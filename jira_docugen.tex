% generated from JIRA project LVV
% using template at <template>.
% Collecting ATM data from folder: "/Data Management/Acceptance|LDM-639"
% using docsteady version 1.2rc24.post6+g778feb8
% Please do not edit -- update information in Jira instead

\section{Test Cases Summary}\label{test-cases-summary}

\begin{longtable}[]{p{2.5cm}p{12cm}p{2cm}}
\toprule
Test Id & Test Name\tabularnewline
\midrule
\endhead
    \hyperref[lvv-t29]{LVV-T29} &
    \href{https://jira.lsstcorp.org/secure/Tests.jspa\#/testCase/LVV-T29}{Verify implementation of Raw Science Image Data Acquisition} &  Defined \tabularnewline
    \hyperref[lvv-t30]{LVV-T30} &
    \href{https://jira.lsstcorp.org/secure/Tests.jspa\#/testCase/LVV-T30}{Verify implementation of Wavefront Sensor Data Acquisition} &  Defined \tabularnewline
    \hyperref[lvv-t32]{LVV-T32} &
    \href{https://jira.lsstcorp.org/secure/Tests.jspa\#/testCase/LVV-T32}{Verify implementation of Raw Image Assembly} &  Defined \tabularnewline
    \hyperref[lvv-t33]{LVV-T33} &
    \href{https://jira.lsstcorp.org/secure/Tests.jspa\#/testCase/LVV-T33}{Verify implementation of Raw Science Image Metadata} &  Defined \tabularnewline
    \hyperref[lvv-t34]{LVV-T34} &
    \href{https://jira.lsstcorp.org/secure/Tests.jspa\#/testCase/LVV-T34}{Verify implementation of Guider Calibration Data Acquisition} &  Defined \tabularnewline
    \hyperref[lvv-t38]{LVV-T38} &
    \href{https://jira.lsstcorp.org/secure/Tests.jspa\#/testCase/LVV-T38}{Verify implementation of Processed Visit Images} &  Defined \tabularnewline
    \hyperref[lvv-t42]{LVV-T42} &
    \href{https://jira.lsstcorp.org/secure/Tests.jspa\#/testCase/LVV-T42}{Verify implementation of Processed Visit Image Content} &  Defined \tabularnewline
    \hyperref[lvv-t45]{LVV-T45} &
    \href{https://jira.lsstcorp.org/secure/Tests.jspa\#/testCase/LVV-T45}{Verify implementation of Prompt Processing Data Quality Report
Definition} &  Defined \tabularnewline
    \hyperref[lvv-t47]{LVV-T47} &
    \href{https://jira.lsstcorp.org/secure/Tests.jspa\#/testCase/LVV-T47}{Verify implementation of Prompt Processing Calibration Report Definition} &  Defined \tabularnewline
    \hyperref[lvv-t48]{LVV-T48} &
    \href{https://jira.lsstcorp.org/secure/Tests.jspa\#/testCase/LVV-T48}{Verify implementation of Exposure Catalog} &  Defined \tabularnewline
    \hyperref[lvv-t61]{LVV-T61} &
    \href{https://jira.lsstcorp.org/secure/Tests.jspa\#/testCase/LVV-T61}{Verify implementation of Associate Sources to Objects} &  Defined \tabularnewline
    \hyperref[lvv-t65]{LVV-T65} &
    \href{https://jira.lsstcorp.org/secure/Tests.jspa\#/testCase/LVV-T65}{Verify implementation of Source Catalog} &  Defined \tabularnewline
    \hyperref[lvv-t82]{LVV-T82} &
    \href{https://jira.lsstcorp.org/secure/Tests.jspa\#/testCase/LVV-T82}{Verify implementation of Tracking Characterization Changes Between Data
Releases} &  Defined \tabularnewline
    \hyperref[lvv-t83]{LVV-T83} &
    \href{https://jira.lsstcorp.org/secure/Tests.jspa\#/testCase/LVV-T83}{Verify implementation of Bad Pixel Map} &  Defined \tabularnewline
    \hyperref[lvv-t84]{LVV-T84} &
    \href{https://jira.lsstcorp.org/secure/Tests.jspa\#/testCase/LVV-T84}{Verify implementation of Bias Residual Image} &  Defined \tabularnewline
    \hyperref[lvv-t85]{LVV-T85} &
    \href{https://jira.lsstcorp.org/secure/Tests.jspa\#/testCase/LVV-T85}{Verify implementation of Crosstalk Correction Matrix} &  Defined \tabularnewline
    \hyperref[lvv-t88]{LVV-T88} &
    \href{https://jira.lsstcorp.org/secure/Tests.jspa\#/testCase/LVV-T88}{Verify implementation of Calibration Data Products} &  Defined \tabularnewline
    \hyperref[lvv-t89]{LVV-T89} &
    \href{https://jira.lsstcorp.org/secure/Tests.jspa\#/testCase/LVV-T89}{Verify implementation of Calibration Image Provenance} &  Defined \tabularnewline
    \hyperref[lvv-t90]{LVV-T90} &
    \href{https://jira.lsstcorp.org/secure/Tests.jspa\#/testCase/LVV-T90}{Verify implementation of Dark Current Correction Frame} &  Defined \tabularnewline
    \hyperref[lvv-t97]{LVV-T97} &
    \href{https://jira.lsstcorp.org/secure/Tests.jspa\#/testCase/LVV-T97}{Verify implementation of Uniqueness of IDs Across Data Releases} &  Defined \tabularnewline
    \hyperref[lvv-t98]{LVV-T98} &
    \href{https://jira.lsstcorp.org/secure/Tests.jspa\#/testCase/LVV-T98}{Verify implementation of Selection of Datasets} &  Defined \tabularnewline
    \hyperref[lvv-t103]{LVV-T103} &
    \href{https://jira.lsstcorp.org/secure/Tests.jspa\#/testCase/LVV-T103}{Verify implementation of Generate Data Quality Report Within Specified
Time} &  Defined \tabularnewline
    \hyperref[lvv-t112]{LVV-T112} &
    \href{https://jira.lsstcorp.org/secure/Tests.jspa\#/testCase/LVV-T112}{Verify implementation of Alert Filtering Service} &  Defined \tabularnewline
    \hyperref[lvv-t113]{LVV-T113} &
    \href{https://jira.lsstcorp.org/secure/Tests.jspa\#/testCase/LVV-T113}{Verify implementation of Performance Requirements for LSST Alert
Filtering Service} &  Defined \tabularnewline
    \hyperref[lvv-t114]{LVV-T114} &
    \href{https://jira.lsstcorp.org/secure/Tests.jspa\#/testCase/LVV-T114}{Verify implementation of Pre-defined alert filters} &  Defined \tabularnewline
    \hyperref[lvv-t115]{LVV-T115} &
    \href{https://jira.lsstcorp.org/secure/Tests.jspa\#/testCase/LVV-T115}{Verify implementation of Calibration Production Processing} &  Defined \tabularnewline
    \hyperref[lvv-t124]{LVV-T124} &
    \href{https://jira.lsstcorp.org/secure/Tests.jspa\#/testCase/LVV-T124}{Verify implementation of Software Architecture to Enable Community
Re-Use} &  Defined \tabularnewline
    \hyperref[lvv-t126]{LVV-T126} &
    \href{https://jira.lsstcorp.org/secure/Tests.jspa\#/testCase/LVV-T126}{Verify implementation of Image Differencing} &  Defined \tabularnewline
    \hyperref[lvv-t127]{LVV-T127} &
    \href{https://jira.lsstcorp.org/secure/Tests.jspa\#/testCase/LVV-T127}{Verify implementation of Provide Source Detection Software} &  Defined \tabularnewline
    \hyperref[lvv-t129]{LVV-T129} &
    \href{https://jira.lsstcorp.org/secure/Tests.jspa\#/testCase/LVV-T129}{Verify implementation of Provide Calibrated Photometry} &  Defined \tabularnewline
    \hyperref[lvv-t131]{LVV-T131} &
    \href{https://jira.lsstcorp.org/secure/Tests.jspa\#/testCase/LVV-T131}{Verify implementation of Provide User Interface Services} &  Defined \tabularnewline
    \hyperref[lvv-t133]{LVV-T133} &
    \href{https://jira.lsstcorp.org/secure/Tests.jspa\#/testCase/LVV-T133}{Verify implementation of Provide Beam Projector Coordinate Calculation
Software} &  Defined \tabularnewline
    \hyperref[lvv-t136]{LVV-T136} &
    \href{https://jira.lsstcorp.org/secure/Tests.jspa\#/testCase/LVV-T136}{Verify implementation of Data Product and Raw Data Access} &  Defined \tabularnewline
    \hyperref[lvv-t137]{LVV-T137} &
    \href{https://jira.lsstcorp.org/secure/Tests.jspa\#/testCase/LVV-T137}{Verify implementation of Data Product Ingest} &  Defined \tabularnewline
    \hyperref[lvv-t140]{LVV-T140} &
    \href{https://jira.lsstcorp.org/secure/Tests.jspa\#/testCase/LVV-T140}{Verify implementation of Production Orchestration} &  Defined \tabularnewline
    \hyperref[lvv-t141]{LVV-T141} &
    \href{https://jira.lsstcorp.org/secure/Tests.jspa\#/testCase/LVV-T141}{Verify implementation of Production Monitoring} &  Defined \tabularnewline
    \hyperref[lvv-t150]{LVV-T150} &
    \href{https://jira.lsstcorp.org/secure/Tests.jspa\#/testCase/LVV-T150}{Verify implementation of Maintain Archive Publicly Accessible} &  Defined \tabularnewline
    \hyperref[lvv-t153]{LVV-T153} &
    \href{https://jira.lsstcorp.org/secure/Tests.jspa\#/testCase/LVV-T153}{Verify implementation of Provide Engineering and Facility Database
Archive} &  Defined \tabularnewline
    \hyperref[lvv-t183]{LVV-T183} &
    \href{https://jira.lsstcorp.org/secure/Tests.jspa\#/testCase/LVV-T183}{Verify implementation of DMS Communication with OCS} &  Defined \tabularnewline
    \hyperref[lvv-t385]{LVV-T385} &
    \href{https://jira.lsstcorp.org/secure/Tests.jspa\#/testCase/LVV-T385}{Verify implementation of minimum number of simultaneous retrievals of
CCD-sized coadd cutouts} &  Defined \tabularnewline
    \hyperref[lvv-t1252]{LVV-T1252} &
    \href{https://jira.lsstcorp.org/secure/Tests.jspa\#/testCase/LVV-T1252}{Verify number of simultaneous alert filter users} &  Defined \tabularnewline
    \hyperref[lvv-t1332]{LVV-T1332} &
    \href{https://jira.lsstcorp.org/secure/Tests.jspa\#/testCase/LVV-T1332}{Verify implementation of maximum time for retrieval of CCD-sized coadd
cutouts} &  Defined \tabularnewline
    \hyperref[lvv-t10]{LVV-T10} &
    \href{https://jira.lsstcorp.org/secure/Tests.jspa\#/testCase/LVV-T10}{DRP-00-00: Installation of the Data Release Production v14.0 science
payload} &  Approved \tabularnewline
    \hyperref[lvv-t11]{LVV-T11} &
    \href{https://jira.lsstcorp.org/secure/Tests.jspa\#/testCase/LVV-T11}{DRP-00-05: Execution of the DRP Science Payload by the Batch Production
Service} &  Approved \tabularnewline
    \hyperref[lvv-t12]{LVV-T12} &
    \href{https://jira.lsstcorp.org/secure/Tests.jspa\#/testCase/LVV-T12}{DRP-00-10: Data Release Includes Required Data Products} &  Approved \tabularnewline
    \hyperref[lvv-t13]{LVV-T13} &
    \href{https://jira.lsstcorp.org/secure/Tests.jspa\#/testCase/LVV-T13}{DRP-00-15: Scientific Verification of Source Catalog} &  Approved \tabularnewline
    \hyperref[lvv-t14]{LVV-T14} &
    \href{https://jira.lsstcorp.org/secure/Tests.jspa\#/testCase/LVV-T14}{DRP-00-25: Scientific Verification of Object Catalog} &  Approved \tabularnewline
    \hyperref[lvv-t15]{LVV-T15} &
    \href{https://jira.lsstcorp.org/secure/Tests.jspa\#/testCase/LVV-T15}{DRP-00-30: Scientific Verification of Processed Visit Images} &  Approved \tabularnewline
    \hyperref[lvv-t16]{LVV-T16} &
    \href{https://jira.lsstcorp.org/secure/Tests.jspa\#/testCase/LVV-T16}{DRP-00-35: Scientific Verification of Coadd Images} &  Approved \tabularnewline
    \hyperref[lvv-t17]{LVV-T17} &
    \href{https://jira.lsstcorp.org/secure/Tests.jspa\#/testCase/LVV-T17}{AG-00-00: Installation of the Alert Generation v16.0 science payload.} &  Approved \tabularnewline
    \hyperref[lvv-t18]{LVV-T18} &
    \href{https://jira.lsstcorp.org/secure/Tests.jspa\#/testCase/LVV-T18}{AG-00-05: Alert Generation Produces Required Data Products} &  Approved \tabularnewline
    \hyperref[lvv-t19]{LVV-T19} &
    \href{https://jira.lsstcorp.org/secure/Tests.jspa\#/testCase/LVV-T19}{AG-00-10: Scientific Verification of Processed Visit Images} &  Approved \tabularnewline
    \hyperref[lvv-t20]{LVV-T20} &
    \href{https://jira.lsstcorp.org/secure/Tests.jspa\#/testCase/LVV-T20}{AG-00-15: Scientific Verification of Difference Images} &  Approved \tabularnewline
    \hyperref[lvv-t21]{LVV-T21} &
    \href{https://jira.lsstcorp.org/secure/Tests.jspa\#/testCase/LVV-T21}{AG-00-20: Scientific Verification of DIASource Catalog} &  Approved \tabularnewline
    \hyperref[lvv-t22]{LVV-T22} &
    \href{https://jira.lsstcorp.org/secure/Tests.jspa\#/testCase/LVV-T22}{AG-00-25: Scientific Verification of DIAObject Catalog} &  Approved \tabularnewline
    \hyperref[lvv-t28]{LVV-T28} &
    \href{https://jira.lsstcorp.org/secure/Tests.jspa\#/testCase/LVV-T28}{Verify implementation of measurements in catalogs from PVIs} &  Approved \tabularnewline
    \hyperref[lvv-t39]{LVV-T39} &
    \href{https://jira.lsstcorp.org/secure/Tests.jspa\#/testCase/LVV-T39}{Verify implementation of Generate Photometric Zeropoint for Visit Image} &  Approved \tabularnewline
    \hyperref[lvv-t40]{LVV-T40} &
    \href{https://jira.lsstcorp.org/secure/Tests.jspa\#/testCase/LVV-T40}{Verify implementation of Generate WCS for Visit Images} &  Approved \tabularnewline
    \hyperref[lvv-t41]{LVV-T41} &
    \href{https://jira.lsstcorp.org/secure/Tests.jspa\#/testCase/LVV-T41}{Verify implementation of Generate PSF for Visit Images} &  Approved \tabularnewline
    \hyperref[lvv-t43]{LVV-T43} &
    \href{https://jira.lsstcorp.org/secure/Tests.jspa\#/testCase/LVV-T43}{Verify implementation of Background Model Calculation} &  Approved \tabularnewline
    \hyperref[lvv-t125]{LVV-T125} &
    \href{https://jira.lsstcorp.org/secure/Tests.jspa\#/testCase/LVV-T125}{Verify implementation of Simulated Data} &  Approved \tabularnewline
    \hyperref[lvv-t132]{LVV-T132} &
    \href{https://jira.lsstcorp.org/secure/Tests.jspa\#/testCase/LVV-T132}{Verify implementation of Pre-cursor and Real Data} &  Approved \tabularnewline
    \hyperref[lvv-t144]{LVV-T144} &
    \href{https://jira.lsstcorp.org/secure/Tests.jspa\#/testCase/LVV-T144}{Verify implementation of Task Specification} &  Approved \tabularnewline
    \hyperref[lvv-t145]{LVV-T145} &
    \href{https://jira.lsstcorp.org/secure/Tests.jspa\#/testCase/LVV-T145}{Verify implementation of Task Configuration} &  Approved \tabularnewline
    \hyperref[lvv-t146]{LVV-T146} &
    \href{https://jira.lsstcorp.org/secure/Tests.jspa\#/testCase/LVV-T146}{Verify implementation of DMS Initialization Component} &  Approved \tabularnewline
    \hyperref[lvv-t149]{LVV-T149} &
    \href{https://jira.lsstcorp.org/secure/Tests.jspa\#/testCase/LVV-T149}{Verify implementation of Catalog Queries} &  Approved \tabularnewline
    \hyperref[lvv-t151]{LVV-T151} &
    \href{https://jira.lsstcorp.org/secure/Tests.jspa\#/testCase/LVV-T151}{Verify Implementation of Catalog Export Formats From the Notebook Aspect} &  Approved \tabularnewline
    \hyperref[lvv-t216]{LVV-T216} &
    \href{https://jira.lsstcorp.org/secure/Tests.jspa\#/testCase/LVV-T216}{Installation of the Alert Distribution payloads.} &  Approved \tabularnewline
    \hyperref[lvv-t217]{LVV-T217} &
    \href{https://jira.lsstcorp.org/secure/Tests.jspa\#/testCase/LVV-T217}{Full Stream Alert Distribution} &  Approved \tabularnewline
    \hyperref[lvv-t218]{LVV-T218} &
    \href{https://jira.lsstcorp.org/secure/Tests.jspa\#/testCase/LVV-T218}{Simple Filtering of the LSST Alert Stream} &  Approved \tabularnewline
    \hyperref[lvv-t62]{LVV-T62} &
    \href{https://jira.lsstcorp.org/secure/Tests.jspa\#/testCase/LVV-T62}{Verify implementation of Provide PSF for Coadded Images} &  Approved \tabularnewline
    \hyperref[lvv-t283]{LVV-T283} &
    \href{https://jira.lsstcorp.org/secure/Tests.jspa\#/testCase/LVV-T283}{RAS-00-00: Writing well-formed raw image} &  Approved \tabularnewline
    \hyperref[lvv-t285]{LVV-T285} &
    \href{https://jira.lsstcorp.org/secure/Tests.jspa\#/testCase/LVV-T285}{RAS-00-10: Raw images in Observatory Operations Data Service} &  Approved \tabularnewline
    \hyperref[lvv-t286]{LVV-T286} &
    \href{https://jira.lsstcorp.org/secure/Tests.jspa\#/testCase/LVV-T286}{RAS-00-20: Raw image are part of the permanent record of survey via DBB} &  Approved \tabularnewline
    \hyperref[lvv-t287]{LVV-T287} &
    \href{https://jira.lsstcorp.org/secure/Tests.jspa\#/testCase/LVV-T287}{RAS-00-30: Raw Image Archiving Availability, Throughput, Reliability,
and Heterogeneity} &  Approved \tabularnewline
    \hyperref[lvv-t362]{LVV-T362} &
    \href{https://jira.lsstcorp.org/secure/Tests.jspa\#/testCase/LVV-T362}{Installation of the LSST Science Pipelines Payloads} &  Approved \tabularnewline
    \hyperref[lvv-t363]{LVV-T363} &
    \href{https://jira.lsstcorp.org/secure/Tests.jspa\#/testCase/LVV-T363}{Science Pipelines Release Documentation} &  Approved \tabularnewline
    \hyperref[lvv-t368]{LVV-T368} &
    \href{https://jira.lsstcorp.org/secure/Tests.jspa\#/testCase/LVV-T368}{Loading and processing Camera test data} &  Approved \tabularnewline
    \hyperref[lvv-t374]{LVV-T374} &
    \href{https://jira.lsstcorp.org/secure/Tests.jspa\#/testCase/LVV-T374}{Ingesting Camera test data} &  Approved \tabularnewline
    \hyperref[lvv-t376]{LVV-T376} &
    \href{https://jira.lsstcorp.org/secure/Tests.jspa\#/testCase/LVV-T376}{Verify the Calculation of Ellipticity Residuals and Correlations} &  Approved \tabularnewline
    \hyperref[lvv-t377]{LVV-T377} &
    \href{https://jira.lsstcorp.org/secure/Tests.jspa\#/testCase/LVV-T377}{Verify Calculation of Photometric Performance Metrics} &  Approved \tabularnewline
    \hyperref[lvv-t378]{LVV-T378} &
    \href{https://jira.lsstcorp.org/secure/Tests.jspa\#/testCase/LVV-T378}{Verify Calculation of Astrometric Performance Metrics} &  Approved \tabularnewline
    \hyperref[lvv-t454]{LVV-T454} &
    \href{https://jira.lsstcorp.org/secure/Tests.jspa\#/testCase/LVV-T454}{LDM-503-8 Enable LSP viewing of spectrograph data.} &  Approved \tabularnewline
    \hyperref[lvv-t1085]{LVV-T1085} &
    \href{https://jira.lsstcorp.org/secure/Tests.jspa\#/testCase/LVV-T1085}{Short Queries Functional Test} &  Approved \tabularnewline
    \hyperref[lvv-t1086]{LVV-T1086} &
    \href{https://jira.lsstcorp.org/secure/Tests.jspa\#/testCase/LVV-T1086}{Full Table Scans Functional Test} &  Approved \tabularnewline
    \hyperref[lvv-t1087]{LVV-T1087} &
    \href{https://jira.lsstcorp.org/secure/Tests.jspa\#/testCase/LVV-T1087}{Full Table Joins Functional Test} &  Approved \tabularnewline
    \hyperref[lvv-t1088]{LVV-T1088} &
    \href{https://jira.lsstcorp.org/secure/Tests.jspa\#/testCase/LVV-T1088}{Concurrent Scans Scaling Test} &  Approved \tabularnewline
    \hyperref[lvv-t1089]{LVV-T1089} &
    \href{https://jira.lsstcorp.org/secure/Tests.jspa\#/testCase/LVV-T1089}{Load Test} &  Approved \tabularnewline
    \hyperref[lvv-t1090]{LVV-T1090} &
    \href{https://jira.lsstcorp.org/secure/Tests.jspa\#/testCase/LVV-T1090}{Heavy Load Test} &  Approved \tabularnewline
    \hyperref[lvv-t1168]{LVV-T1168} &
    \href{https://jira.lsstcorp.org/secure/Tests.jspa\#/testCase/LVV-T1168}{Verify Summit - Base Network Integration} &  Approved \tabularnewline
    \hyperref[lvv-t1232]{LVV-T1232} &
    \href{https://jira.lsstcorp.org/secure/Tests.jspa\#/testCase/LVV-T1232}{Verify Implementation of Catalog Export Formats From the Portal Aspect} &  Approved \tabularnewline
    \hyperref[lvv-t1240]{LVV-T1240} &
    \href{https://jira.lsstcorp.org/secure/Tests.jspa\#/testCase/LVV-T1240}{Verify implementation of minimum astrometric standards per CCD} &  Approved \tabularnewline
    \hyperref[lvv-t1264]{LVV-T1264} &
    \href{https://jira.lsstcorp.org/secure/Tests.jspa\#/testCase/LVV-T1264}{Verify implementation of archiving camera test data} &  Approved \tabularnewline
    \hyperref[lvv-t1549]{LVV-T1549} &
    \href{https://jira.lsstcorp.org/secure/Tests.jspa\#/testCase/LVV-T1549}{LDM-503-6 Comcam verification readiness} &  Approved \tabularnewline
    \hyperref[lvv-t1550]{LVV-T1550} &
    \href{https://jira.lsstcorp.org/secure/Tests.jspa\#/testCase/LVV-T1550}{LDM-503-10 DAQ Validation} &  Approved \tabularnewline
    \hyperref[lvv-t1745]{LVV-T1745} &
    \href{https://jira.lsstcorp.org/secure/Tests.jspa\#/testCase/LVV-T1745}{Verify calculation of median relative astrometric measurement error on
20 arcminute scales} &  Approved \tabularnewline
    \hyperref[lvv-t1746]{LVV-T1746} &
    \href{https://jira.lsstcorp.org/secure/Tests.jspa\#/testCase/LVV-T1746}{Verify calculation of fraction of relative astrometric measurement error
on 5 arcminute scales exceeding outlier limit} &  Approved \tabularnewline
    \hyperref[lvv-t1747]{LVV-T1747} &
    \href{https://jira.lsstcorp.org/secure/Tests.jspa\#/testCase/LVV-T1747}{Verify calculation of relative astrometric measurement error on 5
arcminute scales} &  Approved \tabularnewline
    \hyperref[lvv-t1748]{LVV-T1748} &
    \href{https://jira.lsstcorp.org/secure/Tests.jspa\#/testCase/LVV-T1748}{Verify calculation of median error in absolute position for RA, Dec axes} &  Approved \tabularnewline
    \hyperref[lvv-t1749]{LVV-T1749} &
    \href{https://jira.lsstcorp.org/secure/Tests.jspa\#/testCase/LVV-T1749}{Verify calculation of fraction of relative astrometric measurement error
on 20 arcminute scales exceeding outlier limit} &  Approved \tabularnewline
    \hyperref[lvv-t1750]{LVV-T1750} &
    \href{https://jira.lsstcorp.org/secure/Tests.jspa\#/testCase/LVV-T1750}{Verify calculation of separations relative to r-band exceeding color
difference outlier limit} &  Approved \tabularnewline
    \hyperref[lvv-t1751]{LVV-T1751} &
    \href{https://jira.lsstcorp.org/secure/Tests.jspa\#/testCase/LVV-T1751}{Verify calculation of median relative astrometric measurement error on
200 arcminute scales} &  Approved \tabularnewline
    \hyperref[lvv-t1752]{LVV-T1752} &
    \href{https://jira.lsstcorp.org/secure/Tests.jspa\#/testCase/LVV-T1752}{Verify calculation of fraction of relative astrometric measurement error
on 200 arcminute scales exceeding outlier limit} &  Approved \tabularnewline
    \hyperref[lvv-t1753]{LVV-T1753} &
    \href{https://jira.lsstcorp.org/secure/Tests.jspa\#/testCase/LVV-T1753}{Verify calculation of RMS difference of separations relative to r-band} &  Approved \tabularnewline
    \hyperref[lvv-t1754]{LVV-T1754} &
    \href{https://jira.lsstcorp.org/secure/Tests.jspa\#/testCase/LVV-T1754}{Verify calculation of residual PSF ellipticity correlations for
separations less than 5 arcmin} &  Approved \tabularnewline
    \hyperref[lvv-t1755]{LVV-T1755} &
    \href{https://jira.lsstcorp.org/secure/Tests.jspa\#/testCase/LVV-T1755}{Verify calculation of residual PSF ellipticity correlations for
separations less than 1 arcmin} &  Approved \tabularnewline
    \hyperref[lvv-t1756]{LVV-T1756} &
    \href{https://jira.lsstcorp.org/secure/Tests.jspa\#/testCase/LVV-T1756}{Verify calculation of photometric repeatability in uzy filters} &  Approved \tabularnewline
    \hyperref[lvv-t1757]{LVV-T1757} &
    \href{https://jira.lsstcorp.org/secure/Tests.jspa\#/testCase/LVV-T1757}{Verify calculation of photometric repeatability in gri filters} &  Approved \tabularnewline
    \hyperref[lvv-t1758]{LVV-T1758} &
    \href{https://jira.lsstcorp.org/secure/Tests.jspa\#/testCase/LVV-T1758}{Verify calculation of photometric outliers in uzy bands} &  Approved \tabularnewline
    \hyperref[lvv-t1759]{LVV-T1759} &
    \href{https://jira.lsstcorp.org/secure/Tests.jspa\#/testCase/LVV-T1759}{Verify calculation of photometric outliers in gri bands} &  Approved \tabularnewline
    \hyperref[lvv-t1946]{LVV-T1946} &
    \href{https://jira.lsstcorp.org/secure/Tests.jspa\#/testCase/LVV-T1946}{Verify implementation of measurements in catalogs from coadds} &  Approved \tabularnewline
    \hyperref[lvv-t1947]{LVV-T1947} &
    \href{https://jira.lsstcorp.org/secure/Tests.jspa\#/testCase/LVV-T1947}{Verify implementation of measurements in catalogs from difference images} &  Approved \tabularnewline
    \hyperref[lvv-t23]{LVV-T23} &
    \href{https://jira.lsstcorp.org/secure/Tests.jspa\#/testCase/LVV-T23}{Verify implementation of Storing Approximations of Per-pixel Metadata} &  Draft \tabularnewline
    \hyperref[lvv-t24]{LVV-T24} &
    \href{https://jira.lsstcorp.org/secure/Tests.jspa\#/testCase/LVV-T24}{Verify implementation of Computing Derived Quantities} &  Draft \tabularnewline
    \hyperref[lvv-t25]{LVV-T25} &
    \href{https://jira.lsstcorp.org/secure/Tests.jspa\#/testCase/LVV-T25}{Verify implementation of Denormalizing Database Tables} &  Draft \tabularnewline
    \hyperref[lvv-t26]{LVV-T26} &
    \href{https://jira.lsstcorp.org/secure/Tests.jspa\#/testCase/LVV-T26}{Verify implementation of Maximum Likelihood Values and Covariances} &  Draft \tabularnewline
    \hyperref[lvv-t27]{LVV-T27} &
    \href{https://jira.lsstcorp.org/secure/Tests.jspa\#/testCase/LVV-T27}{Verify implementation of Data Availability} &  Draft \tabularnewline
    \hyperref[lvv-t31]{LVV-T31} &
    \href{https://jira.lsstcorp.org/secure/Tests.jspa\#/testCase/LVV-T31}{Verify implementation of Crosstalk Corrected Science Image Data
Acquisition} &  Draft \tabularnewline
    \hyperref[lvv-t35]{LVV-T35} &
    \href{https://jira.lsstcorp.org/secure/Tests.jspa\#/testCase/LVV-T35}{Verify implementation of Nightly Data Accessible Within 24 hrs} &  Draft \tabularnewline
    \hyperref[lvv-t36]{LVV-T36} &
    \href{https://jira.lsstcorp.org/secure/Tests.jspa\#/testCase/LVV-T36}{Verify implementation of Difference Exposures} &  Draft \tabularnewline
    \hyperref[lvv-t37]{LVV-T37} &
    \href{https://jira.lsstcorp.org/secure/Tests.jspa\#/testCase/LVV-T37}{Verify implementation of Difference Exposure Attributes} &  Draft \tabularnewline
    \hyperref[lvv-t44]{LVV-T44} &
    \href{https://jira.lsstcorp.org/secure/Tests.jspa\#/testCase/LVV-T44}{Verify implementation of Documenting Image Characterization} &  Draft \tabularnewline
    \hyperref[lvv-t46]{LVV-T46} &
    \href{https://jira.lsstcorp.org/secure/Tests.jspa\#/testCase/LVV-T46}{Verify implementation of Prompt Processing Performance Report Definition} &  Draft \tabularnewline
    \hyperref[lvv-t49]{LVV-T49} &
    \href{https://jira.lsstcorp.org/secure/Tests.jspa\#/testCase/LVV-T49}{Verify implementation of DIASource Catalog} &  Draft \tabularnewline
    \hyperref[lvv-t50]{LVV-T50} &
    \href{https://jira.lsstcorp.org/secure/Tests.jspa\#/testCase/LVV-T50}{Verify implementation of Faint DIASource Measurements} &  Draft \tabularnewline
    \hyperref[lvv-t51]{LVV-T51} &
    \href{https://jira.lsstcorp.org/secure/Tests.jspa\#/testCase/LVV-T51}{Verify implementation of DIAObject Catalog} &  Draft \tabularnewline
    \hyperref[lvv-t52]{LVV-T52} &
    \href{https://jira.lsstcorp.org/secure/Tests.jspa\#/testCase/LVV-T52}{Verify implementation of DIAObject Attributes} &  Draft \tabularnewline
    \hyperref[lvv-t53]{LVV-T53} &
    \href{https://jira.lsstcorp.org/secure/Tests.jspa\#/testCase/LVV-T53}{Verify implementation of SSObject Catalog} &  Draft \tabularnewline
    \hyperref[lvv-t54]{LVV-T54} &
    \href{https://jira.lsstcorp.org/secure/Tests.jspa\#/testCase/LVV-T54}{Verify implementation of Alert Content} &  Draft \tabularnewline
    \hyperref[lvv-t55]{LVV-T55} &
    \href{https://jira.lsstcorp.org/secure/Tests.jspa\#/testCase/LVV-T55}{Verify implementation of DIAForcedSource Catalog} &  Draft \tabularnewline
    \hyperref[lvv-t56]{LVV-T56} &
    \href{https://jira.lsstcorp.org/secure/Tests.jspa\#/testCase/LVV-T56}{Verify implementation of Characterizing Variability} &  Draft \tabularnewline
    \hyperref[lvv-t57]{LVV-T57} &
    \href{https://jira.lsstcorp.org/secure/Tests.jspa\#/testCase/LVV-T57}{Verify implementation of Calculating SSObject Parameters} &  Draft \tabularnewline
    \hyperref[lvv-t58]{LVV-T58} &
    \href{https://jira.lsstcorp.org/secure/Tests.jspa\#/testCase/LVV-T58}{Verify implementation of Matching DIASources to Objects} &  Draft \tabularnewline
    \hyperref[lvv-t59]{LVV-T59} &
    \href{https://jira.lsstcorp.org/secure/Tests.jspa\#/testCase/LVV-T59}{Verify implementation of Regenerating L1 Data Products During Data
Release Processing} &  Draft \tabularnewline
    \hyperref[lvv-t60]{LVV-T60} &
    \href{https://jira.lsstcorp.org/secure/Tests.jspa\#/testCase/LVV-T60}{Verify implementation of Publishing predicted visit schedule} &  Draft \tabularnewline
    \hyperref[lvv-t63]{LVV-T63} &
    \href{https://jira.lsstcorp.org/secure/Tests.jspa\#/testCase/LVV-T63}{Verify implementation of Produce Images for EPO} &  Draft \tabularnewline
    \hyperref[lvv-t64]{LVV-T64} &
    \href{https://jira.lsstcorp.org/secure/Tests.jspa\#/testCase/LVV-T64}{Verify implementation of Coadded Image Provenance} &  Draft \tabularnewline
    \hyperref[lvv-t66]{LVV-T66} &
    \href{https://jira.lsstcorp.org/secure/Tests.jspa\#/testCase/LVV-T66}{Verify implementation of Forced-Source Catalog} &  Draft \tabularnewline
    \hyperref[lvv-t67]{LVV-T67} &
    \href{https://jira.lsstcorp.org/secure/Tests.jspa\#/testCase/LVV-T67}{Verify implementation of Object Catalog} &  Draft \tabularnewline
    \hyperref[lvv-t68]{LVV-T68} &
    \href{https://jira.lsstcorp.org/secure/Tests.jspa\#/testCase/LVV-T68}{Verify implementation of Provide Photometric Redshifts of Galaxies} &  Draft \tabularnewline
    \hyperref[lvv-t69]{LVV-T69} &
    \href{https://jira.lsstcorp.org/secure/Tests.jspa\#/testCase/LVV-T69}{Verify implementation of Object Characterization} &  Draft \tabularnewline
    \hyperref[lvv-t71]{LVV-T71} &
    \href{https://jira.lsstcorp.org/secure/Tests.jspa\#/testCase/LVV-T71}{Verify implementation of Detecting extended low surface brightness
objects} &  Draft \tabularnewline
    \hyperref[lvv-t72]{LVV-T72} &
    \href{https://jira.lsstcorp.org/secure/Tests.jspa\#/testCase/LVV-T72}{Verify implementation of Coadd Image Method Constraints} &  Draft \tabularnewline
    \hyperref[lvv-t73]{LVV-T73} &
    \href{https://jira.lsstcorp.org/secure/Tests.jspa\#/testCase/LVV-T73}{Verify implementation of Deep Detection Coadds} &  Draft \tabularnewline
    \hyperref[lvv-t74]{LVV-T74} &
    \href{https://jira.lsstcorp.org/secure/Tests.jspa\#/testCase/LVV-T74}{Verify implementation of Template Coadds} &  Draft \tabularnewline
    \hyperref[lvv-t75]{LVV-T75} &
    \href{https://jira.lsstcorp.org/secure/Tests.jspa\#/testCase/LVV-T75}{Verify implementation of Multi-band Coadds} &  Draft \tabularnewline
    \hyperref[lvv-t76]{LVV-T76} &
    \href{https://jira.lsstcorp.org/secure/Tests.jspa\#/testCase/LVV-T76}{Verify implementation of All-Sky Visualization of Data Releases} &  Draft \tabularnewline
    \hyperref[lvv-t77]{LVV-T77} &
    \href{https://jira.lsstcorp.org/secure/Tests.jspa\#/testCase/LVV-T77}{Verify implementation of Best Seeing Coadds} &  Draft \tabularnewline
    \hyperref[lvv-t78]{LVV-T78} &
    \href{https://jira.lsstcorp.org/secure/Tests.jspa\#/testCase/LVV-T78}{Verify implementation of Persisting Data Products} &  Draft \tabularnewline
    \hyperref[lvv-t79]{LVV-T79} &
    \href{https://jira.lsstcorp.org/secure/Tests.jspa\#/testCase/LVV-T79}{Verify implementation of PSF-Matched Coadds} &  Draft \tabularnewline
    \hyperref[lvv-t80]{LVV-T80} &
    \href{https://jira.lsstcorp.org/secure/Tests.jspa\#/testCase/LVV-T80}{Verify implementation of Detecting faint variable objects} &  Draft \tabularnewline
    \hyperref[lvv-t81]{LVV-T81} &
    \href{https://jira.lsstcorp.org/secure/Tests.jspa\#/testCase/LVV-T81}{Verify implementation of Targeted Coadds} &  Draft \tabularnewline
    \hyperref[lvv-t86]{LVV-T86} &
    \href{https://jira.lsstcorp.org/secure/Tests.jspa\#/testCase/LVV-T86}{Verify implementation of Illumination Correction Frame} &  Draft \tabularnewline
    \hyperref[lvv-t87]{LVV-T87} &
    \href{https://jira.lsstcorp.org/secure/Tests.jspa\#/testCase/LVV-T87}{Verify implementation of Monochromatic Flatfield Data Cube} &  Draft \tabularnewline
    \hyperref[lvv-t91]{LVV-T91} &
    \href{https://jira.lsstcorp.org/secure/Tests.jspa\#/testCase/LVV-T91}{Verify implementation of Fringe Correction Frame} &  Draft \tabularnewline
    \hyperref[lvv-t92]{LVV-T92} &
    \href{https://jira.lsstcorp.org/secure/Tests.jspa\#/testCase/LVV-T92}{Verify implementation of Processing of Data From Special Programs} &  Draft \tabularnewline
    \hyperref[lvv-t93]{LVV-T93} &
    \href{https://jira.lsstcorp.org/secure/Tests.jspa\#/testCase/LVV-T93}{Verify implementation of Level 1 Processing of Special Programs Data} &  Draft \tabularnewline
    \hyperref[lvv-t94]{LVV-T94} &
    \href{https://jira.lsstcorp.org/secure/Tests.jspa\#/testCase/LVV-T94}{Verify implementation of Special Programs Database} &  Draft \tabularnewline
    \hyperref[lvv-t95]{LVV-T95} &
    \href{https://jira.lsstcorp.org/secure/Tests.jspa\#/testCase/LVV-T95}{Verify implementation of Constraints on Level 1 Special Program Products
Generation} &  Draft \tabularnewline
    \hyperref[lvv-t96]{LVV-T96} &
    \href{https://jira.lsstcorp.org/secure/Tests.jspa\#/testCase/LVV-T96}{Verify implementation of Query Repeatability} &  Draft \tabularnewline
    \hyperref[lvv-t99]{LVV-T99} &
    \href{https://jira.lsstcorp.org/secure/Tests.jspa\#/testCase/LVV-T99}{Verify implementation of Processing of Datasets} &  Draft \tabularnewline
    \hyperref[lvv-t100]{LVV-T100} &
    \href{https://jira.lsstcorp.org/secure/Tests.jspa\#/testCase/LVV-T100}{Verify implementation of Transparent Data Access} &  Draft \tabularnewline
    \hyperref[lvv-t101]{LVV-T101} &
    \href{https://jira.lsstcorp.org/secure/Tests.jspa\#/testCase/LVV-T101}{Verify implementation of Transient Alert Distribution} &  Draft \tabularnewline
    \hyperref[lvv-t102]{LVV-T102} &
    \href{https://jira.lsstcorp.org/secure/Tests.jspa\#/testCase/LVV-T102}{Verify implementation of Solar System Objects Available Within Specified
Time} &  Draft \tabularnewline
    \hyperref[lvv-t104]{LVV-T104} &
    \href{https://jira.lsstcorp.org/secure/Tests.jspa\#/testCase/LVV-T104}{Verify implementation of Generate DMS Performance Report Within
Specified Time} &  Draft \tabularnewline
    \hyperref[lvv-t105]{LVV-T105} &
    \href{https://jira.lsstcorp.org/secure/Tests.jspa\#/testCase/LVV-T105}{Verify implementation of Generate Calibration Report Within Specified
Time} &  Draft \tabularnewline
    \hyperref[lvv-t106]{LVV-T106} &
    \href{https://jira.lsstcorp.org/secure/Tests.jspa\#/testCase/LVV-T106}{Verify implementation of Calibration Images Available Within Specified
Time} &  Draft \tabularnewline
    \hyperref[lvv-t107]{LVV-T107} &
    \href{https://jira.lsstcorp.org/secure/Tests.jspa\#/testCase/LVV-T107}{Verify implementation of Level-1 Production Completeness} &  Draft \tabularnewline
    \hyperref[lvv-t108]{LVV-T108} &
    \href{https://jira.lsstcorp.org/secure/Tests.jspa\#/testCase/LVV-T108}{Verify implementation of Level 1 Source Association} &  Draft \tabularnewline
    \hyperref[lvv-t109]{LVV-T109} &
    \href{https://jira.lsstcorp.org/secure/Tests.jspa\#/testCase/LVV-T109}{Verify implementation of SSObject Precovery} &  Draft \tabularnewline
    \hyperref[lvv-t110]{LVV-T110} &
    \href{https://jira.lsstcorp.org/secure/Tests.jspa\#/testCase/LVV-T110}{Verify implementation of DIASource Precovery} &  Draft \tabularnewline
    \hyperref[lvv-t111]{LVV-T111} &
    \href{https://jira.lsstcorp.org/secure/Tests.jspa\#/testCase/LVV-T111}{Verify implementation of Use of External Orbit Catalogs} &  Draft \tabularnewline
    \hyperref[lvv-t116]{LVV-T116} &
    \href{https://jira.lsstcorp.org/secure/Tests.jspa\#/testCase/LVV-T116}{Verify implementation of Associating Objects across data releases} &  Draft \tabularnewline
    \hyperref[lvv-t117]{LVV-T117} &
    \href{https://jira.lsstcorp.org/secure/Tests.jspa\#/testCase/LVV-T117}{Verify implementation of DAC resource allocation for Level 3 processing} &  Draft \tabularnewline
    \hyperref[lvv-t118]{LVV-T118} &
    \href{https://jira.lsstcorp.org/secure/Tests.jspa\#/testCase/LVV-T118}{Verify implementation of Level 3 Data Product Self Consistency} &  Draft \tabularnewline
    \hyperref[lvv-t119]{LVV-T119} &
    \href{https://jira.lsstcorp.org/secure/Tests.jspa\#/testCase/LVV-T119}{Verify implementation of Provenance for Level 3 processing at DACs} &  Draft \tabularnewline
    \hyperref[lvv-t120]{LVV-T120} &
    \href{https://jira.lsstcorp.org/secure/Tests.jspa\#/testCase/LVV-T120}{Verify implementation of Software framework for Level 3 catalog
processing} &  Draft \tabularnewline
    \hyperref[lvv-t121]{LVV-T121} &
    \href{https://jira.lsstcorp.org/secure/Tests.jspa\#/testCase/LVV-T121}{Verify implementation of Software framework for Level 3 image processing} &  Draft \tabularnewline
    \hyperref[lvv-t122]{LVV-T122} &
    \href{https://jira.lsstcorp.org/secure/Tests.jspa\#/testCase/LVV-T122}{Verify implementation of Level 3 Data Import} &  Draft \tabularnewline
    \hyperref[lvv-t123]{LVV-T123} &
    \href{https://jira.lsstcorp.org/secure/Tests.jspa\#/testCase/LVV-T123}{Verify implementation of Access Controls of Level 3 Data Products} &  Draft \tabularnewline
    \hyperref[lvv-t128]{LVV-T128} &
    \href{https://jira.lsstcorp.org/secure/Tests.jspa\#/testCase/LVV-T128}{Verify implementation Provide Astrometric Model} &  Draft \tabularnewline
    \hyperref[lvv-t130]{LVV-T130} &
    \href{https://jira.lsstcorp.org/secure/Tests.jspa\#/testCase/LVV-T130}{Verify implementation of Enable a Range of Shape Measurement Approaches} &  Draft \tabularnewline
    \hyperref[lvv-t134]{LVV-T134} &
    \href{https://jira.lsstcorp.org/secure/Tests.jspa\#/testCase/LVV-T134}{Verify implementation of Provide Image Access Services} &  Draft \tabularnewline
    \hyperref[lvv-t138]{LVV-T138} &
    \href{https://jira.lsstcorp.org/secure/Tests.jspa\#/testCase/LVV-T138}{Verify implementation of Bulk Download Service} &  Draft \tabularnewline
    \hyperref[lvv-t142]{LVV-T142} &
    \href{https://jira.lsstcorp.org/secure/Tests.jspa\#/testCase/LVV-T142}{Verify implementation of Production Fault Tolerance} &  Draft \tabularnewline
    \hyperref[lvv-t147]{LVV-T147} &
    \href{https://jira.lsstcorp.org/secure/Tests.jspa\#/testCase/LVV-T147}{Verify implementation of Control of Level-1 Production} &  Draft \tabularnewline
    \hyperref[lvv-t148]{LVV-T148} &
    \href{https://jira.lsstcorp.org/secure/Tests.jspa\#/testCase/LVV-T148}{Verify implementation of Unique Processing Coverage} &  Draft \tabularnewline
    \hyperref[lvv-t152]{LVV-T152} &
    \href{https://jira.lsstcorp.org/secure/Tests.jspa\#/testCase/LVV-T152}{Verify implementation of Keep Historical Alert Archive} &  Draft \tabularnewline
    \hyperref[lvv-t154]{LVV-T154} &
    \href{https://jira.lsstcorp.org/secure/Tests.jspa\#/testCase/LVV-T154}{Verify implementation of Raw Data Archiving Reliability} &  Draft \tabularnewline
    \hyperref[lvv-t155]{LVV-T155} &
    \href{https://jira.lsstcorp.org/secure/Tests.jspa\#/testCase/LVV-T155}{Verify implementation of Un-Archived Data Product Cache} &  Draft \tabularnewline
    \hyperref[lvv-t156]{LVV-T156} &
    \href{https://jira.lsstcorp.org/secure/Tests.jspa\#/testCase/LVV-T156}{Verify implementation of Regenerate Un-archived Data Products} &  Draft \tabularnewline
    \hyperref[lvv-t157]{LVV-T157} &
    \href{https://jira.lsstcorp.org/secure/Tests.jspa\#/testCase/LVV-T157}{Verify implementation Level 1 Data Product Access} &  Draft \tabularnewline
    \hyperref[lvv-t158]{LVV-T158} &
    \href{https://jira.lsstcorp.org/secure/Tests.jspa\#/testCase/LVV-T158}{Verify implementation Level 1 and 2 Catalog Access} &  Draft \tabularnewline
    \hyperref[lvv-t159]{LVV-T159} &
    \href{https://jira.lsstcorp.org/secure/Tests.jspa\#/testCase/LVV-T159}{Verify implementation of Regenerating Data Products from Previous Data
Releases} &  Draft \tabularnewline
    \hyperref[lvv-t160]{LVV-T160} &
    \href{https://jira.lsstcorp.org/secure/Tests.jspa\#/testCase/LVV-T160}{Verify implementation of Providing a Precovery Service} &  Draft \tabularnewline
    \hyperref[lvv-t161]{LVV-T161} &
    \href{https://jira.lsstcorp.org/secure/Tests.jspa\#/testCase/LVV-T161}{Verify implementation of Logging of catalog queries} &  Draft \tabularnewline
    \hyperref[lvv-t162]{LVV-T162} &
    \href{https://jira.lsstcorp.org/secure/Tests.jspa\#/testCase/LVV-T162}{Verify implementation of Access to Previous Data Releases} &  Draft \tabularnewline
    \hyperref[lvv-t163]{LVV-T163} &
    \href{https://jira.lsstcorp.org/secure/Tests.jspa\#/testCase/LVV-T163}{Verify implementation of Data Access Services} &  Draft \tabularnewline
    \hyperref[lvv-t164]{LVV-T164} &
    \href{https://jira.lsstcorp.org/secure/Tests.jspa\#/testCase/LVV-T164}{Verify implementation of Operations Subsets} &  Draft \tabularnewline
    \hyperref[lvv-t165]{LVV-T165} &
    \href{https://jira.lsstcorp.org/secure/Tests.jspa\#/testCase/LVV-T165}{Verify implementation of Subsets Support} &  Draft \tabularnewline
    \hyperref[lvv-t166]{LVV-T166} &
    \href{https://jira.lsstcorp.org/secure/Tests.jspa\#/testCase/LVV-T166}{Verify implementation of Access Services Performance} &  Draft \tabularnewline
    \hyperref[lvv-t167]{LVV-T167} &
    \href{https://jira.lsstcorp.org/secure/Tests.jspa\#/testCase/LVV-T167}{Verify Capability to serve older Data Releases at Full Performance} &  Draft \tabularnewline
    \hyperref[lvv-t168]{LVV-T168} &
    \href{https://jira.lsstcorp.org/secure/Tests.jspa\#/testCase/LVV-T168}{Verify design of Data Access Services allows Evolution of the LSST Data
Model} &  Draft \tabularnewline
    \hyperref[lvv-t169]{LVV-T169} &
    \href{https://jira.lsstcorp.org/secure/Tests.jspa\#/testCase/LVV-T169}{Verify implementation of Older Release Behavior} &  Draft \tabularnewline
    \hyperref[lvv-t170]{LVV-T170} &
    \href{https://jira.lsstcorp.org/secure/Tests.jspa\#/testCase/LVV-T170}{Verify implementation of Query Availability} &  Draft \tabularnewline
    \hyperref[lvv-t171]{LVV-T171} &
    \href{https://jira.lsstcorp.org/secure/Tests.jspa\#/testCase/LVV-T171}{Verify implementation of Pipeline Availability} &  Draft \tabularnewline
    \hyperref[lvv-t172]{LVV-T172} &
    \href{https://jira.lsstcorp.org/secure/Tests.jspa\#/testCase/LVV-T172}{Verify implementation of Optimization of Cost, Reliability and
Availability} &  Draft \tabularnewline
    \hyperref[lvv-t173]{LVV-T173} &
    \href{https://jira.lsstcorp.org/secure/Tests.jspa\#/testCase/LVV-T173}{Verify implementation of Pipeline Throughput} &  Draft \tabularnewline
    \hyperref[lvv-t174]{LVV-T174} &
    \href{https://jira.lsstcorp.org/secure/Tests.jspa\#/testCase/LVV-T174}{Verify implementation of Re-processing Capacity} &  Draft \tabularnewline
    \hyperref[lvv-t175]{LVV-T175} &
    \href{https://jira.lsstcorp.org/secure/Tests.jspa\#/testCase/LVV-T175}{Verify implementation of Temporary Storage for Communications Links} &  Draft \tabularnewline
    \hyperref[lvv-t176]{LVV-T176} &
    \href{https://jira.lsstcorp.org/secure/Tests.jspa\#/testCase/LVV-T176}{Verify implementation of Infrastructure Sizing for ``catching up''} &  Draft \tabularnewline
    \hyperref[lvv-t177]{LVV-T177} &
    \href{https://jira.lsstcorp.org/secure/Tests.jspa\#/testCase/LVV-T177}{Verify implementation of Incorporate Fault-Tolerance} &  Draft \tabularnewline
    \hyperref[lvv-t178]{LVV-T178} &
    \href{https://jira.lsstcorp.org/secure/Tests.jspa\#/testCase/LVV-T178}{Verify implementation of Incorporate Autonomics} &  Draft \tabularnewline
    \hyperref[lvv-t179]{LVV-T179} &
    \href{https://jira.lsstcorp.org/secure/Tests.jspa\#/testCase/LVV-T179}{Verify implementation of Compute Platform Heterogeneity} &  Draft \tabularnewline
    \hyperref[lvv-t180]{LVV-T180} &
    \href{https://jira.lsstcorp.org/secure/Tests.jspa\#/testCase/LVV-T180}{Verify implementation of Data Management Unscheduled Downtime} &  Draft \tabularnewline
    \hyperref[lvv-t181]{LVV-T181} &
    \href{https://jira.lsstcorp.org/secure/Tests.jspa\#/testCase/LVV-T181}{Verify Base Voice Over IP (VOIP)} &  Draft \tabularnewline
    \hyperref[lvv-t182]{LVV-T182} &
    \href{https://jira.lsstcorp.org/secure/Tests.jspa\#/testCase/LVV-T182}{Verify implementation of Prefer Computing and Storage Down} &  Draft \tabularnewline
    \hyperref[lvv-t185]{LVV-T185} &
    \href{https://jira.lsstcorp.org/secure/Tests.jspa\#/testCase/LVV-T185}{Verify implementation of Summit to Base Network Availability} &  Draft \tabularnewline
    \hyperref[lvv-t186]{LVV-T186} &
    \href{https://jira.lsstcorp.org/secure/Tests.jspa\#/testCase/LVV-T186}{Verify implementation of Summit to Base Network Reliability} &  Draft \tabularnewline
    \hyperref[lvv-t187]{LVV-T187} &
    \href{https://jira.lsstcorp.org/secure/Tests.jspa\#/testCase/LVV-T187}{Verify implementation of Summit to Base Network Secondary Link} &  Draft \tabularnewline
    \hyperref[lvv-t188]{LVV-T188} &
    \href{https://jira.lsstcorp.org/secure/Tests.jspa\#/testCase/LVV-T188}{Verify implementation of Summit to Base Network Ownership and Operation} &  Draft \tabularnewline
    \hyperref[lvv-t189]{LVV-T189} &
    \href{https://jira.lsstcorp.org/secure/Tests.jspa\#/testCase/LVV-T189}{Verify implementation of Base Facility Infrastructure} &  Draft \tabularnewline
    \hyperref[lvv-t190]{LVV-T190} &
    \href{https://jira.lsstcorp.org/secure/Tests.jspa\#/testCase/LVV-T190}{Verify implementation of Base Facility Co-Location with Existing
Facility} &  Draft \tabularnewline
    \hyperref[lvv-t191]{LVV-T191} &
    \href{https://jira.lsstcorp.org/secure/Tests.jspa\#/testCase/LVV-T191}{Verify implementation of Commissioning Cluster} &  Draft \tabularnewline
    \hyperref[lvv-t192]{LVV-T192} &
    \href{https://jira.lsstcorp.org/secure/Tests.jspa\#/testCase/LVV-T192}{Verify implementation of Base Wireless LAN (WiFi)} &  Draft \tabularnewline
    \hyperref[lvv-t193]{LVV-T193} &
    \href{https://jira.lsstcorp.org/secure/Tests.jspa\#/testCase/LVV-T193}{Verify implementation of Base to Archive Network} &  Draft \tabularnewline
    \hyperref[lvv-t194]{LVV-T194} &
    \href{https://jira.lsstcorp.org/secure/Tests.jspa\#/testCase/LVV-T194}{Verify implementation of Base to Archive Network Availability} &  Draft \tabularnewline
    \hyperref[lvv-t195]{LVV-T195} &
    \href{https://jira.lsstcorp.org/secure/Tests.jspa\#/testCase/LVV-T195}{Verify implementation of Base to Archive Network Reliability} &  Draft \tabularnewline
    \hyperref[lvv-t196]{LVV-T196} &
    \href{https://jira.lsstcorp.org/secure/Tests.jspa\#/testCase/LVV-T196}{Verify implementation of Base to Archive Network Secondary Link} &  Draft \tabularnewline
    \hyperref[lvv-t197]{LVV-T197} &
    \href{https://jira.lsstcorp.org/secure/Tests.jspa\#/testCase/LVV-T197}{Verify implementation of Archive Center} &  Draft \tabularnewline
    \hyperref[lvv-t198]{LVV-T198} &
    \href{https://jira.lsstcorp.org/secure/Tests.jspa\#/testCase/LVV-T198}{Verify implementation of Archive Center Disaster Recovery} &  Draft \tabularnewline
    \hyperref[lvv-t199]{LVV-T199} &
    \href{https://jira.lsstcorp.org/secure/Tests.jspa\#/testCase/LVV-T199}{Verify implementation of Archive Center Co-Location with Existing
Facility} &  Draft \tabularnewline
    \hyperref[lvv-t200]{LVV-T200} &
    \href{https://jira.lsstcorp.org/secure/Tests.jspa\#/testCase/LVV-T200}{Verify implementation of Archive to Data Access Center Network} &  Draft \tabularnewline
    \hyperref[lvv-t201]{LVV-T201} &
    \href{https://jira.lsstcorp.org/secure/Tests.jspa\#/testCase/LVV-T201}{Verify implementation of Archive to Data Access Center Network
Availability} &  Draft \tabularnewline
    \hyperref[lvv-t202]{LVV-T202} &
    \href{https://jira.lsstcorp.org/secure/Tests.jspa\#/testCase/LVV-T202}{Verify implementation of Archive to Data Access Center Network
Reliability} &  Draft \tabularnewline
    \hyperref[lvv-t203]{LVV-T203} &
    \href{https://jira.lsstcorp.org/secure/Tests.jspa\#/testCase/LVV-T203}{Verify implementation of Archive to Data Access Center Network Secondary
Link} &  Draft \tabularnewline
    \hyperref[lvv-t204]{LVV-T204} &
    \href{https://jira.lsstcorp.org/secure/Tests.jspa\#/testCase/LVV-T204}{Verify implementation of Access to catalogs for external Level 3
processing} &  Draft \tabularnewline
    \hyperref[lvv-t205]{LVV-T205} &
    \href{https://jira.lsstcorp.org/secure/Tests.jspa\#/testCase/LVV-T205}{Verify implementation of Access to input catalogs for DAC-based Level 3
processing} &  Draft \tabularnewline
    \hyperref[lvv-t206]{LVV-T206} &
    \href{https://jira.lsstcorp.org/secure/Tests.jspa\#/testCase/LVV-T206}{Verify implementation of Federation with external catalogs} &  Draft \tabularnewline
    \hyperref[lvv-t207]{LVV-T207} &
    \href{https://jira.lsstcorp.org/secure/Tests.jspa\#/testCase/LVV-T207}{Verify implementation of Access to images for external Level 3
processing} &  Draft \tabularnewline
    \hyperref[lvv-t208]{LVV-T208} &
    \href{https://jira.lsstcorp.org/secure/Tests.jspa\#/testCase/LVV-T208}{Verify implementation of Access to input images for DAC-based Level 3
processing} &  Draft \tabularnewline
    \hyperref[lvv-t209]{LVV-T209} &
    \href{https://jira.lsstcorp.org/secure/Tests.jspa\#/testCase/LVV-T209}{Verify implementation of Data Access Centers} &  Draft \tabularnewline
    \hyperref[lvv-t210]{LVV-T210} &
    \href{https://jira.lsstcorp.org/secure/Tests.jspa\#/testCase/LVV-T210}{Verify implementation of Data Access Center Simultaneous Connections} &  Draft \tabularnewline
    \hyperref[lvv-t211]{LVV-T211} &
    \href{https://jira.lsstcorp.org/secure/Tests.jspa\#/testCase/LVV-T211}{Verify implementation of Data Access Center Geographical Distribution} &  Draft \tabularnewline
    \hyperref[lvv-t212]{LVV-T212} &
    \href{https://jira.lsstcorp.org/secure/Tests.jspa\#/testCase/LVV-T212}{Verify implementation of No Limit on Data Access Centers} &  Draft \tabularnewline
    \hyperref[lvv-t284]{LVV-T284} &
    \href{https://jira.lsstcorp.org/secure/Tests.jspa\#/testCase/LVV-T284}{RAS-00-05: (LDM-503-8b) Writing data from CCOB to the DBB for further
data processing} &  Draft \tabularnewline
    \hyperref[lvv-t1097]{LVV-T1097} &
    \href{https://jira.lsstcorp.org/secure/Tests.jspa\#/testCase/LVV-T1097}{Verify Summit Facility Network Implementation} &  Draft \tabularnewline
    \hyperref[lvv-t1250]{LVV-T1250} &
    \href{https://jira.lsstcorp.org/secure/Tests.jspa\#/testCase/LVV-T1250}{Verify implementation of minimum number of simultaneous DM EFD query
users} &  Draft \tabularnewline
    \hyperref[lvv-t1251]{LVV-T1251} &
    \href{https://jira.lsstcorp.org/secure/Tests.jspa\#/testCase/LVV-T1251}{Verify implementation of maximum time to retrieve DM EFD query results} &  Draft \tabularnewline
    \hyperref[lvv-t1276]{LVV-T1276} &
    \href{https://jira.lsstcorp.org/secure/Tests.jspa\#/testCase/LVV-T1276}{Verify implementation of latency of reporting optical transients} &  Draft \tabularnewline
    \hyperref[lvv-t1277]{LVV-T1277} &
    \href{https://jira.lsstcorp.org/secure/Tests.jspa\#/testCase/LVV-T1277}{Verify processing of maximum number of calibration exposures} &  Draft \tabularnewline
    \hyperref[lvv-t1524]{LVV-T1524} &
    \href{https://jira.lsstcorp.org/secure/Tests.jspa\#/testCase/LVV-T1524}{Verify Implementation of Exporting MOCs as FITS} &  Draft \tabularnewline
    \hyperref[lvv-t1525]{LVV-T1525} &
    \href{https://jira.lsstcorp.org/secure/Tests.jspa\#/testCase/LVV-T1525}{Verify Implementation of Linkage Between HiPS Maps and Coadded Images} &  Draft \tabularnewline
    \hyperref[lvv-t1526]{LVV-T1526} &
    \href{https://jira.lsstcorp.org/secure/Tests.jspa\#/testCase/LVV-T1526}{Verify Availability of Secure and Authenticated HiPS Service} &  Draft \tabularnewline
    \hyperref[lvv-t1527]{LVV-T1527} &
    \href{https://jira.lsstcorp.org/secure/Tests.jspa\#/testCase/LVV-T1527}{Verify Support for HiPS Visualization} &  Draft \tabularnewline
    \hyperref[lvv-t1528]{LVV-T1528} &
    \href{https://jira.lsstcorp.org/secure/Tests.jspa\#/testCase/LVV-T1528}{Verify Visualization of MOCs via Science Platform} &  Draft \tabularnewline
    \hyperref[lvv-t1529]{LVV-T1529} &
    \href{https://jira.lsstcorp.org/secure/Tests.jspa\#/testCase/LVV-T1529}{Verify Production of All-Sky HiPS Map} &  Draft \tabularnewline
    \hyperref[lvv-t1530]{LVV-T1530} &
    \href{https://jira.lsstcorp.org/secure/Tests.jspa\#/testCase/LVV-T1530}{Verify Production of Multi-Order Coverage Maps for Survey Data} &  Draft \tabularnewline
    \hyperref[lvv-t1556]{LVV-T1556} &
    \href{https://jira.lsstcorp.org/secure/Tests.jspa\#/testCase/LVV-T1556}{LDM-503-10B Large Scale CCOB Data Access} &  Draft \tabularnewline
    \hyperref[lvv-t1560]{LVV-T1560} &
    \href{https://jira.lsstcorp.org/secure/Tests.jspa\#/testCase/LVV-T1560}{Verify archiving of processing provenance} &  Draft \tabularnewline
    \hyperref[lvv-t1561]{LVV-T1561} &
    \href{https://jira.lsstcorp.org/secure/Tests.jspa\#/testCase/LVV-T1561}{Verify provenance availability to science users} &  Draft \tabularnewline
    \hyperref[lvv-t1562]{LVV-T1562} &
    \href{https://jira.lsstcorp.org/secure/Tests.jspa\#/testCase/LVV-T1562}{Verify availability of re-run tools} &  Draft \tabularnewline
    \hyperref[lvv-t1563]{LVV-T1563} &
    \href{https://jira.lsstcorp.org/secure/Tests.jspa\#/testCase/LVV-T1563}{Verify re-run on different system produces the same results} &  Draft \tabularnewline
    \hyperref[lvv-t1564]{LVV-T1564} &
    \href{https://jira.lsstcorp.org/secure/Tests.jspa\#/testCase/LVV-T1564}{Verify re-run on similar system produces the same results} &  Draft \tabularnewline
    \hyperref[lvv-t1612]{LVV-T1612} &
    \href{https://jira.lsstcorp.org/secure/Tests.jspa\#/testCase/LVV-T1612}{Verify Summit - Base Network Integration (System Level)} &  Draft \tabularnewline
    \hyperref[lvv-t1830]{LVV-T1830} &
    \href{https://jira.lsstcorp.org/secure/Tests.jspa\#/testCase/LVV-T1830}{Verify Implementation of Scientific Visualization of Camera Image Data} &  Draft \tabularnewline
    \hyperref[lvv-t1831]{LVV-T1831} &
    \href{https://jira.lsstcorp.org/secure/Tests.jspa\#/testCase/LVV-T1831}{Verify Implementation of Data Management Nightly Reporting} &  Draft \tabularnewline
    \hyperref[lvv-t1836]{LVV-T1836} &
    \href{https://jira.lsstcorp.org/secure/Tests.jspa\#/testCase/LVV-T1836}{Verify calculation of resolved-to-unresolved flux ratio errors} &  Draft \tabularnewline
    \hyperref[lvv-t1837]{LVV-T1837} &
    \href{https://jira.lsstcorp.org/secure/Tests.jspa\#/testCase/LVV-T1837}{Verify calculation of band-to-band color zero-point accuracy} &  Draft \tabularnewline
    \hyperref[lvv-t1838]{LVV-T1838} &
    \href{https://jira.lsstcorp.org/secure/Tests.jspa\#/testCase/LVV-T1838}{Verify calculation of image fraction affected by ghosts} &  Draft \tabularnewline
    \hyperref[lvv-t1839]{LVV-T1839} &
    \href{https://jira.lsstcorp.org/secure/Tests.jspa\#/testCase/LVV-T1839}{Verify calculation of RMS width of photometric zeropoint} &  Draft \tabularnewline
    \hyperref[lvv-t1840]{LVV-T1840} &
    \href{https://jira.lsstcorp.org/secure/Tests.jspa\#/testCase/LVV-T1840}{Verify calculation of sky brightness precision} &  Draft \tabularnewline
    \hyperref[lvv-t1841]{LVV-T1841} &
    \href{https://jira.lsstcorp.org/secure/Tests.jspa\#/testCase/LVV-T1841}{Verify calculation of scientifically unusable pixel fraction} &  Draft \tabularnewline
    \hyperref[lvv-t1842]{LVV-T1842} &
    \href{https://jira.lsstcorp.org/secure/Tests.jspa\#/testCase/LVV-T1842}{Verify calculation of zeropoint error fraction exceeding the outlier
limit} &  Draft \tabularnewline
    \hyperref[lvv-t1843]{LVV-T1843} &
    \href{https://jira.lsstcorp.org/secure/Tests.jspa\#/testCase/LVV-T1843}{Verify calculation of significance of imperfect crosstalk corrections} &  Draft \tabularnewline
    \hyperref[lvv-t1844]{LVV-T1844} &
    \href{https://jira.lsstcorp.org/secure/Tests.jspa\#/testCase/LVV-T1844}{Verify calculation of u-band photometric zero-point RMS} &  Draft \tabularnewline
    \hyperref[lvv-t1845]{LVV-T1845} &
    \href{https://jira.lsstcorp.org/secure/Tests.jspa\#/testCase/LVV-T1845}{Verify accuracy of photometric transformation to physical scale} &  Draft \tabularnewline
    \hyperref[lvv-t1846]{LVV-T1846} &
    \href{https://jira.lsstcorp.org/secure/Tests.jspa\#/testCase/LVV-T1846}{Verify calculation of band-to-band color zero-point accuracy including
u-band} &  Draft \tabularnewline
    \hyperref[lvv-t1847]{LVV-T1847} &
    \href{https://jira.lsstcorp.org/secure/Tests.jspa\#/testCase/LVV-T1847}{Verify calculation of sensor fraction with unusable pixels} &  Draft \tabularnewline
    \hyperref[lvv-t1862]{LVV-T1862} &
    \href{https://jira.lsstcorp.org/secure/Tests.jspa\#/testCase/LVV-T1862}{Verify determining effectiveness of dark current frame} &  Draft \tabularnewline
    \hyperref[lvv-t1863]{LVV-T1863} &
    \href{https://jira.lsstcorp.org/secure/Tests.jspa\#/testCase/LVV-T1863}{Verify ability to process Special Programs data alongside normal
processing} &  Draft \tabularnewline
    \hyperref[lvv-t1865]{LVV-T1865} &
    \href{https://jira.lsstcorp.org/secure/Tests.jspa\#/testCase/LVV-T1865}{Verify implementation of time to L1 public release for Special Programs} &  Draft \tabularnewline
    \hyperref[lvv-t1866]{LVV-T1866} &
    \href{https://jira.lsstcorp.org/secure/Tests.jspa\#/testCase/LVV-T1866}{Verify latency of reporting optical transients from Special Programs} &  Draft \tabularnewline
    \hyperref[lvv-t1867]{LVV-T1867} &
    \href{https://jira.lsstcorp.org/secure/Tests.jspa\#/testCase/LVV-T1867}{Verify implementation of at least numStreams alert streams supported} &  Draft \tabularnewline
    \hyperref[lvv-t1868]{LVV-T1868} &
    \href{https://jira.lsstcorp.org/secure/Tests.jspa\#/testCase/LVV-T1868}{Verify implementation of alert streams distributed within latency limit} &  Draft \tabularnewline
\bottomrule
\end{longtable}

\newpage

\section{Active Test Cases}

This section documents all active test cases that have a status in the Jira/ATM system of Draft, Defined or Approved.

\subsection{ Defined Test Cases}

\subsubsection{LVV-T29 - Verify implementation of Raw Science Image Data Acquisition}\label{lvv-t29}

\begin{longtable}[]{llllll}
\toprule
Version & Status & Priority & Verification Type & Owner
\\\midrule
1 & Defined & Normal &
Test & Kian-Tat Lim
\\\bottomrule
\multicolumn{6}{c}{ Open \href{https://jira.lsstcorp.org/secure/Tests.jspa\#/testCase/LVV-T29}{LVV-T29} in Jira } \\
\end{longtable}

\paragraph{Verification Elements}\mbox{}\\

\begin{itemize}
\item \href{https://jira.lsstcorp.org/browse/LVV-8}{LVV-8} - DMS-REQ-0018-V-01: Raw Science Image Data Acquisition

\end{itemize}

\paragraph{Test Items}\mbox{}\\

Verify acquisition of raw data from L1 Test Stand DAQ while simulating
all modes








\paragraph{Test Procedure}\mbox{}\\
\begin{tabular}{p{4cm}p{12cm}}
\toprule
Step 1
& Description \\ \hline
\end{tabular}
{\scriptsize
{Ingest raw data from L1 Test Stand DAQ, simulating each observing
mode\\
}

}
\begin{tabular}{p{3cm}p{13cm}}
\hline
            & Expected Result \\ \hline
\end{tabular}

\begin{tabular}{p{4cm}p{12cm}}
\toprule
Step 2
& Description \\ \hline
\end{tabular}
{\scriptsize
O{bserve image and its metadata is present and queryable in the Data
Backbone.}

}
\begin{tabular}{p{3cm}p{13cm}}
\hline
            & Expected Result \\ \hline
\end{tabular}
{\scriptsize
Well-formed image data with appropriate associated metadata.

}

\subsubsection{LVV-T30 - Verify implementation of Wavefront Sensor Data Acquisition}\label{lvv-t30}

\begin{longtable}[]{llllll}
\toprule
Version & Status & Priority & Verification Type & Owner
\\\midrule
1 & Defined & Normal &
Test & Kian-Tat Lim
\\\bottomrule
\multicolumn{6}{c}{ Open \href{https://jira.lsstcorp.org/secure/Tests.jspa\#/testCase/LVV-T30}{LVV-T30} in Jira } \\
\end{longtable}

\paragraph{Verification Elements}\mbox{}\\

\begin{itemize}
\item \href{https://jira.lsstcorp.org/browse/LVV-9}{LVV-9} - DMS-REQ-0020-V-01: Wavefront Sensor Data Acquisition

\end{itemize}

\paragraph{Test Items}\mbox{}\\

Verify successful ingestion of wavefront sensor data from L1 Test Stand
DAQ while simulating all modes.








\paragraph{Test Procedure}\mbox{}\\
\begin{tabular}{p{4cm}p{12cm}}
\toprule
Step 1
& Description \\ \hline
\end{tabular}
{\scriptsize
{Ingest wavefront sensor data from L1 Test Stand DAQ while simulating
all modes}

}
\begin{tabular}{p{3cm}p{13cm}}
\hline
            & Expected Result \\ \hline
\end{tabular}

\begin{tabular}{p{4cm}p{12cm}}
\toprule
Step 2
& Description \\ \hline
\end{tabular}
{\scriptsize
Observe wavefront sensor data and metadata archived in the Data
Backbone.

}
\begin{tabular}{p{3cm}p{13cm}}
\hline
            & Expected Result \\ \hline
\end{tabular}
{\scriptsize
Well-formed wavefront sensor image data with appropriate associated
metadata.

}

\subsubsection{LVV-T32 - Verify implementation of Raw Image Assembly}\label{lvv-t32}

\begin{longtable}[]{llllll}
\toprule
Version & Status & Priority & Verification Type & Owner
\\\midrule
1 & Defined & Normal &
Test & Kian-Tat Lim
\\\bottomrule
\multicolumn{6}{c}{ Open \href{https://jira.lsstcorp.org/secure/Tests.jspa\#/testCase/LVV-T32}{LVV-T32} in Jira } \\
\end{longtable}

\paragraph{Verification Elements}\mbox{}\\

\begin{itemize}
\item \href{https://jira.lsstcorp.org/browse/LVV-11}{LVV-11} - DMS-REQ-0024-V-01: Raw Image Assembly

\end{itemize}

\paragraph{Test Items}\mbox{}\\

Verify that the raw exposure data from all readout channels in a sensor
can be assembled into a single image, and that all required/relevant
metadata are associated with the image data.~








\paragraph{Test Procedure}\mbox{}\\
\begin{tabular}{p{4cm}p{12cm}}
\toprule
Step 1
& Description \\ \hline
\end{tabular}
{\scriptsize
Ingest data from the L1 Camera Test Stand DAQ.

}
\begin{tabular}{p{3cm}p{13cm}}
\hline
            & Expected Result \\ \hline
\end{tabular}

\begin{tabular}{p{4cm}p{12cm}}
\toprule
Step 2
& Description \\ \hline
\end{tabular}
{\scriptsize
Simulate all different modes of data gathering.

}
\begin{tabular}{p{3cm}p{13cm}}
\hline
            & Expected Result \\ \hline
\end{tabular}

\begin{tabular}{p{4cm}p{12cm}}
\toprule
Step 3
& Description \\ \hline
\end{tabular}
{\scriptsize
Verify that a raw image is constructed in correct format.

}
\begin{tabular}{p{3cm}p{13cm}}
\hline
            & Expected Result \\ \hline
\end{tabular}
{\scriptsize
A single raw image combining data from all readout channels for a given
sensor.~

}

\begin{tabular}{p{4cm}p{12cm}}
\toprule
Step 4
& Description \\ \hline
\end{tabular}
{\scriptsize
Verify that a raw image is constructed with correct metadata.

}
\begin{tabular}{p{3cm}p{13cm}}
\hline
            & Expected Result \\ \hline
\end{tabular}
{\scriptsize
Image header or ancillary table contains the required metadata about the
observing context in which data were gathered.

}

\subsubsection{LVV-T33 - Verify implementation of Raw Science Image Metadata}\label{lvv-t33}

\begin{longtable}[]{llllll}
\toprule
Version & Status & Priority & Verification Type & Owner
\\\midrule
1 & Defined & Normal &
Test & Kian-Tat Lim
\\\bottomrule
\multicolumn{6}{c}{ Open \href{https://jira.lsstcorp.org/secure/Tests.jspa\#/testCase/LVV-T33}{LVV-T33} in Jira } \\
\end{longtable}

\paragraph{Verification Elements}\mbox{}\\

\begin{itemize}
\item \href{https://jira.lsstcorp.org/browse/LVV-28}{LVV-28} - DMS-REQ-0068-V-01: Raw Science Image Metadata

\item \href{https://jira.lsstcorp.org/browse/LVV-1234}{LVV-1234} - OSS-REQ-0122-V-01: Provenance

\end{itemize}

\paragraph{Test Items}\mbox{}\\

Verify successful ingestion of raw data from L1 Test Stand DAQ and that
image metadata is present and queryable.


\paragraph{Predecessors}\mbox{}\\
\href{https://jira.lsstcorp.org/secure/Tests.jspa\#/testCase/LVV-T29}{LVV-T29},
​\href{https://jira.lsstcorp.org/secure/Tests.jspa\#/testCase/LVV-T32}{LVV-T32}​​​






\paragraph{Test Procedure}\mbox{}\\
\begin{tabular}{p{4cm}p{12cm}}
\toprule
Step 1
& Description \\ \hline
\end{tabular}
{\scriptsize
Identify (or gather) a dataset of raw science images.

}
\begin{tabular}{p{3cm}p{13cm}}
\hline
            & Expected Result \\ \hline
\end{tabular}

\begin{tabular}{p{4cm}p{12cm}}
\toprule
Step 2
& Description \\ \hline
\end{tabular}
{\scriptsize
Verify that time of exposure start/end, site metadata, telescope
metadata, and camera metadata are stored in DMS
system.\\[2\baselineskip]

}
\begin{tabular}{p{3cm}p{13cm}}
\hline
            & Expected Result \\ \hline
\end{tabular}
{\scriptsize
Raw image data contain the required metadata.

}

\subsubsection{LVV-T34 - Verify implementation of Guider Calibration Data Acquisition}\label{lvv-t34}

\begin{longtable}[]{llllll}
\toprule
Version & Status & Priority & Verification Type & Owner
\\\midrule
1 & Defined & Normal &
Test & Kian-Tat Lim
\\\bottomrule
\multicolumn{6}{c}{ Open \href{https://jira.lsstcorp.org/secure/Tests.jspa\#/testCase/LVV-T34}{LVV-T34} in Jira } \\
\end{longtable}

\paragraph{Verification Elements}\mbox{}\\

\begin{itemize}
\item \href{https://jira.lsstcorp.org/browse/LVV-96}{LVV-96} - DMS-REQ-0265-V-01: Guider Calibration Data Acquisition

\end{itemize}

\paragraph{Test Items}\mbox{}\\

{Verify successful}\\
{~1. Ingestion of calibration frames from L1 Test Stand DAQ}\\
{~2. Execution of CPP payloads}\\
{~3. Availability of observed guider calibration products}








\paragraph{Test Procedure}\mbox{}\\
\begin{tabular}{p{4cm}p{12cm}}
\toprule
Step 1
& Description \\ \hline
\end{tabular}
{\scriptsize
{Ingest calibration frames for the guider sensors from L1 Test Stand
DAQ}

}
\begin{tabular}{p{3cm}p{13cm}}
\hline
            & Expected Result \\ \hline
\end{tabular}

\begin{tabular}{p{4cm}p{12cm}}
\toprule
Step 2-1
{\scriptsize from \hyperref[lvv-t1060]{LVV-T1060} }
& Description \\ \hline
\end{tabular}
{\scriptsize
Execute the Calibration Products Production payload. The payload uses
raw calibration images and information from the Transformed EFD to
generate a subset of Master Calibration Images and Calibration Database
entries in the Data Backbone.

}
\begin{tabular}{p{3cm}p{13cm}}
\hline
            & Expected Result \\ \hline
\end{tabular}

\begin{tabular}{p{4cm}p{12cm}}
\toprule
Step 2-2
{\scriptsize from \hyperref[lvv-t1060]{LVV-T1060} }
& Description \\ \hline
\end{tabular}
{\scriptsize
Confirm that the expected Master Calibration images and Calibration
Database entries are present and well-formed.

}
\begin{tabular}{p{3cm}p{13cm}}
\hline
            & Expected Result \\ \hline
\end{tabular}

\begin{tabular}{p{4cm}p{12cm}}
\toprule
Step 3
& Description \\ \hline
\end{tabular}
{\scriptsize
Observe that guider calibration products have been produced.

}
\begin{tabular}{p{3cm}p{13cm}}
\hline
            & Expected Result \\ \hline
\end{tabular}
{\scriptsize
Well-formed calibration frames for the guider sensors.

}

\subsubsection{LVV-T38 - Verify implementation of Processed Visit Images}\label{lvv-t38}

\begin{longtable}[]{llllll}
\toprule
Version & Status & Priority & Verification Type & Owner
\\\midrule
1 & Defined & Normal &
Test & Eric Bellm
\\\bottomrule
\multicolumn{6}{c}{ Open \href{https://jira.lsstcorp.org/secure/Tests.jspa\#/testCase/LVV-T38}{LVV-T38} in Jira } \\
\end{longtable}

\paragraph{Verification Elements}\mbox{}\\

\begin{itemize}
\item \href{https://jira.lsstcorp.org/browse/LVV-29}{LVV-29} - DMS-REQ-0069-V-01: Processed Visit Images

\end{itemize}

\paragraph{Test Items}\mbox{}\\

Verify that the DMS\\
1. Successfully produces Processed Visit Images, where the instrument
signature has been removed.\\
2. Successfully combines images obtained during a standard visit.








\paragraph{Test Procedure}\mbox{}\\
\begin{tabular}{p{4cm}p{12cm}}
\toprule
Step 1
& Description \\ \hline
\end{tabular}
{\scriptsize
Identify suitable precursor datasets containing unprocessed raw images.

}
\begin{tabular}{p{3cm}p{13cm}}
\hline
            & Expected Result \\ \hline
\end{tabular}

\begin{tabular}{p{4cm}p{12cm}}
\toprule
Step 2-1
{\scriptsize from \hyperref[lvv-t987]{LVV-T987} }
& Description \\ \hline
\end{tabular}
{\scriptsize
Identify the path to the data repository, which we will refer to as
`DATA/path', then execute the following:

}
\begin{tabular}{p{3cm}p{13cm}}
\hline
            & Example Code \\ \hline
\end{tabular}
{\scriptsize
\begin{verbatim}
import lsst.daf.persistence as dafPersist
butler = dafPersist.Butler(inputs='DATA/path')
\end{verbatim}

}
\begin{tabular}{p{3cm}p{13cm}}
\hline
            & Expected Result \\ \hline
\end{tabular}
{\scriptsize
Butler repo available for reading.

}

\begin{tabular}{p{4cm}p{12cm}}
\toprule
Step 3
& Description \\ \hline
\end{tabular}
{\scriptsize
Run the Prompt Processing payload on these data. ~Verify that Processed
Visit Images are generated at correct size and with significant
instrumental artifacts removed.

}
\begin{tabular}{p{3cm}p{13cm}}
\hline
            & Expected Result \\ \hline
\end{tabular}
{\scriptsize
Raw precursor dataset images have been processed into Processed Visit
Images, with instrumental artifacts corrected.

}

\subsubsection{LVV-T42 - Verify implementation of Processed Visit Image Content}\label{lvv-t42}

\begin{longtable}[]{llllll}
\toprule
Version & Status & Priority & Verification Type & Owner
\\\midrule
1 & Defined & Normal &
Test & Jim Bosch
\\\bottomrule
\multicolumn{6}{c}{ Open \href{https://jira.lsstcorp.org/secure/Tests.jspa\#/testCase/LVV-T42}{LVV-T42} in Jira } \\
\end{longtable}

\paragraph{Verification Elements}\mbox{}\\

\begin{itemize}
\item \href{https://jira.lsstcorp.org/browse/LVV-31}{LVV-31} - DMS-REQ-0072-V-01: Processed Visit Image Content

\end{itemize}

\paragraph{Test Items}\mbox{}\\

Verify that Processed Visit Images produced by the DRP and AP pipelines
include the observed data, a mask array, a variance array, a PSF model,
and a WCS model.








\paragraph{Test Procedure}\mbox{}\\
\begin{tabular}{p{4cm}p{12cm}}
\toprule
Step 1-1
{\scriptsize from \hyperref[lvv-t987]{LVV-T987} }
& Description \\ \hline
\end{tabular}
{\scriptsize
Identify the path to the data repository, which we will refer to as
`DATA/path', then execute the following:

}
\begin{tabular}{p{3cm}p{13cm}}
\hline
            & Example Code \\ \hline
\end{tabular}
{\scriptsize
\begin{verbatim}
import lsst.daf.persistence as dafPersist
butler = dafPersist.Butler(inputs='DATA/path')
\end{verbatim}

}
\begin{tabular}{p{3cm}p{13cm}}
\hline
            & Expected Result \\ \hline
\end{tabular}
{\scriptsize
Butler repo available for reading.

}

\begin{tabular}{p{4cm}p{12cm}}
\toprule
Step 2
& Description \\ \hline
\end{tabular}
{\scriptsize
Ingest the data from an appropriate processed dataset.

}
\begin{tabular}{p{3cm}p{13cm}}
\hline
            & Expected Result \\ \hline
\end{tabular}

\begin{tabular}{p{4cm}p{12cm}}
\toprule
Step 3
& Description \\ \hline
\end{tabular}
{\scriptsize
Select a single visit from the dataset, and extract its WCS object,
calexp image, psf model, and source list.

}
\begin{tabular}{p{3cm}p{13cm}}
\hline
            & Expected Result \\ \hline
\end{tabular}

\begin{tabular}{p{4cm}p{12cm}}
\toprule
Step 4
& Description \\ \hline
\end{tabular}
{\scriptsize
Inspect the calexp image to ensure that

\begin{enumerate}
\tightlist
\item
  A well-formed image is present,
\item
  The variance plane is present and well-behaved,
\item
  Mask planes are present and contain information about defects.
\end{enumerate}

}
\begin{tabular}{p{3cm}p{13cm}}
\hline
            & Expected Result \\ \hline
\end{tabular}
{\scriptsize
An astronomical image with mask and variance planes. This can be readily
visualized using Firefly, which displays mask planes by default.

}

\begin{tabular}{p{4cm}p{12cm}}
\toprule
Step 5
& Description \\ \hline
\end{tabular}
{\scriptsize
Plot images of the PSF model at various points, and verify that the PSF
differs with position.

}
\begin{tabular}{p{3cm}p{13cm}}
\hline
            & Expected Result \\ \hline
\end{tabular}
{\scriptsize
A ``star-like'' image of the PSF evaluated at various positions. The PSF
should vary slightly with position (this could be readily visualized by
taking a difference of PSFs at two positions).

}

\begin{tabular}{p{4cm}p{12cm}}
\toprule
Step 6
& Description \\ \hline
\end{tabular}
{\scriptsize
Starting from the XY pixel coordinates of the sources, apply the WCS to
obtain RA, Dec coordinates. Plot these positions and confirm that they
match the expected values from the WCS object.

}
\begin{tabular}{p{3cm}p{13cm}}
\hline
            & Expected Result \\ \hline
\end{tabular}
{\scriptsize
RA, Dec coordinates that are returned should be near the central
position of the visit coordinate as given in either the calexp metadata
or the WCS.

}

\begin{tabular}{p{4cm}p{12cm}}
\toprule
Step 7
& Description \\ \hline
\end{tabular}
{\scriptsize
Repeat steps 2-6, but now with difference images created by the Alert
Production pipeline (for example, in the `ap\_verify` test data
processing).

}
\begin{tabular}{p{3cm}p{13cm}}
\hline
            & Expected Result \\ \hline
\end{tabular}

\subsubsection{LVV-T45 - Verify implementation of Prompt Processing Data Quality Report
Definition}\label{lvv-t45}

\begin{longtable}[]{llllll}
\toprule
Version & Status & Priority & Verification Type & Owner
\\\midrule
1 & Defined & Normal &
Test & Eric Bellm
\\\bottomrule
\multicolumn{6}{c}{ Open \href{https://jira.lsstcorp.org/secure/Tests.jspa\#/testCase/LVV-T45}{LVV-T45} in Jira } \\
\end{longtable}

\paragraph{Verification Elements}\mbox{}\\

\begin{itemize}
\item \href{https://jira.lsstcorp.org/browse/LVV-39}{LVV-39} - DMS-REQ-0097-V-01: Level 1 Data Quality Report Definition

\end{itemize}

\paragraph{Test Items}\mbox{}\\

Verify that the DMS produces a Prompt Processing Data Quality Report.
~Specifically check absolute value and temporal variation of\\
1. Photometric zeropoint\\
2. Sky brightness\\
3. Seeing\\
4. PSF\\
5. Detection efficiency








\paragraph{Test Procedure}\mbox{}\\
\begin{tabular}{p{4cm}p{12cm}}
\toprule
Step 1
& Description \\ \hline
\end{tabular}
{\scriptsize
Ingest raw data from L1 Test Stand DAQ.

}
\begin{tabular}{p{3cm}p{13cm}}
\hline
            & Expected Result \\ \hline
\end{tabular}

\begin{tabular}{p{4cm}p{12cm}}
\toprule
Step 2-1
{\scriptsize from \hyperref[lvv-t866]{LVV-T866} }
& Description \\ \hline
\end{tabular}
{\scriptsize
Perform the steps of Alert Production (including, but not necessarily
limited to, single frame processing, ISR, source detection/measurement,
PSF estimation, photometric and astrometric calibration, difference
imaging, DIASource detection/measurement, source association). During
Operations, it is presumed that these are automated for a given
dataset.~

}
\begin{tabular}{p{3cm}p{13cm}}
\hline
            & Expected Result \\ \hline
\end{tabular}
{\scriptsize
An output dataset including difference images and DIASource and
DIAObject measurements.

}

\begin{tabular}{p{4cm}p{12cm}}
\toprule
Step 2-2
{\scriptsize from \hyperref[lvv-t866]{LVV-T866} }
& Description \\ \hline
\end{tabular}
{\scriptsize
Verify that the expected data products have been produced, and that
catalogs contain reasonable values for measured quantities of interest.

}
\begin{tabular}{p{3cm}p{13cm}}
\hline
            & Expected Result \\ \hline
\end{tabular}

\begin{tabular}{p{4cm}p{12cm}}
\toprule
Step 3
& Description \\ \hline
\end{tabular}
{\scriptsize
Load the Prompt Processing QC reports, and observe that a dynamically
updated Data Quality Report has become available at the relevant UI.

}
\begin{tabular}{p{3cm}p{13cm}}
\hline
            & Expected Result \\ \hline
\end{tabular}
{\scriptsize
A Prompt Processing QC report is available via a UI, and contains
information about the photometric zeropoint, sky brightness, seeing,
PSF, and detection efficiency, and possibly other relevant quantities.

}

\begin{tabular}{p{4cm}p{12cm}}
\toprule
Step 4
& Description \\ \hline
\end{tabular}
{\scriptsize
Check that a static report is created and archived in a
readily-accessible location.

}
\begin{tabular}{p{3cm}p{13cm}}
\hline
            & Expected Result \\ \hline
\end{tabular}
{\scriptsize
Persistence of a static QC report in an accessible location, containing
the same information as in the report from Step 3.

}

\subsubsection{LVV-T47 - Verify implementation of Prompt Processing Calibration Report Definition}\label{lvv-t47}

\begin{longtable}[]{llllll}
\toprule
Version & Status & Priority & Verification Type & Owner
\\\midrule
1 & Defined & Normal &
Test & Eric Bellm
\\\bottomrule
\multicolumn{6}{c}{ Open \href{https://jira.lsstcorp.org/secure/Tests.jspa\#/testCase/LVV-T47}{LVV-T47} in Jira } \\
\end{longtable}

\paragraph{Verification Elements}\mbox{}\\

\begin{itemize}
\item \href{https://jira.lsstcorp.org/browse/LVV-43}{LVV-43} - DMS-REQ-0101-V-01: Level 1 Calibration Report Definition

\end{itemize}

\paragraph{Test Items}\mbox{}\\

Verify that the DMS produces a Prompt Processing Calibration Report.
~Specifically check that this report is capable of identifying when
aspects of the telescope or camera are changing with time.








\paragraph{Test Procedure}\mbox{}\\
\begin{tabular}{p{4cm}p{12cm}}
\toprule
Step 1
& Description \\ \hline
\end{tabular}
{\scriptsize
Identify precursor and simulated calibration datasets on which to run
the L1 calibration pipeline.

}
\begin{tabular}{p{3cm}p{13cm}}
\hline
            & Expected Result \\ \hline
\end{tabular}

\begin{tabular}{p{4cm}p{12cm}}
\toprule
Step 2-1
{\scriptsize from \hyperref[lvv-t1059]{LVV-T1059} }
& Description \\ \hline
\end{tabular}
{\scriptsize
Execute the Daily Calibration Products Update payload. The payload uses
raw calibration images and information from the Transformed EFD to
generate a subset of Master Calibration Images and Calibration Database
entries in the Data Backbone.

}
\begin{tabular}{p{3cm}p{13cm}}
\hline
            & Expected Result \\ \hline
\end{tabular}

\begin{tabular}{p{4cm}p{12cm}}
\toprule
Step 2-2
{\scriptsize from \hyperref[lvv-t1059]{LVV-T1059} }
& Description \\ \hline
\end{tabular}
{\scriptsize
Confirm that the expected Master Calibration images and Calibration
Database entries are present and well-formed.

}
\begin{tabular}{p{3cm}p{13cm}}
\hline
            & Expected Result \\ \hline
\end{tabular}

\begin{tabular}{p{4cm}p{12cm}}
\toprule
Step 3
& Description \\ \hline
\end{tabular}
{\scriptsize
Check that a dynamic report is created that triggers alerts if
calibrations go out of range.~

}
\begin{tabular}{p{3cm}p{13cm}}
\hline
            & Expected Result \\ \hline
\end{tabular}
{\scriptsize
A dynamic report is available via UI to users, and if any out-of-spec
changes have occurred, alerts have been issued.

}

\begin{tabular}{p{4cm}p{12cm}}
\toprule
Step 4
& Description \\ \hline
\end{tabular}
{\scriptsize
Check that a static report is created and archived in a
readily-accessible location.

}
\begin{tabular}{p{3cm}p{13cm}}
\hline
            & Expected Result \\ \hline
\end{tabular}
{\scriptsize
An archived version of the calibration report is available and will be
retained in a static file format.

}

\subsubsection{LVV-T48 - Verify implementation of Exposure Catalog}\label{lvv-t48}

\begin{longtable}[]{llllll}
\toprule
Version & Status & Priority & Verification Type & Owner
\\\midrule
1 & Defined & Normal &
Test & Jim Bosch
\\\bottomrule
\multicolumn{6}{c}{ Open \href{https://jira.lsstcorp.org/secure/Tests.jspa\#/testCase/LVV-T48}{LVV-T48} in Jira } \\
\end{longtable}

\paragraph{Verification Elements}\mbox{}\\

\begin{itemize}
\item \href{https://jira.lsstcorp.org/browse/LVV-97}{LVV-97} - DMS-REQ-0266-V-01: Exposure Catalog

\end{itemize}

\paragraph{Test Items}\mbox{}\\

Verify that the DMS creates an Exposure Catalog that includes\\
1. Observation datetime, exposure time\\
2. Filter\\
3. Dome, telescope orientation and status\\
4. Calibration status\\
5. Airmass and zenith\\
6. Environmental information\\
7. Per-sensor information








\paragraph{Test Procedure}\mbox{}\\
\begin{tabular}{p{4cm}p{12cm}}
\toprule
Step 1
& Description \\ \hline
\end{tabular}
{\scriptsize
Verify that Exposure Catalogs contain the required elements. At present,
the form of the exposure catalog is not defined. This information can be
found for a given Butler repo from the metadata, but will ultimately be
aggregated into a database/table summarizing available exposures.

}
\begin{tabular}{p{3cm}p{13cm}}
\hline
            & Expected Result \\ \hline
\end{tabular}
{\scriptsize
A list of the required metadata for a set of exposures is returned and
both human- and machine-readable.

}

\subsubsection{LVV-T61 - Verify implementation of Associate Sources to Objects}\label{lvv-t61}

\begin{longtable}[]{llllll}
\toprule
Version & Status & Priority & Verification Type & Owner
\\\midrule
1 & Defined & Normal &
Test & Jim Bosch
\\\bottomrule
\multicolumn{6}{c}{ Open \href{https://jira.lsstcorp.org/secure/Tests.jspa\#/testCase/LVV-T61}{LVV-T61} in Jira } \\
\end{longtable}

\paragraph{Verification Elements}\mbox{}\\

\begin{itemize}
\item \href{https://jira.lsstcorp.org/browse/LVV-16}{LVV-16} - DMS-REQ-0034-V-01: Associate Sources to Objects

\end{itemize}

\paragraph{Test Items}\mbox{}\\

Verify that each Source record contains an ID that associates it with a
best guess at the Object it corresponds to.








\paragraph{Test Procedure}\mbox{}\\
\begin{tabular}{p{4cm}p{12cm}}
\toprule
Step 1-1
{\scriptsize from \hyperref[lvv-t987]{LVV-T987} }
& Description \\ \hline
\end{tabular}
{\scriptsize
Identify the path to the data repository, which we will refer to as
`DATA/path', then execute the following:

}
\begin{tabular}{p{3cm}p{13cm}}
\hline
            & Example Code \\ \hline
\end{tabular}
{\scriptsize
\begin{verbatim}
import lsst.daf.persistence as dafPersist
butler = dafPersist.Butler(inputs='DATA/path')
\end{verbatim}

}
\begin{tabular}{p{3cm}p{13cm}}
\hline
            & Expected Result \\ \hline
\end{tabular}
{\scriptsize
Butler repo available for reading.

}

\begin{tabular}{p{4cm}p{12cm}}
\toprule
Step 2
& Description \\ \hline
\end{tabular}
{\scriptsize
Read a dataset via the Butler and extract its source and object
catalogs.

}
\begin{tabular}{p{3cm}p{13cm}}
\hline
            & Expected Result \\ \hline
\end{tabular}

\begin{tabular}{p{4cm}p{12cm}}
\toprule
Step 3
& Description \\ \hline
\end{tabular}
{\scriptsize
Verify that sources have objects

}
\begin{tabular}{p{3cm}p{13cm}}
\hline
            & Expected Result \\ \hline
\end{tabular}

\begin{tabular}{p{4cm}p{12cm}}
\toprule
Step 4
& Description \\ \hline
\end{tabular}
{\scriptsize
Verify that objects list sources that seem reasonably near them.

}
\begin{tabular}{p{3cm}p{13cm}}
\hline
            & Expected Result \\ \hline
\end{tabular}

\subsubsection{LVV-T65 - Verify implementation of Source Catalog}\label{lvv-t65}

\begin{longtable}[]{llllll}
\toprule
Version & Status & Priority & Verification Type & Owner
\\\midrule
1 & Defined & Normal &
Test & Jim Bosch
\\\bottomrule
\multicolumn{6}{c}{ Open \href{https://jira.lsstcorp.org/secure/Tests.jspa\#/testCase/LVV-T65}{LVV-T65} in Jira } \\
\end{longtable}

\paragraph{Verification Elements}\mbox{}\\

\begin{itemize}
\item \href{https://jira.lsstcorp.org/browse/LVV-98}{LVV-98} - DMS-REQ-0267-V-01: Source Catalog

\end{itemize}

\paragraph{Test Items}\mbox{}\\

Verify that all Sources produced by the DRP pipelines contain the
entries listed in DMS-REQ-0267.








\paragraph{Test Procedure}\mbox{}\\
\begin{tabular}{p{4cm}p{12cm}}
\toprule
Step 1
& Description \\ \hline
\end{tabular}
{\scriptsize
Identify a suitable small dataset to process through the DRP.

}
\begin{tabular}{p{3cm}p{13cm}}
\hline
            & Expected Result \\ \hline
\end{tabular}

\begin{tabular}{p{4cm}p{12cm}}
\toprule
Step 2-1
{\scriptsize from \hyperref[lvv-t1064]{LVV-T1064} }
& Description \\ \hline
\end{tabular}
{\scriptsize
Process data with the Data Release Production payload, starting from raw
science images and generating science data products, placing them in the
Data Backbone.

}
\begin{tabular}{p{3cm}p{13cm}}
\hline
            & Expected Result \\ \hline
\end{tabular}

\begin{tabular}{p{4cm}p{12cm}}
\toprule
Step 3
& Description \\ \hline
\end{tabular}
{\scriptsize
Confirm that source catalogs have been produced for single visits and
coadds, and that it contains the required measurements.

}
\begin{tabular}{p{3cm}p{13cm}}
\hline
            & Expected Result \\ \hline
\end{tabular}
{\scriptsize
A source catalog containing the measured attributes (and associated
errors), including location on the focal plane; a static point-source
model fit to world coordinates and flux; a centroid and adaptive
moments; and surface brightnesses through elliptical multiple apertures
that are concentric, PSF-homogenized, and logarithmically spaced in
intensity.

}

\subsubsection{LVV-T82 - Verify implementation of Tracking Characterization Changes Between Data
Releases}\label{lvv-t82}

\begin{longtable}[]{llllll}
\toprule
Version & Status & Priority & Verification Type & Owner
\\\midrule
1 & Defined & Normal &
Test & Jim Bosch
\\\bottomrule
\multicolumn{6}{c}{ Open \href{https://jira.lsstcorp.org/secure/Tests.jspa\#/testCase/LVV-T82}{LVV-T82} in Jira } \\
\end{longtable}

\paragraph{Verification Elements}\mbox{}\\

\begin{itemize}
\item \href{https://jira.lsstcorp.org/browse/LVV-170}{LVV-170} - DMS-REQ-0339-V-01: Tracking Characterization Changes Between Data
Releases

\end{itemize}

\paragraph{Test Items}\mbox{}\\

Verify that small-area subsets of a DR can be retained when most of that
DR is retired, for comparison with future DRs.








\paragraph{Test Procedure}\mbox{}\\
\begin{tabular}{p{4cm}p{12cm}}
\toprule
Step 1
& Description \\ \hline
\end{tabular}
{\scriptsize
Prepare a second DRP run -\textgreater{} DPDD with different
configuration parameters for this second test Data Release.

}
\begin{tabular}{p{3cm}p{13cm}}
\hline
            & Expected Result \\ \hline
\end{tabular}

\begin{tabular}{p{4cm}p{12cm}}
\toprule
Step 2-1
{\scriptsize from \hyperref[lvv-t1064]{LVV-T1064} }
& Description \\ \hline
\end{tabular}
{\scriptsize
Process data with the Data Release Production payload, starting from raw
science images and generating science data products, placing them in the
Data Backbone.

}
\begin{tabular}{p{3cm}p{13cm}}
\hline
            & Expected Result \\ \hline
\end{tabular}

\begin{tabular}{p{4cm}p{12cm}}
\toprule
Step 3
& Description \\ \hline
\end{tabular}
{\scriptsize
Stage subset of products from first test Data Release to separate
storage.

}
\begin{tabular}{p{3cm}p{13cm}}
\hline
            & Expected Result \\ \hline
\end{tabular}

\begin{tabular}{p{4cm}p{12cm}}
\toprule
Step 4
& Description \\ \hline
\end{tabular}
{\scriptsize
Scientifically compare the results of the subset of that region of sky
to those in the second test Data Release comparing the results of the
DRP Scientific Verification tests.

}
\begin{tabular}{p{3cm}p{13cm}}
\hline
            & Expected Result \\ \hline
\end{tabular}
{\scriptsize
Diagnostic plots quantifying the differences between scientific outputs
between the first and second test datasets.

}

\subsubsection{LVV-T83 - Verify implementation of Bad Pixel Map}\label{lvv-t83}

\begin{longtable}[]{llllll}
\toprule
Version & Status & Priority & Verification Type & Owner
\\\midrule
1 & Defined & Normal &
Test & Robert Lupton
\\\bottomrule
\multicolumn{6}{c}{ Open \href{https://jira.lsstcorp.org/secure/Tests.jspa\#/testCase/LVV-T83}{LVV-T83} in Jira } \\
\end{longtable}

\paragraph{Verification Elements}\mbox{}\\

\begin{itemize}
\item \href{https://jira.lsstcorp.org/browse/LVV-22}{LVV-22} - DMS-REQ-0059-V-01: Bad Pixel Map

\end{itemize}

\paragraph{Test Items}\mbox{}\\

Verify that the DMS can produce a map of detector pixels that suffer
from pathologies, and that these pathologies are encoded in at least
32-bit values.








\paragraph{Test Procedure}\mbox{}\\
\begin{tabular}{p{4cm}p{12cm}}
\toprule
Step 1
& Description \\ \hline
\end{tabular}
{\scriptsize
Interrogate the calibRegistry for the metadata associated with a bad
pixel map, where the validity range contains the date of interest.

}
\begin{tabular}{p{3cm}p{13cm}}
\hline
            & Expected Result \\ \hline
\end{tabular}
{\scriptsize
A bad pixel map for the requested date has been returned.

}

\begin{tabular}{p{4cm}p{12cm}}
\toprule
Step 2
& Description \\ \hline
\end{tabular}
{\scriptsize
Check that the bad pixel pathologies are encoded as at least 32-bit
values, and that the various pathologies are represented by different
encoding.

}
\begin{tabular}{p{3cm}p{13cm}}
\hline
            & Expected Result \\ \hline
\end{tabular}
{\scriptsize
Bad pixel values can be decoded to determine their pathologies using
their 32-bit values.

}

\subsubsection{LVV-T84 - Verify implementation of Bias Residual Image}\label{lvv-t84}

\begin{longtable}[]{llllll}
\toprule
Version & Status & Priority & Verification Type & Owner
\\\midrule
1 & Defined & Normal &
Test & Robert Lupton
\\\bottomrule
\multicolumn{6}{c}{ Open \href{https://jira.lsstcorp.org/secure/Tests.jspa\#/testCase/LVV-T84}{LVV-T84} in Jira } \\
\end{longtable}

\paragraph{Verification Elements}\mbox{}\\

\begin{itemize}
\item \href{https://jira.lsstcorp.org/browse/LVV-23}{LVV-23} - DMS-REQ-0060-V-01: Bias Residual Image

\end{itemize}

\paragraph{Test Items}\mbox{}\\

Verify that DMS can construct a bias residual image that corrects for
temporally-stable bias structures.\\
Verify that DMS can do this on demand.








\paragraph{Test Procedure}\mbox{}\\
\begin{tabular}{p{4cm}p{12cm}}
\toprule
Step 1
& Description \\ \hline
\end{tabular}
{\scriptsize
Identify the location of an appropriate precursor dataset.

}
\begin{tabular}{p{3cm}p{13cm}}
\hline
            & Expected Result \\ \hline
\end{tabular}

\begin{tabular}{p{4cm}p{12cm}}
\toprule
Step 2-1
{\scriptsize from \hyperref[lvv-t987]{LVV-T987} }
& Description \\ \hline
\end{tabular}
{\scriptsize
Identify the path to the data repository, which we will refer to as
`DATA/path', then execute the following:

}
\begin{tabular}{p{3cm}p{13cm}}
\hline
            & Example Code \\ \hline
\end{tabular}
{\scriptsize
\begin{verbatim}
import lsst.daf.persistence as dafPersist
butler = dafPersist.Butler(inputs='DATA/path')
\end{verbatim}

}
\begin{tabular}{p{3cm}p{13cm}}
\hline
            & Expected Result \\ \hline
\end{tabular}
{\scriptsize
Butler repo available for reading.

}

\begin{tabular}{p{4cm}p{12cm}}
\toprule
Step 3
& Description \\ \hline
\end{tabular}
{\scriptsize
Import the standard libraries required for the rest of this test:

}
\begin{tabular}{p{3cm}p{13cm}}
\hline
            & Example Code \\ \hline
\end{tabular}
{\scriptsize
import osimport lsst.afw.display as afwDisplay\\
from lsst.daf.persistence import Butler\\
from lsst.ip.isr import IsrTask\\
from firefly\_client import FireflyClient\\
from IPython.display import IFrame

}
\begin{tabular}{p{3cm}p{13cm}}
\hline
            & Expected Result \\ \hline
\end{tabular}

\begin{tabular}{p{4cm}p{12cm}}
\toprule
Step 4
& Description \\ \hline
\end{tabular}
{\scriptsize
Ingest the dataset from step 1 using the Butler (e.g., following example
code below).

}
\begin{tabular}{p{3cm}p{13cm}}
\hline
            & Example Code \\ \hline
\end{tabular}
{\scriptsize
butler = Butler(\$REPOSITORY\_PATH)\\
raw = butler.get(``raw'', visit=\$VISIT\_ID, detector=2)\\
bias = butler.get(``bias'', visit=\$VISIT\_ID, detector=2)

}
\begin{tabular}{p{3cm}p{13cm}}
\hline
            & Expected Result \\ \hline
\end{tabular}

\begin{tabular}{p{4cm}p{12cm}}
\toprule
Step 5
& Description \\ \hline
\end{tabular}
{\scriptsize
Display the bias image and inspect that its pixels contain unique
values.

}
\begin{tabular}{p{3cm}p{13cm}}
\hline
            & Expected Result \\ \hline
\end{tabular}
{\scriptsize
A relatively flat image showing the bias level with roughly Poisson
noise.

}

\begin{tabular}{p{4cm}p{12cm}}
\toprule
Step 6
& Description \\ \hline
\end{tabular}
{\scriptsize
Configure and run an Instrument Signature Removal (ISR) task on the raw
data. Most corrections are disabled for simplicity, but the bias frame
is applied.\\[2\baselineskip]

}
\begin{tabular}{p{3cm}p{13cm}}
\hline
            & Example Code \\ \hline
\end{tabular}
{\scriptsize
isr\_config = IsrTask.ConfigClass()\\
isr\_config.doDark=False\\
isr\_config.doFlat=False\\
isr\_config.doFringe=False\\
isr\_config.doDefect=False\\
isr\_config.doAddDistortionModel=False\\
isr\_config.doLinearize=False\\
isr = IsrTask(config=isr\_config)\\
result = isr.run(raw, bias=bias)

}
\begin{tabular}{p{3cm}p{13cm}}
\hline
            & Expected Result \\ \hline
\end{tabular}
{\scriptsize
A trimmed, bias-corrected image in `result`.

}

\begin{tabular}{p{4cm}p{12cm}}
\toprule
Step 7
& Description \\ \hline
\end{tabular}
{\scriptsize
Display the `result` image and confirm that the bias correction has been
performed.

}
\begin{tabular}{p{3cm}p{13cm}}
\hline
            & Expected Result \\ \hline
\end{tabular}
{\scriptsize
A displayed image with bias removed (i.e., typical background counts
reduced relative to the raw frame).

}

\subsubsection{LVV-T85 - Verify implementation of Crosstalk Correction Matrix}\label{lvv-t85}

\begin{longtable}[]{llllll}
\toprule
Version & Status & Priority & Verification Type & Owner
\\\midrule
1 & Defined & Normal &
Test & Robert Lupton
\\\bottomrule
\multicolumn{6}{c}{ Open \href{https://jira.lsstcorp.org/secure/Tests.jspa\#/testCase/LVV-T85}{LVV-T85} in Jira } \\
\end{longtable}

\paragraph{Verification Elements}\mbox{}\\

\begin{itemize}
\item \href{https://jira.lsstcorp.org/browse/LVV-24}{LVV-24} - DMS-REQ-0061-V-01: Crosstalk Correction Matrix

\end{itemize}

\paragraph{Test Items}\mbox{}\\

Verify that the DMS can generate a cross-talk correction matrix from
appropriate calibration data.\\
Verify that the DMS can measure the effectiveness of the cross-talk
correction matrix.








\paragraph{Test Procedure}\mbox{}\\
\begin{tabular}{p{4cm}p{12cm}}
\toprule
Step 1
& Description \\ \hline
\end{tabular}
{\scriptsize
Identify an appropriate calibration dataset that can be used to derive
the crosstalk correction matrix.

}
\begin{tabular}{p{3cm}p{13cm}}
\hline
            & Expected Result \\ \hline
\end{tabular}

\begin{tabular}{p{4cm}p{12cm}}
\toprule
Step 2-1
{\scriptsize from \hyperref[lvv-t1060]{LVV-T1060} }
& Description \\ \hline
\end{tabular}
{\scriptsize
Execute the Calibration Products Production payload. The payload uses
raw calibration images and information from the Transformed EFD to
generate a subset of Master Calibration Images and Calibration Database
entries in the Data Backbone.

}
\begin{tabular}{p{3cm}p{13cm}}
\hline
            & Expected Result \\ \hline
\end{tabular}

\begin{tabular}{p{4cm}p{12cm}}
\toprule
Step 2-2
{\scriptsize from \hyperref[lvv-t1060]{LVV-T1060} }
& Description \\ \hline
\end{tabular}
{\scriptsize
Confirm that the expected Master Calibration images and Calibration
Database entries are present and well-formed.

}
\begin{tabular}{p{3cm}p{13cm}}
\hline
            & Expected Result \\ \hline
\end{tabular}

\begin{tabular}{p{4cm}p{12cm}}
\toprule
Step 3
& Description \\ \hline
\end{tabular}
{\scriptsize
Confirm that the crosstalk correction matrix is produced and persisted.

}
\begin{tabular}{p{3cm}p{13cm}}
\hline
            & Expected Result \\ \hline
\end{tabular}
{\scriptsize
A correction matrix quantifying what fraction of the signal detected in
any given amplifier on each sensor in the focal plane appears in any
other amplifier.

}

\begin{tabular}{p{4cm}p{12cm}}
\toprule
Step 4
& Description \\ \hline
\end{tabular}
{\scriptsize
Apply the crosstalk correction to simulated images, and confirm that the
correction is performing as expected.

}
\begin{tabular}{p{3cm}p{13cm}}
\hline
            & Expected Result \\ \hline
\end{tabular}
{\scriptsize
A noticeable difference between images before and after applying the
correction.

}

\subsubsection{LVV-T88 - Verify implementation of Calibration Data Products}\label{lvv-t88}

\begin{longtable}[]{llllll}
\toprule
Version & Status & Priority & Verification Type & Owner
\\\midrule
1 & Defined & Normal &
Test & Robert Lupton
\\\bottomrule
\multicolumn{6}{c}{ Open \href{https://jira.lsstcorp.org/secure/Tests.jspa\#/testCase/LVV-T88}{LVV-T88} in Jira } \\
\end{longtable}

\paragraph{Verification Elements}\mbox{}\\

\begin{itemize}
\item \href{https://jira.lsstcorp.org/browse/LVV-57}{LVV-57} - DMS-REQ-0130-V-01: Calibration Data Products

\end{itemize}

\paragraph{Test Items}\mbox{}\\

Verify that the DMS can produce and archive the required Calibration
Data Products: cross talk correction, bias, dark, monochromatic dome
flats, broad-band flats, fringe correction, and illumination
corrections.








\paragraph{Test Procedure}\mbox{}\\
\begin{tabular}{p{4cm}p{12cm}}
\toprule
Step 1
& Description \\ \hline
\end{tabular}
{\scriptsize
Identify a suitable set of calibration frames, including biases, dark
frames, and flat-field frames.

}
\begin{tabular}{p{3cm}p{13cm}}
\hline
            & Expected Result \\ \hline
\end{tabular}

\begin{tabular}{p{4cm}p{12cm}}
\toprule
Step 2-1
{\scriptsize from \hyperref[lvv-t1060]{LVV-T1060} }
& Description \\ \hline
\end{tabular}
{\scriptsize
Execute the Calibration Products Production payload. The payload uses
raw calibration images and information from the Transformed EFD to
generate a subset of Master Calibration Images and Calibration Database
entries in the Data Backbone.

}
\begin{tabular}{p{3cm}p{13cm}}
\hline
            & Expected Result \\ \hline
\end{tabular}

\begin{tabular}{p{4cm}p{12cm}}
\toprule
Step 2-2
{\scriptsize from \hyperref[lvv-t1060]{LVV-T1060} }
& Description \\ \hline
\end{tabular}
{\scriptsize
Confirm that the expected Master Calibration images and Calibration
Database entries are present and well-formed.

}
\begin{tabular}{p{3cm}p{13cm}}
\hline
            & Expected Result \\ \hline
\end{tabular}

\begin{tabular}{p{4cm}p{12cm}}
\toprule
Step 3
& Description \\ \hline
\end{tabular}
{\scriptsize
Confirm that the expected data products are created, and that they have
the expected properties.

}
\begin{tabular}{p{3cm}p{13cm}}
\hline
            & Expected Result \\ \hline
\end{tabular}
{\scriptsize
A full set of calibration data products has been created, and they are
well-formed.

}

\begin{tabular}{p{4cm}p{12cm}}
\toprule
Step 4
& Description \\ \hline
\end{tabular}
{\scriptsize
Test that the calibration products are archived, and can readily be
applied to science data to produce the desired corrections.

}
\begin{tabular}{p{3cm}p{13cm}}
\hline
            & Expected Result \\ \hline
\end{tabular}
{\scriptsize
Confirmation that application of the calibration products to processed
data has the desired effects.

}

\subsubsection{LVV-T89 - Verify implementation of Calibration Image Provenance}\label{lvv-t89}

\begin{longtable}[]{llllll}
\toprule
Version & Status & Priority & Verification Type & Owner
\\\midrule
1 & Defined & Normal &
Test & Robert Lupton
\\\bottomrule
\multicolumn{6}{c}{ Open \href{https://jira.lsstcorp.org/secure/Tests.jspa\#/testCase/LVV-T89}{LVV-T89} in Jira } \\
\end{longtable}

\paragraph{Verification Elements}\mbox{}\\

\begin{itemize}
\item \href{https://jira.lsstcorp.org/browse/LVV-59}{LVV-59} - DMS-REQ-0132-V-01: Calibration Image Provenance

\item \href{https://jira.lsstcorp.org/browse/LVV-1234}{LVV-1234} - OSS-REQ-0122-V-01: Provenance

\end{itemize}

\paragraph{Test Items}\mbox{}\\

Verify that the DMS records the required provenance information for the
Calibration Data Products.








\paragraph{Test Procedure}\mbox{}\\
\begin{tabular}{p{4cm}p{12cm}}
\toprule
Step 1
& Description \\ \hline
\end{tabular}
{\scriptsize
Ingest an appropriate precursor calibration dataset into a Butler repo.

}
\begin{tabular}{p{3cm}p{13cm}}
\hline
            & Expected Result \\ \hline
\end{tabular}

\begin{tabular}{p{4cm}p{12cm}}
\toprule
Step 2-1
{\scriptsize from \hyperref[lvv-t1060]{LVV-T1060} }
& Description \\ \hline
\end{tabular}
{\scriptsize
Execute the Calibration Products Production payload. The payload uses
raw calibration images and information from the Transformed EFD to
generate a subset of Master Calibration Images and Calibration Database
entries in the Data Backbone.

}
\begin{tabular}{p{3cm}p{13cm}}
\hline
            & Expected Result \\ \hline
\end{tabular}

\begin{tabular}{p{4cm}p{12cm}}
\toprule
Step 2-2
{\scriptsize from \hyperref[lvv-t1060]{LVV-T1060} }
& Description \\ \hline
\end{tabular}
{\scriptsize
Confirm that the expected Master Calibration images and Calibration
Database entries are present and well-formed.

}
\begin{tabular}{p{3cm}p{13cm}}
\hline
            & Expected Result \\ \hline
\end{tabular}

\begin{tabular}{p{4cm}p{12cm}}
\toprule
Step 3
& Description \\ \hline
\end{tabular}
{\scriptsize
Load the relevant database/Butler data product, and observe that all
provenance information has been retained.

}
\begin{tabular}{p{3cm}p{13cm}}
\hline
            & Expected Result \\ \hline
\end{tabular}
{\scriptsize
A dataset consisting of calibration images, with provenance information
recorded and properly associated with the calibration images.

}

\subsubsection{LVV-T90 - Verify implementation of Dark Current Correction Frame}\label{lvv-t90}

\begin{longtable}[]{llllll}
\toprule
Version & Status & Priority & Verification Type & Owner
\\\midrule
1 & Defined & Normal &
Test & Robert Lupton
\\\bottomrule
\multicolumn{6}{c}{ Open \href{https://jira.lsstcorp.org/secure/Tests.jspa\#/testCase/LVV-T90}{LVV-T90} in Jira } \\
\end{longtable}

\paragraph{Verification Elements}\mbox{}\\

\begin{itemize}
\item \href{https://jira.lsstcorp.org/browse/LVV-113}{LVV-113} - DMS-REQ-0282-V-01: Dark Current Correction Frame Creation

\end{itemize}

\paragraph{Test Items}\mbox{}\\

Verify that the DMS can produce a dark correction frame calibration
product.








\paragraph{Test Procedure}\mbox{}\\
\begin{tabular}{p{4cm}p{12cm}}
\toprule
Step 1
& Description \\ \hline
\end{tabular}
{\scriptsize
Identify the path to a dataset containing dark frames (i.e., exposures
taken with the shutter closed).

}
\begin{tabular}{p{3cm}p{13cm}}
\hline
            & Expected Result \\ \hline
\end{tabular}

\begin{tabular}{p{4cm}p{12cm}}
\toprule
Step 2
& Description \\ \hline
\end{tabular}
{\scriptsize
Execute the relevant steps from `cp\_pipe` (the calibration pipeline) to
produce dark correction frames.

}
\begin{tabular}{p{3cm}p{13cm}}
\hline
            & Expected Result \\ \hline
\end{tabular}

\begin{tabular}{p{4cm}p{12cm}}
\toprule
Step 3
& Description \\ \hline
\end{tabular}
{\scriptsize
Inspect the resulting dark correction frame to confirm that it appears
as expected.

}
\begin{tabular}{p{3cm}p{13cm}}
\hline
            & Expected Result \\ \hline
\end{tabular}
{\scriptsize
A well-formed dark correction frame is present and accessible via the
Data Butler.

}

\subsubsection{LVV-T97 - Verify implementation of Uniqueness of IDs Across Data Releases}\label{lvv-t97}

\begin{longtable}[]{llllll}
\toprule
Version & Status & Priority & Verification Type & Owner
\\\midrule
1 & Defined & Normal &
Test & Kian-Tat Lim
\\\bottomrule
\multicolumn{6}{c}{ Open \href{https://jira.lsstcorp.org/secure/Tests.jspa\#/testCase/LVV-T97}{LVV-T97} in Jira } \\
\end{longtable}

\paragraph{Verification Elements}\mbox{}\\

\begin{itemize}
\item \href{https://jira.lsstcorp.org/browse/LVV-123}{LVV-123} - DMS-REQ-0292-V-01: Uniqueness of IDs Across Data Releases

\end{itemize}

\paragraph{Test Items}\mbox{}\\

Verify that the IDs of Objects, Sources, DIAObjects, and DIASources from
different Data Releases are unique.








\paragraph{Test Procedure}\mbox{}\\
\begin{tabular}{p{4cm}p{12cm}}
\toprule
Step 1
& Description \\ \hline
\end{tabular}
{\scriptsize
Identify an appropriate precursor dataset to be processed through Data
Release Production.

}
\begin{tabular}{p{3cm}p{13cm}}
\hline
            & Expected Result \\ \hline
\end{tabular}

\begin{tabular}{p{4cm}p{12cm}}
\toprule
Step 2-1
{\scriptsize from \hyperref[lvv-t1064]{LVV-T1064} }
& Description \\ \hline
\end{tabular}
{\scriptsize
Process data with the Data Release Production payload, starting from raw
science images and generating science data products, placing them in the
Data Backbone.

}
\begin{tabular}{p{3cm}p{13cm}}
\hline
            & Expected Result \\ \hline
\end{tabular}

\begin{tabular}{p{4cm}p{12cm}}
\toprule
Step 3-1
{\scriptsize from \hyperref[lvv-t987]{LVV-T987} }
& Description \\ \hline
\end{tabular}
{\scriptsize
Identify the path to the data repository, which we will refer to as
`DATA/path', then execute the following:

}
\begin{tabular}{p{3cm}p{13cm}}
\hline
            & Example Code \\ \hline
\end{tabular}
{\scriptsize
\begin{verbatim}
import lsst.daf.persistence as dafPersist
butler = dafPersist.Butler(inputs='DATA/path')
\end{verbatim}

}
\begin{tabular}{p{3cm}p{13cm}}
\hline
            & Expected Result \\ \hline
\end{tabular}
{\scriptsize
Butler repo available for reading.

}

\begin{tabular}{p{4cm}p{12cm}}
\toprule
Step 4
& Description \\ \hline
\end{tabular}
{\scriptsize
After running the DRP payload multiple times, load the resulting data
products (both data release and prompt products) using the Butler.

}
\begin{tabular}{p{3cm}p{13cm}}
\hline
            & Expected Result \\ \hline
\end{tabular}
{\scriptsize
Multiple datasets resulting from processing of the same input data.

}

\begin{tabular}{p{4cm}p{12cm}}
\toprule
Step 5
& Description \\ \hline
\end{tabular}
{\scriptsize
Inspect the IDs in the multiple data products and confirm that all IDs
are unique.

}
\begin{tabular}{p{3cm}p{13cm}}
\hline
            & Expected Result \\ \hline
\end{tabular}
{\scriptsize
No IDs are repeated between multiple processings of the identical input
dataset.

}

\subsubsection{LVV-T98 - Verify implementation of Selection of Datasets}\label{lvv-t98}

\begin{longtable}[]{llllll}
\toprule
Version & Status & Priority & Verification Type & Owner
\\\midrule
1 & Defined & Normal &
Test & Kian-Tat Lim
\\\bottomrule
\multicolumn{6}{c}{ Open \href{https://jira.lsstcorp.org/secure/Tests.jspa\#/testCase/LVV-T98}{LVV-T98} in Jira } \\
\end{longtable}

\paragraph{Verification Elements}\mbox{}\\

\begin{itemize}
\item \href{https://jira.lsstcorp.org/browse/LVV-124}{LVV-124} - DMS-REQ-0293-V-01: Selection of Datasets

\end{itemize}

\paragraph{Test Items}\mbox{}\\

Verify that the DMS can identify and retrieve datasets consisting of
logical groupings of Exposures, metadata, provenance, etc., or other
groupings that are processed or produced as a logical unit.








\paragraph{Test Procedure}\mbox{}\\
\begin{tabular}{p{4cm}p{12cm}}
\toprule
Step 1-1
{\scriptsize from \hyperref[lvv-t987]{LVV-T987} }
& Description \\ \hline
\end{tabular}
{\scriptsize
Identify the path to the data repository, which we will refer to as
`DATA/path', then execute the following:

}
\begin{tabular}{p{3cm}p{13cm}}
\hline
            & Example Code \\ \hline
\end{tabular}
{\scriptsize
\begin{verbatim}
import lsst.daf.persistence as dafPersist
butler = dafPersist.Butler(inputs='DATA/path')
\end{verbatim}

}
\begin{tabular}{p{3cm}p{13cm}}
\hline
            & Expected Result \\ \hline
\end{tabular}
{\scriptsize
Butler repo available for reading.

}

\begin{tabular}{p{4cm}p{12cm}}
\toprule
Step 2
& Description \\ \hline
\end{tabular}
{\scriptsize
Ingest data from an appropriate processed dataset.

}
\begin{tabular}{p{3cm}p{13cm}}
\hline
            & Expected Result \\ \hline
\end{tabular}

\begin{tabular}{p{4cm}p{12cm}}
\toprule
Step 3
& Description \\ \hline
\end{tabular}
{\scriptsize
Observe retrieval of single Processed Visit Image (PVI) with metadata.

}
\begin{tabular}{p{3cm}p{13cm}}
\hline
            & Expected Result \\ \hline
\end{tabular}
{\scriptsize
A PVI and its associated metadata.

}

\begin{tabular}{p{4cm}p{12cm}}
\toprule
Step 4
& Description \\ \hline
\end{tabular}
{\scriptsize
Observe retrieval of multiple PVIs with metadata.

}
\begin{tabular}{p{3cm}p{13cm}}
\hline
            & Expected Result \\ \hline
\end{tabular}
{\scriptsize
A set of PVIs and their associated metadata.

}

\begin{tabular}{p{4cm}p{12cm}}
\toprule
Step 5
& Description \\ \hline
\end{tabular}
{\scriptsize
Observe retrieval of coadd patch with metadata and provenance
information.

}
\begin{tabular}{p{3cm}p{13cm}}
\hline
            & Expected Result \\ \hline
\end{tabular}
{\scriptsize
An image of coadded data in a patch, along with its metadata and
information describing the provenance of the patch constituents.

}

\begin{tabular}{p{4cm}p{12cm}}
\toprule
Step 6
& Description \\ \hline
\end{tabular}
{\scriptsize
Observe retrieval of subset of rows in each of the above catalogs.

}
\begin{tabular}{p{3cm}p{13cm}}
\hline
            & Expected Result \\ \hline
\end{tabular}

\subsubsection{LVV-T103 - Verify implementation of Generate Data Quality Report Within Specified
Time}\label{lvv-t103}

\begin{longtable}[]{llllll}
\toprule
Version & Status & Priority & Verification Type & Owner
\\\midrule
1 & Defined & Normal &
Test & Kian-Tat Lim
\\\bottomrule
\multicolumn{6}{c}{ Open \href{https://jira.lsstcorp.org/secure/Tests.jspa\#/testCase/LVV-T103}{LVV-T103} in Jira } \\
\end{longtable}

\paragraph{Verification Elements}\mbox{}\\

\begin{itemize}
\item \href{https://jira.lsstcorp.org/browse/LVV-38}{LVV-38} - DMS-REQ-0096-V-01: Generate Data Quality Report Within Specified Time

\end{itemize}

\paragraph{Test Items}\mbox{}\\

Verify that the DMS can generate a nightly L1 Data Quality Report within
\textbf{dqReportComplTime = 4{[}hour{]}}, in both human- and
machine-readable formats.








\paragraph{Test Procedure}\mbox{}\\
\begin{tabular}{p{4cm}p{12cm}}
\toprule
Step 1
& Description \\ \hline
\end{tabular}
{\scriptsize
Execute single-day operations rehearsal

}
\begin{tabular}{p{3cm}p{13cm}}
\hline
            & Expected Result \\ \hline
\end{tabular}

\begin{tabular}{p{4cm}p{12cm}}
\toprule
Step 2
& Description \\ \hline
\end{tabular}
{\scriptsize
After \textbf{dqReportComplTime = 4{[}hour{]}~}has passed, confirm (via
timestamps) that the data quality report has been generated within
\textbf{dqReportComplTime = 4{[}hour{]},} and that it contains the
correct contents.

}
\begin{tabular}{p{3cm}p{13cm}}
\hline
            & Expected Result \\ \hline
\end{tabular}
{\scriptsize
Both human- and machine-readable versions of the L1 Data Quality Report
are available with dqReportComplTime.

}

\subsubsection{LVV-T112 - Verify implementation of Alert Filtering Service}\label{lvv-t112}

\begin{longtable}[]{llllll}
\toprule
Version & Status & Priority & Verification Type & Owner
\\\midrule
1 & Defined & Normal &
Test & Eric Bellm
\\\bottomrule
\multicolumn{6}{c}{ Open \href{https://jira.lsstcorp.org/secure/Tests.jspa\#/testCase/LVV-T112}{LVV-T112} in Jira } \\
\end{longtable}

\paragraph{Verification Elements}\mbox{}\\

\begin{itemize}
\item \href{https://jira.lsstcorp.org/browse/LVV-173}{LVV-173} - DMS-REQ-0342-V-01: Alert Filtering Service

\end{itemize}

\paragraph{Test Items}\mbox{}\\

Verify that user-defined filters can be used to generate a basic alert
filtering service.








\paragraph{Test Procedure}\mbox{}\\
\begin{tabular}{p{4cm}p{12cm}}
\toprule
Step 1
& Description \\ \hline
\end{tabular}
{\scriptsize
Identify a suitable precursor dataset for processing through the Alert
Production pipeline.

}
\begin{tabular}{p{3cm}p{13cm}}
\hline
            & Expected Result \\ \hline
\end{tabular}

\begin{tabular}{p{4cm}p{12cm}}
\toprule
Step 2-1
{\scriptsize from \hyperref[lvv-t866]{LVV-T866} }
& Description \\ \hline
\end{tabular}
{\scriptsize
Perform the steps of Alert Production (including, but not necessarily
limited to, single frame processing, ISR, source detection/measurement,
PSF estimation, photometric and astrometric calibration, difference
imaging, DIASource detection/measurement, source association). During
Operations, it is presumed that these are automated for a given
dataset.~

}
\begin{tabular}{p{3cm}p{13cm}}
\hline
            & Expected Result \\ \hline
\end{tabular}
{\scriptsize
An output dataset including difference images and DIASource and
DIAObject measurements.

}

\begin{tabular}{p{4cm}p{12cm}}
\toprule
Step 2-2
{\scriptsize from \hyperref[lvv-t866]{LVV-T866} }
& Description \\ \hline
\end{tabular}
{\scriptsize
Verify that the expected data products have been produced, and that
catalogs contain reasonable values for measured quantities of interest.

}
\begin{tabular}{p{3cm}p{13cm}}
\hline
            & Expected Result \\ \hline
\end{tabular}

\begin{tabular}{p{4cm}p{12cm}}
\toprule
Step 3
& Description \\ \hline
\end{tabular}
{\scriptsize
Confirm that alerts are generated, and that an Alert Distribution
service is making them available via a stream.

}
\begin{tabular}{p{3cm}p{13cm}}
\hline
            & Expected Result \\ \hline
\end{tabular}
{\scriptsize
Via either a UI or API, confirmation that a stream of alerts are
available.

}

\begin{tabular}{p{4cm}p{12cm}}
\toprule
Step 4
& Description \\ \hline
\end{tabular}
{\scriptsize
Confirm that a UI (or API) exists that allows users to define simple
filters. Define a filter, and observe both the full and the filtered
alert streams to confirm that the filter has reduced the volume of
alerts.

}
\begin{tabular}{p{3cm}p{13cm}}
\hline
            & Expected Result \\ \hline
\end{tabular}
{\scriptsize
The user-defined filter has reduced the number of alerts being received
relative to the full stream.

}

\subsubsection{LVV-T113 - Verify implementation of Performance Requirements for LSST Alert
Filtering Service}\label{lvv-t113}

\begin{longtable}[]{llllll}
\toprule
Version & Status & Priority & Verification Type & Owner
\\\midrule
1 & Defined & Normal &
Test & Eric Bellm
\\\bottomrule
\multicolumn{6}{c}{ Open \href{https://jira.lsstcorp.org/secure/Tests.jspa\#/testCase/LVV-T113}{LVV-T113} in Jira } \\
\end{longtable}

\paragraph{Verification Elements}\mbox{}\\

\begin{itemize}
\item \href{https://jira.lsstcorp.org/browse/LVV-174}{LVV-174} - DMS-REQ-0343-V-01: Number of full-size alerts

\end{itemize}

\paragraph{Test Items}\mbox{}\\

Verify that the DMS alert filter service provides sufficient bandwidth
for \textbf{numBrokerUsers = 100} simultaneously-operating brokers to
receive up to \textbf{numBrokerAlerts = 20} alerts per visit.








\paragraph{Test Procedure}\mbox{}\\
\begin{tabular}{p{4cm}p{12cm}}
\toprule
Step 1
& Description \\ \hline
\end{tabular}
{\scriptsize
Create a simulated alert stream.

}
\begin{tabular}{p{3cm}p{13cm}}
\hline
            & Expected Result \\ \hline
\end{tabular}

\begin{tabular}{p{4cm}p{12cm}}
\toprule
Step 2
& Description \\ \hline
\end{tabular}
{\scriptsize
Simultaneously execute user-defined alert filters for at least
\textbf{numBrokerUsers = 100} users, and confirm that the system
successfully filters the stream as requested. Confirm that the bandwidth
requirement of \textbf{numBrokerAlerts = 20} per user was met.

}
\begin{tabular}{p{3cm}p{13cm}}
\hline
            & Expected Result \\ \hline
\end{tabular}
{\scriptsize
All of the (simulated) users successfully receive their requested
filtered alerts, with \textbf{numBrokerAlerts = 20~}per user.

}

\subsubsection{LVV-T114 - Verify implementation of Pre-defined alert filters}\label{lvv-t114}

\begin{longtable}[]{llllll}
\toprule
Version & Status & Priority & Verification Type & Owner
\\\midrule
1 & Defined & Normal &
Test & Eric Bellm
\\\bottomrule
\multicolumn{6}{c}{ Open \href{https://jira.lsstcorp.org/secure/Tests.jspa\#/testCase/LVV-T114}{LVV-T114} in Jira } \\
\end{longtable}

\paragraph{Verification Elements}\mbox{}\\

\begin{itemize}
\item \href{https://jira.lsstcorp.org/browse/LVV-179}{LVV-179} - DMS-REQ-0348-V-01: Pre-defined alert filters

\end{itemize}

\paragraph{Test Items}\mbox{}\\

Verify that users of the Alert Filtering service can use a predefined
set of filters.








\paragraph{Test Procedure}\mbox{}\\
\begin{tabular}{p{4cm}p{12cm}}
\toprule
Step 1
& Description \\ \hline
\end{tabular}
{\scriptsize
Create a simulated alert stream. Confirm that alerts are generated, and
that an Alert Distribution service is making them available.

}
\begin{tabular}{p{3cm}p{13cm}}
\hline
            & Expected Result \\ \hline
\end{tabular}
{\scriptsize
A stream of alerts that is confirmed to be generated and distributed.

}

\begin{tabular}{p{4cm}p{12cm}}
\toprule
Step 2
& Description \\ \hline
\end{tabular}
{\scriptsize
Confirm that a UI (or API) exists that presents users some pre-defined
filters.

}
\begin{tabular}{p{3cm}p{13cm}}
\hline
            & Expected Result \\ \hline
\end{tabular}
{\scriptsize
The UI (or API) for accessing alert streams has some pre-defined filters
available for users.

}

\begin{tabular}{p{4cm}p{12cm}}
\toprule
Step 3
& Description \\ \hline
\end{tabular}
{\scriptsize
Select one of the pre-defined filters, and confirm that the results have
been properly filtered.

}
\begin{tabular}{p{3cm}p{13cm}}
\hline
            & Expected Result \\ \hline
\end{tabular}
{\scriptsize
After applying the pre-defined filter, the number of alerts has
decreased relative to the raw stream.~

}

\subsubsection{LVV-T115 - Verify implementation of Calibration Production Processing}\label{lvv-t115}

\begin{longtable}[]{llllll}
\toprule
Version & Status & Priority & Verification Type & Owner
\\\midrule
1 & Defined & Normal &
Test & Kian-Tat Lim
\\\bottomrule
\multicolumn{6}{c}{ Open \href{https://jira.lsstcorp.org/secure/Tests.jspa\#/testCase/LVV-T115}{LVV-T115} in Jira } \\
\end{longtable}

\paragraph{Verification Elements}\mbox{}\\

\begin{itemize}
\item \href{https://jira.lsstcorp.org/browse/LVV-120}{LVV-120} - DMS-REQ-0289-V-01: Calibration Production Processing

\end{itemize}

\paragraph{Test Items}\mbox{}\\

Execute CPP on a variety of representative cadences, and verify that the
calibration pipeline correctly produces necessary calibration products.








\paragraph{Test Procedure}\mbox{}\\
\begin{tabular}{p{4cm}p{12cm}}
\toprule
Step 1
& Description \\ \hline
\end{tabular}
{\scriptsize
Identify a suitable set of calibration frames, including biases, dark
frames, and flat-field frames.

}
\begin{tabular}{p{3cm}p{13cm}}
\hline
            & Expected Result \\ \hline
\end{tabular}

\begin{tabular}{p{4cm}p{12cm}}
\toprule
Step 2-1
{\scriptsize from \hyperref[lvv-t1060]{LVV-T1060} }
& Description \\ \hline
\end{tabular}
{\scriptsize
Execute the Calibration Products Production payload. The payload uses
raw calibration images and information from the Transformed EFD to
generate a subset of Master Calibration Images and Calibration Database
entries in the Data Backbone.

}
\begin{tabular}{p{3cm}p{13cm}}
\hline
            & Expected Result \\ \hline
\end{tabular}

\begin{tabular}{p{4cm}p{12cm}}
\toprule
Step 2-2
{\scriptsize from \hyperref[lvv-t1060]{LVV-T1060} }
& Description \\ \hline
\end{tabular}
{\scriptsize
Confirm that the expected Master Calibration images and Calibration
Database entries are present and well-formed.

}
\begin{tabular}{p{3cm}p{13cm}}
\hline
            & Expected Result \\ \hline
\end{tabular}

\begin{tabular}{p{4cm}p{12cm}}
\toprule
Step 3
& Description \\ \hline
\end{tabular}
{\scriptsize
Confirm that the expected data products are created, and that they have
the expected properties.

}
\begin{tabular}{p{3cm}p{13cm}}
\hline
            & Expected Result \\ \hline
\end{tabular}
{\scriptsize
Repos containing valid calibration products that are well-formed and
ready to be applied to processed datasets.

}

\subsubsection{LVV-T124 - Verify implementation of Software Architecture to Enable Community
Re-Use}\label{lvv-t124}

\begin{longtable}[]{llllll}
\toprule
Version & Status & Priority & Verification Type & Owner
\\\midrule
1 & Defined & Normal &
Test & Simon Krughoff
\\\bottomrule
\multicolumn{6}{c}{ Open \href{https://jira.lsstcorp.org/secure/Tests.jspa\#/testCase/LVV-T124}{LVV-T124} in Jira } \\
\end{longtable}

\paragraph{Verification Elements}\mbox{}\\

\begin{itemize}
\item \href{https://jira.lsstcorp.org/browse/LVV-139}{LVV-139} - DMS-REQ-0308-V-01: Software Architecture to Enable Community Re-Use

\end{itemize}

\paragraph{Test Items}\mbox{}\\

Show that the LSST software is capable of being executed in multiple
contexts: single user instance, batch processing, continuous
integration.\\
Also show that the algorithms can be reconfigured and, if desired,
completely replaced at run time.








\paragraph{Test Procedure}\mbox{}\\
\begin{tabular}{p{4cm}p{12cm}}
\toprule
Step 1-1
{\scriptsize from \hyperref[lvv-t860]{LVV-T860} }
& Description \\ \hline
\end{tabular}
{\scriptsize
The `path` that you will use depends on where you are running the
science pipelines. Options:\\[2\baselineskip]

\begin{itemize}
\tightlist
\item
  local (newinstall.sh - based
  install):{[}path\_to\_installation{]}/loadLSST.bash
\item
  development cluster (``lsst-dev''):
  /software/lsstsw/stack/loadLSST.bash
\item
  LSP Notebook aspect (from a terminal):
  /opt/lsst/software/stack/loadLSST.bash
\end{itemize}

From the command line, execute the commands below in the example
code:\\[2\baselineskip]

}
\begin{tabular}{p{3cm}p{13cm}}
\hline
            & Example Code \\ \hline
\end{tabular}
{\scriptsize
source `path`\\
setup lsst\_distrib

}
\begin{tabular}{p{3cm}p{13cm}}
\hline
            & Expected Result \\ \hline
\end{tabular}
{\scriptsize
Science pipeline software is available for use. If additional packages
are needed (for example, `obs' packages such as `obs\_subaru`), then
additional `setup` commands will be necessary.\\[2\baselineskip]To check
versions in use, type:\\
eups list -s

}

\begin{tabular}{p{4cm}p{12cm}}
\toprule
Step 2
& Description \\ \hline
\end{tabular}
{\scriptsize
Using curated test datasets for multiple precursor instruments, verify
and log that the prototype DRP pipelines execute successfully in three
contexts:\\
1. The CI system\\
2. On a single user system: laptop, desktop, or notebook running in the
Notebook aspect of the LSP.\\
3. Project workflow system.

}
\begin{tabular}{p{3cm}p{13cm}}
\hline
            & Expected Result \\ \hline
\end{tabular}

\begin{tabular}{p{4cm}p{12cm}}
\toprule
Step 3
& Description \\ \hline
\end{tabular}
{\scriptsize
Using a template testing notebook in the Notebook aspect of the LSP,
verify and log the following:\\
1. Individual pipeline steps (tasks) are importable and executable on
their own. ~this is not comprehensive, but demonstrative.\\
2. Individual pipeline steps may be overridden by configuration.\\
3. Users can implement a custom pipeline step and insert i into the
processing flow via configuration.

}
\begin{tabular}{p{3cm}p{13cm}}
\hline
            & Expected Result \\ \hline
\end{tabular}

\begin{tabular}{p{4cm}p{12cm}}
\toprule
Step 4-1
{\scriptsize from \hyperref[lvv-t987]{LVV-T987} }
& Description \\ \hline
\end{tabular}
{\scriptsize
Identify the path to the data repository, which we will refer to as
`DATA/path', then execute the following:

}
\begin{tabular}{p{3cm}p{13cm}}
\hline
            & Example Code \\ \hline
\end{tabular}
{\scriptsize
\begin{verbatim}
import lsst.daf.persistence as dafPersist
butler = dafPersist.Butler(inputs='DATA/path')
\end{verbatim}

}
\begin{tabular}{p{3cm}p{13cm}}
\hline
            & Expected Result \\ \hline
\end{tabular}
{\scriptsize
Butler repo available for reading.

}

\begin{tabular}{p{4cm}p{12cm}}
\toprule
Step 5
& Description \\ \hline
\end{tabular}
{\scriptsize
Read the resulting dataset using the Bulter, and confirm that it
produced the desired data products.

}
\begin{tabular}{p{3cm}p{13cm}}
\hline
            & Expected Result \\ \hline
\end{tabular}

\begin{tabular}{p{4cm}p{12cm}}
\toprule
Step 6
& Description \\ \hline
\end{tabular}
{\scriptsize
Run subset of full DRP from previous step on an individual node. ~Was
this organizationally easy? ~Did the performance scale appropriately?

}
\begin{tabular}{p{3cm}p{13cm}}
\hline
            & Expected Result \\ \hline
\end{tabular}

\begin{tabular}{p{4cm}p{12cm}}
\toprule
Step 7
& Description \\ \hline
\end{tabular}
{\scriptsize
Re-run aperture correction on subset. ~Verify that same results as DRP
run are achieved.

}
\begin{tabular}{p{3cm}p{13cm}}
\hline
            & Expected Result \\ \hline
\end{tabular}

\begin{tabular}{p{4cm}p{12cm}}
\toprule
Step 8
& Description \\ \hline
\end{tabular}
{\scriptsize
Re-run photometric redshift estimation algorithm on subset coadd
catalogs. ~Verify that same results are achieved as from full DRP.

}
\begin{tabular}{p{3cm}p{13cm}}
\hline
            & Expected Result \\ \hline
\end{tabular}

\subsubsection{LVV-T126 - Verify implementation of Image Differencing}\label{lvv-t126}

\begin{longtable}[]{llllll}
\toprule
Version & Status & Priority & Verification Type & Owner
\\\midrule
1 & Defined & Normal &
Test & Eric Bellm
\\\bottomrule
\multicolumn{6}{c}{ Open \href{https://jira.lsstcorp.org/secure/Tests.jspa\#/testCase/LVV-T126}{LVV-T126} in Jira } \\
\end{longtable}

\paragraph{Verification Elements}\mbox{}\\

\begin{itemize}
\item \href{https://jira.lsstcorp.org/browse/LVV-14}{LVV-14} - DMS-REQ-0032-V-01: Image Differencing

\end{itemize}

\paragraph{Test Items}\mbox{}\\

Verify that the DMS can perform image differencing from single exposures
and coadds.








\paragraph{Test Procedure}\mbox{}\\
\begin{tabular}{p{4cm}p{12cm}}
\toprule
Step 1
& Description \\ \hline
\end{tabular}
{\scriptsize
Identify a repository containing data that have been processed through
the difference imaging pipeline. (e.g., the HiTS 2015 data that are
processed monthly for testing)

}
\begin{tabular}{p{3cm}p{13cm}}
\hline
            & Expected Result \\ \hline
\end{tabular}
{\scriptsize
A dataset containing calexps, difference images, and source catalogs (of
diaSrcs).

}

\begin{tabular}{p{4cm}p{12cm}}
\toprule
Step 2-1
{\scriptsize from \hyperref[lvv-t987]{LVV-T987} }
& Description \\ \hline
\end{tabular}
{\scriptsize
Identify the path to the data repository, which we will refer to as
`DATA/path', then execute the following:

}
\begin{tabular}{p{3cm}p{13cm}}
\hline
            & Example Code \\ \hline
\end{tabular}
{\scriptsize
\begin{verbatim}
import lsst.daf.persistence as dafPersist
butler = dafPersist.Butler(inputs='DATA/path')
\end{verbatim}

}
\begin{tabular}{p{3cm}p{13cm}}
\hline
            & Expected Result \\ \hline
\end{tabular}
{\scriptsize
Butler repo available for reading.

}

\begin{tabular}{p{4cm}p{12cm}}
\toprule
Step 3
& Description \\ \hline
\end{tabular}
{\scriptsize
Extract a `calexp`, a `deepDiff\_differenceExp`, and the
`deepDiff\_diaSrc` catalog of measurements.

}
\begin{tabular}{p{3cm}p{13cm}}
\hline
            & Expected Result \\ \hline
\end{tabular}
{\scriptsize
Well-formed images and catalogs containing the calexp from the visit
image and the difference image, and measurements of sources from the
difference image.

}

\begin{tabular}{p{4cm}p{12cm}}
\toprule
Step 4
& Description \\ \hline
\end{tabular}
{\scriptsize
Confirm (by visual inspection) that the difference image is mostly blank
sky (i.e., has had a template of the same field subtracted), and that
the source catalog contains sources with photometric and astrometric
measurements.

}
\begin{tabular}{p{3cm}p{13cm}}
\hline
            & Expected Result \\ \hline
\end{tabular}
{\scriptsize
A mostly blank image (with perhaps some artifacts due to imperfect
subtraction) and a catalog of sources detected/measured from that image.

}

\subsubsection{LVV-T127 - Verify implementation of Provide Source Detection Software}\label{lvv-t127}

\begin{longtable}[]{llllll}
\toprule
Version & Status & Priority & Verification Type & Owner
\\\midrule
1 & Defined & Normal &
Test & Robert Lupton
\\\bottomrule
\multicolumn{6}{c}{ Open \href{https://jira.lsstcorp.org/secure/Tests.jspa\#/testCase/LVV-T127}{LVV-T127} in Jira } \\
\end{longtable}

\paragraph{Verification Elements}\mbox{}\\

\begin{itemize}
\item \href{https://jira.lsstcorp.org/browse/LVV-15}{LVV-15} - DMS-REQ-0033-V-01: Provide Source Detection Software

\end{itemize}

\paragraph{Test Items}\mbox{}\\

Verify that the DMS provides source detection software that can be
applied to calibrated images, including both difference images and
coadds. This will be verified using simulated data, but could also be
done by inserting artificial sources into existing datasets.








\paragraph{Test Procedure}\mbox{}\\
\begin{tabular}{p{4cm}p{12cm}}
\toprule
Step 1
& Description \\ \hline
\end{tabular}
{\scriptsize
Run DRP and AP processing, including source detection and measurement
algorithms, on a small portion of the data from a simulated dataset.

}
\begin{tabular}{p{3cm}p{13cm}}
\hline
            & Expected Result \\ \hline
\end{tabular}
{\scriptsize
Source catalogs containing measurements of all sources detected in the
input images.

}

\begin{tabular}{p{4cm}p{12cm}}
\toprule
Step 2
& Description \\ \hline
\end{tabular}
{\scriptsize
Confirm that the output repos contain catalogs of source detections.
Compare these output catalogs to the original simulated source catalogs,
and confirm that a large fraction of the sources within a reasonable
signal-to-noise range were recovered.

}
\begin{tabular}{p{3cm}p{13cm}}
\hline
            & Expected Result \\ \hline
\end{tabular}
{\scriptsize
Most sources above a reasonable S/N threshold were detected, and their
measured fluxes are reasonably close to the simulated inputs.

}

\subsubsection{LVV-T129 - Verify implementation of Provide Calibrated Photometry}\label{lvv-t129}

\begin{longtable}[]{llllll}
\toprule
Version & Status & Priority & Verification Type & Owner
\\\midrule
1 & Defined & Normal &
Test & Robert Lupton
\\\bottomrule
\multicolumn{6}{c}{ Open \href{https://jira.lsstcorp.org/secure/Tests.jspa\#/testCase/LVV-T129}{LVV-T129} in Jira } \\
\end{longtable}

\paragraph{Verification Elements}\mbox{}\\

\begin{itemize}
\item \href{https://jira.lsstcorp.org/browse/LVV-18}{LVV-18} - DMS-REQ-0043-V-01: Provide Calibrated Photometry

\end{itemize}

\paragraph{Test Items}\mbox{}\\

Verify that the DMS provides photometry calibrated in AB mags and fluxes
(in nJy) for all measured objects and sources. Must be tested for both
DRP and AP products.








\paragraph{Test Procedure}\mbox{}\\
\begin{tabular}{p{4cm}p{12cm}}
\toprule
Step 1-1
{\scriptsize from \hyperref[lvv-t987]{LVV-T987} }
& Description \\ \hline
\end{tabular}
{\scriptsize
Identify the path to the data repository, which we will refer to as
`DATA/path', then execute the following:

}
\begin{tabular}{p{3cm}p{13cm}}
\hline
            & Example Code \\ \hline
\end{tabular}
{\scriptsize
\begin{verbatim}
import lsst.daf.persistence as dafPersist
butler = dafPersist.Butler(inputs='DATA/path')
\end{verbatim}

}
\begin{tabular}{p{3cm}p{13cm}}
\hline
            & Expected Result \\ \hline
\end{tabular}
{\scriptsize
Butler repo available for reading.

}

\begin{tabular}{p{4cm}p{12cm}}
\toprule
Step 2
& Description \\ \hline
\end{tabular}
{\scriptsize
Ingest the data products from an appropriate DRP-processed dataset.

}
\begin{tabular}{p{3cm}p{13cm}}
\hline
            & Expected Result \\ \hline
\end{tabular}

\begin{tabular}{p{4cm}p{12cm}}
\toprule
Step 3
& Description \\ \hline
\end{tabular}
{\scriptsize
Confirm that AB-calibrated magnitudes and fluxes are available for all
measured Sources and Objects. {[}An enhanced verification could include
matching the sources to an external source catalog and comparing the
magnitudes to show that they are well-calibrated.{]}

}
\begin{tabular}{p{3cm}p{13cm}}
\hline
            & Expected Result \\ \hline
\end{tabular}
{\scriptsize
Calibrated fluxes and magnitudes are available for all sources, as well
as tools to convert measured fluxes to magnitudes (and vice-versa).

}

\begin{tabular}{p{4cm}p{12cm}}
\toprule
Step 4
& Description \\ \hline
\end{tabular}
{\scriptsize
Ingest the data products from an appropriate AP processing dataset.

}
\begin{tabular}{p{3cm}p{13cm}}
\hline
            & Expected Result \\ \hline
\end{tabular}

\begin{tabular}{p{4cm}p{12cm}}
\toprule
Step 5
& Description \\ \hline
\end{tabular}
{\scriptsize
Confirm that AB-calibrated magnitudes and fluxes are available for all
measured Sources, DIASources, and Objects. {[}An enhanced verification
could include matching the sources to an external source catalog and
comparing the magnitudes to show that they are well-calibrated.{]}

}
\begin{tabular}{p{3cm}p{13cm}}
\hline
            & Expected Result \\ \hline
\end{tabular}
{\scriptsize
Calibrated fluxes and magnitudes are available for all Sources,
DIASources, and Objects, as well as tools to convert measured fluxes to
magnitudes (and vice-versa).

}

\subsubsection{LVV-T131 - Verify implementation of Provide User Interface Services}\label{lvv-t131}

\begin{longtable}[]{llllll}
\toprule
Version & Status & Priority & Verification Type & Owner
\\\midrule
1 & Defined & Normal &
Test & Gregory Dubois-Felsmann
\\\bottomrule
\multicolumn{6}{c}{ Open \href{https://jira.lsstcorp.org/secure/Tests.jspa\#/testCase/LVV-T131}{LVV-T131} in Jira } \\
\end{longtable}

\paragraph{Verification Elements}\mbox{}\\

\begin{itemize}
\item \href{https://jira.lsstcorp.org/browse/LVV-63}{LVV-63} - DMS-REQ-0160-V-01: Provide User Interface Services

\end{itemize}

\paragraph{Test Items}\mbox{}\\

Verify the availability and functionality of the broad range of user
interface services called for in the requirement, as applied to both
Nightly and DRP data. ~This will primarily be done by verifications
performed at the LSST Science Platform level, based on the requirements
in \citeds{LDM-554}; however, a high-level set of tests corresponding to the
DMS-REQ-0160 requirement are defined below.



\paragraph{Environment Needs}\mbox{}\\


\subparagraph{Hardware}\mbox{}\\
As noted in Verification Configuration, the systems required to carry
out the tests include both an ``inside'' test execution platform - the
ability to execute test notebooks within the Science Platform Notebook
Aspect - and an ``outside'' test execution platform with connectivity to
the Science Platform instance under test that is comparable to that
available to offsite science users.



\paragraph{Test Procedure}\mbox{}\\
\begin{tabular}{p{4cm}p{12cm}}
\toprule
Step 1
& Description \\ \hline
\end{tabular}
{\scriptsize
\textbf{Establishment of test coordinates:}\\
Establish sky positions and surrounding regions (e.g., cones or
polygons), field sizes, filter bands, and temporal epochs for the tests
that are consistent with the known content of the test dataset, whether
precursor or LSST commissioning data.\\
Establishing sky positions should include pre-determining the
corresponding LSST ``tract and patch'' identifiers.\\
If the plan to not keep all calibrated single-epoch images on disk is
still in place at the time of the test, identify for use in the test
both images that are, and are not, on disk.\\
Establish target image boundaries, projections, and pixel scales to be
used for resampling tests. ~Ensure that at least some of these test
conditions include coadded image boundaries that cross tract and patch
boundaries, and single-epoch image boundaries that cross focal plane
raft boundaries.

}
\begin{tabular}{p{3cm}p{13cm}}
\hline
            & Expected Result \\ \hline
\end{tabular}

\begin{tabular}{p{4cm}p{12cm}}
\toprule
Step 2
& Description \\ \hline
\end{tabular}
{\scriptsize
\textbf{Butler image access:}\\
From within the Notebook Aspect, verify that coadded images for the
identified regions of sky and filter bands are accessible via the
Butler. ~Verify that the same images are available whether obtained by
direct reference to the previous established tract/patch identifiers or
by the use of LSST stack code for retrieving images based on sky
coordinates.\\
From within the Notebook Aspect, verify that single-epoch raw images for
the selected locations and times are available. ~Verify that calibrated
images (PVIs) for the selected locations and times are available;
depending on the details of the test dataset, verify that PVIs still on
disk can be retrieved immediately.\\
Verify that lists or tables of image metadata, not just individual
images, can be retrieved. ~E.g., a list of all the single-epoch images
covering a selected sky location.

}
\begin{tabular}{p{3cm}p{13cm}}
\hline
            & Expected Result \\ \hline
\end{tabular}

\begin{tabular}{p{4cm}p{12cm}}
\toprule
Step 3
& Description \\ \hline
\end{tabular}
{\scriptsize
\textbf{Programmatic PVI re-creation:}\\
From within the Notebook Aspect, verify that the recreation on demand of
a PVI can be performed. ~Ideally, this should be done as follows:

\begin{itemize}
\tightlist
\item
  Verify that recreation of a PVI that \emph{is} still available works
  and that it reproduces the original PVI exactly (except for provenance
  metadata that must be different) or within the reasonable ability of
  processing systems to do so (e.g., taking into account that the
  original calibration and the recreation may have run on different CPU
  architectures).
\item
  The test conditions should ensure the verification that a recreation
  was actually performed, i.e., that the still-available PVI was not
  returned instead.
\item
  Note that it does not appear to be a requirement that \emph{at Butler
  level} recreation on demand of PVIs is a completely transparent
  process. ~If this \emph{is} decided to be a requirement, the test must
  also verify that it has been satisfied. ~If it is \emph{not} a
  requirement, verify that adequate documentation on the PVI-recreation
  process (e.g., the SuperTasks and configuration to be used) is
  available.
\end{itemize}

}
\begin{tabular}{p{3cm}p{13cm}}
\hline
            & Expected Result \\ \hline
\end{tabular}

\begin{tabular}{p{4cm}p{12cm}}
\toprule
Step 4
& Description \\ \hline
\end{tabular}
{\scriptsize
\textbf{Butler catalog access:}\\
From within the Notebook Aspect, verify that all the catalog data
products described in the DPDD can be retrieved for the coordinates
selected above via the Butler. (This test should include access to
SSObject data, but the details of how such a test would depend on the
coordinate selections require additional thought.)

}
\begin{tabular}{p{3cm}p{13cm}}
\hline
            & Expected Result \\ \hline
\end{tabular}

\begin{tabular}{p{4cm}p{12cm}}
\toprule
Step 5
& Description \\ \hline
\end{tabular}
{\scriptsize
\textbf{LSST-stack-based resampling/reprojection:}\\
Verify the availability of software in the LSST stack, and associated
documentation, that permits the resampling of LSST images to different
pixel grids and projections.\\
Exercise this capability for the test conditions selected in Step 1
above.\\
Perform photometric and astrometric tests on the resulting resampled
images to provide evidence that the transformations performed were
correct to the accuracy supported by the data.

}
\begin{tabular}{p{3cm}p{13cm}}
\hline
            & Expected Result \\ \hline
\end{tabular}

\begin{tabular}{p{4cm}p{12cm}}
\toprule
Step 6
& Description \\ \hline
\end{tabular}
{\scriptsize
\textbf{Comment:}\\
The following API Aspect test steps should be carried out on the
required ``offsite-like'' test platform, to ensure that their success
does not reflect any privileged access given to processes inside the
Data Access Center or other Science Platform instance. ~However, at
least a small sampling of them should \emph{also} be carried out
\emph{within} the Science Platform environment, i.e., in the Notebook
Aspect, and the results compared.

}
\begin{tabular}{p{3cm}p{13cm}}
\hline
            & Expected Result \\ \hline
\end{tabular}

\begin{tabular}{p{4cm}p{12cm}}
\toprule
Step 7
& Description \\ \hline
\end{tabular}
{\scriptsize
\textbf{API Aspect image access:}\\
Using IVOA services such as the Registry and ObsTAP, from the
``offsite-like'' test platform, verify that the existence of the classes
of image data products foreseen in the DPDD can be determined.\\
Verify that ObsTAP and/or SIAv2 can be used to find the same images and
lists of images for the established test coordinates that were retrieved
via the Butler in Step 2 above.\\
Verify that the selected images are retrievable from the Web services.\\
Verify that the retrieved images are identical in their pixel content
and metadata.\\
The tests must include both coadded and single-epoch images.

}
\begin{tabular}{p{3cm}p{13cm}}
\hline
            & Expected Result \\ \hline
\end{tabular}

\begin{tabular}{p{4cm}p{12cm}}
\toprule
Step 8
& Description \\ \hline
\end{tabular}
{\scriptsize
\textbf{API Aspect image transformations:}\\
Verify that image cutouts and resamplings can be performed via the IVOA
SODA service, and that the results are identical to those obtained for
the same parameters from the LSST-stack-based tests in Step 5.\\
(The requirements for supported reprojections, if any, in the SODA
service have not been established at the time of writing.)

}
\begin{tabular}{p{3cm}p{13cm}}
\hline
            & Expected Result \\ \hline
\end{tabular}

\begin{tabular}{p{4cm}p{12cm}}
\toprule
Step 9
& Description \\ \hline
\end{tabular}
{\scriptsize
\textbf{API Aspect catalog data access:}\\
Verify that the IVOA Registry, RegTAP, TAP\_SCHEMA, and other relevant
mechanisms can be used to discover the existence of all the catalog data
products foreseen in the DPDD.\\
Using the IVOA TAP service, verify that all the catalog data products
foreseen in the DPDD can be retrieved for the coordinates determined in
Step 1. ~Verify that their scientific content is the same as when they
are retrieved via the Butler.

}
\begin{tabular}{p{3cm}p{13cm}}
\hline
            & Expected Result \\ \hline
\end{tabular}

\begin{tabular}{p{4cm}p{12cm}}
\toprule
Step 10
& Description \\ \hline
\end{tabular}
{\scriptsize
\textbf{Comment:}\\
The Portal Aspect tests below should be carried out from a web browser
on an ``offsite-like'' test platform, to ensure that no privileged
access provided to intra-data-center clients is relied upon.

}
\begin{tabular}{p{3cm}p{13cm}}
\hline
            & Expected Result \\ \hline
\end{tabular}

\begin{tabular}{p{4cm}p{12cm}}
\toprule
Step 11
& Description \\ \hline
\end{tabular}
{\scriptsize
\textbf{Portal Aspect data browsing:}\\
Verify that the Portal Aspect can be used to discover the existence of
all the data products foreseen in the DPDD. ~Verify that the UI permits
locating the data for the coordinates selected in Step 1 by visual
means, e.g., by zooming and panning in from an all-sky view.\\
Verify that the UI permits locating the data by typing in coordinates as
well.

}
\begin{tabular}{p{3cm}p{13cm}}
\hline
            & Expected Result \\ \hline
\end{tabular}

\begin{tabular}{p{4cm}p{12cm}}
\toprule
Step 12
& Description \\ \hline
\end{tabular}
{\scriptsize
\textbf{Portal Aspect image access:}\\
Verify that the Portal Aspect allows both the retrieval of ``original''
image data, i.e., in its native LSST pixel projection and with full
metadata, as well as retrieval of on-demand UI cutouts of coadded image
data for selected locations.

}
\begin{tabular}{p{3cm}p{13cm}}
\hline
            & Expected Result \\ \hline
\end{tabular}

\begin{tabular}{p{4cm}p{12cm}}
\toprule
Step 13
& Description \\ \hline
\end{tabular}
{\scriptsize
\textbf{Portal Aspect catalog query and visualization:}\\
Verify that the Portal Aspect allows graphical querying of DPDD catalog
data, both coadded and single-epoch, for selected regions of sky and/or
with selected properties, and supports the visualization of the results
(including histogramming, scatterplots, time series, table
manipulations, and overplotting on image data).\\
(Note that the Science Platform requirements, LDM-554, lay out a
detailed set of requirements on the selection and visualization of
catalog data.)

}
\begin{tabular}{p{3cm}p{13cm}}
\hline
            & Expected Result \\ \hline
\end{tabular}

\begin{tabular}{p{4cm}p{12cm}}
\toprule
Step 14
& Description \\ \hline
\end{tabular}
{\scriptsize
\textbf{Portal Aspect data download:}\\
Verify that data identified and/or visualized in the Portal Aspect can
be downloaded to the remote system running the web browser in which the
Portal is displayed, as well as to the User Workspace.

}
\begin{tabular}{p{3cm}p{13cm}}
\hline
            & Expected Result \\ \hline
\end{tabular}

\subsubsection{LVV-T133 - Verify implementation of Provide Beam Projector Coordinate Calculation
Software}\label{lvv-t133}

\begin{longtable}[]{llllll}
\toprule
Version & Status & Priority & Verification Type & Owner
\\\midrule
1 & Defined & Normal &
Test & Robert Lupton
\\\bottomrule
\multicolumn{6}{c}{ Open \href{https://jira.lsstcorp.org/secure/Tests.jspa\#/testCase/LVV-T133}{LVV-T133} in Jira } \\
\end{longtable}

\paragraph{Verification Elements}\mbox{}\\

\begin{itemize}
\item \href{https://jira.lsstcorp.org/browse/LVV-182}{LVV-182} - DMS-REQ-0351-V-01: Provide Beam Projector Coordinate Calculation
Software

\end{itemize}

\paragraph{Test Items}\mbox{}\\

Verify that the DMS provides software to calculate coordinates relating
the collimated beam projector position and telescope pupil position to
the illumination position on the telescope optical elements and focal
plane.








\paragraph{Test Procedure}\mbox{}\\
\begin{tabular}{p{4cm}p{12cm}}
\toprule
Step 1
& Description \\ \hline
\end{tabular}
{\scriptsize
On the LSST development cluster or notebook aspect, git clone the repo
containing the CBP package: \url{https://github.com/lsst/cbp}

}
\begin{tabular}{p{3cm}p{13cm}}
\hline
            & Expected Result \\ \hline
\end{tabular}

\begin{tabular}{p{4cm}p{12cm}}
\toprule
Step 2
& Description \\ \hline
\end{tabular}
{\scriptsize
Follow the steps in the package README to install the package.

}
\begin{tabular}{p{3cm}p{13cm}}
\hline
            & Expected Result \\ \hline
\end{tabular}

\begin{tabular}{p{4cm}p{12cm}}
\toprule
Step 3
& Description \\ \hline
\end{tabular}
{\scriptsize
Confirm that the package can be loaded in python, and that some of the
tests in the `tests/` folder will execute.

}
\begin{tabular}{p{3cm}p{13cm}}
\hline
            & Expected Result \\ \hline
\end{tabular}
{\scriptsize
Successful execution of test scripts, which demonstrate the calculation
of beam projector coordinates.

}

\subsubsection{LVV-T136 - Verify implementation of Data Product and Raw Data Access}\label{lvv-t136}

\begin{longtable}[]{llllll}
\toprule
Version & Status & Priority & Verification Type & Owner
\\\midrule
1 & Defined & Normal &
Test & Colin Slater
\\\bottomrule
\multicolumn{6}{c}{ Open \href{https://jira.lsstcorp.org/secure/Tests.jspa\#/testCase/LVV-T136}{LVV-T136} in Jira } \\
\end{longtable}

\paragraph{Verification Elements}\mbox{}\\

\begin{itemize}
\item \href{https://jira.lsstcorp.org/browse/LVV-129}{LVV-129} - DMS-REQ-0298-V-01: Data Product and Raw Data Access

\end{itemize}

\paragraph{Test Items}\mbox{}\\

Verify that available image, file, and catalog data products, and their
metadata and provenance information, can be listed and retrieved.








\paragraph{Test Procedure}\mbox{}\\
\begin{tabular}{p{4cm}p{12cm}}
\toprule
Step 1
& Description \\ \hline
\end{tabular}
{\scriptsize
Details of the Gen3 Butler and ObsTAP tables are still being worked out.
The general overview of this test will be to use some combination of the
Gen3 Butler and TAP access to the ObsTAP tables to test that the
required access is provided.

}
\begin{tabular}{p{3cm}p{13cm}}
\hline
            & Expected Result \\ \hline
\end{tabular}
{\scriptsize
Verification that the relevant data products and their related tables,
metadata, and provenance information are available and readily
accessible.

}

\subsubsection{LVV-T137 - Verify implementation of Data Product Ingest}\label{lvv-t137}

\begin{longtable}[]{llllll}
\toprule
Version & Status & Priority & Verification Type & Owner
\\\midrule
1 & Defined & Normal &
Test & Colin Slater
\\\bottomrule
\multicolumn{6}{c}{ Open \href{https://jira.lsstcorp.org/secure/Tests.jspa\#/testCase/LVV-T137}{LVV-T137} in Jira } \\
\end{longtable}

\paragraph{Verification Elements}\mbox{}\\

\begin{itemize}
\item \href{https://jira.lsstcorp.org/browse/LVV-130}{LVV-130} - DMS-REQ-0299-V-01: Data Product Ingest

\end{itemize}

\paragraph{Test Items}\mbox{}\\

Verify that data products can be ingested.








\paragraph{Test Procedure}\mbox{}\\
\begin{tabular}{p{4cm}p{12cm}}
\toprule
Step 1
& Description \\ \hline
\end{tabular}
{\scriptsize
Identify a suitable set of raw data to be run through ``mini-DRP''
processing.

}
\begin{tabular}{p{3cm}p{13cm}}
\hline
            & Expected Result \\ \hline
\end{tabular}

\begin{tabular}{p{4cm}p{12cm}}
\toprule
Step 2-1
{\scriptsize from \hyperref[lvv-t1064]{LVV-T1064} }
& Description \\ \hline
\end{tabular}
{\scriptsize
Process data with the Data Release Production payload, starting from raw
science images and generating science data products, placing them in the
Data Backbone.

}
\begin{tabular}{p{3cm}p{13cm}}
\hline
            & Expected Result \\ \hline
\end{tabular}

\begin{tabular}{p{4cm}p{12cm}}
\toprule
Step 3-1
{\scriptsize from \hyperref[lvv-t987]{LVV-T987} }
& Description \\ \hline
\end{tabular}
{\scriptsize
Identify the path to the data repository, which we will refer to as
`DATA/path', then execute the following:

}
\begin{tabular}{p{3cm}p{13cm}}
\hline
            & Example Code \\ \hline
\end{tabular}
{\scriptsize
\begin{verbatim}
import lsst.daf.persistence as dafPersist
butler = dafPersist.Butler(inputs='DATA/path')
\end{verbatim}

}
\begin{tabular}{p{3cm}p{13cm}}
\hline
            & Expected Result \\ \hline
\end{tabular}
{\scriptsize
Butler repo available for reading.

}

\begin{tabular}{p{4cm}p{12cm}}
\toprule
Step 4
& Description \\ \hline
\end{tabular}
{\scriptsize
Confirm that the data products from the DRP processing have been
ingested into the Data Backbone.

}
\begin{tabular}{p{3cm}p{13cm}}
\hline
            & Expected Result \\ \hline
\end{tabular}
{\scriptsize
Processed images, catalogs, calibration information, and other related
data products are present and accessible via the Butler.

}

\subsubsection{LVV-T140 - Verify implementation of Production Orchestration}\label{lvv-t140}

\begin{longtable}[]{llllll}
\toprule
Version & Status & Priority & Verification Type & Owner
\\\midrule
1 & Defined & Normal &
Test & Robert Gruendl
\\\bottomrule
\multicolumn{6}{c}{ Open \href{https://jira.lsstcorp.org/secure/Tests.jspa\#/testCase/LVV-T140}{LVV-T140} in Jira } \\
\end{longtable}

\paragraph{Verification Elements}\mbox{}\\

\begin{itemize}
\item \href{https://jira.lsstcorp.org/browse/LVV-133}{LVV-133} - DMS-REQ-0302-V-01: Production Orchestration

\end{itemize}

\paragraph{Test Items}\mbox{}\\

Demonstrate use to orchestration software to perform real-time and batch
production on LSST compute platform(s).








\paragraph{Test Procedure}\mbox{}\\
\begin{tabular}{p{4cm}p{12cm}}
\toprule
Step 1
& Description \\ \hline
\end{tabular}
{\scriptsize
Identify an appropriate precursor dataset.

}
\begin{tabular}{p{3cm}p{13cm}}
\hline
            & Expected Result \\ \hline
\end{tabular}

\begin{tabular}{p{4cm}p{12cm}}
\toprule
Step 2
& Description \\ \hline
\end{tabular}
{\scriptsize
Execute a batch processing job using the orchestration system, and
confirm (manually and/or via QA tools typically used for HSC
reprocessing) that the pipeline executed and produced all expected
products (or error logs in cases of failure).

}
\begin{tabular}{p{3cm}p{13cm}}
\hline
            & Expected Result \\ \hline
\end{tabular}
{\scriptsize
Calexp single-visit and coadd images, and associated catalogs, are
present in a Butler repository. Logs of the processing are available to
be inspected for identification of problems in the processing.

}

\subsubsection{LVV-T141 - Verify implementation of Production Monitoring}\label{lvv-t141}

\begin{longtable}[]{llllll}
\toprule
Version & Status & Priority & Verification Type & Owner
\\\midrule
1 & Defined & Normal &
Test & Robert Gruendl
\\\bottomrule
\multicolumn{6}{c}{ Open \href{https://jira.lsstcorp.org/secure/Tests.jspa\#/testCase/LVV-T141}{LVV-T141} in Jira } \\
\end{longtable}

\paragraph{Verification Elements}\mbox{}\\

\begin{itemize}
\item \href{https://jira.lsstcorp.org/browse/LVV-134}{LVV-134} - DMS-REQ-0303-V-01: Production Monitoring

\end{itemize}

\paragraph{Test Items}\mbox{}\\

Demonstrate monitoring capabilities that give real-time view of pipeline
execution and production systems usage/load.


\paragraph{Predecessors}\mbox{}\\
\href{https://jira.lsstcorp.org/secure/Tests.jspa\#/testCase/LVV-T140}{LVV-T140}​​​​






\paragraph{Test Procedure}\mbox{}\\
\begin{tabular}{p{4cm}p{12cm}}
\toprule
Step 1-1
{\scriptsize from \hyperref[lvv-t1064]{LVV-T1064} }
& Description \\ \hline
\end{tabular}
{\scriptsize
Process data with the Data Release Production payload, starting from raw
science images and generating science data products, placing them in the
Data Backbone.

}
\begin{tabular}{p{3cm}p{13cm}}
\hline
            & Expected Result \\ \hline
\end{tabular}

\begin{tabular}{p{4cm}p{12cm}}
\toprule
Step 2
& Description \\ \hline
\end{tabular}
{\scriptsize
While DRP processing is executing, monitor the progress and resource
usage of processing.

}
\begin{tabular}{p{3cm}p{13cm}}
\hline
            & Expected Result \\ \hline
\end{tabular}
{\scriptsize
Ability to monitor in real-time the orchestrated production processing,
including resource usage.

}

\subsubsection{LVV-T150 - Verify implementation of Maintain Archive Publicly Accessible}\label{lvv-t150}

\begin{longtable}[]{llllll}
\toprule
Version & Status & Priority & Verification Type & Owner
\\\midrule
1 & Defined & Normal &
Test & Colin Slater
\\\bottomrule
\multicolumn{6}{c}{ Open \href{https://jira.lsstcorp.org/secure/Tests.jspa\#/testCase/LVV-T150}{LVV-T150} in Jira } \\
\end{longtable}

\paragraph{Verification Elements}\mbox{}\\

\begin{itemize}
\item \href{https://jira.lsstcorp.org/browse/LVV-34}{LVV-34} - DMS-REQ-0077-V-01: Maintain Archive Publicly Accessible

\end{itemize}

\paragraph{Test Items}\mbox{}\\

Verify that prior data releases remain accessible.








\paragraph{Test Procedure}\mbox{}\\
\begin{tabular}{p{4cm}p{12cm}}
\toprule
Step 1
& Description \\ \hline
\end{tabular}
{\scriptsize
Confirm that at least two data releases (the most recent, and one
previous) are accessible to users (and can be queried) from the standard
channels.~~

}
\begin{tabular}{p{3cm}p{13cm}}
\hline
            & Expected Result \\ \hline
\end{tabular}
{\scriptsize
Simple queries return catalog data from the data releases that are
available in QSERV.

}

\begin{tabular}{p{4cm}p{12cm}}
\toprule
Step 2
& Description \\ \hline
\end{tabular}
{\scriptsize
Confirm that previous data releases are accessible for bulk download
(perhaps with significant latency) from tape or other bulk store, and
that the downloaded tables contain the expected data products.

}
\begin{tabular}{p{3cm}p{13cm}}
\hline
            & Expected Result \\ \hline
\end{tabular}
{\scriptsize
A download of an entire previous data release from its bulk store.

}

\subsubsection{LVV-T153 - Verify implementation of Provide Engineering and Facility Database
Archive}\label{lvv-t153}

\begin{longtable}[]{llllll}
\toprule
Version & Status & Priority & Verification Type & Owner
\\\midrule
1 & Defined & Normal &
Test & Robert Gruendl
\\\bottomrule
\multicolumn{6}{c}{ Open \href{https://jira.lsstcorp.org/secure/Tests.jspa\#/testCase/LVV-T153}{LVV-T153} in Jira } \\
\end{longtable}

\paragraph{Verification Elements}\mbox{}\\

\begin{itemize}
\item \href{https://jira.lsstcorp.org/browse/LVV-44}{LVV-44} - DMS-REQ-0102-V-01: Provide Engineering \& Facility Database Archive

\end{itemize}

\paragraph{Test Items}\mbox{}\\

Demonstrate Engineering and Facilities Data (images, associated
metadata, and observatory environment and control data) are archived and
available for public access within \textbf{L1PublicT (24 hours)}.~








\paragraph{Test Procedure}\mbox{}\\
\begin{tabular}{p{4cm}p{12cm}}
\toprule
Step 1
& Description \\ \hline
\end{tabular}
{\scriptsize
Execute a single-day operations rehearsal, ingesting (simulated) OCS
commands into the EFD.~

}
\begin{tabular}{p{3cm}p{13cm}}
\hline
            & Expected Result \\ \hline
\end{tabular}

\begin{tabular}{p{4cm}p{12cm}}
\toprule
Step 2
& Description \\ \hline
\end{tabular}
{\scriptsize
Wait at least \textbf{L1PublicT=24} hours, then access the archived EFD.
Confirm that the data products are present in the archived EFD
after~\textbf{L1PublicT=24} hours have elapsed.

}
\begin{tabular}{p{3cm}p{13cm}}
\hline
            & Expected Result \\ \hline
\end{tabular}
{\scriptsize
The EFD contains the simulated OCS commands, and they were ingested
within~\textbf{L1PublicT=24} hours of the operations rehearsal.

}

\begin{tabular}{p{4cm}p{12cm}}
\toprule
Step 3
& Description \\ \hline
\end{tabular}
{\scriptsize
From the public access portal to the EFD, execute a query and
demonstrate that the data are publicly available.

}
\begin{tabular}{p{3cm}p{13cm}}
\hline
            & Expected Result \\ \hline
\end{tabular}
{\scriptsize
A query at the public interface to the EFD successfully executes and
returns EFD data.

}

\subsubsection{LVV-T183 - Verify implementation of DMS Communication with OCS}\label{lvv-t183}

\begin{longtable}[]{llllll}
\toprule
Version & Status & Priority & Verification Type & Owner
\\\midrule
1 & Defined & Normal &
Test & Gregory Dubois-Felsmann
\\\bottomrule
\multicolumn{6}{c}{ Open \href{https://jira.lsstcorp.org/secure/Tests.jspa\#/testCase/LVV-T183}{LVV-T183} in Jira } \\
\end{longtable}

\paragraph{Verification Elements}\mbox{}\\

\begin{itemize}
\item \href{https://jira.lsstcorp.org/browse/LVV-146}{LVV-146} - DMS-REQ-0315-V-01: DMS Communication with OCS

\end{itemize}

\paragraph{Test Items}\mbox{}\\

Verify that the DMS at the Base Facility can receive commands from the
OCS and send command responses, events, and telemetry back. ~Verified by
Early Integration activities and during AuxTel commissioning.








\paragraph{Test Procedure}\mbox{}\\
\begin{tabular}{p{4cm}p{12cm}}
\toprule
Step 1
& Description \\ \hline
\end{tabular}
{\scriptsize
From the Base Site, connect to the (simulated) OCS telemetry stream.

}
\begin{tabular}{p{3cm}p{13cm}}
\hline
            & Expected Result \\ \hline
\end{tabular}

\begin{tabular}{p{4cm}p{12cm}}
\toprule
Step 2
& Description \\ \hline
\end{tabular}
{\scriptsize
Send a command to the OCS, and observe that the command has been
executed.

}
\begin{tabular}{p{3cm}p{13cm}}
\hline
            & Expected Result \\ \hline
\end{tabular}
{\scriptsize
Confirmation that the OCS command successfully executed.

}

\begin{tabular}{p{4cm}p{12cm}}
\toprule
Step 3
& Description \\ \hline
\end{tabular}
{\scriptsize
Extract information from the telemetry being broadcast by the OCS, and
ensure that these data are readable.

}
\begin{tabular}{p{3cm}p{13cm}}
\hline
            & Expected Result \\ \hline
\end{tabular}
{\scriptsize
A readable extract from the OCS telemetry stream.

}

\subsubsection{LVV-T385 - Verify implementation of minimum number of simultaneous retrievals of
CCD-sized coadd cutouts}\label{lvv-t385}

\begin{longtable}[]{llllll}
\toprule
Version & Status & Priority & Verification Type & Owner
\\\midrule
1 & Defined & Normal &
Test & Leanne Guy
\\\bottomrule
\multicolumn{6}{c}{ Open \href{https://jira.lsstcorp.org/secure/Tests.jspa\#/testCase/LVV-T385}{LVV-T385} in Jira } \\
\end{longtable}

\paragraph{Verification Elements}\mbox{}\\

\begin{itemize}
\item \href{https://jira.lsstcorp.org/browse/LVV-3394}{LVV-3394} - DMS-REQ-0377-V-01: Min number of simultaneous single-CCD coadd cutout
image users

\end{itemize}

\paragraph{Test Items}\mbox{}\\

Verify that at least \textbf{ccdRetrievalUsers = 20~}users can
simultaneously retrieve a single CCD-sized coadd cutout using the IVOA
SODA protocol.~








\paragraph{Test Procedure}\mbox{}\\
\begin{tabular}{p{4cm}p{12cm}}
\toprule
Step 1
& Description \\ \hline
\end{tabular}
{\scriptsize
Confirm that CCD-sized cutouts from coadds, also containing mask and
variance planes, are available on the SODA server. If none are
available, copy an image (or some images) to the server.

}
\begin{tabular}{p{3cm}p{13cm}}
\hline
            & Expected Result \\ \hline
\end{tabular}
{\scriptsize
At least one CCD-sized coadd cutout is available, and is a well-formed
image.

}

\begin{tabular}{p{4cm}p{12cm}}
\toprule
Step 2
& Description \\ \hline
\end{tabular}
{\scriptsize
Simulate SODA queries by at least \textbf{ccdRetrievalUsers = 20~}users
at the same time.

}
\begin{tabular}{p{3cm}p{13cm}}
\hline
            & Expected Result \\ \hline
\end{tabular}

\begin{tabular}{p{4cm}p{12cm}}
\toprule
Step 3
& Description \\ \hline
\end{tabular}
{\scriptsize
Confirm that all simulated users retrieved the desired image(s), and
that the returned images are well-formed, with (at least) image, mask,
and variance planes.

}
\begin{tabular}{p{3cm}p{13cm}}
\hline
            & Expected Result \\ \hline
\end{tabular}
{\scriptsize
All of the simulated~\textbf{ccdRetrievalUsers = 20~}users retrieved
images within the specified time (see related Verification Element and
Test Case).

}

\subsubsection{LVV-T1252 - Verify number of simultaneous alert filter users}\label{lvv-t1252}

\begin{longtable}[]{llllll}
\toprule
Version & Status & Priority & Verification Type & Owner
\\\midrule
1 & Defined & Normal &
Test & Eric Bellm
\\\bottomrule
\multicolumn{6}{c}{ Open \href{https://jira.lsstcorp.org/secure/Tests.jspa\#/testCase/LVV-T1252}{LVV-T1252} in Jira } \\
\end{longtable}

\paragraph{Verification Elements}\mbox{}\\

\begin{itemize}
\item \href{https://jira.lsstcorp.org/browse/LVV-9748}{LVV-9748} - DMS-REQ-0343-V-02: Number of simultaneous users

\end{itemize}

\paragraph{Test Items}\mbox{}\\

Verify that the DMS alert filter service supports \textbf{numBrokerUsers
= 100~}simultaneous brokers.








\paragraph{Test Procedure}\mbox{}\\
\begin{tabular}{p{4cm}p{12cm}}
\toprule
Step 1
& Description \\ \hline
\end{tabular}
{\scriptsize
Create a simulated alert stream.

}
\begin{tabular}{p{3cm}p{13cm}}
\hline
            & Expected Result \\ \hline
\end{tabular}

\begin{tabular}{p{4cm}p{12cm}}
\toprule
Step 2
& Description \\ \hline
\end{tabular}
{\scriptsize
Simultaneously execute user-defined alert filters for at least
\textbf{numBrokerUsers = 100} users, and confirm that the system
successfully filters the stream as requested. Confirm that the bandwidth
requirement of \textbf{numBrokerAlerts = 20~}per user was
met.Simultaneously execute user-defined alert filters for at least 100
users, and confirm that the system successfully filters the stream as
requested.

}
\begin{tabular}{p{3cm}p{13cm}}
\hline
            & Expected Result \\ \hline
\end{tabular}
{\scriptsize
All of the (simulated) \textbf{numBrokerUsers = 100~}users successfully
receive their requested filtered alerts.

}

\subsubsection{LVV-T1332 - Verify implementation of maximum time for retrieval of CCD-sized coadd
cutouts}\label{lvv-t1332}

\begin{longtable}[]{llllll}
\toprule
Version & Status & Priority & Verification Type & Owner
\\\midrule
1 & Defined & Normal &
Test & Leanne Guy
\\\bottomrule
\multicolumn{6}{c}{ Open \href{https://jira.lsstcorp.org/secure/Tests.jspa\#/testCase/LVV-T1332}{LVV-T1332} in Jira } \\
\end{longtable}

\paragraph{Verification Elements}\mbox{}\\

\begin{itemize}
\item \href{https://jira.lsstcorp.org/browse/LVV-9797}{LVV-9797} - DMS-REQ-0377-V-02: Max time to retrieve single-CCD coadd cutout image

\end{itemize}

\paragraph{Test Items}\mbox{}\\

Verify that at least \textbf{ccdRetrievalUsers = 20~}users can retrieve
CCD-sized coadd cutouts using the IVOA SODA protocol within a maximum
retrieval time of~\textbf{ccdRetrievalTime = 15 seconds}.








\paragraph{Test Procedure}\mbox{}\\
\begin{tabular}{p{4cm}p{12cm}}
\toprule
Step 1
& Description \\ \hline
\end{tabular}
{\scriptsize
Confirm that CCD-sized cutouts from coadds, also containing mask and
variance planes, are available on the SODA server. If none are
available, copy an image (or some images) to the server.

}
\begin{tabular}{p{3cm}p{13cm}}
\hline
            & Expected Result \\ \hline
\end{tabular}
{\scriptsize
At least one CCD-sized coadd cutout is available, and is a well-formed
image.

}

\begin{tabular}{p{4cm}p{12cm}}
\toprule
Step 2
& Description \\ \hline
\end{tabular}
{\scriptsize
Simulate SODA queries by at least \textbf{ccdRetrievalUsers = 20~}users
at the same time.

}
\begin{tabular}{p{3cm}p{13cm}}
\hline
            & Expected Result \\ \hline
\end{tabular}

\begin{tabular}{p{4cm}p{12cm}}
\toprule
Step 3
& Description \\ \hline
\end{tabular}
{\scriptsize
Monitor the time that each query takes to complete, and confirm that all
simulated users retrieved the desired image(s)
within~\textbf{ccdRetrievalTime = 15 seconds.}

}
\begin{tabular}{p{3cm}p{13cm}}
\hline
            & Expected Result \\ \hline
\end{tabular}
{\scriptsize
All of the simulated \textbf{ccdRetrievalUsers = 20~}users retrieved
images within~\textbf{ccdRetrievalTime = 15 seconds.}

}

\subsection{ Approved Test Cases}

\subsubsection{LVV-T10 - DRP-00-00: Installation of the Data Release Production v14.0 science
payload}\label{lvv-t10}

\begin{longtable}[]{llllll}
\toprule
Version & Status & Priority & Verification Type & Owner
\\\midrule
1 & Approved & Normal &
Test & Jim Bosch
\\\bottomrule
\multicolumn{6}{c}{ Open \href{https://jira.lsstcorp.org/secure/Tests.jspa\#/testCase/LVV-T10}{LVV-T10} in Jira } \\
\end{longtable}

\paragraph{Verification Elements}\mbox{}\\

\begin{itemize}
\item \href{https://jira.lsstcorp.org/browse/LVV-139}{LVV-139} - DMS-REQ-0308-V-01: Software Architecture to Enable Community Re-Use

\end{itemize}

\paragraph{Test Items}\mbox{}\\

This test will check:

\begin{itemize}
\tightlist
\item
  That the Data Release Production science payload is available for
  distribution from documented channels;
\item
  That the Data Release Production science payload can be installed on
  LSST Data Facility-managed systems.
\end{itemize}



\paragraph{Environment Needs}\mbox{}\\

\subparagraph{Software}\mbox{}\\
All prerequisite packages listed at https://pipelines.lsst.io/install/
prereqs/centos.html must be available on the test system and on the
LSST-VC compute node.

\subparagraph{Hardware}\mbox{}\\
This test case shall be executed on a developer system at the LSST Data
Facility which serves as the ``head node'' or otherwise provides access
to filesystems shared by the LSST Verification Cluster (LSST-VC). We
assume that this system will be lsst-dev01.ncsa.illinois.edu and the
filesystem will be a GPFS-based system mounted at /software. The test
also requires access to one LSST-VC compute node.

\paragraph{Input Specification}\mbox{}\\
No input data is required for this test case.

\paragraph{Output Specification}\mbox{}\\
The Data Release Production science payload will be made available on a
shared filesystem accessible from LSST-VC compute notes.

\paragraph{Test Procedure}\mbox{}\\
\begin{tabular}{p{4cm}p{12cm}}
\toprule
Step 1
& Description \\ \hline
\end{tabular}
{\scriptsize
Release 14.0 of the LSST Science Pipelines will be installed into the
GPFS filesystem accessible at /software on lsst-dev01 following the
instructions at https://pipelines.lsst. io/install/newinstall.html.

}
\begin{tabular}{p{3cm}p{13cm}}
\hline
            & Expected Result \\ \hline
\end{tabular}

\begin{tabular}{p{4cm}p{12cm}}
\toprule
Step 2
& Description \\ \hline
\end{tabular}
{\scriptsize
The lsst\_distrib top level package will be enabled:\\
\hspace*{0.333em} ~ ~ ~source /software/lsstsw/stack3/loadLSST.bash\\
\hspace*{0.333em} ~ ~ ~setup lsst\_distrib

}
\begin{tabular}{p{3cm}p{13cm}}
\hline
            & Expected Result \\ \hline
\end{tabular}

\begin{tabular}{p{4cm}p{12cm}}
\toprule
Step 3
& Description \\ \hline
\end{tabular}
{\scriptsize
The ``LSST Stack Demo'' package will be downloaded onto the test system
from https: //github.com/lsst/lsst\_dm\_stack\_demo/releases/tag/14.0
and uncompressed.

}
\begin{tabular}{p{3cm}p{13cm}}
\hline
            & Expected Result \\ \hline
\end{tabular}

\begin{tabular}{p{4cm}p{12cm}}
\toprule
Step 4
& Description \\ \hline
\end{tabular}
{\scriptsize
The demo package will be executed by following the instructions in its
``README`` file. The string ``Ok.`` should be returned.

}
\begin{tabular}{p{3cm}p{13cm}}
\hline
            & Expected Result \\ \hline
\end{tabular}

\begin{tabular}{p{4cm}p{12cm}}
\toprule
Step 5
& Description \\ \hline
\end{tabular}
{\scriptsize
A shell on an LSST-VC compute node will now be obtained by executing:\\
\hspace*{0.333em} ~ ~ ~\$ srun -I --pty bash

}
\begin{tabular}{p{3cm}p{13cm}}
\hline
            & Expected Result \\ \hline
\end{tabular}

\begin{tabular}{p{4cm}p{12cm}}
\toprule
Step 6
& Description \\ \hline
\end{tabular}
{\scriptsize
The demo package will be executed on the compute node and the same
result obtained.

}
\begin{tabular}{p{3cm}p{13cm}}
\hline
            & Expected Result \\ \hline
\end{tabular}

\subsubsection{LVV-T11 - DRP-00-05: Execution of the DRP Science Payload by the Batch Production
Service}\label{lvv-t11}

\begin{longtable}[]{llllll}
\toprule
Version & Status & Priority & Verification Type & Owner
\\\midrule
1 & Approved & Normal &
Test & Jim Bosch
\\\bottomrule
\multicolumn{6}{c}{ Open \href{https://jira.lsstcorp.org/secure/Tests.jspa\#/testCase/LVV-T11}{LVV-T11} in Jira } \\
\end{longtable}

\paragraph{Verification Elements}\mbox{}\\

\begin{itemize}
\item \href{https://jira.lsstcorp.org/browse/LVV-46}{LVV-46} - DMS-REQ-0106-V-01: Coadded Image Provenance

\item \href{https://jira.lsstcorp.org/browse/LVV-124}{LVV-124} - DMS-REQ-0293-V-01: Selection of Datasets

\item \href{https://jira.lsstcorp.org/browse/LVV-134}{LVV-134} - DMS-REQ-0303-V-01: Production Monitoring

\item \href{https://jira.lsstcorp.org/browse/LVV-133}{LVV-133} - DMS-REQ-0302-V-01: Production Orchestration

\item \href{https://jira.lsstcorp.org/browse/LVV-136}{LVV-136} - DMS-REQ-0305-V-01: Task Specification

\item \href{https://jira.lsstcorp.org/browse/LVV-137}{LVV-137} - DMS-REQ-0306-V-01: Task Configuration

\item \href{https://jira.lsstcorp.org/browse/LVV-62}{LVV-62} - DMS-REQ-0158-V-01: Provide Pipeline Construction Services

\end{itemize}

\paragraph{Test Items}\mbox{}\\

This test will check that the DRP Science Payload can be executed using
a specific version of the Batch Production Service provided by the LSST
Data Facility. Since the outputs are stored in the Data Backbone, it too
is a component of this test.


\paragraph{Predecessors}\mbox{}\\
LVV-T10 (DRP-00-00)

\paragraph{Environment Needs}\mbox{}\\

\subparagraph{Software}\mbox{}\\
All the necessary software will be pre-installed. The software includes
the science pipeline codes as well as the Data Management system codes
(Batch Processing Service, Data Backbone).\\
For LDM-503-2, Python 2 versions of software will be used. The science
pipeline codes will be provided via the LSST DM Software Stack, release
14.0. The Batch Processing Service and the Data Backbone are initial
versions using the DESDM Framework packages. LSST-specific plugins as
well as DRP pipeline integration codes are also pre-installed. All
python DESDM Framework packages, plugins and integration codes exist in
the lsst-dm github with tag 1.01. The DESDM prerequisites come from the
official DESDM eups package firstcut Y4N+5. They are installed using
DESDMâ\euro{}™s eups install process.\\
The ticket branch tickets/DM-12291 of the LSST Software Stack packages
meas\_base, pipe\_tasks, and obs\_subaru will be used to change the
patch ID naming convention. This is due to issues of having commas in
the filenames and data IDs, as discussed in RFC-361; the solution has
been agreed in RFC-365 for future implementation in DM-11874, DM-11875,
and DM-11876.\\
For LDM-503-2, the software will be installed into the GPFS space at
/project/production/ on LSST-VC. A single eups prototype package will be
defined to encompass the above mentioned software.

\subparagraph{Hardware}\mbox{}\\
This test case shall be executed on a testbed at the LSST Data Facility.
For LDM-503-2, this testbed includes:

\begin{itemize}
\tightlist
\item
  LSST Verification Cluster (LSST-VC) with Slurm Job Scheduler
\item
  Submit and compute nodes with read/write access to various GPFS shared
  filesystems:

  \begin{enumerate}
  \tightlist
  \item
    Filesystem containing the software stack
  \item
    Filesystem for the submit side temporary outputs
  \item
    Filesystem being used by the prototype DataBackbone.(This means that
    the frame-work can use a cp transfer protocol between the job and
    the Data Backbone and does not require additional transfer services
    to be running.)
  \item
    Filesystem for the individual job scratch directories.
  \end{enumerate}
\item
  Single node Oracle database (version 12c)
\item
  Submit node (lsst-dev01.ncsa.illinois.edu) running the HTCondor
  Central Manager (ver- sion 8.7.3).
\end{itemize}

\paragraph{Input Specification}\mbox{}\\
A small number of selected tracts of the Hyper Suprime-Cam dataset will
be used along with appropriate calibration datasets.\\
For LDM-503-2, the three tracts of the Hyper Suprime-Cam ``RC1''
dataset, as defined Appendix A.1.1, will be used. The calibration
dataset will be the 20170105 version, defined as per DMTR- 31. Raw files
known to fail processCcd will be blacklisted.\\[2\baselineskip]

\paragraph{Output Specification}\mbox{}\\
The output data products will be available from the Data Backbone.\\
For LDM-503-2, the output data products will be available on the LSST-VC
via a shared filesystem and advanced data discovery is done via SQL
queries against the Oracle database.

\paragraph{Test Procedure}\mbox{}\\
\begin{tabular}{p{4cm}p{12cm}}
\toprule
Step 1
& Description \\ \hline
\end{tabular}
{\scriptsize
Setup. The LSST Science Pipelines and the DESDM Framework, plugins, and
integration codes as described in Environment - Software paragraph have
already been installed. The Operator merely sets up the expanded stack
using eups.

}
\begin{tabular}{p{3cm}p{13cm}}
\hline
            & Expected Result \\ \hline
\end{tabular}

\begin{tabular}{p{4cm}p{12cm}}
\toprule
Step 2
& Description \\ \hline
\end{tabular}
{\scriptsize
Input raw and calibration data must exist in the Data Backbone. If not,
the data will be ingested into Data Backbone.

}
\begin{tabular}{p{3cm}p{13cm}}
\hline
            & Expected Result \\ \hline
\end{tabular}

\begin{tabular}{p{4cm}p{12cm}}
\toprule
Step 3
& Description \\ \hline
\end{tabular}
{\scriptsize
The operator tags and blacklists input data as appropriate for test (see
Input Specifications ยง4.2.5).

}
\begin{tabular}{p{3cm}p{13cm}}
\hline
            & Expected Result \\ \hline
\end{tabular}

\begin{tabular}{p{4cm}p{12cm}}
\toprule
Step 4
& Description \\ \hline
\end{tabular}
{\scriptsize
Given the LSST Science Pipelines version, the operator will generate the
full config files and schema files (Test Configuration ยง4.2.7), which
are then ingested into the Data Backbone.

}
\begin{tabular}{p{3cm}p{13cm}}
\hline
            & Expected Result \\ \hline
\end{tabular}

\begin{tabular}{p{4cm}p{12cm}}
\toprule
Step 5
& Description \\ \hline
\end{tabular}
{\scriptsize
Write a DRP pipeline workflow definition file from scratch or modify an
existing file from github making its operations- and dataset-specific
inputs match this test.

\begin{itemize}
\tightlist
\item
  (a) For LDM-503-2, the pipeline workflow definition file is written in
  a workflow control language (wcl) format as used by the DESDM
  Framework.
\end{itemize}

}
\begin{tabular}{p{3cm}p{13cm}}
\hline
            & Expected Result \\ \hline
\end{tabular}

\begin{tabular}{p{4cm}p{12cm}}
\toprule
Step 6
& Description \\ \hline
\end{tabular}
{\scriptsize
Make special hardware requests (e.g., disk or compute node reservations)
if needed.

}
\begin{tabular}{p{3cm}p{13cm}}
\hline
            & Expected Result \\ \hline
\end{tabular}

\begin{tabular}{p{4cm}p{12cm}}
\toprule
Step 7
& Description \\ \hline
\end{tabular}
{\scriptsize
Execution starts. If HTCondor processes are not already running, start
HTCondor processes on compute nodes. This step makes the compute nodes
join the HTCondor Central Manager to cre- ate a working HTCondor Pool.

}
\begin{tabular}{p{3cm}p{13cm}}
\hline
            & Expected Result \\ \hline
\end{tabular}

\begin{tabular}{p{4cm}p{12cm}}
\toprule
Step 8
& Description \\ \hline
\end{tabular}
{\scriptsize
The execution for each tract of the input data in the Input
specification section (ยง4.2.5) will be submitted to the hardware
specified in ``Environmental Needs - Hardware'' section (ยง4.2.4.1)
using the configuration specified in ``Test configuration'' section
(ยง4.2.7).

}
\begin{tabular}{p{3cm}p{13cm}}
\hline
            & Expected Result \\ \hline
\end{tabular}

\begin{tabular}{p{4cm}p{12cm}}
\toprule
Step 9
& Description \\ \hline
\end{tabular}
{\scriptsize
During execution, the operator will use software to demonstrate the
ability to check the processing status.

\begin{itemize}
\tightlist
\item
  (a) ForLDM-503-2,the available Batch Production Service monitoring
  software consists of two commands: one to summarize currently
  submitted processing, one to get more details of single submission.
\end{itemize}

}
\begin{tabular}{p{3cm}p{13cm}}
\hline
            & Expected Result \\ \hline
\end{tabular}

\begin{tabular}{p{4cm}p{12cm}}
\toprule
Step 10
& Description \\ \hline
\end{tabular}
{\scriptsize
If the processing attempt completes, the attempt is marked as completed
and tagged as potential for the next test steps. These campaign tags are
used to make pre-release QA queries simpler.

}
\begin{tabular}{p{3cm}p{13cm}}
\hline
            & Expected Result \\ \hline
\end{tabular}

\begin{tabular}{p{4cm}p{12cm}}
\toprule
Step 11
& Description \\ \hline
\end{tabular}
{\scriptsize
If the processing attempt fails, the attempt is marked as failed.

}
\begin{tabular}{p{3cm}p{13cm}}
\hline
            & Expected Result \\ \hline
\end{tabular}

\begin{tabular}{p{4cm}p{12cm}}
\toprule
Step 12
& Description \\ \hline
\end{tabular}
{\scriptsize
If the processing attempt fails due to certain infrastructure
configuration or transient instability (e.g., network blips), the
processing of the tract can be tried again after the problem is
resolved.

}
\begin{tabular}{p{3cm}p{13cm}}
\hline
            & Expected Result \\ \hline
\end{tabular}

\begin{tabular}{p{4cm}p{12cm}}
\toprule
Step 13
& Description \\ \hline
\end{tabular}
{\scriptsize
Checks.~When the execution finishes, the success of the execution will
be verified by checking the existence of the expected output data.

\begin{itemize}
\tightlist
\item
  (a) For each of the expected data products types (listed in ยง4.3.2)
  and each of the expected units (visits, patches, etc), verify the data
  product is in the Data Backbone and has filesize greater than zero via
  DB queries.
\item
  (b) Verify the physical and location information in Data Backbone DB
  matches theData Backbone filesystem and vice-versa.
\end{itemize}

}
\begin{tabular}{p{3cm}p{13cm}}
\hline
            & Expected Result \\ \hline
\end{tabular}

\begin{tabular}{p{4cm}p{12cm}}
\toprule
Step 14
& Description \\ \hline
\end{tabular}
{\scriptsize
Check that for each data product type has appropriate metadata saved for
each file as defined in ``Test Configuration'' section (ยง4.2.7).

}
\begin{tabular}{p{3cm}p{13cm}}
\hline
            & Expected Result \\ \hline
\end{tabular}

\begin{tabular}{p{4cm}p{12cm}}
\toprule
Step 15
& Description \\ \hline
\end{tabular}
{\scriptsize
Check provenance

\begin{itemize}
\tightlist
\item
  (a) Verify that each file can be linked with the step and processing
  attempt that created it via the Data Backbone.
\item
  (b) Verify that the information linking input files to each step was
  saved to the Oracle database.
\end{itemize}

}
\begin{tabular}{p{3cm}p{13cm}}
\hline
            & Expected Result \\ \hline
\end{tabular}

\begin{tabular}{p{4cm}p{12cm}}
\toprule
Step 16
& Description \\ \hline
\end{tabular}
{\scriptsize
Check runtime metrics, such as the number of executions of each code,
wallclock per step, wallclock per tract, etc.

}
\begin{tabular}{p{3cm}p{13cm}}
\hline
            & Expected Result \\ \hline
\end{tabular}

\subsubsection{LVV-T12 - DRP-00-10: Data Release Includes Required Data Products}\label{lvv-t12}

\begin{longtable}[]{llllll}
\toprule
Version & Status & Priority & Verification Type & Owner
\\\midrule
1 & Approved & Normal &
Test & Jim Bosch
\\\bottomrule
\multicolumn{6}{c}{ Open \href{https://jira.lsstcorp.org/secure/Tests.jspa\#/testCase/LVV-T12}{LVV-T12} in Jira } \\
\end{longtable}

\paragraph{Verification Elements}\mbox{}\\

\begin{itemize}
\item \href{https://jira.lsstcorp.org/browse/LVV-165}{LVV-165} - DMS-REQ-0334-V-01: Persisting Data Products

\item \href{https://jira.lsstcorp.org/browse/LVV-98}{LVV-98} - DMS-REQ-0267-V-01: Source Catalog

\item \href{https://jira.lsstcorp.org/browse/LVV-99}{LVV-99} - DMS-REQ-0268-V-01: Forced-Source Catalog

\item \href{https://jira.lsstcorp.org/browse/LVV-106}{LVV-106} - DMS-REQ-0275-V-01: Object Catalog

\item \href{https://jira.lsstcorp.org/browse/LVV-110}{LVV-110} - DMS-REQ-0279-V-01: Deep Detection Coadds

\item \href{https://jira.lsstcorp.org/browse/LVV-125}{LVV-125} - DMS-REQ-0294-V-01: Processing of Datasets

\end{itemize}

\paragraph{Test Items}\mbox{}\\

This test will check that the basic data products which should be in an
data release are generated by execution of the science payload.\\
These products will include:

\begin{itemize}
\tightlist
\item
  Source catalogs, derived from PVIs and coadded images (DMS-REQ-0267 \&
  DMS-REQ-0277);
\item
  Forced source catalogs (DMS-REQ-0268);
\item
  Object catalogs (DMS-REQ-0275);
\item
  Processed visit images (PVIs; DMS-REQ-0069);
\item
  Coadded images (DMS-REQ-0279);
\end{itemize}


\paragraph{Predecessors}\mbox{}\\
LVV-T10 (DRP-00-00)

\paragraph{Environment Needs}\mbox{}\\

\subparagraph{Software}\mbox{}\\
Release 14.0 of the DM Software Stack will be pre-installed (following
the procedure described in DRP-00-00)

\subparagraph{Hardware}\mbox{}\\
The test shall be carried out on a machine with at least 16 GB of RAM
and multiple CPU cores which has access to the /datasets shared (GPFS)
filesystem at the LSST Data Facility.

\paragraph{Input Specification}\mbox{}\\
A complete processing of the Hyper Suprime-Cam ``RC1'' dataset (Appendix
A.1.1 through the DRP Science Payload.\\
This dataset shall be made available in a standard LSST data repository,
accessible via the ``Data Butler''.\\
It is not required that all combinations of visit and CCD have been
processed successfully: a number of failures are expected. However,
documentation to describe processing failures should be provided.

\paragraph{Output Specification}\mbox{}\\
None.

\paragraph{Test Procedure}\mbox{}\\
\begin{tabular}{p{4cm}p{12cm}}
\toprule
Step 1
& Description \\ \hline
\end{tabular}
{\scriptsize
The DM Stack shall be initialized using the loadLSST script (as
described in DRP-00-00).

}
\begin{tabular}{p{3cm}p{13cm}}
\hline
            & Expected Result \\ \hline
\end{tabular}

\begin{tabular}{p{4cm}p{12cm}}
\toprule
Step 2
& Description \\ \hline
\end{tabular}
{\scriptsize
A ``Data Butler'' will be initialized to access the repository.

}
\begin{tabular}{p{3cm}p{13cm}}
\hline
            & Expected Result \\ \hline
\end{tabular}

\begin{tabular}{p{4cm}p{12cm}}
\toprule
Step 3
& Description \\ \hline
\end{tabular}
{\scriptsize
For each of the expected data products types (listed in Test Items
section ยง4.3.2) and each of the expected units (PVIs, coadds, etc), the
data product will be retrieved from the Butler and verified to be
non-empty.

}
\begin{tabular}{p{3cm}p{13cm}}
\hline
            & Expected Result \\ \hline
\end{tabular}

\subsubsection{LVV-T13 - DRP-00-15: Scientific Verification of Source Catalog}\label{lvv-t13}

\begin{longtable}[]{llllll}
\toprule
Version & Status & Priority & Verification Type & Owner
\\\midrule
1 & Approved & Normal &
Test & Jim Bosch
\\\bottomrule
\multicolumn{6}{c}{ Open \href{https://jira.lsstcorp.org/secure/Tests.jspa\#/testCase/LVV-T13}{LVV-T13} in Jira } \\
\end{longtable}

\paragraph{Verification Elements}\mbox{}\\

\begin{itemize}
\item \href{https://jira.lsstcorp.org/browse/LVV-165}{LVV-165} - DMS-REQ-0334-V-01: Persisting Data Products

\item \href{https://jira.lsstcorp.org/browse/LVV-98}{LVV-98} - DMS-REQ-0267-V-01: Source Catalog

\item \href{https://jira.lsstcorp.org/browse/LVV-178}{LVV-178} - DMS-REQ-0347-V-01: Measurements in catalogs

\item \href{https://jira.lsstcorp.org/browse/LVV-162}{LVV-162} - DMS-REQ-0331-V-01: Computing Derived Quantities

\end{itemize}

\paragraph{Test Items}\mbox{}\\

This test will check that the source catalogs delivered by the DRP
science payload meet the requirements laid down by \citeds{LSE-61}.\\
Specifically, this will demonstrate that:

\begin{itemize}
\tightlist
\item
  Measurements in the catalog are presented in flux units
  (DMS-REQ-0347);
\item
  Derived quantities are provided in pre-computed columns
  (DMS-REQ-0331);
\item
  Aperture corrections for different photometry algorithms are
  consistent.
\item
  Photometry measurements are consistent with reference catalog
  photometry (including sources not used in photometric calibration).
\item
  Astrometry measurements are consistent with reference catalog
  positions (including sources not used in astrometric calibration).
\end{itemize}

This test does not include quantitative targets for the science quality
criteria; we instead require for each test that we be able to quickly
construct a plot in which such a target can be visualized.


\paragraph{Predecessors}\mbox{}\\
lvv-t10 (DRP-00-00), lvv-t12 (DRP-00-10)

\paragraph{Environment Needs}\mbox{}\\

\subparagraph{Software}\mbox{}\\
Release 14.0 of the DM Software Stack will be pre-installed (following
the procedure described in DRP-00-00).

\subparagraph{Hardware}\mbox{}\\
The test shall be carried out on a machine with at least 16 GB of RAM
and multiple CPU cores which has access to the /datasets shared (GPFS)
filesystem at the LSST Data Facility.

\paragraph{Input Specification}\mbox{}\\
A complete processing of the Hyper Suprime-Cam ``RC1'' dataset (Appendix
A.1.1 through the DRP Science Payload.\\
This dataset shall be made available in a standard LSST data repository,
accessible via the ``Data Butler''.\\
It is not required that all combinations of visit and CCD have been
processed successfully: a number of failures are expected. However,
documentation to describe processing failures should be provided.

\paragraph{Output Specification}\mbox{}\\
None.

\paragraph{Test Procedure}\mbox{}\\
\begin{tabular}{p{4cm}p{12cm}}
\toprule
Step 1
& Description \\ \hline
\end{tabular}
{\scriptsize
The DM Stack shall be initialized using the loadLSST script (as
described in LVV-T10 - DRP-00-00).

}
\begin{tabular}{p{3cm}p{13cm}}
\hline
            & Expected Result \\ \hline
\end{tabular}

\begin{tabular}{p{4cm}p{12cm}}
\toprule
Step 2
& Description \\ \hline
\end{tabular}
{\scriptsize
A ``Data Butler'' will be initialized to access the repository.

}
\begin{tabular}{p{3cm}p{13cm}}
\hline
            & Expected Result \\ \hline
\end{tabular}

\begin{tabular}{p{4cm}p{12cm}}
\toprule
Step 3
& Description \\ \hline
\end{tabular}
{\scriptsize
Scripts from the pipe\_analysis package will be run on every visit to
check for the presence of data products and make plots.

}
\begin{tabular}{p{3cm}p{13cm}}
\hline
            & Expected Result \\ \hline
\end{tabular}

\subsubsection{LVV-T14 - DRP-00-25: Scientific Verification of Object Catalog}\label{lvv-t14}

\begin{longtable}[]{llllll}
\toprule
Version & Status & Priority & Verification Type & Owner
\\\midrule
1 & Approved & Normal &
Test & Jim Bosch
\\\bottomrule
\multicolumn{6}{c}{ Open \href{https://jira.lsstcorp.org/secure/Tests.jspa\#/testCase/LVV-T14}{LVV-T14} in Jira } \\
\end{longtable}

\paragraph{Verification Elements}\mbox{}\\

\begin{itemize}
\item \href{https://jira.lsstcorp.org/browse/LVV-165}{LVV-165} - DMS-REQ-0334-V-01: Persisting Data Products

\item \href{https://jira.lsstcorp.org/browse/LVV-106}{LVV-106} - DMS-REQ-0275-V-01: Object Catalog

\item \href{https://jira.lsstcorp.org/browse/LVV-178}{LVV-178} - DMS-REQ-0347-V-01: Measurements in catalogs

\item \href{https://jira.lsstcorp.org/browse/LVV-162}{LVV-162} - DMS-REQ-0331-V-01: Computing Derived Quantities

\end{itemize}

\paragraph{Test Items}\mbox{}\\

This test will check that the object catalogs delivered by the DRP
science payload meet the requirements laid down by \citeds{LSE-61}.\\
Specifically, this will demonstrate that:

\begin{itemize}
\tightlist
\item
  Measurements in the catalog are presented in flux units
  (DMS-REQ-0347);
\item
  Derived quantities are provided in pre-computed columns
  (DMS-REQ-0331);
\item
  Aperture corrections for different photometry algorithms are
  consistent.
\item
  PSF models correctly predict the ellipticities of stars over each
  tract.
\item
  Photometry measurements are consistent with reference catalog
  photometry (including sources not used in photometric calibration).
\item
  Astrometry measurements are consistent with reference catalog
  positions (including sources not used in astrometric calibration).
\item
  Forced and unforced photometry measurements are consistent.
\item
  The slope of the stellar locus in color-color space is not a function
  of position on the sky.
\end{itemize}

This test does not include quantitative targets for the science quality
criteria; we instead re- quire for each test that we be able to quickly
construct a plot in which such a target can be visualized.\\
All science quality tests in this section shall distinguish between
blended and isolated objects.


\paragraph{Predecessors}\mbox{}\\
LVV-T10 (DRP-00-00)\\
LVV-T12 (DRP-00-10)

\paragraph{Environment Needs}\mbox{}\\

\subparagraph{Software}\mbox{}\\
Release 14.0 of the DM Software Stack will be pre-installed (following
the procedure described in DRP-00-00).

\subparagraph{Hardware}\mbox{}\\
The test shall be carried out on a machine with at least 16 GB of RAM
and multiple CPU cores which has access to the /datasets shared (GPFS)
filesystem at the LSST Data Facility.

\paragraph{Input Specification}\mbox{}\\
A complete processing of the Hyper Suprime-Cam ``RC1'' dataset (Appendix
A.1.1 through the DRP Science Payload.\\
This dataset shall be made available in a standard LSST data repository,
accessible via the ``Data Butler''.\\
It is not required that all combinations of visit and CCD have been
processed successfully: a number of failures are expected. However,
documentation to describe processing failures should be provided.

\paragraph{Output Specification}\mbox{}\\
None.

\paragraph{Test Procedure}\mbox{}\\
\begin{tabular}{p{4cm}p{12cm}}
\toprule
Step 1
& Description \\ \hline
\end{tabular}
{\scriptsize
The DM Stack shall be initialized using the loadLSST script (as
described in LVV-T10 - DRP-00-00).

}
\begin{tabular}{p{3cm}p{13cm}}
\hline
            & Expected Result \\ \hline
\end{tabular}

\begin{tabular}{p{4cm}p{12cm}}
\toprule
Step 2
& Description \\ \hline
\end{tabular}
{\scriptsize
A ``Data Butler'' will be initialized to access the repository.

}
\begin{tabular}{p{3cm}p{13cm}}
\hline
            & Expected Result \\ \hline
\end{tabular}

\begin{tabular}{p{4cm}p{12cm}}
\toprule
Step 3
& Description \\ \hline
\end{tabular}
{\scriptsize
Scripts from the pipe\_analysis package will be run on every tract to
check for the presence of data products and make plots

}
\begin{tabular}{p{3cm}p{13cm}}
\hline
            & Expected Result \\ \hline
\end{tabular}

\subsubsection{LVV-T15 - DRP-00-30: Scientific Verification of Processed Visit Images}\label{lvv-t15}

\begin{longtable}[]{llllll}
\toprule
Version & Status & Priority & Verification Type & Owner
\\\midrule
1 & Approved & Normal &
Test & Jim Bosch
\\\bottomrule
\multicolumn{6}{c}{ Open \href{https://jira.lsstcorp.org/secure/Tests.jspa\#/testCase/LVV-T15}{LVV-T15} in Jira } \\
\end{longtable}

\paragraph{Verification Elements}\mbox{}\\

\begin{itemize}
\item \href{https://jira.lsstcorp.org/browse/LVV-165}{LVV-165} - DMS-REQ-0334-V-01: Persisting Data Products

\item \href{https://jira.lsstcorp.org/browse/LVV-29}{LVV-29} - DMS-REQ-0069-V-01: Processed Visit Images

\item \href{https://jira.lsstcorp.org/browse/LVV-158}{LVV-158} - DMS-REQ-0327-V-01: Background Model Calculation

\item \href{https://jira.lsstcorp.org/browse/LVV-12}{LVV-12} - DMS-REQ-0029-V-01: Generate Photometric Zeropoint for Visit Image

\item \href{https://jira.lsstcorp.org/browse/LVV-30}{LVV-30} - DMS-REQ-0070-V-01: Generate PSF for Visit Images

\item \href{https://jira.lsstcorp.org/browse/LVV-13}{LVV-13} - DMS-REQ-0030-V-01: Absolute accuracy of WCS

\item \href{https://jira.lsstcorp.org/browse/LVV-31}{LVV-31} - DMS-REQ-0072-V-01: Processed Visit Image Content

\end{itemize}

\paragraph{Test Items}\mbox{}\\

This test will check that the Processed Visit Images (PVIs) delivered by
the DRP science payload meet the requirements laid down by \citeds{LSE-61}.\\
Specifically, this will demonstrate that:

\begin{itemize}
\tightlist
\item
  Processed visit images have been generated and persisted during
  payload execution;
\item
  Each PVI includes a background model (DMS-REQ-0327), photometric
  zero-point (DMS- REQ-0029), spatially-varying PSF (DMS-REQ-0070) and
  WCS (DMS-REQ-0030).
\item
  Saturated pixels are correctly masked.
\item
  Pixels affected by cosmic rays are correctly masked.
\item
  The background is not oversubtracted around bright objects.
\end{itemize}

This test does not include quantitative targets for the science quality
criteria; we instead re- quire for each test that we be able to quickly
construct a plot or display summary images that allow such a target can
be visualized.


\paragraph{Predecessors}\mbox{}\\
LVV-T10\\
LVV-T12

\paragraph{Environment Needs}\mbox{}\\

\subparagraph{Software}\mbox{}\\
Release 14.0 of the DM Software Stack will be pre-installed (following
the procedure described in DRP-00-00).\\[2\baselineskip]

\subparagraph{Hardware}\mbox{}\\
The test shall be carried out on a machine with at least 16 GB of RAM
and multiple CPU cores which has access to the /datasets shared (GPFS)
filesystem at the LSST Data Facility.

\paragraph{Input Specification}\mbox{}\\
A complete processing of the Hyper Suprime-Cam ``RC1'' dataset (Appendix
A.1.1 through the DRP Science Payload.\\
This dataset shall be made available in a standard LSST data repository,
accessible via the ``Data Butler''.\\
It is not required that all combinations of visit and CCD have been
processed successfully: a number of failures are expected. However,
documentation to describe processing failures should be provided.

\paragraph{Output Specification}\mbox{}\\
None.

\paragraph{Test Procedure}\mbox{}\\
\begin{tabular}{p{4cm}p{12cm}}
\toprule
Step 1
& Description \\ \hline
\end{tabular}
{\scriptsize
The DM Stack shall be initialized using the loadLSST script (as
described in LVV-T10 - DRP-00-00).

}
\begin{tabular}{p{3cm}p{13cm}}
\hline
            & Expected Result \\ \hline
\end{tabular}

\begin{tabular}{p{4cm}p{12cm}}
\toprule
Step 2
& Description \\ \hline
\end{tabular}
{\scriptsize
A ``Data Butler'' will be initialized to access the repository.

}
\begin{tabular}{p{3cm}p{13cm}}
\hline
            & Expected Result \\ \hline
\end{tabular}

\begin{tabular}{p{4cm}p{12cm}}
\toprule
Step 3
& Description \\ \hline
\end{tabular}
{\scriptsize
For each processed CCD, the PVI will be retrieved from the Butler, and
the existence of all components described in section Test Items
(ยง4.6.2) will be verified.

}
\begin{tabular}{p{3cm}p{13cm}}
\hline
            & Expected Result \\ \hline
\end{tabular}

\begin{tabular}{p{4cm}p{12cm}}
\toprule
Step 4
& Description \\ \hline
\end{tabular}
{\scriptsize
Scripts from the pipe\_analysis package will be run on every visit to
check for the presence of data products and make plots

}
\begin{tabular}{p{3cm}p{13cm}}
\hline
            & Expected Result \\ \hline
\end{tabular}

\begin{tabular}{p{4cm}p{12cm}}
\toprule
Step 5
& Description \\ \hline
\end{tabular}
{\scriptsize
Five sensors will be chosen at random from each of two visits and
inspected by eye for unmasked artifacts.

}
\begin{tabular}{p{3cm}p{13cm}}
\hline
            & Expected Result \\ \hline
\end{tabular}

\subsubsection{LVV-T16 - DRP-00-35: Scientific Verification of Coadd Images}\label{lvv-t16}

\begin{longtable}[]{llllll}
\toprule
Version & Status & Priority & Verification Type & Owner
\\\midrule
1 & Approved & Normal &
Test & Jim Bosch
\\\bottomrule
\multicolumn{6}{c}{ Open \href{https://jira.lsstcorp.org/secure/Tests.jspa\#/testCase/LVV-T16}{LVV-T16} in Jira } \\
\end{longtable}

\paragraph{Verification Elements}\mbox{}\\

\begin{itemize}
\item \href{https://jira.lsstcorp.org/browse/LVV-165}{LVV-165} - DMS-REQ-0334-V-01: Persisting Data Products

\item \href{https://jira.lsstcorp.org/browse/LVV-110}{LVV-110} - DMS-REQ-0279-V-01: Deep Detection Coadds

\item \href{https://jira.lsstcorp.org/browse/LVV-109}{LVV-109} - DMS-REQ-0278-V-01: Coadd Image Method Constraints

\item \href{https://jira.lsstcorp.org/browse/LVV-20}{LVV-20} - DMS-REQ-0047-V-01: Provide PSF for Coadded Images

\end{itemize}

\paragraph{Test Items}\mbox{}\\

This test will check that the coadded images delivered by the DRP
science payload meet the requirements laid down by \citeds{LSE-61}.\\
Specifically, this will demonstrate that:

\begin{itemize}
\tightlist
\item
  Coadds have been generated and persisted during payload execution;~
\item
  Each coadd provides a spatially varying PSF model (DMS-REQ-0047).
\item
  Saturated pixels are correctly masked.
\item
  Pixels affected by satellite trails and ghosts are rejected from the
  coadd.
\item
  The background is not oversubtracted around bright objects.
\end{itemize}

This test does not include quantitative targets for the science quality
criteria; we instead require for each test that we be able to quickly
construct a plot or display summary images that allow such a target can
be visualized.\\[2\baselineskip]


\paragraph{Predecessors}\mbox{}\\
LVV-T10 (DRP-00-00)\\
LVV-T12 (DRP-00-10)

\paragraph{Environment Needs}\mbox{}\\

\subparagraph{Software}\mbox{}\\
Release 14.0 of the DM Software Stack will be pre-installed (following
the procedure described in DRP-00-00).

\subparagraph{Hardware}\mbox{}\\
The test shall be carried out on a machine with at least 16 GB of RAM
and multiple CPU cores which has access to the /datasets shared (GPFS)
filesystem at the LSST Data Facility.

\paragraph{Input Specification}\mbox{}\\
A complete processing of the Hyper Suprime-Cam ``RC1'' dataset (Appendix
A.1.1 through the DRP Science Payload.\\
This dataset shall be made available in a standard LSST data repository,
accessible via the ``Data Butler''.\\
It is not required that all combinations of visit and CCD have been
processed successfully: a number of failures are expected. However,
documentation to describe processing failures should be provided.

\paragraph{Output Specification}\mbox{}\\
None.

\paragraph{Test Procedure}\mbox{}\\
\begin{tabular}{p{4cm}p{12cm}}
\toprule
Step 1
& Description \\ \hline
\end{tabular}
{\scriptsize
The DM Stack shall be initialized using the loadLSST script (as
described in LVV-T10 - DRP-00-00)

}
\begin{tabular}{p{3cm}p{13cm}}
\hline
            & Expected Result \\ \hline
\end{tabular}

\begin{tabular}{p{4cm}p{12cm}}
\toprule
Step 2
& Description \\ \hline
\end{tabular}
{\scriptsize
A ``Data Butler'' will be initialized to access the repository.

}
\begin{tabular}{p{3cm}p{13cm}}
\hline
            & Expected Result \\ \hline
\end{tabular}

\begin{tabular}{p{4cm}p{12cm}}
\toprule
Step 3
& Description \\ \hline
\end{tabular}
{\scriptsize
For each combination of tract/patch/filter, the PVI will be retrieved
from the Butler, and the existence of all components described in Test
items section ยง4.6.2 will be verified.

}
\begin{tabular}{p{3cm}p{13cm}}
\hline
            & Expected Result \\ \hline
\end{tabular}

\begin{tabular}{p{4cm}p{12cm}}
\toprule
Step 4
& Description \\ \hline
\end{tabular}
{\scriptsize
Scripts from the pipe\_analysis package will be run on every visit to
check for the presence of data products and make plots

}
\begin{tabular}{p{3cm}p{13cm}}
\hline
            & Expected Result \\ \hline
\end{tabular}

\begin{tabular}{p{4cm}p{12cm}}
\toprule
Step 5
& Description \\ \hline
\end{tabular}
{\scriptsize
Ten patches will be chosen at random and inspected by eye for unmasked
artifacts.

}
\begin{tabular}{p{3cm}p{13cm}}
\hline
            & Expected Result \\ \hline
\end{tabular}

\subsubsection{LVV-T17 - AG-00-00: Installation of the Alert Generation v16.0 science payload.}\label{lvv-t17}

\begin{longtable}[]{llllll}
\toprule
Version & Status & Priority & Verification Type & Owner
\\\midrule
1 & Approved & Normal &
Test & Eric Bellm
\\\bottomrule
\multicolumn{6}{c}{ Open \href{https://jira.lsstcorp.org/secure/Tests.jspa\#/testCase/LVV-T17}{LVV-T17} in Jira } \\
\end{longtable}

\paragraph{Verification Elements}\mbox{}\\

\begin{itemize}
\item \href{https://jira.lsstcorp.org/browse/LVV-139}{LVV-139} - DMS-REQ-0308-V-01: Software Architecture to Enable Community Re-Use

\end{itemize}

\paragraph{Test Items}\mbox{}\\

This test will check:

\begin{itemize}
\tightlist
\item
  That the Alert Generation science payload is available for
  distribution from documented channels;
\item
  That the Alert Generation science payload can be installed on LSST
  Data Facility-managed systems.
\end{itemize}


\paragraph{Predecessors}\mbox{}\\
None.

\paragraph{Environment Needs}\mbox{}\\

\subparagraph{Software}\mbox{}\\
All prerequisite packages listed at
https://pipelines.lsst.io/install/prereqs/centos.html must be available
on the test system and on the LSST-VC compute node.

\subparagraph{Hardware}\mbox{}\\
This test case shall be executed on a developer system at NCSA which
serves as the ``head node'' or otherwise provides access to filesystems
shared by the LSST Verification Cluster (LSST-VC). We assume that this
system will be lsst-dev01.ncsa.illinois.edu and the filesystem will be a
GPFS-based system mounted at /software.\\
The test also requires access to one LSST-VC compute node.

\paragraph{Input Specification}\mbox{}\\
No input data is required for this test case.

\paragraph{Output Specification}\mbox{}\\
The Alert Generation science payload will be made available on a shared
filesystem accessible from LSST-VC compute notes.

\paragraph{Test Procedure}\mbox{}\\
\begin{tabular}{p{4cm}p{12cm}}
\toprule
Step 1
& Description \\ \hline
\end{tabular}
{\scriptsize
Release 16.0 of the LSST Science Pipelines will be installed into the
GPFS filesystem accessible at /software on lsst-dev01 following the
instructions at \url{https://pipelines.lsst.io/install/newinstall.html}
.

}
\begin{tabular}{p{3cm}p{13cm}}
\hline
            & Expected Result \\ \hline
\end{tabular}

\begin{tabular}{p{4cm}p{12cm}}
\toprule
Step 2
& Description \\ \hline
\end{tabular}
{\scriptsize
The lsst\_distrib top level package will be
enabled:\\[2\baselineskip]\hspace*{0.333em} ~ ~ ~source
/software/lsstsw/stack3/loadLSST.bash\\
\hspace*{0.333em} ~ ~ ~setup lsst\_distrib

}
\begin{tabular}{p{3cm}p{13cm}}
\hline
            & Expected Result \\ \hline
\end{tabular}

\begin{tabular}{p{4cm}p{12cm}}
\toprule
Step 3
& Description \\ \hline
\end{tabular}
{\scriptsize
The ``LSST Stack Demo'' package will be downloaded onto the test system
from
\href{https://github.com/lsst/lsst_dm_stack_demo/releases/tag/14.0}{https://github.com/lsst/lsst\_dm\_stack\_demo/releases/tag/16.0}
and uncompressed.

}
\begin{tabular}{p{3cm}p{13cm}}
\hline
            & Expected Result \\ \hline
\end{tabular}

\begin{tabular}{p{4cm}p{12cm}}
\toprule
Step 4
& Description \\ \hline
\end{tabular}
{\scriptsize
The demo package will be executed by following the instructions in its
``README`` file. The string ``Ok.`` should be returned. Specifically, we
execute:\\
\hspace*{0.333em} ~ ~ ~setup obs\_sdss\\
\hspace*{0.333em} ~ ~ ~./bin/demo.sh\\
\hspace*{0.333em} ~ ~ ~python bin/compare
expected/Linux64/detected-sources.txt

}
\begin{tabular}{p{3cm}p{13cm}}
\hline
            & Expected Result \\ \hline
\end{tabular}

\begin{tabular}{p{4cm}p{12cm}}
\toprule
Step 5
& Description \\ \hline
\end{tabular}
{\scriptsize
A shell on an LSST-VC compute node will now be obtained by executing:\\
\hspace*{0.333em} ~ ~\$ srun -I --pty bash

}
\begin{tabular}{p{3cm}p{13cm}}
\hline
            & Expected Result \\ \hline
\end{tabular}

\begin{tabular}{p{4cm}p{12cm}}
\toprule
Step 6
& Description \\ \hline
\end{tabular}
{\scriptsize
The demo package will be executed on the compute node and the same
result obtained.

}
\begin{tabular}{p{3cm}p{13cm}}
\hline
            & Expected Result \\ \hline
\end{tabular}

\begin{tabular}{p{4cm}p{12cm}}
\toprule
Step 7
& Description \\ \hline
\end{tabular}
{\scriptsize
The Alert Production datasets and packages are not yet part of
lsst\_distrib and so must be installed separately. They will be
installed as follows on the GPFS
filesystem:\\[2\baselineskip]\hspace*{0.333em} ~ setup git\_lfs\\
\hspace*{0.333em} ~ git clone
https://github.com/lsst/ap\_verify\_hits2015.git\\[2\baselineskip]\hspace*{0.333em}
~ export AP\_VERIFY\_HITS2015\_DIR=\$PWD/ap\_verify\_hits2015 cd
\$AP\_VERIFY\_HITS2015\_DIR\\
\hspace*{0.333em} ~ setup -r .\\
\hspace*{0.333em} ~ cd-\\
\hspace*{0.333em} ~\\
\hspace*{0.333em} ~ setup obs\_decam\\
\hspace*{0.333em} ~ git clone
https://github.com/lsst-dm/ap\_association\\
\hspace*{0.333em} ~ cd ap\_association\\
\hspace*{0.333em} ~ setup -k -r .\\
\hspace*{0.333em} ~ scons\\
\hspace*{0.333em} ~ cd-\\
\hspace*{0.333em} ~\\
\hspace*{0.333em} ~ git clone https://github.com/lsst-dm/ap\_pipe\\
\hspace*{0.333em} ~ cd ap\_pipe\\
\hspace*{0.333em} ~ setup -k -r .\\
\hspace*{0.333em} ~ scons\\
\hspace*{0.333em} ~ cd-\\
\hspace*{0.333em} ~\\
\hspace*{0.333em} ~ git clone https://github.com/lsst-dm/ap\_verify\\
\hspace*{0.333em} ~ cd ap\_verify\\
\hspace*{0.333em} ~ setup -k -r .\\
\hspace*{0.333em} ~ scons\\
\hspace*{0.333em} ~ cd-\\[2\baselineskip]and any errors or failures
reported.

}
\begin{tabular}{p{3cm}p{13cm}}
\hline
            & Expected Result \\ \hline
\end{tabular}

\subsubsection{LVV-T18 - AG-00-05: Alert Generation Produces Required Data Products}\label{lvv-t18}

\begin{longtable}[]{llllll}
\toprule
Version & Status & Priority & Verification Type & Owner
\\\midrule
1 & Approved & Normal &
Test & Eric Bellm
\\\bottomrule
\multicolumn{6}{c}{ Open \href{https://jira.lsstcorp.org/secure/Tests.jspa\#/testCase/LVV-T18}{LVV-T18} in Jira } \\
\end{longtable}

\paragraph{Verification Elements}\mbox{}\\

\begin{itemize}
\item \href{https://jira.lsstcorp.org/browse/LVV-29}{LVV-29} - DMS-REQ-0069-V-01: Processed Visit Images

\item \href{https://jira.lsstcorp.org/browse/LVV-7}{LVV-7} - DMS-REQ-0010-V-01: Difference Exposures

\item \href{https://jira.lsstcorp.org/browse/LVV-100}{LVV-100} - DMS-REQ-0269-V-01: DIASource Catalog

\item \href{https://jira.lsstcorp.org/browse/LVV-102}{LVV-102} - DMS-REQ-0271-V-01: Max nearby galaxies associated with DIASource

\end{itemize}

\paragraph{Test Items}\mbox{}\\

This test will check that the basic data products produced by Alert
Generation are generated by execution of the science payload.\\
These products will include:

\begin{itemize}
\tightlist
\item
  Processed visit images (PVIs; DMS-REQ-0069);
\item
  Difference Exposures (DMS-REQ-0010);
\item
  DIASource catalogs (DMS-REQ-0269);
\item
  DIAObject catalogs (DMS-REQ-0271);
\end{itemize}


\paragraph{Predecessors}\mbox{}\\
LVV-T17 (AG-00-00)

\paragraph{Environment Needs}\mbox{}\\

\subparagraph{Software}\mbox{}\\
Release 16.0 of the DM Software Stack will be pre-installed (following
the procedure described in AG-00-00).

\subparagraph{Hardware}\mbox{}\\
The test shall be carried out on a machine with at least 16 GB of RAM
and multiple CPU cores which has access to the /datasets shared (GPFS)
filesystem at the LSST Data Facility.

\paragraph{Input Specification}\mbox{}\\
A complete processing of the DECam ``HiTS'' dataset, as defined at
https://dmtn-039.lsst.io/ and
https://github.com/lsst/ap\_verify\_hits2015, through the Alert
Generation science payload.\\
This dataset shall be made available in a standard LSST data repository,
accessible via the ``Data Butler''.\\
It is not required that all combinations of visit and CCD have been
processed successfully: a number of failures are expected. However,
documentation to describe processing failures should be provided.

\paragraph{Output Specification}\mbox{}\\
None.

\paragraph{Test Procedure}\mbox{}\\
\begin{tabular}{p{4cm}p{12cm}}
\toprule
Step 1
& Description \\ \hline
\end{tabular}
{\scriptsize
The DM Stack and Alert Processing packaged shall be initialized as
described in LVT-T17 (AG-00-00).

}
\begin{tabular}{p{3cm}p{13cm}}
\hline
            & Expected Result \\ \hline
\end{tabular}

\begin{tabular}{p{4cm}p{12cm}}
\toprule
Step 2
& Description \\ \hline
\end{tabular}
{\scriptsize
The alert generation processing will be executed using the verification
cluster:\\[2\baselineskip]```bash\\
python ap\_verify/bin/prepare\_demo\_slurm\_files.py\\
\# At present we must run a single ccd+visit to handle ingestion
before\\
\# parallel processing can begin\\
./ap\_verify/bin/exec\_demo\_run\_1ccd.sh 410915 25\\
ln -s ap\_verify/bin/demo\_run.sl\\
ln -s ap\_verify/bin/demo\_cmds.conf\\
sbatch demo\_run.sl\\
```\\[2\baselineskip]and any errors or failures reported.

}
\begin{tabular}{p{3cm}p{13cm}}
\hline
            & Expected Result \\ \hline
\end{tabular}

\begin{tabular}{p{4cm}p{12cm}}
\toprule
Step 3
& Description \\ \hline
\end{tabular}
{\scriptsize
A ``Data Butler'' will be initialized to access the repository.

}
\begin{tabular}{p{3cm}p{13cm}}
\hline
            & Expected Result \\ \hline
\end{tabular}

\begin{tabular}{p{4cm}p{12cm}}
\toprule
Step 4
& Description \\ \hline
\end{tabular}
{\scriptsize
For each of the expected data products types (listed in ยง4.2.2) and
each of the expected units (PVIs, catalogs, etc.), the data product will
be retrieved from the Butler and verified to be non-empty.

}
\begin{tabular}{p{3cm}p{13cm}}
\hline
            & Expected Result \\ \hline
\end{tabular}

\begin{tabular}{p{4cm}p{12cm}}
\toprule
Step 5
& Description \\ \hline
\end{tabular}
{\scriptsize
DIAObjects are currently only stored in a database, without shims to the
Butler, so the existence of the database table and its non-empty
contents will be verified by directly accessing it using sqlite3 and
executing appropriate SQL queries.

}
\begin{tabular}{p{3cm}p{13cm}}
\hline
            & Expected Result \\ \hline
\end{tabular}

\subsubsection{LVV-T19 - AG-00-10: Scientific Verification of Processed Visit Images}\label{lvv-t19}

\begin{longtable}[]{llllll}
\toprule
Version & Status & Priority & Verification Type & Owner
\\\midrule
1 & Approved & Normal &
Test & Eric Bellm
\\\bottomrule
\multicolumn{6}{c}{ Open \href{https://jira.lsstcorp.org/secure/Tests.jspa\#/testCase/LVV-T19}{LVV-T19} in Jira } \\
\end{longtable}

\paragraph{Verification Elements}\mbox{}\\

\begin{itemize}
\item \href{https://jira.lsstcorp.org/browse/LVV-29}{LVV-29} - DMS-REQ-0069-V-01: Processed Visit Images

\item \href{https://jira.lsstcorp.org/browse/LVV-158}{LVV-158} - DMS-REQ-0327-V-01: Background Model Calculation

\item \href{https://jira.lsstcorp.org/browse/LVV-12}{LVV-12} - DMS-REQ-0029-V-01: Generate Photometric Zeropoint for Visit Image

\item \href{https://jira.lsstcorp.org/browse/LVV-30}{LVV-30} - DMS-REQ-0070-V-01: Generate PSF for Visit Images

\item \href{https://jira.lsstcorp.org/browse/LVV-13}{LVV-13} - DMS-REQ-0030-V-01: Absolute accuracy of WCS

\item \href{https://jira.lsstcorp.org/browse/LVV-31}{LVV-31} - DMS-REQ-0072-V-01: Processed Visit Image Content

\end{itemize}

\paragraph{Test Items}\mbox{}\\

This test will check that the Processed Visit Images (PVIs) delivered by
the alert generation science payload meet the requirements laid down by
\citeds{LSE-61}.\\
Specifically, this will demonstrate that:

\begin{itemize}
\tightlist
\item
  Processed visit images have been generated and persisted during
  payload execution;
\item
  Each PVI includes a science pixel array, a mask array, and a variance
  array. (DMS-REQ-0072).
\item
  Each PVI includes a background model (DMS-REQ-0327), photometric
  zero-point (DMS- REQ-0029), spatially-varying PSF (DMS-REQ-0070) and
  WCS (DMS-REQ-0030).
\item
  Saturated pixels are correctly masked.
\item
  Pixels affected by cosmic rays are correctly masked.
\item
  The background is not oversubtracted around bright objects.
\end{itemize}

This test does not include quantitative targets for the science quality
criteria.


\paragraph{Predecessors}\mbox{}\\
LVT-T17 (AG-00-00)\\
LVT-T18 (AG-00-05)

\paragraph{Environment Needs}\mbox{}\\

\subparagraph{Software}\mbox{}\\
Release 14.0 of the DM Software Stack will be pre-installed (following
the procedure described in AG-00-00).

\subparagraph{Hardware}\mbox{}\\
The test shall be carried out on a machine with at least 16 GB of RAM
and multiple CPU cores which has access to the /datasets shared (GPFS)
filesystem at the LSST Data Facility.

\paragraph{Input Specification}\mbox{}\\
A complete processing of the DECam ``HiTS'' dataset, as defined at
https://dmtn-039.lsst.io/ and
https://github.com/lsst/ap\_verify\_hits2015, through the Alert
Generation science payload.\\
This dataset shall be made available in a standard LSST data repository,
accessible via the ``Data Butler''.\\
It is not required that all combinations of visit and CCD have been
processed successfully: a number of failures are expected. However,
documentation to describe processing failures should be provided.

\paragraph{Output Specification}\mbox{}\\
None.

\paragraph{Test Procedure}\mbox{}\\
\begin{tabular}{p{4cm}p{12cm}}
\toprule
Step 1
& Description \\ \hline
\end{tabular}
{\scriptsize
The DM Stack shall be initialized using the loadLSST script (as
described in LVV-T17 - AG-00-00).

}
\begin{tabular}{p{3cm}p{13cm}}
\hline
            & Expected Result \\ \hline
\end{tabular}

\begin{tabular}{p{4cm}p{12cm}}
\toprule
Step 2
& Description \\ \hline
\end{tabular}
{\scriptsize
A ``Data Butler'' will be initialized to access the repository.

}
\begin{tabular}{p{3cm}p{13cm}}
\hline
            & Expected Result \\ \hline
\end{tabular}

\begin{tabular}{p{4cm}p{12cm}}
\toprule
Step 3
& Description \\ \hline
\end{tabular}
{\scriptsize
For each processed CCD, the PVI will be retrieved from the Butler, and
the existence of all components described in ยง4.3.2 will be verified.

}
\begin{tabular}{p{3cm}p{13cm}}
\hline
            & Expected Result \\ \hline
\end{tabular}

\begin{tabular}{p{4cm}p{12cm}}
\toprule
Step 4
& Description \\ \hline
\end{tabular}
{\scriptsize
Five sensors will be chosen at random from each of two visits and
inspected by eye for unmasked artifacts.

}
\begin{tabular}{p{3cm}p{13cm}}
\hline
            & Expected Result \\ \hline
\end{tabular}

\subsubsection{LVV-T20 - AG-00-15: Scientific Verification of Difference Images}\label{lvv-t20}

\begin{longtable}[]{llllll}
\toprule
Version & Status & Priority & Verification Type & Owner
\\\midrule
1 & Approved & Normal &
Test & Eric Bellm
\\\bottomrule
\multicolumn{6}{c}{ Open \href{https://jira.lsstcorp.org/secure/Tests.jspa\#/testCase/LVV-T20}{LVV-T20} in Jira } \\
\end{longtable}

\paragraph{Verification Elements}\mbox{}\\

\begin{itemize}
\item \href{https://jira.lsstcorp.org/browse/LVV-7}{LVV-7} - DMS-REQ-0010-V-01: Difference Exposures

\item \href{https://jira.lsstcorp.org/browse/LVV-32}{LVV-32} - DMS-REQ-0074-V-01: Difference Exposure Attributes

\end{itemize}

\paragraph{Test Items}\mbox{}\\

This test will check that the difference images delivered by the Alert
Generation science pay- load meet the requirements laid down by
\citeds{LSE-61}.\\
Specifically, this will demonstrate that:

\begin{itemize}
\tightlist
\item
  Difference images have been generated and persisted during payload
  execution;
\item
  Each difference image includes information about the identity of the
  input exposures, and metadata such as a representation of the PSF
  matching kernel (DMS-REQ-0074);
\item
  Masks are correctly propagated from the input images.
\end{itemize}

This test does not include quantitative targets for the science quality
criteria.


\paragraph{Predecessors}\mbox{}\\
LVV-T17 (AG-00-00)\\
LVV-T18 (AG-00-05)

\paragraph{Environment Needs}\mbox{}\\

\subparagraph{Software}\mbox{}\\
Release 14.0 of the DM Software Stack will be pre-installed (following
the procedure described in AG-00-00).

\subparagraph{Hardware}\mbox{}\\
The test shall be carried out on a machine with at least 16 GB of RAM
and multiple CPU cores which has access to the /datasets shared (GPFS)
filesystem at the LSST Data Facility.

\paragraph{Input Specification}\mbox{}\\
A complete processing of the DECam ``HiTS'' dataset, as defined at
https://dmtn-039.lsst. io/ and
https://github.com/lsst/ap\_verify\_hits2015, through the Alert
Generation science payload.\\
This dataset shall be made available in a standard LSST data repository,
accessible via the ``Data Butler''.\\
It is not required that all combinations of visit and CCD have been
processed successfully: a number of failures are expected. However,
documentation to describe processing failures should be provided.

\paragraph{Output Specification}\mbox{}\\
None.

\paragraph{Test Procedure}\mbox{}\\
\begin{tabular}{p{4cm}p{12cm}}
\toprule
Step 1
& Description \\ \hline
\end{tabular}
{\scriptsize
The DM Stack shall be initialized using the loadLSST script (as
described in LVV-T-17 AG-00-00).

}
\begin{tabular}{p{3cm}p{13cm}}
\hline
            & Expected Result \\ \hline
\end{tabular}

\begin{tabular}{p{4cm}p{12cm}}
\toprule
Step 2
& Description \\ \hline
\end{tabular}
{\scriptsize
A ``Data Butler'' will be initialized to access the repository.

}
\begin{tabular}{p{3cm}p{13cm}}
\hline
            & Expected Result \\ \hline
\end{tabular}

\begin{tabular}{p{4cm}p{12cm}}
\toprule
Step 3
& Description \\ \hline
\end{tabular}
{\scriptsize
For each processed CCD, the difference image will be retrieved from the
Butler, and the existence of all components described in ยง4.4.2 will be
verified.

}
\begin{tabular}{p{3cm}p{13cm}}
\hline
            & Expected Result \\ \hline
\end{tabular}

\begin{tabular}{p{4cm}p{12cm}}
\toprule
Step 4
& Description \\ \hline
\end{tabular}
{\scriptsize
Five sensors will be chosen at random from each of two visits and the
masks of the input and difference images compared by eye.

}
\begin{tabular}{p{3cm}p{13cm}}
\hline
            & Expected Result \\ \hline
\end{tabular}

\subsubsection{LVV-T21 - AG-00-20: Scientific Verification of DIASource Catalog}\label{lvv-t21}

\begin{longtable}[]{llllll}
\toprule
Version & Status & Priority & Verification Type & Owner
\\\midrule
1 & Approved & Normal &
Test & Eric Bellm
\\\bottomrule
\multicolumn{6}{c}{ Open \href{https://jira.lsstcorp.org/secure/Tests.jspa\#/testCase/LVV-T21}{LVV-T21} in Jira } \\
\end{longtable}

\paragraph{Verification Elements}\mbox{}\\

\begin{itemize}
\item \href{https://jira.lsstcorp.org/browse/LVV-100}{LVV-100} - DMS-REQ-0269-V-01: DIASource Catalog

\item \href{https://jira.lsstcorp.org/browse/LVV-101}{LVV-101} - DMS-REQ-0270-V-01: Faint DIASource Measurements

\item \href{https://jira.lsstcorp.org/browse/LVV-178}{LVV-178} - DMS-REQ-0347-V-01: Measurements in catalogs

\item \href{https://jira.lsstcorp.org/browse/LVV-162}{LVV-162} - DMS-REQ-0331-V-01: Computing Derived Quantities

\item \href{https://jira.lsstcorp.org/browse/LVV-18}{LVV-18} - DMS-REQ-0043-V-01: Provide Calibrated Photometry

\end{itemize}

\paragraph{Test Items}\mbox{}\\

This test will check that the difference image source catalogs delivered
by the Alert Generation science payload meet the requirements laid down
by \citeds{LSE-61}.

\begin{itemize}
\tightlist
\item
  Specifically, this will demonstrate that:
\item
  Measurements in the catalog are presented in flux units
  (DMS-REQ-0347);
\item
  Each DIASource record contains an appropriate subset of the attributes
  required by DMS-REQ-0269. In particular, the LDM-503-3-era pipeline is
  expected to provide DIASource positions (sky and focal plane), fluxes,
  and flags indicative of issues encountered during processing.
\item
  Faint DIASources satisfying additional criteria are stored
  (DMS-REQ-0270).
\item
  Derived quantities are provided in pre-computed columns
  (DMS-REQ-0331);
\end{itemize}

This test does not include quantitative targets for the science quality
criteria.\\[2\baselineskip]


\paragraph{Predecessors}\mbox{}\\
LVT-T17 (AG-00-00)\\
LVT-T18 (AG-00-05)

\paragraph{Environment Needs}\mbox{}\\

\subparagraph{Software}\mbox{}\\
Release 14.0 of the DM Software Stack will be pre-installed (following
the procedure described in AG-00-00).

\subparagraph{Hardware}\mbox{}\\
The test shall be carried out on a machine with at least 16 GB of RAM
and multiple CPU cores which has access to the /datasets shared (GPFS)
filesystem at the LSST Data Facility.

\paragraph{Input Specification}\mbox{}\\
A complete processing of the DECam ``HiTS'' dataset, as defined at
https://dmtn-039.lsst. io/ and
https://github.com/lsst/ap\_verify\_hits2015, through the Alert
Generation science payload.\\
This dataset shall be made available in a standard LSST data repository,
accessible via the ``Data Butler''.\\
It is not required that all combinations of visit and CCD have been
processed successfully: a number of failures are expected. However,
documentation to describe processing failures should be provided.

\paragraph{Output Specification}\mbox{}\\
None.

\paragraph{Test Procedure}\mbox{}\\
\begin{tabular}{p{4cm}p{12cm}}
\toprule
Step 1
& Description \\ \hline
\end{tabular}
{\scriptsize
The DM Stack shall be initialized using the loadLSST script (as
described in LVV-T17 - AG-00-00).

}
\begin{tabular}{p{3cm}p{13cm}}
\hline
            & Expected Result \\ \hline
\end{tabular}

\begin{tabular}{p{4cm}p{12cm}}
\toprule
Step 2
& Description \\ \hline
\end{tabular}
{\scriptsize
A ``Data Butler'' will be initialized to access the repository.

}
\begin{tabular}{p{3cm}p{13cm}}
\hline
            & Expected Result \\ \hline
\end{tabular}

\begin{tabular}{p{4cm}p{12cm}}
\toprule
Step 3
& Description \\ \hline
\end{tabular}
{\scriptsize
DIASource records will be accessed by querying the Butler, then examined
interactively at a Python prompt.

}
\begin{tabular}{p{3cm}p{13cm}}
\hline
            & Expected Result \\ \hline
\end{tabular}

\subsubsection{LVV-T22 - AG-00-25: Scientific Verification of DIAObject Catalog}\label{lvv-t22}

\begin{longtable}[]{llllll}
\toprule
Version & Status & Priority & Verification Type & Owner
\\\midrule
1 & Approved & Normal &
Test & Eric Bellm
\\\bottomrule
\multicolumn{6}{c}{ Open \href{https://jira.lsstcorp.org/secure/Tests.jspa\#/testCase/LVV-T22}{LVV-T22} in Jira } \\
\end{longtable}

\paragraph{Verification Elements}\mbox{}\\

\begin{itemize}
\item \href{https://jira.lsstcorp.org/browse/LVV-116}{LVV-116} - DMS-REQ-0285-V-01: Level 1 Source Association

\item \href{https://jira.lsstcorp.org/browse/LVV-102}{LVV-102} - DMS-REQ-0271-V-01: Max nearby galaxies associated with DIASource

\item \href{https://jira.lsstcorp.org/browse/LVV-103}{LVV-103} - DMS-REQ-0272-V-01: DIAObject Attributes

\item \href{https://jira.lsstcorp.org/browse/LVV-178}{LVV-178} - DMS-REQ-0347-V-01: Measurements in catalogs

\item \href{https://jira.lsstcorp.org/browse/LVV-162}{LVV-162} - DMS-REQ-0331-V-01: Computing Derived Quantities

\item \href{https://jira.lsstcorp.org/browse/LVV-18}{LVV-18} - DMS-REQ-0043-V-01: Provide Calibrated Photometry

\end{itemize}

\paragraph{Test Items}\mbox{}\\

This test will check that the DIAObject catalogs delivered by the Alert
Generation science pay- load meet the requirements laid down by
\citeds{LSE-61}.\\
Specifically, this will demonstrate that:

\begin{itemize}
\tightlist
\item
  DIAObjects are recorded with unique identifiers (DMS-REQ-0271);
\item
  Measurements in the catalog are presented in flux units
  (DMS-REQ-0347);
\item
  EachDIAObjectrecordcontainscontainsanappropriatesetofsummaryattributes(DMS-
  REQ-0271 and DMS-REQ-0272). Note:

  \begin{itemize}
  \tightlist
  \item
    This test is executed independently of the Data Release Production
    system. Hence, DIAObjects are not associated to Objects, and the
    association metadata specified by DMS-REQ-0271 is not expected to be
    available.
  \item
    TheLDM-503-3erapipelineisnotexpectedtocalculateorpersistallattributesspec-
    ified by DMS-REQ-0272 requirement.
  \end{itemize}
\item
  Relevant derived quantities are provided in pre-computed columns
  (DMS-REQ-0331);~
\end{itemize}

This test does not include quantitative targets for the science quality
criteria.


\paragraph{Predecessors}\mbox{}\\
LVT-T17 (AG-00-00)\\
LVT-T18 (AG-00-05)

\paragraph{Environment Needs}\mbox{}\\

\subparagraph{Software}\mbox{}\\
Release 14.0 of the DM Software Stack will be pre-installed (following
the procedure described in AG-00-00).

\subparagraph{Hardware}\mbox{}\\
The test shall be carried out on a machine with at least 16 GB of RAM
and multiple CPU cores which has access to the /datasets shared (GPFS)
filesystem at the LSST Data Facility.

\paragraph{Input Specification}\mbox{}\\
A complete processing of the DECam ``HiTS'' dataset, as defined at
https://dmtn-039.lsst. io/ and
https://github.com/lsst/ap\_verify\_hits2015, through the Alert
Generation science payload.\\
This dataset shall be made available in a standard LSST data repository,
accessible via the ``Data Butler''.\\
It is not required that all combinations of visit and CCD have been
processed successfully: a number of failures are expected. However,
documentation to describe processing failures should be provided.

\paragraph{Output Specification}\mbox{}\\
None.

\paragraph{Test Procedure}\mbox{}\\
\begin{tabular}{p{4cm}p{12cm}}
\toprule
Step 1
& Description \\ \hline
\end{tabular}
{\scriptsize
The DM Stack shall be initialized using the loadLSST script (as
described in LVV-T17 - AG-00-00).

}
\begin{tabular}{p{3cm}p{13cm}}
\hline
            & Expected Result \\ \hline
\end{tabular}

\begin{tabular}{p{4cm}p{12cm}}
\toprule
Step 2
& Description \\ \hline
\end{tabular}
{\scriptsize
sqlite3 or Pythonâ\euro{}™s sqlalchemy module will be used to access the
Level 1 database.

}
\begin{tabular}{p{3cm}p{13cm}}
\hline
            & Expected Result \\ \hline
\end{tabular}

\subsubsection{LVV-T28 - Verify implementation of measurements in catalogs from PVIs}\label{lvv-t28}

\begin{longtable}[]{llllll}
\toprule
Version & Status & Priority & Verification Type & Owner
\\\midrule
1 & Approved & Normal &
Test & Colin Slater
\\\bottomrule
\multicolumn{6}{c}{ Open \href{https://jira.lsstcorp.org/secure/Tests.jspa\#/testCase/LVV-T28}{LVV-T28} in Jira } \\
\end{longtable}

\paragraph{Verification Elements}\mbox{}\\

\begin{itemize}
\item \href{https://jira.lsstcorp.org/browse/LVV-178}{LVV-178} - DMS-REQ-0347-V-01: Measurements in catalogs

\end{itemize}

\paragraph{Test Items}\mbox{}\\

Verify that source measurements in catalogs containing measurements from
processed visit images are in flux units.








\paragraph{Test Procedure}\mbox{}\\
\begin{tabular}{p{4cm}p{12cm}}
\toprule
Step 1-1
{\scriptsize from \hyperref[lvv-t987]{LVV-T987} }
& Description \\ \hline
\end{tabular}
{\scriptsize
Identify the path to the data repository, which we will refer to as
`DATA/path', then execute the following:

}
\begin{tabular}{p{3cm}p{13cm}}
\hline
            & Example Code \\ \hline
\end{tabular}
{\scriptsize
\begin{verbatim}
import lsst.daf.persistence as dafPersist
butler = dafPersist.Butler(inputs='DATA/path')
\end{verbatim}

}
\begin{tabular}{p{3cm}p{13cm}}
\hline
            & Expected Result \\ \hline
\end{tabular}
{\scriptsize
Butler repo available for reading.

}

\begin{tabular}{p{4cm}p{12cm}}
\toprule
Step 2
& Description \\ \hline
\end{tabular}
{\scriptsize
Identify and read an appropriate processed precursor dataset containing
coadds with the Butler.~

}
\begin{tabular}{p{3cm}p{13cm}}
\hline
            & Expected Result \\ \hline
\end{tabular}

\begin{tabular}{p{4cm}p{12cm}}
\toprule
Step 3
& Description \\ \hline
\end{tabular}
{\scriptsize
Verify that the single-visit catalog provides measurements in flux
units.

}
\begin{tabular}{p{3cm}p{13cm}}
\hline
            & Expected Result \\ \hline
\end{tabular}
{\scriptsize
Confirmation of measurements in catalogs encoded in flux units.

}

\subsubsection{LVV-T39 - Verify implementation of Generate Photometric Zeropoint for Visit Image}\label{lvv-t39}

\begin{longtable}[]{llllll}
\toprule
Version & Status & Priority & Verification Type & Owner
\\\midrule
1 & Approved & Normal &
Test & Jim Bosch
\\\bottomrule
\multicolumn{6}{c}{ Open \href{https://jira.lsstcorp.org/secure/Tests.jspa\#/testCase/LVV-T39}{LVV-T39} in Jira } \\
\end{longtable}

\paragraph{Verification Elements}\mbox{}\\

\begin{itemize}
\item \href{https://jira.lsstcorp.org/browse/LVV-12}{LVV-12} - DMS-REQ-0029-V-01: Generate Photometric Zeropoint for Visit Image

\end{itemize}

\paragraph{Test Items}\mbox{}\\

Verify that Processed Visit Image data products produced by the DRP and
AP pipelines include the parameters of a model that relates the observed
flux on the image to physical flux units.








\paragraph{Test Procedure}\mbox{}\\
\begin{tabular}{p{4cm}p{12cm}}
\toprule
Step 1
& Description \\ \hline
\end{tabular}
{\scriptsize
Identify a dataset with processed visit images in multiple filters.

}
\begin{tabular}{p{3cm}p{13cm}}
\hline
            & Expected Result \\ \hline
\end{tabular}

\begin{tabular}{p{4cm}p{12cm}}
\toprule
Step 2-1
{\scriptsize from \hyperref[lvv-t987]{LVV-T987} }
& Description \\ \hline
\end{tabular}
{\scriptsize
Identify the path to the data repository, which we will refer to as
`DATA/path', then execute the following:

}
\begin{tabular}{p{3cm}p{13cm}}
\hline
            & Example Code \\ \hline
\end{tabular}
{\scriptsize
\begin{verbatim}
import lsst.daf.persistence as dafPersist
butler = dafPersist.Butler(inputs='DATA/path')
\end{verbatim}

}
\begin{tabular}{p{3cm}p{13cm}}
\hline
            & Expected Result \\ \hline
\end{tabular}
{\scriptsize
Butler repo available for reading.

}

\begin{tabular}{p{4cm}p{12cm}}
\toprule
Step 3
& Description \\ \hline
\end{tabular}
{\scriptsize
Extract the photometric zeropoint from the source catalog associated
with a visit image. Repeat this for all available filters, and confirm
that the zeropoint has been set, and has a reasonable value.

}
\begin{tabular}{p{3cm}p{13cm}}
\hline
            & Expected Result \\ \hline
\end{tabular}
{\scriptsize
A zeropoint that enables one to convert the measured fluxes to
magnitudes.

}

\begin{tabular}{p{4cm}p{12cm}}
\toprule
Step 4
& Description \\ \hline
\end{tabular}
{\scriptsize
Extract fluxes for some sources, and convert them to magnitudes. Confirm
that the distribution spans a reasonable range.

}
\begin{tabular}{p{3cm}p{13cm}}
\hline
            & Expected Result \\ \hline
\end{tabular}
{\scriptsize
In most cases, well-measured magnitudes (i.e., for high S/N
measurements) should be between 12 to 28 for all bands.

}

\subsubsection{LVV-T40 - Verify implementation of Generate WCS for Visit Images}\label{lvv-t40}

\begin{longtable}[]{llllll}
\toprule
Version & Status & Priority & Verification Type & Owner
\\\midrule
1 & Approved & Normal &
Test & Jim Bosch
\\\bottomrule
\multicolumn{6}{c}{ Open \href{https://jira.lsstcorp.org/secure/Tests.jspa\#/testCase/LVV-T40}{LVV-T40} in Jira } \\
\end{longtable}

\paragraph{Verification Elements}\mbox{}\\

\begin{itemize}
\item \href{https://jira.lsstcorp.org/browse/LVV-13}{LVV-13} - DMS-REQ-0030-V-01: Absolute accuracy of WCS

\end{itemize}

\paragraph{Test Items}\mbox{}\\

Verify that Processed Visit Images produced by the AP and DRP pipelines
include FITS WCS accurate to specified \textbf{astrometricAccuracy} over
the bounds of the image.








\paragraph{Test Procedure}\mbox{}\\
\begin{tabular}{p{4cm}p{12cm}}
\toprule
Step 1
& Description \\ \hline
\end{tabular}
{\scriptsize
Identify an appropriate processed dataset for this test.

}
\begin{tabular}{p{3cm}p{13cm}}
\hline
            & Expected Result \\ \hline
\end{tabular}
{\scriptsize
A dataset with Processed Visit Images available.

}

\begin{tabular}{p{4cm}p{12cm}}
\toprule
Step 2-1
{\scriptsize from \hyperref[lvv-t987]{LVV-T987} }
& Description \\ \hline
\end{tabular}
{\scriptsize
Identify the path to the data repository, which we will refer to as
`DATA/path', then execute the following:

}
\begin{tabular}{p{3cm}p{13cm}}
\hline
            & Example Code \\ \hline
\end{tabular}
{\scriptsize
\begin{verbatim}
import lsst.daf.persistence as dafPersist
butler = dafPersist.Butler(inputs='DATA/path')
\end{verbatim}

}
\begin{tabular}{p{3cm}p{13cm}}
\hline
            & Expected Result \\ \hline
\end{tabular}
{\scriptsize
Butler repo available for reading.

}

\begin{tabular}{p{4cm}p{12cm}}
\toprule
Step 3
& Description \\ \hline
\end{tabular}
{\scriptsize
Select a single visit from the dataset, and extract its WCS object and
the source list.

}
\begin{tabular}{p{3cm}p{13cm}}
\hline
            & Expected Result \\ \hline
\end{tabular}
{\scriptsize
A table containing detected sources, and a WCS object associated with
that catalog.

}

\begin{tabular}{p{4cm}p{12cm}}
\toprule
Step 4
& Description \\ \hline
\end{tabular}
{\scriptsize
Confirm that each CCD within the visit image contains at
least~\textbf{astrometricMinStandards~}astrometric standards that were
used in deriving the astrometric solution.

}
\begin{tabular}{p{3cm}p{13cm}}
\hline
            & Expected Result \\ \hline
\end{tabular}
{\scriptsize
At least \textbf{astrometricMinStandards} from each CCD\textbf{~}were
used in determining the WCS solution.

}

\begin{tabular}{p{4cm}p{12cm}}
\toprule
Step 5
& Description \\ \hline
\end{tabular}
{\scriptsize
Starting from the XY pixel coordinates of the sources, apply the WCS to
obtain RA, Dec coordinates.\\[2\baselineskip]

}
\begin{tabular}{p{3cm}p{13cm}}
\hline
            & Expected Result \\ \hline
\end{tabular}
{\scriptsize
A list of RA, Dec coordinates for all sources in the catalog.

}

\begin{tabular}{p{4cm}p{12cm}}
\toprule
Step 6
& Description \\ \hline
\end{tabular}
{\scriptsize
We will assume that Gaia provides a source of ``truth.'' Match the
source list to Gaia DR2, and calculate the positional offset between the
test data and the Gaia catalog.

}
\begin{tabular}{p{3cm}p{13cm}}
\hline
            & Expected Result \\ \hline
\end{tabular}
{\scriptsize
A matched catalog of sources in common between the test source list and
Gaia DR2.

}

\begin{tabular}{p{4cm}p{12cm}}
\toprule
Step 7
& Description \\ \hline
\end{tabular}
{\scriptsize
Apply appropriate cuts to extract the optimal dataset for comparison,
then calculate statistics (median, 1-sigma range, etc.; also plot a
histogram) of the offsets in milliarcseconds. Confirm that the offset is
less than \textbf{astrometricAccuracy}.

}
\begin{tabular}{p{3cm}p{13cm}}
\hline
            & Expected Result \\ \hline
\end{tabular}
{\scriptsize
Histogram and relevant statistics needed to confirm that the WCS
transformation is accurate.

}

\begin{tabular}{p{4cm}p{12cm}}
\toprule
Step 8
& Description \\ \hline
\end{tabular}
{\scriptsize
Repeat Step 5, but for subregions of the image, to confirm that the
accuracy criterion is met at all positions.

}
\begin{tabular}{p{3cm}p{13cm}}
\hline
            & Expected Result \\ \hline
\end{tabular}
{\scriptsize
\textbf{astrometricAccuracy~}requirement is met over the entire image.

}

\subsubsection{LVV-T41 - Verify implementation of Generate PSF for Visit Images}\label{lvv-t41}

\begin{longtable}[]{llllll}
\toprule
Version & Status & Priority & Verification Type & Owner
\\\midrule
1 & Approved & Normal &
Test & Jim Bosch
\\\bottomrule
\multicolumn{6}{c}{ Open \href{https://jira.lsstcorp.org/secure/Tests.jspa\#/testCase/LVV-T41}{LVV-T41} in Jira } \\
\end{longtable}

\paragraph{Verification Elements}\mbox{}\\

\begin{itemize}
\item \href{https://jira.lsstcorp.org/browse/LVV-30}{LVV-30} - DMS-REQ-0070-V-01: Generate PSF for Visit Images

\end{itemize}

\paragraph{Test Items}\mbox{}\\

Verify that Processed Visit Images produced by the DRP and AP pipelines
are associated with a model from which one can obtain an image of the
PSF given a point on the image.








\paragraph{Test Procedure}\mbox{}\\
\begin{tabular}{p{4cm}p{12cm}}
\toprule
Step 1
& Description \\ \hline
\end{tabular}
{\scriptsize
Identify a dataset with processed visit images in multiple filters.

}
\begin{tabular}{p{3cm}p{13cm}}
\hline
            & Expected Result \\ \hline
\end{tabular}

\begin{tabular}{p{4cm}p{12cm}}
\toprule
Step 2-1
{\scriptsize from \hyperref[lvv-t987]{LVV-T987} }
& Description \\ \hline
\end{tabular}
{\scriptsize
Identify the path to the data repository, which we will refer to as
`DATA/path', then execute the following:

}
\begin{tabular}{p{3cm}p{13cm}}
\hline
            & Example Code \\ \hline
\end{tabular}
{\scriptsize
\begin{verbatim}
import lsst.daf.persistence as dafPersist
butler = dafPersist.Butler(inputs='DATA/path')
\end{verbatim}

}
\begin{tabular}{p{3cm}p{13cm}}
\hline
            & Expected Result \\ \hline
\end{tabular}
{\scriptsize
Butler repo available for reading.

}

\begin{tabular}{p{4cm}p{12cm}}
\toprule
Step 3
& Description \\ \hline
\end{tabular}
{\scriptsize
Select Objects classified as point sources on at least 10 different
processed visit images (including all bands). ~Evaluate the PSF model at
the positions of these Objects, and verify that subtracting a scaled
version of the PSF model from the processed visit image yields residuals
consistent with pure noise.

}
\begin{tabular}{p{3cm}p{13cm}}
\hline
            & Expected Result \\ \hline
\end{tabular}
{\scriptsize
Images with the PSF model subtracted, leaving only residuals that are
consistent with being noise.

}

\subsubsection{LVV-T43 - Verify implementation of Background Model Calculation}\label{lvv-t43}

\begin{longtable}[]{llllll}
\toprule
Version & Status & Priority & Verification Type & Owner
\\\midrule
1 & Approved & Normal &
Test & Jim Bosch
\\\bottomrule
\multicolumn{6}{c}{ Open \href{https://jira.lsstcorp.org/secure/Tests.jspa\#/testCase/LVV-T43}{LVV-T43} in Jira } \\
\end{longtable}

\paragraph{Verification Elements}\mbox{}\\

\begin{itemize}
\item \href{https://jira.lsstcorp.org/browse/LVV-158}{LVV-158} - DMS-REQ-0327-V-01: Background Model Calculation

\end{itemize}

\paragraph{Test Items}\mbox{}\\

Verify that Processed Visit Images produced by the DRP and AP pipelines
have had a model of the background subtracted, and that this model is
persisted in a way that permits the background subtracted from any CCD
to be retrieved along with the image for that CCD.


\paragraph{Predecessors}\mbox{}\\
\href{https://jira.lsstcorp.org/secure/Tests.jspa\#/testCase/127}{LVV-T15}\\
\href{https://jira.lsstcorp.org/secure/Tests.jspa\#/testCase/131}{LVV-T19}






\paragraph{Test Procedure}\mbox{}\\
\begin{tabular}{p{4cm}p{12cm}}
\toprule
Step 1
& Description \\ \hline
\end{tabular}
{\scriptsize
Identify a dataset with processed visit images in multiple filters.

}
\begin{tabular}{p{3cm}p{13cm}}
\hline
            & Expected Result \\ \hline
\end{tabular}

\begin{tabular}{p{4cm}p{12cm}}
\toprule
Step 2-1
{\scriptsize from \hyperref[lvv-t987]{LVV-T987} }
& Description \\ \hline
\end{tabular}
{\scriptsize
Identify the path to the data repository, which we will refer to as
`DATA/path', then execute the following:

}
\begin{tabular}{p{3cm}p{13cm}}
\hline
            & Example Code \\ \hline
\end{tabular}
{\scriptsize
\begin{verbatim}
import lsst.daf.persistence as dafPersist
butler = dafPersist.Butler(inputs='DATA/path')
\end{verbatim}

}
\begin{tabular}{p{3cm}p{13cm}}
\hline
            & Expected Result \\ \hline
\end{tabular}
{\scriptsize
Butler repo available for reading.

}

\begin{tabular}{p{4cm}p{12cm}}
\toprule
Step 3
& Description \\ \hline
\end{tabular}
{\scriptsize
Display an image of the background model for a full CCD. Repeat this for
all available filters, and confirm that the background is smoothly
varying and defined over the full CCD.

}
\begin{tabular}{p{3cm}p{13cm}}
\hline
            & Expected Result \\ \hline
\end{tabular}
{\scriptsize
Well-formed background covering the entire CCD for all CCDs in all
filters.

}

\subsubsection{LVV-T125 - Verify implementation of Simulated Data}\label{lvv-t125}

\begin{longtable}[]{llllll}
\toprule
Version & Status & Priority & Verification Type & Owner
\\\midrule
1 & Approved & Normal &
Test & Robert Lupton
\\\bottomrule
\multicolumn{6}{c}{ Open \href{https://jira.lsstcorp.org/secure/Tests.jspa\#/testCase/LVV-T125}{LVV-T125} in Jira } \\
\end{longtable}

\paragraph{Verification Elements}\mbox{}\\

\begin{itemize}
\item \href{https://jira.lsstcorp.org/browse/LVV-6}{LVV-6} - DMS-REQ-0009-V-01: Simulated Data

\end{itemize}

\paragraph{Test Items}\mbox{}\\

Verify that the DMS can inject simulated data into data products for
testing.








\paragraph{Test Procedure}\mbox{}\\
\begin{tabular}{p{4cm}p{12cm}}
\toprule
Step 1
& Description \\ \hline
\end{tabular}
{\scriptsize
Identify a dataset that has been (or can be readily) processed through
single-frame processing and coaddition.

}
\begin{tabular}{p{3cm}p{13cm}}
\hline
            & Expected Result \\ \hline
\end{tabular}
{\scriptsize
The `calexp` and `deepCoadd\_calexp` images and their associated source
catalogs are created.

}

\begin{tabular}{p{4cm}p{12cm}}
\toprule
Step 2
& Description \\ \hline
\end{tabular}
{\scriptsize
Roughly determine the coordinates of a bounding box that is contained
within the images that were processed.

}
\begin{tabular}{p{3cm}p{13cm}}
\hline
            & Expected Result \\ \hline
\end{tabular}
{\scriptsize
RA, Dec boundaries of a region in which to generate fake sources.

}

\begin{tabular}{p{4cm}p{12cm}}
\toprule
Step 3
& Description \\ \hline
\end{tabular}
{\scriptsize
Generate a catalog in the correct format for `insertFakes` to accept.
The catalog should specify positions and magnitudes of stars (and
optionally, parameters specifying galaxy shape, if galaxies are also
being inserted).

}
\begin{tabular}{p{3cm}p{13cm}}
\hline
            & Expected Result \\ \hline
\end{tabular}
{\scriptsize
An input catalog of fake source positions and magnitudes to be inserted
into the images.

}

\begin{tabular}{p{4cm}p{12cm}}
\toprule
Step 4
& Description \\ \hline
\end{tabular}
{\scriptsize
Execute `insertFakes.py` on the repository, specifying the input catalog
from the previous step.

}
\begin{tabular}{p{3cm}p{13cm}}
\hline
            & Expected Result \\ \hline
\end{tabular}
{\scriptsize
A repository with images that have fake sources inserted.

}

\begin{tabular}{p{4cm}p{12cm}}
\toprule
Step 5
& Description \\ \hline
\end{tabular}
{\scriptsize
Run `multiBandDriver.py` on the repository, specifying the fake-source
repository as the input.

}
\begin{tabular}{p{3cm}p{13cm}}
\hline
            & Expected Result \\ \hline
\end{tabular}
{\scriptsize
`calexp` and coadd images containing the artificial sources and sources
catalogs that contain their measurements along with the sources detected
in the original run.

}

\begin{tabular}{p{4cm}p{12cm}}
\toprule
Step 6
& Description \\ \hline
\end{tabular}
{\scriptsize
Confirm that the injected sources appear in the images and the catalogs.

}
\begin{tabular}{p{3cm}p{13cm}}
\hline
            & Expected Result \\ \hline
\end{tabular}
{\scriptsize
Fake sources and their measured properties are recoverable.

}

\subsubsection{LVV-T132 - Verify implementation of Pre-cursor and Real Data}\label{lvv-t132}

\begin{longtable}[]{llllll}
\toprule
Version & Status & Priority & Verification Type & Owner
\\\midrule
1 & Approved & Normal &
Test & Robert Gruendl
\\\bottomrule
\multicolumn{6}{c}{ Open \href{https://jira.lsstcorp.org/secure/Tests.jspa\#/testCase/LVV-T132}{LVV-T132} in Jira } \\
\end{longtable}

\paragraph{Verification Elements}\mbox{}\\

\begin{itemize}
\item \href{https://jira.lsstcorp.org/browse/LVV-127}{LVV-127} - DMS-REQ-0296-V-01: Pre-cursor, and Real Data

\end{itemize}

\paragraph{Test Items}\mbox{}\\

Demonstrate that pixel-oriented data from astronomical imaging cameras
(precursor or otherwise) can be processed using LSST Science Algorithms
and organized for access through the Data Butler Access Client. ~








\paragraph{Test Procedure}\mbox{}\\
\begin{tabular}{p{4cm}p{12cm}}
\toprule
Step 1
& Description \\ \hline
\end{tabular}
{\scriptsize
Confirm that the CI jobs used to test DRP processing successfully run.
These jobs use precursor datasets from cameras other than LSST.

}
\begin{tabular}{p{3cm}p{13cm}}
\hline
            & Expected Result \\ \hline
\end{tabular}

\begin{tabular}{p{4cm}p{12cm}}
\toprule
Step 2
& Description \\ \hline
\end{tabular}
{\scriptsize
For the precursor dataset, instantiate the Butler, load the data
products, and confirm that they exist as expected.

}
\begin{tabular}{p{3cm}p{13cm}}
\hline
            & Expected Result \\ \hline
\end{tabular}
{\scriptsize
Processed images, catalogs, calibration information, and other related
data products are present and accessible via the Butler.

}

\subsubsection{LVV-T144 - Verify implementation of Task Specification}\label{lvv-t144}

\begin{longtable}[]{llllll}
\toprule
Version & Status & Priority & Verification Type & Owner
\\\midrule
1 & Approved & Normal &
Test & Kian-Tat Lim
\\\bottomrule
\multicolumn{6}{c}{ Open \href{https://jira.lsstcorp.org/secure/Tests.jspa\#/testCase/LVV-T144}{LVV-T144} in Jira } \\
\end{longtable}

\paragraph{Verification Elements}\mbox{}\\

\begin{itemize}
\item \href{https://jira.lsstcorp.org/browse/LVV-136}{LVV-136} - DMS-REQ-0305-V-01: Task Specification

\end{itemize}

\paragraph{Test Items}\mbox{}\\

Verify that the DMS provides the ability to define a new or modified
pipeline task without recompilation.








\paragraph{Test Procedure}\mbox{}\\
\begin{tabular}{p{4cm}p{12cm}}
\toprule
Step 1
& Description \\ \hline
\end{tabular}
{\scriptsize
Inspect software architecture. ~Verify that there exist Tasks that can
be run and configured without re-compilation.

}
\begin{tabular}{p{3cm}p{13cm}}
\hline
            & Expected Result \\ \hline
\end{tabular}
{\scriptsize
Confirmation that the software architecture has allowed for
reconfiguring and running Tasks without recompilation.

}

\begin{tabular}{p{4cm}p{12cm}}
\toprule
Step 2
& Description \\ \hline
\end{tabular}
{\scriptsize
Verify that an example science algorithm can be run through one of these
Tasks.~ Three examples from different areas: source measurement, image
subtraction, and photometric-redshift estimation.

}
\begin{tabular}{p{3cm}p{13cm}}
\hline
            & Expected Result \\ \hline
\end{tabular}
{\scriptsize
Successful Task execution with different configurations, including
confirmation that the outputs are different from tasks with altered
configurations.

}

\subsubsection{LVV-T145 - Verify implementation of Task Configuration}\label{lvv-t145}

\begin{longtable}[]{llllll}
\toprule
Version & Status & Priority & Verification Type & Owner
\\\midrule
1 & Approved & Normal &
Test & Robert Lupton
\\\bottomrule
\multicolumn{6}{c}{ Open \href{https://jira.lsstcorp.org/secure/Tests.jspa\#/testCase/LVV-T145}{LVV-T145} in Jira } \\
\end{longtable}

\paragraph{Verification Elements}\mbox{}\\

\begin{itemize}
\item \href{https://jira.lsstcorp.org/browse/LVV-137}{LVV-137} - DMS-REQ-0306-V-01: Task Configuration

\end{itemize}

\paragraph{Test Items}\mbox{}\\

Verify that the DMS software provides configuration control to define,
override, and verify the configuration for a DMS Task.








\paragraph{Test Procedure}\mbox{}\\
\begin{tabular}{p{4cm}p{12cm}}
\toprule
Step 1
& Description \\ \hline
\end{tabular}
{\scriptsize
Inspect software design to verify that one can define the configuration
for a Task.

}
\begin{tabular}{p{3cm}p{13cm}}
\hline
            & Expected Result \\ \hline
\end{tabular}

\begin{tabular}{p{4cm}p{12cm}}
\toprule
Step 2
& Description \\ \hline
\end{tabular}
{\scriptsize
Run a Task with a known invalid configuration. ~Verify that the error is
caught before the science algorithm executes.

}
\begin{tabular}{p{3cm}p{13cm}}
\hline
            & Expected Result \\ \hline
\end{tabular}

\begin{tabular}{p{4cm}p{12cm}}
\toprule
Step 3
& Description \\ \hline
\end{tabular}
{\scriptsize
Run a simple task with two different configurations that make a material
difference for a Task. ~E.g., specify a different source detection
threshold. ~Verify that the configuration is different between the two
runs through difference in recorded provenance and in results.

}
\begin{tabular}{p{3cm}p{13cm}}
\hline
            & Expected Result \\ \hline
\end{tabular}

\subsubsection{LVV-T146 - Verify implementation of DMS Initialization Component}\label{lvv-t146}

\begin{longtable}[]{llllll}
\toprule
Version & Status & Priority & Verification Type & Owner
\\\midrule
1 & Approved & Normal &
Test & Robert Gruendl
\\\bottomrule
\multicolumn{6}{c}{ Open \href{https://jira.lsstcorp.org/secure/Tests.jspa\#/testCase/LVV-T146}{LVV-T146} in Jira } \\
\end{longtable}

\paragraph{Verification Elements}\mbox{}\\

\begin{itemize}
\item \href{https://jira.lsstcorp.org/browse/LVV-128}{LVV-128} - DMS-REQ-0297-V-01: DMS Initialization Component

\end{itemize}

\paragraph{Test Items}\mbox{}\\

Demonstrate that the DMS can be initialized in a safe state that will
not allow data corruption/loss.








\paragraph{Test Procedure}\mbox{}\\
\begin{tabular}{p{4cm}p{12cm}}
\toprule
Step 1
& Description \\ \hline
\end{tabular}
{\scriptsize
Power-cycle all of the DM systems at each Facility.

}
\begin{tabular}{p{3cm}p{13cm}}
\hline
            & Expected Result \\ \hline
\end{tabular}
{\scriptsize
Restart of all DM systems.

}

\begin{tabular}{p{4cm}p{12cm}}
\toprule
Step 2
& Description \\ \hline
\end{tabular}
{\scriptsize
Observe each system and ensure that it has recovered in a properly
initialized state.

}
\begin{tabular}{p{3cm}p{13cm}}
\hline
            & Expected Result \\ \hline
\end{tabular}
{\scriptsize
Systems are all active and initialized for their designated purpose.

}

\subsubsection{LVV-T149 - Verify implementation of Catalog Queries}\label{lvv-t149}

\begin{longtable}[]{llllll}
\toprule
Version & Status & Priority & Verification Type & Owner
\\\midrule
1 & Approved & Normal &
Test & Colin Slater
\\\bottomrule
\multicolumn{6}{c}{ Open \href{https://jira.lsstcorp.org/secure/Tests.jspa\#/testCase/LVV-T149}{LVV-T149} in Jira } \\
\end{longtable}

\paragraph{Verification Elements}\mbox{}\\

\begin{itemize}
\item \href{https://jira.lsstcorp.org/browse/LVV-33}{LVV-33} - DMS-REQ-0075-V-01: Catalog Queries

\end{itemize}

\paragraph{Test Items}\mbox{}\\

Verify that SQL, or a similar structured language, can be used to query
catalogs.








\paragraph{Test Procedure}\mbox{}\\
\begin{tabular}{p{4cm}p{12cm}}
\toprule
Step 1
& Description \\ \hline
\end{tabular}
{\scriptsize
Execute a simple query (for example, the one below) and confirm that it
returns the expected result.

}
\begin{tabular}{p{3cm}p{13cm}}
\hline
            & Example Code \\ \hline
\end{tabular}
{\scriptsize
SELECT * FROM Object WHERE qserv\_areaspec\_box(316.582327, −6.839078,
316.653938, −6.781822)

}
\begin{tabular}{p{3cm}p{13cm}}
\hline
            & Expected Result \\ \hline
\end{tabular}
{\scriptsize
A catalog of objects satisfying the specified constraints.~

}

\begin{tabular}{p{4cm}p{12cm}}
\toprule
Step 2
& Description \\ \hline
\end{tabular}
{\scriptsize
Repeat the query from all available access routes (e.g., an external VO
client, internal DM tools on the development cluster, the Science
Platform query tool, and from within the Notebook Aspect), confirming in
each case that the results are as expected.

}
\begin{tabular}{p{3cm}p{13cm}}
\hline
            & Expected Result \\ \hline
\end{tabular}

\subsubsection{LVV-T151 - Verify Implementation of Catalog Export Formats From the Notebook Aspect}\label{lvv-t151}

\begin{longtable}[]{llllll}
\toprule
Version & Status & Priority & Verification Type & Owner
\\\midrule
1 & Approved & Normal &
Test & Colin Slater
\\\bottomrule
\multicolumn{6}{c}{ Open \href{https://jira.lsstcorp.org/secure/Tests.jspa\#/testCase/LVV-T151}{LVV-T151} in Jira } \\
\end{longtable}

\paragraph{Verification Elements}\mbox{}\\

\begin{itemize}
\item \href{https://jira.lsstcorp.org/browse/LVV-35}{LVV-35} - DMS-REQ-0078-V-01: Catalog Export Formats

\end{itemize}

\paragraph{Test Items}\mbox{}\\

Verify that catalog data is exportable from the notebook aspect in a
variety of community-standard formats.








\paragraph{Test Procedure}\mbox{}\\
\begin{tabular}{p{4cm}p{12cm}}
\toprule
Step 1-1
{\scriptsize from \hyperref[lvv-t837]{LVV-T837} }
& Description \\ \hline
\end{tabular}
{\scriptsize
Authenticate to the notebook aspect of the LSST Science Platform
(NB-LSP). ~This is currently at
https://lsst-lsp-stable.ncsa.illinois.edu/nb.

}
\begin{tabular}{p{3cm}p{13cm}}
\hline
            & Expected Result \\ \hline
\end{tabular}
{\scriptsize
Redirection to the spawner page of the NB-LSP allowing selection of the
containerized stack version and machine flavor.

}

\begin{tabular}{p{4cm}p{12cm}}
\toprule
Step 1-2
{\scriptsize from \hyperref[lvv-t837]{LVV-T837} }
& Description \\ \hline
\end{tabular}
{\scriptsize
Spawn a container by:\\
1) choosing an appropriate stack version: e.g. the latest weekly.\\
2) choosing an appropriate machine flavor: e.g. medium\\
3) click ``Spawn''

}
\begin{tabular}{p{3cm}p{13cm}}
\hline
            & Expected Result \\ \hline
\end{tabular}
{\scriptsize
Redirection to the JupyterLab environment served from the chosen
container containing the correct stack version.

}

\begin{tabular}{p{4cm}p{12cm}}
\toprule
Step 2-1
{\scriptsize from \hyperref[lvv-t838]{LVV-T838} }
& Description \\ \hline
\end{tabular}
{\scriptsize
Open a new launcher by navigating in the top menu bar ``File''
-\textgreater{} ``New Launcher''

}
\begin{tabular}{p{3cm}p{13cm}}
\hline
            & Expected Result \\ \hline
\end{tabular}
{\scriptsize
A launcher window with several sections, potentially with several kernel
versions for each.

}

\begin{tabular}{p{4cm}p{12cm}}
\toprule
Step 2-2
{\scriptsize from \hyperref[lvv-t838]{LVV-T838} }
& Description \\ \hline
\end{tabular}
{\scriptsize
Select the option under ``Notebook'' labeled ``LSST'' by clicking on the
icon.

}
\begin{tabular}{p{3cm}p{13cm}}
\hline
            & Expected Result \\ \hline
\end{tabular}
{\scriptsize
An empty notebook with a single empty cell. ~The kernel show up as
``LSST'' in the top right of the notebook.

}

\begin{tabular}{p{4cm}p{12cm}}
\toprule
Step 3-1
{\scriptsize from \hyperref[lvv-t1207]{LVV-T1207} }
& Description \\ \hline
\end{tabular}
{\scriptsize
Execute a query in a notebook to select a small number of stars. In the
example code below, we query the WISE catalog, then extract the results
to an Astropy table.

}
\begin{tabular}{p{3cm}p{13cm}}
\hline
            & Example Code \\ \hline
\end{tabular}
{\scriptsize
\begin{verbatim}
import pandas
import pyvo
service = pyvo.dal.TAPService('http://lsst-lsp-stable.ncsa.illinois.edu/api/tap')
\end{verbatim}

results = service.search(``SELECT ra, decl, w1mpro\_ep, w2mpro\_ep,
w3mpro\_ep FROM wise\_00.allwise\_p3as\_mep WHERE CONTAINS(POINT('ICRS',
ra, decl), CIRCLE('ICRS', 192.85, 27.13, .2)) = 1'')\\
tab = results.to\_table()

}
\begin{tabular}{p{3cm}p{13cm}}
\hline
            & Expected Result \\ \hline
\end{tabular}

\begin{tabular}{p{4cm}p{12cm}}
\toprule
Step 4
& Description \\ \hline
\end{tabular}
{\scriptsize
Using the example code below, save the files to your storage space on
the LSP Notebook Aspect.\\[2\baselineskip]Confirm that non-empty output
files appear on disk.

}
\begin{tabular}{p{3cm}p{13cm}}
\hline
            & Example Code \\ \hline
\end{tabular}
{\scriptsize
tab.write('test.csv', format='ascii.csv')\\
tab.write('test.vot', format='votable')\\
tab.write('test.fits', format='fits')

}
\begin{tabular}{p{3cm}p{13cm}}
\hline
            & Expected Result \\ \hline
\end{tabular}
{\scriptsize
For the example given here, there should be the following files with the
file size as listed:

\begin{itemize}
\tightlist
\item
  test.csv 5.7M
\item
  test.vot 16M
\item
  test.fits 4.5M
\end{itemize}

}

\begin{tabular}{p{4cm}p{12cm}}
\toprule
Step 5
& Description \\ \hline
\end{tabular}
{\scriptsize
Check that these files contain the same number of rows:

}
\begin{tabular}{p{3cm}p{13cm}}
\hline
            & Example Code \\ \hline
\end{tabular}
{\scriptsize
from astropy.table import Table\\
dat\_csv = Table.read('test.csv', format='ascii.csv')\\
dat\_vot = Table.read('test.vot', format='votable')\\
dat\_fits = Table.read('test.fits',
format='fits')\\[2\baselineskip]import numpy as np\\
print(np.size(dat\_csv), np.size(dat\_vot), np.size(dat\_fits))

}
\begin{tabular}{p{3cm}p{13cm}}
\hline
            & Expected Result \\ \hline
\end{tabular}
{\scriptsize
Print statement produces output ``97058 97058 97058''.

}

\begin{tabular}{p{4cm}p{12cm}}
\toprule
Step 6-1
{\scriptsize from \hyperref[lvv-t1208]{LVV-T1208} }
& Description \\ \hline
\end{tabular}
{\scriptsize
Under the `File' menu at the top of your Jupyter notebook session,
select one of the following:\\[2\baselineskip]

\begin{itemize}
\tightlist
\item
  Save All, Exit, and Log Out
\item
  Exit and Log Out Without Saving
\end{itemize}

}
\begin{tabular}{p{3cm}p{13cm}}
\hline
            & Expected Result \\ \hline
\end{tabular}
{\scriptsize
You will be returned to the LSP landing page:
\url{https://lsst-lsp-stable.ncsa.illinois.edu/} It is now safe to close
the browser window.~

}

\subsubsection{LVV-T216 - Installation of the Alert Distribution payloads.}\label{lvv-t216}

\begin{longtable}[]{llllll}
\toprule
Version & Status & Priority & Verification Type & Owner
\\\midrule
1 & Approved & Normal &
Test & Eric Bellm
\\\bottomrule
\multicolumn{6}{c}{ Open \href{https://jira.lsstcorp.org/secure/Tests.jspa\#/testCase/LVV-T216}{LVV-T216} in Jira } \\
\end{longtable}

\paragraph{Verification Elements}\mbox{}\\

\begin{itemize}
\item \href{https://jira.lsstcorp.org/browse/LVV-139}{LVV-139} - DMS-REQ-0308-V-01: Software Architecture to Enable Community Re-Use

\end{itemize}

\paragraph{Test Items}\mbox{}\\

This test will check:\\

\begin{itemize}
\tightlist
\item
  That the Alert Distribution payloads are available from documented
  channels.
\item
  That the Alert Distribution payloads can be installed on LSST Data
  Facility-managed systems.
\item
  That the Alert Distribution payloads can be executed by LSST Data
  Facility-managed systems.
\end{itemize}



\paragraph{Environment Needs}\mbox{}\\


\subparagraph{Hardware}\mbox{}\\
This test case shall be executed on the Kubernetes Commons at the LDF.\\
As discussed in https://dmtn-028.lsst.io/ and https://dmtn-081.lsst.io/,
the test machine should have at least 16 cores, 64 GB of memory and
access to at least 1.5 TB of shared storage.



\paragraph{Test Procedure}\mbox{}\\
\begin{tabular}{p{4cm}p{12cm}}
\toprule
Step 1
& Description \\ \hline
\end{tabular}
{\scriptsize
Download Kafka Docker image from
https://github.com/lsst-dm/alert\_stream.

}
\begin{tabular}{p{3cm}p{13cm}}
\hline
            & Expected Result \\ \hline
\end{tabular}
{\scriptsize
Runs without error

}

\begin{tabular}{p{4cm}p{12cm}}
\toprule
Step 2
& Description \\ \hline
\end{tabular}
{\scriptsize
Change to the alert\_stream directory and build the docker image.\\

\begin{verbatim}
docker build -t "lsst-kub001:5000/alert_stream"
\end{verbatim}

}
\begin{tabular}{p{3cm}p{13cm}}
\hline
            & Expected Result \\ \hline
\end{tabular}
{\scriptsize
Runs without error

}

\begin{tabular}{p{4cm}p{12cm}}
\toprule
Step 3
& Description \\ \hline
\end{tabular}
{\scriptsize
Register it with Kubernetes\\[2\baselineskip]docker push
lsst-kub001:5000/alert\_stream

}
\begin{tabular}{p{3cm}p{13cm}}
\hline
            & Expected Result \\ \hline
\end{tabular}
{\scriptsize
Runs without error

}

\begin{tabular}{p{4cm}p{12cm}}
\toprule
Step 4
& Description \\ \hline
\end{tabular}
{\scriptsize
From the alert\_stream/kubernetes directory, start Kafka and
Zookeeper:\\[2\baselineskip]

\begin{verbatim}
kubectl create -f zookeeper-service.yaml
kubectl create -f zookeeper-deployment.yaml
kubectl create -f kafka-deployment.yaml
kubectl create -f kafka-service.yaml
\end{verbatim}

(use kubectl get pods/services between each command to check status;
wait until each is ``Running'' before starting the next
command)\\[2\baselineskip]

}
\begin{tabular}{p{3cm}p{13cm}}
\hline
            & Expected Result \\ \hline
\end{tabular}
{\scriptsize
Runs without error

}

\begin{tabular}{p{4cm}p{12cm}}
\toprule
Step 5
& Description \\ \hline
\end{tabular}
{\scriptsize
Confirm Kafka and Zookeeper are listed when
running\\[2\baselineskip]kubectl get
pods\\[2\baselineskip]and\\[2\baselineskip]kubectl get services

}
\begin{tabular}{p{3cm}p{13cm}}
\hline
            & Expected Result \\ \hline
\end{tabular}
{\scriptsize
Output should be similar to:\\[2\baselineskip]kubectl get pods\\
NAME ~ ~ ~ ~ ~ ~ ~ ~ ~ ~ ~ ~READY ~ ~ STATUS ~ ~RESTARTS ~ AGE\\
kafka-768ddf5564-xwgvh ~ ~ ~1/1 ~ ~ ~ Running ~ 0 ~ ~ ~ ~ ~31s\\
zookeeper-f798cc548-mgkpn ~ 1/1 ~ ~ ~ Running ~ 0 ~ ~ ~ ~
~1m\\[2\baselineskip]kubectl get services\\
NAME ~ ~ ~ ~TYPE ~ ~ ~ ~CLUSTER-IP ~ ~ ~EXTERNAL-IP ~ PORT(S) ~ ~ AGE\\
kafka ~ ~ ~ ClusterIP ~ 10.105.19.124 ~ \textless{}none\textgreater{} ~
~ ~ ~9092/TCP ~ ~6s\\
zookeeper ~ ClusterIP ~ 10.97.110.124 ~ \textless{}none\textgreater{} ~
~ ~ ~32181/TCP ~ 2m

}

\subsubsection{LVV-T217 - Full Stream Alert Distribution}\label{lvv-t217}

\begin{longtable}[]{llllll}
\toprule
Version & Status & Priority & Verification Type & Owner
\\\midrule
1 & Approved & Normal &
Test & Eric Bellm
\\\bottomrule
\multicolumn{6}{c}{ Open \href{https://jira.lsstcorp.org/secure/Tests.jspa\#/testCase/LVV-T217}{LVV-T217} in Jira } \\
\end{longtable}

\paragraph{Verification Elements}\mbox{}\\

\begin{itemize}
\item \href{https://jira.lsstcorp.org/browse/LVV-3}{LVV-3} - DMS-REQ-0002-V-01: Transient Alert Distribution

\end{itemize}

\paragraph{Test Items}\mbox{}\\

This test will check that the full stream of LSST alerts can be
distributed to end users.\\[2\baselineskip]Specifically, this will
demonstrate that:

\begin{itemize}
\tightlist
\item
  Serialized alert packets can be loaded into the alert distribution
  system at LSST-relevant scales (10,000 alerts every 39 seconds);
\item
  Alert packets can be retrieved from the queue system at LSST-relevant
  scales.
\end{itemize}


\paragraph{Predecessors}\mbox{}\\
\href{https://jira.lsstcorp.org/secure/Tests.jspa\#/testCase/LVV-T216}{LVV-T216}

\paragraph{Environment Needs}\mbox{}\\

\subparagraph{Software}\mbox{}\\
The Kafka cluster and Zookeeper shall be instantiated according to the
procedure described in
\href{https://jira.lsstcorp.org/secure/Tests.jspa\#/testCase/LVV-T216}{LVV-T216}.

\subparagraph{Hardware}\mbox{}\\
This test case shall be executed on the Kubernetes Commons at the LDF.\\
As discussed in https://dmtn-028.lsst.io/ and https://dmtn-081.lsst.io/,
the test machine should have at least 16 cores, 64 GB of memory and
access to at least 1.5 TB of shared storage.

\paragraph{Input Specification}\mbox{}\\
Input data: A sample of Avro-formatted alert packets.

\paragraph{Output Specification}\mbox{}\\
Multiple Kafka consumers will run and write log files to disk.\\
The logs will include printing every \emph{Nth} alert to to the log as
well as a log summarizing the queue offset.

\paragraph{Test Procedure}\mbox{}\\
\begin{tabular}{p{4cm}p{12cm}}
\toprule
Step 1-1
{\scriptsize from \hyperref[lvv-t866]{LVV-T866} }
& Description \\ \hline
\end{tabular}
{\scriptsize
Perform the steps of Alert Production (including, but not necessarily
limited to, single frame processing, ISR, source detection/measurement,
PSF estimation, photometric and astrometric calibration, difference
imaging, DIASource detection/measurement, source association). During
Operations, it is presumed that these are automated for a given
dataset.~

}
\begin{tabular}{p{3cm}p{13cm}}
\hline
            & Expected Result \\ \hline
\end{tabular}
{\scriptsize
An output dataset including difference images and DIASource and
DIAObject measurements.

}

\begin{tabular}{p{4cm}p{12cm}}
\toprule
Step 1-2
{\scriptsize from \hyperref[lvv-t866]{LVV-T866} }
& Description \\ \hline
\end{tabular}
{\scriptsize
Verify that the expected data products have been produced, and that
catalogs contain reasonable values for measured quantities of interest.

}
\begin{tabular}{p{3cm}p{13cm}}
\hline
            & Expected Result \\ \hline
\end{tabular}

\begin{tabular}{p{4cm}p{12cm}}
\toprule
Step 2
& Description \\ \hline
\end{tabular}
{\scriptsize
Start a consumer that monitors the full stream and logs a deserialized
version of every Nth packet:\\

\begin{verbatim}
kubectl create -f consumerall-deployment.yaml
\end{verbatim}

}
\begin{tabular}{p{3cm}p{13cm}}
\hline
            & Expected Result \\ \hline
\end{tabular}
{\scriptsize
Runs without error

}

\begin{tabular}{p{4cm}p{12cm}}
\toprule
Step 3
& Description \\ \hline
\end{tabular}
{\scriptsize
\begin{verbatim}
Start a producer that reads alert packets from disk and loads them into the Kafka queue:
\end{verbatim}

\begin{verbatim}
kubectl create -f sender-deployment.yaml
\end{verbatim}

}
\begin{tabular}{p{3cm}p{13cm}}
\hline
            & Expected Result \\ \hline
\end{tabular}
{\scriptsize
Runs without error

}

\begin{tabular}{p{4cm}p{12cm}}
\toprule
Step 4
& Description \\ \hline
\end{tabular}
{\scriptsize
Determine the name of the alert sender pod with\\[2\baselineskip]kubectl
get pods\\[2\baselineskip]Examine output log
files.\\[2\baselineskip]kubectl logs \textless{}pod
name\textgreater{}\\[2\baselineskip]Verify that alerts are being sent
within 40 seconds by subtracting the timing measurements.

}
\begin{tabular}{p{3cm}p{13cm}}
\hline
            & Expected Result \\ \hline
\end{tabular}
{\scriptsize
Similar to\\[2\baselineskip]kubectl logs sender-7d6f98586f-nhwfj\\
visit: 1570. ~ ~ time: 1530588618.0313473\\
visits finished: 1 ~ ~ ~time: 1530588653.5614944\\
visit: 1571. ~ ~ time: 1530588657.0087624\\
visits finished: 2 ~ ~ ~time: 1530588692.506188\\
visit: 1572. ~ ~ time: 1530588696.0051727\\
visits finished: 3 ~ ~ ~time: 1530588731.5900314\\[3\baselineskip]

}

\begin{tabular}{p{4cm}p{12cm}}
\toprule
Step 5
& Description \\ \hline
\end{tabular}
{\scriptsize
Determine the name of the consumer pod with\\[2\baselineskip]kubectl get
pods\\[2\baselineskip]Examine output log files.\\[2\baselineskip]kubectl
logs \textless{}pod name\textgreater{}\\[2\baselineskip]The packet log
should show deserialized alert packets with contents matching the input
packets.\\[2\baselineskip]

}
\begin{tabular}{p{3cm}p{13cm}}
\hline
            & Expected Result \\ \hline
\end{tabular}
{\scriptsize
Similar to \{'alertId': 12132024420, `l1dbId': 71776805594116,
`diaSource': \{'diaSourceId':\\
73499448928374785, `ccdVisitId': 2020011570, `diaObjectId':
71776805594116, 'ssO\\
bjectId': None, `parentDiaSourceId': None, `midPointTai': 59595.37041,
'filterNa\\
me': `y', `ra': 172.24912810036074, `decl': -80.64214929176521,
`ra\_decl\_Cov': \{\\
`raSigma': 0.0003428002819418907, `declSigma': 0.00027273103478364646,
'ra\_decl\_\\
Cov': 0.000628734880592674\}, `x': 2979.08837890625, `y':
3843.328857421875, 'x\_y\\
\_Cov': \{'xSigma': 0.6135467886924744, `ySigma': 0.77132648229599,
`x\_y\_Cov': 0.0\\
007463791407644749\}, `apFlux': None, `apFluxErr': None, `snr':
0.366516500711441\\
04, `psFlux': 7.698232025177276e-07, `psRa': None, `psDecl': None,
`ps\_Cov': Non\\
e, `psLnL': None, `psChi2': None, `psNdata': None, `trailFlux': None,
`trailRa':\\
etc.

}

\subsubsection{LVV-T218 - Simple Filtering of the LSST Alert Stream}\label{lvv-t218}

\begin{longtable}[]{llllll}
\toprule
Version & Status & Priority & Verification Type & Owner
\\\midrule
1 & Approved & Normal &
Test & Eric Bellm
\\\bottomrule
\multicolumn{6}{c}{ Open \href{https://jira.lsstcorp.org/secure/Tests.jspa\#/testCase/LVV-T218}{LVV-T218} in Jira } \\
\end{longtable}

\paragraph{Verification Elements}\mbox{}\\

\begin{itemize}
\item \href{https://jira.lsstcorp.org/browse/LVV-173}{LVV-173} - DMS-REQ-0342-V-01: Alert Filtering Service

\item \href{https://jira.lsstcorp.org/browse/LVV-179}{LVV-179} - DMS-REQ-0348-V-01: Pre-defined alert filters

\item \href{https://jira.lsstcorp.org/browse/LVV-174}{LVV-174} - DMS-REQ-0343-V-01: Number of full-size alerts

\end{itemize}

\paragraph{Test Items}\mbox{}\\

This test will demonstrate the LSST Alert Filtering Service that returns
a subset of alerts from the full stream identified by user-provided
filters.\\[2\baselineskip]Specifically, this will demonstrate that:\\

\begin{itemize}
\tightlist
\item
  The filtering service can retrieve alerts from the full alert stream
  and filter them according to their contents; ~ ~
\item
  The filtered subset can be delivered to science users.
\end{itemize}


\paragraph{Predecessors}\mbox{}\\
​\href{https://jira.lsstcorp.org/secure/Tests.jspa\#/testCase/LVV-T216}{LVV-T216}​​​\\
​\href{https://jira.lsstcorp.org/secure/Tests.jspa\#/testCase/LVV-T217}{LVV-T217}​​​

\paragraph{Environment Needs}\mbox{}\\

\subparagraph{Software}\mbox{}\\
The Kafka cluster and Zookeeper shall be instantiated according to the
procedure described in
\href{https://jira.lsstcorp.org/secure/Tests.jspa\#/testCase/LVV-T216}{LVV-T216}.

\subparagraph{Hardware}\mbox{}\\
This test case shall be executed on the Kubernetes Commons at the LDF.\\
As discussed in https://dmtn-028.lsst.io/ and https://dmtn-081.lsst.io/,
the test machine should have at least 16 cores, 64 GB of memory and
access to at least 1.5 TB of shared storage.



\paragraph{Test Procedure}\mbox{}\\
\begin{tabular}{p{4cm}p{12cm}}
\toprule
Step 1-1
{\scriptsize from \hyperref[lvv-t216]{LVV-T216} }
& Description \\ \hline
\end{tabular}
{\scriptsize
Download Kafka Docker image from
https://github.com/lsst-dm/alert\_stream.

}
\begin{tabular}{p{3cm}p{13cm}}
\hline
            & Expected Result \\ \hline
\end{tabular}
{\scriptsize
Runs without error

}

\begin{tabular}{p{4cm}p{12cm}}
\toprule
Step 1-2
{\scriptsize from \hyperref[lvv-t216]{LVV-T216} }
& Description \\ \hline
\end{tabular}
{\scriptsize
Change to the alert\_stream directory and build the docker image.\\

\begin{verbatim}
docker build -t "lsst-kub001:5000/alert_stream"
\end{verbatim}

}
\begin{tabular}{p{3cm}p{13cm}}
\hline
            & Expected Result \\ \hline
\end{tabular}
{\scriptsize
Runs without error

}

\begin{tabular}{p{4cm}p{12cm}}
\toprule
Step 1-3
{\scriptsize from \hyperref[lvv-t216]{LVV-T216} }
& Description \\ \hline
\end{tabular}
{\scriptsize
Register it with Kubernetes\\[2\baselineskip]docker push
lsst-kub001:5000/alert\_stream

}
\begin{tabular}{p{3cm}p{13cm}}
\hline
            & Expected Result \\ \hline
\end{tabular}
{\scriptsize
Runs without error

}

\begin{tabular}{p{4cm}p{12cm}}
\toprule
Step 1-4
{\scriptsize from \hyperref[lvv-t216]{LVV-T216} }
& Description \\ \hline
\end{tabular}
{\scriptsize
From the alert\_stream/kubernetes directory, start Kafka and
Zookeeper:\\[2\baselineskip]

\begin{verbatim}
kubectl create -f zookeeper-service.yaml
kubectl create -f zookeeper-deployment.yaml
kubectl create -f kafka-deployment.yaml
kubectl create -f kafka-service.yaml
\end{verbatim}

(use kubectl get pods/services between each command to check status;
wait until each is ``Running'' before starting the next
command)\\[2\baselineskip]

}
\begin{tabular}{p{3cm}p{13cm}}
\hline
            & Expected Result \\ \hline
\end{tabular}
{\scriptsize
Runs without error

}

\begin{tabular}{p{4cm}p{12cm}}
\toprule
Step 1-5
{\scriptsize from \hyperref[lvv-t216]{LVV-T216} }
& Description \\ \hline
\end{tabular}
{\scriptsize
Confirm Kafka and Zookeeper are listed when
running\\[2\baselineskip]kubectl get
pods\\[2\baselineskip]and\\[2\baselineskip]kubectl get services

}
\begin{tabular}{p{3cm}p{13cm}}
\hline
            & Expected Result \\ \hline
\end{tabular}
{\scriptsize
Output should be similar to:\\[2\baselineskip]kubectl get pods\\
NAME ~ ~ ~ ~ ~ ~ ~ ~ ~ ~ ~ ~READY ~ ~ STATUS ~ ~RESTARTS ~ AGE\\
kafka-768ddf5564-xwgvh ~ ~ ~1/1 ~ ~ ~ Running ~ 0 ~ ~ ~ ~ ~31s\\
zookeeper-f798cc548-mgkpn ~ 1/1 ~ ~ ~ Running ~ 0 ~ ~ ~ ~
~1m\\[2\baselineskip]kubectl get services\\
NAME ~ ~ ~ ~TYPE ~ ~ ~ ~CLUSTER-IP ~ ~ ~EXTERNAL-IP ~ PORT(S) ~ ~ AGE\\
kafka ~ ~ ~ ClusterIP ~ 10.105.19.124 ~ \textless{}none\textgreater{} ~
~ ~ ~9092/TCP ~ ~6s\\
zookeeper ~ ClusterIP ~ 10.97.110.124 ~ \textless{}none\textgreater{} ~
~ ~ ~32181/TCP ~ 2m

}

\begin{tabular}{p{4cm}p{12cm}}
\toprule
Step 2
& Description \\ \hline
\end{tabular}
{\scriptsize
Start 100 consumers that consume the filtered streams and logs a
deserialized version of every Nth packet:\\[2\baselineskip]

\begin{verbatim}
kubectl create -f consumer1-deployment.yaml
kubectl create -f consumer2-deployment.yaml
kubectl create -f consumer3-deployment.yaml
kubectl create -f consumer4-deployment.yaml
kubectl create -f consumer5-deployment.yaml
kubectl create -f consumer6-deployment.yaml
kubectl create -f consumer7-deployment.yaml
kubectl create -f consumer8-deployment.yaml
kubectl create -f consumer9-deployment.yaml
kubectl create -f consumer10-deployment.yaml
\end{verbatim}

}
\begin{tabular}{p{3cm}p{13cm}}
\hline
            & Expected Result \\ \hline
\end{tabular}
{\scriptsize
Runs without error

}

\begin{tabular}{p{4cm}p{12cm}}
\toprule
Step 3
& Description \\ \hline
\end{tabular}
{\scriptsize
Start 5 filter groups:\\

\begin{verbatim}
kubectl create -f filterer1-deployment.yaml
kubectl create -f filterer2-deployment.yaml
kubectl create -f filterer3-deployment.yaml
kubectl create -f filterer4-deployment.yaml
kubectl create -f filterer5-deployment.yaml
\end{verbatim}

}
\begin{tabular}{p{3cm}p{13cm}}
\hline
            & Expected Result \\ \hline
\end{tabular}
{\scriptsize
Runs without error

}

\begin{tabular}{p{4cm}p{12cm}}
\toprule
Step 4
& Description \\ \hline
\end{tabular}
{\scriptsize
Start a producer that reads alert packets from disk and loads them into
the Kafka queue:\\[2\baselineskip]

\begin{verbatim}
kubectl create -f sender-deployment.yaml
\end{verbatim}

}
\begin{tabular}{p{3cm}p{13cm}}
\hline
            & Expected Result \\ \hline
\end{tabular}
{\scriptsize
Runs without error

}

\begin{tabular}{p{4cm}p{12cm}}
\toprule
Step 5
& Description \\ \hline
\end{tabular}
{\scriptsize
Determine the name of the alert sender pod with\\[2\baselineskip]kubectl
get pods\\[2\baselineskip]Examine output log
files.\\[2\baselineskip]kubectl logs \textless{}pod
name\textgreater{}\\[2\baselineskip]Verify that alerts are being sent
within 40 seconds by subtracting the timing measurements.

}
\begin{tabular}{p{3cm}p{13cm}}
\hline
            & Expected Result \\ \hline
\end{tabular}
{\scriptsize
Similar to\\[2\baselineskip]kubectl logs sender-7d6f98586f-nhwfj\\
visit: 1570. ~ ~ time: 1530588618.0313473\\
visits finished: 1 ~ ~ ~time: 1530588653.5614944\\
visit: 1571. ~ ~ time: 1530588657.0087624\\
visits finished: 2 ~ ~ ~time: 1530588692.506188\\
visit: 1572. ~ ~ time: 1530588696.0051727\\
visits finished: 3 ~ ~ ~time: 1530588731.5900314\\[2\baselineskip]

}

\begin{tabular}{p{4cm}p{12cm}}
\toprule
Step 6
& Description \\ \hline
\end{tabular}
{\scriptsize
Determine the name of the consumer pods with\\[2\baselineskip]kubectl
get pods\\[2\baselineskip]Examine output log
files.\\[2\baselineskip]kubectl logs \textless{}pod
name\textgreater{}\\[2\baselineskip]The packet log should show
deserialized alert packets with contents matching the input packets.

}
\begin{tabular}{p{3cm}p{13cm}}
\hline
            & Expected Result \\ \hline
\end{tabular}
{\scriptsize
Similar to\\[2\baselineskip]\{'alertId': 12132024420, `l1dbId':
71776805594116, `diaSource': \{'diaSourceId':\\
73499448928374785, `ccdVisitId': 2020011570, `diaObjectId':
71776805594116, 'ssO\\
bjectId': None, `parentDiaSourceId': None, `midPointTai': 59595.37041,
'filterNa\\
me': `y', `ra': 172.24912810036074, `decl': -80.64214929176521,
`ra\_decl\_Cov': \{\\
`raSigma': 0.0003428002819418907, `declSigma': 0.00027273103478364646,
'ra\_decl\_\\
Cov': 0.000628734880592674\}, `x': 2979.08837890625, `y':
3843.328857421875, 'x\_y\\
\_Cov': \{'xSigma': 0.6135467886924744, `ySigma': 0.77132648229599,
`x\_y\_Cov': 0.0\\
007463791407644749\}, `apFlux': None, `apFluxErr': None, `snr':
0.366516500711441\\
04, `psFlux': 7.698232025177276e-07, `psRa': None, `psDecl': None,
`ps\_Cov': Non\\
e, `psLnL': None, `psChi2': None, `psNdata': None, `trailFlux': None,
`trailRa':\\
etc.

}

\subsubsection{LVV-T62 - Verify implementation of Provide PSF for Coadded Images}\label{lvv-t62}

\begin{longtable}[]{llllll}
\toprule
Version & Status & Priority & Verification Type & Owner
\\\midrule
2 & Approved & Normal &
Test & Jim Bosch
\\\bottomrule
\multicolumn{6}{c}{ Open \href{https://jira.lsstcorp.org/secure/Tests.jspa\#/testCase/LVV-T62}{LVV-T62} in Jira } \\
\end{longtable}

\paragraph{Verification Elements}\mbox{}\\

\begin{itemize}
\item \href{https://jira.lsstcorp.org/browse/LVV-20}{LVV-20} - DMS-REQ-0047-V-01: Provide PSF for Coadded Images

\end{itemize}

\paragraph{Test Items}\mbox{}\\

Verify that all coadd images produced by the DRP pipelines include a
model from which an image of the PSF at any point on the coadd can be
obtained.








\paragraph{Test Procedure}\mbox{}\\
\begin{tabular}{p{4cm}p{12cm}}
\toprule
Step 1
& Description \\ \hline
\end{tabular}
{\scriptsize
Identify a dataset with coadded images in multiple filters.

}
\begin{tabular}{p{3cm}p{13cm}}
\hline
            & Expected Result \\ \hline
\end{tabular}
{\scriptsize
Multi-band data that has been processed through the coaddition stage.

}

\begin{tabular}{p{4cm}p{12cm}}
\toprule
Step 2-1
{\scriptsize from \hyperref[lvv-t987]{LVV-T987} }
& Description \\ \hline
\end{tabular}
{\scriptsize
Identify the path to the data repository, which we will refer to as
`DATA/path', then execute the following:

}
\begin{tabular}{p{3cm}p{13cm}}
\hline
            & Example Code \\ \hline
\end{tabular}
{\scriptsize
\begin{verbatim}
import lsst.daf.persistence as dafPersist
butler = dafPersist.Butler(inputs='DATA/path')
\end{verbatim}

}
\begin{tabular}{p{3cm}p{13cm}}
\hline
            & Expected Result \\ \hline
\end{tabular}
{\scriptsize
Butler repo available for reading.

}

\begin{tabular}{p{4cm}p{12cm}}
\toprule
Step 3
& Description \\ \hline
\end{tabular}
{\scriptsize
Load the exposures, then select Objects classified as point sources on
at least 10 different coadd images (including all bands). Evaluate the
PSF model at the positions of these Objects, and verify that subtracting
a scaled version of the PSF model from the processed visit image yields
residuals consistent with pure noise.

}
\begin{tabular}{p{3cm}p{13cm}}
\hline
            & Expected Result \\ \hline
\end{tabular}
{\scriptsize
Images with the PSF model subtracted, leaving only residuals that are
consistent with being noise.

}

\subsubsection{LVV-T283 - RAS-00-00: Writing well-formed raw image}\label{lvv-t283}

\begin{longtable}[]{llllll}
\toprule
Version & Status & Priority & Verification Type & Owner
\\\midrule
1 & Approved & Normal &
Test & Michelle Butler
\\\bottomrule
\multicolumn{6}{c}{ Open \href{https://jira.lsstcorp.org/secure/Tests.jspa\#/testCase/LVV-T283}{LVV-T283} in Jira } \\
\end{longtable}

\paragraph{Verification Elements}\mbox{}\\

\begin{itemize}
\item \href{https://jira.lsstcorp.org/browse/LVV-8}{LVV-8} - DMS-REQ-0018-V-01: Raw Science Image Data Acquisition

\item \href{https://jira.lsstcorp.org/browse/LVV-9}{LVV-9} - DMS-REQ-0020-V-01: Wavefront Sensor Data Acquisition

\item \href{https://jira.lsstcorp.org/browse/LVV-96}{LVV-96} - DMS-REQ-0265-V-01: Guider Calibration Data Acquisition

\item \href{https://jira.lsstcorp.org/browse/LVV-28}{LVV-28} - DMS-REQ-0068-V-01: Raw Science Image Metadata

\item \href{https://jira.lsstcorp.org/browse/LVV-11}{LVV-11} - DMS-REQ-0024-V-01: Raw Image Assembly

\item \href{https://jira.lsstcorp.org/browse/LVV-146}{LVV-146} - DMS-REQ-0315-V-01: DMS Communication with OCS

\item \href{https://jira.lsstcorp.org/browse/LVV-115}{LVV-115} - DMS-REQ-0284-V-01: Level-1 Production Completeness

\end{itemize}

\paragraph{Test Items}\mbox{}\\

This test will check:\\

\begin{itemize}
\tightlist
\item
  The successful integration of the Pathfinder components with the DM
  Header Service and the Level 1 Archiver;
\item
  That the raw images are well-formed and meet specifications in
  change-controlled documents \citeds{LSE-61};
\end{itemize}

~This Test Case shall be repeated for each of the different cameras
(ATScam, LSSTCam) and sensors (Science, Wavefront, and Guider)
combination.


\paragraph{Predecessors}\mbox{}\\
None.

\paragraph{Environment Needs}\mbox{}\\

\subparagraph{Software}\mbox{}\\
\begin{itemize}
\tightlist
\item
  Level 1 software and services needed to create raw image
\item
  LSST Monitoring Service and plugins specific to monitoring Level 1
  Test Stand and services
\end{itemize}

\subparagraph{Hardware}\mbox{}\\
\begin{itemize}
\tightlist
\item
  Level 1 test stand
\item
  Test machine for LSST Monitoring Service
\end{itemize}

\paragraph{Input Specification}\mbox{}\\
None.

\paragraph{Output Specification}\mbox{}\\
Raw image(s) that follow specifications defined in change-controlled
document \citeds{LSE-61}.

\paragraph{Test Procedure}\mbox{}\\
\begin{tabular}{p{4cm}p{12cm}}
\toprule
Step 1
& Description \\ \hline
\end{tabular}
{\scriptsize
Configure system to pull appropriate data from the DAQ emulator

}
\begin{tabular}{p{3cm}p{13cm}}
\hline
            & Expected Result \\ \hline
\end{tabular}
{\scriptsize
A functional DAQ for images to be received from.~~

}

\begin{tabular}{p{4cm}p{12cm}}
\toprule
Step 2
& Description \\ \hline
\end{tabular}
{\scriptsize
Acquire raw data from DAQ readout and DMHS

}
\begin{tabular}{p{3cm}p{13cm}}
\hline
            & Expected Result \\ \hline
\end{tabular}
{\scriptsize
a raw image and a header from the DMHS~

}

\begin{tabular}{p{4cm}p{12cm}}
\toprule
Step 3
& Description \\ \hline
\end{tabular}
{\scriptsize
Fetch data and reassemble correctly, regardless of CCD/Sensor
manufacturer type (two different types will be used)

}
\begin{tabular}{p{3cm}p{13cm}}
\hline
            & Expected Result \\ \hline
\end{tabular}
{\scriptsize
Build the data into a fits file

}

\begin{tabular}{p{4cm}p{12cm}}
\toprule
Step 4
& Description \\ \hline
\end{tabular}
{\scriptsize
Check completeness and correctness of the raw images including format,
metadata, and image data;

\begin{itemize}
\tightlist
\item
  Check proper fetch and reassembly of image data from camera DAQ
  (correct format and data);
\item
  Check proper merge of header service data with image data;
\item
  Check correct insertion of exposure specific data needed in the data
  file that is not supplied by header service;
\item
  Check minimum required metadata (from requirements document LSE-61)
  exists in raw image header;
\end{itemize}

}
\begin{tabular}{p{3cm}p{13cm}}
\hline
            & Expected Result \\ \hline
\end{tabular}
{\scriptsize
a well formed FITS file with a proper header that has been verified to
be correct.~

}

\begin{tabular}{p{4cm}p{12cm}}
\toprule
Step 5
& Description \\ \hline
\end{tabular}
{\scriptsize
Check that the checksum of the file matches the previously calculated
value that will be passed on to downstream services

}
\begin{tabular}{p{3cm}p{13cm}}
\hline
            & Expected Result \\ \hline
\end{tabular}
{\scriptsize
a MD5sum number generated from the step 4 file.~~

}

\begin{tabular}{p{4cm}p{12cm}}
\toprule
Step 6
& Description \\ \hline
\end{tabular}
{\scriptsize
Check confirmation that the data files arrive at their destination
intact

}
\begin{tabular}{p{3cm}p{13cm}}
\hline
            & Expected Result \\ \hline
\end{tabular}
{\scriptsize
a transfer of the file to the correct location for further retrieval
from other services.~~

}

\begin{tabular}{p{4cm}p{12cm}}
\toprule
Step 7
& Description \\ \hline
\end{tabular}
{\scriptsize
Check that LSST Monitoring Service showed the appropriate information
successfully

}
\begin{tabular}{p{3cm}p{13cm}}
\hline
            & Expected Result \\ \hline
\end{tabular}
{\scriptsize
all systems remained green through out the test, and showed all systems
up and available. ~\\[2\baselineskip]

}

\subsubsection{LVV-T285 - RAS-00-10: Raw images in Observatory Operations Data Service}\label{lvv-t285}

\begin{longtable}[]{llllll}
\toprule
Version & Status & Priority & Verification Type & Owner
\\\midrule
1 & Approved & Normal &
Test & Michelle Butler
\\\bottomrule
\multicolumn{6}{c}{ Open \href{https://jira.lsstcorp.org/secure/Tests.jspa\#/testCase/LVV-T285}{LVV-T285} in Jira } \\
\end{longtable}

\paragraph{Verification Elements}\mbox{}\\

None.

\paragraph{Test Items}\mbox{}\\

This test will check:

\begin{itemize}
\tightlist
\item
  The handoff of a raw image from the Level 1 Archiver to the OODS cache
  manager is successful;
\item
  A recently taken raw image is accessible to the Observatory Operations
  staff at the base and summit;
\end{itemize}

~This Test Case shall be repeated for each of the different cameras
(ATScam, LSSTCam) and sensors (Science, Wavefront, and Guider)
combination.


\paragraph{Predecessors}\mbox{}\\
LVV-T283

\paragraph{Environment Needs}\mbox{}\\

\subparagraph{Software}\mbox{}\\
The following software must be installed:\\[2\baselineskip]

\begin{itemize}
\tightlist
\item
  Level 1 Test Stand (include software from LVV-T283 - RAS-00-00)
\item
  OODS cache manager
\item
  LSST Monitoring Service and plugins specific to monitoring raw images
  and OODS~
\item
  LSST stack for checking raw images
\end{itemize}

\subparagraph{Hardware}\mbox{}\\
To complete all tests in a manner which reflects the real system, the
following hardware is needed. Note: If not testing inter-machine access,
the hardware can be minimized to a single machine outside of the Level 1
Test Stand.

\begin{itemize}
\tightlist
\item
  Level1TestStand(include hardware from LVV-T283 - RAS-00-00)+read/write
  access to OODS cache disk
\item
  Test Machine for OODS cache manager with read/write access to OODS
  cache disk
\item
  Test machine for Observatory Operations staff at ''base'' that can
  access OODS cache disk
\item
  Test machine for Observatory Operations staff at ''summit'' that can
  access OODS cache disk
\item
  Test machine for LSST Monitoring Service
\end{itemize}

Size of cache disk is determined by number of files to be included in
the test.

\paragraph{Input Specification}\mbox{}\\


\paragraph{Output Specification}\mbox{}\\
Raw image(s) that follow format defined in \citeds{LSE-61};\\
Database (may be SQLite file) that enables the raw image(s) to be
accessed via a ``Data Butler''.

\paragraph{Test Procedure}\mbox{}\\
\begin{tabular}{p{4cm}p{12cm}}
\toprule
Step 1
& Description \\ \hline
\end{tabular}
{\scriptsize
Initialize all services configuring the Level 1 Archiver Service so that
the raw images are to be saved to the OODS

}
\begin{tabular}{p{3cm}p{13cm}}
\hline
            & Expected Result \\ \hline
\end{tabular}
{\scriptsize
all camera and services for images are running and reporting green
through the monitoring programs for the services. ~\\[2\baselineskip]

}

\begin{tabular}{p{4cm}p{12cm}}
\toprule
Step 2
& Description \\ \hline
\end{tabular}
{\scriptsize
Acquire a raw image

}
\begin{tabular}{p{3cm}p{13cm}}
\hline
            & Expected Result \\ \hline
\end{tabular}
{\scriptsize
Image present in the input folder.

}

\begin{tabular}{p{4cm}p{12cm}}
\toprule
Step 3
& Description \\ \hline
\end{tabular}
{\scriptsize
\emph{The handoff of the raw image from the Level 1 Archiver Service to
the test OODS automatically occurs\\
}

}
\begin{tabular}{p{3cm}p{13cm}}
\hline
            & Expected Result \\ \hline
\end{tabular}
{\scriptsize
the raw image with a proper header is written to a file area managed by
the OODS\\[2\baselineskip]

}

\begin{tabular}{p{4cm}p{12cm}}
\toprule
Step 4
& Description \\ \hline
\end{tabular}
{\scriptsize
For each of the expected raw images, verify that the checksum matches
the original Level 1 checksum

}
\begin{tabular}{p{3cm}p{13cm}}
\hline
            & Expected Result \\ \hline
\end{tabular}
{\scriptsize
checksum of the file is checked against the file for verification that
the OODS has the correct file and it matches the original md5sum of the
FITS file.\\[2\baselineskip]

}

\begin{tabular}{p{4cm}p{12cm}}
\toprule
Step 5
& Description \\ \hline
\end{tabular}
{\scriptsize
Check that LSST Monitoring Service showed the appropriate information
successfully

}
\begin{tabular}{p{3cm}p{13cm}}
\hline
            & Expected Result \\ \hline
\end{tabular}
{\scriptsize
Make sure all camera and OODS systems were available thorughout this
test.~ ~

}

\subsubsection{LVV-T286 - RAS-00-20: Raw image are part of the permanent record of survey via DBB}\label{lvv-t286}

\begin{longtable}[]{llllll}
\toprule
Version & Status & Priority & Verification Type & Owner
\\\midrule
1 & Approved & Normal &
Test & Michelle Butler
\\\bottomrule
\multicolumn{6}{c}{ Open \href{https://jira.lsstcorp.org/secure/Tests.jspa\#/testCase/LVV-T286}{LVV-T286} in Jira } \\
\end{longtable}

\paragraph{Verification Elements}\mbox{}\\

\begin{itemize}
\item \href{https://jira.lsstcorp.org/browse/LVV-28}{LVV-28} - DMS-REQ-0068-V-01: Raw Science Image Metadata

\item \href{https://jira.lsstcorp.org/browse/LVV-177}{LVV-177} - DMS-REQ-0346-V-01: Data Availability

\item \href{https://jira.lsstcorp.org/browse/LVV-115}{LVV-115} - DMS-REQ-0284-V-01: Level-1 Production Completeness

\end{itemize}

\paragraph{Test Items}\mbox{}\\

This test will check:\\[2\baselineskip]

\begin{itemize}
\tightlist
\item
  That the handoff of a raw image from the Level 1 Archiver Service to
  the DBB buffer manager is successful;
\item
  That the raw image is ingested into the Data Backbone successfully;
\item
  That the monitoring of the above items is successful;
\end{itemize}

This Test Case shall be repeated for each of the different cameras
(ATScam, LSSTCam) and sensors (Science, Wavefront, and Guider)
combination.\\[2\baselineskip]Note: For a complete check of the various
aspects of what it means for a raw image to be in the Data Backbone, see
the tests for the Data Backbone.


\paragraph{Predecessors}\mbox{}\\
LVV-T283

\paragraph{Environment Needs}\mbox{}\\

\subparagraph{Software}\mbox{}\\
\begin{itemize}
\tightlist
\item
  Level 1 Test Stand
\item
  DBB buffer manager
\item
  DBB raw image ingestion
\item
  DBB database
\item
  LSST Monitoring Service and plugins specific to monitoring raw images,
  DBB buffer manager, and DBB
\end{itemize}

\subparagraph{Hardware}\mbox{}\\
\begin{itemize}
\tightlist
\item
  Level 1 Test Stand (include hardware from LVV-T-283 - RAS-00-00) +
  read/write access to DBB buffer disk;
\item
  Test Machine for DBB buffer manager with read/write access to DBB
  buffer disk;
\item
  Test machine for each DBB endpoint with read/write access to DBB disk;
\item
  Test machine for LSST Monitoring Service
\end{itemize}

Size of buffer disk and DBB disk is determined by number of files to be
included in the test.\\[2\baselineskip]Note: If not testing
inter-machine operability, then the hardware can be minimized to a
single machine outside of the Level 1 test stand.

\paragraph{Input Specification}\mbox{}\\
​​​​​None

\paragraph{Output Specification}\mbox{}\\
\begin{itemize}
\tightlist
\item
  Raw image(s) are saved to storage and replicated to correct locations
  with checksums that match original Level 1 checksum;
\item
  Database containing information of the following types: physical,
  location, science metadata, provenance as specified in \citeds{LSE-61};
\item
  Both image(s) and database entries replicated correctly;
\end{itemize}

\paragraph{Test Procedure}\mbox{}\\
\begin{tabular}{p{4cm}p{12cm}}
\toprule
Step 1
& Description \\ \hline
\end{tabular}
{\scriptsize
Initialize all services configuring the Level 1 Archiver Service so that
the raw images are to be archived to the DBB

}
\begin{tabular}{p{3cm}p{13cm}}
\hline
            & Expected Result \\ \hline
\end{tabular}
{\scriptsize
all services for the camera images and the DBB services are all running
and ready for data.~~

}

\begin{tabular}{p{4cm}p{12cm}}
\toprule
Step 2
& Description \\ \hline
\end{tabular}
{\scriptsize
Acquire a raw image (see LVV-T283 - RAS-00-00)\\

}
\begin{tabular}{p{3cm}p{13cm}}
\hline
            & Expected Result \\ \hline
\end{tabular}
{\scriptsize
have a raw Fits file with proper header.~~

}

\begin{tabular}{p{4cm}p{12cm}}
\toprule
Step 3
& Description \\ \hline
\end{tabular}
{\scriptsize
After the automatic handoff of the raw image between the Level 1
Archiver Service and the DBB buffer manager, the raw image will
automatically be ingested into the Data Backbone

}
\begin{tabular}{p{3cm}p{13cm}}
\hline
            & Expected Result \\ \hline
\end{tabular}
{\scriptsize
the DBB file systems will have the file, and metadata and providence
will be recorded in the consolidated DB. ~ The file will also be
replicated to mulitple locations for DR.~~

}

\begin{tabular}{p{4cm}p{12cm}}
\toprule
Step 4
& Description \\ \hline
\end{tabular}
{\scriptsize
Check that the raw image is accessible at each ~DBB endpoint and matches
original Level 1 checksum

}
\begin{tabular}{p{3cm}p{13cm}}
\hline
            & Expected Result \\ \hline
\end{tabular}
{\scriptsize
data resides at NCSA DBB end point, and Chile end point and match with
the same checksum.~~

}

\begin{tabular}{p{4cm}p{12cm}}
\toprule
Step 5
& Description \\ \hline
\end{tabular}
{\scriptsize
Check that LSST Monitoring Service showed the appropriate information
successfully

}
\begin{tabular}{p{3cm}p{13cm}}
\hline
            & Expected Result \\ \hline
\end{tabular}
{\scriptsize
all related systems remained up during this test.~~

}

\begin{tabular}{p{4cm}p{12cm}}
\toprule
Step 6
& Description \\ \hline
\end{tabular}
{\scriptsize
More complete tests of the DBB can be done by running the DBB service
tests on the raw image(s). These would check correctness and
completeness of the data stored in the database as well as checking that
the file has been replicated to all required places

}
\begin{tabular}{p{3cm}p{13cm}}
\hline
            & Expected Result \\ \hline
\end{tabular}
{\scriptsize
These would be more tests of when things go wrong to make sure that the
DBB is able to continue to work, and not be in the way of taking images
from the camera\\[2\baselineskip]

}

\subsubsection{LVV-T287 - RAS-00-30: Raw Image Archiving Availability, Throughput, Reliability,
and Heterogeneity}\label{lvv-t287}

\begin{longtable}[]{llllll}
\toprule
Version & Status & Priority & Verification Type & Owner
\\\midrule
1 & Approved & Normal &
Test & Michelle Butler
\\\bottomrule
\multicolumn{6}{c}{ Open \href{https://jira.lsstcorp.org/secure/Tests.jspa\#/testCase/LVV-T287}{LVV-T287} in Jira } \\
\end{longtable}

\paragraph{Verification Elements}\mbox{}\\

\begin{itemize}
\item \href{https://jira.lsstcorp.org/browse/LVV-5}{LVV-5} - DMS-REQ-0008-V-01: Pipeline Availability

\item \href{https://jira.lsstcorp.org/browse/LVV-65}{LVV-65} - DMS-REQ-0162-V-01: Pipeline Throughput

\item \href{https://jira.lsstcorp.org/browse/LVV-68}{LVV-68} - DMS-REQ-0165-V-01: Infrastructure Sizing for ``catching up''

\item \href{https://jira.lsstcorp.org/browse/LVV-70}{LVV-70} - DMS-REQ-0167-V-01: Incorporate Autonomics

\item \href{https://jira.lsstcorp.org/browse/LVV-145}{LVV-145} - DMS-REQ-0314-V-01: Compute Platform Heterogeneity

\item \href{https://jira.lsstcorp.org/browse/LVV-149}{LVV-149} - DMS-REQ-0318-V-01: Data Management Unscheduled Downtime

\item \href{https://jira.lsstcorp.org/browse/LVV-140}{LVV-140} - DMS-REQ-0309-V-01: Raw Data Archiving Reliability

\end{itemize}

\paragraph{Test Items}\mbox{}\\

This test will check:\\[2\baselineskip]

\begin{itemize}
\tightlist
\item
  Raw Image Archiving meets availability requirements;
\item
  Raw Image Archiving meets throughput requirements;
\item
  Raw Image Archiving meets reliability requirements;
\item
  Raw Image Archiving meets heterogeneity requirements;
\end{itemize}

This test case need to be completed when more information is available.








\paragraph{Test Procedure}\mbox{}\\
\begin{tabular}{p{4cm}p{12cm}}
\toprule
Step 1
& Description \\ \hline
\end{tabular}
{\scriptsize
these will be filled out as the service becomes more known as to what
the availablility, throughput, reliability and heterogeneity are.
~\\[2\baselineskip]

}
\begin{tabular}{p{3cm}p{13cm}}
\hline
            & Expected Result \\ \hline
\end{tabular}
{\scriptsize
The archive system will stay up through thick and thin and perform like
it's suppose to.\\[2\baselineskip]

}

\subsubsection{LVV-T362 - Installation of the LSST Science Pipelines Payloads}\label{lvv-t362}

\begin{longtable}[]{llllll}
\toprule
Version & Status & Priority & Verification Type & Owner
\\\midrule
1 & Approved & Normal &
Test & John Swinbank
\\\bottomrule
\multicolumn{6}{c}{ Open \href{https://jira.lsstcorp.org/secure/Tests.jspa\#/testCase/LVV-T362}{LVV-T362} in Jira } \\
\end{longtable}

\paragraph{Verification Elements}\mbox{}\\

\begin{itemize}
\item \href{https://jira.lsstcorp.org/browse/LVV-29}{LVV-29} - DMS-REQ-0069-V-01: Processed Visit Images

\item \href{https://jira.lsstcorp.org/browse/LVV-98}{LVV-98} - DMS-REQ-0267-V-01: Source Catalog

\item \href{https://jira.lsstcorp.org/browse/LVV-139}{LVV-139} - DMS-REQ-0308-V-01: Software Architecture to Enable Community Re-Use

\item \href{https://jira.lsstcorp.org/browse/LVV-127}{LVV-127} - DMS-REQ-0296-V-01: Pre-cursor, and Real Data

\item \href{https://jira.lsstcorp.org/browse/LVV-15}{LVV-15} - DMS-REQ-0033-V-01: Provide Source Detection Software

\end{itemize}

\paragraph{Test Items}\mbox{}\\

This test will check that:

\begin{itemize}
\tightlist
\item
  The Alert Production Pipeline payload is available for installation
  from documented channels;
\item
  The Data Release Production Pipeline payload is available for
  installation from documented channels;
\item
  The Calibration Products Production Pipeline payload is available for
  installation from documented channels;
\item
  These payloads can be installed on systems at the LSST Data Facility
  following available documentation;
\item
  The installed pipeline payloads are capable of successfully executing
  basic integration tests.
\end{itemize}

Note that this test assumes a 2018-era packaging of the Science
Pipelines software, in which all the above payloads are represented by a
single ``meta-package'', lsst\_distrib.



\paragraph{Environment Needs}\mbox{}\\

\subparagraph{Software}\mbox{}\\
Science Pipelines prerequisite software, as documented at
https://pipelines.lsst.io/, must be installed on the target system.

\subparagraph{Hardware}\mbox{}\\
This test requires a workstation or equivalent system running an
operating system supported by the LSST Science Pipelines.



\paragraph{Test Procedure}\mbox{}\\
\begin{tabular}{p{4cm}p{12cm}}
\toprule
Step 1
& Description \\ \hline
\end{tabular}
{\scriptsize
The LSST Science Pipelines, described by the lsst\_distrib meta-package,
should be installed following the documentation available at
https://pipelines.lsst.io/. The suggested Conda environment will be used
to ensure that a supported execution environment is available.

}
\begin{tabular}{p{3cm}p{13cm}}
\hline
            & Expected Result \\ \hline
\end{tabular}
{\scriptsize
Detailed output will depend on the installation method chosen, but will
confirm the successful installation of the Science Pipelines.

}

\begin{tabular}{p{4cm}p{12cm}}
\toprule
Step 2
& Description \\ \hline
\end{tabular}
{\scriptsize
The lsst\_distrib top-level metapackage will be enabled. Assuming that
the software has been installed at
\$\{LSST\_DIR\}:\\[2\baselineskip]\hspace*{0.333em} ~ ~ ~source
\$\{LSST\_DIR\}/loadLSST.bash\\
\hspace*{0.333em} ~ ~ ~setup lsst\_distrib

}
\begin{tabular}{p{3cm}p{13cm}}
\hline
            & Expected Result \\ \hline
\end{tabular}
{\scriptsize
Nothing is printed. The command\\[2\baselineskip]\hspace*{0.333em} ~eups
list -s lsst\_distrib\\[2\baselineskip]may be used to confirm that the
correct version of the codebase has been installed.

}

\begin{tabular}{p{4cm}p{12cm}}
\toprule
Step 3
& Description \\ \hline
\end{tabular}
{\scriptsize
The ``LSST Stack Demo'' package will be downloaded onto the test system
from https://github.com/lsst/lsst\_dm\_stack\_demo/releases. The version
corresponding to to the version of the Science Pipelines under test
should be chosen.

}
\begin{tabular}{p{3cm}p{13cm}}
\hline
            & Expected Result \\ \hline
\end{tabular}
{\scriptsize
Depends on the tool selected by the user for downloading.

}

\begin{tabular}{p{4cm}p{12cm}}
\toprule
Step 4
& Description \\ \hline
\end{tabular}
{\scriptsize
The stack demo package is uncompressed into a directory \$\{DEMO\_DIR\}.

}
\begin{tabular}{p{3cm}p{13cm}}
\hline
            & Expected Result \\ \hline
\end{tabular}
{\scriptsize
Depends on options given to the tar command. Should confirm the
availability of the stack demo source.

}

\begin{tabular}{p{4cm}p{12cm}}
\toprule
Step 5
& Description \\ \hline
\end{tabular}
{\scriptsize
The demo package will be executed by following the instructions in its
README file.~

}
\begin{tabular}{p{3cm}p{13cm}}
\hline
            & Expected Result \\ \hline
\end{tabular}
{\scriptsize
Successful execution will result in the string ``Ok'' being returned.

}

\subsubsection{LVV-T363 - Science Pipelines Release Documentation}\label{lvv-t363}

\begin{longtable}[]{llllll}
\toprule
Version & Status & Priority & Verification Type & Owner
\\\midrule
1 & Approved & Normal &
Inspection & John Swinbank
\\\bottomrule
\multicolumn{6}{c}{ Open \href{https://jira.lsstcorp.org/secure/Tests.jspa\#/testCase/LVV-T363}{LVV-T363} in Jira } \\
\end{longtable}

\paragraph{Verification Elements}\mbox{}\\

\begin{itemize}
\item \href{https://jira.lsstcorp.org/browse/LVV-139}{LVV-139} - DMS-REQ-0308-V-01: Software Architecture to Enable Community Re-Use

\item \href{https://jira.lsstcorp.org/browse/LVV-3402}{LVV-3402} - DMS-REQ-0360-V-01: Median astrometric error on 20 arcmin scales

\end{itemize}

\paragraph{Test Items}\mbox{}\\

This test will check:

\begin{itemize}
\tightlist
\item
  That a particular Science Pipelines release is adequately described by
  documentation at the https://pipelines.lsst.io/ site;
\item
  That the Science Pipelines release is accompanied by a
  characterization report which describes its scientific performance.
\end{itemize}



\paragraph{Environment Needs}\mbox{}\\

\subparagraph{Software}\mbox{}\\
A web browser.

\subparagraph{Hardware}\mbox{}\\
A device with internet access.



\paragraph{Test Procedure}\mbox{}\\
\begin{tabular}{p{4cm}p{12cm}}
\toprule
Step 1
& Description \\ \hline
\end{tabular}
{\scriptsize
Load the Science Pipelines website at https://pipelines.lsst.io/.

}
\begin{tabular}{p{3cm}p{13cm}}
\hline
            & Expected Result \\ \hline
\end{tabular}
{\scriptsize
The website is displayed.

}

\begin{tabular}{p{4cm}p{12cm}}
\toprule
Step 2
& Description \\ \hline
\end{tabular}
{\scriptsize
Identify documentation for the release under test. This should be
clearly labelled on the documentation site.\\[2\baselineskip]If the
latest release is being tested, the default page loaded when visiting
https://pipelines.lsst.io/ should be the documentation
required.\\[2\baselineskip]If this test is for another release, the site
should present clear instructions for changing the edition (or version)
of the documentation being examined, and documentation for the release
under test should be available.

}
\begin{tabular}{p{3cm}p{13cm}}
\hline
            & Expected Result \\ \hline
\end{tabular}
{\scriptsize
The documentation for the release under test is displayed.

}

\begin{tabular}{p{4cm}p{12cm}}
\toprule
Step 3
& Description \\ \hline
\end{tabular}
{\scriptsize
Inspect the documentation to ensure that it refers to the release under
test, and that it provides:

\begin{itemize}
\tightlist
\item
  Release notes, describing changes in this release relative to the
  previous;
\item
  Installation instructions, together with a list of supported platforms
  and prerequisites;
\item
  Getting started information.
\end{itemize}

}
\begin{tabular}{p{3cm}p{13cm}}
\hline
            & Expected Result \\ \hline
\end{tabular}
{\scriptsize
The user is satisfied that the required information is available.

}

\begin{tabular}{p{4cm}p{12cm}}
\toprule
Step 4
& Description \\ \hline
\end{tabular}
{\scriptsize
Locate the Characterization Metric Report corresponding to this release.
It should be linked from the main release documentation.

}
\begin{tabular}{p{3cm}p{13cm}}
\hline
            & Expected Result \\ \hline
\end{tabular}
{\scriptsize
The user is satisfied that the report is available.

}

\begin{tabular}{p{4cm}p{12cm}}
\toprule
Step 5
& Description \\ \hline
\end{tabular}
{\scriptsize
Verify that the characterization metric report describes the scientific
performance of the release in terms of a selection of performance
metrics drawn from high-level requirements documentation (the Science
Requirements Document, LPM-17; the LSST System Requirements, LSE-29;
and/or the Observatory System Specifications, LSE-30).

}
\begin{tabular}{p{3cm}p{13cm}}
\hline
            & Expected Result \\ \hline
\end{tabular}
{\scriptsize
Metric values describing the performance of the release, for example as
computed by validate\_drp, are described in the report.

}

\subsubsection{LVV-T368 - Loading and processing Camera test data}\label{lvv-t368}

\begin{longtable}[]{llllll}
\toprule
Version & Status & Priority & Verification Type & Owner
\\\midrule
2 & Approved & Normal &
Test & John Swinbank
\\\bottomrule
\multicolumn{6}{c}{ Open \href{https://jira.lsstcorp.org/secure/Tests.jspa\#/testCase/LVV-T368}{LVV-T368} in Jira } \\
\end{longtable}

\paragraph{Verification Elements}\mbox{}\\

\begin{itemize}
\item \href{https://jira.lsstcorp.org/browse/LVV-129}{LVV-129} - DMS-REQ-0298-V-01: Data Product and Raw Data Access

\item \href{https://jira.lsstcorp.org/browse/LVV-63}{LVV-63} - DMS-REQ-0160-V-01: Provide User Interface Services

\item \href{https://jira.lsstcorp.org/browse/LVV-23}{LVV-23} - DMS-REQ-0060-V-01: Bias Residual Image

\end{itemize}

\paragraph{Test Items}\mbox{}\\

This test will check:

\begin{itemize}
\tightlist
\item
  That Camera test data is available for processing in the LSST Data
  Facility, and accessible through the LSST Science Platform;
\item
  That the Data Management I/O abstraction (the ``Data Butler'') can
  load that data into the Science Platform environment;
\item
  That Data Management algorithmic ``tasks'' can be executed to process
  that data;
\item
  That results can be displayed in the Firefly display tool.
\end{itemize}


\paragraph{Predecessors}\mbox{}\\
Executing LVV-T374 will satisfy the preconditions for this test,
assuming that \$REPOSITORY\_PATH is set equal to the output location
used in LVV-T374.

\paragraph{Environment Needs}\mbox{}\\

\subparagraph{Software}\mbox{}\\
The LSST Science Pipelines version w\_2018\_45 must be available within
the Notebook Aspect of the LSST Science Platform.

\subparagraph{Hardware}\mbox{}\\
This test assumes the availability of the Notebook and Portal aspects of
the LSST Science Platform, deployed at
https://lsst-lspdev.ncsa.illinois.edu.



\paragraph{Test Procedure}\mbox{}\\
\begin{tabular}{p{4cm}p{12cm}}
\toprule
Step 1
& Description \\ \hline
\end{tabular}
{\scriptsize
Connect to the Notebook Aspect of the Science Platform following the
instructions at https://nb.lsst.io/. Log in, and ``spawn'' a new machine
with image ``Weekly 2018\_45`` and size ``small''.

}
\begin{tabular}{p{3cm}p{13cm}}
\hline
            & Expected Result \\ \hline
\end{tabular}
{\scriptsize
The JupyterLab environment appears.

}

\begin{tabular}{p{4cm}p{12cm}}
\toprule
Step 2
& Description \\ \hline
\end{tabular}
{\scriptsize
Create a terminal session. Use it to set up the LSST tools, then
download and build version 5c12b06e6 of
obs\_lsst:\\[2\baselineskip]\hspace*{0.333em} ~\$ source
/opt/lsst/software/stack/loadLSST.bash\\
\hspace*{0.333em} ~\$ setup lsst\_distrib\\
\hspace*{0.333em} ~\$ git clone https://github.com/lsst/obs\_lsst.git\\
\hspace*{0.333em} ~\$ cd obs\_lsst\\
\hspace*{0.333em} ~\$ git checkout 5c12b06e6\\
\hspace*{0.333em} ~\$ setup -k -r .\\
\hspace*{0.333em} ~\$ scons\\[2\baselineskip]Arrange for obs\_lsst to
automatically be added to the environment when starting a new
notebook:\\[2\baselineskip]\hspace*{0.333em} ~\$ echo ``setup -j -r
\textasciitilde{}/obs\_lsst'' \textgreater{}\textgreater{}
\textasciitilde{}/notebooks/.user\_setups\\[2\baselineskip]Exit the
terminal.

}
\begin{tabular}{p{3cm}p{13cm}}
\hline
            & Expected Result \\ \hline
\end{tabular}
{\scriptsize
No errors are seen during execution of the provided commands.

}

\begin{tabular}{p{4cm}p{12cm}}
\toprule
Step 3
& Description \\ \hline
\end{tabular}
{\scriptsize
Create a new ``LSST'' notebook.\\[2\baselineskip]Import the standard
libraries required for the rest of this
test:\\[2\baselineskip]\hspace*{0.333em} ~import os\\
\hspace*{0.333em} ~import lsst.afw.display as afwDisplay\\
\hspace*{0.333em} ~from lsst.daf.persistence import Butler\\
\hspace*{0.333em} ~from lsst.ip.isr import IsrTask\\
\hspace*{0.333em} ~from firefly\_client import FireflyClient\\
\hspace*{0.333em} ~from IPython.display import
IFrame\\[2\baselineskip]and execute the cell.

}
\begin{tabular}{p{3cm}p{13cm}}
\hline
            & Expected Result \\ \hline
\end{tabular}
{\scriptsize
Nothing is printed.

}

\begin{tabular}{p{4cm}p{12cm}}
\toprule
Step 4
& Description \\ \hline
\end{tabular}
{\scriptsize
Create a Data Butler client, and use it to retrieve the data which will
be used for this test.\\[2\baselineskip]\hspace*{0.333em} ~butler =
Butler(\$REPOSITORY\_PATH)\\
\hspace*{0.333em} ~raw = butler.get(``raw'', visit=\$VISIT\_ID,
detector=2)\\
\hspace*{0.333em} ~bias = butler.get(``bias'', visit=\$VISIT\_ID,
detector=2)

}
\begin{tabular}{p{3cm}p{13cm}}
\hline
            & Expected Result \\ \hline
\end{tabular}
{\scriptsize
Nothing is printed.

}

\begin{tabular}{p{4cm}p{12cm}}
\toprule
Step 5
& Description \\ \hline
\end{tabular}
{\scriptsize
Initialize the Firefly display
system:\\[2\baselineskip]\hspace*{0.333em} ~my\_channel =
`\{\}\_test\_channel'.format(os.environ{[}'USER'{]})\\
\hspace*{0.333em} ~server = `https://lsst-lspdev.ncsa.illinois.edu'\\
\hspace*{0.333em}
~ff='\{\}/firefly/slate.html?\_\_wsch=\{\}'.format(server,
my\_channel)\\
\hspace*{0.333em} ~IFrame(ff,800,600)\\
\hspace*{0.333em} ~afwDisplay.setDefaultBackend('firefly')\\
\hspace*{0.333em} ~afw\_display = afwDisplay.getDisplay(frame=1,\\
\hspace*{0.333em} ~ ~ ~ ~ ~ ~ ~ ~ ~ ~ ~ ~ ~ ~ ~ ~ ~ ~
~name=my\_channel)\\[2\baselineskip]Click on the link provided after
executing the above.

}
\begin{tabular}{p{3cm}p{13cm}}
\hline
            & Expected Result \\ \hline
\end{tabular}
{\scriptsize
A Firefly window is shown.

}

\begin{tabular}{p{4cm}p{12cm}}
\toprule
Step 6
& Description \\ \hline
\end{tabular}
{\scriptsize
Display the raw image data in the Firefly
window:\\[2\baselineskip]\hspace*{0.333em} afw\_display.mtv(raw)

}
\begin{tabular}{p{3cm}p{13cm}}
\hline
            & Expected Result \\ \hline
\end{tabular}
{\scriptsize
Raw image data is displayed.

}

\begin{tabular}{p{4cm}p{12cm}}
\toprule
Step 7
& Description \\ \hline
\end{tabular}
{\scriptsize
Configure and run an Instrument Signature Removal (ISR) task on the raw
data. Most corrections are disabled for simplicity. but the bias frame
is applied.\\
\hspace*{0.333em}

\begin{verbatim}
   isr_config = IsrTask.ConfigClass()
   isr_config.doDark=False
   isr_config.doFlat=False
   isr_config.doFringe=False
   isr_config.doDefect=False
   isr_config.doAddDistortionModel=False
   isr_config.doLinearize=False
   isr = IsrTask(config=isr_config)
   result = isr.run(raw, bias=bias)
\end{verbatim}

}
\begin{tabular}{p{3cm}p{13cm}}
\hline
            & Expected Result \\ \hline
\end{tabular}
{\scriptsize
Nothing is printed.

}

\begin{tabular}{p{4cm}p{12cm}}
\toprule
Step 8
& Description \\ \hline
\end{tabular}
{\scriptsize
Display the corrected image data in the Firefly
window:\\[2\baselineskip]\hspace*{0.333em}
~afw\_display.mtv(result.exposure)

}
\begin{tabular}{p{3cm}p{13cm}}
\hline
            & Expected Result \\ \hline
\end{tabular}
{\scriptsize
Processed (trimmed, bias-subtracted) image data is displayed.

}

\subsubsection{LVV-T374 - Ingesting Camera test data}\label{lvv-t374}

\begin{longtable}[]{llllll}
\toprule
Version & Status & Priority & Verification Type & Owner
\\\midrule
1 & Approved & Normal &
Test & John Swinbank
\\\bottomrule
\multicolumn{6}{c}{ Open \href{https://jira.lsstcorp.org/secure/Tests.jspa\#/testCase/LVV-T374}{LVV-T374} in Jira } \\
\end{longtable}

\paragraph{Verification Elements}\mbox{}\\

\begin{itemize}
\item \href{https://jira.lsstcorp.org/browse/LVV-130}{LVV-130} - DMS-REQ-0299-V-01: Data Product Ingest

\item \href{https://jira.lsstcorp.org/browse/LVV-129}{LVV-129} - DMS-REQ-0298-V-01: Data Product and Raw Data Access

\end{itemize}

\paragraph{Test Items}\mbox{}\\

This test will check:

\begin{itemize}
\tightlist
\item
  That raw Camera test data is available on a filesystem in the LSST
  Data Facility;
\item
  That raw Camera test data can be ingested and made available through
  the Data Management I/O abstraction (the ``Data Butler'').
\end{itemize}



\paragraph{Environment Needs}\mbox{}\\

\subparagraph{Software}\mbox{}\\
The LSST Science Pipelines version w\_2018\_45 must be available within
the Notebook Aspect of the LSST Science Platform.

\subparagraph{Hardware}\mbox{}\\
This test assumes the availability of the Notebook aspect of the LSST
Science Platform, deployed at https://lsst-lspdev.ncsa.illinois.edu.



\paragraph{Test Procedure}\mbox{}\\
\begin{tabular}{p{4cm}p{12cm}}
\toprule
Step 1
& Description \\ \hline
\end{tabular}
{\scriptsize
Connect to the Notebook Aspect of the Science Platform following the
instructions at https://nb.lsst.io/. Log in, and ``spawn'' a new machine
with image ``Weekly 2018\_45`` and size ``large''.

}
\begin{tabular}{p{3cm}p{13cm}}
\hline
            & Expected Result \\ \hline
\end{tabular}
{\scriptsize
The JupyterLab environment appears.

}

\begin{tabular}{p{4cm}p{12cm}}
\toprule
Step 2
& Description \\ \hline
\end{tabular}
{\scriptsize
Create a terminal session. Use it to set up the LSST tools, then
download and build version 5c12b06e6 of
obs\_lsst:\\[2\baselineskip]\hspace*{0.333em} ~\$ source
/opt/lsst/software/stack/loadLSST.bash\\
\hspace*{0.333em} ~\$ setup lsst\_distrib\\
\hspace*{0.333em} ~\$ git clone https://github.com/lsst/obs\_lsst.git\\
\hspace*{0.333em} ~\$ cd obs\_lsst\\
\hspace*{0.333em} ~\$ git checkout 5c12b06e6\\
\hspace*{0.333em} ~\$ setup -k -r .\\
\hspace*{0.333em} ~\$ scons

}
\begin{tabular}{p{3cm}p{13cm}}
\hline
            & Expected Result \\ \hline
\end{tabular}
{\scriptsize
No errors are seen during execution of the provided commands.

}

\begin{tabular}{p{4cm}p{12cm}}
\toprule
Step 3
& Description \\ \hline
\end{tabular}
{\scriptsize
Ingest RTM-007 test data by executing the following
commands:\\[2\baselineskip]\hspace*{0.333em}
~OUTPUT\_REPO\_DIR=\$OUTPUT\_DATA\_DIR\\
\hspace*{0.333em} ~INPUT\_DATA\_DIR=\$INPUT\_DATA\_DIR\\
\hspace*{0.333em} ~mkdir -p \$OUTPUT\_REPO\_DIR\\
\hspace*{0.333em} ~echo ``lsst.obs.lsst.ts8.Ts8Mapper'' \textgreater{}
\$OUTPUT\_REPO\_DIR/\_mapper\\
\hspace*{0.333em} ~ingestImages.py \$OUTPUT\_REPO\_DIR
\$INPUT\_DATA\_DIR/*/*.fits\\
\hspace*{0.333em} ~constructBias.py \$OUTPUT\_REPO\_DIR --rerun calibs
--id imageType=BIAS --batch-type smp --cores 4\\
\hspace*{0.333em} ~ingestCalibs.py \$OUTPUT\_REPO\_DIR --calibType bias
\$OUTPUT\_REPO\_DIR/rerun/calibs/bias/*/*.fits --validity 9999 --output
\$OUTPUT\_REPO\_DIR/CALIB
--mode=link\\[2\baselineskip]Where:\\[2\baselineskip]\hspace*{0.333em}
~\$OUTPUT\_DATA\_DIR is some location on shared storage to which the
user has write permission;\\
\hspace*{0.333em} ~\$INPUT\_DATA\_DIR is defined in the test case
description.

}
\begin{tabular}{p{3cm}p{13cm}}
\hline
            & Expected Result \\ \hline
\end{tabular}
{\scriptsize
Many status messages are logged to screen, and the command exits with
status 0.

}

\begin{tabular}{p{4cm}p{12cm}}
\toprule
Step 4
& Description \\ \hline
\end{tabular}
{\scriptsize
Demonstrate that raw and bias data for visit \$VISIT\_ID have been made
available in the repository. Load a Python interpreter (run
â\euro{}œpython") and execute the
following:\\[2\baselineskip]\hspace*{0.333em} ~from lsst.daf.persistence
import Butler\\
\hspace*{0.333em} ~visit\_id = \$VISIT\_ID\\
\hspace*{0.333em} ~b = Butler(\$OUTPUT\_DATA\_DIR)\\
\hspace*{0.333em} ~b.get(``raw'', visit=visit\_id, detector=2)\\
\hspace*{0.333em} ~b.get(``bias'', visit=visit\_id, detector=2)

}
\begin{tabular}{p{3cm}p{13cm}}
\hline
            & Expected Result \\ \hline
\end{tabular}
{\scriptsize
Each call to b.get() returns an instance of an ExposureF object.
Warnings about lack of dark-time or WCS information may be ignored.

}

\subsubsection{LVV-T376 - Verify the Calculation of Ellipticity Residuals and Correlations}\label{lvv-t376}

\begin{longtable}[]{llllll}
\toprule
Version & Status & Priority & Verification Type & Owner
\\\midrule
1 & Approved & Normal &
Test & Leanne Guy
\\\bottomrule
\multicolumn{6}{c}{ Open \href{https://jira.lsstcorp.org/secure/Tests.jspa\#/testCase/LVV-T376}{LVV-T376} in Jira } \\
\end{longtable}

\paragraph{Verification Elements}\mbox{}\\

\begin{itemize}
\item \href{https://jira.lsstcorp.org/browse/LVV-3404}{LVV-3404} - DMS-REQ-0362-V-01: Median residual PSF ellipticity correlations on 5
arcmin scales

\item \href{https://jira.lsstcorp.org/browse/LVV-9780}{LVV-9780} - DMS-REQ-0362-V-02: Max fraction of excess ellipticity residuals on 1 and
5 arcmin scales

\end{itemize}

\paragraph{Test Items}\mbox{}\\

Verify that the DMS includes software to enable the calculation of the
ellipticity residuals and correlation metrics defined in the OSS.~








\paragraph{Test Procedure}\mbox{}\\
\begin{tabular}{p{4cm}p{12cm}}
\toprule
Step 1-1
{\scriptsize from \hyperref[lvv-t987]{LVV-T987} }
& Description \\ \hline
\end{tabular}
{\scriptsize
Identify the path to the data repository, which we will refer to as
`DATA/path', then execute the following:

}
\begin{tabular}{p{3cm}p{13cm}}
\hline
            & Example Code \\ \hline
\end{tabular}
{\scriptsize
\begin{verbatim}
import lsst.daf.persistence as dafPersist
butler = dafPersist.Butler(inputs='DATA/path')
\end{verbatim}

}
\begin{tabular}{p{3cm}p{13cm}}
\hline
            & Expected Result \\ \hline
\end{tabular}
{\scriptsize
Butler repo available for reading.

}

\begin{tabular}{p{4cm}p{12cm}}
\toprule
Step 2
& Description \\ \hline
\end{tabular}
{\scriptsize
Point the butler to an appropriate (precursor or simulated) dataset
containing data in all filters, that is sufficient for the purposes of
measuring astrometric performance metrics.

}
\begin{tabular}{p{3cm}p{13cm}}
\hline
            & Expected Result \\ \hline
\end{tabular}

\begin{tabular}{p{4cm}p{12cm}}
\toprule
Step 3
& Description \\ \hline
\end{tabular}
{\scriptsize
Execute the LSST Stack package `validate\_drp` (or an alternate package
that is relevant) on this dataset to perform the measurements of the
metrics.

}
\begin{tabular}{p{3cm}p{13cm}}
\hline
            & Expected Result \\ \hline
\end{tabular}
{\scriptsize
Measurements of validation metrics and the presence of QA plots
resulting from the validation pipeline.

}

\begin{tabular}{p{4cm}p{12cm}}
\toprule
Step 4
& Description \\ \hline
\end{tabular}
{\scriptsize
Compare measured ellipticity correlations to known (for simulated data)
or measured (if using precursor data) values from input (precursor or
simulated) data, and confirm that the output values for all of the
ellipticity performance metrics are as expected.

}
\begin{tabular}{p{3cm}p{13cm}}
\hline
            & Expected Result \\ \hline
\end{tabular}
{\scriptsize
Measured ellipticity metrics that are within reasonable values given the
(known) input dataset.

}

\subsubsection{LVV-T377 - Verify Calculation of Photometric Performance Metrics}\label{lvv-t377}

\begin{longtable}[]{llllll}
\toprule
Version & Status & Priority & Verification Type & Owner
\\\midrule
1 & Approved & Normal &
Test & Leanne Guy
\\\bottomrule
\multicolumn{6}{c}{ Open \href{https://jira.lsstcorp.org/secure/Tests.jspa\#/testCase/LVV-T377}{LVV-T377} in Jira } \\
\end{longtable}

\paragraph{Verification Elements}\mbox{}\\

\begin{itemize}
\item \href{https://jira.lsstcorp.org/browse/LVV-9751}{LVV-9751} - DMS-REQ-0359-V-02: Max fraction of sensors with excess unusable pixels

\item \href{https://jira.lsstcorp.org/browse/LVV-9757}{LVV-9757} - DMS-REQ-0359-V-08: Max cross-talk imperfections

\item \href{https://jira.lsstcorp.org/browse/LVV-9755}{LVV-9755} - DMS-REQ-0359-V-06: Accuracy of photometric transformation

\item \href{https://jira.lsstcorp.org/browse/LVV-9756}{LVV-9756} - DMS-REQ-0359-V-07: RMS width of zero point in u-band

\item \href{https://jira.lsstcorp.org/browse/LVV-9753}{LVV-9753} - DMS-REQ-0359-V-04: Accuracy of zero point for colors with u-band

\item \href{https://jira.lsstcorp.org/browse/LVV-9762}{LVV-9762} - DMS-REQ-0359-V-13: Max sky brightness error

\item \href{https://jira.lsstcorp.org/browse/LVV-9760}{LVV-9760} - DMS-REQ-0359-V-11: Fraction of zero point outliers

\item \href{https://jira.lsstcorp.org/browse/LVV-9761}{LVV-9761} - DMS-REQ-0359-V-12: Max fraction of unusable pixels per sensor

\item \href{https://jira.lsstcorp.org/browse/LVV-9764}{LVV-9764} - DMS-REQ-0359-V-15: Percentage of image area with ghosts

\item \href{https://jira.lsstcorp.org/browse/LVV-9766}{LVV-9766} - DMS-REQ-0359-V-17: Max RMS of resolved/unresolved flux ratio

\item \href{https://jira.lsstcorp.org/browse/LVV-9763}{LVV-9763} - DMS-REQ-0359-V-14: RMS width of zero point in all bands except u

\item \href{https://jira.lsstcorp.org/browse/LVV-9765}{LVV-9765} - DMS-REQ-0359-V-16: Accuracy of zero point for colors without u-band

\end{itemize}

\paragraph{Test Items}\mbox{}\\

Verify that the DMS system provides software to calculate photometric
performance metrics, and that the algorithms are properly calculating
the desired quantities. Note that because the DMS requirement is that
the software shall be provided (and not on the actual measured values of
the metrics), we verify all of the requirements via a single test case.








\paragraph{Test Procedure}\mbox{}\\
\begin{tabular}{p{4cm}p{12cm}}
\toprule
Step 1-1
{\scriptsize from \hyperref[lvv-t987]{LVV-T987} }
& Description \\ \hline
\end{tabular}
{\scriptsize
Identify the path to the data repository, which we will refer to as
`DATA/path', then execute the following:

}
\begin{tabular}{p{3cm}p{13cm}}
\hline
            & Example Code \\ \hline
\end{tabular}
{\scriptsize
\begin{verbatim}
import lsst.daf.persistence as dafPersist
butler = dafPersist.Butler(inputs='DATA/path')
\end{verbatim}

}
\begin{tabular}{p{3cm}p{13cm}}
\hline
            & Expected Result \\ \hline
\end{tabular}
{\scriptsize
Butler repo available for reading.

}

\begin{tabular}{p{4cm}p{12cm}}
\toprule
Step 2
& Description \\ \hline
\end{tabular}
{\scriptsize
Point the butler to a simulated dataset containing data in all filters,
that is sufficient for the purposes of measuring photometric performance
metrics.

}
\begin{tabular}{p{3cm}p{13cm}}
\hline
            & Expected Result \\ \hline
\end{tabular}

\begin{tabular}{p{4cm}p{12cm}}
\toprule
Step 3
& Description \\ \hline
\end{tabular}
{\scriptsize
Execute the LSST Stack package `validate\_drp` (or an alternate package
that is relevant) on this dataset to perform the measurements of the
metrics.

}
\begin{tabular}{p{3cm}p{13cm}}
\hline
            & Expected Result \\ \hline
\end{tabular}
{\scriptsize
Measurements of validation metrics and the presence of QA plots
resulting from the validation pipeline.

}

\begin{tabular}{p{4cm}p{12cm}}
\toprule
Step 4
& Description \\ \hline
\end{tabular}
{\scriptsize
Compare measured photometry to known values from input simulated data,
and confirm that the output values for all of the photometric
performance metrics are as expected.

}
\begin{tabular}{p{3cm}p{13cm}}
\hline
            & Expected Result \\ \hline
\end{tabular}
{\scriptsize
Measured astrometry metrics that are within reasonable values given the
(known) input dataset.

}

\subsubsection{LVV-T378 - Verify Calculation of Astrometric Performance Metrics}\label{lvv-t378}

\begin{longtable}[]{llllll}
\toprule
Version & Status & Priority & Verification Type & Owner
\\\midrule
1 & Approved & Normal &
Test & Leanne Guy
\\\bottomrule
\multicolumn{6}{c}{ Open \href{https://jira.lsstcorp.org/secure/Tests.jspa\#/testCase/LVV-T378}{LVV-T378} in Jira } \\
\end{longtable}

\paragraph{Verification Elements}\mbox{}\\

\begin{itemize}
\item \href{https://jira.lsstcorp.org/browse/LVV-9778}{LVV-9778} - DMS-REQ-0360-V-12: RMS difference between r-band and other filter
separation

\item \href{https://jira.lsstcorp.org/browse/LVV-9777}{LVV-9777} - DMS-REQ-0360-V-11: Max fraction of r-band color difference outliers

\item \href{https://jira.lsstcorp.org/browse/LVV-9779}{LVV-9779} - DMS-REQ-0360-V-13: Max fraction exceeding limit on 200 arcmin scales

\item \href{https://jira.lsstcorp.org/browse/LVV-9773}{LVV-9773} - DMS-REQ-0360-V-07: Outlier limit on 5 arcmin scales

\item \href{https://jira.lsstcorp.org/browse/LVV-9770}{LVV-9770} - DMS-REQ-0360-V-05: Outlier limit on 20 arcmin scales

\item \href{https://jira.lsstcorp.org/browse/LVV-9775}{LVV-9775} - DMS-REQ-0360-V-09: Outlier limit on 200 arcmin scales

\item \href{https://jira.lsstcorp.org/browse/LVV-9769}{LVV-9769} - DMS-REQ-0360-V-04: Median absolute error in RA, Dec

\item \href{https://jira.lsstcorp.org/browse/LVV-9774}{LVV-9774} - DMS-REQ-0360-V-08: Median astrometric error on 200 arcmin scales

\item \href{https://jira.lsstcorp.org/browse/LVV-9768}{LVV-9768} - DMS-REQ-0360-V-03: Median astrometric error on 5 arcmin scales

\item \href{https://jira.lsstcorp.org/browse/LVV-9771}{LVV-9771} - DMS-REQ-0360-V-06: Color difference outlier limit relative to r-band

\item \href{https://jira.lsstcorp.org/browse/LVV-9776}{LVV-9776} - DMS-REQ-0360-V-10: Max fraction exceeding limit on 20 arcmin scales

\item \href{https://jira.lsstcorp.org/browse/LVV-9767}{LVV-9767} - DMS-REQ-0360-V-02: Max fraction exceeding limit on 5 arcmin scales

\end{itemize}

\paragraph{Test Items}\mbox{}\\

Verify that the DMS system provides software to calculate astrometric
performance metrics, and that the algorithms are properly calculating
the desired quantities. Note that because the DMS requirement is that
the software shall be provided (and not on the actual measured values of
the metrics), we verify all of the requirements via a single test case.








\paragraph{Test Procedure}\mbox{}\\
\begin{tabular}{p{4cm}p{12cm}}
\toprule
Step 1-1
{\scriptsize from \hyperref[lvv-t987]{LVV-T987} }
& Description \\ \hline
\end{tabular}
{\scriptsize
Identify the path to the data repository, which we will refer to as
`DATA/path', then execute the following:

}
\begin{tabular}{p{3cm}p{13cm}}
\hline
            & Example Code \\ \hline
\end{tabular}
{\scriptsize
\begin{verbatim}
import lsst.daf.persistence as dafPersist
butler = dafPersist.Butler(inputs='DATA/path')
\end{verbatim}

}
\begin{tabular}{p{3cm}p{13cm}}
\hline
            & Expected Result \\ \hline
\end{tabular}
{\scriptsize
Butler repo available for reading.

}

\begin{tabular}{p{4cm}p{12cm}}
\toprule
Step 2
& Description \\ \hline
\end{tabular}
{\scriptsize
Point the butler to an appropriate (precursor or simulated) dataset
containing data in all filters, that is sufficient for the purposes of
measuring astrometric performance metrics.

}
\begin{tabular}{p{3cm}p{13cm}}
\hline
            & Expected Result \\ \hline
\end{tabular}

\begin{tabular}{p{4cm}p{12cm}}
\toprule
Step 3
& Description \\ \hline
\end{tabular}
{\scriptsize
Execute the LSST Stack package `validate\_drp` (or an alternate package
that is relevant) on this dataset to perform the measurements of the
metrics.

}
\begin{tabular}{p{3cm}p{13cm}}
\hline
            & Expected Result \\ \hline
\end{tabular}
{\scriptsize
Measurements of validation metrics and the presence of QA plots
resulting from the validation pipeline.

}

\begin{tabular}{p{4cm}p{12cm}}
\toprule
Step 4
& Description \\ \hline
\end{tabular}
{\scriptsize
Compare measured astrometry to known (for simulated data) or measured
(if using precursor data) values from input (precursor or simulated)
data, and confirm that the output values for all of the astrometric
performance metrics are as expected.

}
\begin{tabular}{p{3cm}p{13cm}}
\hline
            & Expected Result \\ \hline
\end{tabular}
{\scriptsize
Measured astrometry metrics that are within reasonable values given the
(known) input dataset.

}

\subsubsection{LVV-T454 - LDM-503-8 Enable LSP viewing of spectrograph data.}\label{lvv-t454}

\begin{longtable}[]{llllll}
\toprule
Version & Status & Priority & Verification Type & Owner
\\\midrule
1 & Approved & Normal &
Test & Michelle Gower
\\\bottomrule
\multicolumn{6}{c}{ Open \href{https://jira.lsstcorp.org/secure/Tests.jspa\#/testCase/LVV-T454}{LVV-T454} in Jira } \\
\end{longtable}

\paragraph{Verification Elements}\mbox{}\\

\begin{itemize}
\item \href{https://jira.lsstcorp.org/browse/LVV-140}{LVV-140} - DMS-REQ-0309-V-01: Raw Data Archiving Reliability

\end{itemize}

\paragraph{Test Items}\mbox{}\\

\begin{itemize}
\tightlist
\item
  Acquire spectrograph image data, transfer that data to NCSA, ingest
  data into a Butler (G2 or G3 when available), and enable viewing of
  data on LSP. ~
\end{itemize}


\paragraph{Predecessors}\mbox{}\\
LDM-503-4b

\paragraph{Environment Needs}\mbox{}\\


\subparagraph{Hardware}\mbox{}\\
ATS storage server system housed with spectrograph.~ ~Receiver system at
NCSA for data.~~



\paragraph{Test Procedure}\mbox{}\\
\begin{tabular}{p{4cm}p{12cm}}
\toprule
Step 1
& Description \\ \hline
\end{tabular}
{\scriptsize
Have data on the ATS archiver system from the spectrograph.~

}
\begin{tabular}{p{3cm}p{13cm}}
\hline
            & Expected Result \\ \hline
\end{tabular}
{\scriptsize
Well formed files on the ATS system that need to be transferred to NCSA
for further analysis

}

\begin{tabular}{p{4cm}p{12cm}}
\toprule
Step 2
& Description \\ \hline
\end{tabular}
{\scriptsize
A first few iterations is the human runs script to transfer data to NCSA
through secure pipeline. ~after the process is unchanging/solid, a
cronjob starts up data ``sync'' process. ~

}
\begin{tabular}{p{3cm}p{13cm}}
\hline
            & Expected Result \\ \hline
\end{tabular}
{\scriptsize
Data is transferred to NCSA, and is located in NCSA file
systems.\\[2\baselineskip]

}

\begin{tabular}{p{4cm}p{12cm}}
\toprule
Step 3
& Description \\ \hline
\end{tabular}
{\scriptsize
All files transferred have a ButlerG2 (or G3 when ready) ingest
process.~

}
\begin{tabular}{p{3cm}p{13cm}}
\hline
            & Expected Result \\ \hline
\end{tabular}
{\scriptsize
files now can be accessed by Butler access methods\\[2\baselineskip]

}

\begin{tabular}{p{4cm}p{12cm}}
\toprule
Step 4
& Description \\ \hline
\end{tabular}
{\scriptsize
LSP processes can now view spectrograph generate files~

}
\begin{tabular}{p{3cm}p{13cm}}
\hline
            & Expected Result \\ \hline
\end{tabular}
{\scriptsize
LSP jupyter notebooks can view spectrograph files.\\[2\baselineskip]

}

\subsubsection{LVV-T1085 - Short Queries Functional Test}\label{lvv-t1085}

\begin{longtable}[]{llllll}
\toprule
Version & Status & Priority & Verification Type & Owner
\\\midrule
1 & Approved & Normal &
Test & Fritz Mueller
\\\bottomrule
\multicolumn{6}{c}{ Open \href{https://jira.lsstcorp.org/secure/Tests.jspa\#/testCase/LVV-T1085}{LVV-T1085} in Jira } \\
\end{longtable}

\paragraph{Verification Elements}\mbox{}\\

\begin{itemize}
\item \href{https://jira.lsstcorp.org/browse/LVV-33}{LVV-33} - DMS-REQ-0075-V-01: Catalog Queries

\item \href{https://jira.lsstcorp.org/browse/LVV-9787}{LVV-9787} - DMS-REQ-0356-V-04: Max time to retrieve low-volume query results

\end{itemize}

\paragraph{Test Items}\mbox{}\\

The objective of this test is to ensure that the short queries are
performing as expected and establish a timing baseline benchmark for
these types of queries.








\paragraph{Test Procedure}\mbox{}\\
\begin{tabular}{p{4cm}p{12cm}}
\toprule
Step 1
& Description \\ \hline
\end{tabular}
{\scriptsize
Execute single object selection:\\[2\baselineskip]\textbf{SELECT} *
\textbf{FROM} Object~\textbf{WHERE} deepSourceId =
9292041530376264\\[2\baselineskip]and record execution time.

}
\begin{tabular}{p{3cm}p{13cm}}
\hline
            & Expected Result \\ \hline
\end{tabular}
{\scriptsize
Query runs in less than 10 seconds.

}

\begin{tabular}{p{4cm}p{12cm}}
\toprule
Step 2
& Description \\ \hline
\end{tabular}
{\scriptsize
Execute spatial area selection from
Object:\\[2\baselineskip]\textbf{SELECT COUNT(*)} \textbf{FROM} Object
\textbf{WHERE}~\\

~qserv\_areaspec\_box(316.582327, −6.839078, 316.653938, −6.781822)

and record execution time.

}
\begin{tabular}{p{3cm}p{13cm}}
\hline
            & Expected Result \\ \hline
\end{tabular}
{\scriptsize
Query runs in less than 10 seconds.

}

\subsubsection{LVV-T1086 - Full Table Scans Functional Test}\label{lvv-t1086}

\begin{longtable}[]{llllll}
\toprule
Version & Status & Priority & Verification Type & Owner
\\\midrule
1 & Approved & Normal &
Test & Fritz Mueller
\\\bottomrule
\multicolumn{6}{c}{ Open \href{https://jira.lsstcorp.org/secure/Tests.jspa\#/testCase/LVV-T1086}{LVV-T1086} in Jira } \\
\end{longtable}

\paragraph{Verification Elements}\mbox{}\\

\begin{itemize}
\item \href{https://jira.lsstcorp.org/browse/LVV-33}{LVV-33} - DMS-REQ-0075-V-01: Catalog Queries

\item \href{https://jira.lsstcorp.org/browse/LVV-188}{LVV-188} - DMS-REQ-0357-V-01: Result latency for high-volume full-sky queries on
the Object table

\item \href{https://jira.lsstcorp.org/browse/LVV-185}{LVV-185} - DMS-REQ-0354-V-01: Result latency for high-volume complex queries

\end{itemize}

\paragraph{Test Items}\mbox{}\\

The objective of this test is to ensure that the full table scan queries
are performing as expected and establish a timing baseline benchmark for
these types of queries.








\paragraph{Test Procedure}\mbox{}\\
\begin{tabular}{p{4cm}p{12cm}}
\toprule
Step 1
& Description \\ \hline
\end{tabular}
{\scriptsize
Execute query:\\[2\baselineskip]\textbf{SELECT} ra , decl , u\_psfFlux ,
g\_psfFlux , r\_psfFlux \textbf{FROM} Object\\
\textbf{WHERE} y\_shapeIxx \textbf{BETWEEN} 20 \textbf{AND}
20.1\\[3\baselineskip]and record execution time and output size.

}
\begin{tabular}{p{3cm}p{13cm}}
\hline
            & Expected Result \\ \hline
\end{tabular}
{\scriptsize
Query expected to run in less than 1 hour.\\[2\baselineskip]

}

\begin{tabular}{p{4cm}p{12cm}}
\toprule
Step 2
& Description \\ \hline
\end{tabular}
{\scriptsize
Execute query:\\[2\baselineskip]\textbf{SELECT} COUNT(*) \textbf{FROM}
Source \textbf{WHERE} flux\_sinc \textbf{BETWEEN} 1 \textbf{AND}
1.1\\[2\baselineskip]and record the execution time

}
\begin{tabular}{p{3cm}p{13cm}}
\hline
            & Expected Result \\ \hline
\end{tabular}
{\scriptsize
Query expected to run in less than 12 hours.

}

\begin{tabular}{p{4cm}p{12cm}}
\toprule
Step 3
& Description \\ \hline
\end{tabular}
{\scriptsize
Execute query:\\[2\baselineskip]\textbf{SELECT} COUNT(*) \textbf{FROM}
ForcedSource \textbf{WHERE} psfFlux \textbf{BETWEEN} 0.1 \textbf{AND}
0.2\\[2\baselineskip]and record the execution time

}
\begin{tabular}{p{3cm}p{13cm}}
\hline
            & Expected Result \\ \hline
\end{tabular}
{\scriptsize
Query expected to run in less than 12 hours.

}

\subsubsection{LVV-T1087 - Full Table Joins Functional Test}\label{lvv-t1087}

\begin{longtable}[]{llllll}
\toprule
Version & Status & Priority & Verification Type & Owner
\\\midrule
1 & Approved & Normal &
Test & Fritz Mueller
\\\bottomrule
\multicolumn{6}{c}{ Open \href{https://jira.lsstcorp.org/secure/Tests.jspa\#/testCase/LVV-T1087}{LVV-T1087} in Jira } \\
\end{longtable}

\paragraph{Verification Elements}\mbox{}\\

\begin{itemize}
\item \href{https://jira.lsstcorp.org/browse/LVV-33}{LVV-33} - DMS-REQ-0075-V-01: Catalog Queries

\item \href{https://jira.lsstcorp.org/browse/LVV-185}{LVV-185} - DMS-REQ-0354-V-01: Result latency for high-volume complex queries

\end{itemize}

\paragraph{Test Items}\mbox{}\\

The objective of this test is to ensure that the full table join queries
are performing as expected and establish a timing baseline benchmark for
these types of queries.








\paragraph{Test Procedure}\mbox{}\\
\begin{tabular}{p{4cm}p{12cm}}
\toprule
Step 1
& Description \\ \hline
\end{tabular}
{\scriptsize
Execute query:\\[2\baselineskip]\textbf{SELECT} o.deepSourceId,
s.objectId, s.id, o.ra, o.decl\\
\textbf{~ ~ FROM} Object o, Source s WHERE o.deepSourceId=s.objectId\\
\hspace*{0.333em} ~ \textbf{AND} s . flux\_sinc \textbf{BETWEEN} 0.3
\textbf{AND} 0.31\\[2\baselineskip]and record execution time.

}
\begin{tabular}{p{3cm}p{13cm}}
\hline
            & Expected Result \\ \hline
\end{tabular}
{\scriptsize
Query expected to run in less than 12 hours.

}

\begin{tabular}{p{4cm}p{12cm}}
\toprule
Step 2
& Description \\ \hline
\end{tabular}
{\scriptsize
Execute query:\\[2\baselineskip]\textbf{SELECT} o.deepSourceId,
f.psfFlux \textbf{FROM} Object o, ForcedSource f\\
\textbf{~ ~ WHERE} o.deepSourceId=f.deepSourceId\\
\textbf{~ ~ AND} f . psfFlux \textbf{BETWEEN} 0.13 \textbf{AND}
0.14\\[2\baselineskip]and record execution time.

}
\begin{tabular}{p{3cm}p{13cm}}
\hline
            & Expected Result \\ \hline
\end{tabular}
{\scriptsize
Query expected to run in less than 12 hours.

}

\subsubsection{LVV-T1088 - Concurrent Scans Scaling Test}\label{lvv-t1088}

\begin{longtable}[]{llllll}
\toprule
Version & Status & Priority & Verification Type & Owner
\\\midrule
1 & Approved & Normal &
Test & Fritz Mueller
\\\bottomrule
\multicolumn{6}{c}{ Open \href{https://jira.lsstcorp.org/secure/Tests.jspa\#/testCase/LVV-T1088}{LVV-T1088} in Jira } \\
\end{longtable}

\paragraph{Verification Elements}\mbox{}\\

\begin{itemize}
\item \href{https://jira.lsstcorp.org/browse/LVV-185}{LVV-185} - DMS-REQ-0354-V-01: Result latency for high-volume complex queries

\item \href{https://jira.lsstcorp.org/browse/LVV-188}{LVV-188} - DMS-REQ-0357-V-01: Result latency for high-volume full-sky queries on
the Object table

\item \href{https://jira.lsstcorp.org/browse/LVV-3403}{LVV-3403} - DMS-REQ-0361-V-01: Simultaneous users for high-volume queries

\end{itemize}

\paragraph{Test Items}\mbox{}\\

This test will show that average completion-time of full-scan queries of
the Object catalog table grows sub-linearly with respect to the number
of simultaneously active full-scan queries, within the limits of machine
resource exhaustion.








\paragraph{Test Procedure}\mbox{}\\
\begin{tabular}{p{4cm}p{12cm}}
\toprule
Step 1
& Description \\ \hline
\end{tabular}
{\scriptsize
Repeat steps 2 through 5 below, where ``pool of interest'' is taken
first to be ``FTSObj'' and subsequently ``FTSSrc'':

}
\begin{tabular}{p{3cm}p{13cm}}
\hline
            & Expected Result \\ \hline
\end{tabular}
{\scriptsize
At end of each pass, a graph indicating scan scaling rate and machine
resource exhaustion cutoff.

}

\begin{tabular}{p{4cm}p{12cm}}
\toprule
Step 2
& Description \\ \hline
\end{tabular}
{\scriptsize
Inspect and modify the CONCURRENCY and TARGET\_RATES dictionaries in the
runQueries.py script. Set CONCURRENCY initially to 1 for the query pool
of interest, and to 0 for all other query pools. Set TARGET\_RATES for
the query pool of interest to the yearly value per table in LDM-552,
section 2.2.1.

}
\begin{tabular}{p{3cm}p{13cm}}
\hline
            & Expected Result \\ \hline
\end{tabular}
{\scriptsize
rueQueries.py script updated with appropriate values for test iteration

}

\begin{tabular}{p{4cm}p{12cm}}
\toprule
Step 3
& Description \\ \hline
\end{tabular}
{\scriptsize
Execute the runQueries.py script and let it run for at least one, but
preferably several, query cycles.

}
\begin{tabular}{p{3cm}p{13cm}}
\hline
            & Expected Result \\ \hline
\end{tabular}
{\scriptsize
Test script executes producing log file.

}

\begin{tabular}{p{4cm}p{12cm}}
\toprule
Step 4
& Description \\ \hline
\end{tabular}
{\scriptsize
Examine log file output and compile performance statistics to obtain a
growth curve point for the pool of interest for the test report.

}
\begin{tabular}{p{3cm}p{13cm}}
\hline
            & Expected Result \\ \hline
\end{tabular}
{\scriptsize
Logs indicate either successful test run, providing another growth point
for curve, or errors indicating machine resource exhaustion cutoff has
been reached.

}

\begin{tabular}{p{4cm}p{12cm}}
\toprule
Step 5
& Description \\ \hline
\end{tabular}
{\scriptsize
Adjust the CONCURRENCY value for the pool of interest and repeat from
step 3 to establish the growth trend and machine resource exhaustion
cutoff for the query pool of interest to an acceptable degree of
accuracy.

}
\begin{tabular}{p{3cm}p{13cm}}
\hline
            & Expected Result \\ \hline
\end{tabular}
{\scriptsize
Average query execution time for full scan queries of each class should
be demonstrated to grow sub-linearly in the number of concurrent queries
to the limits of machine resource exhaustion.

}

\subsubsection{LVV-T1089 - Load Test}\label{lvv-t1089}

\begin{longtable}[]{llllll}
\toprule
Version & Status & Priority & Verification Type & Owner
\\\midrule
1 & Approved & Normal &
Test & Fritz Mueller
\\\bottomrule
\multicolumn{6}{c}{ Open \href{https://jira.lsstcorp.org/secure/Tests.jspa\#/testCase/LVV-T1089}{LVV-T1089} in Jira } \\
\end{longtable}

\paragraph{Verification Elements}\mbox{}\\

\begin{itemize}
\item \href{https://jira.lsstcorp.org/browse/LVV-9786}{LVV-9786} - DMS-REQ-0356-V-03: Min number of simultaneous low-volume query users

\item \href{https://jira.lsstcorp.org/browse/LVV-9787}{LVV-9787} - DMS-REQ-0356-V-04: Max time to retrieve low-volume query results

\item \href{https://jira.lsstcorp.org/browse/LVV-188}{LVV-188} - DMS-REQ-0357-V-01: Result latency for high-volume full-sky queries on
the Object table

\item \href{https://jira.lsstcorp.org/browse/LVV-185}{LVV-185} - DMS-REQ-0354-V-01: Result latency for high-volume complex queries

\item \href{https://jira.lsstcorp.org/browse/LVV-3403}{LVV-3403} - DMS-REQ-0361-V-01: Simultaneous users for high-volume queries

\end{itemize}

\paragraph{Test Items}\mbox{}\\

This test will check that Qserv is able to meet average query completion
time targets per query class under a representative load of simultaneous
high and low volume queries while running against an appropriately
scaled test catalog.








\paragraph{Test Procedure}\mbox{}\\
\begin{tabular}{p{4cm}p{12cm}}
\toprule
Step 1
& Description \\ \hline
\end{tabular}
{\scriptsize
Inspect and modify the CONCURRENCY and TARGET\_RATES dictionaries in the
runQueries.py script. ~Set CONCURRENCY and TARGET\_RATES for all pools
to the yearly value per table in LDM-552, section 2.2.1.

}
\begin{tabular}{p{3cm}p{13cm}}
\hline
            & Expected Result \\ \hline
\end{tabular}
{\scriptsize
Script updated with appropriate values.

}

\begin{tabular}{p{4cm}p{12cm}}
\toprule
Step 2
& Description \\ \hline
\end{tabular}
{\scriptsize
Execute the runQueries.py script and let it run for 24 hours.

}
\begin{tabular}{p{3cm}p{13cm}}
\hline
            & Expected Result \\ \hline
\end{tabular}
{\scriptsize
Script runs without error and produces output log.

}

\begin{tabular}{p{4cm}p{12cm}}
\toprule
Step 3
& Description \\ \hline
\end{tabular}
{\scriptsize
Examine log file output and compile average query execution times per
query type; and compare to yearly target values per table in LDM-552,
section 2.2.1.

}
\begin{tabular}{p{3cm}p{13cm}}
\hline
            & Expected Result \\ \hline
\end{tabular}
{\scriptsize
Average query times per query type equal or less than corresponding
yearly target values in LDM-552, section 2.2.1.

}

\subsubsection{LVV-T1090 - Heavy Load Test}\label{lvv-t1090}

\begin{longtable}[]{llllll}
\toprule
Version & Status & Priority & Verification Type & Owner
\\\midrule
1 & Approved & Normal &
Test & Fritz Mueller
\\\bottomrule
\multicolumn{6}{c}{ Open \href{https://jira.lsstcorp.org/secure/Tests.jspa\#/testCase/LVV-T1090}{LVV-T1090} in Jira } \\
\end{longtable}

\paragraph{Verification Elements}\mbox{}\\

\begin{itemize}
\item \href{https://jira.lsstcorp.org/browse/LVV-9786}{LVV-9786} - DMS-REQ-0356-V-03: Min number of simultaneous low-volume query users

\item \href{https://jira.lsstcorp.org/browse/LVV-9787}{LVV-9787} - DMS-REQ-0356-V-04: Max time to retrieve low-volume query results

\item \href{https://jira.lsstcorp.org/browse/LVV-188}{LVV-188} - DMS-REQ-0357-V-01: Result latency for high-volume full-sky queries on
the Object table

\item \href{https://jira.lsstcorp.org/browse/LVV-185}{LVV-185} - DMS-REQ-0354-V-01: Result latency for high-volume complex queries

\item \href{https://jira.lsstcorp.org/browse/LVV-3403}{LVV-3403} - DMS-REQ-0361-V-01: Simultaneous users for high-volume queries

\end{itemize}

\paragraph{Test Items}\mbox{}\\

This test will check that Qserv is able to meet average query completion
time targets per query class under a higher than average load of
simultaneous high and low volume queries while running against an
appropriately scaled test catalog.








\paragraph{Test Procedure}\mbox{}\\
\begin{tabular}{p{4cm}p{12cm}}
\toprule
Step 1
& Description \\ \hline
\end{tabular}
{\scriptsize
Inspect and modify the CONCURRENCY and TARGET\_RATES dictionaries in the
runQueries.py script. ~Set CONCURRENCY and TARGET\_RATES for LV query
pool to 2020 value per table in LDM-552, section 2.2.1.~ Set CONCURRENCY
and TARGET\_RATES for all other query pools to values in next column
over from current year column (or to 2020 values +10\% if year is 2020)
per table in LDM-552, section 2.2.1.

}
\begin{tabular}{p{3cm}p{13cm}}
\hline
            & Expected Result \\ \hline
\end{tabular}
{\scriptsize
Script updated with appropriate values.

}

\begin{tabular}{p{4cm}p{12cm}}
\toprule
Step 2
& Description \\ \hline
\end{tabular}
{\scriptsize
Execute the runQueries.py script and let it run for 24 hrs.

}
\begin{tabular}{p{3cm}p{13cm}}
\hline
            & Expected Result \\ \hline
\end{tabular}
{\scriptsize
Script runs without error and produces output log.

}

\begin{tabular}{p{4cm}p{12cm}}
\toprule
Step 3
& Description \\ \hline
\end{tabular}
{\scriptsize
Examine log file output and compile average query execution times per
query type.

}
\begin{tabular}{p{3cm}p{13cm}}
\hline
            & Expected Result \\ \hline
\end{tabular}
{\scriptsize
Average query times per query type equal or less than corresponding
yearly target values in LDM-552, section 2.2.1.

}

\subsubsection{LVV-T1168 - Verify Summit - Base Network Integration}\label{lvv-t1168}

\begin{longtable}[]{llllll}
\toprule
Version & Status & Priority & Verification Type & Owner
\\\midrule
1 & Approved & Normal &
Inspection & Jeff Kantor
\\\bottomrule
\multicolumn{6}{c}{ Open \href{https://jira.lsstcorp.org/secure/Tests.jspa\#/testCase/LVV-T1168}{LVV-T1168} in Jira } \\
\end{longtable}

\paragraph{Verification Elements}\mbox{}\\

\begin{itemize}
\item \href{https://jira.lsstcorp.org/browse/LVV-73}{LVV-73} - DMS-REQ-0171-V-01: Summit to Base Network

\end{itemize}

\paragraph{Test Items}\mbox{}\\

Verify the integration of the summit to base network by demonstrating a
sustained and uninterrupted transfer of data between summit and base
over 1 day period at or exceeding rates specified in \citeds{LDM-142}. Done in 3
phases in collaboration with equipment/installation vendors (see test
procedure).


\paragraph{Predecessors}\mbox{}\\
See pre-conditions by phase above.

\paragraph{Environment Needs}\mbox{}\\

\subparagraph{Software}\mbox{}\\
perfsonar on DTN.

\subparagraph{Hardware}\mbox{}\\
OTDR, DTN.

\paragraph{Input Specification}\mbox{}\\
PMCS DMTC-7400-2330 COMPLETE\\
By phase:

\begin{enumerate}
\tightlist
\item
  Posts from Cerro Pachon to AURA Gatehouse repaired/improved. ~Fiber
  installed on posts from Cerro Pachon to AURA Gatehouse. ~Fiber
  installed from AURA Gatehouse to AURA compound in La Serena. OTDR
  purchased.
\item
  AURA DWDM installed in caseta on Cerro Pachon and in existing computer
  room in La Serena. ~DTN installed in La Serena. ~DTN loaded with
  software and test data staged.
\item
  Base Data Center (BDC) ready for installation of LSST DWDM. ~Fiber
  connecting existing computer room to BDC. ~LSST DWDM equipment
  installed in Summit Computer Room and BDC.
\end{enumerate}

\paragraph{Output Specification}\mbox{}\\
Fiber tested to within acceptable Db. ~Bandwidth, latency within
specifications.

\paragraph{Test Procedure}\mbox{}\\
\begin{tabular}{p{4cm}p{12cm}}
\toprule
Step 1
& Description \\ \hline
\end{tabular}
{\scriptsize
Test optical fiber with OTDR:\\
Installation of fiber optic cables and Optical Time Domain Reflector
(OTDR) fiber testing (completed 20170602
\href{https://docushare.lsstcorp.org/docushare/dsweb/Get/Document-26270/RD10\%20Report\%20of\%20delivery\%20of\%20LS\%20-\%20AG\%20fiber\%20from\%20Telefonica\%20to\%20REUNA.pdf}{REUNA
deliverable RD10})

}
\begin{tabular}{p{3cm}p{13cm}}
\hline
            & Test Data \\ \hline
\end{tabular}
{\scriptsize
OTDR generated optical data

}
\begin{tabular}{p{3cm}p{13cm}}
\hline
            & Expected Result \\ \hline
\end{tabular}
{\scriptsize
Fiber tested to within acceptable Db.

}

\begin{tabular}{p{4cm}p{12cm}}
\toprule
Step 2
& Description \\ \hline
\end{tabular}
{\scriptsize
Test AURA DWDM:\\
Installation of AURA DWDM and Data Transfer Node (DTN) (completed
20171218
\href{https://docushare.lsst.org/docushare/dsweb/Get/DMTR-82/DMTR-82.pdf}{DMTR-82})

}
\begin{tabular}{p{3cm}p{13cm}}
\hline
            & Test Data \\ \hline
\end{tabular}
{\scriptsize
DTN perfSonar generated data

}
\begin{tabular}{p{3cm}p{13cm}}
\hline
            & Expected Result \\ \hline
\end{tabular}
{\scriptsize
Summit - Base bandwidth and latency within specifications

}

\begin{tabular}{p{4cm}p{12cm}}
\toprule
Step 3
& Description \\ \hline
\end{tabular}
{\scriptsize
Test LSST DWDM:\\
Installation of LSST DWDM and Bit Error Rate Tester (BERT) data
(completed 20190505
\href{https://docushare.lsstcorp.org/docushare/dsweb/View/Collection-7743}{collection-7743},
20191108
\href{https://docushare.lsstcorp.org/docushare/dsweb/Get/Document-35302/DAQ\%20DWDM\%20connection\%20tests\%2020191109.pptx}{DAQ
DWDM Connection Tests})

}
\begin{tabular}{p{3cm}p{13cm}}
\hline
            & Test Data \\ \hline
\end{tabular}
{\scriptsize
BERT generated data

}
\begin{tabular}{p{3cm}p{13cm}}
\hline
            & Expected Result \\ \hline
\end{tabular}
{\scriptsize
Summit - Base bandwidth, latency, bit error rate within specifications

}

\subsubsection{LVV-T1232 - Verify Implementation of Catalog Export Formats From the Portal Aspect}\label{lvv-t1232}

\begin{longtable}[]{llllll}
\toprule
Version & Status & Priority & Verification Type & Owner
\\\midrule
1 & Approved & Normal &
Test & Colin Slater
\\\bottomrule
\multicolumn{6}{c}{ Open \href{https://jira.lsstcorp.org/secure/Tests.jspa\#/testCase/LVV-T1232}{LVV-T1232} in Jira } \\
\end{longtable}

\paragraph{Verification Elements}\mbox{}\\

\begin{itemize}
\item \href{https://jira.lsstcorp.org/browse/LVV-35}{LVV-35} - DMS-REQ-0078-V-01: Catalog Export Formats

\end{itemize}

\paragraph{Test Items}\mbox{}\\

Verify that catalog data is exportable from the portal aspect in a
variety of community-standard formats.








\paragraph{Test Procedure}\mbox{}\\
\begin{tabular}{p{4cm}p{12cm}}
\toprule
Step 1-1
{\scriptsize from \hyperref[lvv-t849]{LVV-T849} }
& Description \\ \hline
\end{tabular}
{\scriptsize
Navigate to the Portal Aspect endpoint. ~The stable version should be
used for this test and is currently located at:
https://lsst-lsp-stable.ncsa.illinois.edu/portal/app/ .

}
\begin{tabular}{p{3cm}p{13cm}}
\hline
            & Expected Result \\ \hline
\end{tabular}
{\scriptsize
A credential-entry screen should be displayed.

}

\begin{tabular}{p{4cm}p{12cm}}
\toprule
Step 1-2
{\scriptsize from \hyperref[lvv-t849]{LVV-T849} }
& Description \\ \hline
\end{tabular}
{\scriptsize
Enter a valid set of credentials for an LSST user with LSP access on the
instance under test.

}
\begin{tabular}{p{3cm}p{13cm}}
\hline
            & Expected Result \\ \hline
\end{tabular}
{\scriptsize
The Portal Aspect UI should be displayed following authentication.

}

\begin{tabular}{p{4cm}p{12cm}}
\toprule
Step 2
& Description \\ \hline
\end{tabular}
{\scriptsize
Select query type ``ADQL''.

}
\begin{tabular}{p{3cm}p{13cm}}
\hline
            & Expected Result \\ \hline
\end{tabular}

\begin{tabular}{p{4cm}p{12cm}}
\toprule
Step 3
& Description \\ \hline
\end{tabular}
{\scriptsize
Execute the example query given in the example code below by entering
the text in the ADQL Query box, then clicking ``Search'' at the lower
left corner of the page.

}
\begin{tabular}{p{3cm}p{13cm}}
\hline
            & Example Code \\ \hline
\end{tabular}
{\scriptsize
SELECT cntr, ra, decl, w1mpro\_ep, w2mpro\_ep, w3mpro\_ep FROM
wise\_00.allwise\_p3as\_mep WHERE CONTAINS(POINT('ICRS', ra, decl),
CIRCLE('ICRS', 192.85, 27.13, .2)) = 1

}
\begin{tabular}{p{3cm}p{13cm}}
\hline
            & Expected Result \\ \hline
\end{tabular}
{\scriptsize
A new page will load with the search results as a table, with some plots
as well.

}

\begin{tabular}{p{4cm}p{12cm}}
\toprule
Step 4
& Description \\ \hline
\end{tabular}
{\scriptsize
Click the icon that looks like a floppy disk (it says ``Save the content
as an IPAC, CSV, or TSV table'' when you mouse over it).

}
\begin{tabular}{p{3cm}p{13cm}}
\hline
            & Expected Result \\ \hline
\end{tabular}

\begin{tabular}{p{4cm}p{12cm}}
\toprule
Step 5
& Description \\ \hline
\end{tabular}
{\scriptsize
\begin{itemize}
\tightlist
\item
  Select ``CSV'', then specify a destination to save the file on your
  local computer.
\item
  Select ``VOTable'', then specify a destination to save the file on
  your local computer.
\item
  Select ``FITS'', then specify a destination to save the file on your
  local computer.
\end{itemize}

}
\begin{tabular}{p{3cm}p{13cm}}
\hline
            & Expected Result \\ \hline
\end{tabular}

\begin{tabular}{p{4cm}p{12cm}}
\toprule
Step 6
& Description \\ \hline
\end{tabular}
{\scriptsize
Open each of the files (either in TOPCAT, or using Astropy io tools).
Confirm that the data tables are well-formed, and that each table
contains the same columns and the same number of rows.

}
\begin{tabular}{p{3cm}p{13cm}}
\hline
            & Expected Result \\ \hline
\end{tabular}

\begin{tabular}{p{4cm}p{12cm}}
\toprule
Step 7-1
{\scriptsize from \hyperref[lvv-t850]{LVV-T850} }
& Description \\ \hline
\end{tabular}
{\scriptsize
Currently, there is no logout mechanism on the portal.\\
This should be updated as the system matures.\\[2\baselineskip]Simply
close the browser window.

}
\begin{tabular}{p{3cm}p{13cm}}
\hline
            & Expected Result \\ \hline
\end{tabular}
{\scriptsize
Closed browser window. ~When navigating to the portal endpoint, expect
to execute the steps in LVV-T849.

}

\subsubsection{LVV-T1240 - Verify implementation of minimum astrometric standards per CCD}\label{lvv-t1240}

\begin{longtable}[]{llllll}
\toprule
Version & Status & Priority & Verification Type & Owner
\\\midrule
1 & Approved & Normal &
Test & Jim Bosch
\\\bottomrule
\multicolumn{6}{c}{ Open \href{https://jira.lsstcorp.org/secure/Tests.jspa\#/testCase/LVV-T1240}{LVV-T1240} in Jira } \\
\end{longtable}

\paragraph{Verification Elements}\mbox{}\\

\begin{itemize}
\item \href{https://jira.lsstcorp.org/browse/LVV-9741}{LVV-9741} - DMS-REQ-0030-V-02: Minimum astrometric standards per CCD

\end{itemize}

\paragraph{Test Items}\mbox{}\\

Verify that each CCD in a processed dataset had its astrometric solution
determined by at least~\textbf{astrometricMinStandards = 5~}astrometric
standards.








\paragraph{Test Procedure}\mbox{}\\
\begin{tabular}{p{4cm}p{12cm}}
\toprule
Step 1
& Description \\ \hline
\end{tabular}
{\scriptsize
Identify an appropriate processed dataset for this test.

}
\begin{tabular}{p{3cm}p{13cm}}
\hline
            & Expected Result \\ \hline
\end{tabular}
{\scriptsize
A dataset with Processed Visit Images.

}

\begin{tabular}{p{4cm}p{12cm}}
\toprule
Step 2-1
{\scriptsize from \hyperref[lvv-t987]{LVV-T987} }
& Description \\ \hline
\end{tabular}
{\scriptsize
Identify the path to the data repository, which we will refer to as
`DATA/path', then execute the following:

}
\begin{tabular}{p{3cm}p{13cm}}
\hline
            & Example Code \\ \hline
\end{tabular}
{\scriptsize
\begin{verbatim}
import lsst.daf.persistence as dafPersist
butler = dafPersist.Butler(inputs='DATA/path')
\end{verbatim}

}
\begin{tabular}{p{3cm}p{13cm}}
\hline
            & Expected Result \\ \hline
\end{tabular}
{\scriptsize
Butler repo available for reading.

}

\begin{tabular}{p{4cm}p{12cm}}
\toprule
Step 3
& Description \\ \hline
\end{tabular}
{\scriptsize
Select a single visit from the dataset, and extract its calibration
data. For a subset of CCDs, check how many astrometric standards
contributed to the solution. Confirm that this number is at
least~\textbf{astrometricMinStandards = 5.}

}
\begin{tabular}{p{3cm}p{13cm}}
\hline
            & Expected Result \\ \hline
\end{tabular}
{\scriptsize
At least \textbf{astrometricMinStandards} from each CCD\textbf{~}were
used in determining the WCS solution.

}

\subsubsection{LVV-T1264 - Verify implementation of archiving camera test data}\label{lvv-t1264}

\begin{longtable}[]{llllll}
\toprule
Version & Status & Priority & Verification Type & Owner
\\\midrule
1 & Approved & Normal &
Test & Robert Gruendl
\\\bottomrule
\multicolumn{6}{c}{ Open \href{https://jira.lsstcorp.org/secure/Tests.jspa\#/testCase/LVV-T1264}{LVV-T1264} in Jira } \\
\end{longtable}

\paragraph{Verification Elements}\mbox{}\\

\begin{itemize}
\item \href{https://jira.lsstcorp.org/browse/LVV-9637}{LVV-9637} - DMS-REQ-0372-V-01: Archiving Camera Test Data

\end{itemize}

\paragraph{Test Items}\mbox{}\\

Verify that a subset of camera test data has been ingested into Butler
repos and is available through standard data access tools.








\paragraph{Test Procedure}\mbox{}\\
\begin{tabular}{p{4cm}p{12cm}}
\toprule
Step 1
& Description \\ \hline
\end{tabular}
{\scriptsize
Obtain some data on a camera test stand.

}
\begin{tabular}{p{3cm}p{13cm}}
\hline
            & Expected Result \\ \hline
\end{tabular}

\begin{tabular}{p{4cm}p{12cm}}
\toprule
Step 2
& Description \\ \hline
\end{tabular}
{\scriptsize
Wait a sufficient amount of time, then confirm that automatic
transfer/ingest of the data has occurred, and a repo is available at
NCSA.

}
\begin{tabular}{p{3cm}p{13cm}}
\hline
            & Expected Result \\ \hline
\end{tabular}
{\scriptsize
The data is present at NCSA in non-empty repos.

}

\begin{tabular}{p{4cm}p{12cm}}
\toprule
Step 3
& Description \\ \hline
\end{tabular}
{\scriptsize
Identify the relevant Butler repo of ingested camera test stand data.

}
\begin{tabular}{p{3cm}p{13cm}}
\hline
            & Expected Result \\ \hline
\end{tabular}

\begin{tabular}{p{4cm}p{12cm}}
\toprule
Step 4-1
{\scriptsize from \hyperref[lvv-t987]{LVV-T987} }
& Description \\ \hline
\end{tabular}
{\scriptsize
Identify the path to the data repository, which we will refer to as
`DATA/path', then execute the following:

}
\begin{tabular}{p{3cm}p{13cm}}
\hline
            & Example Code \\ \hline
\end{tabular}
{\scriptsize
\begin{verbatim}
import lsst.daf.persistence as dafPersist
butler = dafPersist.Butler(inputs='DATA/path')
\end{verbatim}

}
\begin{tabular}{p{3cm}p{13cm}}
\hline
            & Expected Result \\ \hline
\end{tabular}
{\scriptsize
Butler repo available for reading.

}

\begin{tabular}{p{4cm}p{12cm}}
\toprule
Step 5
& Description \\ \hline
\end{tabular}
{\scriptsize
Read various repo data products with the Butler, and confirm that they
contain the expected data.

}
\begin{tabular}{p{3cm}p{13cm}}
\hline
            & Expected Result \\ \hline
\end{tabular}
{\scriptsize
Camera test stand data that is well-formed.

}

\subsubsection{LVV-T1549 - LDM-503-6 Comcam verification readiness}\label{lvv-t1549}

\begin{longtable}[]{llllll}
\toprule
Version & Status & Priority & Verification Type & Owner
\\\midrule
1 & Approved & Normal &
Demonstration & Michelle Butler
\\\bottomrule
\multicolumn{6}{c}{ Open \href{https://jira.lsstcorp.org/secure/Tests.jspa\#/testCase/LVV-T1549}{LVV-T1549} in Jira } \\
\end{longtable}

\paragraph{Verification Elements}\mbox{}\\

\begin{itemize}
\item \href{https://jira.lsstcorp.org/browse/LVV-9}{LVV-9} - DMS-REQ-0020-V-01: Wavefront Sensor Data Acquisition

\item \href{https://jira.lsstcorp.org/browse/LVV-8}{LVV-8} - DMS-REQ-0018-V-01: Raw Science Image Data Acquisition

\item \href{https://jira.lsstcorp.org/browse/LVV-28}{LVV-28} - DMS-REQ-0068-V-01: Raw Science Image Metadata

\item \href{https://jira.lsstcorp.org/browse/LVV-11}{LVV-11} - DMS-REQ-0024-V-01: Raw Image Assembly

\item \href{https://jira.lsstcorp.org/browse/LVV-146}{LVV-146} - DMS-REQ-0315-V-01: DMS Communication with OCS

\end{itemize}

\paragraph{Test Items}\mbox{}\\

Verify that ComCam has all the services running and verified working for
retrieving an image from the ComCam DAQ and store it on file systems at
the LDF for viewing by RSP. ~








\paragraph{Test Procedure}\mbox{}\\
\begin{tabular}{p{4cm}p{12cm}}
\toprule
Step 1
& Description \\ \hline
\end{tabular}
{\scriptsize
ComCam-DAQ produces an image~

}
\begin{tabular}{p{3cm}p{13cm}}
\hline
            & Test Data \\ \hline
\end{tabular}
{\scriptsize
DAQ produces a SAL message that a image has been created~

}
\begin{tabular}{p{3cm}p{13cm}}
\hline
            & Expected Result \\ \hline
\end{tabular}
{\scriptsize
in memory file created in DAQ

}

\begin{tabular}{p{4cm}p{12cm}}
\toprule
Step 2
& Description \\ \hline
\end{tabular}
{\scriptsize
ComCam-archiver and ComCam-forwarder build image with proper header from
ComCam-header service~

}
\begin{tabular}{p{3cm}p{13cm}}
\hline
            & Test Data \\ \hline
\end{tabular}
{\scriptsize
Good image file with proper header with all 9 CCDs~

}
\begin{tabular}{p{3cm}p{13cm}}
\hline
            & Expected Result \\ \hline
\end{tabular}
{\scriptsize
9 image files all with individual headers and then 1 header for all 9
images too. ~\\[2\baselineskip]

}

\begin{tabular}{p{4cm}p{12cm}}
\toprule
Step 3
& Description \\ \hline
\end{tabular}
{\scriptsize
ComCam-archiver/forwarder transfers the file to the l1-handoff machine.~

}
\begin{tabular}{p{3cm}p{13cm}}
\hline
            & Test Data \\ \hline
\end{tabular}
{\scriptsize
l1-handoff machine has image file now on local disk.~~

}
\begin{tabular}{p{3cm}p{13cm}}
\hline
            & Expected Result \\ \hline
\end{tabular}
{\scriptsize
image file now found on disk on L1-handoff with hardlinks to 2 different
file systems (OODS and DBB) services. ~\\[3\baselineskip]

}

\begin{tabular}{p{4cm}p{12cm}}
\toprule
Step 4
& Description \\ \hline
\end{tabular}
{\scriptsize
OODS service running and ingests the image file into Butler/G3 (or Gen2)
and readies the file systems for the commissioning cluster at the Base
to be able to mount and see the new files. ~ ~

}
\begin{tabular}{p{3cm}p{13cm}}
\hline
            & Test Data \\ \hline
\end{tabular}
{\scriptsize
Image file ingested to local butler for Base~

}
\begin{tabular}{p{3cm}p{13cm}}
\hline
            & Expected Result \\ \hline
\end{tabular}
{\scriptsize
Image file ingested\\[2\baselineskip]

}

\begin{tabular}{p{4cm}p{12cm}}
\toprule
Step 5
& Description \\ \hline
\end{tabular}
{\scriptsize
DBB transfers the file to NCSA thorough the DBB-gateway machines and DTN
nodes at the base.~~

}
\begin{tabular}{p{3cm}p{13cm}}
\hline
            & Expected Result \\ \hline
\end{tabular}
{\scriptsize
data file arrives at file systems at NCSA\\[2\baselineskip]

}

\begin{tabular}{p{4cm}p{12cm}}
\toprule
Step 6
& Description \\ \hline
\end{tabular}
{\scriptsize
Files are ingested into the butler/G3 at NCSA and moved to file systems
that are viewable by the RSP. ~

}
\begin{tabular}{p{3cm}p{13cm}}
\hline
            & Expected Result \\ \hline
\end{tabular}
{\scriptsize
data can be seen and retrieved by RSP. ~

}

\subsubsection{LVV-T1550 - LDM-503-10 DAQ Validation}\label{lvv-t1550}

\begin{longtable}[]{llllll}
\toprule
Version & Status & Priority & Verification Type & Owner
\\\midrule
1 & Approved & Normal &
Demonstration & Michelle Butler
\\\bottomrule
\multicolumn{6}{c}{ Open \href{https://jira.lsstcorp.org/secure/Tests.jspa\#/testCase/LVV-T1550}{LVV-T1550} in Jira } \\
\end{longtable}

\paragraph{Verification Elements}\mbox{}\\

\begin{itemize}
\item \href{https://jira.lsstcorp.org/browse/LVV-8}{LVV-8} - DMS-REQ-0018-V-01: Raw Science Image Data Acquisition

\item \href{https://jira.lsstcorp.org/browse/LVV-28}{LVV-28} - DMS-REQ-0068-V-01: Raw Science Image Metadata

\item \href{https://jira.lsstcorp.org/browse/LVV-11}{LVV-11} - DMS-REQ-0024-V-01: Raw Image Assembly

\end{itemize}

\paragraph{Test Items}\mbox{}\\

Verify that the DAQ can talk to test machines at the BDC through the
DWDM network.~


\paragraph{Predecessors}\mbox{}\\
DAQ network at the base; forwarders and L1 handoff machine must be
available to the DAQ COB at the summit, and forwarders and other test
machines must be configured and set up on the BDC networks.~






\paragraph{Test Procedure}\mbox{}\\
\begin{tabular}{p{4cm}p{12cm}}
\toprule
Step 1
& Description \\ \hline
\end{tabular}
{\scriptsize
have DAQ produce image at the summit~

}
\begin{tabular}{p{3cm}p{13cm}}
\hline
            & Expected Result \\ \hline
\end{tabular}
{\scriptsize
Image on At-archiver~

}

\begin{tabular}{p{4cm}p{12cm}}
\toprule
Step 2
& Description \\ \hline
\end{tabular}
{\scriptsize
The forwarder at the BDC should be able to have communication with the
DAQ that the image was taken, and be able to see the file.
~\\[2\baselineskip]

}
\begin{tabular}{p{3cm}p{13cm}}
\hline
            & Expected Result \\ \hline
\end{tabular}
{\scriptsize
Image available for the forwarder at the base.~~

}

\begin{tabular}{p{4cm}p{12cm}}
\toprule
Step 3
& Description \\ \hline
\end{tabular}
{\scriptsize
Communication between the forwarder and the DAQ are in place with
messages being exchanged.~ ~

}
\begin{tabular}{p{3cm}p{13cm}}
\hline
            & Expected Result \\ \hline
\end{tabular}
{\scriptsize
if messages can be exchanged, the communication has been established.
~\\[3\baselineskip]

}

\subsubsection{LVV-T1745 - Verify calculation of median relative astrometric measurement error on
20 arcminute scales}\label{lvv-t1745}

\begin{longtable}[]{llllll}
\toprule
Version & Status & Priority & Verification Type & Owner
\\\midrule
1 & Approved & Normal &
Test & Jeffrey Carlin
\\\bottomrule
\multicolumn{6}{c}{ Open \href{https://jira.lsstcorp.org/secure/Tests.jspa\#/testCase/LVV-T1745}{LVV-T1745} in Jira } \\
\end{longtable}

\paragraph{Verification Elements}\mbox{}\\

\begin{itemize}
\item \href{https://jira.lsstcorp.org/browse/LVV-3402}{LVV-3402} - DMS-REQ-0360-V-01: Median astrometric error on 20 arcmin scales

\end{itemize}

\paragraph{Test Items}\mbox{}\\

Verify that the DM system has provided the code to calculate the median
relative astrometric measurement error on 20 arcminute scales and assess
whether it meets the requirement that it shall be no more than AM2 = 10
milliarcseconds.








\paragraph{Test Procedure}\mbox{}\\
\begin{tabular}{p{4cm}p{12cm}}
\toprule
Step 1
& Description \\ \hline
\end{tabular}
{\scriptsize
Identify a dataset containing at least one field with multiple
overlapping visits.

}
\begin{tabular}{p{3cm}p{13cm}}
\hline
            & Expected Result \\ \hline
\end{tabular}
{\scriptsize
A dataset that has been ingested into a Butler repository.

}

\begin{tabular}{p{4cm}p{12cm}}
\toprule
Step 2-1
{\scriptsize from \hyperref[lvv-t860]{LVV-T860} }
& Description \\ \hline
\end{tabular}
{\scriptsize
The `path` that you will use depends on where you are running the
science pipelines. Options:\\[2\baselineskip]

\begin{itemize}
\tightlist
\item
  local (newinstall.sh - based
  install):{[}path\_to\_installation{]}/loadLSST.bash
\item
  development cluster (``lsst-dev''):
  /software/lsstsw/stack/loadLSST.bash
\item
  LSP Notebook aspect (from a terminal):
  /opt/lsst/software/stack/loadLSST.bash
\end{itemize}

From the command line, execute the commands below in the example
code:\\[2\baselineskip]

}
\begin{tabular}{p{3cm}p{13cm}}
\hline
            & Example Code \\ \hline
\end{tabular}
{\scriptsize
source `path`\\
setup lsst\_distrib

}
\begin{tabular}{p{3cm}p{13cm}}
\hline
            & Expected Result \\ \hline
\end{tabular}
{\scriptsize
Science pipeline software is available for use. If additional packages
are needed (for example, `obs' packages such as `obs\_subaru`), then
additional `setup` commands will be necessary.\\[2\baselineskip]To check
versions in use, type:\\
eups list -s

}

\begin{tabular}{p{4cm}p{12cm}}
\toprule
Step 3-1
{\scriptsize from \hyperref[lvv-t1744]{LVV-T1744} }
& Description \\ \hline
\end{tabular}
{\scriptsize
Execute `validate\_drp` on a repository containing precursor data.
Identify the path to the data, which we will call `DATA/path', then
execute the following (with additional flags specified as needed):

}
\begin{tabular}{p{3cm}p{13cm}}
\hline
            & Example Code \\ \hline
\end{tabular}
{\scriptsize
validateDrp.py `DATA/path`

}
\begin{tabular}{p{3cm}p{13cm}}
\hline
            & Expected Result \\ \hline
\end{tabular}
{\scriptsize
JSON files (and associated figures) containing the Measurements and any
associated ``extras.''

}

\begin{tabular}{p{4cm}p{12cm}}
\toprule
Step 4
& Description \\ \hline
\end{tabular}
{\scriptsize
Confirm that the metric AM2 has been calculated, and that its values are
reasonable.

}
\begin{tabular}{p{3cm}p{13cm}}
\hline
            & Expected Result \\ \hline
\end{tabular}
{\scriptsize
A JSON file (and/or a report generated from that JSON file)
demonstrating that AM2 has been calculated.

}

\subsubsection{LVV-T1746 - Verify calculation of fraction of relative astrometric measurement error
on 5 arcminute scales exceeding outlier limit}\label{lvv-t1746}

\begin{longtable}[]{llllll}
\toprule
Version & Status & Priority & Verification Type & Owner
\\\midrule
1 & Approved & Normal &
Test & Jeffrey Carlin
\\\bottomrule
\multicolumn{6}{c}{ Open \href{https://jira.lsstcorp.org/secure/Tests.jspa\#/testCase/LVV-T1746}{LVV-T1746} in Jira } \\
\end{longtable}

\paragraph{Verification Elements}\mbox{}\\

\begin{itemize}
\item \href{https://jira.lsstcorp.org/browse/LVV-9767}{LVV-9767} - DMS-REQ-0360-V-02: Max fraction exceeding limit on 5 arcmin scales

\item \href{https://jira.lsstcorp.org/browse/LVV-9773}{LVV-9773} - DMS-REQ-0360-V-07: Outlier limit on 5 arcmin scales

\end{itemize}

\paragraph{Test Items}\mbox{}\\

Verify that the DM system has provided the code to calculate the maximum
fraction of relative astrometric measurements on 5 arcminute scales that
exceed the 5 arcminute outlier limit \textbf{AD1 = 20 milliarcseconds},
and assess whether it meets the requirement that it shall be less than
\textbf{AF1 = 10 percent.}








\paragraph{Test Procedure}\mbox{}\\
\begin{tabular}{p{4cm}p{12cm}}
\toprule
Step 1
& Description \\ \hline
\end{tabular}
{\scriptsize
Identify a dataset containing at least one field with multiple
overlapping visits.

}
\begin{tabular}{p{3cm}p{13cm}}
\hline
            & Expected Result \\ \hline
\end{tabular}
{\scriptsize
A dataset that has been ingested into a Butler repository.

}

\begin{tabular}{p{4cm}p{12cm}}
\toprule
Step 2-1
{\scriptsize from \hyperref[lvv-t860]{LVV-T860} }
& Description \\ \hline
\end{tabular}
{\scriptsize
The `path` that you will use depends on where you are running the
science pipelines. Options:\\[2\baselineskip]

\begin{itemize}
\tightlist
\item
  local (newinstall.sh - based
  install):{[}path\_to\_installation{]}/loadLSST.bash
\item
  development cluster (``lsst-dev''):
  /software/lsstsw/stack/loadLSST.bash
\item
  LSP Notebook aspect (from a terminal):
  /opt/lsst/software/stack/loadLSST.bash
\end{itemize}

From the command line, execute the commands below in the example
code:\\[2\baselineskip]

}
\begin{tabular}{p{3cm}p{13cm}}
\hline
            & Example Code \\ \hline
\end{tabular}
{\scriptsize
source `path`\\
setup lsst\_distrib

}
\begin{tabular}{p{3cm}p{13cm}}
\hline
            & Expected Result \\ \hline
\end{tabular}
{\scriptsize
Science pipeline software is available for use. If additional packages
are needed (for example, `obs' packages such as `obs\_subaru`), then
additional `setup` commands will be necessary.\\[2\baselineskip]To check
versions in use, type:\\
eups list -s

}

\begin{tabular}{p{4cm}p{12cm}}
\toprule
Step 3-1
{\scriptsize from \hyperref[lvv-t1744]{LVV-T1744} }
& Description \\ \hline
\end{tabular}
{\scriptsize
Execute `validate\_drp` on a repository containing precursor data.
Identify the path to the data, which we will call `DATA/path', then
execute the following (with additional flags specified as needed):

}
\begin{tabular}{p{3cm}p{13cm}}
\hline
            & Example Code \\ \hline
\end{tabular}
{\scriptsize
validateDrp.py `DATA/path`

}
\begin{tabular}{p{3cm}p{13cm}}
\hline
            & Expected Result \\ \hline
\end{tabular}
{\scriptsize
JSON files (and associated figures) containing the Measurements and any
associated ``extras.''

}

\begin{tabular}{p{4cm}p{12cm}}
\toprule
Step 4
& Description \\ \hline
\end{tabular}
{\scriptsize
Confirm that the metric AF1 has been calculated using the outlier limit
AD1, and that its values are reasonable.

}
\begin{tabular}{p{3cm}p{13cm}}
\hline
            & Expected Result \\ \hline
\end{tabular}
{\scriptsize
A JSON file (and/or a report generated from that JSON file)
demonstrating that AF1 has been calculated (and used the limit AD1).

}

\subsubsection{LVV-T1747 - Verify calculation of relative astrometric measurement error on 5
arcminute scales}\label{lvv-t1747}

\begin{longtable}[]{llllll}
\toprule
Version & Status & Priority & Verification Type & Owner
\\\midrule
1 & Approved & Normal &
Test & Jeffrey Carlin
\\\bottomrule
\multicolumn{6}{c}{ Open \href{https://jira.lsstcorp.org/secure/Tests.jspa\#/testCase/LVV-T1747}{LVV-T1747} in Jira } \\
\end{longtable}

\paragraph{Verification Elements}\mbox{}\\

\begin{itemize}
\item \href{https://jira.lsstcorp.org/browse/LVV-9768}{LVV-9768} - DMS-REQ-0360-V-03: Median astrometric error on 5 arcmin scales

\end{itemize}

\paragraph{Test Items}\mbox{}\\

Verify that the DM system has provided the code to calculate the
relative astrometric measurement error on 5 arcminute scales, and assess
whether it meets the requirement that it shall be less
than\textbf{~\textbf{AM1 = 10 milliarcseconds.}}








\paragraph{Test Procedure}\mbox{}\\
\begin{tabular}{p{4cm}p{12cm}}
\toprule
Step 1
& Description \\ \hline
\end{tabular}
{\scriptsize
Identify a dataset containing at least one field with multiple
overlapping visits.

}
\begin{tabular}{p{3cm}p{13cm}}
\hline
            & Expected Result \\ \hline
\end{tabular}
{\scriptsize
A dataset that has been ingested into a Butler repository.

}

\begin{tabular}{p{4cm}p{12cm}}
\toprule
Step 2-1
{\scriptsize from \hyperref[lvv-t860]{LVV-T860} }
& Description \\ \hline
\end{tabular}
{\scriptsize
The `path` that you will use depends on where you are running the
science pipelines. Options:\\[2\baselineskip]

\begin{itemize}
\tightlist
\item
  local (newinstall.sh - based
  install):{[}path\_to\_installation{]}/loadLSST.bash
\item
  development cluster (``lsst-dev''):
  /software/lsstsw/stack/loadLSST.bash
\item
  LSP Notebook aspect (from a terminal):
  /opt/lsst/software/stack/loadLSST.bash
\end{itemize}

From the command line, execute the commands below in the example
code:\\[2\baselineskip]

}
\begin{tabular}{p{3cm}p{13cm}}
\hline
            & Example Code \\ \hline
\end{tabular}
{\scriptsize
source `path`\\
setup lsst\_distrib

}
\begin{tabular}{p{3cm}p{13cm}}
\hline
            & Expected Result \\ \hline
\end{tabular}
{\scriptsize
Science pipeline software is available for use. If additional packages
are needed (for example, `obs' packages such as `obs\_subaru`), then
additional `setup` commands will be necessary.\\[2\baselineskip]To check
versions in use, type:\\
eups list -s

}

\begin{tabular}{p{4cm}p{12cm}}
\toprule
Step 3-1
{\scriptsize from \hyperref[lvv-t1744]{LVV-T1744} }
& Description \\ \hline
\end{tabular}
{\scriptsize
Execute `validate\_drp` on a repository containing precursor data.
Identify the path to the data, which we will call `DATA/path', then
execute the following (with additional flags specified as needed):

}
\begin{tabular}{p{3cm}p{13cm}}
\hline
            & Example Code \\ \hline
\end{tabular}
{\scriptsize
validateDrp.py `DATA/path`

}
\begin{tabular}{p{3cm}p{13cm}}
\hline
            & Expected Result \\ \hline
\end{tabular}
{\scriptsize
JSON files (and associated figures) containing the Measurements and any
associated ``extras.''

}

\begin{tabular}{p{4cm}p{12cm}}
\toprule
Step 4
& Description \\ \hline
\end{tabular}
{\scriptsize
Confirm that the metric AM1 has been calculated, and that its values are
reasonable.

}
\begin{tabular}{p{3cm}p{13cm}}
\hline
            & Expected Result \\ \hline
\end{tabular}
{\scriptsize
A JSON file (and/or a report generated from that JSON file)
demonstrating that AM1 has been calculated.

}

\subsubsection{LVV-T1748 - Verify calculation of median error in absolute position for RA, Dec axes}\label{lvv-t1748}

\begin{longtable}[]{llllll}
\toprule
Version & Status & Priority & Verification Type & Owner
\\\midrule
1 & Approved & Normal &
Test & Jeffrey Carlin
\\\bottomrule
\multicolumn{6}{c}{ Open \href{https://jira.lsstcorp.org/secure/Tests.jspa\#/testCase/LVV-T1748}{LVV-T1748} in Jira } \\
\end{longtable}

\paragraph{Verification Elements}\mbox{}\\

\begin{itemize}
\item \href{https://jira.lsstcorp.org/browse/LVV-9769}{LVV-9769} - DMS-REQ-0360-V-04: Median absolute error in RA, Dec

\end{itemize}

\paragraph{Test Items}\mbox{}\\

Verify that the DM system has provided the code to calculate the median
error in absolute position for each axis, RA and DEC, and assess whether
it meets the requirement that it shall be less than \textbf{AA1 = 50
milliarcseconds}.








\paragraph{Test Procedure}\mbox{}\\
\begin{tabular}{p{4cm}p{12cm}}
\toprule
Step 1
& Description \\ \hline
\end{tabular}
{\scriptsize
Identify a dataset containing at least one field with multiple
overlapping visits.

}
\begin{tabular}{p{3cm}p{13cm}}
\hline
            & Expected Result \\ \hline
\end{tabular}
{\scriptsize
A dataset that has been ingested into a Butler repository.

}

\begin{tabular}{p{4cm}p{12cm}}
\toprule
Step 2-1
{\scriptsize from \hyperref[lvv-t860]{LVV-T860} }
& Description \\ \hline
\end{tabular}
{\scriptsize
The `path` that you will use depends on where you are running the
science pipelines. Options:\\[2\baselineskip]

\begin{itemize}
\tightlist
\item
  local (newinstall.sh - based
  install):{[}path\_to\_installation{]}/loadLSST.bash
\item
  development cluster (``lsst-dev''):
  /software/lsstsw/stack/loadLSST.bash
\item
  LSP Notebook aspect (from a terminal):
  /opt/lsst/software/stack/loadLSST.bash
\end{itemize}

From the command line, execute the commands below in the example
code:\\[2\baselineskip]

}
\begin{tabular}{p{3cm}p{13cm}}
\hline
            & Example Code \\ \hline
\end{tabular}
{\scriptsize
source `path`\\
setup lsst\_distrib

}
\begin{tabular}{p{3cm}p{13cm}}
\hline
            & Expected Result \\ \hline
\end{tabular}
{\scriptsize
Science pipeline software is available for use. If additional packages
are needed (for example, `obs' packages such as `obs\_subaru`), then
additional `setup` commands will be necessary.\\[2\baselineskip]To check
versions in use, type:\\
eups list -s

}

\begin{tabular}{p{4cm}p{12cm}}
\toprule
Step 3-1
{\scriptsize from \hyperref[lvv-t1744]{LVV-T1744} }
& Description \\ \hline
\end{tabular}
{\scriptsize
Execute `validate\_drp` on a repository containing precursor data.
Identify the path to the data, which we will call `DATA/path', then
execute the following (with additional flags specified as needed):

}
\begin{tabular}{p{3cm}p{13cm}}
\hline
            & Example Code \\ \hline
\end{tabular}
{\scriptsize
validateDrp.py `DATA/path`

}
\begin{tabular}{p{3cm}p{13cm}}
\hline
            & Expected Result \\ \hline
\end{tabular}
{\scriptsize
JSON files (and associated figures) containing the Measurements and any
associated ``extras.''

}

\begin{tabular}{p{4cm}p{12cm}}
\toprule
Step 4
& Description \\ \hline
\end{tabular}
{\scriptsize
Confirm that the metric AA1 has been calculated, and that its values are
reasonable.

}
\begin{tabular}{p{3cm}p{13cm}}
\hline
            & Expected Result \\ \hline
\end{tabular}
{\scriptsize
A JSON file (and/or a report generated from that JSON file)
demonstrating that AA1 has been calculated.

}

\subsubsection{LVV-T1749 - Verify calculation of fraction of relative astrometric measurement error
on 20 arcminute scales exceeding outlier limit}\label{lvv-t1749}

\begin{longtable}[]{llllll}
\toprule
Version & Status & Priority & Verification Type & Owner
\\\midrule
1 & Approved & Normal &
Test & Jeffrey Carlin
\\\bottomrule
\multicolumn{6}{c}{ Open \href{https://jira.lsstcorp.org/secure/Tests.jspa\#/testCase/LVV-T1749}{LVV-T1749} in Jira } \\
\end{longtable}

\paragraph{Verification Elements}\mbox{}\\

\begin{itemize}
\item \href{https://jira.lsstcorp.org/browse/LVV-9776}{LVV-9776} - DMS-REQ-0360-V-10: Max fraction exceeding limit on 20 arcmin scales

\item \href{https://jira.lsstcorp.org/browse/LVV-9770}{LVV-9770} - DMS-REQ-0360-V-05: Outlier limit on 20 arcmin scales

\end{itemize}

\paragraph{Test Items}\mbox{}\\

Verify that the DM system has provided the code to calculate the maximum
fraction of relative astrometric measurements on 20 arcminute scales
that exceed the 20 arcminute outlier limit \textbf{AD2 = 20
milliarcseconds}, and assess whether it meets the requirement that it
shall be less than \textbf{AF2 = 10 percent.}








\paragraph{Test Procedure}\mbox{}\\
\begin{tabular}{p{4cm}p{12cm}}
\toprule
Step 1
& Description \\ \hline
\end{tabular}
{\scriptsize
Identify a dataset containing at least one field with multiple
overlapping visits.

}
\begin{tabular}{p{3cm}p{13cm}}
\hline
            & Expected Result \\ \hline
\end{tabular}
{\scriptsize
A dataset that has been ingested into a Butler repository.

}

\begin{tabular}{p{4cm}p{12cm}}
\toprule
Step 2-1
{\scriptsize from \hyperref[lvv-t860]{LVV-T860} }
& Description \\ \hline
\end{tabular}
{\scriptsize
The `path` that you will use depends on where you are running the
science pipelines. Options:\\[2\baselineskip]

\begin{itemize}
\tightlist
\item
  local (newinstall.sh - based
  install):{[}path\_to\_installation{]}/loadLSST.bash
\item
  development cluster (``lsst-dev''):
  /software/lsstsw/stack/loadLSST.bash
\item
  LSP Notebook aspect (from a terminal):
  /opt/lsst/software/stack/loadLSST.bash
\end{itemize}

From the command line, execute the commands below in the example
code:\\[2\baselineskip]

}
\begin{tabular}{p{3cm}p{13cm}}
\hline
            & Example Code \\ \hline
\end{tabular}
{\scriptsize
source `path`\\
setup lsst\_distrib

}
\begin{tabular}{p{3cm}p{13cm}}
\hline
            & Expected Result \\ \hline
\end{tabular}
{\scriptsize
Science pipeline software is available for use. If additional packages
are needed (for example, `obs' packages such as `obs\_subaru`), then
additional `setup` commands will be necessary.\\[2\baselineskip]To check
versions in use, type:\\
eups list -s

}

\begin{tabular}{p{4cm}p{12cm}}
\toprule
Step 3-1
{\scriptsize from \hyperref[lvv-t1744]{LVV-T1744} }
& Description \\ \hline
\end{tabular}
{\scriptsize
Execute `validate\_drp` on a repository containing precursor data.
Identify the path to the data, which we will call `DATA/path', then
execute the following (with additional flags specified as needed):

}
\begin{tabular}{p{3cm}p{13cm}}
\hline
            & Example Code \\ \hline
\end{tabular}
{\scriptsize
validateDrp.py `DATA/path`

}
\begin{tabular}{p{3cm}p{13cm}}
\hline
            & Expected Result \\ \hline
\end{tabular}
{\scriptsize
JSON files (and associated figures) containing the Measurements and any
associated ``extras.''

}

\begin{tabular}{p{4cm}p{12cm}}
\toprule
Step 4
& Description \\ \hline
\end{tabular}
{\scriptsize
Confirm that the metric AF2 has been calculated using the outlier limit
AD2, and that its values are reasonable.

}
\begin{tabular}{p{3cm}p{13cm}}
\hline
            & Expected Result \\ \hline
\end{tabular}
{\scriptsize
A JSON file (and/or a report generated from that JSON file)
demonstrating that AF2 has been calculated (and used the limit AD2).

}

\subsubsection{LVV-T1750 - Verify calculation of separations relative to r-band exceeding color
difference outlier limit}\label{lvv-t1750}

\begin{longtable}[]{llllll}
\toprule
Version & Status & Priority & Verification Type & Owner
\\\midrule
1 & Approved & Normal &
Test & Jeffrey Carlin
\\\bottomrule
\multicolumn{6}{c}{ Open \href{https://jira.lsstcorp.org/secure/Tests.jspa\#/testCase/LVV-T1750}{LVV-T1750} in Jira } \\
\end{longtable}

\paragraph{Verification Elements}\mbox{}\\

\begin{itemize}
\item \href{https://jira.lsstcorp.org/browse/LVV-9771}{LVV-9771} - DMS-REQ-0360-V-06: Color difference outlier limit relative to r-band

\item \href{https://jira.lsstcorp.org/browse/LVV-9777}{LVV-9777} - DMS-REQ-0360-V-11: Max fraction of r-band color difference outliers

\end{itemize}

\paragraph{Test Items}\mbox{}\\

Verify that the DM system has provided the code to calculate the
separations measured relative to the r-band that exceed the color
difference outlier limit \textbf{AB2 = 20 milliarcseconds}, and assess
whether it meets the requirement that it shall be less than \textbf{ABF1
= 10 percent.~}








\paragraph{Test Procedure}\mbox{}\\
\begin{tabular}{p{4cm}p{12cm}}
\toprule
Step 1
& Description \\ \hline
\end{tabular}
{\scriptsize
Identify a dataset containing at least one field with multiple
overlapping visits, and including at least one visit in r-band.

}
\begin{tabular}{p{3cm}p{13cm}}
\hline
            & Expected Result \\ \hline
\end{tabular}
{\scriptsize
A dataset that has been ingested into a Butler repository.

}

\begin{tabular}{p{4cm}p{12cm}}
\toprule
Step 2-1
{\scriptsize from \hyperref[lvv-t860]{LVV-T860} }
& Description \\ \hline
\end{tabular}
{\scriptsize
The `path` that you will use depends on where you are running the
science pipelines. Options:\\[2\baselineskip]

\begin{itemize}
\tightlist
\item
  local (newinstall.sh - based
  install):{[}path\_to\_installation{]}/loadLSST.bash
\item
  development cluster (``lsst-dev''):
  /software/lsstsw/stack/loadLSST.bash
\item
  LSP Notebook aspect (from a terminal):
  /opt/lsst/software/stack/loadLSST.bash
\end{itemize}

From the command line, execute the commands below in the example
code:\\[2\baselineskip]

}
\begin{tabular}{p{3cm}p{13cm}}
\hline
            & Example Code \\ \hline
\end{tabular}
{\scriptsize
source `path`\\
setup lsst\_distrib

}
\begin{tabular}{p{3cm}p{13cm}}
\hline
            & Expected Result \\ \hline
\end{tabular}
{\scriptsize
Science pipeline software is available for use. If additional packages
are needed (for example, `obs' packages such as `obs\_subaru`), then
additional `setup` commands will be necessary.\\[2\baselineskip]To check
versions in use, type:\\
eups list -s

}

\begin{tabular}{p{4cm}p{12cm}}
\toprule
Step 3-1
{\scriptsize from \hyperref[lvv-t1744]{LVV-T1744} }
& Description \\ \hline
\end{tabular}
{\scriptsize
Execute `validate\_drp` on a repository containing precursor data.
Identify the path to the data, which we will call `DATA/path', then
execute the following (with additional flags specified as needed):

}
\begin{tabular}{p{3cm}p{13cm}}
\hline
            & Example Code \\ \hline
\end{tabular}
{\scriptsize
validateDrp.py `DATA/path`

}
\begin{tabular}{p{3cm}p{13cm}}
\hline
            & Expected Result \\ \hline
\end{tabular}
{\scriptsize
JSON files (and associated figures) containing the Measurements and any
associated ``extras.''

}

\begin{tabular}{p{4cm}p{12cm}}
\toprule
Step 4
& Description \\ \hline
\end{tabular}
{\scriptsize
Confirm that the metric ABF1 has been calculated using the outlier limit
AB2, and that its values are reasonable.

}
\begin{tabular}{p{3cm}p{13cm}}
\hline
            & Expected Result \\ \hline
\end{tabular}
{\scriptsize
A JSON file (and/or a report generated from that JSON file)
demonstrating that ABF1 has been calculated (and used the limit AB2).

}

\subsubsection{LVV-T1751 - Verify calculation of median relative astrometric measurement error on
200 arcminute scales}\label{lvv-t1751}

\begin{longtable}[]{llllll}
\toprule
Version & Status & Priority & Verification Type & Owner
\\\midrule
1 & Approved & Normal &
Test & Jeffrey Carlin
\\\bottomrule
\multicolumn{6}{c}{ Open \href{https://jira.lsstcorp.org/secure/Tests.jspa\#/testCase/LVV-T1751}{LVV-T1751} in Jira } \\
\end{longtable}

\paragraph{Verification Elements}\mbox{}\\

\begin{itemize}
\item \href{https://jira.lsstcorp.org/browse/LVV-9774}{LVV-9774} - DMS-REQ-0360-V-08: Median astrometric error on 200 arcmin scales

\end{itemize}

\paragraph{Test Items}\mbox{}\\

Verify that the DM system has provided the code to calculate the median
relative astrometric measurement error on 200 arcminute scales and
assess whether it meets the requirement that it shall be no more than
AM3 = 15 milliarcseconds.








\paragraph{Test Procedure}\mbox{}\\
\begin{tabular}{p{4cm}p{12cm}}
\toprule
Step 1
& Description \\ \hline
\end{tabular}
{\scriptsize
Identify a dataset containing at least one field with multiple
overlapping visits, and that covers an area larger than 200 arcminutes.

}
\begin{tabular}{p{3cm}p{13cm}}
\hline
            & Expected Result \\ \hline
\end{tabular}
{\scriptsize
A dataset that has been ingested into a Butler repository.

}

\begin{tabular}{p{4cm}p{12cm}}
\toprule
Step 2-1
{\scriptsize from \hyperref[lvv-t860]{LVV-T860} }
& Description \\ \hline
\end{tabular}
{\scriptsize
The `path` that you will use depends on where you are running the
science pipelines. Options:\\[2\baselineskip]

\begin{itemize}
\tightlist
\item
  local (newinstall.sh - based
  install):{[}path\_to\_installation{]}/loadLSST.bash
\item
  development cluster (``lsst-dev''):
  /software/lsstsw/stack/loadLSST.bash
\item
  LSP Notebook aspect (from a terminal):
  /opt/lsst/software/stack/loadLSST.bash
\end{itemize}

From the command line, execute the commands below in the example
code:\\[2\baselineskip]

}
\begin{tabular}{p{3cm}p{13cm}}
\hline
            & Example Code \\ \hline
\end{tabular}
{\scriptsize
source `path`\\
setup lsst\_distrib

}
\begin{tabular}{p{3cm}p{13cm}}
\hline
            & Expected Result \\ \hline
\end{tabular}
{\scriptsize
Science pipeline software is available for use. If additional packages
are needed (for example, `obs' packages such as `obs\_subaru`), then
additional `setup` commands will be necessary.\\[2\baselineskip]To check
versions in use, type:\\
eups list -s

}

\begin{tabular}{p{4cm}p{12cm}}
\toprule
Step 3-1
{\scriptsize from \hyperref[lvv-t1744]{LVV-T1744} }
& Description \\ \hline
\end{tabular}
{\scriptsize
Execute `validate\_drp` on a repository containing precursor data.
Identify the path to the data, which we will call `DATA/path', then
execute the following (with additional flags specified as needed):

}
\begin{tabular}{p{3cm}p{13cm}}
\hline
            & Example Code \\ \hline
\end{tabular}
{\scriptsize
validateDrp.py `DATA/path`

}
\begin{tabular}{p{3cm}p{13cm}}
\hline
            & Expected Result \\ \hline
\end{tabular}
{\scriptsize
JSON files (and associated figures) containing the Measurements and any
associated ``extras.''

}

\begin{tabular}{p{4cm}p{12cm}}
\toprule
Step 4
& Description \\ \hline
\end{tabular}
{\scriptsize
Confirm that the metric AM3 has been calculated, and that its values are
reasonable.

}
\begin{tabular}{p{3cm}p{13cm}}
\hline
            & Expected Result \\ \hline
\end{tabular}
{\scriptsize
A JSON file (and/or a report generated from that JSON file)
demonstrating that AM3 has been calculated.

}

\subsubsection{LVV-T1752 - Verify calculation of fraction of relative astrometric measurement error
on 200 arcminute scales exceeding outlier limit}\label{lvv-t1752}

\begin{longtable}[]{llllll}
\toprule
Version & Status & Priority & Verification Type & Owner
\\\midrule
1 & Approved & Normal &
Test & Jeffrey Carlin
\\\bottomrule
\multicolumn{6}{c}{ Open \href{https://jira.lsstcorp.org/secure/Tests.jspa\#/testCase/LVV-T1752}{LVV-T1752} in Jira } \\
\end{longtable}

\paragraph{Verification Elements}\mbox{}\\

\begin{itemize}
\item \href{https://jira.lsstcorp.org/browse/LVV-9779}{LVV-9779} - DMS-REQ-0360-V-13: Max fraction exceeding limit on 200 arcmin scales

\end{itemize}

\paragraph{Test Items}\mbox{}\\

Verify that the DM system has provided the code to calculate the maximum
fraction of relative astrometric measurements on 200 arcminute scales
that exceed the 200 arcminute outlier limit \textbf{AD3 = 30
milliarcseconds}, and assess whether it meets the requirement that it
shall be less than \textbf{AF3 = 10 percent.}








\paragraph{Test Procedure}\mbox{}\\
\begin{tabular}{p{4cm}p{12cm}}
\toprule
Step 1
& Description \\ \hline
\end{tabular}
{\scriptsize
Identify a dataset containing at least one field with multiple
overlapping visits, and that covers an area larger than 200 arcminutes.

}
\begin{tabular}{p{3cm}p{13cm}}
\hline
            & Expected Result \\ \hline
\end{tabular}
{\scriptsize
A dataset that has been ingested into a Butler repository.

}

\begin{tabular}{p{4cm}p{12cm}}
\toprule
Step 2-1
{\scriptsize from \hyperref[lvv-t860]{LVV-T860} }
& Description \\ \hline
\end{tabular}
{\scriptsize
The `path` that you will use depends on where you are running the
science pipelines. Options:\\[2\baselineskip]

\begin{itemize}
\tightlist
\item
  local (newinstall.sh - based
  install):{[}path\_to\_installation{]}/loadLSST.bash
\item
  development cluster (``lsst-dev''):
  /software/lsstsw/stack/loadLSST.bash
\item
  LSP Notebook aspect (from a terminal):
  /opt/lsst/software/stack/loadLSST.bash
\end{itemize}

From the command line, execute the commands below in the example
code:\\[2\baselineskip]

}
\begin{tabular}{p{3cm}p{13cm}}
\hline
            & Example Code \\ \hline
\end{tabular}
{\scriptsize
source `path`\\
setup lsst\_distrib

}
\begin{tabular}{p{3cm}p{13cm}}
\hline
            & Expected Result \\ \hline
\end{tabular}
{\scriptsize
Science pipeline software is available for use. If additional packages
are needed (for example, `obs' packages such as `obs\_subaru`), then
additional `setup` commands will be necessary.\\[2\baselineskip]To check
versions in use, type:\\
eups list -s

}

\begin{tabular}{p{4cm}p{12cm}}
\toprule
Step 3-1
{\scriptsize from \hyperref[lvv-t1744]{LVV-T1744} }
& Description \\ \hline
\end{tabular}
{\scriptsize
Execute `validate\_drp` on a repository containing precursor data.
Identify the path to the data, which we will call `DATA/path', then
execute the following (with additional flags specified as needed):

}
\begin{tabular}{p{3cm}p{13cm}}
\hline
            & Example Code \\ \hline
\end{tabular}
{\scriptsize
validateDrp.py `DATA/path`

}
\begin{tabular}{p{3cm}p{13cm}}
\hline
            & Expected Result \\ \hline
\end{tabular}
{\scriptsize
JSON files (and associated figures) containing the Measurements and any
associated ``extras.''

}

\begin{tabular}{p{4cm}p{12cm}}
\toprule
Step 4
& Description \\ \hline
\end{tabular}
{\scriptsize
Confirm that the metric AF3 has been calculated using the outlier limit
AD3, and that its values are reasonable.

}
\begin{tabular}{p{3cm}p{13cm}}
\hline
            & Expected Result \\ \hline
\end{tabular}
{\scriptsize
A JSON file (and/or a report generated from that JSON file)
demonstrating that AF3 has been calculated (and used the limit AD3).

}

\subsubsection{LVV-T1753 - Verify calculation of RMS difference of separations relative to r-band}\label{lvv-t1753}

\begin{longtable}[]{llllll}
\toprule
Version & Status & Priority & Verification Type & Owner
\\\midrule
1 & Approved & Normal &
Test & Jeffrey Carlin
\\\bottomrule
\multicolumn{6}{c}{ Open \href{https://jira.lsstcorp.org/secure/Tests.jspa\#/testCase/LVV-T1753}{LVV-T1753} in Jira } \\
\end{longtable}

\paragraph{Verification Elements}\mbox{}\\

\begin{itemize}
\item \href{https://jira.lsstcorp.org/browse/LVV-9778}{LVV-9778} - DMS-REQ-0360-V-12: RMS difference between r-band and other filter
separation

\end{itemize}

\paragraph{Test Items}\mbox{}\\

Verify that the DM system has provided the code to calculate the
separations measured relative to the r-band, and assess whether it meets
the requirement that it shall be less than \textbf{AB1 =
10~milliarcseconds.}








\paragraph{Test Procedure}\mbox{}\\
\begin{tabular}{p{4cm}p{12cm}}
\toprule
Step 1
& Description \\ \hline
\end{tabular}
{\scriptsize
Identify a dataset containing at least one field with multiple
overlapping visits, and including at least one visit in r-band.

}
\begin{tabular}{p{3cm}p{13cm}}
\hline
            & Expected Result \\ \hline
\end{tabular}
{\scriptsize
A dataset that has been ingested into a Butler repository.

}

\begin{tabular}{p{4cm}p{12cm}}
\toprule
Step 2-1
{\scriptsize from \hyperref[lvv-t860]{LVV-T860} }
& Description \\ \hline
\end{tabular}
{\scriptsize
The `path` that you will use depends on where you are running the
science pipelines. Options:\\[2\baselineskip]

\begin{itemize}
\tightlist
\item
  local (newinstall.sh - based
  install):{[}path\_to\_installation{]}/loadLSST.bash
\item
  development cluster (``lsst-dev''):
  /software/lsstsw/stack/loadLSST.bash
\item
  LSP Notebook aspect (from a terminal):
  /opt/lsst/software/stack/loadLSST.bash
\end{itemize}

From the command line, execute the commands below in the example
code:\\[2\baselineskip]

}
\begin{tabular}{p{3cm}p{13cm}}
\hline
            & Example Code \\ \hline
\end{tabular}
{\scriptsize
source `path`\\
setup lsst\_distrib

}
\begin{tabular}{p{3cm}p{13cm}}
\hline
            & Expected Result \\ \hline
\end{tabular}
{\scriptsize
Science pipeline software is available for use. If additional packages
are needed (for example, `obs' packages such as `obs\_subaru`), then
additional `setup` commands will be necessary.\\[2\baselineskip]To check
versions in use, type:\\
eups list -s

}

\begin{tabular}{p{4cm}p{12cm}}
\toprule
Step 3-1
{\scriptsize from \hyperref[lvv-t1744]{LVV-T1744} }
& Description \\ \hline
\end{tabular}
{\scriptsize
Execute `validate\_drp` on a repository containing precursor data.
Identify the path to the data, which we will call `DATA/path', then
execute the following (with additional flags specified as needed):

}
\begin{tabular}{p{3cm}p{13cm}}
\hline
            & Example Code \\ \hline
\end{tabular}
{\scriptsize
validateDrp.py `DATA/path`

}
\begin{tabular}{p{3cm}p{13cm}}
\hline
            & Expected Result \\ \hline
\end{tabular}
{\scriptsize
JSON files (and associated figures) containing the Measurements and any
associated ``extras.''

}

\begin{tabular}{p{4cm}p{12cm}}
\toprule
Step 4
& Description \\ \hline
\end{tabular}
{\scriptsize
Confirm that the metric AB1 has been calculated, and that its values are
reasonable.

}
\begin{tabular}{p{3cm}p{13cm}}
\hline
            & Expected Result \\ \hline
\end{tabular}
{\scriptsize
A JSON file (and/or a report generated from that JSON file)
demonstrating that AB1 has been calculated.

}

\subsubsection{LVV-T1754 - Verify calculation of residual PSF ellipticity correlations for
separations less than 5 arcmin}\label{lvv-t1754}

\begin{longtable}[]{llllll}
\toprule
Version & Status & Priority & Verification Type & Owner
\\\midrule
1 & Approved & Normal &
Test & Jeffrey Carlin
\\\bottomrule
\multicolumn{6}{c}{ Open \href{https://jira.lsstcorp.org/secure/Tests.jspa\#/testCase/LVV-T1754}{LVV-T1754} in Jira } \\
\end{longtable}

\paragraph{Verification Elements}\mbox{}\\

\begin{itemize}
\item \href{https://jira.lsstcorp.org/browse/LVV-3404}{LVV-3404} - DMS-REQ-0362-V-01: Median residual PSF ellipticity correlations on 5
arcmin scales

\end{itemize}

\paragraph{Test Items}\mbox{}\\

Verify that the DM system has provided the code to calculate the median
residual PSF ellipticity correlations averaged over an arbitrary field
of view for separations less than 5 arcmin, and assess whether it meets
the requirement that it shall be no greater than \textbf{TE2 =
1.0e-7{[}arcminuteSeparationCorrelation{]}.}








\paragraph{Test Procedure}\mbox{}\\
\begin{tabular}{p{4cm}p{12cm}}
\toprule
Step 1
& Description \\ \hline
\end{tabular}
{\scriptsize
Identify a dataset containing at least one field with multiple
overlapping visits.

}
\begin{tabular}{p{3cm}p{13cm}}
\hline
            & Expected Result \\ \hline
\end{tabular}
{\scriptsize
A dataset that has been ingested into a Butler repository.

}

\begin{tabular}{p{4cm}p{12cm}}
\toprule
Step 2-1
{\scriptsize from \hyperref[lvv-t860]{LVV-T860} }
& Description \\ \hline
\end{tabular}
{\scriptsize
The `path` that you will use depends on where you are running the
science pipelines. Options:\\[2\baselineskip]

\begin{itemize}
\tightlist
\item
  local (newinstall.sh - based
  install):{[}path\_to\_installation{]}/loadLSST.bash
\item
  development cluster (``lsst-dev''):
  /software/lsstsw/stack/loadLSST.bash
\item
  LSP Notebook aspect (from a terminal):
  /opt/lsst/software/stack/loadLSST.bash
\end{itemize}

From the command line, execute the commands below in the example
code:\\[2\baselineskip]

}
\begin{tabular}{p{3cm}p{13cm}}
\hline
            & Example Code \\ \hline
\end{tabular}
{\scriptsize
source `path`\\
setup lsst\_distrib

}
\begin{tabular}{p{3cm}p{13cm}}
\hline
            & Expected Result \\ \hline
\end{tabular}
{\scriptsize
Science pipeline software is available for use. If additional packages
are needed (for example, `obs' packages such as `obs\_subaru`), then
additional `setup` commands will be necessary.\\[2\baselineskip]To check
versions in use, type:\\
eups list -s

}

\begin{tabular}{p{4cm}p{12cm}}
\toprule
Step 3-1
{\scriptsize from \hyperref[lvv-t1744]{LVV-T1744} }
& Description \\ \hline
\end{tabular}
{\scriptsize
Execute `validate\_drp` on a repository containing precursor data.
Identify the path to the data, which we will call `DATA/path', then
execute the following (with additional flags specified as needed):

}
\begin{tabular}{p{3cm}p{13cm}}
\hline
            & Example Code \\ \hline
\end{tabular}
{\scriptsize
validateDrp.py `DATA/path`

}
\begin{tabular}{p{3cm}p{13cm}}
\hline
            & Expected Result \\ \hline
\end{tabular}
{\scriptsize
JSON files (and associated figures) containing the Measurements and any
associated ``extras.''

}

\begin{tabular}{p{4cm}p{12cm}}
\toprule
Step 4
& Description \\ \hline
\end{tabular}
{\scriptsize
Confirm that the metric TE2 has been calculated, and that its values are
reasonable.

}
\begin{tabular}{p{3cm}p{13cm}}
\hline
            & Expected Result \\ \hline
\end{tabular}
{\scriptsize
A JSON file (and/or a report generated from that JSON file)
demonstrating that TE2 has been calculated.

}

\subsubsection{LVV-T1755 - Verify calculation of residual PSF ellipticity correlations for
separations less than 1 arcmin}\label{lvv-t1755}

\begin{longtable}[]{llllll}
\toprule
Version & Status & Priority & Verification Type & Owner
\\\midrule
1 & Approved & Normal &
Test & Jeffrey Carlin
\\\bottomrule
\multicolumn{6}{c}{ Open \href{https://jira.lsstcorp.org/secure/Tests.jspa\#/testCase/LVV-T1755}{LVV-T1755} in Jira } \\
\end{longtable}

\paragraph{Verification Elements}\mbox{}\\

\begin{itemize}
\item \href{https://jira.lsstcorp.org/browse/LVV-9782}{LVV-9782} - DMS-REQ-0362-V-04: Median residual PSF ellipticity correlations on 1
arcmin scales

\end{itemize}

\paragraph{Test Items}\mbox{}\\

Verify that the DM system has provided the code to calculate the median
residual PSF ellipticity correlations averaged over an arbitrary field
of view for separations less than 1 arcmin, and assess whether it meets
the requirement that it shall be no greater than \textbf{TE1 =
2.0e-5{[}arcminuteSeparationCorrelation{]}.}








\paragraph{Test Procedure}\mbox{}\\
\begin{tabular}{p{4cm}p{12cm}}
\toprule
Step 1
& Description \\ \hline
\end{tabular}
{\scriptsize
Identify a dataset containing at least one field with multiple
overlapping visits.

}
\begin{tabular}{p{3cm}p{13cm}}
\hline
            & Expected Result \\ \hline
\end{tabular}
{\scriptsize
A dataset that has been ingested into a Butler repository.

}

\begin{tabular}{p{4cm}p{12cm}}
\toprule
Step 2-1
{\scriptsize from \hyperref[lvv-t860]{LVV-T860} }
& Description \\ \hline
\end{tabular}
{\scriptsize
The `path` that you will use depends on where you are running the
science pipelines. Options:\\[2\baselineskip]

\begin{itemize}
\tightlist
\item
  local (newinstall.sh - based
  install):{[}path\_to\_installation{]}/loadLSST.bash
\item
  development cluster (``lsst-dev''):
  /software/lsstsw/stack/loadLSST.bash
\item
  LSP Notebook aspect (from a terminal):
  /opt/lsst/software/stack/loadLSST.bash
\end{itemize}

From the command line, execute the commands below in the example
code:\\[2\baselineskip]

}
\begin{tabular}{p{3cm}p{13cm}}
\hline
            & Example Code \\ \hline
\end{tabular}
{\scriptsize
source `path`\\
setup lsst\_distrib

}
\begin{tabular}{p{3cm}p{13cm}}
\hline
            & Expected Result \\ \hline
\end{tabular}
{\scriptsize
Science pipeline software is available for use. If additional packages
are needed (for example, `obs' packages such as `obs\_subaru`), then
additional `setup` commands will be necessary.\\[2\baselineskip]To check
versions in use, type:\\
eups list -s

}

\begin{tabular}{p{4cm}p{12cm}}
\toprule
Step 3-1
{\scriptsize from \hyperref[lvv-t1744]{LVV-T1744} }
& Description \\ \hline
\end{tabular}
{\scriptsize
Execute `validate\_drp` on a repository containing precursor data.
Identify the path to the data, which we will call `DATA/path', then
execute the following (with additional flags specified as needed):

}
\begin{tabular}{p{3cm}p{13cm}}
\hline
            & Example Code \\ \hline
\end{tabular}
{\scriptsize
validateDrp.py `DATA/path`

}
\begin{tabular}{p{3cm}p{13cm}}
\hline
            & Expected Result \\ \hline
\end{tabular}
{\scriptsize
JSON files (and associated figures) containing the Measurements and any
associated ``extras.''

}

\begin{tabular}{p{4cm}p{12cm}}
\toprule
Step 4
& Description \\ \hline
\end{tabular}
{\scriptsize
Confirm that the metric TE1 has been calculated, and that its values are
reasonable.

}
\begin{tabular}{p{3cm}p{13cm}}
\hline
            & Expected Result \\ \hline
\end{tabular}
{\scriptsize
A JSON file (and/or a report generated from that JSON file)
demonstrating that TE1 has been calculated.

}

\subsubsection{LVV-T1756 - Verify calculation of photometric repeatability in uzy filters}\label{lvv-t1756}

\begin{longtable}[]{llllll}
\toprule
Version & Status & Priority & Verification Type & Owner
\\\midrule
1 & Approved & Normal &
Test & Jeffrey Carlin
\\\bottomrule
\multicolumn{6}{c}{ Open \href{https://jira.lsstcorp.org/secure/Tests.jspa\#/testCase/LVV-T1756}{LVV-T1756} in Jira } \\
\end{longtable}

\paragraph{Verification Elements}\mbox{}\\

\begin{itemize}
\item \href{https://jira.lsstcorp.org/browse/LVV-3401}{LVV-3401} - DMS-REQ-0359-V-01: RMS photometric repeatability in uzy

\end{itemize}

\paragraph{Test Items}\mbox{}\\

Verify that the DM system has provided the code to calculate the RMS
photometric repeatability of bright non-saturated unresolved point
sources in the u, z, and y filters, and assess whether it meets the
requirement that it shall be less than \textbf{PA1uzy = 7.5
millimagnitudes}.








\paragraph{Test Procedure}\mbox{}\\
\begin{tabular}{p{4cm}p{12cm}}
\toprule
Step 1
& Description \\ \hline
\end{tabular}
{\scriptsize
Identify a dataset containing at least one field in each of the u, z,
and y filters with multiple overlapping visits.

}
\begin{tabular}{p{3cm}p{13cm}}
\hline
            & Expected Result \\ \hline
\end{tabular}
{\scriptsize
A dataset that has been ingested into a Butler repository.

}

\begin{tabular}{p{4cm}p{12cm}}
\toprule
Step 2-1
{\scriptsize from \hyperref[lvv-t1744]{LVV-T1744} }
& Description \\ \hline
\end{tabular}
{\scriptsize
Execute `validate\_drp` on a repository containing precursor data.
Identify the path to the data, which we will call `DATA/path', then
execute the following (with additional flags specified as needed):

}
\begin{tabular}{p{3cm}p{13cm}}
\hline
            & Example Code \\ \hline
\end{tabular}
{\scriptsize
validateDrp.py `DATA/path`

}
\begin{tabular}{p{3cm}p{13cm}}
\hline
            & Expected Result \\ \hline
\end{tabular}
{\scriptsize
JSON files (and associated figures) containing the Measurements and any
associated ``extras.''

}

\begin{tabular}{p{4cm}p{12cm}}
\toprule
Step 3
& Description \\ \hline
\end{tabular}
{\scriptsize
Confirm that the metric PA1uzy has been calculated, and that its values
are reasonable.

}
\begin{tabular}{p{3cm}p{13cm}}
\hline
            & Expected Result \\ \hline
\end{tabular}
{\scriptsize
A JSON file (and/or a report generated from that JSON file)
demonstrating that PA1uzy has been calculated.

}

\subsubsection{LVV-T1757 - Verify calculation of photometric repeatability in gri filters}\label{lvv-t1757}

\begin{longtable}[]{llllll}
\toprule
Version & Status & Priority & Verification Type & Owner
\\\midrule
1 & Approved & Normal &
Test & Jeffrey Carlin
\\\bottomrule
\multicolumn{6}{c}{ Open \href{https://jira.lsstcorp.org/secure/Tests.jspa\#/testCase/LVV-T1757}{LVV-T1757} in Jira } \\
\end{longtable}

\paragraph{Verification Elements}\mbox{}\\

\begin{itemize}
\item \href{https://jira.lsstcorp.org/browse/LVV-9759}{LVV-9759} - DMS-REQ-0359-V-10: RMS photometric repeatability in gri

\end{itemize}

\paragraph{Test Items}\mbox{}\\

Verify that the DM system has provided the code to calculate the RMS
photometric repeatability of bright non-saturated unresolved point
sources in the g, r, and i filters, and assess whether it meets the
requirement that it shall be less than \textbf{PA1gri = 5.0
millimagnitudes}.








\paragraph{Test Procedure}\mbox{}\\
\begin{tabular}{p{4cm}p{12cm}}
\toprule
Step 1
& Description \\ \hline
\end{tabular}
{\scriptsize
Identify a dataset containing at least one field in each of the g, r,
and i filters with multiple overlapping visits.

}
\begin{tabular}{p{3cm}p{13cm}}
\hline
            & Expected Result \\ \hline
\end{tabular}
{\scriptsize
A dataset that has been ingested into a Butler repository.

}

\begin{tabular}{p{4cm}p{12cm}}
\toprule
Step 2-1
{\scriptsize from \hyperref[lvv-t1744]{LVV-T1744} }
& Description \\ \hline
\end{tabular}
{\scriptsize
Execute `validate\_drp` on a repository containing precursor data.
Identify the path to the data, which we will call `DATA/path', then
execute the following (with additional flags specified as needed):

}
\begin{tabular}{p{3cm}p{13cm}}
\hline
            & Example Code \\ \hline
\end{tabular}
{\scriptsize
validateDrp.py `DATA/path`

}
\begin{tabular}{p{3cm}p{13cm}}
\hline
            & Expected Result \\ \hline
\end{tabular}
{\scriptsize
JSON files (and associated figures) containing the Measurements and any
associated ``extras.''

}

\begin{tabular}{p{4cm}p{12cm}}
\toprule
Step 3
& Description \\ \hline
\end{tabular}
{\scriptsize
Confirm that the metric PA1gri has been calculated, and that its values
are reasonable.

}
\begin{tabular}{p{3cm}p{13cm}}
\hline
            & Expected Result \\ \hline
\end{tabular}
{\scriptsize
A JSON file (and/or a report generated from that JSON file)
demonstrating that PA1gri has been calculated.

}

\subsubsection{LVV-T1758 - Verify calculation of photometric outliers in uzy bands}\label{lvv-t1758}

\begin{longtable}[]{llllll}
\toprule
Version & Status & Priority & Verification Type & Owner
\\\midrule
1 & Approved & Normal &
Test & Jeffrey Carlin
\\\bottomrule
\multicolumn{6}{c}{ Open \href{https://jira.lsstcorp.org/secure/Tests.jspa\#/testCase/LVV-T1758}{LVV-T1758} in Jira } \\
\end{longtable}

\paragraph{Verification Elements}\mbox{}\\

\begin{itemize}
\item \href{https://jira.lsstcorp.org/browse/LVV-9758}{LVV-9758} - DMS-REQ-0359-V-09: Repeatability outlier limit in uzy

\item \href{https://jira.lsstcorp.org/browse/LVV-9752}{LVV-9752} - DMS-REQ-0359-V-03: Max fraction of outliers among non-saturated sources

\end{itemize}

\paragraph{Test Items}\mbox{}\\

Verify that the DM system has provided the code to calculate the
photometric repeatability in the u, z, and y filters, and assess whether
it meets the requirement that no more than \textbf{PF1 =
10{[}percent{]}} of the repeatability outliers exceed the outlier limit
of \textbf{PA2uzy = 22.5 millimagnitudes}.~








\paragraph{Test Procedure}\mbox{}\\
\begin{tabular}{p{4cm}p{12cm}}
\toprule
Step 1
& Description \\ \hline
\end{tabular}
{\scriptsize
Identify a dataset containing at least one field in each of the u, z,
and y filters with multiple overlapping visits.

}
\begin{tabular}{p{3cm}p{13cm}}
\hline
            & Expected Result \\ \hline
\end{tabular}
{\scriptsize
A dataset that has been ingested into a Butler repository.

}

\begin{tabular}{p{4cm}p{12cm}}
\toprule
Step 2-1
{\scriptsize from \hyperref[lvv-t860]{LVV-T860} }
& Description \\ \hline
\end{tabular}
{\scriptsize
The `path` that you will use depends on where you are running the
science pipelines. Options:\\[2\baselineskip]

\begin{itemize}
\tightlist
\item
  local (newinstall.sh - based
  install):{[}path\_to\_installation{]}/loadLSST.bash
\item
  development cluster (``lsst-dev''):
  /software/lsstsw/stack/loadLSST.bash
\item
  LSP Notebook aspect (from a terminal):
  /opt/lsst/software/stack/loadLSST.bash
\end{itemize}

From the command line, execute the commands below in the example
code:\\[2\baselineskip]

}
\begin{tabular}{p{3cm}p{13cm}}
\hline
            & Example Code \\ \hline
\end{tabular}
{\scriptsize
source `path`\\
setup lsst\_distrib

}
\begin{tabular}{p{3cm}p{13cm}}
\hline
            & Expected Result \\ \hline
\end{tabular}
{\scriptsize
Science pipeline software is available for use. If additional packages
are needed (for example, `obs' packages such as `obs\_subaru`), then
additional `setup` commands will be necessary.\\[2\baselineskip]To check
versions in use, type:\\
eups list -s

}

\begin{tabular}{p{4cm}p{12cm}}
\toprule
Step 3-1
{\scriptsize from \hyperref[lvv-t1744]{LVV-T1744} }
& Description \\ \hline
\end{tabular}
{\scriptsize
Execute `validate\_drp` on a repository containing precursor data.
Identify the path to the data, which we will call `DATA/path', then
execute the following (with additional flags specified as needed):

}
\begin{tabular}{p{3cm}p{13cm}}
\hline
            & Example Code \\ \hline
\end{tabular}
{\scriptsize
validateDrp.py `DATA/path`

}
\begin{tabular}{p{3cm}p{13cm}}
\hline
            & Expected Result \\ \hline
\end{tabular}
{\scriptsize
JSON files (and associated figures) containing the Measurements and any
associated ``extras.''

}

\begin{tabular}{p{4cm}p{12cm}}
\toprule
Step 4
& Description \\ \hline
\end{tabular}
{\scriptsize
Confirm that the metric PA2uzy has been calculated using the threshold
PF1, and that its values are reasonable.

}
\begin{tabular}{p{3cm}p{13cm}}
\hline
            & Expected Result \\ \hline
\end{tabular}
{\scriptsize
A JSON file (and/or a report generated from that JSON file)
demonstrating that PA2uzy has been calculated (and that it used PF1).

}

\subsubsection{LVV-T1759 - Verify calculation of photometric outliers in gri bands}\label{lvv-t1759}

\begin{longtable}[]{llllll}
\toprule
Version & Status & Priority & Verification Type & Owner
\\\midrule
1 & Approved & Normal &
Test & Jeffrey Carlin
\\\bottomrule
\multicolumn{6}{c}{ Open \href{https://jira.lsstcorp.org/secure/Tests.jspa\#/testCase/LVV-T1759}{LVV-T1759} in Jira } \\
\end{longtable}

\paragraph{Verification Elements}\mbox{}\\

\begin{itemize}
\item \href{https://jira.lsstcorp.org/browse/LVV-9752}{LVV-9752} - DMS-REQ-0359-V-03: Max fraction of outliers among non-saturated sources

\item \href{https://jira.lsstcorp.org/browse/LVV-9754}{LVV-9754} - DMS-REQ-0359-V-05: Repeatability outlier limit in gri

\end{itemize}

\paragraph{Test Items}\mbox{}\\

Verify that the DM system has provided the code to calculate the
photometric repeatability in the g, r, and i filters, and assess whether
it meets the requirement that no more than \textbf{PF1 =
10{[}percent{]}} of the repeatability outliers exceed the outlier limit
of \textbf{PA2gri = 15 millimagnitudes}.








\paragraph{Test Procedure}\mbox{}\\
\begin{tabular}{p{4cm}p{12cm}}
\toprule
Step 1
& Description \\ \hline
\end{tabular}
{\scriptsize
Identify a dataset containing at least one field in each of the g, r,
and i filters with multiple overlapping visits.

}
\begin{tabular}{p{3cm}p{13cm}}
\hline
            & Expected Result \\ \hline
\end{tabular}
{\scriptsize
A dataset that has been ingested into a Butler repository.

}

\begin{tabular}{p{4cm}p{12cm}}
\toprule
Step 2-1
{\scriptsize from \hyperref[lvv-t860]{LVV-T860} }
& Description \\ \hline
\end{tabular}
{\scriptsize
The `path` that you will use depends on where you are running the
science pipelines. Options:\\[2\baselineskip]

\begin{itemize}
\tightlist
\item
  local (newinstall.sh - based
  install):{[}path\_to\_installation{]}/loadLSST.bash
\item
  development cluster (``lsst-dev''):
  /software/lsstsw/stack/loadLSST.bash
\item
  LSP Notebook aspect (from a terminal):
  /opt/lsst/software/stack/loadLSST.bash
\end{itemize}

From the command line, execute the commands below in the example
code:\\[2\baselineskip]

}
\begin{tabular}{p{3cm}p{13cm}}
\hline
            & Example Code \\ \hline
\end{tabular}
{\scriptsize
source `path`\\
setup lsst\_distrib

}
\begin{tabular}{p{3cm}p{13cm}}
\hline
            & Expected Result \\ \hline
\end{tabular}
{\scriptsize
Science pipeline software is available for use. If additional packages
are needed (for example, `obs' packages such as `obs\_subaru`), then
additional `setup` commands will be necessary.\\[2\baselineskip]To check
versions in use, type:\\
eups list -s

}

\begin{tabular}{p{4cm}p{12cm}}
\toprule
Step 3-1
{\scriptsize from \hyperref[lvv-t1744]{LVV-T1744} }
& Description \\ \hline
\end{tabular}
{\scriptsize
Execute `validate\_drp` on a repository containing precursor data.
Identify the path to the data, which we will call `DATA/path', then
execute the following (with additional flags specified as needed):

}
\begin{tabular}{p{3cm}p{13cm}}
\hline
            & Example Code \\ \hline
\end{tabular}
{\scriptsize
validateDrp.py `DATA/path`

}
\begin{tabular}{p{3cm}p{13cm}}
\hline
            & Expected Result \\ \hline
\end{tabular}
{\scriptsize
JSON files (and associated figures) containing the Measurements and any
associated ``extras.''

}

\begin{tabular}{p{4cm}p{12cm}}
\toprule
Step 4
& Description \\ \hline
\end{tabular}
{\scriptsize
Confirm that the metric PA2gri has been calculated using the threshold
PF1, and that its values are reasonable.

}
\begin{tabular}{p{3cm}p{13cm}}
\hline
            & Expected Result \\ \hline
\end{tabular}
{\scriptsize
A JSON file (and/or a report generated from that JSON file)
demonstrating that PA2gri has been calculated (and that it used PF1).

}

\subsubsection{LVV-T1946 - Verify implementation of measurements in catalogs from coadds}\label{lvv-t1946}

\begin{longtable}[]{llllll}
\toprule
Version & Status & Priority & Verification Type & Owner
\\\midrule
1 & Approved & Normal &
Test & Jeffrey Carlin
\\\bottomrule
\multicolumn{6}{c}{ Open \href{https://jira.lsstcorp.org/secure/Tests.jspa\#/testCase/LVV-T1946}{LVV-T1946} in Jira } \\
\end{longtable}

\paragraph{Verification Elements}\mbox{}\\

\begin{itemize}
\item \href{https://jira.lsstcorp.org/browse/LVV-178}{LVV-178} - DMS-REQ-0347-V-01: Measurements in catalogs

\end{itemize}

\paragraph{Test Items}\mbox{}\\

Verify that source measurements in catalogs containing measurements from
coadd images are in flux units.








\paragraph{Test Procedure}\mbox{}\\
\begin{tabular}{p{4cm}p{12cm}}
\toprule
Step 1-1
{\scriptsize from \hyperref[lvv-t987]{LVV-T987} }
& Description \\ \hline
\end{tabular}
{\scriptsize
Identify the path to the data repository, which we will refer to as
`DATA/path', then execute the following:

}
\begin{tabular}{p{3cm}p{13cm}}
\hline
            & Example Code \\ \hline
\end{tabular}
{\scriptsize
\begin{verbatim}
import lsst.daf.persistence as dafPersist
butler = dafPersist.Butler(inputs='DATA/path')
\end{verbatim}

}
\begin{tabular}{p{3cm}p{13cm}}
\hline
            & Expected Result \\ \hline
\end{tabular}
{\scriptsize
Butler repo available for reading.

}

\begin{tabular}{p{4cm}p{12cm}}
\toprule
Step 2
& Description \\ \hline
\end{tabular}
{\scriptsize
Identify and read an appropriate processed precursor dataset containing
coadds with the Butler.

}
\begin{tabular}{p{3cm}p{13cm}}
\hline
            & Expected Result \\ \hline
\end{tabular}

\begin{tabular}{p{4cm}p{12cm}}
\toprule
Step 3
& Description \\ \hline
\end{tabular}
{\scriptsize
Verify that the coadd catalog provides measurements in flux units.

}
\begin{tabular}{p{3cm}p{13cm}}
\hline
            & Expected Result \\ \hline
\end{tabular}
{\scriptsize
Confirmation of measurements in catalogs encoded in flux units.

}

\subsubsection{LVV-T1947 - Verify implementation of measurements in catalogs from difference images}\label{lvv-t1947}

\begin{longtable}[]{llllll}
\toprule
Version & Status & Priority & Verification Type & Owner
\\\midrule
1 & Approved & Normal &
Test & Jeffrey Carlin
\\\bottomrule
\multicolumn{6}{c}{ Open \href{https://jira.lsstcorp.org/secure/Tests.jspa\#/testCase/LVV-T1947}{LVV-T1947} in Jira } \\
\end{longtable}

\paragraph{Verification Elements}\mbox{}\\

\begin{itemize}
\item \href{https://jira.lsstcorp.org/browse/LVV-178}{LVV-178} - DMS-REQ-0347-V-01: Measurements in catalogs

\end{itemize}

\paragraph{Test Items}\mbox{}\\

Verify that source measurements in catalogs containing measurements from
difference images are in flux units.








\paragraph{Test Procedure}\mbox{}\\
\begin{tabular}{p{4cm}p{12cm}}
\toprule
Step 1-1
{\scriptsize from \hyperref[lvv-t987]{LVV-T987} }
& Description \\ \hline
\end{tabular}
{\scriptsize
Identify the path to the data repository, which we will refer to as
`DATA/path', then execute the following:

}
\begin{tabular}{p{3cm}p{13cm}}
\hline
            & Example Code \\ \hline
\end{tabular}
{\scriptsize
\begin{verbatim}
import lsst.daf.persistence as dafPersist
butler = dafPersist.Butler(inputs='DATA/path')
\end{verbatim}

}
\begin{tabular}{p{3cm}p{13cm}}
\hline
            & Expected Result \\ \hline
\end{tabular}
{\scriptsize
Butler repo available for reading.

}

\begin{tabular}{p{4cm}p{12cm}}
\toprule
Step 2
& Description \\ \hline
\end{tabular}
{\scriptsize
Identify and read an appropriate processed precursor dataset containing
difference images with the Butler.

}
\begin{tabular}{p{3cm}p{13cm}}
\hline
            & Expected Result \\ \hline
\end{tabular}

\begin{tabular}{p{4cm}p{12cm}}
\toprule
Step 3
& Description \\ \hline
\end{tabular}
{\scriptsize
Verify that the difference image source catalog provides measurements in
flux units.

}
\begin{tabular}{p{3cm}p{13cm}}
\hline
            & Expected Result \\ \hline
\end{tabular}
{\scriptsize
Confirmation of measurements in catalogs encoded in flux units.

}

\subsection{ Draft Test Cases}

\subsubsection{LVV-T23 - Verify implementation of Storing Approximations of Per-pixel Metadata}\label{lvv-t23}

\begin{longtable}[]{llllll}
\toprule
Version & Status & Priority & Verification Type & Owner
\\\midrule
1 & Draft & Normal &
Test & Simon Krughoff
\\\bottomrule
\multicolumn{6}{c}{ Open \href{https://jira.lsstcorp.org/secure/Tests.jspa\#/testCase/LVV-T23}{LVV-T23} in Jira } \\
\end{longtable}

\paragraph{Verification Elements}\mbox{}\\

\begin{itemize}
\item \href{https://jira.lsstcorp.org/browse/LVV-157}{LVV-157} - DMS-REQ-0326-V-01: Storing Approximations of Per-pixel Metadata

\end{itemize}

\paragraph{Test Items}\mbox{}\\

\textbf{Test Items}\\[2\baselineskip]Show that the compressed form depth
and mask maps adequately represents the exact version of the same
information.








\paragraph{Test Procedure}\mbox{}\\
\begin{tabular}{p{4cm}p{12cm}}
\toprule
Step 1-1
{\scriptsize from \hyperref[lvv-t860]{LVV-T860} }
& Description \\ \hline
\end{tabular}
{\scriptsize
The `path` that you will use depends on where you are running the
science pipelines. Options:\\[2\baselineskip]

\begin{itemize}
\tightlist
\item
  local (newinstall.sh - based
  install):{[}path\_to\_installation{]}/loadLSST.bash
\item
  development cluster (``lsst-dev''):
  /software/lsstsw/stack/loadLSST.bash
\item
  LSP Notebook aspect (from a terminal):
  /opt/lsst/software/stack/loadLSST.bash
\end{itemize}

From the command line, execute the commands below in the example
code:\\[2\baselineskip]

}
\begin{tabular}{p{3cm}p{13cm}}
\hline
            & Example Code \\ \hline
\end{tabular}
{\scriptsize
source `path`\\
setup lsst\_distrib

}
\begin{tabular}{p{3cm}p{13cm}}
\hline
            & Expected Result \\ \hline
\end{tabular}
{\scriptsize
Science pipeline software is available for use. If additional packages
are needed (for example, `obs' packages such as `obs\_subaru`), then
additional `setup` commands will be necessary.\\[2\baselineskip]To check
versions in use, type:\\
eups list -s

}

\begin{tabular}{p{4cm}p{12cm}}
\toprule
Step 2-1
{\scriptsize from \hyperref[lvv-t987]{LVV-T987} }
& Description \\ \hline
\end{tabular}
{\scriptsize
Identify the path to the data repository, which we will refer to as
`DATA/path', then execute the following:

}
\begin{tabular}{p{3cm}p{13cm}}
\hline
            & Example Code \\ \hline
\end{tabular}
{\scriptsize
\begin{verbatim}
import lsst.daf.persistence as dafPersist
butler = dafPersist.Butler(inputs='DATA/path')
\end{verbatim}

}
\begin{tabular}{p{3cm}p{13cm}}
\hline
            & Expected Result \\ \hline
\end{tabular}
{\scriptsize
Butler repo available for reading.

}

\begin{tabular}{p{4cm}p{12cm}}
\toprule
Step 3
& Description \\ \hline
\end{tabular}
{\scriptsize
For each of the expected data products types (listed in Test Items
section ยง4.3.2) and each of the expected units (PVIs, coadds, etc),
retrieve the data product from the Butler and verify that it is
non-empty.

}
\begin{tabular}{p{3cm}p{13cm}}
\hline
            & Expected Result \\ \hline
\end{tabular}

\begin{tabular}{p{4cm}p{12cm}}
\toprule
Step 4
& Description \\ \hline
\end{tabular}
{\scriptsize
Create the coadd pixel level depth map for the HSC PDR dataset.

}
\begin{tabular}{p{3cm}p{13cm}}
\hline
            & Expected Result \\ \hline
\end{tabular}

\begin{tabular}{p{4cm}p{12cm}}
\toprule
Step 5
& Description \\ \hline
\end{tabular}
{\scriptsize
Generate compressed representation of the pixel level depth map.

}
\begin{tabular}{p{3cm}p{13cm}}
\hline
            & Expected Result \\ \hline
\end{tabular}

\begin{tabular}{p{4cm}p{12cm}}
\toprule
Step 6
& Description \\ \hline
\end{tabular}
{\scriptsize
Create the coadd pixel level mask map for the HSC PDR dataset.

}
\begin{tabular}{p{3cm}p{13cm}}
\hline
            & Expected Result \\ \hline
\end{tabular}

\begin{tabular}{p{4cm}p{12cm}}
\toprule
Step 7
& Description \\ \hline
\end{tabular}
{\scriptsize
Generate compressed representation of the mask map.

}
\begin{tabular}{p{3cm}p{13cm}}
\hline
            & Expected Result \\ \hline
\end{tabular}

\begin{tabular}{p{4cm}p{12cm}}
\toprule
Step 8
& Description \\ \hline
\end{tabular}
{\scriptsize
Sample randomly from both the pixel level and compressed depth maps.
~Compare the distribution of depths sampled from the pixel level depth
map to that sampled from the compressed representation.

}
\begin{tabular}{p{3cm}p{13cm}}
\hline
            & Expected Result \\ \hline
\end{tabular}

\begin{tabular}{p{4cm}p{12cm}}
\toprule
Step 9
& Description \\ \hline
\end{tabular}
{\scriptsize
Divide the mask planes into two groups: INFO and BAD. ~BAD flags are any
that would cause a particular pixel to be excluded from processing: e.g.
EDGE, SAT, BAD. ~Sample masks from both the pixel level mask map and the
compressed mask map.\\[2\baselineskip]For each sample, compute
sum(mask\_pixel xor mask\_compressed). ~Produce the distribution of the
number of bits that differ between the samples.\\[2\baselineskip]Repeat
for both the INFO flags and the BAD flags.

}
\begin{tabular}{p{3cm}p{13cm}}
\hline
            & Expected Result \\ \hline
\end{tabular}

\subsubsection{LVV-T24 - Verify implementation of Computing Derived Quantities}\label{lvv-t24}

\begin{longtable}[]{llllll}
\toprule
Version & Status & Priority & Verification Type & Owner
\\\midrule
1 & Draft & Normal &
Test & Melissa Graham
\\\bottomrule
\multicolumn{6}{c}{ Open \href{https://jira.lsstcorp.org/secure/Tests.jspa\#/testCase/LVV-T24}{LVV-T24} in Jira } \\
\end{longtable}

\paragraph{Verification Elements}\mbox{}\\

\begin{itemize}
\item \href{https://jira.lsstcorp.org/browse/LVV-162}{LVV-162} - DMS-REQ-0331-V-01: Computing Derived Quantities

\end{itemize}

\paragraph{Test Items}\mbox{}\\

To confirm that common derived quantities (apparent magnitude, FWHM in
arcsec, ellipticity) are available to an end-user by, e.g., ensuring a
color-color diagram is easy to construction, fitting functions to
derived data, or generating other common scientific derivatives.








\paragraph{Test Procedure}\mbox{}\\
\begin{tabular}{p{4cm}p{12cm}}
\toprule
Step 1-1
{\scriptsize from \hyperref[lvv-t860]{LVV-T860} }
& Description \\ \hline
\end{tabular}
{\scriptsize
The `path` that you will use depends on where you are running the
science pipelines. Options:\\[2\baselineskip]

\begin{itemize}
\tightlist
\item
  local (newinstall.sh - based
  install):{[}path\_to\_installation{]}/loadLSST.bash
\item
  development cluster (``lsst-dev''):
  /software/lsstsw/stack/loadLSST.bash
\item
  LSP Notebook aspect (from a terminal):
  /opt/lsst/software/stack/loadLSST.bash
\end{itemize}

From the command line, execute the commands below in the example
code:\\[2\baselineskip]

}
\begin{tabular}{p{3cm}p{13cm}}
\hline
            & Example Code \\ \hline
\end{tabular}
{\scriptsize
source `path`\\
setup lsst\_distrib

}
\begin{tabular}{p{3cm}p{13cm}}
\hline
            & Expected Result \\ \hline
\end{tabular}
{\scriptsize
Science pipeline software is available for use. If additional packages
are needed (for example, `obs' packages such as `obs\_subaru`), then
additional `setup` commands will be necessary.\\[2\baselineskip]To check
versions in use, type:\\
eups list -s

}

\begin{tabular}{p{4cm}p{12cm}}
\toprule
Step 2-1
{\scriptsize from \hyperref[lvv-t987]{LVV-T987} }
& Description \\ \hline
\end{tabular}
{\scriptsize
Identify the path to the data repository, which we will refer to as
`DATA/path', then execute the following:

}
\begin{tabular}{p{3cm}p{13cm}}
\hline
            & Example Code \\ \hline
\end{tabular}
{\scriptsize
\begin{verbatim}
import lsst.daf.persistence as dafPersist
butler = dafPersist.Butler(inputs='DATA/path')
\end{verbatim}

}
\begin{tabular}{p{3cm}p{13cm}}
\hline
            & Expected Result \\ \hline
\end{tabular}
{\scriptsize
Butler repo available for reading.

}

\begin{tabular}{p{4cm}p{12cm}}
\toprule
Step 3
& Description \\ \hline
\end{tabular}
{\scriptsize
For each of the expected data product types (listed in Test Items
section ยง4.3.2) and each of the expected units (PVIs, coadds, etc),
retrieve the data product from the Butler and verify it to be non-empty.

}
\begin{tabular}{p{3cm}p{13cm}}
\hline
            & Expected Result \\ \hline
\end{tabular}

\begin{tabular}{p{4cm}p{12cm}}
\toprule
Step 4
& Description \\ \hline
\end{tabular}
{\scriptsize
Load into DPDD+Science Platform

}
\begin{tabular}{p{3cm}p{13cm}}
\hline
            & Expected Result \\ \hline
\end{tabular}

\begin{tabular}{p{4cm}p{12cm}}
\toprule
Step 5
& Description \\ \hline
\end{tabular}
{\scriptsize
Constructing color-color diagram and fitting stellar locus in Science
Platform.

}
\begin{tabular}{p{3cm}p{13cm}}
\hline
            & Expected Result \\ \hline
\end{tabular}

\begin{tabular}{p{4cm}p{12cm}}
\toprule
Step 6
& Description \\ \hline
\end{tabular}
{\scriptsize
Invite three members of commissioning team to create color-color diagram
from coadd catalogs based on merged coadd reference catalog.

}
\begin{tabular}{p{3cm}p{13cm}}
\hline
            & Expected Result \\ \hline
\end{tabular}

\subsubsection{LVV-T25 - Verify implementation of Denormalizing Database Tables}\label{lvv-t25}

\begin{longtable}[]{llllll}
\toprule
Version & Status & Priority & Verification Type & Owner
\\\midrule
1 & Draft & Normal &
Test & Colin Slater
\\\bottomrule
\multicolumn{6}{c}{ Open \href{https://jira.lsstcorp.org/secure/Tests.jspa\#/testCase/LVV-T25}{LVV-T25} in Jira } \\
\end{longtable}

\paragraph{Verification Elements}\mbox{}\\

\begin{itemize}
\item \href{https://jira.lsstcorp.org/browse/LVV-163}{LVV-163} - DMS-REQ-0332-V-01: Denormalizing Database Tables

\end{itemize}

\paragraph{Test Items}\mbox{}\\

Verify that commonly useful views of data are easy to obtain through the
Science Platform.








\paragraph{Test Procedure}\mbox{}\\
\begin{tabular}{p{4cm}p{12cm}}
\toprule
Step 1
& Description \\ \hline
\end{tabular}
{\scriptsize
Connect to the Science Platform's portal query interface.

}
\begin{tabular}{p{3cm}p{13cm}}
\hline
            & Expected Result \\ \hline
\end{tabular}

\begin{tabular}{p{4cm}p{12cm}}
\toprule
Step 2
& Description \\ \hline
\end{tabular}
{\scriptsize
List the available views in the database.

}
\begin{tabular}{p{3cm}p{13cm}}
\hline
            & Expected Result \\ \hline
\end{tabular}

\begin{tabular}{p{4cm}p{12cm}}
\toprule
Step 3
& Description \\ \hline
\end{tabular}
{\scriptsize
{Take 20 sampled queries and determine which are easily done on views
and which require complicated joins. Discuss the complicated ones and
determine if any could be simplified by adding additional views.}

}
\begin{tabular}{p{3cm}p{13cm}}
\hline
            & Expected Result \\ \hline
\end{tabular}

\subsubsection{LVV-T26 - Verify implementation of Maximum Likelihood Values and Covariances}\label{lvv-t26}

\begin{longtable}[]{llllll}
\toprule
Version & Status & Priority & Verification Type & Owner
\\\midrule
1 & Draft & Normal &
Test & Jim Bosch
\\\bottomrule
\multicolumn{6}{c}{ Open \href{https://jira.lsstcorp.org/secure/Tests.jspa\#/testCase/LVV-T26}{LVV-T26} in Jira } \\
\end{longtable}

\paragraph{Verification Elements}\mbox{}\\

\begin{itemize}
\item \href{https://jira.lsstcorp.org/browse/LVV-164}{LVV-164} - DMS-REQ-0333-V-01: Maximum Likelihood Values and Covariances

\end{itemize}

\paragraph{Test Items}\mbox{}\\

\begin{itemize}
\tightlist
\item
  Check that all measurements in source and object schemas include
  columns containing uncertainties, including covariances between
  jointly-measured quantities.
\item
  Check that all model-fit measurements in source and object schemas
  include columns that report goodness-of-fit.
\item
  Check that most sources and objects with successful measurements
  report finite uncertainty values for those measurements.
\item
  Check that most sources and objects with successful model-fit
  measurements report finite goodness-of-fit values.
\end{itemize}








\paragraph{Test Procedure}\mbox{}\\
\begin{tabular}{p{4cm}p{12cm}}
\toprule
Step 1-1
{\scriptsize from \hyperref[lvv-t860]{LVV-T860} }
& Description \\ \hline
\end{tabular}
{\scriptsize
The `path` that you will use depends on where you are running the
science pipelines. Options:\\[2\baselineskip]

\begin{itemize}
\tightlist
\item
  local (newinstall.sh - based
  install):{[}path\_to\_installation{]}/loadLSST.bash
\item
  development cluster (``lsst-dev''):
  /software/lsstsw/stack/loadLSST.bash
\item
  LSP Notebook aspect (from a terminal):
  /opt/lsst/software/stack/loadLSST.bash
\end{itemize}

From the command line, execute the commands below in the example
code:\\[2\baselineskip]

}
\begin{tabular}{p{3cm}p{13cm}}
\hline
            & Example Code \\ \hline
\end{tabular}
{\scriptsize
source `path`\\
setup lsst\_distrib

}
\begin{tabular}{p{3cm}p{13cm}}
\hline
            & Expected Result \\ \hline
\end{tabular}
{\scriptsize
Science pipeline software is available for use. If additional packages
are needed (for example, `obs' packages such as `obs\_subaru`), then
additional `setup` commands will be necessary.\\[2\baselineskip]To check
versions in use, type:\\
eups list -s

}

\begin{tabular}{p{4cm}p{12cm}}
\toprule
Step 2-1
{\scriptsize from \hyperref[lvv-t987]{LVV-T987} }
& Description \\ \hline
\end{tabular}
{\scriptsize
Identify the path to the data repository, which we will refer to as
`DATA/path', then execute the following:

}
\begin{tabular}{p{3cm}p{13cm}}
\hline
            & Example Code \\ \hline
\end{tabular}
{\scriptsize
\begin{verbatim}
import lsst.daf.persistence as dafPersist
butler = dafPersist.Butler(inputs='DATA/path')
\end{verbatim}

}
\begin{tabular}{p{3cm}p{13cm}}
\hline
            & Expected Result \\ \hline
\end{tabular}
{\scriptsize
Butler repo available for reading.

}

\begin{tabular}{p{4cm}p{12cm}}
\toprule
Step 3
& Description \\ \hline
\end{tabular}
{\scriptsize
For each of the expected data product types (listed in Test Items
section ยง4.3.2) and each of the expected units (PVIs, coadds, etc),
retrieve the data product from the Butler and verify it to be non-empty.

}
\begin{tabular}{p{3cm}p{13cm}}
\hline
            & Expected Result \\ \hline
\end{tabular}

\begin{tabular}{p{4cm}p{12cm}}
\toprule
Step 4
& Description \\ \hline
\end{tabular}
{\scriptsize
Verify that maximum likelihood and covariant quantities are provided.
~Test and manually inspect that they are reasonable (finite,
appropriately normed).

}
\begin{tabular}{p{3cm}p{13cm}}
\hline
            & Expected Result \\ \hline
\end{tabular}

\subsubsection{LVV-T27 - Verify implementation of Data Availability}\label{lvv-t27}

\begin{longtable}[]{llllll}
\toprule
Version & Status & Priority & Verification Type & Owner
\\\midrule
1 & Draft & Normal &
Test & Gregory Dubois-Felsmann
\\\bottomrule
\multicolumn{6}{c}{ Open \href{https://jira.lsstcorp.org/secure/Tests.jspa\#/testCase/LVV-T27}{LVV-T27} in Jira } \\
\end{longtable}

\paragraph{Verification Elements}\mbox{}\\

\begin{itemize}
\item \href{https://jira.lsstcorp.org/browse/LVV-177}{LVV-177} - DMS-REQ-0346-V-01: Data Availability

\end{itemize}

\paragraph{Test Items}\mbox{}\\

Determine if all required categories of raw data (specifically
enumerated: raw exposures, calibration frames, telemetry, configuration
metadata) can be located through the Science Platform and are available
for download. ~Verify through (1) administrative review; (2) checking
with precursor data; (3) checking on early data feeds from the Summit
such as from AuxTel and ComCam.








\paragraph{Test Procedure}\mbox{}\\
\begin{tabular}{p{4cm}p{12cm}}
\toprule
Step 1
& Description \\ \hline
\end{tabular}
{\scriptsize
{Invite two reviewers to review that plan that seems reasonable to
expect the archiving and provision of raw data}

}
\begin{tabular}{p{3cm}p{13cm}}
\hline
            & Expected Result \\ \hline
\end{tabular}

\begin{tabular}{p{4cm}p{12cm}}
\toprule
Step 2
& Description \\ \hline
\end{tabular}
{\scriptsize
Pass a set of HSC data through (equal in size to the first public data
release) the data backbone through ingest and provide interface

}
\begin{tabular}{p{3cm}p{13cm}}
\hline
            & Expected Result \\ \hline
\end{tabular}

\begin{tabular}{p{4cm}p{12cm}}
\toprule
Step 3
& Description \\ \hline
\end{tabular}
{\scriptsize
Track the ingestion of AuxTel data during one month in 2018-2019 and
verify delivery and test download.

}
\begin{tabular}{p{3cm}p{13cm}}
\hline
            & Expected Result \\ \hline
\end{tabular}

\subsubsection{LVV-T31 - Verify implementation of Crosstalk Corrected Science Image Data
Acquisition}\label{lvv-t31}

\begin{longtable}[]{llllll}
\toprule
Version & Status & Priority & Verification Type & Owner
\\\midrule
1 & Draft & Normal &
Test & Kian-Tat Lim
\\\bottomrule
\multicolumn{6}{c}{ Open \href{https://jira.lsstcorp.org/secure/Tests.jspa\#/testCase/LVV-T31}{LVV-T31} in Jira } \\
\end{longtable}

\paragraph{Verification Elements}\mbox{}\\

\begin{itemize}
\item \href{https://jira.lsstcorp.org/browse/LVV-10}{LVV-10} - DMS-REQ-0022-V-01: Crosstalk Corrected Science Image Data Acquisition

\end{itemize}

\paragraph{Test Items}\mbox{}\\

Verify successful ingestion of crosstalk corrected data from L1 Test
Stand DAQ while simulating all modes.








\paragraph{Test Procedure}\mbox{}\\
\begin{tabular}{p{4cm}p{12cm}}
\toprule
Step 1
& Description \\ \hline
\end{tabular}
{\scriptsize
Inject signals of different relative strength

}
\begin{tabular}{p{3cm}p{13cm}}
\hline
            & Expected Result \\ \hline
\end{tabular}

\begin{tabular}{p{4cm}p{12cm}}
\toprule
Step 2
& Description \\ \hline
\end{tabular}
{\scriptsize
Apply Camera cross-talk correction

}
\begin{tabular}{p{3cm}p{13cm}}
\hline
            & Expected Result \\ \hline
\end{tabular}

\begin{tabular}{p{4cm}p{12cm}}
\toprule
Step 3
& Description \\ \hline
\end{tabular}
{\scriptsize
Verify that DMS sytem can import the cross-talk corrected images

}
\begin{tabular}{p{3cm}p{13cm}}
\hline
            & Expected Result \\ \hline
\end{tabular}

\begin{tabular}{p{4cm}p{12cm}}
\toprule
Step 4
& Description \\ \hline
\end{tabular}
{\scriptsize
Verify that images are corrected for crosstalk

}
\begin{tabular}{p{3cm}p{13cm}}
\hline
            & Expected Result \\ \hline
\end{tabular}

\subsubsection{LVV-T35 - Verify implementation of Nightly Data Accessible Within 24 hrs}\label{lvv-t35}

\begin{longtable}[]{llllll}
\toprule
Version & Status & Priority & Verification Type & Owner
\\\midrule
1 & Draft & Normal &
Test & Eric Bellm
\\\bottomrule
\multicolumn{6}{c}{ Open \href{https://jira.lsstcorp.org/secure/Tests.jspa\#/testCase/LVV-T35}{LVV-T35} in Jira } \\
\end{longtable}

\paragraph{Verification Elements}\mbox{}\\

\begin{itemize}
\item \href{https://jira.lsstcorp.org/browse/LVV-175}{LVV-175} - DMS-REQ-0004-V-01: Time to L1 public release

\end{itemize}

\paragraph{Test Items}\mbox{}\\

\textbf{Test Items}\\[2\baselineskip]Verify that\\
1. Alerts are available within OTT1\\
2. Level 1 Data Products are available within L1PublicT\\
3. Solar System Object orbits are available within L1PublicT of the
updated calculations completion on the following night.








\paragraph{Test Procedure}\mbox{}\\
\begin{tabular}{p{4cm}p{12cm}}
\toprule
Step 1-1
{\scriptsize from \hyperref[lvv-t860]{LVV-T860} }
& Description \\ \hline
\end{tabular}
{\scriptsize
The `path` that you will use depends on where you are running the
science pipelines. Options:\\[2\baselineskip]

\begin{itemize}
\tightlist
\item
  local (newinstall.sh - based
  install):{[}path\_to\_installation{]}/loadLSST.bash
\item
  development cluster (``lsst-dev''):
  /software/lsstsw/stack/loadLSST.bash
\item
  LSP Notebook aspect (from a terminal):
  /opt/lsst/software/stack/loadLSST.bash
\end{itemize}

From the command line, execute the commands below in the example
code:\\[2\baselineskip]

}
\begin{tabular}{p{3cm}p{13cm}}
\hline
            & Example Code \\ \hline
\end{tabular}
{\scriptsize
source `path`\\
setup lsst\_distrib

}
\begin{tabular}{p{3cm}p{13cm}}
\hline
            & Expected Result \\ \hline
\end{tabular}
{\scriptsize
Science pipeline software is available for use. If additional packages
are needed (for example, `obs' packages such as `obs\_subaru`), then
additional `setup` commands will be necessary.\\[2\baselineskip]To check
versions in use, type:\\
eups list -s

}

\begin{tabular}{p{4cm}p{12cm}}
\toprule
Step 2-1
{\scriptsize from \hyperref[lvv-t866]{LVV-T866} }
& Description \\ \hline
\end{tabular}
{\scriptsize
Perform the steps of Alert Production (including, but not necessarily
limited to, single frame processing, ISR, source detection/measurement,
PSF estimation, photometric and astrometric calibration, difference
imaging, DIASource detection/measurement, source association). During
Operations, it is presumed that these are automated for a given
dataset.~

}
\begin{tabular}{p{3cm}p{13cm}}
\hline
            & Expected Result \\ \hline
\end{tabular}
{\scriptsize
An output dataset including difference images and DIASource and
DIAObject measurements.

}

\begin{tabular}{p{4cm}p{12cm}}
\toprule
Step 2-2
{\scriptsize from \hyperref[lvv-t866]{LVV-T866} }
& Description \\ \hline
\end{tabular}
{\scriptsize
Verify that the expected data products have been produced, and that
catalogs contain reasonable values for measured quantities of interest.

}
\begin{tabular}{p{3cm}p{13cm}}
\hline
            & Expected Result \\ \hline
\end{tabular}

\begin{tabular}{p{4cm}p{12cm}}
\toprule
Step 3
& Description \\ \hline
\end{tabular}
{\scriptsize
Time processing of data starting from (pre-ingested) raw files until an
alert is available for distribution; verify that this time is less than
OTT1.

}
\begin{tabular}{p{3cm}p{13cm}}
\hline
            & Expected Result \\ \hline
\end{tabular}

\begin{tabular}{p{4cm}p{12cm}}
\toprule
Step 4
& Description \\ \hline
\end{tabular}
{\scriptsize
Time processing of data starting from (pre-ingested) raw files until the
required data products are available in the Science Platform. Verify
that this time is less than L1PublicT.

}
\begin{tabular}{p{3cm}p{13cm}}
\hline
            & Expected Result \\ \hline
\end{tabular}

\begin{tabular}{p{4cm}p{12cm}}
\toprule
Step 5
& Description \\ \hline
\end{tabular}
{\scriptsize
Run MOPS on 1 night equivalent of LSST observing worth of precursor data
and verify that Solar System Object orbits can be updated within 24
hours.

}
\begin{tabular}{p{3cm}p{13cm}}
\hline
            & Expected Result \\ \hline
\end{tabular}

\begin{tabular}{p{4cm}p{12cm}}
\toprule
Step 6
& Description \\ \hline
\end{tabular}
{\scriptsize
Record time between completion of MOPS processing and availability of
the updated SSObject catalogue through the Science Platform; verify this
time is less than L1PublicT.

}
\begin{tabular}{p{3cm}p{13cm}}
\hline
            & Expected Result \\ \hline
\end{tabular}

\subsubsection{LVV-T36 - Verify implementation of Difference Exposures}\label{lvv-t36}

\begin{longtable}[]{llllll}
\toprule
Version & Status & Priority & Verification Type & Owner
\\\midrule
1 & Draft & Normal &
Test & Eric Bellm
\\\bottomrule
\multicolumn{6}{c}{ Open \href{https://jira.lsstcorp.org/secure/Tests.jspa\#/testCase/LVV-T36}{LVV-T36} in Jira } \\
\end{longtable}

\paragraph{Verification Elements}\mbox{}\\

\begin{itemize}
\item \href{https://jira.lsstcorp.org/browse/LVV-7}{LVV-7} - DMS-REQ-0010-V-01: Difference Exposures

\end{itemize}

\paragraph{Test Items}\mbox{}\\

Verify successful creation of a\\
1. PSF-matched template image for a given Processed Visit Image\\
2. Difference Exposure from each Processed Visit Image








\paragraph{Test Procedure}\mbox{}\\
\begin{tabular}{p{4cm}p{12cm}}
\toprule
Step 1-1
{\scriptsize from \hyperref[lvv-t860]{LVV-T860} }
& Description \\ \hline
\end{tabular}
{\scriptsize
The `path` that you will use depends on where you are running the
science pipelines. Options:\\[2\baselineskip]

\begin{itemize}
\tightlist
\item
  local (newinstall.sh - based
  install):{[}path\_to\_installation{]}/loadLSST.bash
\item
  development cluster (``lsst-dev''):
  /software/lsstsw/stack/loadLSST.bash
\item
  LSP Notebook aspect (from a terminal):
  /opt/lsst/software/stack/loadLSST.bash
\end{itemize}

From the command line, execute the commands below in the example
code:\\[2\baselineskip]

}
\begin{tabular}{p{3cm}p{13cm}}
\hline
            & Example Code \\ \hline
\end{tabular}
{\scriptsize
source `path`\\
setup lsst\_distrib

}
\begin{tabular}{p{3cm}p{13cm}}
\hline
            & Expected Result \\ \hline
\end{tabular}
{\scriptsize
Science pipeline software is available for use. If additional packages
are needed (for example, `obs' packages such as `obs\_subaru`), then
additional `setup` commands will be necessary.\\[2\baselineskip]To check
versions in use, type:\\
eups list -s

}

\begin{tabular}{p{4cm}p{12cm}}
\toprule
Step 2-1
{\scriptsize from \hyperref[lvv-t866]{LVV-T866} }
& Description \\ \hline
\end{tabular}
{\scriptsize
Perform the steps of Alert Production (including, but not necessarily
limited to, single frame processing, ISR, source detection/measurement,
PSF estimation, photometric and astrometric calibration, difference
imaging, DIASource detection/measurement, source association). During
Operations, it is presumed that these are automated for a given
dataset.~

}
\begin{tabular}{p{3cm}p{13cm}}
\hline
            & Expected Result \\ \hline
\end{tabular}
{\scriptsize
An output dataset including difference images and DIASource and
DIAObject measurements.

}

\begin{tabular}{p{4cm}p{12cm}}
\toprule
Step 2-2
{\scriptsize from \hyperref[lvv-t866]{LVV-T866} }
& Description \\ \hline
\end{tabular}
{\scriptsize
Verify that the expected data products have been produced, and that
catalogs contain reasonable values for measured quantities of interest.

}
\begin{tabular}{p{3cm}p{13cm}}
\hline
            & Expected Result \\ \hline
\end{tabular}

\begin{tabular}{p{4cm}p{12cm}}
\toprule
Step 3
& Description \\ \hline
\end{tabular}
{\scriptsize
Demonstrate successful creation of a template image from HSC PDF and
DECAM HiTS data. ~Demonstrate successful creation of a Difference
Exposure for at least 10 other images from survey, ideally at a range of
arimass. ~In particular, HiTS has 2013A u-band data. ~While the Blanco
4-m does have an ADC, there are still some chromatic effects and we
should demonstrate that we can successfully produce Difference Exposures
and templates for diferent airmass bins.

}
\begin{tabular}{p{3cm}p{13cm}}
\hline
            & Expected Result \\ \hline
\end{tabular}

\subsubsection{LVV-T37 - Verify implementation of Difference Exposure Attributes}\label{lvv-t37}

\begin{longtable}[]{llllll}
\toprule
Version & Status & Priority & Verification Type & Owner
\\\midrule
1 & Draft & Normal &
Test & Eric Bellm
\\\bottomrule
\multicolumn{6}{c}{ Open \href{https://jira.lsstcorp.org/secure/Tests.jspa\#/testCase/LVV-T37}{LVV-T37} in Jira } \\
\end{longtable}

\paragraph{Verification Elements}\mbox{}\\

\begin{itemize}
\item \href{https://jira.lsstcorp.org/browse/LVV-32}{LVV-32} - DMS-REQ-0074-V-01: Difference Exposure Attributes

\item \href{https://jira.lsstcorp.org/browse/LVV-1234}{LVV-1234} - OSS-REQ-0122-V-01: Provenance

\end{itemize}

\paragraph{Test Items}\mbox{}\\

Verify that for each Difference Exposure the DMS stores\\
1. The identify of the input exposures and related provenance
information\\
2. Metadata attributes of the subtraction, including the PSF-matching
kernel used.








\paragraph{Test Procedure}\mbox{}\\
\begin{tabular}{p{4cm}p{12cm}}
\toprule
Step 1-1
{\scriptsize from \hyperref[lvv-t860]{LVV-T860} }
& Description \\ \hline
\end{tabular}
{\scriptsize
The `path` that you will use depends on where you are running the
science pipelines. Options:\\[2\baselineskip]

\begin{itemize}
\tightlist
\item
  local (newinstall.sh - based
  install):{[}path\_to\_installation{]}/loadLSST.bash
\item
  development cluster (``lsst-dev''):
  /software/lsstsw/stack/loadLSST.bash
\item
  LSP Notebook aspect (from a terminal):
  /opt/lsst/software/stack/loadLSST.bash
\end{itemize}

From the command line, execute the commands below in the example
code:\\[2\baselineskip]

}
\begin{tabular}{p{3cm}p{13cm}}
\hline
            & Example Code \\ \hline
\end{tabular}
{\scriptsize
source `path`\\
setup lsst\_distrib

}
\begin{tabular}{p{3cm}p{13cm}}
\hline
            & Expected Result \\ \hline
\end{tabular}
{\scriptsize
Science pipeline software is available for use. If additional packages
are needed (for example, `obs' packages such as `obs\_subaru`), then
additional `setup` commands will be necessary.\\[2\baselineskip]To check
versions in use, type:\\
eups list -s

}

\begin{tabular}{p{4cm}p{12cm}}
\toprule
Step 2-1
{\scriptsize from \hyperref[lvv-t866]{LVV-T866} }
& Description \\ \hline
\end{tabular}
{\scriptsize
Perform the steps of Alert Production (including, but not necessarily
limited to, single frame processing, ISR, source detection/measurement,
PSF estimation, photometric and astrometric calibration, difference
imaging, DIASource detection/measurement, source association). During
Operations, it is presumed that these are automated for a given
dataset.~

}
\begin{tabular}{p{3cm}p{13cm}}
\hline
            & Expected Result \\ \hline
\end{tabular}
{\scriptsize
An output dataset including difference images and DIASource and
DIAObject measurements.

}

\begin{tabular}{p{4cm}p{12cm}}
\toprule
Step 2-2
{\scriptsize from \hyperref[lvv-t866]{LVV-T866} }
& Description \\ \hline
\end{tabular}
{\scriptsize
Verify that the expected data products have been produced, and that
catalogs contain reasonable values for measured quantities of interest.

}
\begin{tabular}{p{3cm}p{13cm}}
\hline
            & Expected Result \\ \hline
\end{tabular}

\begin{tabular}{p{4cm}p{12cm}}
\toprule
Step 3
& Description \\ \hline
\end{tabular}
{\scriptsize
For each of HSC PDR and DECAM HiTS data: set up three different
templates and run subtractions on 10 different images from at least two
different filters. ~Verify that we can recover the provenance
information about which template was used for each subtraction, which
input images were used for that template, and that we can successfull
extract the PSF matching kernel.

}
\begin{tabular}{p{3cm}p{13cm}}
\hline
            & Expected Result \\ \hline
\end{tabular}

\subsubsection{LVV-T44 - Verify implementation of Documenting Image Characterization}\label{lvv-t44}

\begin{longtable}[]{llllll}
\toprule
Version & Status & Priority & Verification Type & Owner
\\\midrule
1 & Draft & Normal &
Test & Jim Bosch
\\\bottomrule
\multicolumn{6}{c}{ Open \href{https://jira.lsstcorp.org/secure/Tests.jspa\#/testCase/LVV-T44}{LVV-T44} in Jira } \\
\end{longtable}

\paragraph{Verification Elements}\mbox{}\\

\begin{itemize}
\item \href{https://jira.lsstcorp.org/browse/LVV-159}{LVV-159} - DMS-REQ-0328-V-01: Documenting Image Characterization

\end{itemize}

\paragraph{Test Items}\mbox{}\\

Verify that the persisted format for Processed Visit Images and
associated instrument-signature-removal data products is documented.








\paragraph{Test Procedure}\mbox{}\\
\begin{tabular}{p{4cm}p{12cm}}
\toprule
Step 1
& Description \\ \hline
\end{tabular}
{\scriptsize
Delegate to Alert Production

}
\begin{tabular}{p{3cm}p{13cm}}
\hline
            & Expected Result \\ \hline
\end{tabular}

\subsubsection{LVV-T46 - Verify implementation of Prompt Processing Performance Report Definition}\label{lvv-t46}

\begin{longtable}[]{llllll}
\toprule
Version & Status & Priority & Verification Type & Owner
\\\midrule
1 & Draft & Normal &
Test & Eric Bellm
\\\bottomrule
\multicolumn{6}{c}{ Open \href{https://jira.lsstcorp.org/secure/Tests.jspa\#/testCase/LVV-T46}{LVV-T46} in Jira } \\
\end{longtable}

\paragraph{Verification Elements}\mbox{}\\

\begin{itemize}
\item \href{https://jira.lsstcorp.org/browse/LVV-41}{LVV-41} - DMS-REQ-0099-V-01: Level 1 Performance Report Definition

\end{itemize}

\paragraph{Test Items}\mbox{}\\

Verify that the DMS produces a Prompt Processing Performance Report.
~Specifically check that the number of observations that describe each
of the following:\\
1. Successfully processed, recoverable failures, unrecoverable
failures.\\
2. Archived\\
3. Result in science.\\[2\baselineskip]This is testing more the
processing rather than the observatory system.








\paragraph{Test Procedure}\mbox{}\\
\begin{tabular}{p{4cm}p{12cm}}
\toprule
Step 1
& Description \\ \hline
\end{tabular}
{\scriptsize
Execute single-day operations rehearsal, observe report

}
\begin{tabular}{p{3cm}p{13cm}}
\hline
            & Expected Result \\ \hline
\end{tabular}

\subsubsection{LVV-T49 - Verify implementation of DIASource Catalog}\label{lvv-t49}

\begin{longtable}[]{llllll}
\toprule
Version & Status & Priority & Verification Type & Owner
\\\midrule
1 & Draft & Normal &
Test & Eric Bellm
\\\bottomrule
\multicolumn{6}{c}{ Open \href{https://jira.lsstcorp.org/secure/Tests.jspa\#/testCase/LVV-T49}{LVV-T49} in Jira } \\
\end{longtable}

\paragraph{Verification Elements}\mbox{}\\

\begin{itemize}
\item \href{https://jira.lsstcorp.org/browse/LVV-100}{LVV-100} - DMS-REQ-0269-V-01: DIASource Catalog

\end{itemize}

\paragraph{Test Items}\mbox{}\\

Verify that the DMS produces a Source catalog from Difference Exposures
with the required attributes.








\paragraph{Test Procedure}\mbox{}\\
\begin{tabular}{p{4cm}p{12cm}}
\toprule
Step 1-1
{\scriptsize from \hyperref[lvv-t860]{LVV-T860} }
& Description \\ \hline
\end{tabular}
{\scriptsize
The `path` that you will use depends on where you are running the
science pipelines. Options:\\[2\baselineskip]

\begin{itemize}
\tightlist
\item
  local (newinstall.sh - based
  install):{[}path\_to\_installation{]}/loadLSST.bash
\item
  development cluster (``lsst-dev''):
  /software/lsstsw/stack/loadLSST.bash
\item
  LSP Notebook aspect (from a terminal):
  /opt/lsst/software/stack/loadLSST.bash
\end{itemize}

From the command line, execute the commands below in the example
code:\\[2\baselineskip]

}
\begin{tabular}{p{3cm}p{13cm}}
\hline
            & Example Code \\ \hline
\end{tabular}
{\scriptsize
source `path`\\
setup lsst\_distrib

}
\begin{tabular}{p{3cm}p{13cm}}
\hline
            & Expected Result \\ \hline
\end{tabular}
{\scriptsize
Science pipeline software is available for use. If additional packages
are needed (for example, `obs' packages such as `obs\_subaru`), then
additional `setup` commands will be necessary.\\[2\baselineskip]To check
versions in use, type:\\
eups list -s

}

\begin{tabular}{p{4cm}p{12cm}}
\toprule
Step 2-1
{\scriptsize from \hyperref[lvv-t866]{LVV-T866} }
& Description \\ \hline
\end{tabular}
{\scriptsize
Perform the steps of Alert Production (including, but not necessarily
limited to, single frame processing, ISR, source detection/measurement,
PSF estimation, photometric and astrometric calibration, difference
imaging, DIASource detection/measurement, source association). During
Operations, it is presumed that these are automated for a given
dataset.~

}
\begin{tabular}{p{3cm}p{13cm}}
\hline
            & Expected Result \\ \hline
\end{tabular}
{\scriptsize
An output dataset including difference images and DIASource and
DIAObject measurements.

}

\begin{tabular}{p{4cm}p{12cm}}
\toprule
Step 2-2
{\scriptsize from \hyperref[lvv-t866]{LVV-T866} }
& Description \\ \hline
\end{tabular}
{\scriptsize
Verify that the expected data products have been produced, and that
catalogs contain reasonable values for measured quantities of interest.

}
\begin{tabular}{p{3cm}p{13cm}}
\hline
            & Expected Result \\ \hline
\end{tabular}

\begin{tabular}{p{4cm}p{12cm}}
\toprule
Step 3-1
{\scriptsize from \hyperref[lvv-t987]{LVV-T987} }
& Description \\ \hline
\end{tabular}
{\scriptsize
Identify the path to the data repository, which we will refer to as
`DATA/path', then execute the following:

}
\begin{tabular}{p{3cm}p{13cm}}
\hline
            & Example Code \\ \hline
\end{tabular}
{\scriptsize
\begin{verbatim}
import lsst.daf.persistence as dafPersist
butler = dafPersist.Butler(inputs='DATA/path')
\end{verbatim}

}
\begin{tabular}{p{3cm}p{13cm}}
\hline
            & Expected Result \\ \hline
\end{tabular}
{\scriptsize
Butler repo available for reading.

}

\begin{tabular}{p{4cm}p{12cm}}
\toprule
Step 4
& Description \\ \hline
\end{tabular}
{\scriptsize
Verify that products are produced for DIASource catalog

}
\begin{tabular}{p{3cm}p{13cm}}
\hline
            & Expected Result \\ \hline
\end{tabular}

\subsubsection{LVV-T50 - Verify implementation of Faint DIASource Measurements}\label{lvv-t50}

\begin{longtable}[]{llllll}
\toprule
Version & Status & Priority & Verification Type & Owner
\\\midrule
1 & Draft & Normal &
Test & Eric Bellm
\\\bottomrule
\multicolumn{6}{c}{ Open \href{https://jira.lsstcorp.org/secure/Tests.jspa\#/testCase/LVV-T50}{LVV-T50} in Jira } \\
\end{longtable}

\paragraph{Verification Elements}\mbox{}\\

\begin{itemize}
\item \href{https://jira.lsstcorp.org/browse/LVV-101}{LVV-101} - DMS-REQ-0270-V-01: Faint DIASource Measurements

\end{itemize}

\paragraph{Test Items}\mbox{}\\

Verify that the DMS can produces DIASources measurements for sources
below the nominal S/N cutoff that satisfy additional criteria.








\paragraph{Test Procedure}\mbox{}\\
\begin{tabular}{p{4cm}p{12cm}}
\toprule
Step 1-1
{\scriptsize from \hyperref[lvv-t860]{LVV-T860} }
& Description \\ \hline
\end{tabular}
{\scriptsize
The `path` that you will use depends on where you are running the
science pipelines. Options:\\[2\baselineskip]

\begin{itemize}
\tightlist
\item
  local (newinstall.sh - based
  install):{[}path\_to\_installation{]}/loadLSST.bash
\item
  development cluster (``lsst-dev''):
  /software/lsstsw/stack/loadLSST.bash
\item
  LSP Notebook aspect (from a terminal):
  /opt/lsst/software/stack/loadLSST.bash
\end{itemize}

From the command line, execute the commands below in the example
code:\\[2\baselineskip]

}
\begin{tabular}{p{3cm}p{13cm}}
\hline
            & Example Code \\ \hline
\end{tabular}
{\scriptsize
source `path`\\
setup lsst\_distrib

}
\begin{tabular}{p{3cm}p{13cm}}
\hline
            & Expected Result \\ \hline
\end{tabular}
{\scriptsize
Science pipeline software is available for use. If additional packages
are needed (for example, `obs' packages such as `obs\_subaru`), then
additional `setup` commands will be necessary.\\[2\baselineskip]To check
versions in use, type:\\
eups list -s

}

\begin{tabular}{p{4cm}p{12cm}}
\toprule
Step 2-1
{\scriptsize from \hyperref[lvv-t866]{LVV-T866} }
& Description \\ \hline
\end{tabular}
{\scriptsize
Perform the steps of Alert Production (including, but not necessarily
limited to, single frame processing, ISR, source detection/measurement,
PSF estimation, photometric and astrometric calibration, difference
imaging, DIASource detection/measurement, source association). During
Operations, it is presumed that these are automated for a given
dataset.~

}
\begin{tabular}{p{3cm}p{13cm}}
\hline
            & Expected Result \\ \hline
\end{tabular}
{\scriptsize
An output dataset including difference images and DIASource and
DIAObject measurements.

}

\begin{tabular}{p{4cm}p{12cm}}
\toprule
Step 2-2
{\scriptsize from \hyperref[lvv-t866]{LVV-T866} }
& Description \\ \hline
\end{tabular}
{\scriptsize
Verify that the expected data products have been produced, and that
catalogs contain reasonable values for measured quantities of interest.

}
\begin{tabular}{p{3cm}p{13cm}}
\hline
            & Expected Result \\ \hline
\end{tabular}

\begin{tabular}{p{4cm}p{12cm}}
\toprule
Step 3
& Description \\ \hline
\end{tabular}
{\scriptsize
As an example of selecting with constrains, Re-run source detection as
an afterburner to select isolated sources (defined as more than 2
arcseconds away from any other objects in the single-image-depth
catalog) that are fainter than the fiducial transSNR cut.

}
\begin{tabular}{p{3cm}p{13cm}}
\hline
            & Expected Result \\ \hline
\end{tabular}

\subsubsection{LVV-T51 - Verify implementation of DIAObject Catalog}\label{lvv-t51}

\begin{longtable}[]{llllll}
\toprule
Version & Status & Priority & Verification Type & Owner
\\\midrule
1 & Draft & Normal &
Test & Eric Bellm
\\\bottomrule
\multicolumn{6}{c}{ Open \href{https://jira.lsstcorp.org/secure/Tests.jspa\#/testCase/LVV-T51}{LVV-T51} in Jira } \\
\end{longtable}

\paragraph{Verification Elements}\mbox{}\\

\begin{itemize}
\item \href{https://jira.lsstcorp.org/browse/LVV-102}{LVV-102} - DMS-REQ-0271-V-01: Max nearby galaxies associated with DIASource

\end{itemize}

\paragraph{Test Items}\mbox{}\\

Verify that the DIAObject includes a unique ID, identifiers for nearest
stars and nearest galaxies, and probability of matching to static
Object.








\paragraph{Test Procedure}\mbox{}\\
\begin{tabular}{p{4cm}p{12cm}}
\toprule
Step 1-1
{\scriptsize from \hyperref[lvv-t866]{LVV-T866} }
& Description \\ \hline
\end{tabular}
{\scriptsize
Perform the steps of Alert Production (including, but not necessarily
limited to, single frame processing, ISR, source detection/measurement,
PSF estimation, photometric and astrometric calibration, difference
imaging, DIASource detection/measurement, source association). During
Operations, it is presumed that these are automated for a given
dataset.~

}
\begin{tabular}{p{3cm}p{13cm}}
\hline
            & Expected Result \\ \hline
\end{tabular}
{\scriptsize
An output dataset including difference images and DIASource and
DIAObject measurements.

}

\begin{tabular}{p{4cm}p{12cm}}
\toprule
Step 1-2
{\scriptsize from \hyperref[lvv-t866]{LVV-T866} }
& Description \\ \hline
\end{tabular}
{\scriptsize
Verify that the expected data products have been produced, and that
catalogs contain reasonable values for measured quantities of interest.

}
\begin{tabular}{p{3cm}p{13cm}}
\hline
            & Expected Result \\ \hline
\end{tabular}

\begin{tabular}{p{4cm}p{12cm}}
\toprule
Step 2-1
{\scriptsize from \hyperref[lvv-t987]{LVV-T987} }
& Description \\ \hline
\end{tabular}
{\scriptsize
Identify the path to the data repository, which we will refer to as
`DATA/path', then execute the following:

}
\begin{tabular}{p{3cm}p{13cm}}
\hline
            & Example Code \\ \hline
\end{tabular}
{\scriptsize
\begin{verbatim}
import lsst.daf.persistence as dafPersist
butler = dafPersist.Butler(inputs='DATA/path')
\end{verbatim}

}
\begin{tabular}{p{3cm}p{13cm}}
\hline
            & Expected Result \\ \hline
\end{tabular}
{\scriptsize
Butler repo available for reading.

}

\begin{tabular}{p{4cm}p{12cm}}
\toprule
Step 3
& Description \\ \hline
\end{tabular}
{\scriptsize
Verify that DIAObjects have diaNearbyObjMaxStar and
diaNearbyObjMaxGalaxies that point to the Object catalog and are within
dianNearbyObjRadius; the probability of association; and the required
DIAObject properties.

}
\begin{tabular}{p{3cm}p{13cm}}
\hline
            & Expected Result \\ \hline
\end{tabular}

\subsubsection{LVV-T52 - Verify implementation of DIAObject Attributes}\label{lvv-t52}

\begin{longtable}[]{llllll}
\toprule
Version & Status & Priority & Verification Type & Owner
\\\midrule
1 & Draft & Normal &
Test & Eric Bellm
\\\bottomrule
\multicolumn{6}{c}{ Open \href{https://jira.lsstcorp.org/secure/Tests.jspa\#/testCase/LVV-T52}{LVV-T52} in Jira } \\
\end{longtable}

\paragraph{Verification Elements}\mbox{}\\

\begin{itemize}
\item \href{https://jira.lsstcorp.org/browse/LVV-103}{LVV-103} - DMS-REQ-0272-V-01: DIAObject Attributes

\end{itemize}

\paragraph{Test Items}\mbox{}\\

Verify that the DMS provides summary attributes for each DIAObject,
including periodicity measures.








\paragraph{Test Procedure}\mbox{}\\
\begin{tabular}{p{4cm}p{12cm}}
\toprule
Step 1-1
{\scriptsize from \hyperref[lvv-t866]{LVV-T866} }
& Description \\ \hline
\end{tabular}
{\scriptsize
Perform the steps of Alert Production (including, but not necessarily
limited to, single frame processing, ISR, source detection/measurement,
PSF estimation, photometric and astrometric calibration, difference
imaging, DIASource detection/measurement, source association). During
Operations, it is presumed that these are automated for a given
dataset.~

}
\begin{tabular}{p{3cm}p{13cm}}
\hline
            & Expected Result \\ \hline
\end{tabular}
{\scriptsize
An output dataset including difference images and DIASource and
DIAObject measurements.

}

\begin{tabular}{p{4cm}p{12cm}}
\toprule
Step 1-2
{\scriptsize from \hyperref[lvv-t866]{LVV-T866} }
& Description \\ \hline
\end{tabular}
{\scriptsize
Verify that the expected data products have been produced, and that
catalogs contain reasonable values for measured quantities of interest.

}
\begin{tabular}{p{3cm}p{13cm}}
\hline
            & Expected Result \\ \hline
\end{tabular}

\begin{tabular}{p{4cm}p{12cm}}
\toprule
Step 2-1
{\scriptsize from \hyperref[lvv-t987]{LVV-T987} }
& Description \\ \hline
\end{tabular}
{\scriptsize
Identify the path to the data repository, which we will refer to as
`DATA/path', then execute the following:

}
\begin{tabular}{p{3cm}p{13cm}}
\hline
            & Example Code \\ \hline
\end{tabular}
{\scriptsize
\begin{verbatim}
import lsst.daf.persistence as dafPersist
butler = dafPersist.Butler(inputs='DATA/path')
\end{verbatim}

}
\begin{tabular}{p{3cm}p{13cm}}
\hline
            & Expected Result \\ \hline
\end{tabular}
{\scriptsize
Butler repo available for reading.

}

\begin{tabular}{p{4cm}p{12cm}}
\toprule
Step 3
& Description \\ \hline
\end{tabular}
{\scriptsize
Confirm that the DIAObjects include summary attributes as specified.

}
\begin{tabular}{p{3cm}p{13cm}}
\hline
            & Expected Result \\ \hline
\end{tabular}

\subsubsection{LVV-T53 - Verify implementation of SSObject Catalog}\label{lvv-t53}

\begin{longtable}[]{llllll}
\toprule
Version & Status & Priority & Verification Type & Owner
\\\midrule
1 & Draft & Normal &
Test & Eric Bellm
\\\bottomrule
\multicolumn{6}{c}{ Open \href{https://jira.lsstcorp.org/secure/Tests.jspa\#/testCase/LVV-T53}{LVV-T53} in Jira } \\
\end{longtable}

\paragraph{Verification Elements}\mbox{}\\

\begin{itemize}
\item \href{https://jira.lsstcorp.org/browse/LVV-104}{LVV-104} - DMS-REQ-0273-V-01: SSObject Catalog

\end{itemize}

\paragraph{Test Items}\mbox{}\\

Verify that the DMS produces a catalog of Solar System Objects identify
from Moving Object Processing.\\
Verify that the SSObject catalog includes orbital elements and
additional related quanitites.








\paragraph{Test Procedure}\mbox{}\\
\begin{tabular}{p{4cm}p{12cm}}
\toprule
Step 1-1
{\scriptsize from \hyperref[lvv-t866]{LVV-T866} }
& Description \\ \hline
\end{tabular}
{\scriptsize
Perform the steps of Alert Production (including, but not necessarily
limited to, single frame processing, ISR, source detection/measurement,
PSF estimation, photometric and astrometric calibration, difference
imaging, DIASource detection/measurement, source association). During
Operations, it is presumed that these are automated for a given
dataset.~

}
\begin{tabular}{p{3cm}p{13cm}}
\hline
            & Expected Result \\ \hline
\end{tabular}
{\scriptsize
An output dataset including difference images and DIASource and
DIAObject measurements.

}

\begin{tabular}{p{4cm}p{12cm}}
\toprule
Step 1-2
{\scriptsize from \hyperref[lvv-t866]{LVV-T866} }
& Description \\ \hline
\end{tabular}
{\scriptsize
Verify that the expected data products have been produced, and that
catalogs contain reasonable values for measured quantities of interest.

}
\begin{tabular}{p{3cm}p{13cm}}
\hline
            & Expected Result \\ \hline
\end{tabular}

\begin{tabular}{p{4cm}p{12cm}}
\toprule
Step 2-1
{\scriptsize from \hyperref[lvv-t901]{LVV-T901} }
& Description \\ \hline
\end{tabular}
{\scriptsize
Perform the steps of Moving Object Pipeline (MOPS) processing on newly
detected DIASources, and generate Solar System data products including
Solar System objects with associated Keplerian orbits, errors, and
detected DIASources. This includes running processes to link DIASource
detections within a night (called tracklets), to link these tracklets
across multiple nights (into tracks), to fit the tracks with an orbital
model to identify those tracks that are consistent with an asteroid
orbit, to match these new orbits with existing SSObjects, and to update
the SSObject table. ~ ~ ~ ~ ~ ~ ~ ~ ~ ~ ~ ~ ~ ~ ~ ~ ~ ~ ~~

}
\begin{tabular}{p{3cm}p{13cm}}
\hline
            & Expected Result \\ \hline
\end{tabular}
{\scriptsize
An output dataset consisting of an updated SSObject database with
SSObjects both added and pruned as the orbital fits have been refined,
and an updated DIASource database with DIASources assigned and
unassigned to SSObjects.

}

\begin{tabular}{p{4cm}p{12cm}}
\toprule
Step 2-2
{\scriptsize from \hyperref[lvv-t901]{LVV-T901} }
& Description \\ \hline
\end{tabular}
{\scriptsize
Verify that the expected data products have been produced, and that
catalogs contain reasonable values for measured quantities of interest.

}
\begin{tabular}{p{3cm}p{13cm}}
\hline
            & Expected Result \\ \hline
\end{tabular}

\begin{tabular}{p{4cm}p{12cm}}
\toprule
Step 3-1
{\scriptsize from \hyperref[lvv-t987]{LVV-T987} }
& Description \\ \hline
\end{tabular}
{\scriptsize
Identify the path to the data repository, which we will refer to as
`DATA/path', then execute the following:

}
\begin{tabular}{p{3cm}p{13cm}}
\hline
            & Example Code \\ \hline
\end{tabular}
{\scriptsize
\begin{verbatim}
import lsst.daf.persistence as dafPersist
butler = dafPersist.Butler(inputs='DATA/path')
\end{verbatim}

}
\begin{tabular}{p{3cm}p{13cm}}
\hline
            & Expected Result \\ \hline
\end{tabular}
{\scriptsize
Butler repo available for reading.

}

\begin{tabular}{p{4cm}p{12cm}}
\toprule
Step 4
& Description \\ \hline
\end{tabular}
{\scriptsize
Inspect SSObject catalog and verify the presence of the required
elements (​\href{https://jira.lsstcorp.org/browse/LVV-104}{LVV-104)}​​​.

}
\begin{tabular}{p{3cm}p{13cm}}
\hline
            & Expected Result \\ \hline
\end{tabular}

\subsubsection{LVV-T54 - Verify implementation of Alert Content}\label{lvv-t54}

\begin{longtable}[]{llllll}
\toprule
Version & Status & Priority & Verification Type & Owner
\\\midrule
1 & Draft & Normal &
Test & Eric Bellm
\\\bottomrule
\multicolumn{6}{c}{ Open \href{https://jira.lsstcorp.org/secure/Tests.jspa\#/testCase/LVV-T54}{LVV-T54} in Jira } \\
\end{longtable}

\paragraph{Verification Elements}\mbox{}\\

\begin{itemize}
\item \href{https://jira.lsstcorp.org/browse/LVV-105}{LVV-105} - DMS-REQ-0274-V-01: Alert Content

\end{itemize}

\paragraph{Test Items}\mbox{}\\

Verify that the DMS creates an Alert for each detected DIASource\\
Verify that this Alert is broadcasted using community protocols\\
Verify that the context of the Alert packet match requirements.








\paragraph{Test Procedure}\mbox{}\\
\begin{tabular}{p{4cm}p{12cm}}
\toprule
Step 1-1
{\scriptsize from \hyperref[lvv-t866]{LVV-T866} }
& Description \\ \hline
\end{tabular}
{\scriptsize
Perform the steps of Alert Production (including, but not necessarily
limited to, single frame processing, ISR, source detection/measurement,
PSF estimation, photometric and astrometric calibration, difference
imaging, DIASource detection/measurement, source association). During
Operations, it is presumed that these are automated for a given
dataset.~

}
\begin{tabular}{p{3cm}p{13cm}}
\hline
            & Expected Result \\ \hline
\end{tabular}
{\scriptsize
An output dataset including difference images and DIASource and
DIAObject measurements.

}

\begin{tabular}{p{4cm}p{12cm}}
\toprule
Step 1-2
{\scriptsize from \hyperref[lvv-t866]{LVV-T866} }
& Description \\ \hline
\end{tabular}
{\scriptsize
Verify that the expected data products have been produced, and that
catalogs contain reasonable values for measured quantities of interest.

}
\begin{tabular}{p{3cm}p{13cm}}
\hline
            & Expected Result \\ \hline
\end{tabular}

\begin{tabular}{p{4cm}p{12cm}}
\toprule
Step 2
& Description \\ \hline
\end{tabular}
{\scriptsize
Examine the serialized alert packets to confirm the presence of the
required elements
(\href{https://jira.lsstcorp.org/browse/LVV-105}{LVV-105}).~ ~ ~ ~ ~ ~ ~
~ ~ ~ ~ ~ ~ ~ ~ ~ ~

}
\begin{tabular}{p{3cm}p{13cm}}
\hline
            & Expected Result \\ \hline
\end{tabular}

\subsubsection{LVV-T55 - Verify implementation of DIAForcedSource Catalog}\label{lvv-t55}

\begin{longtable}[]{llllll}
\toprule
Version & Status & Priority & Verification Type & Owner
\\\midrule
1 & Draft & Normal &
Test & Eric Bellm
\\\bottomrule
\multicolumn{6}{c}{ Open \href{https://jira.lsstcorp.org/secure/Tests.jspa\#/testCase/LVV-T55}{LVV-T55} in Jira } \\
\end{longtable}

\paragraph{Verification Elements}\mbox{}\\

\begin{itemize}
\item \href{https://jira.lsstcorp.org/browse/LVV-148}{LVV-148} - DMS-REQ-0317-V-01: DIAForcedSource Catalog

\end{itemize}

\paragraph{Test Items}\mbox{}\\

Verify that the DMS produces a DIAForcedSource Catalog and that the
catalog contains measured fluxes for DIAObjects.








\paragraph{Test Procedure}\mbox{}\\
\begin{tabular}{p{4cm}p{12cm}}
\toprule
Step 1-1
{\scriptsize from \hyperref[lvv-t866]{LVV-T866} }
& Description \\ \hline
\end{tabular}
{\scriptsize
Perform the steps of Alert Production (including, but not necessarily
limited to, single frame processing, ISR, source detection/measurement,
PSF estimation, photometric and astrometric calibration, difference
imaging, DIASource detection/measurement, source association). During
Operations, it is presumed that these are automated for a given
dataset.~

}
\begin{tabular}{p{3cm}p{13cm}}
\hline
            & Expected Result \\ \hline
\end{tabular}
{\scriptsize
An output dataset including difference images and DIASource and
DIAObject measurements.

}

\begin{tabular}{p{4cm}p{12cm}}
\toprule
Step 1-2
{\scriptsize from \hyperref[lvv-t866]{LVV-T866} }
& Description \\ \hline
\end{tabular}
{\scriptsize
Verify that the expected data products have been produced, and that
catalogs contain reasonable values for measured quantities of interest.

}
\begin{tabular}{p{3cm}p{13cm}}
\hline
            & Expected Result \\ \hline
\end{tabular}

\begin{tabular}{p{4cm}p{12cm}}
\toprule
Step 2-1
{\scriptsize from \hyperref[lvv-t987]{LVV-T987} }
& Description \\ \hline
\end{tabular}
{\scriptsize
Identify the path to the data repository, which we will refer to as
`DATA/path', then execute the following:

}
\begin{tabular}{p{3cm}p{13cm}}
\hline
            & Example Code \\ \hline
\end{tabular}
{\scriptsize
\begin{verbatim}
import lsst.daf.persistence as dafPersist
butler = dafPersist.Butler(inputs='DATA/path')
\end{verbatim}

}
\begin{tabular}{p{3cm}p{13cm}}
\hline
            & Expected Result \\ \hline
\end{tabular}
{\scriptsize
Butler repo available for reading.

}

\begin{tabular}{p{4cm}p{12cm}}
\toprule
Step 3
& Description \\ \hline
\end{tabular}
{\scriptsize
Confirm that the DIAForcedSource catalog contains measurements for each
source.

}
\begin{tabular}{p{3cm}p{13cm}}
\hline
            & Expected Result \\ \hline
\end{tabular}

\subsubsection{LVV-T56 - Verify implementation of Characterizing Variability}\label{lvv-t56}

\begin{longtable}[]{llllll}
\toprule
Version & Status & Priority & Verification Type & Owner
\\\midrule
1 & Draft & Normal &
Test & Eric Bellm
\\\bottomrule
\multicolumn{6}{c}{ Open \href{https://jira.lsstcorp.org/secure/Tests.jspa\#/testCase/LVV-T56}{LVV-T56} in Jira } \\
\end{longtable}

\paragraph{Verification Elements}\mbox{}\\

\begin{itemize}
\item \href{https://jira.lsstcorp.org/browse/LVV-150}{LVV-150} - DMS-REQ-0319-V-01: Characterizing Variability

\end{itemize}

\paragraph{Test Items}\mbox{}\\

Verify that the variability characterization in the DIAObject catalog
includes data collected within previous ``diaCharacterizationCutoff''
period of time.








\paragraph{Test Procedure}\mbox{}\\
\begin{tabular}{p{4cm}p{12cm}}
\toprule
Step 1-1
{\scriptsize from \hyperref[lvv-t866]{LVV-T866} }
& Description \\ \hline
\end{tabular}
{\scriptsize
Perform the steps of Alert Production (including, but not necessarily
limited to, single frame processing, ISR, source detection/measurement,
PSF estimation, photometric and astrometric calibration, difference
imaging, DIASource detection/measurement, source association). During
Operations, it is presumed that these are automated for a given
dataset.~

}
\begin{tabular}{p{3cm}p{13cm}}
\hline
            & Expected Result \\ \hline
\end{tabular}
{\scriptsize
An output dataset including difference images and DIASource and
DIAObject measurements.

}

\begin{tabular}{p{4cm}p{12cm}}
\toprule
Step 1-2
{\scriptsize from \hyperref[lvv-t866]{LVV-T866} }
& Description \\ \hline
\end{tabular}
{\scriptsize
Verify that the expected data products have been produced, and that
catalogs contain reasonable values for measured quantities of interest.

}
\begin{tabular}{p{3cm}p{13cm}}
\hline
            & Expected Result \\ \hline
\end{tabular}

\begin{tabular}{p{4cm}p{12cm}}
\toprule
Step 2
& Description \\ \hline
\end{tabular}
{\scriptsize
Verify that the issued alerts contain measurements during the
diaCharacterizationCutoff.

}
\begin{tabular}{p{3cm}p{13cm}}
\hline
            & Expected Result \\ \hline
\end{tabular}

\subsubsection{LVV-T57 - Verify implementation of Calculating SSObject Parameters}\label{lvv-t57}

\begin{longtable}[]{llllll}
\toprule
Version & Status & Priority & Verification Type & Owner
\\\midrule
1 & Draft & Normal &
Test & Eric Bellm
\\\bottomrule
\multicolumn{6}{c}{ Open \href{https://jira.lsstcorp.org/secure/Tests.jspa\#/testCase/LVV-T57}{LVV-T57} in Jira } \\
\end{longtable}

\paragraph{Verification Elements}\mbox{}\\

\begin{itemize}
\item \href{https://jira.lsstcorp.org/browse/LVV-154}{LVV-154} - DMS-REQ-0323-V-01: Calculating SSObject Parameters

\end{itemize}

\paragraph{Test Items}\mbox{}\\

Verify that the DMS database provides functions to compute phase angles
and magnitudes in LSST bands for every SSObject.








\paragraph{Test Procedure}\mbox{}\\
\begin{tabular}{p{4cm}p{12cm}}
\toprule
Step 1-1
{\scriptsize from \hyperref[lvv-t866]{LVV-T866} }
& Description \\ \hline
\end{tabular}
{\scriptsize
Perform the steps of Alert Production (including, but not necessarily
limited to, single frame processing, ISR, source detection/measurement,
PSF estimation, photometric and astrometric calibration, difference
imaging, DIASource detection/measurement, source association). During
Operations, it is presumed that these are automated for a given
dataset.~

}
\begin{tabular}{p{3cm}p{13cm}}
\hline
            & Expected Result \\ \hline
\end{tabular}
{\scriptsize
An output dataset including difference images and DIASource and
DIAObject measurements.

}

\begin{tabular}{p{4cm}p{12cm}}
\toprule
Step 1-2
{\scriptsize from \hyperref[lvv-t866]{LVV-T866} }
& Description \\ \hline
\end{tabular}
{\scriptsize
Verify that the expected data products have been produced, and that
catalogs contain reasonable values for measured quantities of interest.

}
\begin{tabular}{p{3cm}p{13cm}}
\hline
            & Expected Result \\ \hline
\end{tabular}

\begin{tabular}{p{4cm}p{12cm}}
\toprule
Step 2-1
{\scriptsize from \hyperref[lvv-t901]{LVV-T901} }
& Description \\ \hline
\end{tabular}
{\scriptsize
Perform the steps of Moving Object Pipeline (MOPS) processing on newly
detected DIASources, and generate Solar System data products including
Solar System objects with associated Keplerian orbits, errors, and
detected DIASources. This includes running processes to link DIASource
detections within a night (called tracklets), to link these tracklets
across multiple nights (into tracks), to fit the tracks with an orbital
model to identify those tracks that are consistent with an asteroid
orbit, to match these new orbits with existing SSObjects, and to update
the SSObject table. ~ ~ ~ ~ ~ ~ ~ ~ ~ ~ ~ ~ ~ ~ ~ ~ ~ ~ ~~

}
\begin{tabular}{p{3cm}p{13cm}}
\hline
            & Expected Result \\ \hline
\end{tabular}
{\scriptsize
An output dataset consisting of an updated SSObject database with
SSObjects both added and pruned as the orbital fits have been refined,
and an updated DIASource database with DIASources assigned and
unassigned to SSObjects.

}

\begin{tabular}{p{4cm}p{12cm}}
\toprule
Step 2-2
{\scriptsize from \hyperref[lvv-t901]{LVV-T901} }
& Description \\ \hline
\end{tabular}
{\scriptsize
Verify that the expected data products have been produced, and that
catalogs contain reasonable values for measured quantities of interest.

}
\begin{tabular}{p{3cm}p{13cm}}
\hline
            & Expected Result \\ \hline
\end{tabular}

\begin{tabular}{p{4cm}p{12cm}}
\toprule
Step 3
& Description \\ \hline
\end{tabular}
{\scriptsize
Computer the phase angle, reduced and absolute asteroid magnitudes for
objects identified in SSObject Catalog

}
\begin{tabular}{p{3cm}p{13cm}}
\hline
            & Expected Result \\ \hline
\end{tabular}

\subsubsection{LVV-T58 - Verify implementation of Matching DIASources to Objects}\label{lvv-t58}

\begin{longtable}[]{llllll}
\toprule
Version & Status & Priority & Verification Type & Owner
\\\midrule
1 & Draft & Normal &
Test & Eric Bellm
\\\bottomrule
\multicolumn{6}{c}{ Open \href{https://jira.lsstcorp.org/secure/Tests.jspa\#/testCase/LVV-T58}{LVV-T58} in Jira } \\
\end{longtable}

\paragraph{Verification Elements}\mbox{}\\

\begin{itemize}
\item \href{https://jira.lsstcorp.org/browse/LVV-155}{LVV-155} - DMS-REQ-0324-V-01: Matching DIASources to Objects

\end{itemize}

\paragraph{Test Items}\mbox{}\\

Verify that a cross-match table is available between DIASources and
Objects.








\paragraph{Test Procedure}\mbox{}\\
\begin{tabular}{p{4cm}p{12cm}}
\toprule
Step 1-1
{\scriptsize from \hyperref[lvv-t866]{LVV-T866} }
& Description \\ \hline
\end{tabular}
{\scriptsize
Perform the steps of Alert Production (including, but not necessarily
limited to, single frame processing, ISR, source detection/measurement,
PSF estimation, photometric and astrometric calibration, difference
imaging, DIASource detection/measurement, source association). During
Operations, it is presumed that these are automated for a given
dataset.~

}
\begin{tabular}{p{3cm}p{13cm}}
\hline
            & Expected Result \\ \hline
\end{tabular}
{\scriptsize
An output dataset including difference images and DIASource and
DIAObject measurements.

}

\begin{tabular}{p{4cm}p{12cm}}
\toprule
Step 1-2
{\scriptsize from \hyperref[lvv-t866]{LVV-T866} }
& Description \\ \hline
\end{tabular}
{\scriptsize
Verify that the expected data products have been produced, and that
catalogs contain reasonable values for measured quantities of interest.

}
\begin{tabular}{p{3cm}p{13cm}}
\hline
            & Expected Result \\ \hline
\end{tabular}

\begin{tabular}{p{4cm}p{12cm}}
\toprule
Step 2-1
{\scriptsize from \hyperref[lvv-t987]{LVV-T987} }
& Description \\ \hline
\end{tabular}
{\scriptsize
Identify the path to the data repository, which we will refer to as
`DATA/path', then execute the following:

}
\begin{tabular}{p{3cm}p{13cm}}
\hline
            & Example Code \\ \hline
\end{tabular}
{\scriptsize
\begin{verbatim}
import lsst.daf.persistence as dafPersist
butler = dafPersist.Butler(inputs='DATA/path')
\end{verbatim}

}
\begin{tabular}{p{3cm}p{13cm}}
\hline
            & Expected Result \\ \hline
\end{tabular}
{\scriptsize
Butler repo available for reading.

}

\begin{tabular}{p{4cm}p{12cm}}
\toprule
Step 3
& Description \\ \hline
\end{tabular}
{\scriptsize
Verify that a cross-match table between the Prompt DIASources and DRP
Objects is available.

}
\begin{tabular}{p{3cm}p{13cm}}
\hline
            & Expected Result \\ \hline
\end{tabular}

\subsubsection{LVV-T59 - Verify implementation of Regenerating L1 Data Products During Data
Release Processing}\label{lvv-t59}

\begin{longtable}[]{llllll}
\toprule
Version & Status & Priority & Verification Type & Owner
\\\midrule
1 & Draft & Normal &
Test & Kian-Tat Lim
\\\bottomrule
\multicolumn{6}{c}{ Open \href{https://jira.lsstcorp.org/secure/Tests.jspa\#/testCase/LVV-T59}{LVV-T59} in Jira } \\
\end{longtable}

\paragraph{Verification Elements}\mbox{}\\

\begin{itemize}
\item \href{https://jira.lsstcorp.org/browse/LVV-156}{LVV-156} - DMS-REQ-0325-V-01: Regenerating L1 Data Products During Data Release
Processing

\end{itemize}

\paragraph{Test Items}\mbox{}\\

Verify that the Prompt Processing data products are regenerated during
DRP.








\paragraph{Test Procedure}\mbox{}\\
\begin{tabular}{p{4cm}p{12cm}}
\toprule
Step 1
& Description \\ \hline
\end{tabular}
{\scriptsize
Execute DRP

}
\begin{tabular}{p{3cm}p{13cm}}
\hline
            & Expected Result \\ \hline
\end{tabular}

\begin{tabular}{p{4cm}p{12cm}}
\toprule
Step 2
& Description \\ \hline
\end{tabular}
{\scriptsize
Observe production of difference image data products

}
\begin{tabular}{p{3cm}p{13cm}}
\hline
            & Expected Result \\ \hline
\end{tabular}

\subsubsection{LVV-T60 - Verify implementation of Publishing predicted visit schedule}\label{lvv-t60}

\begin{longtable}[]{llllll}
\toprule
Version & Status & Priority & Verification Type & Owner
\\\midrule
1 & Draft & Normal &
Test & Eric Bellm
\\\bottomrule
\multicolumn{6}{c}{ Open \href{https://jira.lsstcorp.org/secure/Tests.jspa\#/testCase/LVV-T60}{LVV-T60} in Jira } \\
\end{longtable}

\paragraph{Verification Elements}\mbox{}\\

\begin{itemize}
\item \href{https://jira.lsstcorp.org/browse/LVV-184}{LVV-184} - DMS-REQ-0353-V-01: Publishing predicted visit schedule

\end{itemize}

\paragraph{Test Items}\mbox{}\\

Verify that a predict-visit schedule can be published by the OCS.








\paragraph{Test Procedure}\mbox{}\\
\begin{tabular}{p{4cm}p{12cm}}
\toprule
Step 1
& Description \\ \hline
\end{tabular}
{\scriptsize

}
\begin{tabular}{p{3cm}p{13cm}}
\hline
            & Expected Result \\ \hline
\end{tabular}

\subsubsection{LVV-T63 - Verify implementation of Produce Images for EPO}\label{lvv-t63}

\begin{longtable}[]{llllll}
\toprule
Version & Status & Priority & Verification Type & Owner
\\\midrule
1 & Draft & Normal &
Test & Gregory Dubois-Felsmann
\\\bottomrule
\multicolumn{6}{c}{ Open \href{https://jira.lsstcorp.org/secure/Tests.jspa\#/testCase/LVV-T63}{LVV-T63} in Jira } \\
\end{longtable}

\paragraph{Verification Elements}\mbox{}\\

\begin{itemize}
\item \href{https://jira.lsstcorp.org/browse/LVV-45}{LVV-45} - DMS-REQ-0103-V-01: Produce Images for EPO

\end{itemize}

\paragraph{Test Items}\mbox{}\\

This test will verify that the DRP pipelines produce the image data
products called out in \citeds{LSE-131}. ~Currently this is limited to a color
all-sky HiPS map. ~This will be verified (1) by inspection of pipeline
configurations and (2) in operations rehearsals on precursor data. ~The
production of a usable HiPS map will be verified by browsing it with
community tools.








\paragraph{Test Procedure}\mbox{}\\
\begin{tabular}{p{4cm}p{12cm}}
\toprule
Step 1-1
{\scriptsize from \hyperref[lvv-t987]{LVV-T987} }
& Description \\ \hline
\end{tabular}
{\scriptsize
Identify the path to the data repository, which we will refer to as
`DATA/path', then execute the following:

}
\begin{tabular}{p{3cm}p{13cm}}
\hline
            & Example Code \\ \hline
\end{tabular}
{\scriptsize
\begin{verbatim}
import lsst.daf.persistence as dafPersist
butler = dafPersist.Butler(inputs='DATA/path')
\end{verbatim}

}
\begin{tabular}{p{3cm}p{13cm}}
\hline
            & Expected Result \\ \hline
\end{tabular}
{\scriptsize
Butler repo available for reading.

}

\begin{tabular}{p{4cm}p{12cm}}
\toprule
Step 2
& Description \\ \hline
\end{tabular}
{\scriptsize
For each of the expected data product types needed for creation of HiPS
images, retrieve the data product from the Butler and verify it to be
non-empty.

}
\begin{tabular}{p{3cm}p{13cm}}
\hline
            & Expected Result \\ \hline
\end{tabular}

\begin{tabular}{p{4cm}p{12cm}}
\toprule
Step 3
& Description \\ \hline
\end{tabular}
{\scriptsize
Verify that a HiPS image map covering the LSST survey area, with a
limiting depth yielding 1 arcsecond resolution, has been produced
matching the color prescriptions provided by EPO (in updates to LSE-131
which are expected to be made ``once ComCam data is available'').

}
\begin{tabular}{p{3cm}p{13cm}}
\hline
            & Expected Result \\ \hline
\end{tabular}

\begin{tabular}{p{4cm}p{12cm}}
\toprule
Step 4
& Description \\ \hline
\end{tabular}
{\scriptsize
Place the image map in a location accessible to a Firefly and an Aladin
Lite client, ideally with the client running in the EPO data systems
environment.

}
\begin{tabular}{p{3cm}p{13cm}}
\hline
            & Expected Result \\ \hline
\end{tabular}

\begin{tabular}{p{4cm}p{12cm}}
\toprule
Step 5
& Description \\ \hline
\end{tabular}
{\scriptsize
Use Firefly to manually explore the image map at the largest scales to
verify coverage of the entire sky. ~Sample in various locations to
confirm the 1 arcsecond maximum depth.\\
Confirm using Aladin Lite that the format of the image map is supported
by this common community tool.

}
\begin{tabular}{p{3cm}p{13cm}}
\hline
            & Expected Result \\ \hline
\end{tabular}

\begin{tabular}{p{4cm}p{12cm}}
\toprule
Step 6
& Description \\ \hline
\end{tabular}
{\scriptsize
Verify programmatically, perhaps both by sampling a variety of
locations, and by counting the tiles created at the
1-arcsecond-resolution depth, that the map is complete and meets its
specifications.

}
\begin{tabular}{p{3cm}p{13cm}}
\hline
            & Expected Result \\ \hline
\end{tabular}

\begin{tabular}{p{4cm}p{12cm}}
\toprule
Step 7
& Description \\ \hline
\end{tabular}
{\scriptsize
Apply an IVOA-community HiPS service validation tool, if available, to
the service location.

}
\begin{tabular}{p{3cm}p{13cm}}
\hline
            & Expected Result \\ \hline
\end{tabular}

\begin{tabular}{p{4cm}p{12cm}}
\toprule
Step 8
& Description \\ \hline
\end{tabular}
{\scriptsize
Verify that the HiPS map created is in a location accessible to the EPO
data systems.

}
\begin{tabular}{p{3cm}p{13cm}}
\hline
            & Expected Result \\ \hline
\end{tabular}

\subsubsection{LVV-T64 - Verify implementation of Coadded Image Provenance}\label{lvv-t64}

\begin{longtable}[]{llllll}
\toprule
Version & Status & Priority & Verification Type & Owner
\\\midrule
1 & Draft & Normal &
Test & Jim Bosch
\\\bottomrule
\multicolumn{6}{c}{ Open \href{https://jira.lsstcorp.org/secure/Tests.jspa\#/testCase/LVV-T64}{LVV-T64} in Jira } \\
\end{longtable}

\paragraph{Verification Elements}\mbox{}\\

\begin{itemize}
\item \href{https://jira.lsstcorp.org/browse/LVV-46}{LVV-46} - DMS-REQ-0106-V-01: Coadded Image Provenance

\item \href{https://jira.lsstcorp.org/browse/LVV-1234}{LVV-1234} - OSS-REQ-0122-V-01: Provenance

\end{itemize}

\paragraph{Test Items}\mbox{}\\

Verify that all coadd data products produced by the DRP pipelines are
associated with provenance information that includes the set of input
epochs contributing to that coadd as well as any additional information
needed to exactly produce that coadd.








\paragraph{Test Procedure}\mbox{}\\
\begin{tabular}{p{4cm}p{12cm}}
\toprule
Step 1-1
{\scriptsize from \hyperref[lvv-t860]{LVV-T860} }
& Description \\ \hline
\end{tabular}
{\scriptsize
The `path` that you will use depends on where you are running the
science pipelines. Options:\\[2\baselineskip]

\begin{itemize}
\tightlist
\item
  local (newinstall.sh - based
  install):{[}path\_to\_installation{]}/loadLSST.bash
\item
  development cluster (``lsst-dev''):
  /software/lsstsw/stack/loadLSST.bash
\item
  LSP Notebook aspect (from a terminal):
  /opt/lsst/software/stack/loadLSST.bash
\end{itemize}

From the command line, execute the commands below in the example
code:\\[2\baselineskip]

}
\begin{tabular}{p{3cm}p{13cm}}
\hline
            & Example Code \\ \hline
\end{tabular}
{\scriptsize
source `path`\\
setup lsst\_distrib

}
\begin{tabular}{p{3cm}p{13cm}}
\hline
            & Expected Result \\ \hline
\end{tabular}
{\scriptsize
Science pipeline software is available for use. If additional packages
are needed (for example, `obs' packages such as `obs\_subaru`), then
additional `setup` commands will be necessary.\\[2\baselineskip]To check
versions in use, type:\\
eups list -s

}

\begin{tabular}{p{4cm}p{12cm}}
\toprule
Step 2-1
{\scriptsize from \hyperref[lvv-t987]{LVV-T987} }
& Description \\ \hline
\end{tabular}
{\scriptsize
Identify the path to the data repository, which we will refer to as
`DATA/path', then execute the following:

}
\begin{tabular}{p{3cm}p{13cm}}
\hline
            & Example Code \\ \hline
\end{tabular}
{\scriptsize
\begin{verbatim}
import lsst.daf.persistence as dafPersist
butler = dafPersist.Butler(inputs='DATA/path')
\end{verbatim}

}
\begin{tabular}{p{3cm}p{13cm}}
\hline
            & Expected Result \\ \hline
\end{tabular}
{\scriptsize
Butler repo available for reading.

}

\begin{tabular}{p{4cm}p{12cm}}
\toprule
Step 3
& Description \\ \hline
\end{tabular}
{\scriptsize
For each of the expected data product types and each of the expected
units (PVIs, coadds, etc), retrieve the data product from the Butler and
verify it to be non-empty.

}
\begin{tabular}{p{3cm}p{13cm}}
\hline
            & Expected Result \\ \hline
\end{tabular}

\begin{tabular}{p{4cm}p{12cm}}
\toprule
Step 4
& Description \\ \hline
\end{tabular}
{\scriptsize
Query and verify provenance of input images, and software versions that
went into producing stack.

}
\begin{tabular}{p{3cm}p{13cm}}
\hline
            & Expected Result \\ \hline
\end{tabular}

\begin{tabular}{p{4cm}p{12cm}}
\toprule
Step 5
& Description \\ \hline
\end{tabular}
{\scriptsize
Test re-generating 10 different coadds tract+patches based on the
provenance image given

}
\begin{tabular}{p{3cm}p{13cm}}
\hline
            & Expected Result \\ \hline
\end{tabular}

\subsubsection{LVV-T66 - Verify implementation of Forced-Source Catalog}\label{lvv-t66}

\begin{longtable}[]{llllll}
\toprule
Version & Status & Priority & Verification Type & Owner
\\\midrule
1 & Draft & Normal &
Test & Jim Bosch
\\\bottomrule
\multicolumn{6}{c}{ Open \href{https://jira.lsstcorp.org/secure/Tests.jspa\#/testCase/LVV-T66}{LVV-T66} in Jira } \\
\end{longtable}

\paragraph{Verification Elements}\mbox{}\\

\begin{itemize}
\item \href{https://jira.lsstcorp.org/browse/LVV-99}{LVV-99} - DMS-REQ-0268-V-01: Forced-Source Catalog

\end{itemize}

\paragraph{Test Items}\mbox{}\\

Verify that all ForcedSources produced by the DRP pipelines contain
fluxes measured on difference and direct single-epoch images, associated
uncertainties, an Object ID, and a Visit ID.








\paragraph{Test Procedure}\mbox{}\\
\begin{tabular}{p{4cm}p{12cm}}
\toprule
Step 1-1
{\scriptsize from \hyperref[lvv-t987]{LVV-T987} }
& Description \\ \hline
\end{tabular}
{\scriptsize
Identify the path to the data repository, which we will refer to as
`DATA/path', then execute the following:

}
\begin{tabular}{p{3cm}p{13cm}}
\hline
            & Example Code \\ \hline
\end{tabular}
{\scriptsize
\begin{verbatim}
import lsst.daf.persistence as dafPersist
butler = dafPersist.Butler(inputs='DATA/path')
\end{verbatim}

}
\begin{tabular}{p{3cm}p{13cm}}
\hline
            & Expected Result \\ \hline
\end{tabular}
{\scriptsize
Butler repo available for reading.

}

\begin{tabular}{p{4cm}p{12cm}}
\toprule
Step 2
& Description \\ \hline
\end{tabular}
{\scriptsize
Retrieve the forced-source catalog from the Butler and verify it to be
non-empty.

}
\begin{tabular}{p{3cm}p{13cm}}
\hline
            & Expected Result \\ \hline
\end{tabular}

\begin{tabular}{p{4cm}p{12cm}}
\toprule
Step 3
& Description \\ \hline
\end{tabular}
{\scriptsize
Verify that there exist entries in the forced-photometry table for all
coadd objects for the PVIs on which the object should appear.

}
\begin{tabular}{p{3cm}p{13cm}}
\hline
            & Expected Result \\ \hline
\end{tabular}

\begin{tabular}{p{4cm}p{12cm}}
\toprule
Step 4
& Description \\ \hline
\end{tabular}
{\scriptsize
Verify that there exist entries in a forced-photometry table for each
image for all DIAObjects.\\[2\baselineskip]

}
\begin{tabular}{p{3cm}p{13cm}}
\hline
            & Expected Result \\ \hline
\end{tabular}

\subsubsection{LVV-T67 - Verify implementation of Object Catalog}\label{lvv-t67}

\begin{longtable}[]{llllll}
\toprule
Version & Status & Priority & Verification Type & Owner
\\\midrule
1 & Draft & Normal &
Test & Jim Bosch
\\\bottomrule
\multicolumn{6}{c}{ Open \href{https://jira.lsstcorp.org/secure/Tests.jspa\#/testCase/LVV-T67}{LVV-T67} in Jira } \\
\end{longtable}

\paragraph{Verification Elements}\mbox{}\\

\begin{itemize}
\item \href{https://jira.lsstcorp.org/browse/LVV-106}{LVV-106} - DMS-REQ-0275-V-01: Object Catalog

\end{itemize}

\paragraph{Test Items}\mbox{}\\

Verify that the DRP pipelines produce an Object catalog derived from
detections made on both coadded images and difference images and
measurements performed on coadds and possibly overlapping single-epoch
images.








\paragraph{Test Procedure}\mbox{}\\
\begin{tabular}{p{4cm}p{12cm}}
\toprule
Step 1
& Description \\ \hline
\end{tabular}
{\scriptsize
load LSST DM Stack

}
\begin{tabular}{p{3cm}p{13cm}}
\hline
            & Expected Result \\ \hline
\end{tabular}

\begin{tabular}{p{4cm}p{12cm}}
\toprule
Step 2
& Description \\ \hline
\end{tabular}
{\scriptsize
Run the single-frame processing and self-calibration steps of the DRP
pipeline.~

}
\begin{tabular}{p{3cm}p{13cm}}
\hline
            & Expected Result \\ \hline
\end{tabular}

\begin{tabular}{p{4cm}p{12cm}}
\toprule
Step 3
& Description \\ \hline
\end{tabular}
{\scriptsize
Insert simulated sources into all single-frame images, including:

\begin{itemize}
\tightlist
\item
  static objects (e.g. galaxies), including some too faint to be
  detectable in single-epoch images;
\item
  objects with static positions that are sufficiently bright and
  variable that they should be detectable in single-epoch difference
  images;
\item
  transient objects that appear in only a few epochs;
\item
  stars with significant proper motions and parallaxes, some below the
  single-epoch detection limit
\item
  simulated solar system objects with orbits that can be constrained
  from just the epochs in the test dataset
\end{itemize}

}
\begin{tabular}{p{3cm}p{13cm}}
\hline
            & Expected Result \\ \hline
\end{tabular}

\begin{tabular}{p{4cm}p{12cm}}
\toprule
Step 4
& Description \\ \hline
\end{tabular}
{\scriptsize
Run all remaining DRP pipeline steps.

}
\begin{tabular}{p{3cm}p{13cm}}
\hline
            & Expected Result \\ \hline
\end{tabular}

\begin{tabular}{p{4cm}p{12cm}}
\toprule
Step 5
& Description \\ \hline
\end{tabular}
{\scriptsize
Load data into DRP database

}
\begin{tabular}{p{3cm}p{13cm}}
\hline
            & Expected Result \\ \hline
\end{tabular}

\begin{tabular}{p{4cm}p{12cm}}
\toprule
Step 6
& Description \\ \hline
\end{tabular}
{\scriptsize
Verify that the injected simulated objects are recovered at a rate
consistent with their S/N \emph{when not blended with each other or real
objects}, and that flags indicating how each Object was detected are
consistent with their properties:

\begin{itemize}
\tightlist
\item
  static objects should be detected in coadds only (not difference
  images)
\item
  static-position/variable-flux objects should be detected in coadds and
  possibly difference images
\item
  transient objects should be detected in difference images only
\item
  stars with significant proper motions may be detected in either coadds
  or difference images
\item
  solar system objects should be detected in difference images only.
\end{itemize}

}
\begin{tabular}{p{3cm}p{13cm}}
\hline
            & Expected Result \\ \hline
\end{tabular}

\subsubsection{LVV-T68 - Verify implementation of Provide Photometric Redshifts of Galaxies}\label{lvv-t68}

\begin{longtable}[]{llllll}
\toprule
Version & Status & Priority & Verification Type & Owner
\\\midrule
1 & Draft & Normal &
Test & Jim Bosch
\\\bottomrule
\multicolumn{6}{c}{ Open \href{https://jira.lsstcorp.org/secure/Tests.jspa\#/testCase/LVV-T68}{LVV-T68} in Jira } \\
\end{longtable}

\paragraph{Verification Elements}\mbox{}\\

\begin{itemize}
\item \href{https://jira.lsstcorp.org/browse/LVV-19}{LVV-19} - DMS-REQ-0046-V-01: Provide Photometric Redshifts of Galaxies

\end{itemize}

\paragraph{Test Items}\mbox{}\\

Verify that Object catalogs produced by the DRP Pipeline include
photometric redshift information.








\paragraph{Test Procedure}\mbox{}\\
\begin{tabular}{p{4cm}p{12cm}}
\toprule
Step 1
& Description \\ \hline
\end{tabular}
{\scriptsize
Run DRP processing steps through (at least) final galaxy photometry
measurements.

}
\begin{tabular}{p{3cm}p{13cm}}
\hline
            & Expected Result \\ \hline
\end{tabular}

\begin{tabular}{p{4cm}p{12cm}}
\toprule
Step 2
& Description \\ \hline
\end{tabular}
{\scriptsize
Train photometric redshift algorithm(s) on spectroscopic and
high-accuracy photometric redshift catalogs.

}
\begin{tabular}{p{3cm}p{13cm}}
\hline
            & Expected Result \\ \hline
\end{tabular}

\begin{tabular}{p{4cm}p{12cm}}
\toprule
Step 3
& Description \\ \hline
\end{tabular}
{\scriptsize
Estimate photometric redshifts for all Objects generated by DRP
processing.

}
\begin{tabular}{p{3cm}p{13cm}}
\hline
            & Expected Result \\ \hline
\end{tabular}

\begin{tabular}{p{4cm}p{12cm}}
\toprule
Step 4
& Description \\ \hline
\end{tabular}
{\scriptsize
Load into DRP Database

}
\begin{tabular}{p{3cm}p{13cm}}
\hline
            & Expected Result \\ \hline
\end{tabular}

\begin{tabular}{p{4cm}p{12cm}}
\toprule
Step 5
& Description \\ \hline
\end{tabular}
{\scriptsize
Inspect database to verify that photometric redshifts are present for
all objects

}
\begin{tabular}{p{3cm}p{13cm}}
\hline
            & Expected Result \\ \hline
\end{tabular}

\subsubsection{LVV-T69 - Verify implementation of Object Characterization}\label{lvv-t69}

\begin{longtable}[]{llllll}
\toprule
Version & Status & Priority & Verification Type & Owner
\\\midrule
1 & Draft & Normal &
Test & Jim Bosch
\\\bottomrule
\multicolumn{6}{c}{ Open \href{https://jira.lsstcorp.org/secure/Tests.jspa\#/testCase/LVV-T69}{LVV-T69} in Jira } \\
\end{longtable}

\paragraph{Verification Elements}\mbox{}\\

\begin{itemize}
\item \href{https://jira.lsstcorp.org/browse/LVV-107}{LVV-107} - DMS-REQ-0276-V-01: Object Characterization

\end{itemize}

\paragraph{Test Items}\mbox{}\\

Verify that Object catalogs produced by the DRP pipeline include all
measurements listed in DMS-REQ-0276: a point-source model fit, a
bulge-disk model fit, standard colors, a centroid, adap- tive moments,
Petrosian and Kron fluxes, surface brightness at multiple apertures,
proper motion and parallax, and a variability characterization.








\paragraph{Test Procedure}\mbox{}\\
\begin{tabular}{p{4cm}p{12cm}}
\toprule
Step 1
& Description \\ \hline
\end{tabular}
{\scriptsize
Precursor data, execute DRP, load results, observe catalog contents

}
\begin{tabular}{p{3cm}p{13cm}}
\hline
            & Expected Result \\ \hline
\end{tabular}

\subsubsection{LVV-T71 - Verify implementation of Detecting extended low surface brightness
objects}\label{lvv-t71}

\begin{longtable}[]{llllll}
\toprule
Version & Status & Priority & Verification Type & Owner
\\\midrule
1 & Draft & Normal &
Test & Jim Bosch
\\\bottomrule
\multicolumn{6}{c}{ Open \href{https://jira.lsstcorp.org/secure/Tests.jspa\#/testCase/LVV-T71}{LVV-T71} in Jira } \\
\end{longtable}

\paragraph{Verification Elements}\mbox{}\\

\begin{itemize}
\item \href{https://jira.lsstcorp.org/browse/LVV-180}{LVV-180} - DMS-REQ-0349-V-01: Detecting extended low surface brightness objects

\end{itemize}

\paragraph{Test Items}\mbox{}\\

Verify that low-surface brightness objects (including those whose PSF
S/N is lower than the detection threshold) are detected in coadds.








\paragraph{Test Procedure}\mbox{}\\
\begin{tabular}{p{4cm}p{12cm}}
\toprule
Step 1
& Description \\ \hline
\end{tabular}
{\scriptsize
load LSST DM Stack

}
\begin{tabular}{p{3cm}p{13cm}}
\hline
            & Expected Result \\ \hline
\end{tabular}

\begin{tabular}{p{4cm}p{12cm}}
\toprule
Step 2
& Description \\ \hline
\end{tabular}
{\scriptsize
Run the single-frame processing and self-calibration steps of the DRP
pipeline.

}
\begin{tabular}{p{3cm}p{13cm}}
\hline
            & Expected Result \\ \hline
\end{tabular}

\begin{tabular}{p{4cm}p{12cm}}
\toprule
Step 3
& Description \\ \hline
\end{tabular}
{\scriptsize
Insert simulated low-surface-brightness galaxies (with exponential
profiles) consistently into all calibrated single-epoch images.

}
\begin{tabular}{p{3cm}p{13cm}}
\hline
            & Expected Result \\ \hline
\end{tabular}

\begin{tabular}{p{4cm}p{12cm}}
\toprule
Step 4
& Description \\ \hline
\end{tabular}
{\scriptsize
Run all remaining DRP pipeline steps.

}
\begin{tabular}{p{3cm}p{13cm}}
\hline
            & Expected Result \\ \hline
\end{tabular}

\begin{tabular}{p{4cm}p{12cm}}
\toprule
Step 5
& Description \\ \hline
\end{tabular}
{\scriptsize
​​​​Load data into DRP database

}
\begin{tabular}{p{3cm}p{13cm}}
\hline
            & Expected Result \\ \hline
\end{tabular}

\begin{tabular}{p{4cm}p{12cm}}
\toprule
Step 6
& Description \\ \hline
\end{tabular}
{\scriptsize
Verify that the injected simulated objects are recovered at a rate
consistent with their S/N and true profile \emph{when not blended with
each other or real objects.}

}
\begin{tabular}{p{3cm}p{13cm}}
\hline
            & Expected Result \\ \hline
\end{tabular}

\subsubsection{LVV-T72 - Verify implementation of Coadd Image Method Constraints}\label{lvv-t72}

\begin{longtable}[]{llllll}
\toprule
Version & Status & Priority & Verification Type & Owner
\\\midrule
1 & Draft & Normal &
Test & Jim Bosch
\\\bottomrule
\multicolumn{6}{c}{ Open \href{https://jira.lsstcorp.org/secure/Tests.jspa\#/testCase/LVV-T72}{LVV-T72} in Jira } \\
\end{longtable}

\paragraph{Verification Elements}\mbox{}\\

\begin{itemize}
\item \href{https://jira.lsstcorp.org/browse/LVV-109}{LVV-109} - DMS-REQ-0278-V-01: Coadd Image Method Constraints

\end{itemize}

\paragraph{Test Items}\mbox{}\\

Verify the implementation of how Coadd images are created.








\paragraph{Test Procedure}\mbox{}\\
\begin{tabular}{p{4cm}p{12cm}}
\toprule
Step 1
& Description \\ \hline
\end{tabular}
{\scriptsize
Identify a dataset that has been processed to create coadd images.

}
\begin{tabular}{p{3cm}p{13cm}}
\hline
            & Expected Result \\ \hline
\end{tabular}

\begin{tabular}{p{4cm}p{12cm}}
\toprule
Step 2-1
{\scriptsize from \hyperref[lvv-t987]{LVV-T987} }
& Description \\ \hline
\end{tabular}
{\scriptsize
Identify the path to the data repository, which we will refer to as
`DATA/path', then execute the following:

}
\begin{tabular}{p{3cm}p{13cm}}
\hline
            & Example Code \\ \hline
\end{tabular}
{\scriptsize
\begin{verbatim}
import lsst.daf.persistence as dafPersist
butler = dafPersist.Butler(inputs='DATA/path')
\end{verbatim}

}
\begin{tabular}{p{3cm}p{13cm}}
\hline
            & Expected Result \\ \hline
\end{tabular}
{\scriptsize
Butler repo available for reading.

}

\begin{tabular}{p{4cm}p{12cm}}
\toprule
Step 3
& Description \\ \hline
\end{tabular}
{\scriptsize
Retrieve the coadds in the dataset and verify that they are non-empty.

}
\begin{tabular}{p{3cm}p{13cm}}
\hline
            & Expected Result \\ \hline
\end{tabular}

\begin{tabular}{p{4cm}p{12cm}}
\toprule
Step 4
& Description \\ \hline
\end{tabular}
{\scriptsize
Verify that coadds were created following specification

}
\begin{tabular}{p{3cm}p{13cm}}
\hline
            & Expected Result \\ \hline
\end{tabular}

\subsubsection{LVV-T73 - Verify implementation of Deep Detection Coadds}\label{lvv-t73}

\begin{longtable}[]{llllll}
\toprule
Version & Status & Priority & Verification Type & Owner
\\\midrule
1 & Draft & Normal &
Test & Jim Bosch
\\\bottomrule
\multicolumn{6}{c}{ Open \href{https://jira.lsstcorp.org/secure/Tests.jspa\#/testCase/LVV-T73}{LVV-T73} in Jira } \\
\end{longtable}

\paragraph{Verification Elements}\mbox{}\\

\begin{itemize}
\item \href{https://jira.lsstcorp.org/browse/LVV-110}{LVV-110} - DMS-REQ-0279-V-01: Deep Detection Coadds

\end{itemize}

\paragraph{Test Items}\mbox{}\\

Verify that the DRP pipelines produce a suite of per-band coadded images
that are optimized for depth.








\paragraph{Test Procedure}\mbox{}\\
\begin{tabular}{p{4cm}p{12cm}}
\toprule
Step 1-1
{\scriptsize from \hyperref[lvv-t987]{LVV-T987} }
& Description \\ \hline
\end{tabular}
{\scriptsize
Identify the path to the data repository, which we will refer to as
`DATA/path', then execute the following:

}
\begin{tabular}{p{3cm}p{13cm}}
\hline
            & Example Code \\ \hline
\end{tabular}
{\scriptsize
\begin{verbatim}
import lsst.daf.persistence as dafPersist
butler = dafPersist.Butler(inputs='DATA/path')
\end{verbatim}

}
\begin{tabular}{p{3cm}p{13cm}}
\hline
            & Expected Result \\ \hline
\end{tabular}
{\scriptsize
Butler repo available for reading.

}

\begin{tabular}{p{4cm}p{12cm}}
\toprule
Step 2
& Description \\ \hline
\end{tabular}
{\scriptsize
Verify through inspection that per-filter coadds exist for each
tract+patch possible

}
\begin{tabular}{p{3cm}p{13cm}}
\hline
            & Expected Result \\ \hline
\end{tabular}

\begin{tabular}{p{4cm}p{12cm}}
\toprule
Step 3
& Description \\ \hline
\end{tabular}
{\scriptsize
Verify through inspection that the images used to generate those coadds
met specified conditions

}
\begin{tabular}{p{3cm}p{13cm}}
\hline
            & Expected Result \\ \hline
\end{tabular}

\begin{tabular}{p{4cm}p{12cm}}
\toprule
Step 4
& Description \\ \hline
\end{tabular}
{\scriptsize
Visually inspect a subset of the coadds to verify that they visually
appear reasonable and to be from good quality data.

}
\begin{tabular}{p{3cm}p{13cm}}
\hline
            & Expected Result \\ \hline
\end{tabular}

\subsubsection{LVV-T74 - Verify implementation of Template Coadds}\label{lvv-t74}

\begin{longtable}[]{llllll}
\toprule
Version & Status & Priority & Verification Type & Owner
\\\midrule
1 & Draft & Normal &
Test & Eric Bellm
\\\bottomrule
\multicolumn{6}{c}{ Open \href{https://jira.lsstcorp.org/secure/Tests.jspa\#/testCase/LVV-T74}{LVV-T74} in Jira } \\
\end{longtable}

\paragraph{Verification Elements}\mbox{}\\

\begin{itemize}
\item \href{https://jira.lsstcorp.org/browse/LVV-111}{LVV-111} - DMS-REQ-0280-V-01: Template Coadds

\end{itemize}

\paragraph{Test Items}\mbox{}\\

Verify that the DMS can produce Template Coadds for DIA processing.








\paragraph{Test Procedure}\mbox{}\\
\begin{tabular}{p{4cm}p{12cm}}
\toprule
Step 1-1
{\scriptsize from \hyperref[lvv-t866]{LVV-T866} }
& Description \\ \hline
\end{tabular}
{\scriptsize
Perform the steps of Alert Production (including, but not necessarily
limited to, single frame processing, ISR, source detection/measurement,
PSF estimation, photometric and astrometric calibration, difference
imaging, DIASource detection/measurement, source association). During
Operations, it is presumed that these are automated for a given
dataset.~

}
\begin{tabular}{p{3cm}p{13cm}}
\hline
            & Expected Result \\ \hline
\end{tabular}
{\scriptsize
An output dataset including difference images and DIASource and
DIAObject measurements.

}

\begin{tabular}{p{4cm}p{12cm}}
\toprule
Step 1-2
{\scriptsize from \hyperref[lvv-t866]{LVV-T866} }
& Description \\ \hline
\end{tabular}
{\scriptsize
Verify that the expected data products have been produced, and that
catalogs contain reasonable values for measured quantities of interest.

}
\begin{tabular}{p{3cm}p{13cm}}
\hline
            & Expected Result \\ \hline
\end{tabular}

\begin{tabular}{p{4cm}p{12cm}}
\toprule
Step 2
& Description \\ \hline
\end{tabular}
{\scriptsize
Confirm that the template coadds have been created and are well-formed.

}
\begin{tabular}{p{3cm}p{13cm}}
\hline
            & Expected Result \\ \hline
\end{tabular}

\subsubsection{LVV-T75 - Verify implementation of Multi-band Coadds}\label{lvv-t75}

\begin{longtable}[]{llllll}
\toprule
Version & Status & Priority & Verification Type & Owner
\\\midrule
1 & Draft & Normal &
Test & Jim Bosch
\\\bottomrule
\multicolumn{6}{c}{ Open \href{https://jira.lsstcorp.org/secure/Tests.jspa\#/testCase/LVV-T75}{LVV-T75} in Jira } \\
\end{longtable}

\paragraph{Verification Elements}\mbox{}\\

\begin{itemize}
\item \href{https://jira.lsstcorp.org/browse/LVV-112}{LVV-112} - DMS-REQ-0281-V-01: Multi-band Coadds

\end{itemize}

\paragraph{Test Items}\mbox{}\\

Verify that the DRP pipelines produce multi-band coadds for detection
purposes.








\paragraph{Test Procedure}\mbox{}\\
\begin{tabular}{p{4cm}p{12cm}}
\toprule
Step 1-1
{\scriptsize from \hyperref[lvv-t987]{LVV-T987} }
& Description \\ \hline
\end{tabular}
{\scriptsize
Identify the path to the data repository, which we will refer to as
`DATA/path', then execute the following:

}
\begin{tabular}{p{3cm}p{13cm}}
\hline
            & Example Code \\ \hline
\end{tabular}
{\scriptsize
\begin{verbatim}
import lsst.daf.persistence as dafPersist
butler = dafPersist.Butler(inputs='DATA/path')
\end{verbatim}

}
\begin{tabular}{p{3cm}p{13cm}}
\hline
            & Expected Result \\ \hline
\end{tabular}
{\scriptsize
Butler repo available for reading.

}

\begin{tabular}{p{4cm}p{12cm}}
\toprule
Step 2
& Description \\ \hline
\end{tabular}
{\scriptsize
Verify that deep detection coadds exist based on all
filters.\\[2\baselineskip]

}
\begin{tabular}{p{3cm}p{13cm}}
\hline
            & Expected Result \\ \hline
\end{tabular}

\subsubsection{LVV-T76 - Verify implementation of All-Sky Visualization of Data Releases}\label{lvv-t76}

\begin{longtable}[]{llllll}
\toprule
Version & Status & Priority & Verification Type & Owner
\\\midrule
1 & Draft & Normal &
Test & Simon Krughoff
\\\bottomrule
\multicolumn{6}{c}{ Open \href{https://jira.lsstcorp.org/secure/Tests.jspa\#/testCase/LVV-T76}{LVV-T76} in Jira } \\
\end{longtable}

\paragraph{Verification Elements}\mbox{}\\

\begin{itemize}
\item \href{https://jira.lsstcorp.org/browse/LVV-160}{LVV-160} - DMS-REQ-0329-V-01: All-Sky Visualization of Data Releases

\end{itemize}

\paragraph{Test Items}\mbox{}\\

Show that it's possible to produce large area visualizations from Data
Release data products.








\paragraph{Test Procedure}\mbox{}\\
\begin{tabular}{p{4cm}p{12cm}}
\toprule
Step 1-1
{\scriptsize from \hyperref[lvv-t987]{LVV-T987} }
& Description \\ \hline
\end{tabular}
{\scriptsize
Identify the path to the data repository, which we will refer to as
`DATA/path', then execute the following:

}
\begin{tabular}{p{3cm}p{13cm}}
\hline
            & Example Code \\ \hline
\end{tabular}
{\scriptsize
\begin{verbatim}
import lsst.daf.persistence as dafPersist
butler = dafPersist.Butler(inputs='DATA/path')
\end{verbatim}

}
\begin{tabular}{p{3cm}p{13cm}}
\hline
            & Expected Result \\ \hline
\end{tabular}
{\scriptsize
Butler repo available for reading.

}

\begin{tabular}{p{4cm}p{12cm}}
\toprule
Step 2
& Description \\ \hline
\end{tabular}
{\scriptsize
Run all sky tile generation task to produce the data products necessary
for serving the all sky visualization.

}
\begin{tabular}{p{3cm}p{13cm}}
\hline
            & Expected Result \\ \hline
\end{tabular}

\begin{tabular}{p{4cm}p{12cm}}
\toprule
Step 3
& Description \\ \hline
\end{tabular}
{\scriptsize
Manually perform, and log (including timing where applicable), the
following steps against that all sky visualization application. ~At all
steps take special care to note any missing or un-rendered image
tiles:\\[2\baselineskip]1. Navigate to the all sky viewer and log the
URL, browser and version.\\
2. Zoom to native pixel display (1 image pixel per display pixel)\\
3. Zoom to fit the full PDR footprint\\
4. Zoom to 1/4x native resolution\\
5. Pan to eastern edge of the footprint.\\
6. Pan to western edge of the footprint.\\
7. Navigate to the middle of the footprint.\\
8. Zoom to max magnification

}
\begin{tabular}{p{3cm}p{13cm}}
\hline
            & Expected Result \\ \hline
\end{tabular}

\subsubsection{LVV-T77 - Verify implementation of Best Seeing Coadds}\label{lvv-t77}

\begin{longtable}[]{llllll}
\toprule
Version & Status & Priority & Verification Type & Owner
\\\midrule
1 & Draft & Normal &
Test & Jim Bosch
\\\bottomrule
\multicolumn{6}{c}{ Open \href{https://jira.lsstcorp.org/secure/Tests.jspa\#/testCase/LVV-T77}{LVV-T77} in Jira } \\
\end{longtable}

\paragraph{Verification Elements}\mbox{}\\

\begin{itemize}
\item \href{https://jira.lsstcorp.org/browse/LVV-161}{LVV-161} - DMS-REQ-0330-V-01: Best Seeing Coadds

\end{itemize}

\paragraph{Test Items}\mbox{}\\

Verify that the DRP pipelines produce a suite of per-band coadds with
input images filtered to optimize the size of the effective PSF on the
coadd.








\paragraph{Test Procedure}\mbox{}\\
\begin{tabular}{p{4cm}p{12cm}}
\toprule
Step 1-1
{\scriptsize from \hyperref[lvv-t860]{LVV-T860} }
& Description \\ \hline
\end{tabular}
{\scriptsize
The `path` that you will use depends on where you are running the
science pipelines. Options:\\[2\baselineskip]

\begin{itemize}
\tightlist
\item
  local (newinstall.sh - based
  install):{[}path\_to\_installation{]}/loadLSST.bash
\item
  development cluster (``lsst-dev''):
  /software/lsstsw/stack/loadLSST.bash
\item
  LSP Notebook aspect (from a terminal):
  /opt/lsst/software/stack/loadLSST.bash
\end{itemize}

From the command line, execute the commands below in the example
code:\\[2\baselineskip]

}
\begin{tabular}{p{3cm}p{13cm}}
\hline
            & Example Code \\ \hline
\end{tabular}
{\scriptsize
source `path`\\
setup lsst\_distrib

}
\begin{tabular}{p{3cm}p{13cm}}
\hline
            & Expected Result \\ \hline
\end{tabular}
{\scriptsize
Science pipeline software is available for use. If additional packages
are needed (for example, `obs' packages such as `obs\_subaru`), then
additional `setup` commands will be necessary.\\[2\baselineskip]To check
versions in use, type:\\
eups list -s

}

\begin{tabular}{p{4cm}p{12cm}}
\toprule
Step 2-1
{\scriptsize from \hyperref[lvv-t987]{LVV-T987} }
& Description \\ \hline
\end{tabular}
{\scriptsize
Identify the path to the data repository, which we will refer to as
`DATA/path', then execute the following:

}
\begin{tabular}{p{3cm}p{13cm}}
\hline
            & Example Code \\ \hline
\end{tabular}
{\scriptsize
\begin{verbatim}
import lsst.daf.persistence as dafPersist
butler = dafPersist.Butler(inputs='DATA/path')
\end{verbatim}

}
\begin{tabular}{p{3cm}p{13cm}}
\hline
            & Expected Result \\ \hline
\end{tabular}
{\scriptsize
Butler repo available for reading.

}

\begin{tabular}{p{4cm}p{12cm}}
\toprule
Step 3
& Description \\ \hline
\end{tabular}
{\scriptsize
Explicitly create a coadd for a specified seeing range in each filter.

}
\begin{tabular}{p{3cm}p{13cm}}
\hline
            & Expected Result \\ \hline
\end{tabular}

\begin{tabular}{p{4cm}p{12cm}}
\toprule
Step 4
& Description \\ \hline
\end{tabular}
{\scriptsize
Verify that these coadds exist.

}
\begin{tabular}{p{3cm}p{13cm}}
\hline
            & Expected Result \\ \hline
\end{tabular}

\subsubsection{LVV-T78 - Verify implementation of Persisting Data Products}\label{lvv-t78}

\begin{longtable}[]{llllll}
\toprule
Version & Status & Priority & Verification Type & Owner
\\\midrule
1 & Draft & Normal &
Test & Kian-Tat Lim
\\\bottomrule
\multicolumn{6}{c}{ Open \href{https://jira.lsstcorp.org/secure/Tests.jspa\#/testCase/LVV-T78}{LVV-T78} in Jira } \\
\end{longtable}

\paragraph{Verification Elements}\mbox{}\\

\begin{itemize}
\item \href{https://jira.lsstcorp.org/browse/LVV-165}{LVV-165} - DMS-REQ-0334-V-01: Persisting Data Products

\end{itemize}

\paragraph{Test Items}\mbox{}\\

Verify that per-band deep coadds and best-seeing coadds are present,
kept, and available.








\paragraph{Test Procedure}\mbox{}\\
\begin{tabular}{p{4cm}p{12cm}}
\toprule
Step 1
& Description \\ \hline
\end{tabular}
{\scriptsize
\hypertarget{description-val}{}
Produce some relevant coadds and store them in the Archive

}
\begin{tabular}{p{3cm}p{13cm}}
\hline
            & Expected Result \\ \hline
\end{tabular}

\begin{tabular}{p{4cm}p{12cm}}
\toprule
Step 2
& Description \\ \hline
\end{tabular}
{\scriptsize
Examine the data retention policies for those products

}
\begin{tabular}{p{3cm}p{13cm}}
\hline
            & Expected Result \\ \hline
\end{tabular}

\subsubsection{LVV-T79 - Verify implementation of PSF-Matched Coadds}\label{lvv-t79}

\begin{longtable}[]{llllll}
\toprule
Version & Status & Priority & Verification Type & Owner
\\\midrule
1 & Draft & Normal &
Test & Jim Bosch
\\\bottomrule
\multicolumn{6}{c}{ Open \href{https://jira.lsstcorp.org/secure/Tests.jspa\#/testCase/LVV-T79}{LVV-T79} in Jira } \\
\end{longtable}

\paragraph{Verification Elements}\mbox{}\\

\begin{itemize}
\item \href{https://jira.lsstcorp.org/browse/LVV-166}{LVV-166} - DMS-REQ-0335-V-01: PSF-Matched Coadds

\end{itemize}

\paragraph{Test Items}\mbox{}\\

Verify that the DRP pipelines produce PSF matched coadds.








\paragraph{Test Procedure}\mbox{}\\
\begin{tabular}{p{4cm}p{12cm}}
\toprule
Step 1-1
{\scriptsize from \hyperref[lvv-t987]{LVV-T987} }
& Description \\ \hline
\end{tabular}
{\scriptsize
Identify the path to the data repository, which we will refer to as
`DATA/path', then execute the following:

}
\begin{tabular}{p{3cm}p{13cm}}
\hline
            & Example Code \\ \hline
\end{tabular}
{\scriptsize
\begin{verbatim}
import lsst.daf.persistence as dafPersist
butler = dafPersist.Butler(inputs='DATA/path')
\end{verbatim}

}
\begin{tabular}{p{3cm}p{13cm}}
\hline
            & Expected Result \\ \hline
\end{tabular}
{\scriptsize
Butler repo available for reading.

}

\begin{tabular}{p{4cm}p{12cm}}
\toprule
Step 2
& Description \\ \hline
\end{tabular}
{\scriptsize
Verify that PSF-matched coadds were created.

}
\begin{tabular}{p{3cm}p{13cm}}
\hline
            & Expected Result \\ \hline
\end{tabular}

\subsubsection{LVV-T80 - Verify implementation of Detecting faint variable objects}\label{lvv-t80}

\begin{longtable}[]{llllll}
\toprule
Version & Status & Priority & Verification Type & Owner
\\\midrule
1 & Draft & Normal &
Test & Melissa Graham
\\\bottomrule
\multicolumn{6}{c}{ Open \href{https://jira.lsstcorp.org/secure/Tests.jspa\#/testCase/LVV-T80}{LVV-T80} in Jira } \\
\end{longtable}

\paragraph{Verification Elements}\mbox{}\\

\begin{itemize}
\item \href{https://jira.lsstcorp.org/browse/LVV-168}{LVV-168} - DMS-REQ-0337-V-01: Detecting faint variable objects

\end{itemize}

\paragraph{Test Items}\mbox{}\\

To verify that the Data Release Production pipeline will be able to
detect faint sources with long-term variability (e.g., quasars, proper
motion stars) via, e.g., shorter timescale coadds (month to a few
months).








\paragraph{Test Procedure}\mbox{}\\
\begin{tabular}{p{4cm}p{12cm}}
\toprule
Step 1-1
{\scriptsize from \hyperref[lvv-t866]{LVV-T866} }
& Description \\ \hline
\end{tabular}
{\scriptsize
Perform the steps of Alert Production (including, but not necessarily
limited to, single frame processing, ISR, source detection/measurement,
PSF estimation, photometric and astrometric calibration, difference
imaging, DIASource detection/measurement, source association). During
Operations, it is presumed that these are automated for a given
dataset.~

}
\begin{tabular}{p{3cm}p{13cm}}
\hline
            & Expected Result \\ \hline
\end{tabular}
{\scriptsize
An output dataset including difference images and DIASource and
DIAObject measurements.

}

\begin{tabular}{p{4cm}p{12cm}}
\toprule
Step 1-2
{\scriptsize from \hyperref[lvv-t866]{LVV-T866} }
& Description \\ \hline
\end{tabular}
{\scriptsize
Verify that the expected data products have been produced, and that
catalogs contain reasonable values for measured quantities of interest.

}
\begin{tabular}{p{3cm}p{13cm}}
\hline
            & Expected Result \\ \hline
\end{tabular}

\begin{tabular}{p{4cm}p{12cm}}
\toprule
Step 2-1
{\scriptsize from \hyperref[lvv-t987]{LVV-T987} }
& Description \\ \hline
\end{tabular}
{\scriptsize
Identify the path to the data repository, which we will refer to as
`DATA/path', then execute the following:

}
\begin{tabular}{p{3cm}p{13cm}}
\hline
            & Example Code \\ \hline
\end{tabular}
{\scriptsize
\begin{verbatim}
import lsst.daf.persistence as dafPersist
butler = dafPersist.Butler(inputs='DATA/path')
\end{verbatim}

}
\begin{tabular}{p{3cm}p{13cm}}
\hline
            & Expected Result \\ \hline
\end{tabular}
{\scriptsize
Butler repo available for reading.

}

\begin{tabular}{p{4cm}p{12cm}}
\toprule
Step 3
& Description \\ \hline
\end{tabular}
{\scriptsize
Identify 100 objects from Gaia with proper motions high enough to have
detectably moved during HSC observations.

}
\begin{tabular}{p{3cm}p{13cm}}
\hline
            & Expected Result \\ \hline
\end{tabular}

\begin{tabular}{p{4cm}p{12cm}}
\toprule
Step 4
& Description \\ \hline
\end{tabular}
{\scriptsize
Measure reported proper motion of these objects in DM Stack processing.
~Verify that it is consistent with Gaia objects.

}
\begin{tabular}{p{3cm}p{13cm}}
\hline
            & Expected Result \\ \hline
\end{tabular}

\begin{tabular}{p{4cm}p{12cm}}
\toprule
Step 5
& Description \\ \hline
\end{tabular}
{\scriptsize
Identify 100 quasars from color-space or existing extragalactic
spectroscopic catalog.

}
\begin{tabular}{p{3cm}p{13cm}}
\hline
            & Expected Result \\ \hline
\end{tabular}

\begin{tabular}{p{4cm}p{12cm}}
\toprule
Step 6
& Description \\ \hline
\end{tabular}
{\scriptsize
Measure lightcurves of these quasars. ~Determine if structure function
is reasonable (may require at least a year to determine if the structure
function of 100 quasars is ``reasonable'').

}
\begin{tabular}{p{3cm}p{13cm}}
\hline
            & Expected Result \\ \hline
\end{tabular}

\begin{tabular}{p{4cm}p{12cm}}
\toprule
Step 7
& Description \\ \hline
\end{tabular}
{\scriptsize
(Alternative: if faint variable source can be injected into the input
data, test to see if they are recovered).

}
\begin{tabular}{p{3cm}p{13cm}}
\hline
            & Expected Result \\ \hline
\end{tabular}
{\scriptsize
(This Alternative would enable us not only to tell if faint variable
objects are detected, but exactly which kinds, how faint, and with what
efficiency.)

}

\subsubsection{LVV-T81 - Verify implementation of Targeted Coadds}\label{lvv-t81}

\begin{longtable}[]{llllll}
\toprule
Version & Status & Priority & Verification Type & Owner
\\\midrule
1 & Draft & Normal &
Test & Jim Bosch
\\\bottomrule
\multicolumn{6}{c}{ Open \href{https://jira.lsstcorp.org/secure/Tests.jspa\#/testCase/LVV-T81}{LVV-T81} in Jira } \\
\end{longtable}

\paragraph{Verification Elements}\mbox{}\\

\begin{itemize}
\item \href{https://jira.lsstcorp.org/browse/LVV-169}{LVV-169} - DMS-REQ-0338-V-01: Targeted Coadds

\end{itemize}

\paragraph{Test Items}\mbox{}\\

Verify that small sections of any coadd produced by the DRP pipelines
can be retained, even if the full coadd is not.








\paragraph{Test Procedure}\mbox{}\\
\begin{tabular}{p{4cm}p{12cm}}
\toprule
Step 1
& Description \\ \hline
\end{tabular}
{\scriptsize
Remove DR from disk

}
\begin{tabular}{p{3cm}p{13cm}}
\hline
            & Expected Result \\ \hline
\end{tabular}

\begin{tabular}{p{4cm}p{12cm}}
\toprule
Step 2
& Description \\ \hline
\end{tabular}
{\scriptsize
Observe retention of designated coadd sections

}
\begin{tabular}{p{3cm}p{13cm}}
\hline
            & Expected Result \\ \hline
\end{tabular}

\begin{tabular}{p{4cm}p{12cm}}
\toprule
Step 3
& Description \\ \hline
\end{tabular}
{\scriptsize
Observe accessibility of designated coadd sections via simulated DAC LSP
instance

}
\begin{tabular}{p{3cm}p{13cm}}
\hline
            & Expected Result \\ \hline
\end{tabular}

\subsubsection{LVV-T86 - Verify implementation of Illumination Correction Frame}\label{lvv-t86}

\begin{longtable}[]{llllll}
\toprule
Version & Status & Priority & Verification Type & Owner
\\\midrule
1 & Draft & Normal &
Test & Robert Lupton
\\\bottomrule
\multicolumn{6}{c}{ Open \href{https://jira.lsstcorp.org/secure/Tests.jspa\#/testCase/LVV-T86}{LVV-T86} in Jira } \\
\end{longtable}

\paragraph{Verification Elements}\mbox{}\\

\begin{itemize}
\item \href{https://jira.lsstcorp.org/browse/LVV-25}{LVV-25} - DMS-REQ-0062-V-01: Illumination Correction Frame

\end{itemize}

\paragraph{Test Items}\mbox{}\\

Verify that the DMS can produce an illumination correction frame
calibration product.\\
Verify that the DMS can determine the effectiveness of an illumination
correction and determine how often it should be updated.








\paragraph{Test Procedure}\mbox{}\\
\begin{tabular}{p{4cm}p{12cm}}
\toprule
Step 1
& Description \\ \hline
\end{tabular}
{\scriptsize
Delegate to CPP

}
\begin{tabular}{p{3cm}p{13cm}}
\hline
            & Expected Result \\ \hline
\end{tabular}

\subsubsection{LVV-T87 - Verify implementation of Monochromatic Flatfield Data Cube}\label{lvv-t87}

\begin{longtable}[]{llllll}
\toprule
Version & Status & Priority & Verification Type & Owner
\\\midrule
1 & Draft & Normal &
Test & Robert Lupton
\\\bottomrule
\multicolumn{6}{c}{ Open \href{https://jira.lsstcorp.org/secure/Tests.jspa\#/testCase/LVV-T87}{LVV-T87} in Jira } \\
\end{longtable}

\paragraph{Verification Elements}\mbox{}\\

\begin{itemize}
\item \href{https://jira.lsstcorp.org/browse/LVV-26}{LVV-26} - DMS-REQ-0063-V-01: Monochromatic Flatfield Data Cube

\end{itemize}

\paragraph{Test Items}\mbox{}\\

Verify that the DMS can generate a calibration image/cube that corrects
for pixel-to-pixel wavelength-dependent detector response.\\
Verify that the DMS can measure the effectiveness of this monochromatic
flatfield data cube.








\paragraph{Test Procedure}\mbox{}\\
\begin{tabular}{p{4cm}p{12cm}}
\toprule
Step 1
& Description \\ \hline
\end{tabular}
{\scriptsize
Delegate to CPP

}
\begin{tabular}{p{3cm}p{13cm}}
\hline
            & Expected Result \\ \hline
\end{tabular}

\subsubsection{LVV-T91 - Verify implementation of Fringe Correction Frame}\label{lvv-t91}

\begin{longtable}[]{llllll}
\toprule
Version & Status & Priority & Verification Type & Owner
\\\midrule
1 & Draft & Normal &
Test & Robert Lupton
\\\bottomrule
\multicolumn{6}{c}{ Open \href{https://jira.lsstcorp.org/secure/Tests.jspa\#/testCase/LVV-T91}{LVV-T91} in Jira } \\
\end{longtable}

\paragraph{Verification Elements}\mbox{}\\

\begin{itemize}
\item \href{https://jira.lsstcorp.org/browse/LVV-114}{LVV-114} - DMS-REQ-0283-V-01: Fringe Correction Frame

\end{itemize}

\paragraph{Test Items}\mbox{}\\

Verify that the DMS can produce an fringe-correction frame calibration
product.\\
Verify that the DMS can determine the effectiveness of the
fringe-correction frame and determine how often it should be updated.








\paragraph{Test Procedure}\mbox{}\\
\begin{tabular}{p{4cm}p{12cm}}
\toprule
Step 1
& Description \\ \hline
\end{tabular}
{\scriptsize
Delegate to CPP

}
\begin{tabular}{p{3cm}p{13cm}}
\hline
            & Expected Result \\ \hline
\end{tabular}

\subsubsection{LVV-T92 - Verify implementation of Processing of Data From Special Programs}\label{lvv-t92}

\begin{longtable}[]{llllll}
\toprule
Version & Status & Priority & Verification Type & Owner
\\\midrule
1 & Draft & Normal &
Test & Melissa Graham
\\\bottomrule
\multicolumn{6}{c}{ Open \href{https://jira.lsstcorp.org/secure/Tests.jspa\#/testCase/LVV-T92}{LVV-T92} in Jira } \\
\end{longtable}

\paragraph{Verification Elements}\mbox{}\\

\begin{itemize}
\item \href{https://jira.lsstcorp.org/browse/LVV-151}{LVV-151} - DMS-REQ-0320-V-01: Processing of Data From Special Programs

\end{itemize}

\paragraph{Test Items}\mbox{}\\

For a simulated night of observing that includes some special program
observations, show that the SP observations are reduced using their
designated reconfigured pipelines (i.e., that the image metadata is
sufficient to trigger the processing and include all other relevant
images in the processing).








\paragraph{Test Procedure}\mbox{}\\
\begin{tabular}{p{4cm}p{12cm}}
\toprule
Step 1
& Description \\ \hline
\end{tabular}
{\scriptsize
(1) Special Programs data that can be processed by the Prompt pipeline
(i.e., standard visits).\\
Check that all images with the header keyword for SP were processed by
the Prompt pipeline. Check that the Prompt pipeline's data products --
DIASource, DIAObject catalogs and the Alerts -- contain items flagged
with their origin as that SP.

}
\begin{tabular}{p{3cm}p{13cm}}
\hline
            & Expected Result \\ \hline
\end{tabular}

\begin{tabular}{p{4cm}p{12cm}}
\toprule
Step 2
& Description \\ \hline
\end{tabular}
{\scriptsize
(2) Special Programs data that requires `real-time'
(\textasciitilde{}24) processing with a reconfigured pipeline (e.g., DDF
imaging sequence)\\
Check that all images with the header keywords for a given SP were
processed by their reconfigured pipeline. Check that the pipeline's data
products have been updated, and passed their QA.

}
\begin{tabular}{p{3cm}p{13cm}}
\hline
            & Expected Result \\ \hline
\end{tabular}

\begin{tabular}{p{4cm}p{12cm}}
\toprule
Step 3
& Description \\ \hline
\end{tabular}
{\scriptsize
(3) Special Programs data that can (should) be processed by the Data
Release pipeline (e.g., North Ecliptic Spur standard visits).\\
SP data would be added manually to the DRP processing. Check that the
DRP's data products -- Source, Object, CoAdds -- contain items flagged
as originating in that SP.

}
\begin{tabular}{p{3cm}p{13cm}}
\hline
            & Expected Result \\ \hline
\end{tabular}

\subsubsection{LVV-T93 - Verify implementation of Level 1 Processing of Special Programs Data}\label{lvv-t93}

\begin{longtable}[]{llllll}
\toprule
Version & Status & Priority & Verification Type & Owner
\\\midrule
1 & Draft & Normal &
Test & Melissa Graham
\\\bottomrule
\multicolumn{6}{c}{ Open \href{https://jira.lsstcorp.org/secure/Tests.jspa\#/testCase/LVV-T93}{LVV-T93} in Jira } \\
\end{longtable}

\paragraph{Verification Elements}\mbox{}\\

\begin{itemize}
\item \href{https://jira.lsstcorp.org/browse/LVV-152}{LVV-152} - DMS-REQ-0321-V-01: Level 1 Processing of Special Programs Data

\end{itemize}

\paragraph{Test Items}\mbox{}\\

Execute multi-day operations rehearsal. Observe whether Prompt
Processing data products generated in time and confirm whether
processing has completed before the start of the next simulated night.~








\paragraph{Test Procedure}\mbox{}\\
\begin{tabular}{p{4cm}p{12cm}}
\toprule
Step 1
& Description \\ \hline
\end{tabular}
{\scriptsize
If imaging data for a Special Program that requires processing with the
Prompt pipeline was obtained the previous night, check that there exist
DIASources/Objects/Alerts with flags that they originated from the
Special Program.

}
\begin{tabular}{p{3cm}p{13cm}}
\hline
            & Expected Result \\ \hline
\end{tabular}

\begin{tabular}{p{4cm}p{12cm}}
\toprule
Step 2
& Description \\ \hline
\end{tabular}
{\scriptsize
If imaging data for a Special Program that requires prompt processing
with a reconfigured pipeline was obtained the previous night, check that
the relevant data products have been updated.

}
\begin{tabular}{p{3cm}p{13cm}}
\hline
            & Expected Result \\ \hline
\end{tabular}

\subsubsection{LVV-T94 - Verify implementation of Special Programs Database}\label{lvv-t94}

\begin{longtable}[]{llllll}
\toprule
Version & Status & Priority & Verification Type & Owner
\\\midrule
1 & Draft & Normal &
Test & Melissa Graham
\\\bottomrule
\multicolumn{6}{c}{ Open \href{https://jira.lsstcorp.org/secure/Tests.jspa\#/testCase/LVV-T94}{LVV-T94} in Jira } \\
\end{longtable}

\paragraph{Verification Elements}\mbox{}\\

\begin{itemize}
\item \href{https://jira.lsstcorp.org/browse/LVV-153}{LVV-153} - DMS-REQ-0322-V-01: Special Programs Database

\end{itemize}

\paragraph{Test Items}\mbox{}\\

To confirm that data products from Special Programs are based solely on
images obtained as part of SP via, e.g., metadata queries. To confirm
that the SP data products can be joined to Prompt and DRP products by
attempting to do so via, e.g., coordinate table joins, and attempting to
e.g., find the faint counterparts in a Deep Drilling stack to variables
with no Object detections in the DRP coadds.








\paragraph{Test Procedure}\mbox{}\\
\begin{tabular}{p{4cm}p{12cm}}
\toprule
Step 1
& Description \\ \hline
\end{tabular}
{\scriptsize
SP data product: DDF DIAObjects catalog\\
Non-SP data product: WFD DIAObjects catalog\\
Test: join the two catalogs by coordinate (e.g., to get a longer time
baseline for variable stars in the DDF)

}
\begin{tabular}{p{3cm}p{13cm}}
\hline
            & Expected Result \\ \hline
\end{tabular}

\begin{tabular}{p{4cm}p{12cm}}
\toprule
Step 2
& Description \\ \hline
\end{tabular}
{\scriptsize
SP data product: DDF Objects catalog\\
Non-SP data product: WFD DIAObjects catalog\\
Test: join the two catalogs by coordinate to identify faint host
galaxies of transients found in WFD

}
\begin{tabular}{p{3cm}p{13cm}}
\hline
            & Expected Result \\ \hline
\end{tabular}

\subsubsection{LVV-T95 - Verify implementation of Constraints on Level 1 Special Program Products
Generation}\label{lvv-t95}

\begin{longtable}[]{llllll}
\toprule
Version & Status & Priority & Verification Type & Owner
\\\midrule
1 & Draft & Normal &
Test & Melissa Graham
\\\bottomrule
\multicolumn{6}{c}{ Open \href{https://jira.lsstcorp.org/secure/Tests.jspa\#/testCase/LVV-T95}{LVV-T95} in Jira } \\
\end{longtable}

\paragraph{Verification Elements}\mbox{}\\

\begin{itemize}
\item \href{https://jira.lsstcorp.org/browse/LVV-175}{LVV-175} - DMS-REQ-0004-V-01: Time to L1 public release

\item \href{https://jira.lsstcorp.org/browse/LVV-1276}{LVV-1276} - OSS-REQ-0127-V-01: Level 1 Data Product Availability

\end{itemize}

\paragraph{Test Items}\mbox{}\\

Execute single-day operations rehearsal. Observe Prompt Processing data
products generated in time. Confirm that data from Special Programs is
processed with the same latency as required for main survey data:
release of public data within L1publicT and Alerts within OTT1.








\paragraph{Test Procedure}\mbox{}\\
\begin{tabular}{p{4cm}p{12cm}}
\toprule
Step 1-1
{\scriptsize from \hyperref[lvv-t866]{LVV-T866} }
& Description \\ \hline
\end{tabular}
{\scriptsize
Perform the steps of Alert Production (including, but not necessarily
limited to, single frame processing, ISR, source detection/measurement,
PSF estimation, photometric and astrometric calibration, difference
imaging, DIASource detection/measurement, source association). During
Operations, it is presumed that these are automated for a given
dataset.~

}
\begin{tabular}{p{3cm}p{13cm}}
\hline
            & Expected Result \\ \hline
\end{tabular}
{\scriptsize
An output dataset including difference images and DIASource and
DIAObject measurements.

}

\begin{tabular}{p{4cm}p{12cm}}
\toprule
Step 1-2
{\scriptsize from \hyperref[lvv-t866]{LVV-T866} }
& Description \\ \hline
\end{tabular}
{\scriptsize
Verify that the expected data products have been produced, and that
catalogs contain reasonable values for measured quantities of interest.

}
\begin{tabular}{p{3cm}p{13cm}}
\hline
            & Expected Result \\ \hline
\end{tabular}

\begin{tabular}{p{4cm}p{12cm}}
\toprule
Step 2
& Description \\ \hline
\end{tabular}
{\scriptsize
Confirm that Special Program prompt data products have been generated
within 24 hours.

}
\begin{tabular}{p{3cm}p{13cm}}
\hline
            & Expected Result \\ \hline
\end{tabular}

\subsubsection{LVV-T96 - Verify implementation of Query Repeatability}\label{lvv-t96}

\begin{longtable}[]{llllll}
\toprule
Version & Status & Priority & Verification Type & Owner
\\\midrule
1 & Draft & Normal &
Test & Colin Slater
\\\bottomrule
\multicolumn{6}{c}{ Open \href{https://jira.lsstcorp.org/secure/Tests.jspa\#/testCase/LVV-T96}{LVV-T96} in Jira } \\
\end{longtable}

\paragraph{Verification Elements}\mbox{}\\

\begin{itemize}
\item \href{https://jira.lsstcorp.org/browse/LVV-122}{LVV-122} - DMS-REQ-0291-V-01: Query Repeatability

\end{itemize}

\paragraph{Test Items}\mbox{}\\

Verify that prior queries can be rerun with identical results, or with
new additional data for live (Alert Production) databases.








\paragraph{Test Procedure}\mbox{}\\
\begin{tabular}{p{4cm}p{12cm}}
\toprule
Step 1
& Description \\ \hline
\end{tabular}
{\scriptsize
Select and download (deterministic) random subsample of records from
Data Release Object and Source tables.

}
\begin{tabular}{p{3cm}p{13cm}}
\hline
            & Expected Result \\ \hline
\end{tabular}

\begin{tabular}{p{4cm}p{12cm}}
\toprule
Step 2
& Description \\ \hline
\end{tabular}
{\scriptsize
Select and download random subsample of PPDB DIAObject and DIASource
tables.

}
\begin{tabular}{p{3cm}p{13cm}}
\hline
            & Expected Result \\ \hline
\end{tabular}

\begin{tabular}{p{4cm}p{12cm}}
\toprule
Step 3
& Description \\ \hline
\end{tabular}
{\scriptsize
As appropriate, wait for some amount of non-trivial database usage to
occur, such as Prompt Processing ingestion or ingestion of other DRP
database tables.

}
\begin{tabular}{p{3cm}p{13cm}}
\hline
            & Expected Result \\ \hline
\end{tabular}

\begin{tabular}{p{4cm}p{12cm}}
\toprule
Step 4
& Description \\ \hline
\end{tabular}
{\scriptsize
Re-run the queries in steps 1 and 2 and verify that the resulting data
are identical.

}
\begin{tabular}{p{3cm}p{13cm}}
\hline
            & Expected Result \\ \hline
\end{tabular}

\subsubsection{LVV-T99 - Verify implementation of Processing of Datasets}\label{lvv-t99}

\begin{longtable}[]{llllll}
\toprule
Version & Status & Priority & Verification Type & Owner
\\\midrule
1 & Draft & Normal &
Test & Kian-Tat Lim
\\\bottomrule
\multicolumn{6}{c}{ Open \href{https://jira.lsstcorp.org/secure/Tests.jspa\#/testCase/LVV-T99}{LVV-T99} in Jira } \\
\end{longtable}

\paragraph{Verification Elements}\mbox{}\\

\begin{itemize}
\item \href{https://jira.lsstcorp.org/browse/LVV-125}{LVV-125} - DMS-REQ-0294-V-01: Processing of Datasets

\end{itemize}

\paragraph{Test Items}\mbox{}\\

Execute AP and DRP, simulate failures, observe correct processing








\paragraph{Test Procedure}\mbox{}\\
\begin{tabular}{p{4cm}p{12cm}}
\toprule
Step 1
& Description \\ \hline
\end{tabular}
{\scriptsize
Execute AP and DRP

}
\begin{tabular}{p{3cm}p{13cm}}
\hline
            & Expected Result \\ \hline
\end{tabular}

\begin{tabular}{p{4cm}p{12cm}}
\toprule
Step 2
& Description \\ \hline
\end{tabular}
{\scriptsize
~Simulate failures

}
\begin{tabular}{p{3cm}p{13cm}}
\hline
            & Expected Result \\ \hline
\end{tabular}

\begin{tabular}{p{4cm}p{12cm}}
\toprule
Step 3
& Description \\ \hline
\end{tabular}
{\scriptsize
Observe correct processing

}
\begin{tabular}{p{3cm}p{13cm}}
\hline
            & Expected Result \\ \hline
\end{tabular}

\subsubsection{LVV-T100 - Verify implementation of Transparent Data Access}\label{lvv-t100}

\begin{longtable}[]{llllll}
\toprule
Version & Status & Priority & Verification Type & Owner
\\\midrule
1 & Draft & Normal &
Test & Kian-Tat Lim
\\\bottomrule
\multicolumn{6}{c}{ Open \href{https://jira.lsstcorp.org/secure/Tests.jspa\#/testCase/LVV-T100}{LVV-T100} in Jira } \\
\end{longtable}

\paragraph{Verification Elements}\mbox{}\\

\begin{itemize}
\item \href{https://jira.lsstcorp.org/browse/LVV-126}{LVV-126} - DMS-REQ-0295-V-01: Transparent Data Access

\end{itemize}

\paragraph{Test Items}\mbox{}\\

\textbf{Test Items\\
}\\
Observe dataset retrieval from multiple LSP instances








\paragraph{Test Procedure}\mbox{}\\
\begin{tabular}{p{4cm}p{12cm}}
\toprule
Step 1
& Description \\ \hline
\end{tabular}
{\scriptsize
Observe dataset retrieval from multiple LSP instances

}
\begin{tabular}{p{3cm}p{13cm}}
\hline
            & Expected Result \\ \hline
\end{tabular}

\subsubsection{LVV-T101 - Verify implementation of Transient Alert Distribution}\label{lvv-t101}

\begin{longtable}[]{llllll}
\toprule
Version & Status & Priority & Verification Type & Owner
\\\midrule
1 & Draft & Normal &
Test & Kian-Tat Lim
\\\bottomrule
\multicolumn{6}{c}{ Open \href{https://jira.lsstcorp.org/secure/Tests.jspa\#/testCase/LVV-T101}{LVV-T101} in Jira } \\
\end{longtable}

\paragraph{Verification Elements}\mbox{}\\

\begin{itemize}
\item \href{https://jira.lsstcorp.org/browse/LVV-3}{LVV-3} - DMS-REQ-0002-V-01: Transient Alert Distribution

\end{itemize}

\paragraph{Test Items}\mbox{}\\

Precursor or simulated data, execute AP, observe distribution to
simulated clients using standard protocols








\paragraph{Test Procedure}\mbox{}\\
\begin{tabular}{p{4cm}p{12cm}}
\toprule
Step 1
& Description \\ \hline
\end{tabular}
{\scriptsize
Execute AP

}
\begin{tabular}{p{3cm}p{13cm}}
\hline
            & Expected Result \\ \hline
\end{tabular}

\begin{tabular}{p{4cm}p{12cm}}
\toprule
Step 2
& Description \\ \hline
\end{tabular}
{\scriptsize
Observe distribution to simulated clients using standard protocols

}
\begin{tabular}{p{3cm}p{13cm}}
\hline
            & Expected Result \\ \hline
\end{tabular}

\subsubsection{LVV-T102 - Verify implementation of Solar System Objects Available Within Specified
Time}\label{lvv-t102}

\begin{longtable}[]{llllll}
\toprule
Version & Status & Priority & Verification Type & Owner
\\\midrule
1 & Draft & Normal &
Test & Kian-Tat Lim
\\\bottomrule
\multicolumn{6}{c}{ Open \href{https://jira.lsstcorp.org/secure/Tests.jspa\#/testCase/LVV-T102}{LVV-T102} in Jira } \\
\end{longtable}

\paragraph{Verification Elements}\mbox{}\\

\begin{itemize}
\item \href{https://jira.lsstcorp.org/browse/LVV-36}{LVV-36} - DMS-REQ-0089-V-01: Solar System Objects Available Within Specified Time

\item \href{https://jira.lsstcorp.org/browse/LVV-1276}{LVV-1276} - OSS-REQ-0127-V-01: Level 1 Data Product Availability

\item \href{https://jira.lsstcorp.org/browse/LVV-9803}{LVV-9803} - DMS-REQ-0004-V-03: Time to availability of Solar System Object orbits

\end{itemize}

\paragraph{Test Items}\mbox{}\\

Execute single-day operations rehearsal, observe data products generated
in time








\paragraph{Test Procedure}\mbox{}\\
\begin{tabular}{p{4cm}p{12cm}}
\toprule
Step 1
& Description \\ \hline
\end{tabular}
{\scriptsize
Execute single-day operations rehearsal

}
\begin{tabular}{p{3cm}p{13cm}}
\hline
            & Expected Result \\ \hline
\end{tabular}

\begin{tabular}{p{4cm}p{12cm}}
\toprule
Step 2
& Description \\ \hline
\end{tabular}
{\scriptsize
~Observe data products generated in time

}
\begin{tabular}{p{3cm}p{13cm}}
\hline
            & Expected Result \\ \hline
\end{tabular}

\subsubsection{LVV-T104 - Verify implementation of Generate DMS Performance Report Within
Specified Time}\label{lvv-t104}

\begin{longtable}[]{llllll}
\toprule
Version & Status & Priority & Verification Type & Owner
\\\midrule
1 & Draft & Normal &
Test & Kian-Tat Lim
\\\bottomrule
\multicolumn{6}{c}{ Open \href{https://jira.lsstcorp.org/secure/Tests.jspa\#/testCase/LVV-T104}{LVV-T104} in Jira } \\
\end{longtable}

\paragraph{Verification Elements}\mbox{}\\

\begin{itemize}
\item \href{https://jira.lsstcorp.org/browse/LVV-40}{LVV-40} - DMS-REQ-0098-V-01: Generate DMS Performance Report Within Specified Time

\end{itemize}

\paragraph{Test Items}\mbox{}\\

Verify that the DMS can generate a nightly Perfomance Report within
perfReportComplTime








\paragraph{Test Procedure}\mbox{}\\
\begin{tabular}{p{4cm}p{12cm}}
\toprule
Step 1
& Description \\ \hline
\end{tabular}
{\scriptsize
Execute single-day operations rehearsal

}
\begin{tabular}{p{3cm}p{13cm}}
\hline
            & Expected Result \\ \hline
\end{tabular}

\begin{tabular}{p{4cm}p{12cm}}
\toprule
Step 2
& Description \\ \hline
\end{tabular}
{\scriptsize
Observe performance report is generated on time and with correct
contents

}
\begin{tabular}{p{3cm}p{13cm}}
\hline
            & Expected Result \\ \hline
\end{tabular}

\subsubsection{LVV-T105 - Verify implementation of Generate Calibration Report Within Specified
Time}\label{lvv-t105}

\begin{longtable}[]{llllll}
\toprule
Version & Status & Priority & Verification Type & Owner
\\\midrule
1 & Draft & Normal &
Test & Kian-Tat Lim
\\\bottomrule
\multicolumn{6}{c}{ Open \href{https://jira.lsstcorp.org/secure/Tests.jspa\#/testCase/LVV-T105}{LVV-T105} in Jira } \\
\end{longtable}

\paragraph{Verification Elements}\mbox{}\\

\begin{itemize}
\item \href{https://jira.lsstcorp.org/browse/LVV-42}{LVV-42} - DMS-REQ-0100-V-01: Generate Calibration Report Within Specified Time

\end{itemize}

\paragraph{Test Items}\mbox{}\\

Verify that the DMS can generate a night Calibration Report ~in both
human-readable and machine-parseable forms.








\paragraph{Test Procedure}\mbox{}\\
\begin{tabular}{p{4cm}p{12cm}}
\toprule
Step 1
& Description \\ \hline
\end{tabular}
{\scriptsize
Execute single-day operations rehearsal

}
\begin{tabular}{p{3cm}p{13cm}}
\hline
            & Expected Result \\ \hline
\end{tabular}

\begin{tabular}{p{4cm}p{12cm}}
\toprule
Step 2
& Description \\ \hline
\end{tabular}
{\scriptsize
Observe calibration report is generated on time and with correct
contents

}
\begin{tabular}{p{3cm}p{13cm}}
\hline
            & Expected Result \\ \hline
\end{tabular}

\subsubsection{LVV-T106 - Verify implementation of Calibration Images Available Within Specified
Time}\label{lvv-t106}

\begin{longtable}[]{llllll}
\toprule
Version & Status & Priority & Verification Type & Owner
\\\midrule
1 & Draft & Normal &
Test & Kian-Tat Lim
\\\bottomrule
\multicolumn{6}{c}{ Open \href{https://jira.lsstcorp.org/secure/Tests.jspa\#/testCase/LVV-T106}{LVV-T106} in Jira } \\
\end{longtable}

\paragraph{Verification Elements}\mbox{}\\

\begin{itemize}
\item \href{https://jira.lsstcorp.org/browse/LVV-58}{LVV-58} - DMS-REQ-0131-V-01: Time allowed to process calibs

\end{itemize}

\paragraph{Test Items}\mbox{}\\

Execute single-day operations rehearsal, observe data products generated








\paragraph{Test Procedure}\mbox{}\\
\begin{tabular}{p{4cm}p{12cm}}
\toprule
Step 1
& Description \\ \hline
\end{tabular}
{\scriptsize
Identify a dataset of raw calibration exposures containing at least
\textbf{nCalExpProc = 25~}exposures. (If it contains more than 25
exposures, use only 25 for the test.)

}
\begin{tabular}{p{3cm}p{13cm}}
\hline
            & Expected Result \\ \hline
\end{tabular}

\begin{tabular}{p{4cm}p{12cm}}
\toprule
Step 2-1
{\scriptsize from \hyperref[lvv-t1059]{LVV-T1059} }
& Description \\ \hline
\end{tabular}
{\scriptsize
Execute the Daily Calibration Products Update payload. The payload uses
raw calibration images and information from the Transformed EFD to
generate a subset of Master Calibration Images and Calibration Database
entries in the Data Backbone.

}
\begin{tabular}{p{3cm}p{13cm}}
\hline
            & Expected Result \\ \hline
\end{tabular}

\begin{tabular}{p{4cm}p{12cm}}
\toprule
Step 2-2
{\scriptsize from \hyperref[lvv-t1059]{LVV-T1059} }
& Description \\ \hline
\end{tabular}
{\scriptsize
Confirm that the expected Master Calibration images and Calibration
Database entries are present and well-formed.

}
\begin{tabular}{p{3cm}p{13cm}}
\hline
            & Expected Result \\ \hline
\end{tabular}

\begin{tabular}{p{4cm}p{12cm}}
\toprule
Step 3
& Description \\ \hline
\end{tabular}
{\scriptsize
Confirm that the processing completed successfully within
\textbf{calProcTime = 1200 seconds.}

}
\begin{tabular}{p{3cm}p{13cm}}
\hline
            & Expected Result \\ \hline
\end{tabular}
{\scriptsize
Calibration products resulting from processed raw calibration exposures
are present within calProcTime, and are well-formed images.

}

\subsubsection{LVV-T107 - Verify implementation of Level-1 Production Completeness}\label{lvv-t107}

\begin{longtable}[]{llllll}
\toprule
Version & Status & Priority & Verification Type & Owner
\\\midrule
1 & Draft & Normal &
Test & Eric Bellm
\\\bottomrule
\multicolumn{6}{c}{ Open \href{https://jira.lsstcorp.org/secure/Tests.jspa\#/testCase/LVV-T107}{LVV-T107} in Jira } \\
\end{longtable}

\paragraph{Verification Elements}\mbox{}\\

\begin{itemize}
\item \href{https://jira.lsstcorp.org/browse/LVV-115}{LVV-115} - DMS-REQ-0284-V-01: Level-1 Production Completeness

\end{itemize}

\paragraph{Test Items}\mbox{}\\

Verify that the DMS successfully processes all images of sufficiently
quality for processing are eventually processed even after connectivity
failures.


\paragraph{Predecessors}\mbox{}\\
\href{https://jira.lsstcorp.org/secure/Tests.jspa\#/testCase/LVV-T284}{LVV-T284}






\paragraph{Test Procedure}\mbox{}\\
\begin{tabular}{p{4cm}p{12cm}}
\toprule
Step 1
& Description \\ \hline
\end{tabular}
{\scriptsize
Ingest raw data while simulating failures and outages, observe eventual
recovery

}
\begin{tabular}{p{3cm}p{13cm}}
\hline
            & Expected Result \\ \hline
\end{tabular}

\subsubsection{LVV-T108 - Verify implementation of Level 1 Source Association}\label{lvv-t108}

\begin{longtable}[]{llllll}
\toprule
Version & Status & Priority & Verification Type & Owner
\\\midrule
1 & Draft & Normal &
Test & Eric Bellm
\\\bottomrule
\multicolumn{6}{c}{ Open \href{https://jira.lsstcorp.org/secure/Tests.jspa\#/testCase/LVV-T108}{LVV-T108} in Jira } \\
\end{longtable}

\paragraph{Verification Elements}\mbox{}\\

\begin{itemize}
\item \href{https://jira.lsstcorp.org/browse/LVV-116}{LVV-116} - DMS-REQ-0285-V-01: Level 1 Source Association

\end{itemize}

\paragraph{Test Items}\mbox{}\\

Verify that the DMS associates DIASources into a DIAObject or SSObject.








\paragraph{Test Procedure}\mbox{}\\
\begin{tabular}{p{4cm}p{12cm}}
\toprule
Step 1
& Description \\ \hline
\end{tabular}
{\scriptsize
Delegate to AP

}
\begin{tabular}{p{3cm}p{13cm}}
\hline
            & Expected Result \\ \hline
\end{tabular}

\subsubsection{LVV-T109 - Verify implementation of SSObject Precovery}\label{lvv-t109}

\begin{longtable}[]{llllll}
\toprule
Version & Status & Priority & Verification Type & Owner
\\\midrule
1 & Draft & Normal &
Test & Eric Bellm
\\\bottomrule
\multicolumn{6}{c}{ Open \href{https://jira.lsstcorp.org/secure/Tests.jspa\#/testCase/LVV-T109}{LVV-T109} in Jira } \\
\end{longtable}

\paragraph{Verification Elements}\mbox{}\\

\begin{itemize}
\item \href{https://jira.lsstcorp.org/browse/LVV-117}{LVV-117} - DMS-REQ-0286-V-01: SSObject Precovery

\end{itemize}

\paragraph{Test Items}\mbox{}\\

Verify that the DMS associates additional DIAObjects (both forward and
back in time) with objects classified as SSObjects.








\paragraph{Test Procedure}\mbox{}\\
\begin{tabular}{p{4cm}p{12cm}}
\toprule
Step 1
& Description \\ \hline
\end{tabular}
{\scriptsize
Delegate to AP

}
\begin{tabular}{p{3cm}p{13cm}}
\hline
            & Expected Result \\ \hline
\end{tabular}

\subsubsection{LVV-T110 - Verify implementation of DIASource Precovery}\label{lvv-t110}

\begin{longtable}[]{llllll}
\toprule
Version & Status & Priority & Verification Type & Owner
\\\midrule
1 & Draft & Normal &
Test & Eric Bellm
\\\bottomrule
\multicolumn{6}{c}{ Open \href{https://jira.lsstcorp.org/secure/Tests.jspa\#/testCase/LVV-T110}{LVV-T110} in Jira } \\
\end{longtable}

\paragraph{Verification Elements}\mbox{}\\

\begin{itemize}
\item \href{https://jira.lsstcorp.org/browse/LVV-118}{LVV-118} - DMS-REQ-0287-V-01: Max look-back time for precovery

\end{itemize}

\paragraph{Test Items}\mbox{}\\

Verify that DMS performs forced photometry for new DIAObjects at all
available images within the precoveryWindow.








\paragraph{Test Procedure}\mbox{}\\
\begin{tabular}{p{4cm}p{12cm}}
\toprule
Step 1
& Description \\ \hline
\end{tabular}
{\scriptsize
Execute single-day operations rehearsal, observe data products generated
in time

}
\begin{tabular}{p{3cm}p{13cm}}
\hline
            & Expected Result \\ \hline
\end{tabular}

\subsubsection{LVV-T111 - Verify implementation of Use of External Orbit Catalogs}\label{lvv-t111}

\begin{longtable}[]{llllll}
\toprule
Version & Status & Priority & Verification Type & Owner
\\\midrule
1 & Draft & Normal &
Test & Eric Bellm
\\\bottomrule
\multicolumn{6}{c}{ Open \href{https://jira.lsstcorp.org/secure/Tests.jspa\#/testCase/LVV-T111}{LVV-T111} in Jira } \\
\end{longtable}

\paragraph{Verification Elements}\mbox{}\\

\begin{itemize}
\item \href{https://jira.lsstcorp.org/browse/LVV-119}{LVV-119} - DMS-REQ-0288-V-01: Use of External Orbit Catalogs

\end{itemize}

\paragraph{Test Items}\mbox{}\\

Verify that the DMS can make use of external catalogs to improve
identification of SSObjects.








\paragraph{Test Procedure}\mbox{}\\
\begin{tabular}{p{4cm}p{12cm}}
\toprule
Step 1
& Description \\ \hline
\end{tabular}
{\scriptsize
Delegate to AP

}
\begin{tabular}{p{3cm}p{13cm}}
\hline
            & Expected Result \\ \hline
\end{tabular}

\subsubsection{LVV-T116 - Verify implementation of Associating Objects across data releases}\label{lvv-t116}

\begin{longtable}[]{llllll}
\toprule
Version & Status & Priority & Verification Type & Owner
\\\midrule
1 & Draft & Normal &
Test & Kian-Tat Lim
\\\bottomrule
\multicolumn{6}{c}{ Open \href{https://jira.lsstcorp.org/secure/Tests.jspa\#/testCase/LVV-T116}{LVV-T116} in Jira } \\
\end{longtable}

\paragraph{Verification Elements}\mbox{}\\

\begin{itemize}
\item \href{https://jira.lsstcorp.org/browse/LVV-181}{LVV-181} - DMS-REQ-0350-V-01: Associating Objects across data releases

\end{itemize}

\paragraph{Test Items}\mbox{}\\

Load DR, observe queryable association








\paragraph{Test Procedure}\mbox{}\\
\begin{tabular}{p{4cm}p{12cm}}
\toprule
Step 1
& Description \\ \hline
\end{tabular}
{\scriptsize
Load DR

}
\begin{tabular}{p{3cm}p{13cm}}
\hline
            & Expected Result \\ \hline
\end{tabular}

\begin{tabular}{p{4cm}p{12cm}}
\toprule
Step 2
& Description \\ \hline
\end{tabular}
{\scriptsize
Observe queryable association

}
\begin{tabular}{p{3cm}p{13cm}}
\hline
            & Expected Result \\ \hline
\end{tabular}

\subsubsection{LVV-T117 - Verify implementation of DAC resource allocation for Level 3 processing}\label{lvv-t117}

\begin{longtable}[]{llllll}
\toprule
Version & Status & Priority & Verification Type & Owner
\\\midrule
1 & Draft & Normal &
Test & Colin Slater
\\\bottomrule
\multicolumn{6}{c}{ Open \href{https://jira.lsstcorp.org/secure/Tests.jspa\#/testCase/LVV-T117}{LVV-T117} in Jira } \\
\end{longtable}

\paragraph{Verification Elements}\mbox{}\\

\begin{itemize}
\item \href{https://jira.lsstcorp.org/browse/LVV-47}{LVV-47} - DMS-REQ-0119-V-01: DAC resource allocation for Level 3 processing

\end{itemize}

\paragraph{Test Items}\mbox{}\\

Verify that compute time and storage space allocations can be granted to
science users.








\paragraph{Test Procedure}\mbox{}\\
\begin{tabular}{p{4cm}p{12cm}}
\toprule
Step 1
& Description \\ \hline
\end{tabular}
{\scriptsize
Create a test user account for the Science Platform.

}
\begin{tabular}{p{3cm}p{13cm}}
\hline
            & Expected Result \\ \hline
\end{tabular}

\begin{tabular}{p{4cm}p{12cm}}
\toprule
Step 2
& Description \\ \hline
\end{tabular}
{\scriptsize
Set the LSP resource allocations for the test user to very low values.

}
\begin{tabular}{p{3cm}p{13cm}}
\hline
            & Expected Result \\ \hline
\end{tabular}

\begin{tabular}{p{4cm}p{12cm}}
\toprule
Step 3
& Description \\ \hline
\end{tabular}
{\scriptsize
Initiate example batch jobs and notebook sessions that will exceed the
specified resource limits.

}
\begin{tabular}{p{3cm}p{13cm}}
\hline
            & Expected Result \\ \hline
\end{tabular}
{\scriptsize
Quota error.

}

\begin{tabular}{p{4cm}p{12cm}}
\toprule
Step 4
& Description \\ \hline
\end{tabular}
{\scriptsize
Transfer sufficient data volumes into the user workspace and MyDB tables
that would exceed the resource quotas.

}
\begin{tabular}{p{3cm}p{13cm}}
\hline
            & Expected Result \\ \hline
\end{tabular}
{\scriptsize
Quota error.

}

\begin{tabular}{p{4cm}p{12cm}}
\toprule
Step 5
& Description \\ \hline
\end{tabular}
{\scriptsize
Reset the user resource quotas to normal values.

}
\begin{tabular}{p{3cm}p{13cm}}
\hline
            & Expected Result \\ \hline
\end{tabular}

\begin{tabular}{p{4cm}p{12cm}}
\toprule
Step 6
& Description \\ \hline
\end{tabular}
{\scriptsize
Initiate the same example batch jobs and notebook sessions that
previously caused an error.

}
\begin{tabular}{p{3cm}p{13cm}}
\hline
            & Expected Result \\ \hline
\end{tabular}
{\scriptsize
Successful notebook and batch job execution.

}

\begin{tabular}{p{4cm}p{12cm}}
\toprule
Step 7
& Description \\ \hline
\end{tabular}
{\scriptsize
Transfer the same data volumes into the user workspace and MyDB tables
that previously caused an error.

}
\begin{tabular}{p{3cm}p{13cm}}
\hline
            & Expected Result \\ \hline
\end{tabular}
{\scriptsize
Successful data transfer.

}

\subsubsection{LVV-T118 - Verify implementation of Level 3 Data Product Self Consistency}\label{lvv-t118}

\begin{longtable}[]{llllll}
\toprule
Version & Status & Priority & Verification Type & Owner
\\\midrule
1 & Draft & Normal &
Test & Colin Slater
\\\bottomrule
\multicolumn{6}{c}{ Open \href{https://jira.lsstcorp.org/secure/Tests.jspa\#/testCase/LVV-T118}{LVV-T118} in Jira } \\
\end{longtable}

\paragraph{Verification Elements}\mbox{}\\

\begin{itemize}
\item \href{https://jira.lsstcorp.org/browse/LVV-48}{LVV-48} - DMS-REQ-0120-V-01: Level 3 Data Product Self Consistency

\end{itemize}

\paragraph{Test Items}\mbox{}\\

Verify that user-driven Level 3 processing is conducted on consistent
sets of input data.








\paragraph{Test Procedure}\mbox{}\\
\begin{tabular}{p{4cm}p{12cm}}
\toprule
Step 1
& Description \\ \hline
\end{tabular}
{\scriptsize
Execute representative processing on DR in PDAC, observe consistency

}
\begin{tabular}{p{3cm}p{13cm}}
\hline
            & Expected Result \\ \hline
\end{tabular}

\subsubsection{LVV-T119 - Verify implementation of Provenance for Level 3 processing at DACs}\label{lvv-t119}

\begin{longtable}[]{llllll}
\toprule
Version & Status & Priority & Verification Type & Owner
\\\midrule
1 & Draft & Normal &
Test & Colin Slater
\\\bottomrule
\multicolumn{6}{c}{ Open \href{https://jira.lsstcorp.org/secure/Tests.jspa\#/testCase/LVV-T119}{LVV-T119} in Jira } \\
\end{longtable}

\paragraph{Verification Elements}\mbox{}\\

\begin{itemize}
\item \href{https://jira.lsstcorp.org/browse/LVV-49}{LVV-49} - DMS-REQ-0121-V-01: Provenance for Level 3 processing at DACs

\item \href{https://jira.lsstcorp.org/browse/LVV-1234}{LVV-1234} - OSS-REQ-0122-V-01: Provenance

\end{itemize}

\paragraph{Test Items}\mbox{}\\

Verify that provenance information is recorded and accessible for
user-generated Level 3 products.








\paragraph{Test Procedure}\mbox{}\\
\begin{tabular}{p{4cm}p{12cm}}
\toprule
Step 1
& Description \\ \hline
\end{tabular}
{\scriptsize
Execute representative processing on DR in PDAC, observe provenance
recording

}
\begin{tabular}{p{3cm}p{13cm}}
\hline
            & Expected Result \\ \hline
\end{tabular}

\subsubsection{LVV-T120 - Verify implementation of Software framework for Level 3 catalog
processing}\label{lvv-t120}

\begin{longtable}[]{llllll}
\toprule
Version & Status & Priority & Verification Type & Owner
\\\midrule
1 & Draft & Normal &
Test & Colin Slater
\\\bottomrule
\multicolumn{6}{c}{ Open \href{https://jira.lsstcorp.org/secure/Tests.jspa\#/testCase/LVV-T120}{LVV-T120} in Jira } \\
\end{longtable}

\paragraph{Verification Elements}\mbox{}\\

\begin{itemize}
\item \href{https://jira.lsstcorp.org/browse/LVV-53}{LVV-53} - DMS-REQ-0125-V-01: Software framework for Level 3 catalog processing

\end{itemize}

\paragraph{Test Items}\mbox{}\\

Verify that user-driven Level 3 processing can be consistently applied
to all records in a catalog.








\paragraph{Test Procedure}\mbox{}\\
\begin{tabular}{p{4cm}p{12cm}}
\toprule
Step 1
& Description \\ \hline
\end{tabular}
{\scriptsize
Execute representative processing on DR in PDAC, observe recognition of
and recovery from failures

}
\begin{tabular}{p{3cm}p{13cm}}
\hline
            & Expected Result \\ \hline
\end{tabular}

\subsubsection{LVV-T121 - Verify implementation of Software framework for Level 3 image processing}\label{lvv-t121}

\begin{longtable}[]{llllll}
\toprule
Version & Status & Priority & Verification Type & Owner
\\\midrule
1 & Draft & Normal &
Test & Colin Slater
\\\bottomrule
\multicolumn{6}{c}{ Open \href{https://jira.lsstcorp.org/secure/Tests.jspa\#/testCase/LVV-T121}{LVV-T121} in Jira } \\
\end{longtable}

\paragraph{Verification Elements}\mbox{}\\

\begin{itemize}
\item \href{https://jira.lsstcorp.org/browse/LVV-56}{LVV-56} - DMS-REQ-0128-V-01: Software framework for Level 3 image processing

\end{itemize}

\paragraph{Test Items}\mbox{}\\

Verify that user-specified Level 3 processing can be applied to the
desired set of images.








\paragraph{Test Procedure}\mbox{}\\
\begin{tabular}{p{4cm}p{12cm}}
\toprule
Step 1
& Description \\ \hline
\end{tabular}
{\scriptsize
Execute representative processing on DR in PDAC, observe recognition of
and recovery from failures

}
\begin{tabular}{p{3cm}p{13cm}}
\hline
            & Expected Result \\ \hline
\end{tabular}

\subsubsection{LVV-T122 - Verify implementation of Level 3 Data Import}\label{lvv-t122}

\begin{longtable}[]{llllll}
\toprule
Version & Status & Priority & Verification Type & Owner
\\\midrule
1 & Draft & Normal &
Test & Colin Slater
\\\bottomrule
\multicolumn{6}{c}{ Open \href{https://jira.lsstcorp.org/secure/Tests.jspa\#/testCase/LVV-T122}{LVV-T122} in Jira } \\
\end{longtable}

\paragraph{Verification Elements}\mbox{}\\

\begin{itemize}
\item \href{https://jira.lsstcorp.org/browse/LVV-121}{LVV-121} - DMS-REQ-0290-V-01: Level 3 Data Import

\end{itemize}

\paragraph{Test Items}\mbox{}\\

Verify that the Science Platform can ingest data from community-standard
file formats.








\paragraph{Test Procedure}\mbox{}\\
\begin{tabular}{p{4cm}p{12cm}}
\toprule
Step 1
& Description \\ \hline
\end{tabular}
{\scriptsize
Use the Science Platform catalog upload tool to ingest a small example
FITS table.

}
\begin{tabular}{p{3cm}p{13cm}}
\hline
            & Expected Result \\ \hline
\end{tabular}

\begin{tabular}{p{4cm}p{12cm}}
\toprule
Step 2
& Description \\ \hline
\end{tabular}
{\scriptsize
Use the Science Platform catalog upload tool to ingest a small example
CSV table.

}
\begin{tabular}{p{3cm}p{13cm}}
\hline
            & Expected Result \\ \hline
\end{tabular}

\begin{tabular}{p{4cm}p{12cm}}
\toprule
Step 3
& Description \\ \hline
\end{tabular}
{\scriptsize
Use the Science Platform catalog upload tool to ingest a large FITS
table that needs to be spatially-sharded in the database.

}
\begin{tabular}{p{3cm}p{13cm}}
\hline
            & Expected Result \\ \hline
\end{tabular}

\begin{tabular}{p{4cm}p{12cm}}
\toprule
Step 4
& Description \\ \hline
\end{tabular}
{\scriptsize
Perform example queries on each of the three tables to verify that all
data is present.

}
\begin{tabular}{p{3cm}p{13cm}}
\hline
            & Expected Result \\ \hline
\end{tabular}
{\scriptsize
Data returned in the queries is identical to the data uploaded.

}

\subsubsection{LVV-T123 - Verify implementation of Access Controls of Level 3 Data Products}\label{lvv-t123}

\begin{longtable}[]{llllll}
\toprule
Version & Status & Priority & Verification Type & Owner
\\\midrule
1 & Draft & Normal &
Test & Robert Gruendl
\\\bottomrule
\multicolumn{6}{c}{ Open \href{https://jira.lsstcorp.org/secure/Tests.jspa\#/testCase/LVV-T123}{LVV-T123} in Jira } \\
\end{longtable}

\paragraph{Verification Elements}\mbox{}\\

\begin{itemize}
\item \href{https://jira.lsstcorp.org/browse/LVV-171}{LVV-171} - DMS-REQ-0340-V-01: Access Controls of Level 3 Data Products

\end{itemize}

\paragraph{Test Items}\mbox{}\\

This test touches upon the interface between the following areas: IT
Security, Identity Management, LSP Portal, and Parallel Distributed
Database. ~The purpose is to show that access to user generated data
products (previously Level 3) can have a variety of access restrictions
varying from single-user, a list, a named group, or open access.








\paragraph{Test Procedure}\mbox{}\\
\begin{tabular}{p{4cm}p{12cm}}
\toprule
Step 1
& Description \\ \hline
\end{tabular}
{\scriptsize
Configure representative access controls in PDAC, observe proper
restrictions

}
\begin{tabular}{p{3cm}p{13cm}}
\hline
            & Expected Result \\ \hline
\end{tabular}

\subsubsection{LVV-T128 - Verify implementation Provide Astrometric Model}\label{lvv-t128}

\begin{longtable}[]{llllll}
\toprule
Version & Status & Priority & Verification Type & Owner
\\\midrule
1 & Draft & Normal &
Test & Colin Slater
\\\bottomrule
\multicolumn{6}{c}{ Open \href{https://jira.lsstcorp.org/secure/Tests.jspa\#/testCase/LVV-T128}{LVV-T128} in Jira } \\
\end{longtable}

\paragraph{Verification Elements}\mbox{}\\

\begin{itemize}
\item \href{https://jira.lsstcorp.org/browse/LVV-17}{LVV-17} - DMS-REQ-0042-V-01: Provide Astrometric Model

\end{itemize}

\paragraph{Test Items}\mbox{}\\

Verify that an astrometric model is available for Objects and
DIAObjects.








\paragraph{Test Procedure}\mbox{}\\
\begin{tabular}{p{4cm}p{12cm}}
\toprule
Step 1
& Description \\ \hline
\end{tabular}
{\scriptsize
Delegate to AP and DRP

}
\begin{tabular}{p{3cm}p{13cm}}
\hline
            & Expected Result \\ \hline
\end{tabular}

\subsubsection{LVV-T130 - Verify implementation of Enable a Range of Shape Measurement Approaches}\label{lvv-t130}

\begin{longtable}[]{llllll}
\toprule
Version & Status & Priority & Verification Type & Owner
\\\midrule
1 & Draft & Normal &
Test & Colin Slater
\\\bottomrule
\multicolumn{6}{c}{ Open \href{https://jira.lsstcorp.org/secure/Tests.jspa\#/testCase/LVV-T130}{LVV-T130} in Jira } \\
\end{longtable}

\paragraph{Verification Elements}\mbox{}\\

\begin{itemize}
\item \href{https://jira.lsstcorp.org/browse/LVV-21}{LVV-21} - DMS-REQ-0052-V-01: Enable a Range of Shape Measurement Approaches

\end{itemize}

\paragraph{Test Items}\mbox{}\\

Verify that multiple shape measurement algorithms can be used.








\paragraph{Test Procedure}\mbox{}\\
\begin{tabular}{p{4cm}p{12cm}}
\toprule
Step 1
& Description \\ \hline
\end{tabular}
{\scriptsize
Delegate to AP and DRP

}
\begin{tabular}{p{3cm}p{13cm}}
\hline
            & Expected Result \\ \hline
\end{tabular}

\subsubsection{LVV-T134 - Verify implementation of Provide Image Access Services}\label{lvv-t134}

\begin{longtable}[]{llllll}
\toprule
Version & Status & Priority & Verification Type & Owner
\\\midrule
1 & Draft & Normal &
Inspection & Gregory Dubois-Felsmann
\\\bottomrule
\multicolumn{6}{c}{ Open \href{https://jira.lsstcorp.org/secure/Tests.jspa\#/testCase/LVV-T134}{LVV-T134} in Jira } \\
\end{longtable}

\paragraph{Verification Elements}\mbox{}\\

\begin{itemize}
\item \href{https://jira.lsstcorp.org/browse/LVV-27}{LVV-27} - DMS-REQ-0065-V-01: Provide Image Access Services

\end{itemize}

\paragraph{Test Items}\mbox{}\\

Verify that images can be identified and that images and image cut-outs
can be retrieved using the network interfaces - primarily IVOA
standards-based - and Python APIs provided for image access by science
users.








\paragraph{Test Procedure}\mbox{}\\
\begin{tabular}{p{4cm}p{12cm}}
\toprule
Step 1
& Description \\ \hline
\end{tabular}
{\scriptsize
Inspect that the following test cases have been executed and passed:
\href{https://jira.lsstcorp.org/secure/Tests.jspa\#/testCase/LVV-T803}{LVV-T803},
\href{https://jira.lsstcorp.org/secure/Tests.jspa\#/testCase/LVV-T810}{LVV-T810},
\href{https://jira.lsstcorp.org/secure/Tests.jspa\#/testCase/LVV-T811}{LVV-T811},
\href{https://jira.lsstcorp.org/secure/Tests.jspa\#/testCase/LVV-T812}{LVV-T812}.\\[2\baselineskip]The
requirement is fully satisfied by lower-level LSP test cases.

}
\begin{tabular}{p{3cm}p{13cm}}
\hline
            & Expected Result \\ \hline
\end{tabular}
{\scriptsize
Test cases
\href{https://jira.lsstcorp.org/secure/Tests.jspa\#/testCase/LVV-T803}{LVV-T803},
\href{https://jira.lsstcorp.org/secure/Tests.jspa\#/testCase/LVV-T810}{LVV-T810},
\href{https://jira.lsstcorp.org/secure/Tests.jspa\#/testCase/LVV-T811}{LVV-T811},
\href{https://jira.lsstcorp.org/secure/Tests.jspa\#/testCase/LVV-T812}{LVV-T812}
passed without blocking issues.

}

\subsubsection{LVV-T138 - Verify implementation of Bulk Download Service}\label{lvv-t138}

\begin{longtable}[]{llllll}
\toprule
Version & Status & Priority & Verification Type & Owner
\\\midrule
1 & Draft & Normal &
Test & Robert Gruendl
\\\bottomrule
\multicolumn{6}{c}{ Open \href{https://jira.lsstcorp.org/secure/Tests.jspa\#/testCase/LVV-T138}{LVV-T138} in Jira } \\
\end{longtable}

\paragraph{Verification Elements}\mbox{}\\

\begin{itemize}
\item \href{https://jira.lsstcorp.org/browse/LVV-131}{LVV-131} - DMS-REQ-0300-V-01: Bulk Download Service

\end{itemize}

\paragraph{Test Items}\mbox{}\\

Bulk Download








\paragraph{Test Procedure}\mbox{}\\
\begin{tabular}{p{4cm}p{12cm}}
\toprule
Step 1
& Description \\ \hline
\end{tabular}
{\scriptsize
Setup large transfer request and examine the data transfer rates
achieved.

}
\begin{tabular}{p{3cm}p{13cm}}
\hline
            & Expected Result \\ \hline
\end{tabular}

\begin{tabular}{p{4cm}p{12cm}}
\toprule
Step 2
& Description \\ \hline
\end{tabular}
{\scriptsize
Test should be repeated while observing in firehose mode (with LSSTCam)
during science verification to ensure that bulk transfer does not
compromise normal nightly operations.

}
\begin{tabular}{p{3cm}p{13cm}}
\hline
            & Expected Result \\ \hline
\end{tabular}

\subsubsection{LVV-T142 - Verify implementation of Production Fault Tolerance}\label{lvv-t142}

\begin{longtable}[]{llllll}
\toprule
Version & Status & Priority & Verification Type & Owner
\\\midrule
1 & Draft & Normal &
Test & Robert Gruendl
\\\bottomrule
\multicolumn{6}{c}{ Open \href{https://jira.lsstcorp.org/secure/Tests.jspa\#/testCase/LVV-T142}{LVV-T142} in Jira } \\
\end{longtable}

\paragraph{Verification Elements}\mbox{}\\

\begin{itemize}
\item \href{https://jira.lsstcorp.org/browse/LVV-135}{LVV-135} - DMS-REQ-0304-V-01: Production Fault Tolerance

\end{itemize}

\paragraph{Test Items}\mbox{}\\

Demonstrate production systems report faults in pipeline executions and
that system is able to recover. ~Where recovery can mean the ability to
provide production artifacts for examination, return production elements
ready for subsequent use, and/or reset and repeat production attempts.








\paragraph{Test Procedure}\mbox{}\\
\begin{tabular}{p{4cm}p{12cm}}
\toprule
Step 1
& Description \\ \hline
\end{tabular}
{\scriptsize
Execute AP and DRP, simulate failures, observe correct processing

}
\begin{tabular}{p{3cm}p{13cm}}
\hline
            & Expected Result \\ \hline
\end{tabular}

\subsubsection{LVV-T147 - Verify implementation of Control of Level-1 Production}\label{lvv-t147}

\begin{longtable}[]{llllll}
\toprule
Version & Status & Priority & Verification Type & Owner
\\\midrule
1 & Draft & Normal &
Test & Robert Gruendl
\\\bottomrule
\multicolumn{6}{c}{ Open \href{https://jira.lsstcorp.org/secure/Tests.jspa\#/testCase/LVV-T147}{LVV-T147} in Jira } \\
\end{longtable}

\paragraph{Verification Elements}\mbox{}\\

\begin{itemize}
\item \href{https://jira.lsstcorp.org/browse/LVV-132}{LVV-132} - DMS-REQ-0301-V-01: Control of Level-1 Production

\end{itemize}

\paragraph{Test Items}\mbox{}\\

Demonstrate that the DMS can control all Prompt Processing across DMS
facilities.








\paragraph{Test Procedure}\mbox{}\\
\begin{tabular}{p{4cm}p{12cm}}
\toprule
Step 1
& Description \\ \hline
\end{tabular}
{\scriptsize
Observe existence and capability of Prompt DMCS

}
\begin{tabular}{p{3cm}p{13cm}}
\hline
            & Expected Result \\ \hline
\end{tabular}

\subsubsection{LVV-T148 - Verify implementation of Unique Processing Coverage}\label{lvv-t148}

\begin{longtable}[]{llllll}
\toprule
Version & Status & Priority & Verification Type & Owner
\\\midrule
1 & Draft & Normal &
Test & Colin Slater
\\\bottomrule
\multicolumn{6}{c}{ Open \href{https://jira.lsstcorp.org/secure/Tests.jspa\#/testCase/LVV-T148}{LVV-T148} in Jira } \\
\end{longtable}

\paragraph{Verification Elements}\mbox{}\\

\begin{itemize}
\item \href{https://jira.lsstcorp.org/browse/LVV-138}{LVV-138} - DMS-REQ-0307-V-01: Unique Processing Coverage

\end{itemize}

\paragraph{Test Items}\mbox{}\\

Verify that a user-specified criterion can be used to process each
record in a table exactly once.








\paragraph{Test Procedure}\mbox{}\\
\begin{tabular}{p{4cm}p{12cm}}
\toprule
Step 1
& Description \\ \hline
\end{tabular}
{\scriptsize
Execute representative processing, observe lack of duplicates or missing
rows even in the presence of failures

}
\begin{tabular}{p{3cm}p{13cm}}
\hline
            & Expected Result \\ \hline
\end{tabular}

\subsubsection{LVV-T152 - Verify implementation of Keep Historical Alert Archive}\label{lvv-t152}

\begin{longtable}[]{llllll}
\toprule
Version & Status & Priority & Verification Type & Owner
\\\midrule
1 & Draft & Normal &
Test & Eric Bellm
\\\bottomrule
\multicolumn{6}{c}{ Open \href{https://jira.lsstcorp.org/secure/Tests.jspa\#/testCase/LVV-T152}{LVV-T152} in Jira } \\
\end{longtable}

\paragraph{Verification Elements}\mbox{}\\

\begin{itemize}
\item \href{https://jira.lsstcorp.org/browse/LVV-37}{LVV-37} - DMS-REQ-0094-V-01: Keep Historical Alert Archive

\end{itemize}

\paragraph{Test Items}\mbox{}\\

Verify that the DMS preserves and makes accessible an Alert Archive for
reference and for false alert analyses








\paragraph{Test Procedure}\mbox{}\\
\begin{tabular}{p{4cm}p{12cm}}
\toprule
Step 1
& Description \\ \hline
\end{tabular}
{\scriptsize
Simulated alert stream, load Alert DB, observe access to Alert DB

}
\begin{tabular}{p{3cm}p{13cm}}
\hline
            & Expected Result \\ \hline
\end{tabular}

\subsubsection{LVV-T154 - Verify implementation of Raw Data Archiving Reliability}\label{lvv-t154}

\begin{longtable}[]{llllll}
\toprule
Version & Status & Priority & Verification Type & Owner
\\\midrule
1 & Draft & Normal &
Test & Colin Slater
\\\bottomrule
\multicolumn{6}{c}{ Open \href{https://jira.lsstcorp.org/secure/Tests.jspa\#/testCase/LVV-T154}{LVV-T154} in Jira } \\
\end{longtable}

\paragraph{Verification Elements}\mbox{}\\

\begin{itemize}
\item \href{https://jira.lsstcorp.org/browse/LVV-140}{LVV-140} - DMS-REQ-0309-V-01: Raw Data Archiving Reliability

\end{itemize}

\paragraph{Test Items}\mbox{}\\

Verify that raw images are reliably archived.








\paragraph{Test Procedure}\mbox{}\\
\begin{tabular}{p{4cm}p{12cm}}
\toprule
Step 1
& Description \\ \hline
\end{tabular}
{\scriptsize
Analyze sources of loss or corruption after mitigation to compute
estimated reliability

}
\begin{tabular}{p{3cm}p{13cm}}
\hline
            & Expected Result \\ \hline
\end{tabular}

\subsubsection{LVV-T155 - Verify implementation of Un-Archived Data Product Cache}\label{lvv-t155}

\begin{longtable}[]{llllll}
\toprule
Version & Status & Priority & Verification Type & Owner
\\\midrule
1 & Draft & Normal &
Test & Robert Gruendl
\\\bottomrule
\multicolumn{6}{c}{ Open \href{https://jira.lsstcorp.org/secure/Tests.jspa\#/testCase/LVV-T155}{LVV-T155} in Jira } \\
\end{longtable}

\paragraph{Verification Elements}\mbox{}\\

\begin{itemize}
\item \href{https://jira.lsstcorp.org/browse/LVV-141}{LVV-141} - DMS-REQ-0310-V-01: Un-Archived Data Product Cache

\end{itemize}

\paragraph{Test Items}\mbox{}\\

Demonstrate that the DMS provides low-latency storage for at least
I1CacheLifetime (30 days) to keep prompt processing pre-covery images on
hand.








\paragraph{Test Procedure}\mbox{}\\
\begin{tabular}{p{4cm}p{12cm}}
\toprule
Step 1
& Description \\ \hline
\end{tabular}
{\scriptsize
Delegate to DBB

}
\begin{tabular}{p{3cm}p{13cm}}
\hline
            & Expected Result \\ \hline
\end{tabular}

\subsubsection{LVV-T156 - Verify implementation of Regenerate Un-archived Data Products}\label{lvv-t156}

\begin{longtable}[]{llllll}
\toprule
Version & Status & Priority & Verification Type & Owner
\\\midrule
1 & Draft & Normal &
Test & Simon Krughoff
\\\bottomrule
\multicolumn{6}{c}{ Open \href{https://jira.lsstcorp.org/secure/Tests.jspa\#/testCase/LVV-T156}{LVV-T156} in Jira } \\
\end{longtable}

\paragraph{Verification Elements}\mbox{}\\

\begin{itemize}
\item \href{https://jira.lsstcorp.org/browse/LVV-142}{LVV-142} - DMS-REQ-0311-V-01: Regenerate Un-archived Data Products

\end{itemize}

\paragraph{Test Items}\mbox{}\\

Not all of the ancillary data products produced by a data release will
be archived permanently. ~These ancillary products have been promised as
accessible to the community.~ Show that these products can be produced
from an archived data release after the fact.








\paragraph{Test Procedure}\mbox{}\\
\begin{tabular}{p{4cm}p{12cm}}
\toprule
Step 1
& Description \\ \hline
\end{tabular}
{\scriptsize
Run a small DRP processing job and download unarchived data products.

}
\begin{tabular}{p{3cm}p{13cm}}
\hline
            & Expected Result \\ \hline
\end{tabular}

\begin{tabular}{p{4cm}p{12cm}}
\toprule
Step 2
& Description \\ \hline
\end{tabular}
{\scriptsize
Wait for (or force) a processing stack change so that the subsequent
re-processing will be forced to use an older software build.

}
\begin{tabular}{p{3cm}p{13cm}}
\hline
            & Expected Result \\ \hline
\end{tabular}

\begin{tabular}{p{4cm}p{12cm}}
\toprule
Step 3
& Description \\ \hline
\end{tabular}
{\scriptsize
Using provenance information from the products in Step 1, request a
re-processing and compare results with previously unarchived products.

}
\begin{tabular}{p{3cm}p{13cm}}
\hline
            & Expected Result \\ \hline
\end{tabular}

\subsubsection{LVV-T157 - Verify implementation Level 1 Data Product Access}\label{lvv-t157}

\begin{longtable}[]{llllll}
\toprule
Version & Status & Priority & Verification Type & Owner
\\\midrule
1 & Draft & Normal &
Test & Colin Slater
\\\bottomrule
\multicolumn{6}{c}{ Open \href{https://jira.lsstcorp.org/secure/Tests.jspa\#/testCase/LVV-T157}{LVV-T157} in Jira } \\
\end{longtable}

\paragraph{Verification Elements}\mbox{}\\

\begin{itemize}
\item \href{https://jira.lsstcorp.org/browse/LVV-143}{LVV-143} - DMS-REQ-0312-V-01: Level 1 Data Product Access

\end{itemize}

\paragraph{Test Items}\mbox{}\\

Verify that Level 1 Data Products are accessible by science users.








\paragraph{Test Procedure}\mbox{}\\
\begin{tabular}{p{4cm}p{12cm}}
\toprule
Step 1
& Description \\ \hline
\end{tabular}
{\scriptsize
Delegate to LSP

}
\begin{tabular}{p{3cm}p{13cm}}
\hline
            & Expected Result \\ \hline
\end{tabular}

\subsubsection{LVV-T158 - Verify implementation Level 1 and 2 Catalog Access}\label{lvv-t158}

\begin{longtable}[]{llllll}
\toprule
Version & Status & Priority & Verification Type & Owner
\\\midrule
1 & Draft & Normal &
Test & Colin Slater
\\\bottomrule
\multicolumn{6}{c}{ Open \href{https://jira.lsstcorp.org/secure/Tests.jspa\#/testCase/LVV-T158}{LVV-T158} in Jira } \\
\end{longtable}

\paragraph{Verification Elements}\mbox{}\\

\begin{itemize}
\item \href{https://jira.lsstcorp.org/browse/LVV-144}{LVV-144} - DMS-REQ-0313-V-01: Level 1 \& 2 Catalog Access

\end{itemize}

\paragraph{Test Items}\mbox{}\\

Verify that Data Release Products are accessible by science users.








\paragraph{Test Procedure}\mbox{}\\
\begin{tabular}{p{4cm}p{12cm}}
\toprule
Step 1
& Description \\ \hline
\end{tabular}
{\scriptsize
Delegate to LSP

}
\begin{tabular}{p{3cm}p{13cm}}
\hline
            & Expected Result \\ \hline
\end{tabular}

\subsubsection{LVV-T159 - Verify implementation of Regenerating Data Products from Previous Data
Releases}\label{lvv-t159}

\begin{longtable}[]{llllll}
\toprule
Version & Status & Priority & Verification Type & Owner
\\\midrule
1 & Draft & Normal &
Test & Simon Krughoff
\\\bottomrule
\multicolumn{6}{c}{ Open \href{https://jira.lsstcorp.org/secure/Tests.jspa\#/testCase/LVV-T159}{LVV-T159} in Jira } \\
\end{longtable}

\paragraph{Verification Elements}\mbox{}\\

\begin{itemize}
\item \href{https://jira.lsstcorp.org/browse/LVV-167}{LVV-167} - DMS-REQ-0336-V-01: Regenerating Data Products from Previous Data
Releases

\end{itemize}

\paragraph{Test Items}\mbox{}\\

Show that un-archived data products from previous data releases can be
generated using through the LSST Science Platform.








\paragraph{Test Procedure}\mbox{}\\
\begin{tabular}{p{4cm}p{12cm}}
\toprule
Step 1
& Description \\ \hline
\end{tabular}
{\scriptsize
Delegate to LSP

}
\begin{tabular}{p{3cm}p{13cm}}
\hline
            & Expected Result \\ \hline
\end{tabular}

\subsubsection{LVV-T160 - Verify implementation of Providing a Precovery Service}\label{lvv-t160}

\begin{longtable}[]{llllll}
\toprule
Version & Status & Priority & Verification Type & Owner
\\\midrule
1 & Draft & Normal &
Test & Gregory Dubois-Felsmann
\\\bottomrule
\multicolumn{6}{c}{ Open \href{https://jira.lsstcorp.org/secure/Tests.jspa\#/testCase/LVV-T160}{LVV-T160} in Jira } \\
\end{longtable}

\paragraph{Verification Elements}\mbox{}\\

\begin{itemize}
\item \href{https://jira.lsstcorp.org/browse/LVV-172}{LVV-172} - DMS-REQ-0341-V-01: Max elapsed time for precovery results

\end{itemize}

\paragraph{Test Items}\mbox{}\\

Verify that a technical capability to perform user-directed precovery
analyses on difference images exists and that it is exposed through the
LSST Science Platform. ~Verified by testing against precursor
datasets.\\
(Involves: LSP Portal, MOPS and Forced Photometry)








\paragraph{Test Procedure}\mbox{}\\
\begin{tabular}{p{4cm}p{12cm}}
\toprule
Step 1
& Description \\ \hline
\end{tabular}
{\scriptsize
Run Precovery within follow-on Alert Production (i.e. daily
post-processing on 30 day store).

}
\begin{tabular}{p{3cm}p{13cm}}
\hline
            & Expected Result \\ \hline
\end{tabular}

\begin{tabular}{p{4cm}p{12cm}}
\toprule
Step 2
& Description \\ \hline
\end{tabular}
{\scriptsize
Within Science Platform, initiate request to perform precovery for a
list of sources over same period (and longer). ~Include among the
sources for precovery quasars from
\href{https://jira.lsstcorp.org/secure/Tests.jspa\#/testCase/LVV-T80}{LVV-T80}.~

}
\begin{tabular}{p{3cm}p{13cm}}
\hline
            & Expected Result \\ \hline
\end{tabular}

\begin{tabular}{p{4cm}p{12cm}}
\toprule
Step 3
& Description \\ \hline
\end{tabular}
{\scriptsize
Examine the results. ~Compare the results for the period where there is
overlap with precovery run\ldots{} and quasar photometry with those from
\href{https://jira.lsstcorp.org/secure/Tests.jspa\#/testCase/LVV-T80}{LVV-T80}
to verify user service performs as production services.

}
\begin{tabular}{p{3cm}p{13cm}}
\hline
            & Expected Result \\ \hline
\end{tabular}

\subsubsection{LVV-T161 - Verify implementation of Logging of catalog queries}\label{lvv-t161}

\begin{longtable}[]{llllll}
\toprule
Version & Status & Priority & Verification Type & Owner
\\\midrule
1 & Draft & Normal &
Test & Robert Gruendl
\\\bottomrule
\multicolumn{6}{c}{ Open \href{https://jira.lsstcorp.org/secure/Tests.jspa\#/testCase/LVV-T161}{LVV-T161} in Jira } \\
\end{longtable}

\paragraph{Verification Elements}\mbox{}\\

\begin{itemize}
\item \href{https://jira.lsstcorp.org/browse/LVV-176}{LVV-176} - DMS-REQ-0345-V-01: Logging of catalog queries

\end{itemize}

\paragraph{Test Items}\mbox{}\\

Demonstrate logging of queries of LSST databases. ~Logged queries are
globally available to DB administrators but otherwise private excepting
the user that made the query.








\paragraph{Test Procedure}\mbox{}\\
\begin{tabular}{p{4cm}p{12cm}}
\toprule
Step 1
& Description \\ \hline
\end{tabular}
{\scriptsize
Delegate to LSP

}
\begin{tabular}{p{3cm}p{13cm}}
\hline
            & Expected Result \\ \hline
\end{tabular}

\subsubsection{LVV-T162 - Verify implementation of Access to Previous Data Releases}\label{lvv-t162}

\begin{longtable}[]{llllll}
\toprule
Version & Status & Priority & Verification Type & Owner
\\\midrule
1 & Draft & Normal &
Test & Gregory Dubois-Felsmann
\\\bottomrule
\multicolumn{6}{c}{ Open \href{https://jira.lsstcorp.org/secure/Tests.jspa\#/testCase/LVV-T162}{LVV-T162} in Jira } \\
\end{longtable}

\paragraph{Verification Elements}\mbox{}\\

\begin{itemize}
\item \href{https://jira.lsstcorp.org/browse/LVV-189}{LVV-189} - DMS-REQ-0363-V-01: Access to Previous Data Releases

\end{itemize}

\paragraph{Test Items}\mbox{}\\

Verify this high-level requirement, which states that the other data
access requirements, for images and catalogs, all must be satisfied for
multiple data releases. ~Verified by inspection, i.e., by determining
that the data access system components, from middleware through APIs to
user interfaces, are designed to support data from multiple releases, as
well as by direct testing using a synthetic test environment containing
multiple releases.\\
(Involves: Data Backbone, Managed Database, LSP Portal, LSP JupyterLab,
LSP Web APIs, Parallel Distributed Database)








\paragraph{Test Procedure}\mbox{}\\
\begin{tabular}{p{4cm}p{12cm}}
\toprule
Step 1
& Description \\ \hline
\end{tabular}
{\scriptsize
From Science Platform initiate request for image and catalog products
from one of the two release sets.

}
\begin{tabular}{p{3cm}p{13cm}}
\hline
            & Expected Result \\ \hline
\end{tabular}

\begin{tabular}{p{4cm}p{12cm}}
\toprule
Step 2
& Description \\ \hline
\end{tabular}
{\scriptsize
From Science Platform re-issue the same request but specifying the
alternate/earlier release set.

}
\begin{tabular}{p{3cm}p{13cm}}
\hline
            & Expected Result \\ \hline
\end{tabular}

\begin{tabular}{p{4cm}p{12cm}}
\toprule
Step 3
& Description \\ \hline
\end{tabular}
{\scriptsize
Compare results and identify differences that are germaine to the
relevant Data Release Sets are found.

}
\begin{tabular}{p{3cm}p{13cm}}
\hline
            & Expected Result \\ \hline
\end{tabular}

\subsubsection{LVV-T163 - Verify implementation of Data Access Services}\label{lvv-t163}

\begin{longtable}[]{llllll}
\toprule
Version & Status & Priority & Verification Type & Owner
\\\midrule
1 & Draft & Normal &
Test & Robert Gruendl
\\\bottomrule
\multicolumn{6}{c}{ Open \href{https://jira.lsstcorp.org/secure/Tests.jspa\#/testCase/LVV-T163}{LVV-T163} in Jira } \\
\end{longtable}

\paragraph{Verification Elements}\mbox{}\\

\begin{itemize}
\item \href{https://jira.lsstcorp.org/browse/LVV-190}{LVV-190} - DMS-REQ-0364-V-01: Total number of data releases

\end{itemize}

\paragraph{Test Items}\mbox{}\\

Demonstrate that Data Access Services are capable of scaling to serve
data from nDRTot (11) data releases over a surveyYears (10) year survey.








\paragraph{Test Procedure}\mbox{}\\
\begin{tabular}{p{4cm}p{12cm}}
\toprule
Step 1
& Description \\ \hline
\end{tabular}
{\scriptsize
Delegate to LSP

}
\begin{tabular}{p{3cm}p{13cm}}
\hline
            & Expected Result \\ \hline
\end{tabular}

\subsubsection{LVV-T164 - Verify implementation of Operations Subsets}\label{lvv-t164}

\begin{longtable}[]{llllll}
\toprule
Version & Status & Priority & Verification Type & Owner
\\\midrule
1 & Draft & Normal &
Test & Robert Gruendl
\\\bottomrule
\multicolumn{6}{c}{ Open \href{https://jira.lsstcorp.org/secure/Tests.jspa\#/testCase/LVV-T164}{LVV-T164} in Jira } \\
\end{longtable}

\paragraph{Verification Elements}\mbox{}\\

\begin{itemize}
\item \href{https://jira.lsstcorp.org/browse/LVV-191}{LVV-191} - DMS-REQ-0365-V-01: Operations Subsets

\end{itemize}

\paragraph{Test Items}\mbox{}\\

Demonstrate that Data Access Services are designed such that subsets of
a Data Release may be retained and served (made available) after a Data
Release has been superseded. ~ (Data Backbone, Managed Database, LSP
Portal, LSP JupyterLab, LSP Web APIs, Parallel Distributed Database)








\paragraph{Test Procedure}\mbox{}\\
\begin{tabular}{p{4cm}p{12cm}}
\toprule
Step 1
& Description \\ \hline
\end{tabular}
{\scriptsize
Delegate to LSP

}
\begin{tabular}{p{3cm}p{13cm}}
\hline
            & Expected Result \\ \hline
\end{tabular}

\subsubsection{LVV-T165 - Verify implementation of Subsets Support}\label{lvv-t165}

\begin{longtable}[]{llllll}
\toprule
Version & Status & Priority & Verification Type & Owner
\\\midrule
1 & Draft & Normal &
Test & Robert Lupton
\\\bottomrule
\multicolumn{6}{c}{ Open \href{https://jira.lsstcorp.org/secure/Tests.jspa\#/testCase/LVV-T165}{LVV-T165} in Jira } \\
\end{longtable}

\paragraph{Verification Elements}\mbox{}\\

\begin{itemize}
\item \href{https://jira.lsstcorp.org/browse/LVV-192}{LVV-192} - DMS-REQ-0366-V-01: Subsets Support

\end{itemize}

\paragraph{Test Items}\mbox{}\\

Verify that the DMS can provide designated subsets of previous Data
Releases.








\paragraph{Test Procedure}\mbox{}\\
\begin{tabular}{p{4cm}p{12cm}}
\toprule
Step 1
& Description \\ \hline
\end{tabular}
{\scriptsize
Delegate to LSP

}
\begin{tabular}{p{3cm}p{13cm}}
\hline
            & Expected Result \\ \hline
\end{tabular}

\subsubsection{LVV-T166 - Verify implementation of Access Services Performance}\label{lvv-t166}

\begin{longtable}[]{llllll}
\toprule
Version & Status & Priority & Verification Type & Owner
\\\midrule
1 & Draft & Normal &
Test & Robert Gruendl
\\\bottomrule
\multicolumn{6}{c}{ Open \href{https://jira.lsstcorp.org/secure/Tests.jspa\#/testCase/LVV-T166}{LVV-T166} in Jira } \\
\end{longtable}

\paragraph{Verification Elements}\mbox{}\\

\begin{itemize}
\item \href{https://jira.lsstcorp.org/browse/LVV-193}{LVV-193} - DMS-REQ-0367-V-01: Access Services Performance

\end{itemize}

\paragraph{Test Items}\mbox{}\\

Demonstrate monitoring of Data Access Services that give real and
long-time views of system performance and usage.








\paragraph{Test Procedure}\mbox{}\\
\begin{tabular}{p{4cm}p{12cm}}
\toprule
Step 1
& Description \\ \hline
\end{tabular}
{\scriptsize
Delegate to LSP

}
\begin{tabular}{p{3cm}p{13cm}}
\hline
            & Expected Result \\ \hline
\end{tabular}

\subsubsection{LVV-T167 - Verify Capability to serve older Data Releases at Full Performance}\label{lvv-t167}

\begin{longtable}[]{llllll}
\toprule
Version & Status & Priority & Verification Type & Owner
\\\midrule
1 & Draft & Normal &
Test & Robert Gruendl
\\\bottomrule
\multicolumn{6}{c}{ Open \href{https://jira.lsstcorp.org/secure/Tests.jspa\#/testCase/LVV-T167}{LVV-T167} in Jira } \\
\end{longtable}

\paragraph{Verification Elements}\mbox{}\\

\begin{itemize}
\item \href{https://jira.lsstcorp.org/browse/LVV-194}{LVV-194} - DMS-REQ-0368-V-01: Implementation Provisions

\end{itemize}

\paragraph{Test Items}\mbox{}\\

Verify that implementation of the data access services do not preclude
serving all older Data Releases with the same performance requirements
as current Data Releases. ~Note that it is an operational consideration
whether sufficient compute and storage resources would actually be
provisioned to meet those requirements.








\paragraph{Test Procedure}\mbox{}\\
\begin{tabular}{p{4cm}p{12cm}}
\toprule
Step 1
& Description \\ \hline
\end{tabular}
{\scriptsize
Delegate to LSP

}
\begin{tabular}{p{3cm}p{13cm}}
\hline
            & Expected Result \\ \hline
\end{tabular}

\subsubsection{LVV-T168 - Verify design of Data Access Services allows Evolution of the LSST Data
Model}\label{lvv-t168}

\begin{longtable}[]{llllll}
\toprule
Version & Status & Priority & Verification Type & Owner
\\\midrule
1 & Draft & Normal &
Test & Robert Gruendl
\\\bottomrule
\multicolumn{6}{c}{ Open \href{https://jira.lsstcorp.org/secure/Tests.jspa\#/testCase/LVV-T168}{LVV-T168} in Jira } \\
\end{longtable}

\paragraph{Verification Elements}\mbox{}\\

\begin{itemize}
\item \href{https://jira.lsstcorp.org/browse/LVV-195}{LVV-195} - DMS-REQ-0369-V-01: Evolution

\end{itemize}

\paragraph{Test Items}\mbox{}\\

Verify that the design of the Data Access Services are able to
accommodate changes/evolution of the LSST data model from one release to
another.








\paragraph{Test Procedure}\mbox{}\\
\begin{tabular}{p{4cm}p{12cm}}
\toprule
Step 1
& Description \\ \hline
\end{tabular}
{\scriptsize
Delegate to LSP

}
\begin{tabular}{p{3cm}p{13cm}}
\hline
            & Expected Result \\ \hline
\end{tabular}

\subsubsection{LVV-T169 - Verify implementation of Older Release Behavior}\label{lvv-t169}

\begin{longtable}[]{llllll}
\toprule
Version & Status & Priority & Verification Type & Owner
\\\midrule
1 & Draft & Normal &
Test & Gregory Dubois-Felsmann
\\\bottomrule
\multicolumn{6}{c}{ Open \href{https://jira.lsstcorp.org/secure/Tests.jspa\#/testCase/LVV-T169}{LVV-T169} in Jira } \\
\end{longtable}

\paragraph{Verification Elements}\mbox{}\\

\begin{itemize}
\item \href{https://jira.lsstcorp.org/browse/LVV-196}{LVV-196} - DMS-REQ-0370-V-01: Older Release Behavior

\end{itemize}

\paragraph{Test Items}\mbox{}\\

Verify that the components of the data access system are technically
capable of handling data releases beyond the two for which full services
are required. ~DMS-REQ-0364 requires that up to 11 be supported.
~Verified by inspection, i.e., by determination that the system design
and implementation contain the necessary features to support this number
of releases, and by direct test in a synthetic test environment with
multiple releases.\\
(Involves: Data Backbone, Managed Database, LSP Portal, LSP JupyterLab,
LSP Web APIs, Parallel Distributed Database)








\paragraph{Test Procedure}\mbox{}\\
\begin{tabular}{p{4cm}p{12cm}}
\toprule
Step 1
& Description \\ \hline
\end{tabular}
{\scriptsize
Delegate to LSP

}
\begin{tabular}{p{3cm}p{13cm}}
\hline
            & Expected Result \\ \hline
\end{tabular}

\subsubsection{LVV-T170 - Verify implementation of Query Availability}\label{lvv-t170}

\begin{longtable}[]{llllll}
\toprule
Version & Status & Priority & Verification Type & Owner
\\\midrule
1 & Draft & Normal &
Test & Colin Slater
\\\bottomrule
\multicolumn{6}{c}{ Open \href{https://jira.lsstcorp.org/secure/Tests.jspa\#/testCase/LVV-T170}{LVV-T170} in Jira } \\
\end{longtable}

\paragraph{Verification Elements}\mbox{}\\

\begin{itemize}
\item \href{https://jira.lsstcorp.org/browse/LVV-197}{LVV-197} - DMS-REQ-0371-V-01: Query Availability

\end{itemize}

\paragraph{Test Items}\mbox{}\\

Verify that queries continue to be successfully executable over time.








\paragraph{Test Procedure}\mbox{}\\
\begin{tabular}{p{4cm}p{12cm}}
\toprule
Step 1
& Description \\ \hline
\end{tabular}
{\scriptsize
Delegate to LSP

}
\begin{tabular}{p{3cm}p{13cm}}
\hline
            & Expected Result \\ \hline
\end{tabular}

\subsubsection{LVV-T171 - Verify implementation of Pipeline Availability}\label{lvv-t171}

\begin{longtable}[]{llllll}
\toprule
Version & Status & Priority & Verification Type & Owner
\\\midrule
1 & Draft & Normal &
Test & Robert Gruendl
\\\bottomrule
\multicolumn{6}{c}{ Open \href{https://jira.lsstcorp.org/secure/Tests.jspa\#/testCase/LVV-T171}{LVV-T171} in Jira } \\
\end{longtable}

\paragraph{Verification Elements}\mbox{}\\

\begin{itemize}
\item \href{https://jira.lsstcorp.org/browse/LVV-5}{LVV-5} - DMS-REQ-0008-V-01: Pipeline Availability

\end{itemize}

\paragraph{Test Items}\mbox{}\\

Demonstrate that Data Management System pipelines are available for use
without disruptions of greater than productionMaxDowntime (24 hours). ~
This requires a regimented change control process and testing
infrastructure for all pipelines and their underlying software services,
and regimented management and monitoring of compute and networking
resources. ~The list of services covered by this test include: Image and
EFD Archiving, Prompt Processing, OCS Driven Batch, Telemetry Gateway,
Alert Distribution, Alert Filtering, Batch Production, Data Backbone,
Compute/Storage/LAN, Inter-Site Networks, and Service Management and
Monitoring.








\paragraph{Test Procedure}\mbox{}\\
\begin{tabular}{p{4cm}p{12cm}}
\toprule
Step 1
& Description \\ \hline
\end{tabular}
{\scriptsize
Analyze sources of downtime after mitigation to compute estimated
reliability; observe unscheduled downtime of developer, integration, and
pre-production systems

}
\begin{tabular}{p{3cm}p{13cm}}
\hline
            & Expected Result \\ \hline
\end{tabular}

\subsubsection{LVV-T172 - Verify implementation of Optimization of Cost, Reliability and
Availability}\label{lvv-t172}

\begin{longtable}[]{llllll}
\toprule
Version & Status & Priority & Verification Type & Owner
\\\midrule
1 & Draft & Normal &
Test & Robert Gruendl
\\\bottomrule
\multicolumn{6}{c}{ Open \href{https://jira.lsstcorp.org/secure/Tests.jspa\#/testCase/LVV-T172}{LVV-T172} in Jira } \\
\end{longtable}

\paragraph{Verification Elements}\mbox{}\\

\begin{itemize}
\item \href{https://jira.lsstcorp.org/browse/LVV-64}{LVV-64} - DMS-REQ-0161-V-01: Optimization of Cost, Reliability and Availability in
Order

\end{itemize}

\paragraph{Test Items}\mbox{}\\

In matters of cost, system reliability (functioning properly at a given
time) has precedence over system availability (ability to use the system
at a given time). ~ The optimization may be outside the realm of direct
testing as it is more of a system provisioning guideline but on its face
it demands that the Data Management System include failure reporting,
regimented change control, acceptance testing, maintenance and
monitoring.








\paragraph{Test Procedure}\mbox{}\\
\begin{tabular}{p{4cm}p{12cm}}
\toprule
Step 1
& Description \\ \hline
\end{tabular}
{\scriptsize
Analyze resource management policy

}
\begin{tabular}{p{3cm}p{13cm}}
\hline
            & Expected Result \\ \hline
\end{tabular}

\subsubsection{LVV-T173 - Verify implementation of Pipeline Throughput}\label{lvv-t173}

\begin{longtable}[]{llllll}
\toprule
Version & Status & Priority & Verification Type & Owner
\\\midrule
1 & Draft & Normal &
Test & Robert Gruendl
\\\bottomrule
\multicolumn{6}{c}{ Open \href{https://jira.lsstcorp.org/secure/Tests.jspa\#/testCase/LVV-T173}{LVV-T173} in Jira } \\
\end{longtable}

\paragraph{Verification Elements}\mbox{}\\

\begin{itemize}
\item \href{https://jira.lsstcorp.org/browse/LVV-65}{LVV-65} - DMS-REQ-0162-V-01: Pipeline Throughput

\end{itemize}

\paragraph{Test Items}\mbox{}\\

Demonstrate that the Alert Production Pipeline is capable of processing
nRawExpNightMax (2800) science exposures within a (24-nightDurationMax)
12 hour period and issue alerts in offline batch mode.~








\paragraph{Test Procedure}\mbox{}\\
\begin{tabular}{p{4cm}p{12cm}}
\toprule
Step 1
& Description \\ \hline
\end{tabular}
{\scriptsize
Execute single-day operations rehearsal, observe data products generated
in time

}
\begin{tabular}{p{3cm}p{13cm}}
\hline
            & Expected Result \\ \hline
\end{tabular}

\subsubsection{LVV-T174 - Verify implementation of Re-processing Capacity}\label{lvv-t174}

\begin{longtable}[]{llllll}
\toprule
Version & Status & Priority & Verification Type & Owner
\\\midrule
1 & Draft & Normal &
Test & Robert Gruendl
\\\bottomrule
\multicolumn{6}{c}{ Open \href{https://jira.lsstcorp.org/secure/Tests.jspa\#/testCase/LVV-T174}{LVV-T174} in Jira } \\
\end{longtable}

\paragraph{Verification Elements}\mbox{}\\

\begin{itemize}
\item \href{https://jira.lsstcorp.org/browse/LVV-66}{LVV-66} - DMS-REQ-0163-V-01: Re-processing Capacity

\end{itemize}

\paragraph{Test Items}\mbox{}\\

Verify that the DMS has sufficient processing, storage, and network to
reprocess all data within ``drProcessingPeriod'' (1 year) while
maintaining full Prompt Processing capability.








\paragraph{Test Procedure}\mbox{}\\
\begin{tabular}{p{4cm}p{12cm}}
\toprule
Step 1
& Description \\ \hline
\end{tabular}
{\scriptsize
~Analyze sizing model; execute DRP, observe scaling

}
\begin{tabular}{p{3cm}p{13cm}}
\hline
            & Expected Result \\ \hline
\end{tabular}

\subsubsection{LVV-T175 - Verify implementation of Temporary Storage for Communications Links}\label{lvv-t175}

\begin{longtable}[]{llllll}
\toprule
Version & Status & Priority & Verification Type & Owner
\\\midrule
1 & Draft & Normal &
Test & Robert Gruendl
\\\bottomrule
\multicolumn{6}{c}{ Open \href{https://jira.lsstcorp.org/secure/Tests.jspa\#/testCase/LVV-T175}{LVV-T175} in Jira } \\
\end{longtable}

\paragraph{Verification Elements}\mbox{}\\

\begin{itemize}
\item \href{https://jira.lsstcorp.org/browse/LVV-67}{LVV-67} - DMS-REQ-0164-V-01: Temporary Storage for Communications Links

\end{itemize}

\paragraph{Test Items}\mbox{}\\

Demonstrate that storage capacity is present and usable to prevent data
loss if networking is interrupted between summit and base, base and
archive, or archive and DAC. ~The requirement is to have storage
necessary to hold tempStorageReIMTTR (200\%) of the expected raw data
that would arrive during the Mean Time to Repair (summToBaseNetMTTR = 24
hours, baseToArchNetMTTR = 48 hours, ~archToDacNetMTTR = 48 hours).
~This scale is further set by nCalibExpDay + nRawExpNightMax = 450 +
2800 = ~3250 exposures/day.








\paragraph{Test Procedure}\mbox{}\\
\begin{tabular}{p{4cm}p{12cm}}
\toprule
Step 1
& Description \\ \hline
\end{tabular}
{\scriptsize
Analyze sizing model and network/storage design

}
\begin{tabular}{p{3cm}p{13cm}}
\hline
            & Expected Result \\ \hline
\end{tabular}

\subsubsection{LVV-T176 - Verify implementation of Infrastructure Sizing for ``catching up''}\label{lvv-t176}

\begin{longtable}[]{llllll}
\toprule
Version & Status & Priority & Verification Type & Owner
\\\midrule
1 & Draft & Normal &
Test & Robert Gruendl
\\\bottomrule
\multicolumn{6}{c}{ Open \href{https://jira.lsstcorp.org/secure/Tests.jspa\#/testCase/LVV-T176}{LVV-T176} in Jira } \\
\end{longtable}

\paragraph{Verification Elements}\mbox{}\\

\begin{itemize}
\item \href{https://jira.lsstcorp.org/browse/LVV-68}{LVV-68} - DMS-REQ-0165-V-01: Infrastructure Sizing for ``catching up''

\item \href{https://jira.lsstcorp.org/browse/LVV-994}{LVV-994} - OSS-REQ-0051-V-01: Summit-Base Connectivity Loss

\end{itemize}

\paragraph{Test Items}\mbox{}\\

Demonstrate Data Management System has sufficient excess capacity
(compute infrastructure) to process one night's data (2800 exposures)
within 24 hours while also maintaining nightly Alert Production (note
this is very similar to
\href{https://jira.lsstcorp.org/secure/Tests.jspa\#/testCase/LVV-T173}{LVV-T173}).~








\paragraph{Test Procedure}\mbox{}\\
\begin{tabular}{p{4cm}p{12cm}}
\toprule
Step 1
& Description \\ \hline
\end{tabular}
{\scriptsize
Execute single-day operations rehearsal including catch-up after
failure, observe data products generated in time

}
\begin{tabular}{p{3cm}p{13cm}}
\hline
            & Expected Result \\ \hline
\end{tabular}

\subsubsection{LVV-T177 - Verify implementation of Incorporate Fault-Tolerance}\label{lvv-t177}

\begin{longtable}[]{llllll}
\toprule
Version & Status & Priority & Verification Type & Owner
\\\midrule
1 & Draft & Normal &
Test & Robert Gruendl
\\\bottomrule
\multicolumn{6}{c}{ Open \href{https://jira.lsstcorp.org/secure/Tests.jspa\#/testCase/LVV-T177}{LVV-T177} in Jira } \\
\end{longtable}

\paragraph{Verification Elements}\mbox{}\\

\begin{itemize}
\item \href{https://jira.lsstcorp.org/browse/LVV-69}{LVV-69} - DMS-REQ-0166-V-01: Incorporate Fault-Tolerance

\end{itemize}

\paragraph{Test Items}\mbox{}\\

Demonstrate that Data Management Systems have features that prevent data
loss. ~Includes: MD5SUM/checksum verification for data transfer; RAID to
eliminate single-point disk failures; multi-site and tape for disaster
recovery of raw data; multiple site (and tape?) for backup/recovery of
Data Release products; DB transaction logging and backup to maintain DB
integrity. ~ (Note: storage to prevent loss in case of networking
failures is covered in
\href{https://jira.lsstcorp.org/secure/Tests.jspa\#/testCase/LVV-T175}{LVV-T175}
). ~~








\paragraph{Test Procedure}\mbox{}\\
\begin{tabular}{p{4cm}p{12cm}}
\toprule
Step 1
& Description \\ \hline
\end{tabular}
{\scriptsize
Analyze design; execute single-day operations rehearsal including
failures, observe recovery without loss of data

}
\begin{tabular}{p{3cm}p{13cm}}
\hline
            & Expected Result \\ \hline
\end{tabular}

\subsubsection{LVV-T178 - Verify implementation of Incorporate Autonomics}\label{lvv-t178}

\begin{longtable}[]{llllll}
\toprule
Version & Status & Priority & Verification Type & Owner
\\\midrule
1 & Draft & Normal &
Test & Robert Gruendl
\\\bottomrule
\multicolumn{6}{c}{ Open \href{https://jira.lsstcorp.org/secure/Tests.jspa\#/testCase/LVV-T178}{LVV-T178} in Jira } \\
\end{longtable}

\paragraph{Verification Elements}\mbox{}\\

\begin{itemize}
\item \href{https://jira.lsstcorp.org/browse/LVV-70}{LVV-70} - DMS-REQ-0167-V-01: Incorporate Autonomics

\end{itemize}

\paragraph{Test Items}\mbox{}\\

Demonstrate that production systems monitor and report faults. ~Where
possible fault mitigation can include re-start, re-submission, or return
of partial products for triage.








\paragraph{Test Procedure}\mbox{}\\
\begin{tabular}{p{4cm}p{12cm}}
\toprule
Step 1
& Description \\ \hline
\end{tabular}
{\scriptsize
Analyze design; execute single-day operations rehearsal including
failures, observe automated recovery and continuation of processing

}
\begin{tabular}{p{3cm}p{13cm}}
\hline
            & Expected Result \\ \hline
\end{tabular}

\subsubsection{LVV-T179 - Verify implementation of Compute Platform Heterogeneity}\label{lvv-t179}

\begin{longtable}[]{llllll}
\toprule
Version & Status & Priority & Verification Type & Owner
\\\midrule
1 & Draft & Normal &
Test & Robert Gruendl
\\\bottomrule
\multicolumn{6}{c}{ Open \href{https://jira.lsstcorp.org/secure/Tests.jspa\#/testCase/LVV-T179}{LVV-T179} in Jira } \\
\end{longtable}

\paragraph{Verification Elements}\mbox{}\\

\begin{itemize}
\item \href{https://jira.lsstcorp.org/browse/LVV-145}{LVV-145} - DMS-REQ-0314-V-01: Compute Platform Heterogeneity

\end{itemize}

\paragraph{Test Items}\mbox{}\\

Demonstrate that production results are the same (within machine
accuracy) when production occurs on different platforms (OS, kernel,
hardware provisioning).








\paragraph{Test Procedure}\mbox{}\\
\begin{tabular}{p{4cm}p{12cm}}
\toprule
Step 1
& Description \\ \hline
\end{tabular}
{\scriptsize
Configure heterogeneous cluster, execute AP+DRP+LSP, observe correct
functioning

}
\begin{tabular}{p{3cm}p{13cm}}
\hline
            & Expected Result \\ \hline
\end{tabular}

\subsubsection{LVV-T180 - Verify implementation of Data Management Unscheduled Downtime}\label{lvv-t180}

\begin{longtable}[]{llllll}
\toprule
Version & Status & Priority & Verification Type & Owner
\\\midrule
1 & Draft & Normal &
Test & Robert Gruendl
\\\bottomrule
\multicolumn{6}{c}{ Open \href{https://jira.lsstcorp.org/secure/Tests.jspa\#/testCase/LVV-T180}{LVV-T180} in Jira } \\
\end{longtable}

\paragraph{Verification Elements}\mbox{}\\

\begin{itemize}
\item \href{https://jira.lsstcorp.org/browse/LVV-149}{LVV-149} - DMS-REQ-0318-V-01: Data Management Unscheduled Downtime

\end{itemize}

\paragraph{Test Items}\mbox{}\\

This applies only to downtime that would prevent the collection of
survey data. ~Verification means that analysis has occurred to identify
likely hardware failures that would prevent survey operations and that
mitigations that minimize the downtime to less than DMDowntime (1
day/year) are in place. ~Known systems that fall in this category
include: Image and EFD Archiving, Observatory Operations Data, Telemetry
Gateway, Data Backbone, Managed Database, Inter-Site Networks, and
Service Management and Monitoring.~








\paragraph{Test Procedure}\mbox{}\\
\begin{tabular}{p{4cm}p{12cm}}
\toprule
Step 1
& Description \\ \hline
\end{tabular}
{\scriptsize
Analyze likely hardware failures with mitigations to compute estimated
unplanned downtime

}
\begin{tabular}{p{3cm}p{13cm}}
\hline
            & Expected Result \\ \hline
\end{tabular}

\subsubsection{LVV-T181 - Verify Base Voice Over IP (VOIP)}\label{lvv-t181}

\begin{longtable}[]{llllll}
\toprule
Version & Status & Priority & Verification Type & Owner
\\\midrule
1 & Draft & Normal &
Test & Jeff Kantor
\\\bottomrule
\multicolumn{6}{c}{ Open \href{https://jira.lsstcorp.org/secure/Tests.jspa\#/testCase/LVV-T181}{LVV-T181} in Jira } \\
\end{longtable}

\paragraph{Verification Elements}\mbox{}\\

\begin{itemize}
\item \href{https://jira.lsstcorp.org/browse/LVV-18491}{LVV-18491} - DMS-REQ-0352-V-02: Base Voice Over IP (VOIP)

\end{itemize}

\paragraph{Test Items}\mbox{}\\

Verify as-built VOIP at the Base Facility is operational and performs as
expected (i.e. sufficient number of extensions allocated properly, no
frequent drop-outs, no frequent jaggies on video, etc.) on both voice
calls and videoconferening.


\paragraph{Predecessors}\mbox{}\\
PMCS DLP-465 Complete\\
PMCS IT-702 Complete

\paragraph{Environment Needs}\mbox{}\\

\subparagraph{Software}\mbox{}\\
See pre-conditions.

\subparagraph{Hardware}\mbox{}\\
See pre-conditions.



\paragraph{Test Procedure}\mbox{}\\
\begin{tabular}{p{4cm}p{12cm}}
\toprule
Step 1
& Description \\ \hline
\end{tabular}
{\scriptsize
Test voice calls over VOIP system from Base Facility to locations in
~Base and to other Rubin Observatory facilities.

}
\begin{tabular}{p{3cm}p{13cm}}
\hline
            & Expected Result \\ \hline
\end{tabular}
{\scriptsize
As-built VOIP at the Base Facility is operational and performs as
expected (i.e. sufficient number of extensions allocated properly, no
frequent drop-outs, etc.).

}

\begin{tabular}{p{4cm}p{12cm}}
\toprule
Step 2
& Description \\ \hline
\end{tabular}
{\scriptsize
Test video conferences over ~system from Base Facility to locations in
Base and to other Rubin Observatory facilities.

}
\begin{tabular}{p{3cm}p{13cm}}
\hline
            & Expected Result \\ \hline
\end{tabular}
{\scriptsize
Verify (a) plannned and (b) as-built VOIP at the Base Facility is
operational and performs as expected (i.e. no frequent drop-outs, no
frequent audio glitches, no frequent jaggies on video, etc.).

}

\subsubsection{LVV-T182 - Verify implementation of Prefer Computing and Storage Down}\label{lvv-t182}

\begin{longtable}[]{llllll}
\toprule
Version & Status & Priority & Verification Type & Owner
\\\midrule
1 & Draft & Normal &
Test & Robert Gruendl
\\\bottomrule
\multicolumn{6}{c}{ Open \href{https://jira.lsstcorp.org/secure/Tests.jspa\#/testCase/LVV-T182}{LVV-T182} in Jira } \\
\end{longtable}

\paragraph{Verification Elements}\mbox{}\\

\begin{itemize}
\item \href{https://jira.lsstcorp.org/browse/LVV-72}{LVV-72} - DMS-REQ-0170-V-01: Prefer Computing and Storage Down

\end{itemize}

\paragraph{Test Items}\mbox{}\\

Only build compute or storage facilities at the summit that are
justified by operational need or to prevent loss of data during
networking downtimes.








\paragraph{Test Procedure}\mbox{}\\
\begin{tabular}{p{4cm}p{12cm}}
\toprule
Step 1
& Description \\ \hline
\end{tabular}
{\scriptsize
Analyze design

}
\begin{tabular}{p{3cm}p{13cm}}
\hline
            & Expected Result \\ \hline
\end{tabular}

\subsubsection{LVV-T185 - Verify implementation of Summit to Base Network Availability}\label{lvv-t185}

\begin{longtable}[]{llllll}
\toprule
Version & Status & Priority & Verification Type & Owner
\\\midrule
1 & Draft & Normal &
Inspection & Jeff Kantor
\\\bottomrule
\multicolumn{6}{c}{ Open \href{https://jira.lsstcorp.org/secure/Tests.jspa\#/testCase/LVV-T185}{LVV-T185} in Jira } \\
\end{longtable}

\paragraph{Verification Elements}\mbox{}\\

\begin{itemize}
\item \href{https://jira.lsstcorp.org/browse/LVV-74}{LVV-74} - DMS-REQ-0172-V-01: Summit to Base Network Availability

\end{itemize}

\paragraph{Test Items}\mbox{}\\

Verify the availability of Summit to Base Network by demonstrating that
the mean time between failures is less than summToBaseNetMTBF (90 days)
over 1 year.


\paragraph{Predecessors}\mbox{}\\
See pre-conditions.

\paragraph{Environment Needs}\mbox{}\\

\subparagraph{Software}\mbox{}\\
See pre-conditions.

\subparagraph{Hardware}\mbox{}\\
See pre-conditions.



\paragraph{Test Procedure}\mbox{}\\
\begin{tabular}{p{4cm}p{12cm}}
\toprule
Step 1
& Description \\ \hline
\end{tabular}
{\scriptsize
Monitor summit to base networking for at least 1 week

}
\begin{tabular}{p{3cm}p{13cm}}
\hline
            & Test Data \\ \hline
\end{tabular}
{\scriptsize
LATISS, ComCAM, and/or Full Camera data.

}
\begin{tabular}{p{3cm}p{13cm}}
\hline
            & Expected Result \\ \hline
\end{tabular}
{\scriptsize
Summit - base network is operational for 1 week and monitoring data is
collected.

}

\begin{tabular}{p{4cm}p{12cm}}
\toprule
Step 2
& Description \\ \hline
\end{tabular}
{\scriptsize
Extrapolate annual availability, compare with at least 6 months of
historical data on the link.

}
\begin{tabular}{p{3cm}p{13cm}}
\hline
            & Test Data \\ \hline
\end{tabular}
{\scriptsize
Historical and current logs

}
\begin{tabular}{p{3cm}p{13cm}}
\hline
            & Expected Result \\ \hline
\end{tabular}
{\scriptsize
The mean time between failures (MTBF) is projected to be less than
summToBaseNetMTBF (90 days) over 1 year.

}

\subsubsection{LVV-T186 - Verify implementation of Summit to Base Network Reliability}\label{lvv-t186}

\begin{longtable}[]{llllll}
\toprule
Version & Status & Priority & Verification Type & Owner
\\\midrule
1 & Draft & Normal &
Demonstration & Jeff Kantor
\\\bottomrule
\multicolumn{6}{c}{ Open \href{https://jira.lsstcorp.org/secure/Tests.jspa\#/testCase/LVV-T186}{LVV-T186} in Jira } \\
\end{longtable}

\paragraph{Verification Elements}\mbox{}\\

\begin{itemize}
\item \href{https://jira.lsstcorp.org/browse/LVV-75}{LVV-75} - DMS-REQ-0173-V-01: Summit to Base Network Reliability

\end{itemize}

\paragraph{Test Items}\mbox{}\\

Verify the reliability of the summit to base network by demonstrating
reconnection and recovery to transfer of data at or exceeding rates
specified in \citeds{LDM-142} following a cut in network connection, within MTTR
specification. The network operator will provide MTTR data on links
during commissioning and operations.\\[2\baselineskip]


\paragraph{Predecessors}\mbox{}\\
See pre-conditions.

\paragraph{Environment Needs}\mbox{}\\

\subparagraph{Software}\mbox{}\\
See pre-conditions.

\subparagraph{Hardware}\mbox{}\\
See pre-conditions.



\paragraph{Test Procedure}\mbox{}\\
\begin{tabular}{p{4cm}p{12cm}}
\toprule
Step 1
& Description \\ \hline
\end{tabular}
{\scriptsize
Disconnect fiber cable at an endpoint location on the base side of the
Summit - Base fiber.

}
\begin{tabular}{p{3cm}p{13cm}}
\hline
            & Test Data \\ \hline
\end{tabular}
{\scriptsize
LATISS, ComCAM, or FullCam data

}
\begin{tabular}{p{3cm}p{13cm}}
\hline
            & Expected Result \\ \hline
\end{tabular}
{\scriptsize
Fiber is disconnected and the fault is detected by the network
monitoring system.

}

\begin{tabular}{p{4cm}p{12cm}}
\toprule
Step 2
& Description \\ \hline
\end{tabular}
{\scriptsize
Measure the cable with the OTDR to locate the distance from the end
point. Diagnose that it is a break.

}
\begin{tabular}{p{3cm}p{13cm}}
\hline
            & Test Data \\ \hline
\end{tabular}
{\scriptsize
NA

}
\begin{tabular}{p{3cm}p{13cm}}
\hline
            & Expected Result \\ \hline
\end{tabular}
{\scriptsize
OTDR shows the fiber is disconnected (break).

}

\begin{tabular}{p{4cm}p{12cm}}
\toprule
Step 3
& Description \\ \hline
\end{tabular}
{\scriptsize
Elapse time to simulate the following:

\begin{itemize}
\tightlist
\item
  Go to the most inaccessible place which would mean carrying all the
  tools/splicer/generator/tent equipment some ~metres.
\item
  Erect a tent to make the splice
\item
  Start the generator
\item
  Do a splice on some random piece of cable
\item
  At an end point measure the cable again to ensure it is break free.
\item
  Take down and reinstall an isolated pole (not in the actual fiber
  path)
\item
  Put the cable on the pole.
\end{itemize}

}
\begin{tabular}{p{3cm}p{13cm}}
\hline
            & Test Data \\ \hline
\end{tabular}
{\scriptsize
NA

}
\begin{tabular}{p{3cm}p{13cm}}
\hline
            & Expected Result \\ \hline
\end{tabular}
{\scriptsize
Wall clock advances by 24 hours.

}

\begin{tabular}{p{4cm}p{12cm}}
\toprule
Step 4
& Description \\ \hline
\end{tabular}
{\scriptsize
Clean fiber connections. ~Restore connection (e.g. reconnect cable).
~Cycle equipment as necessary to confirm fiber is connected.

}
\begin{tabular}{p{3cm}p{13cm}}
\hline
            & Test Data \\ \hline
\end{tabular}
{\scriptsize
NA

}
\begin{tabular}{p{3cm}p{13cm}}
\hline
            & Expected Result \\ \hline
\end{tabular}
{\scriptsize
Network recovers and resumes sending data.

}

\begin{tabular}{p{4cm}p{12cm}}
\toprule
Step 5
& Description \\ \hline
\end{tabular}
{\scriptsize
Measure with OTDR to ensure back to normal state.

}
\begin{tabular}{p{3cm}p{13cm}}
\hline
            & Test Data \\ \hline
\end{tabular}
{\scriptsize
NA

}
\begin{tabular}{p{3cm}p{13cm}}
\hline
            & Expected Result \\ \hline
\end{tabular}
{\scriptsize
OTDR indicates normal state.

}

\subsubsection{LVV-T187 - Verify implementation of Summit to Base Network Secondary Link}\label{lvv-t187}

\begin{longtable}[]{llllll}
\toprule
Version & Status & Priority & Verification Type & Owner
\\\midrule
1 & Draft & Normal &
Test & Jeff Kantor
\\\bottomrule
\multicolumn{6}{c}{ Open \href{https://jira.lsstcorp.org/secure/Tests.jspa\#/testCase/LVV-T187}{LVV-T187} in Jira } \\
\end{longtable}

\paragraph{Verification Elements}\mbox{}\\

\begin{itemize}
\item \href{https://jira.lsstcorp.org/browse/LVV-76}{LVV-76} - DMS-REQ-0174-V-01: Summit to Base Network Secondary Link

\end{itemize}

\paragraph{Test Items}\mbox{}\\

Verify automated fail-over from primary to secondary equipment in Rubin
Observatory DWDM on simulated failure of primary. ~Verify bandwidth
sufficiency on secondary. ~Verify automated recovery to primary
equipment on simulated restoration of primary. ~Repeat for failure of
Rubin Observatory fiber and fail-over to AURA fiber and DWDM.
~Demonstrate use of secondary in ``catch-up'' mode.


\paragraph{Predecessors}\mbox{}\\
See pre-conditions.

\paragraph{Environment Needs}\mbox{}\\

\subparagraph{Software}\mbox{}\\
See pre-conditions.

\subparagraph{Hardware}\mbox{}\\
See pre-conditions.



\paragraph{Test Procedure}\mbox{}\\
\begin{tabular}{p{4cm}p{12cm}}
\toprule
Step 1
& Description \\ \hline
\end{tabular}
{\scriptsize
Transfer data between summit and base on primary equipment (LSST Summit
- Base) over uninterrupted 1 day period. ~

}
\begin{tabular}{p{3cm}p{13cm}}
\hline
            & Test Data \\ \hline
\end{tabular}
{\scriptsize
LATISS, ComCAM, or FullCAM data.

}
\begin{tabular}{p{3cm}p{13cm}}
\hline
            & Expected Result \\ \hline
\end{tabular}
{\scriptsize
Normal operations.

}

\begin{tabular}{p{4cm}p{12cm}}
\toprule
Step 2
& Description \\ \hline
\end{tabular}
{\scriptsize
Simulate equipment outage by disconnecting power card from primary DWDM
equipment on base side of Summit - Base Fiber.

}
\begin{tabular}{p{3cm}p{13cm}}
\hline
            & Test Data \\ \hline
\end{tabular}
{\scriptsize
NA

}
\begin{tabular}{p{3cm}p{13cm}}
\hline
            & Expected Result \\ \hline
\end{tabular}
{\scriptsize
Network fails over to secondary equipment in \textless{}=60s.

}

\begin{tabular}{p{4cm}p{12cm}}
\toprule
Step 3
& Description \\ \hline
\end{tabular}
{\scriptsize
Transfer data between summit and base over secondary equipment
uninterrupted 1 day period while monitoring network.

}
\begin{tabular}{p{3cm}p{13cm}}
\hline
            & Test Data \\ \hline
\end{tabular}
{\scriptsize
NA

}
\begin{tabular}{p{3cm}p{13cm}}
\hline
            & Expected Result \\ \hline
\end{tabular}
{\scriptsize
Verify that secondary equipment is capable of transferring 1 night of
raw data (nCalibExpDay + nRawExpNightMax = 450 + 2800 = ~3250 exposures)
within summToBaseNet2TransMax (72 hours), i.e. at or exceeding rates
specified in LDM-142.

}

\begin{tabular}{p{4cm}p{12cm}}
\toprule
Step 4
& Description \\ \hline
\end{tabular}
{\scriptsize
Restore ~primary equipment (i.e. reconnect power card to primary
equipment.)

}
\begin{tabular}{p{3cm}p{13cm}}
\hline
            & Test Data \\ \hline
\end{tabular}
{\scriptsize
NA

}
\begin{tabular}{p{3cm}p{13cm}}
\hline
            & Expected Result \\ \hline
\end{tabular}
{\scriptsize
Network recovers to primary in \textless{}= 60s.

}

\begin{tabular}{p{4cm}p{12cm}}
\toprule
Step 5
& Description \\ \hline
\end{tabular}
{\scriptsize
Simulate fiber outage by disconnecting fiber from primary DWDM equipment
on base side of Summit - Base Fiber.

}
\begin{tabular}{p{3cm}p{13cm}}
\hline
            & Test Data \\ \hline
\end{tabular}
{\scriptsize
NA

}
\begin{tabular}{p{3cm}p{13cm}}
\hline
            & Expected Result \\ \hline
\end{tabular}
{\scriptsize
Network fails over to AURA DWDM and fiber.

}

\begin{tabular}{p{4cm}p{12cm}}
\toprule
Step 6
& Description \\ \hline
\end{tabular}
{\scriptsize
Transfer data between summit and base over AURA fiber and equipment
uninterrupted 1 day period while monitoring network.

}
\begin{tabular}{p{3cm}p{13cm}}
\hline
            & Test Data \\ \hline
\end{tabular}
{\scriptsize
LATISS, ComCAM, or FullCAM data.

}
\begin{tabular}{p{3cm}p{13cm}}
\hline
            & Expected Result \\ \hline
\end{tabular}
{\scriptsize
Verify that AURA fiber and equipment is capable of transferring 1 night
of raw data (nCalibExpDay + nRawExpNightMax = 450 + 2800 = ~3250
exposures) within summToBaseNet2TransMax (72 hours), i.e. at or
exceeding rates specified in LDM-142.

}

\begin{tabular}{p{4cm}p{12cm}}
\toprule
Step 7
& Description \\ \hline
\end{tabular}
{\scriptsize
Restore ~primary fiber (i.e. reconnect fiber to Rubin Observatory DWDM
equipment.)

}
\begin{tabular}{p{3cm}p{13cm}}
\hline
            & Expected Result \\ \hline
\end{tabular}
{\scriptsize
Network recovers to Rubin Observatory fiber and DWDM.

}

\begin{tabular}{p{4cm}p{12cm}}
\toprule
Step 8
& Description \\ \hline
\end{tabular}
{\scriptsize
Demonstrate use of secondary in ``catch-up'' mode.

}
\begin{tabular}{p{3cm}p{13cm}}
\hline
            & Test Data \\ \hline
\end{tabular}
{\scriptsize
DAQ data buffer full of images and associated meta-data

}
\begin{tabular}{p{3cm}p{13cm}}
\hline
            & Expected Result \\ \hline
\end{tabular}
{\scriptsize
Images from DAQ buffer and associated metadata are retrievable over
secondary path while current observing data is being transferred over
primary path.

}

\subsubsection{LVV-T188 - Verify implementation of Summit to Base Network Ownership and Operation}\label{lvv-t188}

\begin{longtable}[]{llllll}
\toprule
Version & Status & Priority & Verification Type & Owner
\\\midrule
1 & Draft & Normal &
Inspection & Jeff Kantor
\\\bottomrule
\multicolumn{6}{c}{ Open \href{https://jira.lsstcorp.org/secure/Tests.jspa\#/testCase/LVV-T188}{LVV-T188} in Jira } \\
\end{longtable}

\paragraph{Verification Elements}\mbox{}\\

\begin{itemize}
\item \href{https://jira.lsstcorp.org/browse/LVV-77}{LVV-77} - DMS-REQ-0175-V-01: Summit to Base Network Ownership and Operation

\end{itemize}

\paragraph{Test Items}\mbox{}\\

Verify Summit to Base Network Ownership and Operation by LSST and/or the
operations entity by inspection of construction and operations contracts
and Indefeasible Rights.


\paragraph{Predecessors}\mbox{}\\
PMCS DMTC-7400-2140, -2240, -2330 Complete

\paragraph{Environment Needs}\mbox{}\\

\subparagraph{Software}\mbox{}\\
None

\subparagraph{Hardware}\mbox{}\\
None



\paragraph{Test Procedure}\mbox{}\\
\begin{tabular}{p{4cm}p{12cm}}
\toprule
Step 1
& Description \\ \hline
\end{tabular}
{\scriptsize
Examine contracts with REUNA and telefonica for fiber ownership and
maintenance terms.

}
\begin{tabular}{p{3cm}p{13cm}}
\hline
            & Expected Result \\ \hline
\end{tabular}
{\scriptsize
Rubin Observatory is owner of fibers on AURA property and Summit - Base
DWDM ~and has 15-year IRU for use of fibers on all segments. ~REUNA is
owner of LS - SCL DWDM on AURA property and in Santiago, and is operator
on all fibers and DWDM. ~Telefonica is contracted to maintain fibers not
on AURA property.

}

\subsubsection{LVV-T189 - Verify implementation of Base Facility Infrastructure}\label{lvv-t189}

\begin{longtable}[]{llllll}
\toprule
Version & Status & Priority & Verification Type & Owner
\\\midrule
1 & Draft & Normal &
Test & Robert Gruendl
\\\bottomrule
\multicolumn{6}{c}{ Open \href{https://jira.lsstcorp.org/secure/Tests.jspa\#/testCase/LVV-T189}{LVV-T189} in Jira } \\
\end{longtable}

\paragraph{Verification Elements}\mbox{}\\

\begin{itemize}
\item \href{https://jira.lsstcorp.org/browse/LVV-78}{LVV-78} - DMS-REQ-0176-V-01: Base Facility Infrastructure

\end{itemize}

\paragraph{Test Items}\mbox{}\\

Verify that the (a) planned infrastructure and (b) as-built
infrastructure for the Base Facility satisfies the needs for data
transfer and buffering, a copy of the Archive Facility, and support for
Commissioning.








\paragraph{Test Procedure}\mbox{}\\
\begin{tabular}{p{4cm}p{12cm}}
\toprule
Step 1
& Description \\ \hline
\end{tabular}
{\scriptsize
Analyze design and sizing model

}
\begin{tabular}{p{3cm}p{13cm}}
\hline
            & Expected Result \\ \hline
\end{tabular}

\subsubsection{LVV-T190 - Verify implementation of Base Facility Co-Location with Existing
Facility}\label{lvv-t190}

\begin{longtable}[]{llllll}
\toprule
Version & Status & Priority & Verification Type & Owner
\\\midrule
1 & Draft & Normal &
Test & Robert Gruendl
\\\bottomrule
\multicolumn{6}{c}{ Open \href{https://jira.lsstcorp.org/secure/Tests.jspa\#/testCase/LVV-T190}{LVV-T190} in Jira } \\
\end{longtable}

\paragraph{Verification Elements}\mbox{}\\

\begin{itemize}
\item \href{https://jira.lsstcorp.org/browse/LVV-80}{LVV-80} - DMS-REQ-0178-V-01: Base Facility Co-Location with Existing Facility

\end{itemize}

\paragraph{Test Items}\mbox{}\\

Verify that the Base Facility is located at an existing known supported
facility.








\paragraph{Test Procedure}\mbox{}\\
\begin{tabular}{p{4cm}p{12cm}}
\toprule
Step 1
& Description \\ \hline
\end{tabular}
{\scriptsize
Analyze design

}
\begin{tabular}{p{3cm}p{13cm}}
\hline
            & Expected Result \\ \hline
\end{tabular}

\subsubsection{LVV-T191 - Verify implementation of Commissioning Cluster}\label{lvv-t191}

\begin{longtable}[]{llllll}
\toprule
Version & Status & Priority & Verification Type & Owner
\\\midrule
1 & Draft & Normal &
Test & Robert Gruendl
\\\bottomrule
\multicolumn{6}{c}{ Open \href{https://jira.lsstcorp.org/secure/Tests.jspa\#/testCase/LVV-T191}{LVV-T191} in Jira } \\
\end{longtable}

\paragraph{Verification Elements}\mbox{}\\

\begin{itemize}
\item \href{https://jira.lsstcorp.org/browse/LVV-147}{LVV-147} - DMS-REQ-0316-V-01: Commissioning Cluster

\end{itemize}

\paragraph{Test Items}\mbox{}\\

Verify that the Commissioning Cluster has sufficient Compute/Storage/LAN
at the Base Facility to support Commissioning.








\paragraph{Test Procedure}\mbox{}\\
\begin{tabular}{p{4cm}p{12cm}}
\toprule
Step 1
& Description \\ \hline
\end{tabular}
{\scriptsize
Analyze design and budget

}
\begin{tabular}{p{3cm}p{13cm}}
\hline
            & Expected Result \\ \hline
\end{tabular}

\subsubsection{LVV-T192 - Verify implementation of Base Wireless LAN (WiFi)}\label{lvv-t192}

\begin{longtable}[]{llllll}
\toprule
Version & Status & Priority & Verification Type & Owner
\\\midrule
1 & Draft & Normal &
Test & Jeff Kantor
\\\bottomrule
\multicolumn{6}{c}{ Open \href{https://jira.lsstcorp.org/secure/Tests.jspa\#/testCase/LVV-T192}{LVV-T192} in Jira } \\
\end{longtable}

\paragraph{Verification Elements}\mbox{}\\

\begin{itemize}
\item \href{https://jira.lsstcorp.org/browse/LVV-183}{LVV-183} - DMS-REQ-0352-V-01: Base Wireless LAN (WiFi)

\end{itemize}

\paragraph{Test Items}\mbox{}\\

Verify as-built wireless network at the Base Facility supports
minBaseWiFi bandwidth (1000 Mbs).


\paragraph{Predecessors}\mbox{}\\
PMCS DLP-465 Complete.

\paragraph{Environment Needs}\mbox{}\\

\subparagraph{Software}\mbox{}\\
See pre-conditions.

\subparagraph{Hardware}\mbox{}\\
Desktop with WiFi NIC, email reader, internet browser.



\paragraph{Test Procedure}\mbox{}\\
\begin{tabular}{p{4cm}p{12cm}}
\toprule
Step 1
& Description \\ \hline
\end{tabular}
{\scriptsize
Test internet web browsing and file download, email at summit and base
over wireless.

}
\begin{tabular}{p{3cm}p{13cm}}
\hline
            & Test Data \\ \hline
\end{tabular}
{\scriptsize
NA

}
\begin{tabular}{p{3cm}p{13cm}}
\hline
            & Expected Result \\ \hline
\end{tabular}
{\scriptsize
Verify as-built wireless network at the Base Facility supports
minBaseWiFi bandwidth (1000 Mbs). Verify wireless signal strength meets
or exceeds typical, and average and peak bandwidths meet or exceed
minBaseWiFI bandwidth.

}

\subsubsection{LVV-T193 - Verify implementation of Base to Archive Network}\label{lvv-t193}

\begin{longtable}[]{llllll}
\toprule
Version & Status & Priority & Verification Type & Owner
\\\midrule
1 & Draft & Normal &
Test & Jeff Kantor
\\\bottomrule
\multicolumn{6}{c}{ Open \href{https://jira.lsstcorp.org/secure/Tests.jspa\#/testCase/LVV-T193}{LVV-T193} in Jira } \\
\end{longtable}

\paragraph{Verification Elements}\mbox{}\\

\begin{itemize}
\item \href{https://jira.lsstcorp.org/browse/LVV-81}{LVV-81} - DMS-REQ-0180-V-01: Base to Archive Network

\end{itemize}

\paragraph{Test Items}\mbox{}\\

Verify that the data acquired by a DAQ can be transferred within the
required time, i.e. verify that link is capable of transferring image
for prompt processing in oArchiveMaxTransferTime = 5{[}second{]}, i.e.
at or exceeding rates specified in \citeds{LDM-142}.


\paragraph{Predecessors}\mbox{}\\
PMCS DM-Net-5 Complete

\paragraph{Environment Needs}\mbox{}\\

\subparagraph{Software}\mbox{}\\
See pre-conditions.

\subparagraph{Hardware}\mbox{}\\
See pre-conditions.



\paragraph{Test Procedure}\mbox{}\\
\begin{tabular}{p{4cm}p{12cm}}
\toprule
Step 1
& Description \\ \hline
\end{tabular}
{\scriptsize
Transfer data between base and archive while monitoring the network over
uninterrupted 1 day period (with repeated transfers on normal observing
cadence).

}
\begin{tabular}{p{3cm}p{13cm}}
\hline
            & Test Data \\ \hline
\end{tabular}
{\scriptsize
LATISS, ComCAM, or FullCAM data.

}
\begin{tabular}{p{3cm}p{13cm}}
\hline
            & Expected Result \\ \hline
\end{tabular}
{\scriptsize
Data transfers occur without significant delay or frequent latency
spikes.

}

\begin{tabular}{p{4cm}p{12cm}}
\toprule
Step 2
& Description \\ \hline
\end{tabular}
{\scriptsize
~Analyze the network logs and monitoring system to determine average and
peak latency and packet loss statistics.

}
\begin{tabular}{p{3cm}p{13cm}}
\hline
            & Expected Result \\ \hline
\end{tabular}
{\scriptsize
Data can be transferred within the required time, i.e. verify that link
is capable of transferring image for prompt processing in
oArchiveMaxTransferTime = 5{[}second{]}. Verify transfer of data at or
exceeding rates specified in LDM-142 at least 98\% of the time.

}

\subsubsection{LVV-T194 - Verify implementation of Base to Archive Network Availability}\label{lvv-t194}

\begin{longtable}[]{llllll}
\toprule
Version & Status & Priority & Verification Type & Owner
\\\midrule
1 & Draft & Normal &
Test & Jeff Kantor
\\\bottomrule
\multicolumn{6}{c}{ Open \href{https://jira.lsstcorp.org/secure/Tests.jspa\#/testCase/LVV-T194}{LVV-T194} in Jira } \\
\end{longtable}

\paragraph{Verification Elements}\mbox{}\\

\begin{itemize}
\item \href{https://jira.lsstcorp.org/browse/LVV-82}{LVV-82} - DMS-REQ-0181-V-01: Base to Archive Network Availability

\end{itemize}

\paragraph{Test Items}\mbox{}\\

Verify the availability of the Base to Archive Network communications by
demonstrating that it meets or exceeds a mean time between failures,
measured over a 1-yr period of MTBF \textgreater{} baseToArchNetMTBF
(180{[}day{]})


\paragraph{Predecessors}\mbox{}\\
~PMCS DMTC-7400-2130 Complete






\paragraph{Test Procedure}\mbox{}\\
\begin{tabular}{p{4cm}p{12cm}}
\toprule
Step 1
& Description \\ \hline
\end{tabular}
{\scriptsize
Transfer data between base and archive over uninterrupted 1 week period.

}
\begin{tabular}{p{3cm}p{13cm}}
\hline
            & Test Data \\ \hline
\end{tabular}
{\scriptsize
LATISS, ComCAM, or FullCAM data.

}
\begin{tabular}{p{3cm}p{13cm}}
\hline
            & Expected Result \\ \hline
\end{tabular}
{\scriptsize
Data is successfully transferred during the entire week.

}

\begin{tabular}{p{4cm}p{12cm}}
\toprule
Step 2
& Description \\ \hline
\end{tabular}
{\scriptsize
Analyze monitoring/performance data, compare to historical data, and
extrapolate to a full year, average and peak throughput and latency.

}
\begin{tabular}{p{3cm}p{13cm}}
\hline
            & Test Data \\ \hline
\end{tabular}
{\scriptsize
NA

}
\begin{tabular}{p{3cm}p{13cm}}
\hline
            & Expected Result \\ \hline
\end{tabular}
{\scriptsize
Extrapolated network availability meets baseToArchNetMTBF =
180{[}day{]}. ~Note that this is for complete loss of transfer service
(all paths), not a single path failure with successful fail-over.

}

\subsubsection{LVV-T195 - Verify implementation of Base to Archive Network Reliability}\label{lvv-t195}

\begin{longtable}[]{llllll}
\toprule
Version & Status & Priority & Verification Type & Owner
\\\midrule
1 & Draft & Normal &
Test & Jeff Kantor
\\\bottomrule
\multicolumn{6}{c}{ Open \href{https://jira.lsstcorp.org/secure/Tests.jspa\#/testCase/LVV-T195}{LVV-T195} in Jira } \\
\end{longtable}

\paragraph{Verification Elements}\mbox{}\\

\begin{itemize}
\item \href{https://jira.lsstcorp.org/browse/LVV-83}{LVV-83} - DMS-REQ-0182-V-01: Base to Archive Network Reliability

\end{itemize}

\paragraph{Test Items}\mbox{}\\

Verify Base to Archive Network Reliability by demonstrating that the
network can recover from outages within baseToArchNetMTTR =
48{[}hour{]}.


\paragraph{Predecessors}\mbox{}\\
PMCS DM-NET-5 Complete

\paragraph{Environment Needs}\mbox{}\\

\subparagraph{Software}\mbox{}\\
See pre-conditions.

\subparagraph{Hardware}\mbox{}\\
See pre-conditions.



\paragraph{Test Procedure}\mbox{}\\
\begin{tabular}{p{4cm}p{12cm}}
\toprule
Step 1
& Description \\ \hline
\end{tabular}
{\scriptsize
Disconnect primary fiber on base side of Base - ~Archive network.

}
\begin{tabular}{p{3cm}p{13cm}}
\hline
            & Test Data \\ \hline
\end{tabular}
{\scriptsize
LATISS, ComCAM, or FullCAM data.

}
\begin{tabular}{p{3cm}p{13cm}}
\hline
            & Expected Result \\ \hline
\end{tabular}
{\scriptsize
Network fails over to secondary path.

}

\begin{tabular}{p{4cm}p{12cm}}
\toprule
Step 2
& Description \\ \hline
\end{tabular}
{\scriptsize
Simulate diagnosis and repair by elapsed time.\\[2\baselineskip]

}
\begin{tabular}{p{3cm}p{13cm}}
\hline
            & Test Data \\ \hline
\end{tabular}
{\scriptsize
NA

}
\begin{tabular}{p{3cm}p{13cm}}
\hline
            & Expected Result \\ \hline
\end{tabular}
{\scriptsize
Wall clock advances by 48 hours. ~Data is successfully transferred over
secondary path.

}

\begin{tabular}{p{4cm}p{12cm}}
\toprule
Step 3
& Description \\ \hline
\end{tabular}
{\scriptsize
Reconnect primary fiber on base side of Base - Archive network.

}
\begin{tabular}{p{3cm}p{13cm}}
\hline
            & Test Data \\ \hline
\end{tabular}
{\scriptsize
NA

}
\begin{tabular}{p{3cm}p{13cm}}
\hline
            & Expected Result \\ \hline
\end{tabular}
{\scriptsize
Network recovers to primary path.~

}

\begin{tabular}{p{4cm}p{12cm}}
\toprule
Step 4
& Description \\ \hline
\end{tabular}
{\scriptsize
Analyze fail-over and recovery times. ~Compare to historical data and
extrapolate to MTTR.

}
\begin{tabular}{p{3cm}p{13cm}}
\hline
            & Expected Result \\ \hline
\end{tabular}
{\scriptsize
Verify recovery can occur within baseToArchNetMTTR = 48{[}hour{]}.
Demonstrate reconnection and recovery to transfer of data at or
exceeding rates specified in LDM-142.

}

\subsubsection{LVV-T196 - Verify implementation of Base to Archive Network Secondary Link}\label{lvv-t196}

\begin{longtable}[]{llllll}
\toprule
Version & Status & Priority & Verification Type & Owner
\\\midrule
1 & Draft & Normal &
Test & Jeff Kantor
\\\bottomrule
\multicolumn{6}{c}{ Open \href{https://jira.lsstcorp.org/secure/Tests.jspa\#/testCase/LVV-T196}{LVV-T196} in Jira } \\
\end{longtable}

\paragraph{Verification Elements}\mbox{}\\

\begin{itemize}
\item \href{https://jira.lsstcorp.org/browse/LVV-84}{LVV-84} - DMS-REQ-0183-V-01: Base to Archive Network Secondary Link

\end{itemize}

\paragraph{Test Items}\mbox{}\\

Verify Base to Archive Network Secondary Link failover and capacity, and
subsequent recovery primary. Demonstrate the use of the secondary path
in ``catch-up'' mode.


\paragraph{Predecessors}\mbox{}\\
PMCS DM-NET-5 Complete\\
PMCS DMTC-8000-0990 Complete\\
PMCS DMTC-8100-2130 Complete\\
PMCS DMTC-8100-2530 Complete\\
PMCS DMTC-8200-0600 Complete

\paragraph{Environment Needs}\mbox{}\\

\subparagraph{Software}\mbox{}\\
See pre-conditions.

\subparagraph{Hardware}\mbox{}\\
See pre-conditions.



\paragraph{Test Procedure}\mbox{}\\
\begin{tabular}{p{4cm}p{12cm}}
\toprule
Step 1
& Description \\ \hline
\end{tabular}
{\scriptsize
Transfer data between base and archive on primary links over
uninterrupted 1 day period.

}
\begin{tabular}{p{3cm}p{13cm}}
\hline
            & Test Data \\ \hline
\end{tabular}
{\scriptsize
LATISS, ComCAM, or FullCAM data.

}
\begin{tabular}{p{3cm}p{13cm}}
\hline
            & Expected Result \\ \hline
\end{tabular}
{\scriptsize
Data is successfully transferred over primary link at or exceeding rates
specified in LDM-142 throughout period.

}

\begin{tabular}{p{4cm}p{12cm}}
\toprule
Step 2
& Description \\ \hline
\end{tabular}
{\scriptsize
Simulate outage by disconnecting fiber on primary fiber on Base side of
Base - Archive Network.

}
\begin{tabular}{p{3cm}p{13cm}}
\hline
            & Test Data \\ \hline
\end{tabular}
{\scriptsize
NA

}
\begin{tabular}{p{3cm}p{13cm}}
\hline
            & Expected Result \\ \hline
\end{tabular}
{\scriptsize
Network fails over to secondary links in \textless{}=60s

}

\begin{tabular}{p{4cm}p{12cm}}
\toprule
Step 3
& Description \\ \hline
\end{tabular}
{\scriptsize
Transfer data between base and archive over secondary equipment
uninterrupted 1 day period.

}
\begin{tabular}{p{3cm}p{13cm}}
\hline
            & Test Data \\ \hline
\end{tabular}
{\scriptsize
LATISS, ComCAM, or FullCAM data.

}
\begin{tabular}{p{3cm}p{13cm}}
\hline
            & Expected Result \\ \hline
\end{tabular}
{\scriptsize
Data is successfully transferred over secondary link ~at or exceeding
rates specified in LDM-142 throughout period.

}

\begin{tabular}{p{4cm}p{12cm}}
\toprule
Step 4
& Description \\ \hline
\end{tabular}
{\scriptsize
Restore connection on primary link by reconnecting
fiber.\\[2\baselineskip]

}
\begin{tabular}{p{3cm}p{13cm}}
\hline
            & Test Data \\ \hline
\end{tabular}
{\scriptsize
NA

}
\begin{tabular}{p{3cm}p{13cm}}
\hline
            & Expected Result \\ \hline
\end{tabular}
{\scriptsize
Network recovers to primary.

}

\begin{tabular}{p{4cm}p{12cm}}
\toprule
Step 5
& Description \\ \hline
\end{tabular}
{\scriptsize
Demonstrate use of secondary in catch-up mode.

}
\begin{tabular}{p{3cm}p{13cm}}
\hline
            & Test Data \\ \hline
\end{tabular}
{\scriptsize
DAQ buffer full of images and associated metadata.

}
\begin{tabular}{p{3cm}p{13cm}}
\hline
            & Expected Result \\ \hline
\end{tabular}
{\scriptsize
Images from DAQ buffer and associated metadata are retrievable over
secondary path while current observing data is being transferred over
primary path.

}

\subsubsection{LVV-T197 - Verify implementation of Archive Center}\label{lvv-t197}

\begin{longtable}[]{llllll}
\toprule
Version & Status & Priority & Verification Type & Owner
\\\midrule
1 & Draft & Normal &
Test & Robert Gruendl
\\\bottomrule
\multicolumn{6}{c}{ Open \href{https://jira.lsstcorp.org/secure/Tests.jspa\#/testCase/LVV-T197}{LVV-T197} in Jira } \\
\end{longtable}

\paragraph{Verification Elements}\mbox{}\\

\begin{itemize}
\item \href{https://jira.lsstcorp.org/browse/LVV-85}{LVV-85} - DMS-REQ-0185-V-01: Archive Center

\end{itemize}

\paragraph{Test Items}\mbox{}\\

Verify that the Archive Center is sufficiently provisioned to support
prompt processing, DRP, and data access needs.








\paragraph{Test Procedure}\mbox{}\\
\begin{tabular}{p{4cm}p{12cm}}
\toprule
Step 1
& Description \\ \hline
\end{tabular}
{\scriptsize
Analyze design and sizing model

}
\begin{tabular}{p{3cm}p{13cm}}
\hline
            & Expected Result \\ \hline
\end{tabular}

\subsubsection{LVV-T198 - Verify implementation of Archive Center Disaster Recovery}\label{lvv-t198}

\begin{longtable}[]{llllll}
\toprule
Version & Status & Priority & Verification Type & Owner
\\\midrule
1 & Draft & Normal &
Test & Robert Gruendl
\\\bottomrule
\multicolumn{6}{c}{ Open \href{https://jira.lsstcorp.org/secure/Tests.jspa\#/testCase/LVV-T198}{LVV-T198} in Jira } \\
\end{longtable}

\paragraph{Verification Elements}\mbox{}\\

\begin{itemize}
\item \href{https://jira.lsstcorp.org/browse/LVV-86}{LVV-86} - DMS-REQ-0186-V-01: Archive Center Disaster Recovery

\end{itemize}

\paragraph{Test Items}\mbox{}\\

Verify disaster recovery plan for Archive Center.








\paragraph{Test Procedure}\mbox{}\\
\begin{tabular}{p{4cm}p{12cm}}
\toprule
Step 1
& Description \\ \hline
\end{tabular}
{\scriptsize
Analyze design; simulate storage failure, observe restore from disaster
recovery

}
\begin{tabular}{p{3cm}p{13cm}}
\hline
            & Expected Result \\ \hline
\end{tabular}

\subsubsection{LVV-T199 - Verify implementation of Archive Center Co-Location with Existing
Facility}\label{lvv-t199}

\begin{longtable}[]{llllll}
\toprule
Version & Status & Priority & Verification Type & Owner
\\\midrule
1 & Draft & Normal &
Test & Robert Gruendl
\\\bottomrule
\multicolumn{6}{c}{ Open \href{https://jira.lsstcorp.org/secure/Tests.jspa\#/testCase/LVV-T199}{LVV-T199} in Jira } \\
\end{longtable}

\paragraph{Verification Elements}\mbox{}\\

\begin{itemize}
\item \href{https://jira.lsstcorp.org/browse/LVV-87}{LVV-87} - DMS-REQ-0187-V-01: Archive Center Co-Location with Existing Facility

\end{itemize}

\paragraph{Test Items}\mbox{}\\

Verify the Archive Center is located at an existing supported facility.








\paragraph{Test Procedure}\mbox{}\\
\begin{tabular}{p{4cm}p{12cm}}
\toprule
Step 1
& Description \\ \hline
\end{tabular}
{\scriptsize
Analyze design

}
\begin{tabular}{p{3cm}p{13cm}}
\hline
            & Expected Result \\ \hline
\end{tabular}

\subsubsection{LVV-T200 - Verify implementation of Archive to Data Access Center Network}\label{lvv-t200}

\begin{longtable}[]{llllll}
\toprule
Version & Status & Priority & Verification Type & Owner
\\\midrule
1 & Draft & Normal &
Test & Jeff Kantor
\\\bottomrule
\multicolumn{6}{c}{ Open \href{https://jira.lsstcorp.org/secure/Tests.jspa\#/testCase/LVV-T200}{LVV-T200} in Jira } \\
\end{longtable}

\paragraph{Verification Elements}\mbox{}\\

\begin{itemize}
\item \href{https://jira.lsstcorp.org/browse/LVV-88}{LVV-88} - DMS-REQ-0188-V-01: Archive to Data Access Center Network

\end{itemize}

\paragraph{Test Items}\mbox{}\\

Verify archiving of data to Data Access Center Network at or exceeding
rates specified in \citeds{LDM-142}, i.e at archToDacBandwidth = 10000{[}megabit
per second{]}.


\paragraph{Predecessors}\mbox{}\\
PMCS DMTC-8100-2550 Complete

\paragraph{Environment Needs}\mbox{}\\

\subparagraph{Software}\mbox{}\\
See pre-conditions.

\subparagraph{Hardware}\mbox{}\\
See pre-conditions.



\paragraph{Test Procedure}\mbox{}\\
\begin{tabular}{p{4cm}p{12cm}}
\toprule
Step 1
& Description \\ \hline
\end{tabular}
{\scriptsize
Transfer data from Data Facility to US and Chilean DACs over an
uninterrupted 1 week period.\\[2\baselineskip]

}
\begin{tabular}{p{3cm}p{13cm}}
\hline
            & Test Data \\ \hline
\end{tabular}
{\scriptsize
Data Release

}
\begin{tabular}{p{3cm}p{13cm}}
\hline
            & Expected Result \\ \hline
\end{tabular}
{\scriptsize
Data transfers without significant failures or extended latency spikes

}

\begin{tabular}{p{4cm}p{12cm}}
\toprule
Step 2
& Description \\ \hline
\end{tabular}
{\scriptsize
Analyze network logs and compare with historical data on the links.

}
\begin{tabular}{p{3cm}p{13cm}}
\hline
            & Test Data \\ \hline
\end{tabular}
{\scriptsize
NA

}
\begin{tabular}{p{3cm}p{13cm}}
\hline
            & Expected Result \\ \hline
\end{tabular}
{\scriptsize
The networks can transfer data at archToDacBandwidth = 10000{[}megabit
per second{]}, i.e. at or exceeding rates specified in LDM-142.

}

\subsubsection{LVV-T201 - Verify implementation of Archive to Data Access Center Network
Availability}\label{lvv-t201}

\begin{longtable}[]{llllll}
\toprule
Version & Status & Priority & Verification Type & Owner
\\\midrule
1 & Draft & Normal &
Test & Jeff Kantor
\\\bottomrule
\multicolumn{6}{c}{ Open \href{https://jira.lsstcorp.org/secure/Tests.jspa\#/testCase/LVV-T201}{LVV-T201} in Jira } \\
\end{longtable}

\paragraph{Verification Elements}\mbox{}\\

\begin{itemize}
\item \href{https://jira.lsstcorp.org/browse/LVV-89}{LVV-89} - DMS-REQ-0189-V-01: Archive to Data Access Center Network Availability

\end{itemize}

\paragraph{Test Items}\mbox{}\\

Verify availability of archiving to Data Access Center Network using
test and historical data of or exceeding archToDacNetMTBF= 180{[}day{]}.


\paragraph{Predecessors}\mbox{}\\
PMCS DMTC-8100-2550 Complete

\paragraph{Environment Needs}\mbox{}\\

\subparagraph{Software}\mbox{}\\
See pre-conditions.

\subparagraph{Hardware}\mbox{}\\
See pre-conditions.



\paragraph{Test Procedure}\mbox{}\\
\begin{tabular}{p{4cm}p{12cm}}
\toprule
Step 1
& Description \\ \hline
\end{tabular}
{\scriptsize
Transfer data between archive and DACs over uninterrupted 1 week period.

}
\begin{tabular}{p{3cm}p{13cm}}
\hline
            & Test Data \\ \hline
\end{tabular}
{\scriptsize
Data Release or petabyte-scale test data set

}
\begin{tabular}{p{3cm}p{13cm}}
\hline
            & Expected Result \\ \hline
\end{tabular}
{\scriptsize
Data transfers without failures or extended latency spikes

}

\begin{tabular}{p{4cm}p{12cm}}
\toprule
Step 2
& Description \\ \hline
\end{tabular}
{\scriptsize
Analyze test data and compare to historical data. Extrapolate to 1 year
testimate of MTBF.

}
\begin{tabular}{p{3cm}p{13cm}}
\hline
            & Test Data \\ \hline
\end{tabular}
{\scriptsize
NA

}
\begin{tabular}{p{3cm}p{13cm}}
\hline
            & Expected Result \\ \hline
\end{tabular}
{\scriptsize
Networks can meet archToDacNetMTBF = 180{[}day{]} at or exceeding rates
specified in LDM-142.

}

\subsubsection{LVV-T202 - Verify implementation of Archive to Data Access Center Network
Reliability}\label{lvv-t202}

\begin{longtable}[]{llllll}
\toprule
Version & Status & Priority & Verification Type & Owner
\\\midrule
1 & Draft & Normal &
Test & Jeff Kantor
\\\bottomrule
\multicolumn{6}{c}{ Open \href{https://jira.lsstcorp.org/secure/Tests.jspa\#/testCase/LVV-T202}{LVV-T202} in Jira } \\
\end{longtable}

\paragraph{Verification Elements}\mbox{}\\

\begin{itemize}
\item \href{https://jira.lsstcorp.org/browse/LVV-90}{LVV-90} - DMS-REQ-0190-V-01: Archive to Data Access Center Network Reliability

\end{itemize}

\paragraph{Test Items}\mbox{}\\

Verify the reliability of Archive to Data Access Center Network by
demonstrating successful failover and capacity to the secondary part and
subsequent recovery to primary within or exceeding chToDacNetMTTR =
48{[}hour{]}.


\paragraph{Predecessors}\mbox{}\\
PMCS DMTC-8100-2550 Complete

\paragraph{Environment Needs}\mbox{}\\

\subparagraph{Software}\mbox{}\\
See pre-conditions.

\subparagraph{Hardware}\mbox{}\\
See pre-conditions.



\paragraph{Test Procedure}\mbox{}\\
\begin{tabular}{p{4cm}p{12cm}}
\toprule
Step 1
& Description \\ \hline
\end{tabular}
{\scriptsize
Simulate failure on primary paths by disconnecting fiber at an endpoint
location in the archive on the Archive - ~DACs network.

}
\begin{tabular}{p{3cm}p{13cm}}
\hline
            & Test Data \\ \hline
\end{tabular}
{\scriptsize
NA

}
\begin{tabular}{p{3cm}p{13cm}}
\hline
            & Expected Result \\ \hline
\end{tabular}
{\scriptsize
Networks fail over to secondary paths.

}

\begin{tabular}{p{4cm}p{12cm}}
\toprule
Step 2
& Description \\ \hline
\end{tabular}
{\scriptsize
Monitor transfers on secondary paths for 1 day.

}
\begin{tabular}{p{3cm}p{13cm}}
\hline
            & Expected Result \\ \hline
\end{tabular}
{\scriptsize
Transfers occur without extended failures or extended latency spikes.
~Data transfers on secondary at rates at or above those specified in
LDM-142.

}

\begin{tabular}{p{4cm}p{12cm}}
\toprule
Step 3
& Description \\ \hline
\end{tabular}
{\scriptsize
Simulate repair and recovery period by leaving primary fiber
disconnected for at least 1 day, then reconnecting primary fiber.

}
\begin{tabular}{p{3cm}p{13cm}}
\hline
            & Test Data \\ \hline
\end{tabular}
{\scriptsize
NA

}
\begin{tabular}{p{3cm}p{13cm}}
\hline
            & Expected Result \\ \hline
\end{tabular}
{\scriptsize
Wall clock advances by 1 day. ~Network recovers to primary path. ~Verify
entire process meets chToDacNetMTTR = 48{[}hour{]}.

}

\subsubsection{LVV-T203 - Verify implementation of Archive to Data Access Center Network Secondary
Link}\label{lvv-t203}

\begin{longtable}[]{llllll}
\toprule
Version & Status & Priority & Verification Type & Owner
\\\midrule
1 & Draft & Normal &
Test & Kian-Tat Lim
\\\bottomrule
\multicolumn{6}{c}{ Open \href{https://jira.lsstcorp.org/secure/Tests.jspa\#/testCase/LVV-T203}{LVV-T203} in Jira } \\
\end{longtable}

\paragraph{Verification Elements}\mbox{}\\

\begin{itemize}
\item \href{https://jira.lsstcorp.org/browse/LVV-91}{LVV-91} - DMS-REQ-0191-V-01: Archive to Data Access Center Network Secondary Link

\end{itemize}

\paragraph{Test Items}\mbox{}\\

Verify the Archive to Data Access Center Network via Secondary Link by
simulating a failure on the primary path and capacity on the secondary
path.


\paragraph{Predecessors}\mbox{}\\
PMCS DMTC-8100-2550 Complete

\paragraph{Environment Needs}\mbox{}\\

\subparagraph{Software}\mbox{}\\
See pre-conditions.

\subparagraph{Hardware}\mbox{}\\
See pre-conditions.



\paragraph{Test Procedure}\mbox{}\\
\begin{tabular}{p{4cm}p{12cm}}
\toprule
Step 1
& Description \\ \hline
\end{tabular}
{\scriptsize
Transfer data between Archive and DACs on primary path over
uninterrupted 1 week period.

}
\begin{tabular}{p{3cm}p{13cm}}
\hline
            & Test Data \\ \hline
\end{tabular}
{\scriptsize
Data Release or other petabyte-scale test data set.

}
\begin{tabular}{p{3cm}p{13cm}}
\hline
            & Expected Result \\ \hline
\end{tabular}
{\scriptsize
Data transfers without failures or extended latency spikes, at or
exceeding rates specified in LDM-142 throughout fail-over period.

}

\begin{tabular}{p{4cm}p{12cm}}
\toprule
Step 2
& Description \\ \hline
\end{tabular}
{\scriptsize
Simulate outage on primary path by disconnecting fiber on primary on
Archive side of Archive - DACs networks.

}
\begin{tabular}{p{3cm}p{13cm}}
\hline
            & Test Data \\ \hline
\end{tabular}
{\scriptsize
NA

}
\begin{tabular}{p{3cm}p{13cm}}
\hline
            & Expected Result \\ \hline
\end{tabular}
{\scriptsize
Network fails over to secondary links in \textless{}=
60s.\\[2\baselineskip]

}

\begin{tabular}{p{4cm}p{12cm}}
\toprule
Step 3
& Description \\ \hline
\end{tabular}
{\scriptsize
Transfer data between base and archive over secondary equipment
uninterrupted 1 day period.

}
\begin{tabular}{p{3cm}p{13cm}}
\hline
            & Test Data \\ \hline
\end{tabular}
{\scriptsize
Data Release or other petabyte-scale test data set.

}
\begin{tabular}{p{3cm}p{13cm}}
\hline
            & Expected Result \\ \hline
\end{tabular}
{\scriptsize
Data transfers without failures or extended latency spikes, ~at or
exceeding rates specified in LDM-142 throughout fail-over period.

}

\begin{tabular}{p{4cm}p{12cm}}
\toprule
Step 4
& Description \\ \hline
\end{tabular}
{\scriptsize
Restore connection on primary link (reconnect fiber).

}
\begin{tabular}{p{3cm}p{13cm}}
\hline
            & Test Data \\ \hline
\end{tabular}
{\scriptsize
NA

}
\begin{tabular}{p{3cm}p{13cm}}
\hline
            & Expected Result \\ \hline
\end{tabular}
{\scriptsize
Network recovers to primary in \textless{}= 60s.

}

\subsubsection{LVV-T204 - Verify implementation of Access to catalogs for external Level 3
processing}\label{lvv-t204}

\begin{longtable}[]{llllll}
\toprule
Version & Status & Priority & Verification Type & Owner
\\\midrule
1 & Draft & Normal &
Test & Kian-Tat Lim
\\\bottomrule
\multicolumn{6}{c}{ Open \href{https://jira.lsstcorp.org/secure/Tests.jspa\#/testCase/LVV-T204}{LVV-T204} in Jira } \\
\end{longtable}

\paragraph{Verification Elements}\mbox{}\\

\begin{itemize}
\item \href{https://jira.lsstcorp.org/browse/LVV-50}{LVV-50} - DMS-REQ-0122-V-01: Access to catalogs for external Level 3 processing

\end{itemize}

\paragraph{Test Items}\mbox{}\\

Verify that catalog export, and maintenance/validation tools for Level 3
products to outside of the Data Access Centers.








\paragraph{Test Procedure}\mbox{}\\
\begin{tabular}{p{4cm}p{12cm}}
\toprule
Step 1
& Description \\ \hline
\end{tabular}
{\scriptsize
Execute bulk distribution of DRP catalogs

}
\begin{tabular}{p{3cm}p{13cm}}
\hline
            & Expected Result \\ \hline
\end{tabular}

\begin{tabular}{p{4cm}p{12cm}}
\toprule
Step 2
& Description \\ \hline
\end{tabular}
{\scriptsize
Observe correct transfer and use of maintenance/validation tools

}
\begin{tabular}{p{3cm}p{13cm}}
\hline
            & Expected Result \\ \hline
\end{tabular}

\subsubsection{LVV-T205 - Verify implementation of Access to input catalogs for DAC-based Level 3
processing}\label{lvv-t205}

\begin{longtable}[]{llllll}
\toprule
Version & Status & Priority & Verification Type & Owner
\\\midrule
1 & Draft & Normal &
Test & Robert Gruendl
\\\bottomrule
\multicolumn{6}{c}{ Open \href{https://jira.lsstcorp.org/secure/Tests.jspa\#/testCase/LVV-T205}{LVV-T205} in Jira } \\
\end{longtable}

\paragraph{Verification Elements}\mbox{}\\

\begin{itemize}
\item \href{https://jira.lsstcorp.org/browse/LVV-51}{LVV-51} - DMS-REQ-0123-V-01: Access to input catalogs for DAC-based Level 3
processing

\end{itemize}

\paragraph{Test Items}\mbox{}\\

Verify that data products are available at the Data Access Centers for
use in Level 3 processing.








\paragraph{Test Procedure}\mbox{}\\
\begin{tabular}{p{4cm}p{12cm}}
\toprule
Step 1
& Description \\ \hline
\end{tabular}
{\scriptsize
Load Prompt and DR catalogs into PDAC, observe access via LSP

}
\begin{tabular}{p{3cm}p{13cm}}
\hline
            & Expected Result \\ \hline
\end{tabular}

\subsubsection{LVV-T206 - Verify implementation of Federation with external catalogs}\label{lvv-t206}

\begin{longtable}[]{llllll}
\toprule
Version & Status & Priority & Verification Type & Owner
\\\midrule
1 & Draft & Normal &
Test & Colin Slater
\\\bottomrule
\multicolumn{6}{c}{ Open \href{https://jira.lsstcorp.org/secure/Tests.jspa\#/testCase/LVV-T206}{LVV-T206} in Jira } \\
\end{longtable}

\paragraph{Verification Elements}\mbox{}\\

\begin{itemize}
\item \href{https://jira.lsstcorp.org/browse/LVV-52}{LVV-52} - DMS-REQ-0124-V-01: Federation with external catalogs

\end{itemize}

\paragraph{Test Items}\mbox{}\\

Verify that LSST-produced data can be combined with external datasets.








\paragraph{Test Procedure}\mbox{}\\
\begin{tabular}{p{4cm}p{12cm}}
\toprule
Step 1
& Description \\ \hline
\end{tabular}
{\scriptsize
Load external catalog into PDAC (using VO if possible), observe
federation with other catalogs via LSP

}
\begin{tabular}{p{3cm}p{13cm}}
\hline
            & Expected Result \\ \hline
\end{tabular}

\subsubsection{LVV-T207 - Verify implementation of Access to images for external Level 3
processing}\label{lvv-t207}

\begin{longtable}[]{llllll}
\toprule
Version & Status & Priority & Verification Type & Owner
\\\midrule
1 & Draft & Normal &
Test & Kian-Tat Lim
\\\bottomrule
\multicolumn{6}{c}{ Open \href{https://jira.lsstcorp.org/secure/Tests.jspa\#/testCase/LVV-T207}{LVV-T207} in Jira } \\
\end{longtable}

\paragraph{Verification Elements}\mbox{}\\

\begin{itemize}
\item \href{https://jira.lsstcorp.org/browse/LVV-54}{LVV-54} - DMS-REQ-0126-V-01: Access to images for external Level 3 processing

\end{itemize}

\paragraph{Test Items}\mbox{}\\

Verify that bulk distribution of images, and accompanying
maintenance/validation tools for Level 3 image products to outside of
the Data Access Centers.








\paragraph{Test Procedure}\mbox{}\\
\begin{tabular}{p{4cm}p{12cm}}
\toprule
Step 1
& Description \\ \hline
\end{tabular}
{\scriptsize
Execute bulk distribution of DRP images

}
\begin{tabular}{p{3cm}p{13cm}}
\hline
            & Expected Result \\ \hline
\end{tabular}

\begin{tabular}{p{4cm}p{12cm}}
\toprule
Step 2
& Description \\ \hline
\end{tabular}
{\scriptsize
Observe correct transfer and use of maintenance/validation tools

}
\begin{tabular}{p{3cm}p{13cm}}
\hline
            & Expected Result \\ \hline
\end{tabular}

\subsubsection{LVV-T208 - Verify implementation of Access to input images for DAC-based Level 3
processing}\label{lvv-t208}

\begin{longtable}[]{llllll}
\toprule
Version & Status & Priority & Verification Type & Owner
\\\midrule
1 & Draft & Normal &
Test & Kian-Tat Lim
\\\bottomrule
\multicolumn{6}{c}{ Open \href{https://jira.lsstcorp.org/secure/Tests.jspa\#/testCase/LVV-T208}{LVV-T208} in Jira } \\
\end{longtable}

\paragraph{Verification Elements}\mbox{}\\

\begin{itemize}
\item \href{https://jira.lsstcorp.org/browse/LVV-55}{LVV-55} - DMS-REQ-0127-V-01: Access to input images for DAC-based Level 3
processing

\end{itemize}

\paragraph{Test Items}\mbox{}\\

Verify that prompt processing and DRP products are available at the DACs
for Level 3 processing at the DACs.








\paragraph{Test Procedure}\mbox{}\\
\begin{tabular}{p{4cm}p{12cm}}
\toprule
Step 1
& Description \\ \hline
\end{tabular}
{\scriptsize
Load Prompt and DR images into PDAC

}
\begin{tabular}{p{3cm}p{13cm}}
\hline
            & Expected Result \\ \hline
\end{tabular}

\begin{tabular}{p{4cm}p{12cm}}
\toprule
Step 2
& Description \\ \hline
\end{tabular}
{\scriptsize
Observe access via LSP

}
\begin{tabular}{p{3cm}p{13cm}}
\hline
            & Expected Result \\ \hline
\end{tabular}

\subsubsection{LVV-T209 - Verify implementation of Data Access Centers}\label{lvv-t209}

\begin{longtable}[]{llllll}
\toprule
Version & Status & Priority & Verification Type & Owner
\\\midrule
1 & Draft & Normal &
Analysis & Kian-Tat Lim
\\\bottomrule
\multicolumn{6}{c}{ Open \href{https://jira.lsstcorp.org/secure/Tests.jspa\#/testCase/LVV-T209}{LVV-T209} in Jira } \\
\end{longtable}

\paragraph{Verification Elements}\mbox{}\\

\begin{itemize}
\item \href{https://jira.lsstcorp.org/browse/LVV-92}{LVV-92} - DMS-REQ-0193-V-01: Data Access Centers

\end{itemize}

\paragraph{Test Items}\mbox{}\\

Verify that the Data Access Centers are provisioned with computing
resources necessary to support end-user access to LSST Data Products.








\paragraph{Test Procedure}\mbox{}\\
\begin{tabular}{p{4cm}p{12cm}}
\toprule
Step 1
& Description \\ \hline
\end{tabular}
{\scriptsize
Analyze design

}
\begin{tabular}{p{3cm}p{13cm}}
\hline
            & Expected Result \\ \hline
\end{tabular}

\subsubsection{LVV-T210 - Verify implementation of Data Access Center Simultaneous Connections}\label{lvv-t210}

\begin{longtable}[]{llllll}
\toprule
Version & Status & Priority & Verification Type & Owner
\\\midrule
1 & Draft & Normal &
Test & Kian-Tat Lim
\\\bottomrule
\multicolumn{6}{c}{ Open \href{https://jira.lsstcorp.org/secure/Tests.jspa\#/testCase/LVV-T210}{LVV-T210} in Jira } \\
\end{longtable}

\paragraph{Verification Elements}\mbox{}\\

\begin{itemize}
\item \href{https://jira.lsstcorp.org/browse/LVV-93}{LVV-93} - DMS-REQ-0194-V-01: Data Access Center Simultaneous Connections

\end{itemize}

\paragraph{Test Items}\mbox{}\\

Verify that the each DAC can support at least dacMinConnections
simultaneously








\paragraph{Test Procedure}\mbox{}\\
\begin{tabular}{p{4cm}p{12cm}}
\toprule
Step 1
& Description \\ \hline
\end{tabular}
{\scriptsize
Simulate data access to PDAC

}
\begin{tabular}{p{3cm}p{13cm}}
\hline
            & Expected Result \\ \hline
\end{tabular}

\begin{tabular}{p{4cm}p{12cm}}
\toprule
Step 2
& Description \\ \hline
\end{tabular}
{\scriptsize
Observe scaling

}
\begin{tabular}{p{3cm}p{13cm}}
\hline
            & Expected Result \\ \hline
\end{tabular}

\subsubsection{LVV-T211 - Verify implementation of Data Access Center Geographical Distribution}\label{lvv-t211}

\begin{longtable}[]{llllll}
\toprule
Version & Status & Priority & Verification Type & Owner
\\\midrule
1 & Draft & Normal &
Analysis & Kian-Tat Lim
\\\bottomrule
\multicolumn{6}{c}{ Open \href{https://jira.lsstcorp.org/secure/Tests.jspa\#/testCase/LVV-T211}{LVV-T211} in Jira } \\
\end{longtable}

\paragraph{Verification Elements}\mbox{}\\

\begin{itemize}
\item \href{https://jira.lsstcorp.org/browse/LVV-94}{LVV-94} - DMS-REQ-0196-V-01: Data Access Center Geographical Distribution

\end{itemize}

\paragraph{Test Items}\mbox{}\\

Verify that the DACs are geographically distributed to provide
low-latency access to data-rights community.








\paragraph{Test Procedure}\mbox{}\\
\begin{tabular}{p{4cm}p{12cm}}
\toprule
Step 1
& Description \\ \hline
\end{tabular}
{\scriptsize
Analyze design

}
\begin{tabular}{p{3cm}p{13cm}}
\hline
            & Expected Result \\ \hline
\end{tabular}

\subsubsection{LVV-T212 - Verify implementation of No Limit on Data Access Centers}\label{lvv-t212}

\begin{longtable}[]{llllll}
\toprule
Version & Status & Priority & Verification Type & Owner
\\\midrule
1 & Draft & Normal &
Test & Colin Slater
\\\bottomrule
\multicolumn{6}{c}{ Open \href{https://jira.lsstcorp.org/secure/Tests.jspa\#/testCase/LVV-T212}{LVV-T212} in Jira } \\
\end{longtable}

\paragraph{Verification Elements}\mbox{}\\

\begin{itemize}
\item \href{https://jira.lsstcorp.org/browse/LVV-95}{LVV-95} - DMS-REQ-0197-V-01: No Limit on Data Access Centers

\end{itemize}

\paragraph{Test Items}\mbox{}\\

Verify that additional Data Access Centers can be set up.








\paragraph{Test Procedure}\mbox{}\\
\begin{tabular}{p{4cm}p{12cm}}
\toprule
Step 1
& Description \\ \hline
\end{tabular}
{\scriptsize
Analyze design; instantiate and load simulated DAC, observe correct
functioning

}
\begin{tabular}{p{3cm}p{13cm}}
\hline
            & Expected Result \\ \hline
\end{tabular}

\subsubsection{LVV-T284 - RAS-00-05: (LDM-503-8b) Writing data from CCOB to the DBB for further
data processing}\label{lvv-t284}

\begin{longtable}[]{llllll}
\toprule
Version & Status & Priority & Verification Type & Owner
\\\midrule
1 & Draft & Normal &
Test & Michelle Butler
\\\bottomrule
\multicolumn{6}{c}{ Open \href{https://jira.lsstcorp.org/secure/Tests.jspa\#/testCase/LVV-T284}{LVV-T284} in Jira } \\
\end{longtable}

\paragraph{Verification Elements}\mbox{}\\

\begin{itemize}
\item \href{https://jira.lsstcorp.org/browse/LVV-9}{LVV-9} - DMS-REQ-0020-V-01: Wavefront Sensor Data Acquisition

\item \href{https://jira.lsstcorp.org/browse/LVV-8}{LVV-8} - DMS-REQ-0018-V-01: Raw Science Image Data Acquisition

\item \href{https://jira.lsstcorp.org/browse/LVV-96}{LVV-96} - DMS-REQ-0265-V-01: Guider Calibration Data Acquisition

\item \href{https://jira.lsstcorp.org/browse/LVV-28}{LVV-28} - DMS-REQ-0068-V-01: Raw Science Image Metadata

\item \href{https://jira.lsstcorp.org/browse/LVV-11}{LVV-11} - DMS-REQ-0024-V-01: Raw Image Assembly

\item \href{https://jira.lsstcorp.org/browse/LVV-146}{LVV-146} - DMS-REQ-0315-V-01: DMS Communication with OCS

\item \href{https://jira.lsstcorp.org/browse/LVV-115}{LVV-115} - DMS-REQ-0284-V-01: Level-1 Production Completeness

\end{itemize}

\paragraph{Test Items}\mbox{}\\

This test will check:

\begin{itemize}
\tightlist
\item
  The successful integration of the DAQ archiver components with the
  CCOB
\item
  That the file can then be ingested into the DBB and be retrieved for
  further analysis
\end{itemize}


\paragraph{Predecessors}\mbox{}\\
None.

\paragraph{Environment Needs}\mbox{}\\

\subparagraph{Software}\mbox{}\\
\begin{itemize}
\tightlist
\item
  CCOB device and the software to produce a file to be transferred and
  kept
\item
  DBB software to produce a retrieval file for further processing
\end{itemize}

\subparagraph{Hardware}\mbox{}\\
\begin{itemize}
\tightlist
\item
  CCOB
\item
  Test machine for LSST Monitoring Service
\item
  consolidate DB~
\item
  DBB ingest file system~
\item
  DBB output file system~
\item
  data transfer protocol to move data from CCOB file systems to DBB
  ingest file system~
\end{itemize}

\paragraph{Input Specification}\mbox{}\\
None.

\paragraph{Output Specification}\mbox{}\\
\begin{itemize}
\tightlist
\item
  CCOB (raw image) files that follow specifications;
\item
  DBB files that follow specifications;
\item
  CCOB device directs a human to where a file is wanted to be stored in
  the DBB;
\item
  Transfer the file to the DBB ingest area;
\end{itemize}

\paragraph{Test Procedure}\mbox{}\\
\begin{tabular}{p{4cm}p{12cm}}
\toprule
Step 1
& Description \\ \hline
\end{tabular}
{\scriptsize
CCOB device directs a human to where a raw file is wanted to be stored
in the DBB

}
\begin{tabular}{p{3cm}p{13cm}}
\hline
            & Expected Result \\ \hline
\end{tabular}
{\scriptsize
A file with a unique file name is in a file system somewhere, and the
data is then transferred to NCSA.~ ~

}

\begin{tabular}{p{4cm}p{12cm}}
\toprule
Step 2
& Description \\ \hline
\end{tabular}
{\scriptsize
Move the data from the transferred directory into the DBB foreign file
ingest file system. ~

}
\begin{tabular}{p{3cm}p{13cm}}
\hline
            & Expected Result \\ \hline
\end{tabular}
{\scriptsize
A command is executed by a human with a file name and path to the file
wanted to be stored in the DBB.~ The file is transferred to NCSA's DBB
ingest area.~ ~~

}

\begin{tabular}{p{4cm}p{12cm}}
\toprule
Step 3
& Description \\ \hline
\end{tabular}
{\scriptsize
Have data inspected by scientist for managing that all data was
transferred.~ ~

}
\begin{tabular}{p{3cm}p{13cm}}
\hline
            & Expected Result \\ \hline
\end{tabular}
{\scriptsize
a specific Okay to move forward; or something is
broke.\\[2\baselineskip]

}

\begin{tabular}{p{4cm}p{12cm}}
\toprule
Step 4
& Description \\ \hline
\end{tabular}
{\scriptsize
The DBB is notified of a new file being in the ingest area, and the DBB
ingest is run manually to ingest the CCOB file.~ ~

}
\begin{tabular}{p{3cm}p{13cm}}
\hline
            & Expected Result \\ \hline
\end{tabular}
{\scriptsize
The DBB puts the resulting file into the DBB file systems depending on
what type of file it is. ~The DB is updated with metadata and providence
of the file to be kept. ~ The resulting file system is queryable by the
LSP to find the CCOB raw image.~~

}

\begin{tabular}{p{4cm}p{12cm}}
\toprule
Step 5
& Description \\ \hline
\end{tabular}
{\scriptsize
The LSP can review and use the CCOB raw data file that was stored
originally somewhere else such as slac

}
\begin{tabular}{p{3cm}p{13cm}}
\hline
            & Expected Result \\ \hline
\end{tabular}
{\scriptsize
LSP has the ability to find the file and view/use it.
~\\[2\baselineskip]

}

\subsubsection{LVV-T1097 - Verify Summit Facility Network Implementation}\label{lvv-t1097}

\begin{longtable}[]{llllll}
\toprule
Version & Status & Priority & Verification Type & Owner
\\\midrule
1 & Draft & Normal &
Test & Jeff Kantor
\\\bottomrule
\multicolumn{6}{c}{ Open \href{https://jira.lsstcorp.org/secure/Tests.jspa\#/testCase/LVV-T1097}{LVV-T1097} in Jira } \\
\end{longtable}

\paragraph{Verification Elements}\mbox{}\\

\begin{itemize}
\item \href{https://jira.lsstcorp.org/browse/LVV-71}{LVV-71} - DMS-REQ-0168-V-01: Summit Facility Data Communications

\end{itemize}

\paragraph{Test Items}\mbox{}\\

Verify that data acquired by a AuxTel DAQ can be transferred to Summit
DWDM and loaded in the EFD without problems.


\paragraph{Predecessors}\mbox{}\\
PMCS DMTC-7400-2400 Complete\\
PMCS T\&SC-2600-1545 Complete

\paragraph{Environment Needs}\mbox{}\\

\subparagraph{Software}\mbox{}\\
See pre-conditions

\subparagraph{Hardware}\mbox{}\\
See pre-conditions.\\[2\baselineskip]



\paragraph{Test Procedure}\mbox{}\\
\begin{tabular}{p{4cm}p{12cm}}
\toprule
Step 1
& Description \\ \hline
\end{tabular}
{\scriptsize
Verify the pre-conditions have been satisfied

}
\begin{tabular}{p{3cm}p{13cm}}
\hline
            & Test Data \\ \hline
\end{tabular}
{\scriptsize
NA

}
\begin{tabular}{p{3cm}p{13cm}}
\hline
            & Expected Result \\ \hline
\end{tabular}
{\scriptsize
Pre-conditions are satisfied.

}

\begin{tabular}{p{4cm}p{12cm}}
\toprule
Step 2
& Description \\ \hline
\end{tabular}
{\scriptsize
Control the AuxTel through a night of Observing. ~While observing, read
out LATISS data and transfer to Rubin Observatory Summit DWDM while
monitoring latency.

}
\begin{tabular}{p{3cm}p{13cm}}
\hline
            & Test Data \\ \hline
\end{tabular}
{\scriptsize
LATISS images and metadata

}
\begin{tabular}{p{3cm}p{13cm}}
\hline
            & Expected Result \\ \hline
\end{tabular}
{\scriptsize
Data is fed to DWDM without delays or errors.

}

\begin{tabular}{p{4cm}p{12cm}}
\toprule
Step 3
& Description \\ \hline
\end{tabular}
{\scriptsize
Verify that data acquired by a AuxTel DAQ can be transferred ~and loaded
in EFD without problems.

}
\begin{tabular}{p{3cm}p{13cm}}
\hline
            & Test Data \\ \hline
\end{tabular}
{\scriptsize
LATISS images and metadata

}
\begin{tabular}{p{3cm}p{13cm}}
\hline
            & Expected Result \\ \hline
\end{tabular}
{\scriptsize
Examine the EFD to ensure that the data has been loaded properly.

}

\subsubsection{LVV-T1250 - Verify implementation of minimum number of simultaneous DM EFD query
users}\label{lvv-t1250}

\begin{longtable}[]{llllll}
\toprule
Version & Status & Priority & Verification Type & Owner
\\\midrule
1 & Draft & Normal &
Test & Jeffrey Carlin
\\\bottomrule
\multicolumn{6}{c}{ Open \href{https://jira.lsstcorp.org/secure/Tests.jspa\#/testCase/LVV-T1250}{LVV-T1250} in Jira } \\
\end{longtable}

\paragraph{Verification Elements}\mbox{}\\

\begin{itemize}
\item \href{https://jira.lsstcorp.org/browse/LVV-3400}{LVV-3400} - DMS-REQ-0358-V-01: Min number of simultaneous DM EFD query users

\end{itemize}

\paragraph{Test Items}\mbox{}\\

Verify that the DM EFD can support \textbf{dmEfdQueryUsers~= 5}
simultaneous queries. The additional requirement that each query must
last no more than \textbf{dmEfdQueryTime = 10 seconds~}will be verified
separately in
\href{https://jira.lsstcorp.org/secure/Tests.jspa\#/testCase/LVV-T1251}{LVV-T1251},
but these must be satisfied together.








\paragraph{Test Procedure}\mbox{}\\
\begin{tabular}{p{4cm}p{12cm}}
\toprule
Step 1
& Description \\ \hline
\end{tabular}
{\scriptsize
Send multiple (at least 5) simultaneous queries to the DM EFD.

}
\begin{tabular}{p{3cm}p{13cm}}
\hline
            & Expected Result \\ \hline
\end{tabular}

\begin{tabular}{p{4cm}p{12cm}}
\toprule
Step 2
& Description \\ \hline
\end{tabular}
{\scriptsize
Confirm that (a) the queries executed successfully, and that (b) they
return reasonable results.~

}
\begin{tabular}{p{3cm}p{13cm}}
\hline
            & Expected Result \\ \hline
\end{tabular}

\begin{tabular}{p{4cm}p{12cm}}
\toprule
Step 3
& Description \\ \hline
\end{tabular}
{\scriptsize
Repeat the above steps for different queries, and different numbers of
simultaneous queries, to confirm that the expected performance is met
regardless of the query being executed.

}
\begin{tabular}{p{3cm}p{13cm}}
\hline
            & Expected Result \\ \hline
\end{tabular}

\subsubsection{LVV-T1251 - Verify implementation of maximum time to retrieve DM EFD query results}\label{lvv-t1251}

\begin{longtable}[]{llllll}
\toprule
Version & Status & Priority & Verification Type & Owner
\\\midrule
1 & Draft & Normal &
Test & Jeffrey Carlin
\\\bottomrule
\multicolumn{6}{c}{ Open \href{https://jira.lsstcorp.org/secure/Tests.jspa\#/testCase/LVV-T1251}{LVV-T1251} in Jira } \\
\end{longtable}

\paragraph{Verification Elements}\mbox{}\\

\begin{itemize}
\item \href{https://jira.lsstcorp.org/browse/LVV-9788}{LVV-9788} - DMS-REQ-0358-V-02: Max time to retrieve DM EFD query results

\end{itemize}

\paragraph{Test Items}\mbox{}\\

Verify that the DM EFD can support \textbf{dmEfdQueryUsers~= 5}
simultaneous queries, with each query must executing in no more than
\textbf{dmEfdQueryTime = 10 seconds.~}The requirement on at least 5
simultaneous queries will be verified separately in
\href{https://jira.lsstcorp.org/secure/Tests.jspa\#/testCase/LVV-T1250}{LVV-T1250},\href{https://jira.lsstcorp.org/secure/Tests.jspa\#/testCase/LVV-T1251}{}
but these must be satisfied together.








\paragraph{Test Procedure}\mbox{}\\
\begin{tabular}{p{4cm}p{12cm}}
\toprule
Step 1
& Description \\ \hline
\end{tabular}
{\scriptsize
Send multiple (at least 5) simultaneous queries to the DM EFD.

}
\begin{tabular}{p{3cm}p{13cm}}
\hline
            & Expected Result \\ \hline
\end{tabular}

\begin{tabular}{p{4cm}p{12cm}}
\toprule
Step 2
& Description \\ \hline
\end{tabular}
{\scriptsize
Confirm that (a) the queries executed successfully, and that (b) they
return reasonable results. Check that the time of execution for all
queries was less than 10 seconds.

}
\begin{tabular}{p{3cm}p{13cm}}
\hline
            & Expected Result \\ \hline
\end{tabular}

\begin{tabular}{p{4cm}p{12cm}}
\toprule
Step 3
& Description \\ \hline
\end{tabular}
{\scriptsize
Repeat the above steps for different queries, and different numbers of
simultaneous queries, to confirm that the expected performance is met
regardless of the query being executed.

}
\begin{tabular}{p{3cm}p{13cm}}
\hline
            & Expected Result \\ \hline
\end{tabular}

\subsubsection{LVV-T1276 - Verify implementation of latency of reporting optical transients}\label{lvv-t1276}

\begin{longtable}[]{llllll}
\toprule
Version & Status & Priority & Verification Type & Owner
\\\midrule
1 & Draft & Normal &
Test & Eric Bellm
\\\bottomrule
\multicolumn{6}{c}{ Open \href{https://jira.lsstcorp.org/secure/Tests.jspa\#/testCase/LVV-T1276}{LVV-T1276} in Jira } \\
\end{longtable}

\paragraph{Verification Elements}\mbox{}\\

\begin{itemize}
\item \href{https://jira.lsstcorp.org/browse/LVV-9740}{LVV-9740} - DMS-REQ-0004-V-02: Latency of reporting optical transients

\end{itemize}

\paragraph{Test Items}\mbox{}\\

Verify that alerts are generated for optical transients
within~\textbf{OTT1 = 1 minute~}of the completion of the readout of the
last image.








\paragraph{Test Procedure}\mbox{}\\
\begin{tabular}{p{4cm}p{12cm}}
\toprule
Step 1
& Description \\ \hline
\end{tabular}
{\scriptsize
Identify a precursor dataset containing raw images (and templates), that
is suitable for testing the Alert Production.

}
\begin{tabular}{p{3cm}p{13cm}}
\hline
            & Expected Result \\ \hline
\end{tabular}

\begin{tabular}{p{4cm}p{12cm}}
\toprule
Step 2-1
{\scriptsize from \hyperref[lvv-t866]{LVV-T866} }
& Description \\ \hline
\end{tabular}
{\scriptsize
Perform the steps of Alert Production (including, but not necessarily
limited to, single frame processing, ISR, source detection/measurement,
PSF estimation, photometric and astrometric calibration, difference
imaging, DIASource detection/measurement, source association). During
Operations, it is presumed that these are automated for a given
dataset.~

}
\begin{tabular}{p{3cm}p{13cm}}
\hline
            & Expected Result \\ \hline
\end{tabular}
{\scriptsize
An output dataset including difference images and DIASource and
DIAObject measurements.

}

\begin{tabular}{p{4cm}p{12cm}}
\toprule
Step 2-2
{\scriptsize from \hyperref[lvv-t866]{LVV-T866} }
& Description \\ \hline
\end{tabular}
{\scriptsize
Verify that the expected data products have been produced, and that
catalogs contain reasonable values for measured quantities of interest.

}
\begin{tabular}{p{3cm}p{13cm}}
\hline
            & Expected Result \\ \hline
\end{tabular}

\begin{tabular}{p{4cm}p{12cm}}
\toprule
Step 3
& Description \\ \hline
\end{tabular}
{\scriptsize
Time processing of data starting from (pre-ingested) raw files until an
alert is available for distribution; verify that this time is less than
OTT1.

}
\begin{tabular}{p{3cm}p{13cm}}
\hline
            & Expected Result \\ \hline
\end{tabular}
{\scriptsize
Alerts are received via the alert stream within OTT1=1 minute from the
time the Alert Production payload was executed.

}

\subsubsection{LVV-T1277 - Verify processing of maximum number of calibration exposures}\label{lvv-t1277}

\begin{longtable}[]{llllll}
\toprule
Version & Status & Priority & Verification Type & Owner
\\\midrule
1 & Draft & Normal &
Test & Kian-Tat Lim
\\\bottomrule
\multicolumn{6}{c}{ Open \href{https://jira.lsstcorp.org/secure/Tests.jspa\#/testCase/LVV-T1277}{LVV-T1277} in Jira } \\
\end{longtable}

\paragraph{Verification Elements}\mbox{}\\

\begin{itemize}
\item \href{https://jira.lsstcorp.org/browse/LVV-9745}{LVV-9745} - DMS-REQ-0131-V-02: Max number of calibs to be processed

\end{itemize}

\paragraph{Test Items}\mbox{}\\

Verify that as many as \textbf{nCalExpProc = 25} calibration exposures
can be processed together within time calProcTime.








\paragraph{Test Procedure}\mbox{}\\
\begin{tabular}{p{4cm}p{12cm}}
\toprule
Step 1
& Description \\ \hline
\end{tabular}
{\scriptsize
Identify a dataset of raw calibration exposures containing at least
\textbf{nCalExpProc = 25~}exposures. (If it contains more than 25
exposures, use only 25 for the test.)

}
\begin{tabular}{p{3cm}p{13cm}}
\hline
            & Expected Result \\ \hline
\end{tabular}

\begin{tabular}{p{4cm}p{12cm}}
\toprule
Step 2-1
{\scriptsize from \hyperref[lvv-t1059]{LVV-T1059} }
& Description \\ \hline
\end{tabular}
{\scriptsize
Execute the Daily Calibration Products Update payload. The payload uses
raw calibration images and information from the Transformed EFD to
generate a subset of Master Calibration Images and Calibration Database
entries in the Data Backbone.

}
\begin{tabular}{p{3cm}p{13cm}}
\hline
            & Expected Result \\ \hline
\end{tabular}

\begin{tabular}{p{4cm}p{12cm}}
\toprule
Step 2-2
{\scriptsize from \hyperref[lvv-t1059]{LVV-T1059} }
& Description \\ \hline
\end{tabular}
{\scriptsize
Confirm that the expected Master Calibration images and Calibration
Database entries are present and well-formed.

}
\begin{tabular}{p{3cm}p{13cm}}
\hline
            & Expected Result \\ \hline
\end{tabular}

\begin{tabular}{p{4cm}p{12cm}}
\toprule
Step 3
& Description \\ \hline
\end{tabular}
{\scriptsize
Confirm that the processing completed successfully within
\textbf{calProcTime = 1200 seconds.}

}
\begin{tabular}{p{3cm}p{13cm}}
\hline
            & Expected Result \\ \hline
\end{tabular}
{\scriptsize
Calibration products resulting from processed raw calibration exposures
are present within calProcTime, and are well-formed images.

}

\begin{tabular}{p{4cm}p{12cm}}
\toprule
Step 4
& Description \\ \hline
\end{tabular}
{\scriptsize
Perform the test again with \emph{more than} nCalExpProc = 25 images,
and confirm that the processing completes within~\textbf{calProcTime =
1200 seconds.}

}
\begin{tabular}{p{3cm}p{13cm}}
\hline
            & Expected Result \\ \hline
\end{tabular}
{\scriptsize
Calibration products resulting from processed raw calibration exposures
are present within calProcTime, and are well-formed images. (To verify
that the test with 25 images was not at the limits of what the software
can handle -- should be able to exceed that bare minimum.)

}

\subsubsection{LVV-T1524 - Verify Implementation of Exporting MOCs as FITS}\label{lvv-t1524}

\begin{longtable}[]{llllll}
\toprule
Version & Status & Priority & Verification Type & Owner
\\\midrule
1 & Draft & Normal &
Demonstration & Jeffrey Carlin
\\\bottomrule
\multicolumn{6}{c}{ Open \href{https://jira.lsstcorp.org/secure/Tests.jspa\#/testCase/LVV-T1524}{LVV-T1524} in Jira } \\
\end{longtable}

\paragraph{Verification Elements}\mbox{}\\

\begin{itemize}
\item \href{https://jira.lsstcorp.org/browse/LVV-18222}{LVV-18222} - DMS-REQ-0384-V-01: Export MOCs As FITS\_1

\end{itemize}

\paragraph{Test Items}\mbox{}\\

Verify that the Data Management system provides a means for exporting
the LSST-generated MOCs in the FITS serialization form defined in the
IVOA MOC Recommendation.








\paragraph{Test Procedure}\mbox{}\\
\begin{tabular}{p{4cm}p{12cm}}
\toprule
Step 1
& Description \\ \hline
\end{tabular}
{\scriptsize

}
\begin{tabular}{p{3cm}p{13cm}}
\hline
            & Expected Result \\ \hline
\end{tabular}

\subsubsection{LVV-T1525 - Verify Implementation of Linkage Between HiPS Maps and Coadded Images}\label{lvv-t1525}

\begin{longtable}[]{llllll}
\toprule
Version & Status & Priority & Verification Type & Owner
\\\midrule
1 & Draft & Normal &
Demonstration & Jeffrey Carlin
\\\bottomrule
\multicolumn{6}{c}{ Open \href{https://jira.lsstcorp.org/secure/Tests.jspa\#/testCase/LVV-T1525}{LVV-T1525} in Jira } \\
\end{longtable}

\paragraph{Verification Elements}\mbox{}\\

\begin{itemize}
\item \href{https://jira.lsstcorp.org/browse/LVV-18223}{LVV-18223} - DMS-REQ-0381-V-01: HiPS Linkage to Coadds\_1

\end{itemize}

\paragraph{Test Items}\mbox{}\\

Verify that the HiPS maps produced by the Data Management system provide
for straightforward linkage from the HiPS data to the underlying LSST
coadded images, and that this has been implemented using a mechanism
supported by both the LSST Science Platform and by community tools.








\paragraph{Test Procedure}\mbox{}\\
\begin{tabular}{p{4cm}p{12cm}}
\toprule
Step 1
& Description \\ \hline
\end{tabular}
{\scriptsize

}
\begin{tabular}{p{3cm}p{13cm}}
\hline
            & Expected Result \\ \hline
\end{tabular}

\subsubsection{LVV-T1526 - Verify Availability of Secure and Authenticated HiPS Service}\label{lvv-t1526}

\begin{longtable}[]{llllll}
\toprule
Version & Status & Priority & Verification Type & Owner
\\\midrule
1 & Draft & Normal &
Demonstration & Jeffrey Carlin
\\\bottomrule
\multicolumn{6}{c}{ Open \href{https://jira.lsstcorp.org/secure/Tests.jspa\#/testCase/LVV-T1526}{LVV-T1526} in Jira } \\
\end{longtable}

\paragraph{Verification Elements}\mbox{}\\

\begin{itemize}
\item \href{https://jira.lsstcorp.org/browse/LVV-18224}{LVV-18224} - DMS-REQ-0380-V-01: HiPS Service\_1

\end{itemize}

\paragraph{Test Items}\mbox{}\\

Verify that the Data Management system includes a secure and
authenticated Internet endpoint for an IVOA-compliant HiPS service.
~Confirm that this service is advertised via Registry as well as in the
HiPS community mechanism operated by CDS, or whatever equivalent
mechanism may exist in the LSST operations era.








\paragraph{Test Procedure}\mbox{}\\
\begin{tabular}{p{4cm}p{12cm}}
\toprule
Step 1
& Description \\ \hline
\end{tabular}
{\scriptsize

}
\begin{tabular}{p{3cm}p{13cm}}
\hline
            & Expected Result \\ \hline
\end{tabular}

\subsubsection{LVV-T1527 - Verify Support for HiPS Visualization}\label{lvv-t1527}

\begin{longtable}[]{llllll}
\toprule
Version & Status & Priority & Verification Type & Owner
\\\midrule
1 & Draft & Normal &
Demonstration & Jeffrey Carlin
\\\bottomrule
\multicolumn{6}{c}{ Open \href{https://jira.lsstcorp.org/secure/Tests.jspa\#/testCase/LVV-T1527}{LVV-T1527} in Jira } \\
\end{longtable}

\paragraph{Verification Elements}\mbox{}\\

\begin{itemize}
\item \href{https://jira.lsstcorp.org/browse/LVV-18225}{LVV-18225} - DMS-REQ-0382-V-01: HiPS Visualization\_1

\end{itemize}

\paragraph{Test Items}\mbox{}\\

Verify that the LSST Science Platform supports the visualization of
LSST-generated HiPS image maps as well as other HiPS maps which satisfy
the IVOA HiPS Recommendation. Also verify that integrated behavior is
available, such as the overplotting of catalog entries, comparable to
that provided for individual source images (e.g., PVIs and coadd tiles).








\paragraph{Test Procedure}\mbox{}\\
\begin{tabular}{p{4cm}p{12cm}}
\toprule
Step 1
& Description \\ \hline
\end{tabular}
{\scriptsize

}
\begin{tabular}{p{3cm}p{13cm}}
\hline
            & Expected Result \\ \hline
\end{tabular}

\subsubsection{LVV-T1528 - Verify Visualization of MOCs via Science Platform}\label{lvv-t1528}

\begin{longtable}[]{llllll}
\toprule
Version & Status & Priority & Verification Type & Owner
\\\midrule
1 & Draft & Normal &
Demonstration & Jeffrey Carlin
\\\bottomrule
\multicolumn{6}{c}{ Open \href{https://jira.lsstcorp.org/secure/Tests.jspa\#/testCase/LVV-T1528}{LVV-T1528} in Jira } \\
\end{longtable}

\paragraph{Verification Elements}\mbox{}\\

\begin{itemize}
\item \href{https://jira.lsstcorp.org/browse/LVV-18226}{LVV-18226} - DMS-REQ-0385-V-01: MOC Visualization\_1

\end{itemize}

\paragraph{Test Items}\mbox{}\\

Verify that the LSST Science Platform supports the visualization of the
LSST-generated MOCs as well as other MOCs which satisfy the IVOA MOC
Recommendation.








\paragraph{Test Procedure}\mbox{}\\
\begin{tabular}{p{4cm}p{12cm}}
\toprule
Step 1
& Description \\ \hline
\end{tabular}
{\scriptsize

}
\begin{tabular}{p{3cm}p{13cm}}
\hline
            & Expected Result \\ \hline
\end{tabular}

\subsubsection{LVV-T1529 - Verify Production of All-Sky HiPS Map}\label{lvv-t1529}

\begin{longtable}[]{llllll}
\toprule
Version & Status & Priority & Verification Type & Owner
\\\midrule
1 & Draft & Normal &
Demonstration & Jeffrey Carlin
\\\bottomrule
\multicolumn{6}{c}{ Open \href{https://jira.lsstcorp.org/secure/Tests.jspa\#/testCase/LVV-T1529}{LVV-T1529} in Jira } \\
\end{longtable}

\paragraph{Verification Elements}\mbox{}\\

\begin{itemize}
\item \href{https://jira.lsstcorp.org/browse/LVV-18227}{LVV-18227} - DMS-REQ-0379-V-01: Produce All-Sky HiPS Map\_1

\end{itemize}

\paragraph{Test Items}\mbox{}\\

Verify that Data Release Production includes the production of an
all-sky image map for the existing coadded image area in each filter
band, and at least one pre-defined all-sky color image map, following
the IVOA HiPS Recommendation.








\paragraph{Test Procedure}\mbox{}\\
\begin{tabular}{p{4cm}p{12cm}}
\toprule
Step 1
& Description \\ \hline
\end{tabular}
{\scriptsize

}
\begin{tabular}{p{3cm}p{13cm}}
\hline
            & Expected Result \\ \hline
\end{tabular}

\subsubsection{LVV-T1530 - Verify Production of Multi-Order Coverage Maps for Survey Data}\label{lvv-t1530}

\begin{longtable}[]{llllll}
\toprule
Version & Status & Priority & Verification Type & Owner
\\\midrule
1 & Draft & Normal &
Demonstration & Jeffrey Carlin
\\\bottomrule
\multicolumn{6}{c}{ Open \href{https://jira.lsstcorp.org/secure/Tests.jspa\#/testCase/LVV-T1530}{LVV-T1530} in Jira } \\
\end{longtable}

\paragraph{Verification Elements}\mbox{}\\

\begin{itemize}
\item \href{https://jira.lsstcorp.org/browse/LVV-18228}{LVV-18228} - DMS-REQ-0383-V-01: Produce MOC Maps\_1

\end{itemize}

\paragraph{Test Items}\mbox{}\\

Verify that Data Release Production includes the production of
Multi-Order Coverage maps for the survey data, conformant with the IVOA
MOC recommendation. ~Confirm that separate MOC are produced for each
filter band for the main survey, and additional MOCs are produced to
represent special-programs datasets and other collections of on-sky
data.








\paragraph{Test Procedure}\mbox{}\\
\begin{tabular}{p{4cm}p{12cm}}
\toprule
Step 1
& Description \\ \hline
\end{tabular}
{\scriptsize

}
\begin{tabular}{p{3cm}p{13cm}}
\hline
            & Expected Result \\ \hline
\end{tabular}

\subsubsection{LVV-T1556 - LDM-503-10B Large Scale CCOB Data Access}\label{lvv-t1556}

\begin{longtable}[]{llllll}
\toprule
Version & Status & Priority & Verification Type & Owner
\\\midrule
1 & Draft & Normal &
Demonstration & Michelle Butler
\\\bottomrule
\multicolumn{6}{c}{ Open \href{https://jira.lsstcorp.org/secure/Tests.jspa\#/testCase/LVV-T1556}{LVV-T1556} in Jira } \\
\end{longtable}

\paragraph{Verification Elements}\mbox{}\\

\begin{itemize}
\item \href{https://jira.lsstcorp.org/browse/LVV-8}{LVV-8} - DMS-REQ-0018-V-01: Raw Science Image Data Acquisition

\item \href{https://jira.lsstcorp.org/browse/LVV-9}{LVV-9} - DMS-REQ-0020-V-01: Wavefront Sensor Data Acquisition

\item \href{https://jira.lsstcorp.org/browse/LVV-11}{LVV-11} - DMS-REQ-0024-V-01: Raw Image Assembly

\item \href{https://jira.lsstcorp.org/browse/LVV-146}{LVV-146} - DMS-REQ-0315-V-01: DMS Communication with OCS

\item \href{https://jira.lsstcorp.org/browse/LVV-28}{LVV-28} - DMS-REQ-0068-V-01: Raw Science Image Metadata

\end{itemize}

\paragraph{Test Items}\mbox{}\\

Demonstrate the ability to transfer data from the SLAC test stand or
CCOB with 21 rafts from SLAC and ingested at NCSA and make available
through an instance of the RSP








\paragraph{Test Procedure}\mbox{}\\
\begin{tabular}{p{4cm}p{12cm}}
\toprule
Step 1
& Description \\ \hline
\end{tabular}
{\scriptsize
Have a system at SLAC that has the 21 raft data that needs to be
transferred to NCSA, and all accounts and scripts installed on
environment that can read that data.~ ~

}
\begin{tabular}{p{3cm}p{13cm}}
\hline
            & Test Data \\ \hline
\end{tabular}
{\scriptsize
21 rafts of data with proper headers~

}
\begin{tabular}{p{3cm}p{13cm}}
\hline
            & Expected Result \\ \hline
\end{tabular}
{\scriptsize
scripts are able to transfer the data to NCSA though rsync or bbcp.
~\\[2\baselineskip]

}

\begin{tabular}{p{4cm}p{12cm}}
\toprule
Step 2
& Description \\ \hline
\end{tabular}
{\scriptsize
Data is transferred to NCSA and ingested into Butler~

}
\begin{tabular}{p{3cm}p{13cm}}
\hline
            & Test Data \\ \hline
\end{tabular}
{\scriptsize
21 rafts of data~

}
\begin{tabular}{p{3cm}p{13cm}}
\hline
            & Expected Result \\ \hline
\end{tabular}
{\scriptsize
Data is transferred to NCSA, and can now be see in file systems by the
RSP. ~\\[2\baselineskip]

}

\begin{tabular}{p{4cm}p{12cm}}
\toprule
Step 3
& Description \\ \hline
\end{tabular}
{\scriptsize
using the RSP view the data in the ingested directory~

}
\begin{tabular}{p{3cm}p{13cm}}
\hline
            & Test Data \\ \hline
\end{tabular}
{\scriptsize
21 rafts of data with proper headers and available with Butler.get~

}
\begin{tabular}{p{3cm}p{13cm}}
\hline
            & Expected Result \\ \hline
\end{tabular}
{\scriptsize
data can be viewed.

}

\subsubsection{LVV-T1560 - Verify archiving of processing provenance}\label{lvv-t1560}

\begin{longtable}[]{llllll}
\toprule
Version & Status & Priority & Verification Type & Owner
\\\midrule
1 & Draft & Normal &
Inspection & Jeffrey Carlin
\\\bottomrule
\multicolumn{6}{c}{ Open \href{https://jira.lsstcorp.org/secure/Tests.jspa\#/testCase/LVV-T1560}{LVV-T1560} in Jira } \\
\end{longtable}

\paragraph{Verification Elements}\mbox{}\\

\begin{itemize}
\item \href{https://jira.lsstcorp.org/browse/LVV-18230}{LVV-18230} - DMS-REQ-0386-V-01: Archive Processing Provenance\_1

\end{itemize}

\paragraph{Test Items}\mbox{}\\

Verify that provenance information related to data processing, including
relevant data from other subsystems, has been archived.








\paragraph{Test Procedure}\mbox{}\\
\begin{tabular}{p{4cm}p{12cm}}
\toprule
Step 1
& Description \\ \hline
\end{tabular}
{\scriptsize

}
\begin{tabular}{p{3cm}p{13cm}}
\hline
            & Expected Result \\ \hline
\end{tabular}

\subsubsection{LVV-T1561 - Verify provenance availability to science users}\label{lvv-t1561}

\begin{longtable}[]{llllll}
\toprule
Version & Status & Priority & Verification Type & Owner
\\\midrule
1 & Draft & Normal &
Inspection & Jeffrey Carlin
\\\bottomrule
\multicolumn{6}{c}{ Open \href{https://jira.lsstcorp.org/secure/Tests.jspa\#/testCase/LVV-T1561}{LVV-T1561} in Jira } \\
\end{longtable}

\paragraph{Verification Elements}\mbox{}\\

\begin{itemize}
\item \href{https://jira.lsstcorp.org/browse/LVV-18231}{LVV-18231} - DMS-REQ-0387-V-01: Serve Archived Provenance\_1

\end{itemize}

\paragraph{Test Items}\mbox{}\\

Verify that archived provenance data is available to science users
together with the associated science data products.








\paragraph{Test Procedure}\mbox{}\\
\begin{tabular}{p{4cm}p{12cm}}
\toprule
Step 1
& Description \\ \hline
\end{tabular}
{\scriptsize

}
\begin{tabular}{p{3cm}p{13cm}}
\hline
            & Expected Result \\ \hline
\end{tabular}

\subsubsection{LVV-T1562 - Verify availability of re-run tools}\label{lvv-t1562}

\begin{longtable}[]{llllll}
\toprule
Version & Status & Priority & Verification Type & Owner
\\\midrule
1 & Draft & Normal &
Demonstration & Jeffrey Carlin
\\\bottomrule
\multicolumn{6}{c}{ Open \href{https://jira.lsstcorp.org/secure/Tests.jspa\#/testCase/LVV-T1562}{LVV-T1562} in Jira } \\
\end{longtable}

\paragraph{Verification Elements}\mbox{}\\

\begin{itemize}
\item \href{https://jira.lsstcorp.org/browse/LVV-18232}{LVV-18232} - DMS-REQ-0388-V-01: Provide Re-Run Tools\_1

\end{itemize}

\paragraph{Test Items}\mbox{}\\

Verify that tools are provided to use the archived provenance data to
re-run a data processing operation under the same conditions (including
LSST software version, its configuration parameters, and supporting data
such as calibration frames) as a previous run of that operation.








\paragraph{Test Procedure}\mbox{}\\
\begin{tabular}{p{4cm}p{12cm}}
\toprule
Step 1
& Description \\ \hline
\end{tabular}
{\scriptsize

}
\begin{tabular}{p{3cm}p{13cm}}
\hline
            & Expected Result \\ \hline
\end{tabular}

\subsubsection{LVV-T1563 - Verify re-run on different system produces the same results}\label{lvv-t1563}

\begin{longtable}[]{llllll}
\toprule
Version & Status & Priority & Verification Type & Owner
\\\midrule
1 & Draft & Normal &
Demonstration & Jeffrey Carlin
\\\bottomrule
\multicolumn{6}{c}{ Open \href{https://jira.lsstcorp.org/secure/Tests.jspa\#/testCase/LVV-T1563}{LVV-T1563} in Jira } \\
\end{longtable}

\paragraph{Verification Elements}\mbox{}\\

\begin{itemize}
\item \href{https://jira.lsstcorp.org/browse/LVV-18233}{LVV-18233} - DMS-REQ-0390-V-01: Re-Runs on Other Systems\_1

\end{itemize}

\paragraph{Test Items}\mbox{}\\

Verify that tools are provided to use the archived provenance data to
re-run a data processing operation on different systems, and that the
results produced are the same to the extent computationally feasible.~








\paragraph{Test Procedure}\mbox{}\\
\begin{tabular}{p{4cm}p{12cm}}
\toprule
Step 1
& Description \\ \hline
\end{tabular}
{\scriptsize

}
\begin{tabular}{p{3cm}p{13cm}}
\hline
            & Expected Result \\ \hline
\end{tabular}

\subsubsection{LVV-T1564 - Verify re-run on similar system produces the same results}\label{lvv-t1564}

\begin{longtable}[]{llllll}
\toprule
Version & Status & Priority & Verification Type & Owner
\\\midrule
1 & Draft & Normal &
Demonstration & Jeffrey Carlin
\\\bottomrule
\multicolumn{6}{c}{ Open \href{https://jira.lsstcorp.org/secure/Tests.jspa\#/testCase/LVV-T1564}{LVV-T1564} in Jira } \\
\end{longtable}

\paragraph{Verification Elements}\mbox{}\\

\begin{itemize}
\item \href{https://jira.lsstcorp.org/browse/LVV-18234}{LVV-18234} - DMS-REQ-0389-V-01: Re-Runs on Similar Systems\_1

\end{itemize}

\paragraph{Test Items}\mbox{}\\

Verify that a provenance-based re-run that is run on the same system, or
a system with identically configured hardware and system software,
produces the same results.~








\paragraph{Test Procedure}\mbox{}\\
\begin{tabular}{p{4cm}p{12cm}}
\toprule
Step 1
& Description \\ \hline
\end{tabular}
{\scriptsize

}
\begin{tabular}{p{3cm}p{13cm}}
\hline
            & Expected Result \\ \hline
\end{tabular}

\subsubsection{LVV-T1612 - Verify Summit - Base Network Integration (System Level)}\label{lvv-t1612}

\begin{longtable}[]{llllll}
\toprule
Version & Status & Priority & Verification Type & Owner
\\\midrule
1 & Draft & Normal &
Inspection & Jeff Kantor
\\\bottomrule
\multicolumn{6}{c}{ Open \href{https://jira.lsstcorp.org/secure/Tests.jspa\#/testCase/LVV-T1612}{LVV-T1612} in Jira } \\
\end{longtable}

\paragraph{Verification Elements}\mbox{}\\

\begin{itemize}
\item \href{https://jira.lsstcorp.org/browse/LVV-73}{LVV-73} - DMS-REQ-0171-V-01: Summit to Base Network

\end{itemize}

\paragraph{Test Items}\mbox{}\\

Verify ISO Layer 3 full (22 x 10 Gbps ethernet ports on DAQ side with
test data from DAQ test stand, AURA, Camera DAQ team do test).
Demonstrate transfer of data at or exceeding rates specified in \citeds{LDM-142}.


\paragraph{Predecessors}\mbox{}\\
See pre-conditions.

\paragraph{Environment Needs}\mbox{}\\

\subparagraph{Software}\mbox{}\\
See pre-conditions.

\subparagraph{Hardware}\mbox{}\\
See pre-conditions.



\paragraph{Test Procedure}\mbox{}\\
\begin{tabular}{p{4cm}p{12cm}}
\toprule
Step 1
& Description \\ \hline
\end{tabular}
{\scriptsize
Verify Pre-conditions are satisfied.

}
\begin{tabular}{p{3cm}p{13cm}}
\hline
            & Test Data \\ \hline
\end{tabular}
{\scriptsize
NA

}
\begin{tabular}{p{3cm}p{13cm}}
\hline
            & Expected Result \\ \hline
\end{tabular}
{\scriptsize
Pre-conditions are satisfied.

}

\begin{tabular}{p{4cm}p{12cm}}
\toprule
Step 2
& Description \\ \hline
\end{tabular}
{\scriptsize
Transfer data between summit and base over uninterrupted 1 day period.
~Monitor transfer of data at or exceeding rates specified in LDM-142.

}
\begin{tabular}{p{3cm}p{13cm}}
\hline
            & Test Data \\ \hline
\end{tabular}
{\scriptsize
DAQ pre-loaded data

}
\begin{tabular}{p{3cm}p{13cm}}
\hline
            & Expected Result \\ \hline
\end{tabular}
{\scriptsize
Data transfers at or exceeding rates specified in LDM-142.

}

\subsubsection{LVV-T1830 - Verify Implementation of Scientific Visualization of Camera Image Data}\label{lvv-t1830}

\begin{longtable}[]{llllll}
\toprule
Version & Status & Priority & Verification Type & Owner
\\\midrule
1 & Draft & Normal &
Inspection & Jeffrey Carlin
\\\bottomrule
\multicolumn{6}{c}{ Open \href{https://jira.lsstcorp.org/secure/Tests.jspa\#/testCase/LVV-T1830}{LVV-T1830} in Jira } \\
\end{longtable}

\paragraph{Verification Elements}\mbox{}\\

\begin{itemize}
\item \href{https://jira.lsstcorp.org/browse/LVV-18465}{LVV-18465} - DMS-REQ-0395-V-01: Scientific Visualization of Camera Image Data\_1

\end{itemize}

\paragraph{Test Items}\mbox{}\\

Verify that all scientific visualization of camera image data uses the
coordinate systems defined in \href{https://lse-349.lsst.io/}{LSE-349}.








\paragraph{Test Procedure}\mbox{}\\
\begin{tabular}{p{4cm}p{12cm}}
\toprule
Step 1
& Description \\ \hline
\end{tabular}
{\scriptsize

}
\begin{tabular}{p{3cm}p{13cm}}
\hline
            & Expected Result \\ \hline
\end{tabular}

\subsubsection{LVV-T1831 - Verify Implementation of Data Management Nightly Reporting}\label{lvv-t1831}

\begin{longtable}[]{llllll}
\toprule
Version & Status & Priority & Verification Type & Owner
\\\midrule
1 & Draft & Normal &
Demonstration & Jeffrey Carlin
\\\bottomrule
\multicolumn{6}{c}{ Open \href{https://jira.lsstcorp.org/secure/Tests.jspa\#/testCase/LVV-T1831}{LVV-T1831} in Jira } \\
\end{longtable}

\paragraph{Verification Elements}\mbox{}\\

\begin{itemize}
\item \href{https://jira.lsstcorp.org/browse/LVV-18295}{LVV-18295} - DMS-REQ-0394-V-01: Data Management Nightly Reporting\_1

\end{itemize}

\paragraph{Test Items}\mbox{}\\

Verify that the LSST Data Management subsystem produces a searchable -
interactive nightly report(s), from information published in the EFD by
each subsystem, summarizing performance and behavior over a user defined
period of time (e.g. the previous 24 hours).








\paragraph{Test Procedure}\mbox{}\\
\begin{tabular}{p{4cm}p{12cm}}
\toprule
Step 1
& Description \\ \hline
\end{tabular}
{\scriptsize

}
\begin{tabular}{p{3cm}p{13cm}}
\hline
            & Expected Result \\ \hline
\end{tabular}

\subsubsection{LVV-T1836 - Verify calculation of resolved-to-unresolved flux ratio errors}\label{lvv-t1836}

\begin{longtable}[]{llllll}
\toprule
Version & Status & Priority & Verification Type & Owner
\\\midrule
1 & Draft & Normal &
Test & Jeffrey Carlin
\\\bottomrule
\multicolumn{6}{c}{ Open \href{https://jira.lsstcorp.org/secure/Tests.jspa\#/testCase/LVV-T1836}{LVV-T1836} in Jira } \\
\end{longtable}

\paragraph{Verification Elements}\mbox{}\\

\begin{itemize}
\item \href{https://jira.lsstcorp.org/browse/LVV-9766}{LVV-9766} - DMS-REQ-0359-V-17: Max RMS of resolved/unresolved flux ratio

\end{itemize}

\paragraph{Test Items}\mbox{}\\

Verify that the DM system has provided code to assess whether the
maximum RMS of the ratio of the error in integrated flux measurement
between bright, isolated, resolved sources less than 10 arcsec in
diameter and bright, isolated unresolved point sources is less than
\textbf{ResSource = 2}.








\paragraph{Test Procedure}\mbox{}\\
\begin{tabular}{p{4cm}p{12cm}}
\toprule
Step 1
& Description \\ \hline
\end{tabular}
{\scriptsize

}
\begin{tabular}{p{3cm}p{13cm}}
\hline
            & Expected Result \\ \hline
\end{tabular}

\subsubsection{LVV-T1837 - Verify calculation of band-to-band color zero-point accuracy}\label{lvv-t1837}

\begin{longtable}[]{llllll}
\toprule
Version & Status & Priority & Verification Type & Owner
\\\midrule
1 & Draft & Normal &
Test & Jeffrey Carlin
\\\bottomrule
\multicolumn{6}{c}{ Open \href{https://jira.lsstcorp.org/secure/Tests.jspa\#/testCase/LVV-T1837}{LVV-T1837} in Jira } \\
\end{longtable}

\paragraph{Verification Elements}\mbox{}\\

\begin{itemize}
\item \href{https://jira.lsstcorp.org/browse/LVV-9765}{LVV-9765} - DMS-REQ-0359-V-16: Accuracy of zero point for colors without u-band

\end{itemize}

\paragraph{Test Items}\mbox{}\\

Verify that the DM system provides code to assess whether the accuracy
of absolute band-to-band color zero-points for all colors constructed
from any filter pair, excluding the u-band, is less than \textbf{PA5 = 5
millimagnitudes}.








\paragraph{Test Procedure}\mbox{}\\
\begin{tabular}{p{4cm}p{12cm}}
\toprule
Step 1
& Description \\ \hline
\end{tabular}
{\scriptsize

}
\begin{tabular}{p{3cm}p{13cm}}
\hline
            & Expected Result \\ \hline
\end{tabular}

\subsubsection{LVV-T1838 - Verify calculation of image fraction affected by ghosts}\label{lvv-t1838}

\begin{longtable}[]{llllll}
\toprule
Version & Status & Priority & Verification Type & Owner
\\\midrule
1 & Draft & Normal &
Test & Jeffrey Carlin
\\\bottomrule
\multicolumn{6}{c}{ Open \href{https://jira.lsstcorp.org/secure/Tests.jspa\#/testCase/LVV-T1838}{LVV-T1838} in Jira } \\
\end{longtable}

\paragraph{Verification Elements}\mbox{}\\

\begin{itemize}
\item \href{https://jira.lsstcorp.org/browse/LVV-9764}{LVV-9764} - DMS-REQ-0359-V-15: Percentage of image area with ghosts

\end{itemize}

\paragraph{Test Items}\mbox{}\\

Verify that the DM system provides code to assess whether the percentage
of image area that has ghosts with surface brightness gradient amplitude
of more than 1/3 of the sky noise over 1 arcsec is less than
\textbf{GhostAF = 1 percent}.








\paragraph{Test Procedure}\mbox{}\\
\begin{tabular}{p{4cm}p{12cm}}
\toprule
Step 1
& Description \\ \hline
\end{tabular}
{\scriptsize

}
\begin{tabular}{p{3cm}p{13cm}}
\hline
            & Expected Result \\ \hline
\end{tabular}

\subsubsection{LVV-T1839 - Verify calculation of RMS width of photometric zeropoint}\label{lvv-t1839}

\begin{longtable}[]{llllll}
\toprule
Version & Status & Priority & Verification Type & Owner
\\\midrule
1 & Draft & Normal &
Test & Jeffrey Carlin
\\\bottomrule
\multicolumn{6}{c}{ Open \href{https://jira.lsstcorp.org/secure/Tests.jspa\#/testCase/LVV-T1839}{LVV-T1839} in Jira } \\
\end{longtable}

\paragraph{Verification Elements}\mbox{}\\

\begin{itemize}
\item \href{https://jira.lsstcorp.org/browse/LVV-9763}{LVV-9763} - DMS-REQ-0359-V-14: RMS width of zero point in all bands except u

\end{itemize}

\paragraph{Test Items}\mbox{}\\

Verify that the DM system provides code to assess whether the RMS width
of the internal photometric zero-point (precision of system uniformity
across the sky) for all bands except u-band is less than \textbf{PA3 =
10 millimagnitudes}.








\paragraph{Test Procedure}\mbox{}\\
\begin{tabular}{p{4cm}p{12cm}}
\toprule
Step 1
& Description \\ \hline
\end{tabular}
{\scriptsize

}
\begin{tabular}{p{3cm}p{13cm}}
\hline
            & Expected Result \\ \hline
\end{tabular}

\subsubsection{LVV-T1840 - Verify calculation of sky brightness precision}\label{lvv-t1840}

\begin{longtable}[]{llllll}
\toprule
Version & Status & Priority & Verification Type & Owner
\\\midrule
1 & Draft & Normal &
Test & Jeffrey Carlin
\\\bottomrule
\multicolumn{6}{c}{ Open \href{https://jira.lsstcorp.org/secure/Tests.jspa\#/testCase/LVV-T1840}{LVV-T1840} in Jira } \\
\end{longtable}

\paragraph{Verification Elements}\mbox{}\\

\begin{itemize}
\item \href{https://jira.lsstcorp.org/browse/LVV-9762}{LVV-9762} - DMS-REQ-0359-V-13: Max sky brightness error

\end{itemize}

\paragraph{Test Items}\mbox{}\\

Verify that the DM system provides software to assess whether the
maximum error in the precision of the sky brightness determination is
less than \textbf{SBPrec = 1 percent.}








\paragraph{Test Procedure}\mbox{}\\
\begin{tabular}{p{4cm}p{12cm}}
\toprule
Step 1
& Description \\ \hline
\end{tabular}
{\scriptsize

}
\begin{tabular}{p{3cm}p{13cm}}
\hline
            & Expected Result \\ \hline
\end{tabular}

\subsubsection{LVV-T1841 - Verify calculation of scientifically unusable pixel fraction}\label{lvv-t1841}

\begin{longtable}[]{llllll}
\toprule
Version & Status & Priority & Verification Type & Owner
\\\midrule
1 & Draft & Normal &
Test & Jeffrey Carlin
\\\bottomrule
\multicolumn{6}{c}{ Open \href{https://jira.lsstcorp.org/secure/Tests.jspa\#/testCase/LVV-T1841}{LVV-T1841} in Jira } \\
\end{longtable}

\paragraph{Verification Elements}\mbox{}\\

\begin{itemize}
\item \href{https://jira.lsstcorp.org/browse/LVV-9761}{LVV-9761} - DMS-REQ-0359-V-12: Max fraction of unusable pixels per sensor

\end{itemize}

\paragraph{Test Items}\mbox{}\\

Verify that the DM system provides software to assess whether the
maximum fraction of pixels scientifically unusable per sensor out of the
total allowable fraction of sensors meeting this performance is less
than~\textbf{PixFrac = 1 percent}.








\paragraph{Test Procedure}\mbox{}\\
\begin{tabular}{p{4cm}p{12cm}}
\toprule
Step 1
& Description \\ \hline
\end{tabular}
{\scriptsize

}
\begin{tabular}{p{3cm}p{13cm}}
\hline
            & Expected Result \\ \hline
\end{tabular}

\subsubsection{LVV-T1842 - Verify calculation of zeropoint error fraction exceeding the outlier
limit}\label{lvv-t1842}

\begin{longtable}[]{llllll}
\toprule
Version & Status & Priority & Verification Type & Owner
\\\midrule
1 & Draft & Normal &
Test & Jeffrey Carlin
\\\bottomrule
\multicolumn{6}{c}{ Open \href{https://jira.lsstcorp.org/secure/Tests.jspa\#/testCase/LVV-T1842}{LVV-T1842} in Jira } \\
\end{longtable}

\paragraph{Verification Elements}\mbox{}\\

\begin{itemize}
\item \href{https://jira.lsstcorp.org/browse/LVV-9760}{LVV-9760} - DMS-REQ-0359-V-11: Fraction of zero point outliers

\end{itemize}

\paragraph{Test Items}\mbox{}\\

Verify that the DM system provides software to calculate the fraction of
zeropoint errors that exceed the zero point error outlier limit, and
confirm that it is less than \textbf{PF2 = 10 percent.}








\paragraph{Test Procedure}\mbox{}\\
\begin{tabular}{p{4cm}p{12cm}}
\toprule
Step 1
& Description \\ \hline
\end{tabular}
{\scriptsize

}
\begin{tabular}{p{3cm}p{13cm}}
\hline
            & Expected Result \\ \hline
\end{tabular}

\subsubsection{LVV-T1843 - Verify calculation of significance of imperfect crosstalk corrections}\label{lvv-t1843}

\begin{longtable}[]{llllll}
\toprule
Version & Status & Priority & Verification Type & Owner
\\\midrule
1 & Draft & Normal &
Test & Jeffrey Carlin
\\\bottomrule
\multicolumn{6}{c}{ Open \href{https://jira.lsstcorp.org/secure/Tests.jspa\#/testCase/LVV-T1843}{LVV-T1843} in Jira } \\
\end{longtable}

\paragraph{Verification Elements}\mbox{}\\

\begin{itemize}
\item \href{https://jira.lsstcorp.org/browse/LVV-9757}{LVV-9757} - DMS-REQ-0359-V-08: Max cross-talk imperfections

\end{itemize}

\paragraph{Test Items}\mbox{}\\

Verify that the DM system provides software to assess whether the
maximum local significance integrated over the PSF of imperfect
crosstalk corrections is less than \textbf{Xtalk = 3 sigma}.








\paragraph{Test Procedure}\mbox{}\\
\begin{tabular}{p{4cm}p{12cm}}
\toprule
Step 1
& Description \\ \hline
\end{tabular}
{\scriptsize

}
\begin{tabular}{p{3cm}p{13cm}}
\hline
            & Expected Result \\ \hline
\end{tabular}

\subsubsection{LVV-T1844 - Verify calculation of u-band photometric zero-point RMS}\label{lvv-t1844}

\begin{longtable}[]{llllll}
\toprule
Version & Status & Priority & Verification Type & Owner
\\\midrule
1 & Draft & Normal &
Test & Jeffrey Carlin
\\\bottomrule
\multicolumn{6}{c}{ Open \href{https://jira.lsstcorp.org/secure/Tests.jspa\#/testCase/LVV-T1844}{LVV-T1844} in Jira } \\
\end{longtable}

\paragraph{Verification Elements}\mbox{}\\

\begin{itemize}
\item \href{https://jira.lsstcorp.org/browse/LVV-9756}{LVV-9756} - DMS-REQ-0359-V-07: RMS width of zero point in u-band

\end{itemize}

\paragraph{Test Items}\mbox{}\\

Verify that the DM system provides software to assess whether the RMS
width of internal photometric zero-point (precision of system uniformity
across the sky) in the u-band is less than \textbf{PA3u = 20
millimagnitudes}.








\paragraph{Test Procedure}\mbox{}\\
\begin{tabular}{p{4cm}p{12cm}}
\toprule
Step 1
& Description \\ \hline
\end{tabular}
{\scriptsize

}
\begin{tabular}{p{3cm}p{13cm}}
\hline
            & Expected Result \\ \hline
\end{tabular}

\subsubsection{LVV-T1845 - Verify accuracy of photometric transformation to physical scale}\label{lvv-t1845}

\begin{longtable}[]{llllll}
\toprule
Version & Status & Priority & Verification Type & Owner
\\\midrule
1 & Draft & Normal &
Test & Jeffrey Carlin
\\\bottomrule
\multicolumn{6}{c}{ Open \href{https://jira.lsstcorp.org/secure/Tests.jspa\#/testCase/LVV-T1845}{LVV-T1845} in Jira } \\
\end{longtable}

\paragraph{Verification Elements}\mbox{}\\

\begin{itemize}
\item \href{https://jira.lsstcorp.org/browse/LVV-9755}{LVV-9755} - DMS-REQ-0359-V-06: Accuracy of photometric transformation

\end{itemize}

\paragraph{Test Items}\mbox{}\\

Verify that the DM system provides software to assess whether the
accuracy of the transformation of internal LSST photometry to a physical
scale (e.g. AB magnitudes) is less than \textbf{PA6 = 10
millimagnitudes}.








\paragraph{Test Procedure}\mbox{}\\
\begin{tabular}{p{4cm}p{12cm}}
\toprule
Step 1
& Description \\ \hline
\end{tabular}
{\scriptsize

}
\begin{tabular}{p{3cm}p{13cm}}
\hline
            & Expected Result \\ \hline
\end{tabular}

\subsubsection{LVV-T1846 - Verify calculation of band-to-band color zero-point accuracy including
u-band}\label{lvv-t1846}

\begin{longtable}[]{llllll}
\toprule
Version & Status & Priority & Verification Type & Owner
\\\midrule
1 & Draft & Normal &
Test & Jeffrey Carlin
\\\bottomrule
\multicolumn{6}{c}{ Open \href{https://jira.lsstcorp.org/secure/Tests.jspa\#/testCase/LVV-T1846}{LVV-T1846} in Jira } \\
\end{longtable}

\paragraph{Verification Elements}\mbox{}\\

\begin{itemize}
\item \href{https://jira.lsstcorp.org/browse/LVV-9753}{LVV-9753} - DMS-REQ-0359-V-04: Accuracy of zero point for colors with u-band

\end{itemize}

\paragraph{Test Items}\mbox{}\\

Verify that the DM system provides software to assess whether the
accuracy of absolute band-to-band color zero-points for all colors
constructed from any filter pair, including the u-band, is less than
\textbf{PA5u = 10 millimagnitudes}.








\paragraph{Test Procedure}\mbox{}\\
\begin{tabular}{p{4cm}p{12cm}}
\toprule
Step 1
& Description \\ \hline
\end{tabular}
{\scriptsize

}
\begin{tabular}{p{3cm}p{13cm}}
\hline
            & Expected Result \\ \hline
\end{tabular}

\subsubsection{LVV-T1847 - Verify calculation of sensor fraction with unusable pixels}\label{lvv-t1847}

\begin{longtable}[]{llllll}
\toprule
Version & Status & Priority & Verification Type & Owner
\\\midrule
1 & Draft & Normal &
Test & Jeffrey Carlin
\\\bottomrule
\multicolumn{6}{c}{ Open \href{https://jira.lsstcorp.org/secure/Tests.jspa\#/testCase/LVV-T1847}{LVV-T1847} in Jira } \\
\end{longtable}

\paragraph{Verification Elements}\mbox{}\\

\begin{itemize}
\item \href{https://jira.lsstcorp.org/browse/LVV-9751}{LVV-9751} - DMS-REQ-0359-V-02: Max fraction of sensors with excess unusable pixels

\end{itemize}

\paragraph{Test Items}\mbox{}\\

Verify that the DM system provides software to assess whether the
maximum allowable fraction of sensors with \textbf{PixFrac
\textgreater{} 1} percent scientifically unusable pixels is less
than~\textbf{SensorFraction = 15 percent.}








\paragraph{Test Procedure}\mbox{}\\
\begin{tabular}{p{4cm}p{12cm}}
\toprule
Step 1
& Description \\ \hline
\end{tabular}
{\scriptsize

}
\begin{tabular}{p{3cm}p{13cm}}
\hline
            & Expected Result \\ \hline
\end{tabular}

\subsubsection{LVV-T1862 - Verify determining effectiveness of dark current frame}\label{lvv-t1862}

\begin{longtable}[]{llllll}
\toprule
Version & Status & Priority & Verification Type & Owner
\\\midrule
1 & Draft & Normal &
Test & Jeffrey Carlin
\\\bottomrule
\multicolumn{6}{c}{ Open \href{https://jira.lsstcorp.org/secure/Tests.jspa\#/testCase/LVV-T1862}{LVV-T1862} in Jira } \\
\end{longtable}

\paragraph{Verification Elements}\mbox{}\\

\begin{itemize}
\item \href{https://jira.lsstcorp.org/browse/LVV-18881}{LVV-18881} - DMS-REQ-0282-V-02: Dark Current Correction Frame Effectiveness

\end{itemize}

\paragraph{Test Items}\mbox{}\\

Verify that the DMS can determine the effectiveness of a dark correction
and determine how often it should be updated.


\paragraph{Predecessors}\mbox{}\\
Execution of
\href{https://jira.lsstcorp.org/secure/Tests.jspa\#/testCase/LVV-T90}{LVV-T90}.






\paragraph{Test Procedure}\mbox{}\\
\begin{tabular}{p{4cm}p{12cm}}
\toprule
Step 1
& Description \\ \hline
\end{tabular}
{\scriptsize
Identify the path to a dataset containing dark frames (i.e., exposures
taken with the shutter closed).

}
\begin{tabular}{p{3cm}p{13cm}}
\hline
            & Expected Result \\ \hline
\end{tabular}

\begin{tabular}{p{4cm}p{12cm}}
\toprule
Step 2-1
{\scriptsize from \hyperref[lvv-t1060]{LVV-T1060} }
& Description \\ \hline
\end{tabular}
{\scriptsize
Execute the Calibration Products Production payload. The payload uses
raw calibration images and information from the Transformed EFD to
generate a subset of Master Calibration Images and Calibration Database
entries in the Data Backbone.

}
\begin{tabular}{p{3cm}p{13cm}}
\hline
            & Expected Result \\ \hline
\end{tabular}

\begin{tabular}{p{4cm}p{12cm}}
\toprule
Step 2-2
{\scriptsize from \hyperref[lvv-t1060]{LVV-T1060} }
& Description \\ \hline
\end{tabular}
{\scriptsize
Confirm that the expected Master Calibration images and Calibration
Database entries are present and well-formed.

}
\begin{tabular}{p{3cm}p{13cm}}
\hline
            & Expected Result \\ \hline
\end{tabular}

\begin{tabular}{p{4cm}p{12cm}}
\toprule
Step 3
& Description \\ \hline
\end{tabular}
{\scriptsize
Determining whether the dark correction is being done properly will
require on-sky science data. The dark correction can be applied to these
frames and the results inspected to ensure that the correction was
correctly measured and applied.

}
\begin{tabular}{p{3cm}p{13cm}}
\hline
            & Expected Result \\ \hline
\end{tabular}
{\scriptsize
Applying the dark correction to a dataset produces noticeable
differences between the original frame(s) and the corrected outputs.

}

\subsubsection{LVV-T1863 - Verify ability to process Special Programs data alongside normal
processing}\label{lvv-t1863}

\begin{longtable}[]{llllll}
\toprule
Version & Status & Priority & Verification Type & Owner
\\\midrule
1 & Draft & Normal &
Test & Jeffrey Carlin
\\\bottomrule
\multicolumn{6}{c}{ Open \href{https://jira.lsstcorp.org/secure/Tests.jspa\#/testCase/LVV-T1863}{LVV-T1863} in Jira } \\
\end{longtable}

\paragraph{Verification Elements}\mbox{}\\

\begin{itemize}
\item \href{https://jira.lsstcorp.org/browse/LVV-18847}{LVV-18847} - DMS-REQ-0397-V-01: Prompt/DR Processing of Data from Special Programs\_1

\end{itemize}

\paragraph{Test Items}\mbox{}\\

Verify that Special Programs data can be processed alongside either
prompt-products or data-release processing with little or no extra
effort by DM staff.








\paragraph{Test Procedure}\mbox{}\\
\begin{tabular}{p{4cm}p{12cm}}
\toprule
Step 1
& Description \\ \hline
\end{tabular}
{\scriptsize

}
\begin{tabular}{p{3cm}p{13cm}}
\hline
            & Expected Result \\ \hline
\end{tabular}

\subsubsection{LVV-T1865 - Verify implementation of time to L1 public release for Special Programs}\label{lvv-t1865}

\begin{longtable}[]{llllll}
\toprule
Version & Status & Priority & Verification Type & Owner
\\\midrule
1 & Draft & Normal &
Test & Jeffrey Carlin
\\\bottomrule
\multicolumn{6}{c}{ Open \href{https://jira.lsstcorp.org/secure/Tests.jspa\#/testCase/LVV-T1865}{LVV-T1865} in Jira } \\
\end{longtable}

\paragraph{Verification Elements}\mbox{}\\

\begin{itemize}
\item \href{https://jira.lsstcorp.org/browse/LVV-18229}{LVV-18229} - DMS-REQ-0344-V-01: Time to L1 public release

\end{itemize}

\paragraph{Test Items}\mbox{}\\

~Verify that data from Special Programs are made available via public
release within \textbf{L1PublicT = 24{[}hour{]}} from the acquisition of
science data.








\paragraph{Test Procedure}\mbox{}\\
\begin{tabular}{p{4cm}p{12cm}}
\toprule
Step 1
& Description \\ \hline
\end{tabular}
{\scriptsize

}
\begin{tabular}{p{3cm}p{13cm}}
\hline
            & Expected Result \\ \hline
\end{tabular}

\subsubsection{LVV-T1866 - Verify latency of reporting optical transients from Special Programs}\label{lvv-t1866}

\begin{longtable}[]{llllll}
\toprule
Version & Status & Priority & Verification Type & Owner
\\\midrule
1 & Draft & Normal &
Test & Jeffrey Carlin
\\\bottomrule
\multicolumn{6}{c}{ Open \href{https://jira.lsstcorp.org/secure/Tests.jspa\#/testCase/LVV-T1866}{LVV-T1866} in Jira } \\
\end{longtable}

\paragraph{Verification Elements}\mbox{}\\

\begin{itemize}
\item \href{https://jira.lsstcorp.org/browse/LVV-9744}{LVV-9744} - DMS-REQ-0344-V-02: Latency of reporting optical transients

\end{itemize}

\paragraph{Test Items}\mbox{}\\

Verify that optical transients (Level 1 data products) are reported
within OTT1 = 1 minute of last image readout for Special Programs.








\paragraph{Test Procedure}\mbox{}\\
\begin{tabular}{p{4cm}p{12cm}}
\toprule
Step 1
& Description \\ \hline
\end{tabular}
{\scriptsize

}
\begin{tabular}{p{3cm}p{13cm}}
\hline
            & Expected Result \\ \hline
\end{tabular}

\subsubsection{LVV-T1867 - Verify implementation of at least numStreams alert streams supported}\label{lvv-t1867}

\begin{longtable}[]{llllll}
\toprule
Version & Status & Priority & Verification Type & Owner
\\\midrule
1 & Draft & Normal &
Test & Jeffrey Carlin
\\\bottomrule
\multicolumn{6}{c}{ Open \href{https://jira.lsstcorp.org/secure/Tests.jspa\#/testCase/LVV-T1867}{LVV-T1867} in Jira } \\
\end{longtable}

\paragraph{Verification Elements}\mbox{}\\

\begin{itemize}
\item \href{https://jira.lsstcorp.org/browse/LVV-18297}{LVV-18297} - DMS-REQ-0391-V-01: Alert Stream Distribution nStreams

\end{itemize}

\paragraph{Test Items}\mbox{}\\

Verify that the LSST system supports the transmission of at least
\textbf{numStreams=5} full alert streams out of the alert distribution
system within \textbf{OTT1=1 minute}.~








\paragraph{Test Procedure}\mbox{}\\
\begin{tabular}{p{4cm}p{12cm}}
\toprule
Step 1
& Description \\ \hline
\end{tabular}
{\scriptsize

}
\begin{tabular}{p{3cm}p{13cm}}
\hline
            & Expected Result \\ \hline
\end{tabular}

\subsubsection{LVV-T1868 - Verify implementation of alert streams distributed within latency limit}\label{lvv-t1868}

\begin{longtable}[]{llllll}
\toprule
Version & Status & Priority & Verification Type & Owner
\\\midrule
1 & Draft & Normal &
Test & Jeffrey Carlin
\\\bottomrule
\multicolumn{6}{c}{ Open \href{https://jira.lsstcorp.org/secure/Tests.jspa\#/testCase/LVV-T1868}{LVV-T1868} in Jira } \\
\end{longtable}

\paragraph{Verification Elements}\mbox{}\\

\begin{itemize}
\item \href{https://jira.lsstcorp.org/browse/LVV-18911}{LVV-18911} - DMS-REQ-0391-V-02: Alert Stream Distribution Latency

\end{itemize}

\paragraph{Test Items}\mbox{}\\

Verify that the LSST system supports the transmission of full alert
streams out of the alert distribution system within \textbf{OTT1=1
minute}.








\paragraph{Test Procedure}\mbox{}\\
\begin{tabular}{p{4cm}p{12cm}}
\toprule
Step 1
& Description \\ \hline
\end{tabular}
{\scriptsize

}
\begin{tabular}{p{3cm}p{13cm}}
\hline
            & Expected Result \\ \hline
\end{tabular}


\newpage
\section{Reusable Test Cases}

Test cases in this section are made up of commonly encountered steps that have been factored out into modular, reusable scripts.
These test cases are meant solely for the building of actual tests used for verification, to be inserted in test scripts via the “Call to Test” functionality in Jira/ATM.
They streamline the process of writing test scripts by providing pre-designed steps, while also ensuring homogeneity throughout the test suite.
These reusable modules are not themselves verifying requirements.
Also, these test cases shall not call other reusable test cases in their script.



\subsection{LVV-T216 - Installation of the Alert Distribution payloads.}\label{lvv-t216}

\begin{longtable}[]{llllll}
\toprule
Version & Status & Priority & Verification Type & Owner
\\\midrule
1 & Approved & Normal &
Test & Eric Bellm
\\\bottomrule
\multicolumn{6}{c}{ Open \href{https://jira.lsstcorp.org/secure/Tests.jspa\#/testCase/LVV-T216}{LVV-T216} in Jira } \\
\end{longtable}

\paragraph{Test Items}\mbox{}\\
This test will check:\\

\begin{itemize}
\tightlist
\item
  That the Alert Distribution payloads are available from documented
  channels.
\item
  That the Alert Distribution payloads can be installed on LSST Data
  Facility-managed systems.
\item
  That the Alert Distribution payloads can be executed by LSST Data
  Facility-managed systems.
\end{itemize}



\paragraph{Environment Needs}\mbox{}\\


\subparagraph{Hardware}\mbox{}\\
This test case shall be executed on the Kubernetes Commons at the LDF.\\
As discussed in https://dmtn-028.lsst.io/ and https://dmtn-081.lsst.io/,
the test machine should have at least 16 cores, 64 GB of memory and
access to at least 1.5 TB of shared storage.



\paragraph{Test Procedure}\mbox{}\\
\begin{tabular}{p{4cm}p{12cm}}
\toprule
Step 1
& Description \\ \hline
\end{tabular}
{\scriptsize
Download Kafka Docker image from
https://github.com/lsst-dm/alert\_stream.

}
\begin{tabular}{p{3cm}p{13cm}}
\hline
            & Expected Result \\ \hline
\end{tabular}
{\scriptsize
Runs without error

}

\begin{tabular}{p{4cm}p{12cm}}
\toprule
Step 2
& Description \\ \hline
\end{tabular}
{\scriptsize
Change to the alert\_stream directory and build the docker image.\\

\begin{verbatim}
docker build -t "lsst-kub001:5000/alert_stream"
\end{verbatim}

}
\begin{tabular}{p{3cm}p{13cm}}
\hline
            & Expected Result \\ \hline
\end{tabular}
{\scriptsize
Runs without error

}

\begin{tabular}{p{4cm}p{12cm}}
\toprule
Step 3
& Description \\ \hline
\end{tabular}
{\scriptsize
Register it with Kubernetes\\[2\baselineskip]docker push
lsst-kub001:5000/alert\_stream

}
\begin{tabular}{p{3cm}p{13cm}}
\hline
            & Expected Result \\ \hline
\end{tabular}
{\scriptsize
Runs without error

}

\begin{tabular}{p{4cm}p{12cm}}
\toprule
Step 4
& Description \\ \hline
\end{tabular}
{\scriptsize
From the alert\_stream/kubernetes directory, start Kafka and
Zookeeper:\\[2\baselineskip]

\begin{verbatim}
kubectl create -f zookeeper-service.yaml
kubectl create -f zookeeper-deployment.yaml
kubectl create -f kafka-deployment.yaml
kubectl create -f kafka-service.yaml
\end{verbatim}

(use kubectl get pods/services between each command to check status;
wait until each is ``Running'' before starting the next
command)\\[2\baselineskip]

}
\begin{tabular}{p{3cm}p{13cm}}
\hline
            & Expected Result \\ \hline
\end{tabular}
{\scriptsize
Runs without error

}

\begin{tabular}{p{4cm}p{12cm}}
\toprule
Step 5
& Description \\ \hline
\end{tabular}
{\scriptsize
Confirm Kafka and Zookeeper are listed when
running\\[2\baselineskip]kubectl get
pods\\[2\baselineskip]and\\[2\baselineskip]kubectl get services

}
\begin{tabular}{p{3cm}p{13cm}}
\hline
            & Expected Result \\ \hline
\end{tabular}
{\scriptsize
Output should be similar to:\\[2\baselineskip]kubectl get pods\\
NAME ~ ~ ~ ~ ~ ~ ~ ~ ~ ~ ~ ~READY ~ ~ STATUS ~ ~RESTARTS ~ AGE\\
kafka-768ddf5564-xwgvh ~ ~ ~1/1 ~ ~ ~ Running ~ 0 ~ ~ ~ ~ ~31s\\
zookeeper-f798cc548-mgkpn ~ 1/1 ~ ~ ~ Running ~ 0 ~ ~ ~ ~
~1m\\[2\baselineskip]kubectl get services\\
NAME ~ ~ ~ ~TYPE ~ ~ ~ ~CLUSTER-IP ~ ~ ~EXTERNAL-IP ~ PORT(S) ~ ~ AGE\\
kafka ~ ~ ~ ClusterIP ~ 10.105.19.124 ~ \textless{}none\textgreater{} ~
~ ~ ~9092/TCP ~ ~6s\\
zookeeper ~ ClusterIP ~ 10.97.110.124 ~ \textless{}none\textgreater{} ~
~ ~ ~32181/TCP ~ 2m

}



\subsection{LVV-T837 - Authenticate to Notebook Aspect}\label{lvv-t837}

\begin{longtable}[]{llllll}
\toprule
Version & Status & Priority & Verification Type & Owner
\\\midrule
1 & Draft & Normal &
Test & Jeffrey Carlin
\\\bottomrule
\multicolumn{6}{c}{ Open \href{https://jira.lsstcorp.org/secure/Tests.jspa\#/testCase/LVV-T837}{LVV-T837} in Jira } \\
\end{longtable}

\paragraph{Test Items}\mbox{}\\
Not specifically a test -- modular script to be used in multiple other
Test Scripts.






\paragraph{Input Specification}\mbox{}\\
Must have a user account on the LSP.


\paragraph{Test Procedure}\mbox{}\\
\begin{tabular}{p{4cm}p{12cm}}
\toprule
Step 1
& Description \\ \hline
\end{tabular}
{\scriptsize
Authenticate to the notebook aspect of the LSST Science Platform
(NB-LSP). ~This is currently at
https://lsst-lsp-stable.ncsa.illinois.edu/nb.

}
\begin{tabular}{p{3cm}p{13cm}}
\hline
            & Expected Result \\ \hline
\end{tabular}
{\scriptsize
Redirection to the spawner page of the NB-LSP allowing selection of the
containerized stack version and machine flavor.

}

\begin{tabular}{p{4cm}p{12cm}}
\toprule
Step 2
& Description \\ \hline
\end{tabular}
{\scriptsize
Spawn a container by:\\
1) choosing an appropriate stack version: e.g. the latest weekly.\\
2) choosing an appropriate machine flavor: e.g. medium\\
3) click ``Spawn''

}
\begin{tabular}{p{3cm}p{13cm}}
\hline
            & Expected Result \\ \hline
\end{tabular}
{\scriptsize
Redirection to the JupyterLab environment served from the chosen
container containing the correct stack version.

}



\subsection{LVV-T838 - Access an empty notebook in the Notebook Aspect}\label{lvv-t838}

\begin{longtable}[]{llllll}
\toprule
Version & Status & Priority & Verification Type & Owner
\\\midrule
1 & Draft & Normal &
Test & Simon Krughoff
\\\bottomrule
\multicolumn{6}{c}{ Open \href{https://jira.lsstcorp.org/secure/Tests.jspa\#/testCase/LVV-T838}{LVV-T838} in Jira } \\
\end{longtable}

\paragraph{Test Items}\mbox{}\\
The steps here cover just those necessary to gain access to an empty
notebook after authentication is complete.






\paragraph{Input Specification}\mbox{}\\
Authentication to the Notebook aspect.


\paragraph{Test Procedure}\mbox{}\\
\begin{tabular}{p{4cm}p{12cm}}
\toprule
Step 1
& Description \\ \hline
\end{tabular}
{\scriptsize
Open a new launcher by navigating in the top menu bar ``File''
-\textgreater{} ``New Launcher''

}
\begin{tabular}{p{3cm}p{13cm}}
\hline
            & Expected Result \\ \hline
\end{tabular}
{\scriptsize
A launcher window with several sections, potentially with several kernel
versions for each.

}

\begin{tabular}{p{4cm}p{12cm}}
\toprule
Step 2
& Description \\ \hline
\end{tabular}
{\scriptsize
Select the option under ``Notebook'' labeled ``LSST'' by clicking on the
icon.

}
\begin{tabular}{p{3cm}p{13cm}}
\hline
            & Expected Result \\ \hline
\end{tabular}
{\scriptsize
An empty notebook with a single empty cell. ~The kernel show up as
``LSST'' in the top right of the notebook.

}



\subsection{LVV-T849 - Authenticate to the portal aspect of the LSP}\label{lvv-t849}

\begin{longtable}[]{llllll}
\toprule
Version & Status & Priority & Verification Type & Owner
\\\midrule
2 & Draft & Normal &
Test & Simon Krughoff
\\\bottomrule
\multicolumn{6}{c}{ Open \href{https://jira.lsstcorp.org/secure/Tests.jspa\#/testCase/LVV-T849}{LVV-T849} in Jira } \\
\end{longtable}

\paragraph{Test Items}\mbox{}\\
Obtain an authenticated session in the portal aspect of the LSST Science
Platform








\paragraph{Test Procedure}\mbox{}\\
\begin{tabular}{p{4cm}p{12cm}}
\toprule
Step 1
& Description \\ \hline
\end{tabular}
{\scriptsize
Navigate to the Portal Aspect endpoint. ~The stable version should be
used for this test and is currently located at:
https://lsst-lsp-stable.ncsa.illinois.edu/portal/app/ .

}
\begin{tabular}{p{3cm}p{13cm}}
\hline
            & Expected Result \\ \hline
\end{tabular}
{\scriptsize
A credential-entry screen should be displayed.

}

\begin{tabular}{p{4cm}p{12cm}}
\toprule
Step 2
& Description \\ \hline
\end{tabular}
{\scriptsize
Enter a valid set of credentials for an LSST user with LSP access on the
instance under test.

}
\begin{tabular}{p{3cm}p{13cm}}
\hline
            & Expected Result \\ \hline
\end{tabular}
{\scriptsize
The Portal Aspect UI should be displayed following authentication.

}



\subsection{LVV-T850 - Log out of the portal aspect of the LSP}\label{lvv-t850}

\begin{longtable}[]{llllll}
\toprule
Version & Status & Priority & Verification Type & Owner
\\\midrule
1 & Draft & Normal &
Test & Simon Krughoff
\\\bottomrule
\multicolumn{6}{c}{ Open \href{https://jira.lsstcorp.org/secure/Tests.jspa\#/testCase/LVV-T850}{LVV-T850} in Jira } \\
\end{longtable}

\paragraph{Test Items}\mbox{}\\
Leave the portal aspect of the LSST Science Platform in a clean state








\paragraph{Test Procedure}\mbox{}\\
\begin{tabular}{p{4cm}p{12cm}}
\toprule
Step 1
& Description \\ \hline
\end{tabular}
{\scriptsize
Currently, there is no logout mechanism on the portal.\\
This should be updated as the system matures.\\[2\baselineskip]Simply
close the browser window.

}
\begin{tabular}{p{3cm}p{13cm}}
\hline
            & Expected Result \\ \hline
\end{tabular}
{\scriptsize
Closed browser window. ~When navigating to the portal endpoint, expect
to execute the steps in LVV-T849.

}



\subsection{LVV-T860 - Initialize science pipelines}\label{lvv-t860}

\begin{longtable}[]{llllll}
\toprule
Version & Status & Priority & Verification Type & Owner
\\\midrule
1 & Draft & Normal &
Test & Jeffrey Carlin
\\\bottomrule
\multicolumn{6}{c}{ Open \href{https://jira.lsstcorp.org/secure/Tests.jspa\#/testCase/LVV-T860}{LVV-T860} in Jira } \\
\end{longtable}

\paragraph{Test Items}\mbox{}\\
Initialize the science pipelines software for use.~






\paragraph{Input Specification}\mbox{}\\
An installed software stack, either locally, on `lsst-dev`, or through
the Notebook aspect.


\paragraph{Test Procedure}\mbox{}\\
\begin{tabular}{p{4cm}p{12cm}}
\toprule
Step 1
& Description \\ \hline
\end{tabular}
{\scriptsize
The `path` that you will use depends on where you are running the
science pipelines. Options:\\[2\baselineskip]

\begin{itemize}
\tightlist
\item
  local (newinstall.sh - based
  install):{[}path\_to\_installation{]}/loadLSST.bash
\item
  development cluster (``lsst-dev''):
  /software/lsstsw/stack/loadLSST.bash
\item
  LSP Notebook aspect (from a terminal):
  /opt/lsst/software/stack/loadLSST.bash
\end{itemize}

From the command line, execute the commands below in the example
code:\\[2\baselineskip]

}
\begin{tabular}{p{3cm}p{13cm}}
\hline
            & Example Code \\ \hline
\end{tabular}
{\scriptsize
source `path`\\
setup lsst\_distrib

}
\begin{tabular}{p{3cm}p{13cm}}
\hline
            & Expected Result \\ \hline
\end{tabular}
{\scriptsize
Science pipeline software is available for use. If additional packages
are needed (for example, `obs' packages such as `obs\_subaru`), then
additional `setup` commands will be necessary.\\[2\baselineskip]To check
versions in use, type:\\
eups list -s

}



\subsection{LVV-T866 - Run Alert Production Payload}\label{lvv-t866}

\begin{longtable}[]{llllll}
\toprule
Version & Status & Priority & Verification Type & Owner
\\\midrule
1 & Draft & Normal &
Test & Jeffrey Carlin
\\\bottomrule
\multicolumn{6}{c}{ Open \href{https://jira.lsstcorp.org/secure/Tests.jspa\#/testCase/LVV-T866}{LVV-T866} in Jira } \\
\end{longtable}

\paragraph{Test Items}\mbox{}\\
Execute Alert Production payload on a dataset. Generate all (or a subset
of) Prompt science data products including Alerts (with the exception of
Solar System object orbits) and load them into the Data Backbone and
Prompt Products Database. ~~








\paragraph{Test Procedure}\mbox{}\\
\begin{tabular}{p{4cm}p{12cm}}
\toprule
Step 1
& Description \\ \hline
\end{tabular}
{\scriptsize
Perform the steps of Alert Production (including, but not necessarily
limited to, single frame processing, ISR, source detection/measurement,
PSF estimation, photometric and astrometric calibration, difference
imaging, DIASource detection/measurement, source association). During
Operations, it is presumed that these are automated for a given
dataset.~

}
\begin{tabular}{p{3cm}p{13cm}}
\hline
            & Expected Result \\ \hline
\end{tabular}
{\scriptsize
An output dataset including difference images and DIASource and
DIAObject measurements.

}

\begin{tabular}{p{4cm}p{12cm}}
\toprule
Step 2
& Description \\ \hline
\end{tabular}
{\scriptsize
Verify that the expected data products have been produced, and that
catalogs contain reasonable values for measured quantities of interest.

}
\begin{tabular}{p{3cm}p{13cm}}
\hline
            & Expected Result \\ \hline
\end{tabular}



\subsection{LVV-T901 - Run MOPS payload}\label{lvv-t901}

\begin{longtable}[]{llllll}
\toprule
Version & Status & Priority & Verification Type & Owner
\\\midrule
1 & Draft & Normal &
Test & Jeffrey Carlin
\\\bottomrule
\multicolumn{6}{c}{ Open \href{https://jira.lsstcorp.org/secure/Tests.jspa\#/testCase/LVV-T901}{LVV-T901} in Jira } \\
\end{longtable}

\paragraph{Test Items}\mbox{}\\
Run MOPS payload on a dataset (for example, one night's data). Generate
entries in the MOPS Database and the Prompt Products Database, including
Solar System Object records, measurements, and orbits. Perform precovery
forced photometry of transients.


\paragraph{Predecessors}\mbox{}\\
Uses results loaded into Prompt Products database and Data Backbone
services in
\href{https://jira.lsstcorp.org/secure/Tests.jspa\#/testCase/LVV-T866}{LVV-T866}.






\paragraph{Test Procedure}\mbox{}\\
\begin{tabular}{p{4cm}p{12cm}}
\toprule
Step 1
& Description \\ \hline
\end{tabular}
{\scriptsize
Perform the steps of Moving Object Pipeline (MOPS) processing on newly
detected DIASources, and generate Solar System data products including
Solar System objects with associated Keplerian orbits, errors, and
detected DIASources. This includes running processes to link DIASource
detections within a night (called tracklets), to link these tracklets
across multiple nights (into tracks), to fit the tracks with an orbital
model to identify those tracks that are consistent with an asteroid
orbit, to match these new orbits with existing SSObjects, and to update
the SSObject table. ~ ~ ~ ~ ~ ~ ~ ~ ~ ~ ~ ~ ~ ~ ~ ~ ~ ~ ~~

}
\begin{tabular}{p{3cm}p{13cm}}
\hline
            & Expected Result \\ \hline
\end{tabular}
{\scriptsize
An output dataset consisting of an updated SSObject database with
SSObjects both added and pruned as the orbital fits have been refined,
and an updated DIASource database with DIASources assigned and
unassigned to SSObjects.

}

\begin{tabular}{p{4cm}p{12cm}}
\toprule
Step 2
& Description \\ \hline
\end{tabular}
{\scriptsize
Verify that the expected data products have been produced, and that
catalogs contain reasonable values for measured quantities of interest.

}
\begin{tabular}{p{3cm}p{13cm}}
\hline
            & Expected Result \\ \hline
\end{tabular}



\subsection{LVV-T987 - Instantiate the Butler for reading data}\label{lvv-t987}

\begin{longtable}[]{llllll}
\toprule
Version & Status & Priority & Verification Type & Owner
\\\midrule
1 & Draft & Normal &
Test & Jeffrey Carlin
\\\bottomrule
\multicolumn{6}{c}{ Open \href{https://jira.lsstcorp.org/secure/Tests.jspa\#/testCase/LVV-T987}{LVV-T987} in Jira } \\
\end{longtable}

\paragraph{Test Items}\mbox{}\\
Create a Butler client to read data from an input repository.






\paragraph{Input Specification}\mbox{}\\
\href{https://jira.lsstcorp.org/secure/Tests.jspa\#/testCase/LVV-T860}{LVV-T860}
must be executed to initialize the science pipelines.


\paragraph{Test Procedure}\mbox{}\\
\begin{tabular}{p{4cm}p{12cm}}
\toprule
Step 1
& Description \\ \hline
\end{tabular}
{\scriptsize
Identify the path to the data repository, which we will refer to as
`DATA/path', then execute the following:

}
\begin{tabular}{p{3cm}p{13cm}}
\hline
            & Example Code \\ \hline
\end{tabular}
{\scriptsize
\begin{verbatim}
import lsst.daf.persistence as dafPersist
butler = dafPersist.Butler(inputs='DATA/path')
\end{verbatim}

}
\begin{tabular}{p{3cm}p{13cm}}
\hline
            & Expected Result \\ \hline
\end{tabular}
{\scriptsize
Butler repo available for reading.

}



\subsection{LVV-T1059 - Run Daily Calibration Products Update Payload}\label{lvv-t1059}

\begin{longtable}[]{llllll}
\toprule
Version & Status & Priority & Verification Type & Owner
\\\midrule
1 & Draft & Normal &
Test & Jeffrey Carlin
\\\bottomrule
\multicolumn{6}{c}{ Open \href{https://jira.lsstcorp.org/secure/Tests.jspa\#/testCase/LVV-T1059}{LVV-T1059} in Jira } \\
\end{longtable}

\paragraph{Test Items}\mbox{}\\
Execute the Daily Calibration Products Update payload to create a subset
of Master Calibration images and Calibration Database entries.








\paragraph{Test Procedure}\mbox{}\\
\begin{tabular}{p{4cm}p{12cm}}
\toprule
Step 1
& Description \\ \hline
\end{tabular}
{\scriptsize
Execute the Daily Calibration Products Update payload. The payload uses
raw calibration images and information from the Transformed EFD to
generate a subset of Master Calibration Images and Calibration Database
entries in the Data Backbone.

}
\begin{tabular}{p{3cm}p{13cm}}
\hline
            & Expected Result \\ \hline
\end{tabular}

\begin{tabular}{p{4cm}p{12cm}}
\toprule
Step 2
& Description \\ \hline
\end{tabular}
{\scriptsize
Confirm that the expected Master Calibration images and Calibration
Database entries are present and well-formed.

}
\begin{tabular}{p{3cm}p{13cm}}
\hline
            & Expected Result \\ \hline
\end{tabular}



\subsection{LVV-T1060 - Run Periodic Calibration Products Production Payload}\label{lvv-t1060}

\begin{longtable}[]{llllll}
\toprule
Version & Status & Priority & Verification Type & Owner
\\\midrule
1 & Draft & Normal &
Test & Jeffrey Carlin
\\\bottomrule
\multicolumn{6}{c}{ Open \href{https://jira.lsstcorp.org/secure/Tests.jspa\#/testCase/LVV-T1060}{LVV-T1060} in Jira } \\
\end{longtable}

\paragraph{Test Items}\mbox{}\\
Execute the Calibration Products Production payload to create a subset
of Master Calibration images and Calibration Database entries.








\paragraph{Test Procedure}\mbox{}\\
\begin{tabular}{p{4cm}p{12cm}}
\toprule
Step 1
& Description \\ \hline
\end{tabular}
{\scriptsize
Execute the Calibration Products Production payload. The payload uses
raw calibration images and information from the Transformed EFD to
generate a subset of Master Calibration Images and Calibration Database
entries in the Data Backbone.

}
\begin{tabular}{p{3cm}p{13cm}}
\hline
            & Expected Result \\ \hline
\end{tabular}

\begin{tabular}{p{4cm}p{12cm}}
\toprule
Step 2
& Description \\ \hline
\end{tabular}
{\scriptsize
Confirm that the expected Master Calibration images and Calibration
Database entries are present and well-formed.

}
\begin{tabular}{p{3cm}p{13cm}}
\hline
            & Expected Result \\ \hline
\end{tabular}



\subsection{LVV-T1064 - Run Data Release Production Payload}\label{lvv-t1064}

\begin{longtable}[]{llllll}
\toprule
Version & Status & Priority & Verification Type & Owner
\\\midrule
1 & Draft & Normal &
Test & Jeffrey Carlin
\\\bottomrule
\multicolumn{6}{c}{ Open \href{https://jira.lsstcorp.org/secure/Tests.jspa\#/testCase/LVV-T1064}{LVV-T1064} in Jira } \\
\end{longtable}

\paragraph{Test Items}\mbox{}\\
Execute the Data Release Production payload, starting from raw images
and producing science data products.








\paragraph{Test Procedure}\mbox{}\\
\begin{tabular}{p{4cm}p{12cm}}
\toprule
Step 1
& Description \\ \hline
\end{tabular}
{\scriptsize
Process data with the Data Release Production payload, starting from raw
science images and generating science data products, placing them in the
Data Backbone.

}
\begin{tabular}{p{3cm}p{13cm}}
\hline
            & Expected Result \\ \hline
\end{tabular}



\subsection{LVV-T1207 - Execute a simple ADQL query using the TAP service in the notebook aspect}\label{lvv-t1207}

\begin{longtable}[]{llllll}
\toprule
Version & Status & Priority & Verification Type & Owner
\\\midrule
1 & Draft & Normal &
Test & Jeffrey Carlin
\\\bottomrule
\multicolumn{6}{c}{ Open \href{https://jira.lsstcorp.org/secure/Tests.jspa\#/testCase/LVV-T1207}{LVV-T1207} in Jira } \\
\end{longtable}

\paragraph{Test Items}\mbox{}\\
Extract a small amount of data from a catalog via the LSST TAP service.






\paragraph{Input Specification}\mbox{}\\
One must have access to the LSST Notebook Aspect, and have logged in and
opened an empty notebook.


\paragraph{Test Procedure}\mbox{}\\
\begin{tabular}{p{4cm}p{12cm}}
\toprule
Step 1
& Description \\ \hline
\end{tabular}
{\scriptsize
Execute a query in a notebook to select a small number of stars. In the
example code below, we query the WISE catalog, then extract the results
to an Astropy table.

}
\begin{tabular}{p{3cm}p{13cm}}
\hline
            & Example Code \\ \hline
\end{tabular}
{\scriptsize
\begin{verbatim}
import pandas
import pyvo
service = pyvo.dal.TAPService('http://lsst-lsp-stable.ncsa.illinois.edu/api/tap')
\end{verbatim}

results = service.search(``SELECT ra, decl, w1mpro\_ep, w2mpro\_ep,
w3mpro\_ep FROM wise\_00.allwise\_p3as\_mep WHERE CONTAINS(POINT('ICRS',
ra, decl), CIRCLE('ICRS', 192.85, 27.13, .2)) = 1'')\\
tab = results.to\_table()

}
\begin{tabular}{p{3cm}p{13cm}}
\hline
            & Expected Result \\ \hline
\end{tabular}



\subsection{LVV-T1208 - Log out of the notebook aspect of the LSP}\label{lvv-t1208}

\begin{longtable}[]{llllll}
\toprule
Version & Status & Priority & Verification Type & Owner
\\\midrule
1 & Draft & Normal &
Test & Simon Krughoff
\\\bottomrule
\multicolumn{6}{c}{ Open \href{https://jira.lsstcorp.org/secure/Tests.jspa\#/testCase/LVV-T1208}{LVV-T1208} in Jira } \\
\end{longtable}

\paragraph{Test Items}\mbox{}\\
Leave the notebook aspect of the LSST Science Platform in a clean state








\paragraph{Test Procedure}\mbox{}\\
\begin{tabular}{p{4cm}p{12cm}}
\toprule
Step 1
& Description \\ \hline
\end{tabular}
{\scriptsize
Under the `File' menu at the top of your Jupyter notebook session,
select one of the following:\\[2\baselineskip]

\begin{itemize}
\tightlist
\item
  Save All, Exit, and Log Out
\item
  Exit and Log Out Without Saving
\end{itemize}

}
\begin{tabular}{p{3cm}p{13cm}}
\hline
            & Expected Result \\ \hline
\end{tabular}
{\scriptsize
You will be returned to the LSP landing page:
\url{https://lsst-lsp-stable.ncsa.illinois.edu/} It is now safe to close
the browser window.~

}



\subsection{LVV-T1744 - Run validate\_drp on precursor data}\label{lvv-t1744}

\begin{longtable}[]{llllll}
\toprule
Version & Status & Priority & Verification Type & Owner
\\\midrule
1 & Defined & Normal &
Analysis & Jeffrey Carlin
\\\bottomrule
\multicolumn{6}{c}{ Open \href{https://jira.lsstcorp.org/secure/Tests.jspa\#/testCase/LVV-T1744}{LVV-T1744} in Jira } \\
\end{longtable}

\paragraph{Test Items}\mbox{}\\
Run the validate\_drp code on a precursor dataset to evaluate the
metrics that have been implemented in validate\_drp.








\paragraph{Test Procedure}\mbox{}\\
\begin{tabular}{p{4cm}p{12cm}}
\toprule
Step 1
& Description \\ \hline
\end{tabular}
{\scriptsize
Execute `validate\_drp` on a repository containing precursor data.
Identify the path to the data, which we will call `DATA/path', then
execute the following (with additional flags specified as needed):

}
\begin{tabular}{p{3cm}p{13cm}}
\hline
            & Example Code \\ \hline
\end{tabular}
{\scriptsize
validateDrp.py `DATA/path`

}
\begin{tabular}{p{3cm}p{13cm}}
\hline
            & Expected Result \\ \hline
\end{tabular}
{\scriptsize
JSON files (and associated figures) containing the Measurements and any
associated ``extras.''

}





\newpage
\section{Deprecated Test Cases}

This section includes all test cases that have been marked as deprecated.
These test cases will never be executed again, but have been in the past.
For this reason it is important to keep them in the baseline as a reference.

  \textit{No deprecated test cases found.}

\newpage
\appendix
