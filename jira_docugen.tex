% generated from JIRA project LVV
% using template at /usr/local/lib/python3.7/site-packages/docsteady/templates/dm-spec.latex.jinja2.
% Collecting ATM data from folder: "/Data Management/Acceptance|LDM-639"
% using docsteady version 1.2rc24
% Please do not edit -- update information in Jira instead

\section{Test Cases Summary}\label{test-cases-summary}

\begin{longtable}[]{p{3cm}p{13cm}}
\toprule
Test Id & Test Name\tabularnewline
\midrule
\endhead
    \hyperref[lvv-t10]{LVV-T10} &
    \href{https://jira.lsstcorp.org/secure/Tests.jspa\#/testCase/LVV-T10}{DRP-00-00: Installation of the Data Release Production v14.0 science
payload} \tabularnewline
    \hyperref[lvv-t11]{LVV-T11} &
    \href{https://jira.lsstcorp.org/secure/Tests.jspa\#/testCase/LVV-T11}{DRP-00-05: Execution of the DRP Science Payload by the Batch Production
Service} \tabularnewline
    \hyperref[lvv-t12]{LVV-T12} &
    \href{https://jira.lsstcorp.org/secure/Tests.jspa\#/testCase/LVV-T12}{DRP-00-10: Data Release Includes Required Data Products} \tabularnewline
    \hyperref[lvv-t13]{LVV-T13} &
    \href{https://jira.lsstcorp.org/secure/Tests.jspa\#/testCase/LVV-T13}{DRP-00-15: Scientific Verification of Source Catalog} \tabularnewline
    \hyperref[lvv-t14]{LVV-T14} &
    \href{https://jira.lsstcorp.org/secure/Tests.jspa\#/testCase/LVV-T14}{DRP-00-25: Scientific Verification of Object Catalog} \tabularnewline
    \hyperref[lvv-t15]{LVV-T15} &
    \href{https://jira.lsstcorp.org/secure/Tests.jspa\#/testCase/LVV-T15}{DRP-00-30: Scientific Verification of Processed Visit Images} \tabularnewline
    \hyperref[lvv-t16]{LVV-T16} &
    \href{https://jira.lsstcorp.org/secure/Tests.jspa\#/testCase/LVV-T16}{DRP-00-35: Scientific Verification of Coadd Images} \tabularnewline
    \hyperref[lvv-t17]{LVV-T17} &
    \href{https://jira.lsstcorp.org/secure/Tests.jspa\#/testCase/LVV-T17}{AG-00-00: Installation of the Alert Generation v16.0 science payload.} \tabularnewline
    \hyperref[lvv-t18]{LVV-T18} &
    \href{https://jira.lsstcorp.org/secure/Tests.jspa\#/testCase/LVV-T18}{AG-00-05: Alert Generation Produces Required Data Products} \tabularnewline
    \hyperref[lvv-t19]{LVV-T19} &
    \href{https://jira.lsstcorp.org/secure/Tests.jspa\#/testCase/LVV-T19}{AG-00-10: Scientific Verification of Processed Visit Images} \tabularnewline
    \hyperref[lvv-t20]{LVV-T20} &
    \href{https://jira.lsstcorp.org/secure/Tests.jspa\#/testCase/LVV-T20}{AG-00-15: Scientific Verification of Difference Images} \tabularnewline
    \hyperref[lvv-t21]{LVV-T21} &
    \href{https://jira.lsstcorp.org/secure/Tests.jspa\#/testCase/LVV-T21}{AG-00-20: Scientific Verification of DIASource Catalog} \tabularnewline
    \hyperref[lvv-t22]{LVV-T22} &
    \href{https://jira.lsstcorp.org/secure/Tests.jspa\#/testCase/LVV-T22}{AG-00-25: Scientific Verification of DIAObject Catalog} \tabularnewline
    \hyperref[lvv-t23]{LVV-T23} &
    \href{https://jira.lsstcorp.org/secure/Tests.jspa\#/testCase/LVV-T23}{Verify implementation of Storing Approximations of Per-pixel Metadata} \tabularnewline
    \hyperref[lvv-t24]{LVV-T24} &
    \href{https://jira.lsstcorp.org/secure/Tests.jspa\#/testCase/LVV-T24}{Verify implementation of Computing Derived Quantities} \tabularnewline
    \hyperref[lvv-t25]{LVV-T25} &
    \href{https://jira.lsstcorp.org/secure/Tests.jspa\#/testCase/LVV-T25}{Verify implementation of Denormalizing Database Tables} \tabularnewline
    \hyperref[lvv-t26]{LVV-T26} &
    \href{https://jira.lsstcorp.org/secure/Tests.jspa\#/testCase/LVV-T26}{Verify implementation of Maximum Likelihood Values and Covariances} \tabularnewline
    \hyperref[lvv-t27]{LVV-T27} &
    \href{https://jira.lsstcorp.org/secure/Tests.jspa\#/testCase/LVV-T27}{Verify implementation of Data Availability} \tabularnewline
    \hyperref[lvv-t28]{LVV-T28} &
    \href{https://jira.lsstcorp.org/secure/Tests.jspa\#/testCase/LVV-T28}{Verify implementation of measurements in catalogs from PVIs} \tabularnewline
    \hyperref[lvv-t29]{LVV-T29} &
    \href{https://jira.lsstcorp.org/secure/Tests.jspa\#/testCase/LVV-T29}{Verify implementation of Raw Science Image Data Acquisition} \tabularnewline
    \hyperref[lvv-t30]{LVV-T30} &
    \href{https://jira.lsstcorp.org/secure/Tests.jspa\#/testCase/LVV-T30}{Verify implementation of Wavefront Sensor Data Acquisition} \tabularnewline
    \hyperref[lvv-t31]{LVV-T31} &
    \href{https://jira.lsstcorp.org/secure/Tests.jspa\#/testCase/LVV-T31}{Verify implementation of Crosstalk Corrected Science Image Data
Acquisition} \tabularnewline
    \hyperref[lvv-t32]{LVV-T32} &
    \href{https://jira.lsstcorp.org/secure/Tests.jspa\#/testCase/LVV-T32}{Verify implementation of Raw Image Assembly} \tabularnewline
    \hyperref[lvv-t33]{LVV-T33} &
    \href{https://jira.lsstcorp.org/secure/Tests.jspa\#/testCase/LVV-T33}{Verify implementation of Raw Science Image Metadata} \tabularnewline
    \hyperref[lvv-t34]{LVV-T34} &
    \href{https://jira.lsstcorp.org/secure/Tests.jspa\#/testCase/LVV-T34}{Verify implementation of Guider Calibration Data Acquisition} \tabularnewline
    \hyperref[lvv-t35]{LVV-T35} &
    \href{https://jira.lsstcorp.org/secure/Tests.jspa\#/testCase/LVV-T35}{Verify implementation of Nightly Data Accessible Within 24 hrs} \tabularnewline
    \hyperref[lvv-t36]{LVV-T36} &
    \href{https://jira.lsstcorp.org/secure/Tests.jspa\#/testCase/LVV-T36}{Verify implementation of Difference Exposures} \tabularnewline
    \hyperref[lvv-t37]{LVV-T37} &
    \href{https://jira.lsstcorp.org/secure/Tests.jspa\#/testCase/LVV-T37}{Verify implementation of Difference Exposure Attributes} \tabularnewline
    \hyperref[lvv-t38]{LVV-T38} &
    \href{https://jira.lsstcorp.org/secure/Tests.jspa\#/testCase/LVV-T38}{Verify implementation of Processed Visit Images} \tabularnewline
    \hyperref[lvv-t39]{LVV-T39} &
    \href{https://jira.lsstcorp.org/secure/Tests.jspa\#/testCase/LVV-T39}{Verify implementation of Generate Photometric Zeropoint for Visit Image} \tabularnewline
    \hyperref[lvv-t40]{LVV-T40} &
    \href{https://jira.lsstcorp.org/secure/Tests.jspa\#/testCase/LVV-T40}{Verify implementation of Generate WCS for Visit Images} \tabularnewline
    \hyperref[lvv-t41]{LVV-T41} &
    \href{https://jira.lsstcorp.org/secure/Tests.jspa\#/testCase/LVV-T41}{Verify implementation of Generate PSF for Visit Images} \tabularnewline
    \hyperref[lvv-t42]{LVV-T42} &
    \href{https://jira.lsstcorp.org/secure/Tests.jspa\#/testCase/LVV-T42}{Verify implementation of Processed Visit Image Content} \tabularnewline
    \hyperref[lvv-t43]{LVV-T43} &
    \href{https://jira.lsstcorp.org/secure/Tests.jspa\#/testCase/LVV-T43}{Verify implementation of Background Model Calculation} \tabularnewline
    \hyperref[lvv-t44]{LVV-T44} &
    \href{https://jira.lsstcorp.org/secure/Tests.jspa\#/testCase/LVV-T44}{Verify implementation of Documenting Image Characterization} \tabularnewline
    \hyperref[lvv-t45]{LVV-T45} &
    \href{https://jira.lsstcorp.org/secure/Tests.jspa\#/testCase/LVV-T45}{Verify implementation of Prompt Processing Data Quality Report
Definition} \tabularnewline
    \hyperref[lvv-t46]{LVV-T46} &
    \href{https://jira.lsstcorp.org/secure/Tests.jspa\#/testCase/LVV-T46}{Verify implementation of Prompt Processing Performance Report Definition} \tabularnewline
    \hyperref[lvv-t47]{LVV-T47} &
    \href{https://jira.lsstcorp.org/secure/Tests.jspa\#/testCase/LVV-T47}{Verify implementation of Prompt Processing Calibration Report Definition} \tabularnewline
    \hyperref[lvv-t48]{LVV-T48} &
    \href{https://jira.lsstcorp.org/secure/Tests.jspa\#/testCase/LVV-T48}{Verify implementation of Exposure Catalog} \tabularnewline
    \hyperref[lvv-t49]{LVV-T49} &
    \href{https://jira.lsstcorp.org/secure/Tests.jspa\#/testCase/LVV-T49}{Verify implementation of DIASource Catalog} \tabularnewline
    \hyperref[lvv-t50]{LVV-T50} &
    \href{https://jira.lsstcorp.org/secure/Tests.jspa\#/testCase/LVV-T50}{Verify implementation of Faint DIASource Measurements} \tabularnewline
    \hyperref[lvv-t51]{LVV-T51} &
    \href{https://jira.lsstcorp.org/secure/Tests.jspa\#/testCase/LVV-T51}{Verify implementation of DIAObject Catalog} \tabularnewline
    \hyperref[lvv-t52]{LVV-T52} &
    \href{https://jira.lsstcorp.org/secure/Tests.jspa\#/testCase/LVV-T52}{Verify implementation of DIAObject Attributes} \tabularnewline
    \hyperref[lvv-t53]{LVV-T53} &
    \href{https://jira.lsstcorp.org/secure/Tests.jspa\#/testCase/LVV-T53}{Verify implementation of SSObject Catalog} \tabularnewline
    \hyperref[lvv-t54]{LVV-T54} &
    \href{https://jira.lsstcorp.org/secure/Tests.jspa\#/testCase/LVV-T54}{Verify implementation of Alert Content} \tabularnewline
    \hyperref[lvv-t55]{LVV-T55} &
    \href{https://jira.lsstcorp.org/secure/Tests.jspa\#/testCase/LVV-T55}{Verify implementation of DIAForcedSource Catalog} \tabularnewline
    \hyperref[lvv-t56]{LVV-T56} &
    \href{https://jira.lsstcorp.org/secure/Tests.jspa\#/testCase/LVV-T56}{Verify implementation of Characterizing Variability} \tabularnewline
    \hyperref[lvv-t57]{LVV-T57} &
    \href{https://jira.lsstcorp.org/secure/Tests.jspa\#/testCase/LVV-T57}{Verify implementation of Calculating SSObject Parameters} \tabularnewline
    \hyperref[lvv-t58]{LVV-T58} &
    \href{https://jira.lsstcorp.org/secure/Tests.jspa\#/testCase/LVV-T58}{Verify implementation of Matching DIASources to Objects} \tabularnewline
    \hyperref[lvv-t59]{LVV-T59} &
    \href{https://jira.lsstcorp.org/secure/Tests.jspa\#/testCase/LVV-T59}{Verify implementation of Regenerating L1 Data Products During Data
Release Processing} \tabularnewline
    \hyperref[lvv-t60]{LVV-T60} &
    \href{https://jira.lsstcorp.org/secure/Tests.jspa\#/testCase/LVV-T60}{Verify implementation of Publishing predicted visit schedule} \tabularnewline
    \hyperref[lvv-t61]{LVV-T61} &
    \href{https://jira.lsstcorp.org/secure/Tests.jspa\#/testCase/LVV-T61}{Verify implementation of Associate Sources to Objects} \tabularnewline
    \hyperref[lvv-t62]{LVV-T62} &
    \href{https://jira.lsstcorp.org/secure/Tests.jspa\#/testCase/LVV-T62}{Verify implementation of Provide PSF for Coadded Images} \tabularnewline
    \hyperref[lvv-t63]{LVV-T63} &
    \href{https://jira.lsstcorp.org/secure/Tests.jspa\#/testCase/LVV-T63}{Verify implementation of Produce Images for EPO} \tabularnewline
    \hyperref[lvv-t64]{LVV-T64} &
    \href{https://jira.lsstcorp.org/secure/Tests.jspa\#/testCase/LVV-T64}{Verify implementation of Coadded Image Provenance} \tabularnewline
    \hyperref[lvv-t65]{LVV-T65} &
    \href{https://jira.lsstcorp.org/secure/Tests.jspa\#/testCase/LVV-T65}{Verify implementation of Source Catalog} \tabularnewline
    \hyperref[lvv-t66]{LVV-T66} &
    \href{https://jira.lsstcorp.org/secure/Tests.jspa\#/testCase/LVV-T66}{Verify implementation of Forced-Source Catalog} \tabularnewline
    \hyperref[lvv-t67]{LVV-T67} &
    \href{https://jira.lsstcorp.org/secure/Tests.jspa\#/testCase/LVV-T67}{Verify implementation of Object Catalog} \tabularnewline
    \hyperref[lvv-t68]{LVV-T68} &
    \href{https://jira.lsstcorp.org/secure/Tests.jspa\#/testCase/LVV-T68}{Verify implementation of Provide Photometric Redshifts of Galaxies} \tabularnewline
    \hyperref[lvv-t69]{LVV-T69} &
    \href{https://jira.lsstcorp.org/secure/Tests.jspa\#/testCase/LVV-T69}{Verify implementation of Object Characterization} \tabularnewline
    \hyperref[lvv-t71]{LVV-T71} &
    \href{https://jira.lsstcorp.org/secure/Tests.jspa\#/testCase/LVV-T71}{Verify implementation of Detecting extended low surface brightness
objects} \tabularnewline
    \hyperref[lvv-t72]{LVV-T72} &
    \href{https://jira.lsstcorp.org/secure/Tests.jspa\#/testCase/LVV-T72}{Verify implementation of Coadd Image Method Constraints} \tabularnewline
    \hyperref[lvv-t73]{LVV-T73} &
    \href{https://jira.lsstcorp.org/secure/Tests.jspa\#/testCase/LVV-T73}{Verify implementation of Deep Detection Coadds} \tabularnewline
    \hyperref[lvv-t74]{LVV-T74} &
    \href{https://jira.lsstcorp.org/secure/Tests.jspa\#/testCase/LVV-T74}{Verify implementation of Template Coadds} \tabularnewline
    \hyperref[lvv-t75]{LVV-T75} &
    \href{https://jira.lsstcorp.org/secure/Tests.jspa\#/testCase/LVV-T75}{Verify implementation of Multi-band Coadds} \tabularnewline
    \hyperref[lvv-t76]{LVV-T76} &
    \href{https://jira.lsstcorp.org/secure/Tests.jspa\#/testCase/LVV-T76}{Verify implementation of All-Sky Visualization of Data Releases} \tabularnewline
    \hyperref[lvv-t77]{LVV-T77} &
    \href{https://jira.lsstcorp.org/secure/Tests.jspa\#/testCase/LVV-T77}{Verify implementation of Best Seeing Coadds} \tabularnewline
    \hyperref[lvv-t78]{LVV-T78} &
    \href{https://jira.lsstcorp.org/secure/Tests.jspa\#/testCase/LVV-T78}{Verify implementation of Persisting Data Products} \tabularnewline
    \hyperref[lvv-t79]{LVV-T79} &
    \href{https://jira.lsstcorp.org/secure/Tests.jspa\#/testCase/LVV-T79}{Verify implementation of PSF-Matched Coadds} \tabularnewline
    \hyperref[lvv-t80]{LVV-T80} &
    \href{https://jira.lsstcorp.org/secure/Tests.jspa\#/testCase/LVV-T80}{Verify implementation of Detecting faint variable objects} \tabularnewline
    \hyperref[lvv-t81]{LVV-T81} &
    \href{https://jira.lsstcorp.org/secure/Tests.jspa\#/testCase/LVV-T81}{Verify implementation of Targeted Coadds} \tabularnewline
    \hyperref[lvv-t82]{LVV-T82} &
    \href{https://jira.lsstcorp.org/secure/Tests.jspa\#/testCase/LVV-T82}{Verify implementation of Tracking Characterization Changes Between Data
Releases} \tabularnewline
    \hyperref[lvv-t83]{LVV-T83} &
    \href{https://jira.lsstcorp.org/secure/Tests.jspa\#/testCase/LVV-T83}{Verify implementation of Bad Pixel Map} \tabularnewline
    \hyperref[lvv-t84]{LVV-T84} &
    \href{https://jira.lsstcorp.org/secure/Tests.jspa\#/testCase/LVV-T84}{Verify implementation of Bias Residual Image} \tabularnewline
    \hyperref[lvv-t85]{LVV-T85} &
    \href{https://jira.lsstcorp.org/secure/Tests.jspa\#/testCase/LVV-T85}{Verify implementation of Crosstalk Correction Matrix} \tabularnewline
    \hyperref[lvv-t86]{LVV-T86} &
    \href{https://jira.lsstcorp.org/secure/Tests.jspa\#/testCase/LVV-T86}{Verify implementation of Illumination Correction Frame} \tabularnewline
    \hyperref[lvv-t87]{LVV-T87} &
    \href{https://jira.lsstcorp.org/secure/Tests.jspa\#/testCase/LVV-T87}{Verify implementation of Monochromatic Flatfield Data Cube} \tabularnewline
    \hyperref[lvv-t88]{LVV-T88} &
    \href{https://jira.lsstcorp.org/secure/Tests.jspa\#/testCase/LVV-T88}{Verify implementation of Calibration Data Products} \tabularnewline
    \hyperref[lvv-t89]{LVV-T89} &
    \href{https://jira.lsstcorp.org/secure/Tests.jspa\#/testCase/LVV-T89}{Verify implementation of Calibration Image Provenance} \tabularnewline
    \hyperref[lvv-t90]{LVV-T90} &
    \href{https://jira.lsstcorp.org/secure/Tests.jspa\#/testCase/LVV-T90}{Verify implementation of Dark Current Correction Frame} \tabularnewline
    \hyperref[lvv-t91]{LVV-T91} &
    \href{https://jira.lsstcorp.org/secure/Tests.jspa\#/testCase/LVV-T91}{Verify implementation of Fringe Correction Frame} \tabularnewline
    \hyperref[lvv-t92]{LVV-T92} &
    \href{https://jira.lsstcorp.org/secure/Tests.jspa\#/testCase/LVV-T92}{Verify implementation of Processing of Data From Special Programs} \tabularnewline
    \hyperref[lvv-t93]{LVV-T93} &
    \href{https://jira.lsstcorp.org/secure/Tests.jspa\#/testCase/LVV-T93}{Verify implementation of Level 1 Processing of Special Programs Data} \tabularnewline
    \hyperref[lvv-t94]{LVV-T94} &
    \href{https://jira.lsstcorp.org/secure/Tests.jspa\#/testCase/LVV-T94}{Verify implementation of Special Programs Database} \tabularnewline
    \hyperref[lvv-t95]{LVV-T95} &
    \href{https://jira.lsstcorp.org/secure/Tests.jspa\#/testCase/LVV-T95}{Verify implementation of Constraints on Level 1 Special Program Products
Generation} \tabularnewline
    \hyperref[lvv-t96]{LVV-T96} &
    \href{https://jira.lsstcorp.org/secure/Tests.jspa\#/testCase/LVV-T96}{Verify implementation of Query Repeatability} \tabularnewline
    \hyperref[lvv-t97]{LVV-T97} &
    \href{https://jira.lsstcorp.org/secure/Tests.jspa\#/testCase/LVV-T97}{Verify implementation of Uniqueness of IDs Across Data Releases} \tabularnewline
    \hyperref[lvv-t98]{LVV-T98} &
    \href{https://jira.lsstcorp.org/secure/Tests.jspa\#/testCase/LVV-T98}{Verify implementation of Selection of Datasets} \tabularnewline
    \hyperref[lvv-t99]{LVV-T99} &
    \href{https://jira.lsstcorp.org/secure/Tests.jspa\#/testCase/LVV-T99}{Verify implementation of Processing of Datasets} \tabularnewline
    \hyperref[lvv-t100]{LVV-T100} &
    \href{https://jira.lsstcorp.org/secure/Tests.jspa\#/testCase/LVV-T100}{Verify implementation of Transparent Data Access} \tabularnewline
    \hyperref[lvv-t101]{LVV-T101} &
    \href{https://jira.lsstcorp.org/secure/Tests.jspa\#/testCase/LVV-T101}{Verify implementation of Transient Alert Distribution} \tabularnewline
    \hyperref[lvv-t102]{LVV-T102} &
    \href{https://jira.lsstcorp.org/secure/Tests.jspa\#/testCase/LVV-T102}{Verify implementation of Solar System Objects Available Within Specified
Time} \tabularnewline
    \hyperref[lvv-t103]{LVV-T103} &
    \href{https://jira.lsstcorp.org/secure/Tests.jspa\#/testCase/LVV-T103}{Verify implementation of Generate Data Quality Report Within Specified
Time} \tabularnewline
    \hyperref[lvv-t104]{LVV-T104} &
    \href{https://jira.lsstcorp.org/secure/Tests.jspa\#/testCase/LVV-T104}{Verify implementation of Generate DMS Performance Report Within
Specified Time} \tabularnewline
    \hyperref[lvv-t105]{LVV-T105} &
    \href{https://jira.lsstcorp.org/secure/Tests.jspa\#/testCase/LVV-T105}{Verify implementation of Generate Calibration Report Within Specified
Time} \tabularnewline
    \hyperref[lvv-t106]{LVV-T106} &
    \href{https://jira.lsstcorp.org/secure/Tests.jspa\#/testCase/LVV-T106}{Verify implementation of Calibration Images Available Within Specified
Time} \tabularnewline
    \hyperref[lvv-t107]{LVV-T107} &
    \href{https://jira.lsstcorp.org/secure/Tests.jspa\#/testCase/LVV-T107}{Verify implementation of Level-1 Production Completeness} \tabularnewline
    \hyperref[lvv-t108]{LVV-T108} &
    \href{https://jira.lsstcorp.org/secure/Tests.jspa\#/testCase/LVV-T108}{Verify implementation of Level 1 Source Association} \tabularnewline
    \hyperref[lvv-t109]{LVV-T109} &
    \href{https://jira.lsstcorp.org/secure/Tests.jspa\#/testCase/LVV-T109}{Verify implementation of SSObject Precovery} \tabularnewline
    \hyperref[lvv-t110]{LVV-T110} &
    \href{https://jira.lsstcorp.org/secure/Tests.jspa\#/testCase/LVV-T110}{Verify implementation of DIASource Precovery} \tabularnewline
    \hyperref[lvv-t111]{LVV-T111} &
    \href{https://jira.lsstcorp.org/secure/Tests.jspa\#/testCase/LVV-T111}{Verify implementation of Use of External Orbit Catalogs} \tabularnewline
    \hyperref[lvv-t112]{LVV-T112} &
    \href{https://jira.lsstcorp.org/secure/Tests.jspa\#/testCase/LVV-T112}{Verify implementation of Alert Filtering Service} \tabularnewline
    \hyperref[lvv-t113]{LVV-T113} &
    \href{https://jira.lsstcorp.org/secure/Tests.jspa\#/testCase/LVV-T113}{Verify implementation of Performance Requirements for LSST Alert
Filtering Service} \tabularnewline
    \hyperref[lvv-t114]{LVV-T114} &
    \href{https://jira.lsstcorp.org/secure/Tests.jspa\#/testCase/LVV-T114}{Verify implementation of Pre-defined alert filters} \tabularnewline
    \hyperref[lvv-t115]{LVV-T115} &
    \href{https://jira.lsstcorp.org/secure/Tests.jspa\#/testCase/LVV-T115}{Verify implementation of Calibration Production Processing} \tabularnewline
    \hyperref[lvv-t116]{LVV-T116} &
    \href{https://jira.lsstcorp.org/secure/Tests.jspa\#/testCase/LVV-T116}{Verify implementation of Associating Objects across data releases} \tabularnewline
    \hyperref[lvv-t117]{LVV-T117} &
    \href{https://jira.lsstcorp.org/secure/Tests.jspa\#/testCase/LVV-T117}{Verify implementation of DAC resource allocation for Level 3 processing} \tabularnewline
    \hyperref[lvv-t118]{LVV-T118} &
    \href{https://jira.lsstcorp.org/secure/Tests.jspa\#/testCase/LVV-T118}{Verify implementation of Level 3 Data Product Self Consistency} \tabularnewline
    \hyperref[lvv-t119]{LVV-T119} &
    \href{https://jira.lsstcorp.org/secure/Tests.jspa\#/testCase/LVV-T119}{Verify implementation of Provenance for Level 3 processing at DACs} \tabularnewline
    \hyperref[lvv-t120]{LVV-T120} &
    \href{https://jira.lsstcorp.org/secure/Tests.jspa\#/testCase/LVV-T120}{Verify implementation of Software framework for Level 3 catalog
processing} \tabularnewline
    \hyperref[lvv-t121]{LVV-T121} &
    \href{https://jira.lsstcorp.org/secure/Tests.jspa\#/testCase/LVV-T121}{Verify implementation of Software framework for Level 3 image processing} \tabularnewline
    \hyperref[lvv-t122]{LVV-T122} &
    \href{https://jira.lsstcorp.org/secure/Tests.jspa\#/testCase/LVV-T122}{Verify implementation of Level 3 Data Import} \tabularnewline
    \hyperref[lvv-t123]{LVV-T123} &
    \href{https://jira.lsstcorp.org/secure/Tests.jspa\#/testCase/LVV-T123}{Verify implementation of Access Controls of Level 3 Data Products} \tabularnewline
    \hyperref[lvv-t124]{LVV-T124} &
    \href{https://jira.lsstcorp.org/secure/Tests.jspa\#/testCase/LVV-T124}{Verify implementation of Software Architecture to Enable Community
Re-Use} \tabularnewline
    \hyperref[lvv-t125]{LVV-T125} &
    \href{https://jira.lsstcorp.org/secure/Tests.jspa\#/testCase/LVV-T125}{Verify implementation of Simulated Data} \tabularnewline
    \hyperref[lvv-t126]{LVV-T126} &
    \href{https://jira.lsstcorp.org/secure/Tests.jspa\#/testCase/LVV-T126}{Verify implementation of Image Differencing} \tabularnewline
    \hyperref[lvv-t127]{LVV-T127} &
    \href{https://jira.lsstcorp.org/secure/Tests.jspa\#/testCase/LVV-T127}{Verify implementation of Provide Source Detection Software} \tabularnewline
    \hyperref[lvv-t128]{LVV-T128} &
    \href{https://jira.lsstcorp.org/secure/Tests.jspa\#/testCase/LVV-T128}{Verify implementation Provide Astrometric Model} \tabularnewline
    \hyperref[lvv-t129]{LVV-T129} &
    \href{https://jira.lsstcorp.org/secure/Tests.jspa\#/testCase/LVV-T129}{Verify implementation of Provide Calibrated Photometry} \tabularnewline
    \hyperref[lvv-t130]{LVV-T130} &
    \href{https://jira.lsstcorp.org/secure/Tests.jspa\#/testCase/LVV-T130}{Verify implementation of Enable a Range of Shape Measurement Approaches} \tabularnewline
    \hyperref[lvv-t131]{LVV-T131} &
    \href{https://jira.lsstcorp.org/secure/Tests.jspa\#/testCase/LVV-T131}{Verify implementation of Provide User Interface Services} \tabularnewline
    \hyperref[lvv-t132]{LVV-T132} &
    \href{https://jira.lsstcorp.org/secure/Tests.jspa\#/testCase/LVV-T132}{Verify implementation of Pre-cursor and Real Data} \tabularnewline
    \hyperref[lvv-t133]{LVV-T133} &
    \href{https://jira.lsstcorp.org/secure/Tests.jspa\#/testCase/LVV-T133}{Verify implementation of Provide Beam Projector Coordinate Calculation
Software} \tabularnewline
    \hyperref[lvv-t134]{LVV-T134} &
    \href{https://jira.lsstcorp.org/secure/Tests.jspa\#/testCase/LVV-T134}{Verify implementation of Provide Image Access Services} \tabularnewline
    \hyperref[lvv-t136]{LVV-T136} &
    \href{https://jira.lsstcorp.org/secure/Tests.jspa\#/testCase/LVV-T136}{Verify implementation of Data Product and Raw Data Access} \tabularnewline
    \hyperref[lvv-t137]{LVV-T137} &
    \href{https://jira.lsstcorp.org/secure/Tests.jspa\#/testCase/LVV-T137}{Verify implementation of Data Product Ingest} \tabularnewline
    \hyperref[lvv-t138]{LVV-T138} &
    \href{https://jira.lsstcorp.org/secure/Tests.jspa\#/testCase/LVV-T138}{Verify implementation of Bulk Download Service} \tabularnewline
    \hyperref[lvv-t140]{LVV-T140} &
    \href{https://jira.lsstcorp.org/secure/Tests.jspa\#/testCase/LVV-T140}{Verify implementation of Production Orchestration} \tabularnewline
    \hyperref[lvv-t141]{LVV-T141} &
    \href{https://jira.lsstcorp.org/secure/Tests.jspa\#/testCase/LVV-T141}{Verify implementation of Production Monitoring} \tabularnewline
    \hyperref[lvv-t142]{LVV-T142} &
    \href{https://jira.lsstcorp.org/secure/Tests.jspa\#/testCase/LVV-T142}{Verify implementation of Production Fault Tolerance} \tabularnewline
    \hyperref[lvv-t144]{LVV-T144} &
    \href{https://jira.lsstcorp.org/secure/Tests.jspa\#/testCase/LVV-T144}{Verify implementation of Task Specification} \tabularnewline
    \hyperref[lvv-t145]{LVV-T145} &
    \href{https://jira.lsstcorp.org/secure/Tests.jspa\#/testCase/LVV-T145}{Verify implementation of Task Configuration} \tabularnewline
    \hyperref[lvv-t146]{LVV-T146} &
    \href{https://jira.lsstcorp.org/secure/Tests.jspa\#/testCase/LVV-T146}{Verify implementation of DMS Initialization Component} \tabularnewline
    \hyperref[lvv-t147]{LVV-T147} &
    \href{https://jira.lsstcorp.org/secure/Tests.jspa\#/testCase/LVV-T147}{Verify implementation of Control of Level-1 Production} \tabularnewline
    \hyperref[lvv-t148]{LVV-T148} &
    \href{https://jira.lsstcorp.org/secure/Tests.jspa\#/testCase/LVV-T148}{Verify implementation of Unique Processing Coverage} \tabularnewline
    \hyperref[lvv-t149]{LVV-T149} &
    \href{https://jira.lsstcorp.org/secure/Tests.jspa\#/testCase/LVV-T149}{Verify implementation of Catalog Queries} \tabularnewline
    \hyperref[lvv-t150]{LVV-T150} &
    \href{https://jira.lsstcorp.org/secure/Tests.jspa\#/testCase/LVV-T150}{Verify implementation of Maintain Archive Publicly Accessible} \tabularnewline
    \hyperref[lvv-t151]{LVV-T151} &
    \href{https://jira.lsstcorp.org/secure/Tests.jspa\#/testCase/LVV-T151}{Verify Implementation of Catalog Export Formats From the Notebook Aspect} \tabularnewline
    \hyperref[lvv-t152]{LVV-T152} &
    \href{https://jira.lsstcorp.org/secure/Tests.jspa\#/testCase/LVV-T152}{Verify implementation of Keep Historical Alert Archive} \tabularnewline
    \hyperref[lvv-t153]{LVV-T153} &
    \href{https://jira.lsstcorp.org/secure/Tests.jspa\#/testCase/LVV-T153}{Verify implementation of Provide Engineering and Facility Database
Archive} \tabularnewline
    \hyperref[lvv-t154]{LVV-T154} &
    \href{https://jira.lsstcorp.org/secure/Tests.jspa\#/testCase/LVV-T154}{Verify implementation of Raw Data Archiving Reliability} \tabularnewline
    \hyperref[lvv-t155]{LVV-T155} &
    \href{https://jira.lsstcorp.org/secure/Tests.jspa\#/testCase/LVV-T155}{Verify implementation of Un-Archived Data Product Cache} \tabularnewline
    \hyperref[lvv-t156]{LVV-T156} &
    \href{https://jira.lsstcorp.org/secure/Tests.jspa\#/testCase/LVV-T156}{Verify implementation of Regenerate Un-archived Data Products} \tabularnewline
    \hyperref[lvv-t157]{LVV-T157} &
    \href{https://jira.lsstcorp.org/secure/Tests.jspa\#/testCase/LVV-T157}{Verify implementation Level 1 Data Product Access} \tabularnewline
    \hyperref[lvv-t158]{LVV-T158} &
    \href{https://jira.lsstcorp.org/secure/Tests.jspa\#/testCase/LVV-T158}{Verify implementation Level 1 and 2 Catalog Access} \tabularnewline
    \hyperref[lvv-t159]{LVV-T159} &
    \href{https://jira.lsstcorp.org/secure/Tests.jspa\#/testCase/LVV-T159}{Verify implementation of Regenerating Data Products from Previous Data
Releases} \tabularnewline
    \hyperref[lvv-t160]{LVV-T160} &
    \href{https://jira.lsstcorp.org/secure/Tests.jspa\#/testCase/LVV-T160}{Verify implementation of Providing a Precovery Service} \tabularnewline
    \hyperref[lvv-t161]{LVV-T161} &
    \href{https://jira.lsstcorp.org/secure/Tests.jspa\#/testCase/LVV-T161}{Verify implementation of Logging of catalog queries} \tabularnewline
    \hyperref[lvv-t162]{LVV-T162} &
    \href{https://jira.lsstcorp.org/secure/Tests.jspa\#/testCase/LVV-T162}{Verify implementation of Access to Previous Data Releases} \tabularnewline
    \hyperref[lvv-t163]{LVV-T163} &
    \href{https://jira.lsstcorp.org/secure/Tests.jspa\#/testCase/LVV-T163}{Verify implementation of Data Access Services} \tabularnewline
    \hyperref[lvv-t164]{LVV-T164} &
    \href{https://jira.lsstcorp.org/secure/Tests.jspa\#/testCase/LVV-T164}{Verify implementation of Operations Subsets} \tabularnewline
    \hyperref[lvv-t165]{LVV-T165} &
    \href{https://jira.lsstcorp.org/secure/Tests.jspa\#/testCase/LVV-T165}{Verify implementation of Subsets Support} \tabularnewline
    \hyperref[lvv-t166]{LVV-T166} &
    \href{https://jira.lsstcorp.org/secure/Tests.jspa\#/testCase/LVV-T166}{Verify implementation of Access Services Performance} \tabularnewline
    \hyperref[lvv-t167]{LVV-T167} &
    \href{https://jira.lsstcorp.org/secure/Tests.jspa\#/testCase/LVV-T167}{Verify Capability to serve older Data Releases at Full Performance} \tabularnewline
    \hyperref[lvv-t168]{LVV-T168} &
    \href{https://jira.lsstcorp.org/secure/Tests.jspa\#/testCase/LVV-T168}{Verify design of Data Access Services allows Evolution of the LSST Data
Model} \tabularnewline
    \hyperref[lvv-t169]{LVV-T169} &
    \href{https://jira.lsstcorp.org/secure/Tests.jspa\#/testCase/LVV-T169}{Verify implementation of Older Release Behavior} \tabularnewline
    \hyperref[lvv-t170]{LVV-T170} &
    \href{https://jira.lsstcorp.org/secure/Tests.jspa\#/testCase/LVV-T170}{Verify implementation of Query Availability} \tabularnewline
    \hyperref[lvv-t171]{LVV-T171} &
    \href{https://jira.lsstcorp.org/secure/Tests.jspa\#/testCase/LVV-T171}{Verify implementation of Pipeline Availability} \tabularnewline
    \hyperref[lvv-t172]{LVV-T172} &
    \href{https://jira.lsstcorp.org/secure/Tests.jspa\#/testCase/LVV-T172}{Verify implementation of Optimization of Cost, Reliability and
Availability} \tabularnewline
    \hyperref[lvv-t173]{LVV-T173} &
    \href{https://jira.lsstcorp.org/secure/Tests.jspa\#/testCase/LVV-T173}{Verify implementation of Pipeline Throughput} \tabularnewline
    \hyperref[lvv-t174]{LVV-T174} &
    \href{https://jira.lsstcorp.org/secure/Tests.jspa\#/testCase/LVV-T174}{Verify implementation of Re-processing Capacity} \tabularnewline
    \hyperref[lvv-t175]{LVV-T175} &
    \href{https://jira.lsstcorp.org/secure/Tests.jspa\#/testCase/LVV-T175}{Verify implementation of Temporary Storage for Communications Links} \tabularnewline
    \hyperref[lvv-t176]{LVV-T176} &
    \href{https://jira.lsstcorp.org/secure/Tests.jspa\#/testCase/LVV-T176}{Verify implementation of Infrastructure Sizing for ``catching up''} \tabularnewline
    \hyperref[lvv-t177]{LVV-T177} &
    \href{https://jira.lsstcorp.org/secure/Tests.jspa\#/testCase/LVV-T177}{Verify implementation of Incorporate Fault-Tolerance} \tabularnewline
    \hyperref[lvv-t178]{LVV-T178} &
    \href{https://jira.lsstcorp.org/secure/Tests.jspa\#/testCase/LVV-T178}{Verify implementation of Incorporate Autonomics} \tabularnewline
    \hyperref[lvv-t179]{LVV-T179} &
    \href{https://jira.lsstcorp.org/secure/Tests.jspa\#/testCase/LVV-T179}{Verify implementation of Compute Platform Heterogeneity} \tabularnewline
    \hyperref[lvv-t180]{LVV-T180} &
    \href{https://jira.lsstcorp.org/secure/Tests.jspa\#/testCase/LVV-T180}{Verify implementation of Data Management Unscheduled Downtime} \tabularnewline
    \hyperref[lvv-t181]{LVV-T181} &
    \href{https://jira.lsstcorp.org/secure/Tests.jspa\#/testCase/LVV-T181}{Verify Base Voice Over IP (VOIP)} \tabularnewline
    \hyperref[lvv-t182]{LVV-T182} &
    \href{https://jira.lsstcorp.org/secure/Tests.jspa\#/testCase/LVV-T182}{Verify implementation of Prefer Computing and Storage Down} \tabularnewline
    \hyperref[lvv-t183]{LVV-T183} &
    \href{https://jira.lsstcorp.org/secure/Tests.jspa\#/testCase/LVV-T183}{Verify implementation of DMS Communication with OCS} \tabularnewline
    \hyperref[lvv-t185]{LVV-T185} &
    \href{https://jira.lsstcorp.org/secure/Tests.jspa\#/testCase/LVV-T185}{Verify implementation of Summit to Base Network Availability} \tabularnewline
    \hyperref[lvv-t186]{LVV-T186} &
    \href{https://jira.lsstcorp.org/secure/Tests.jspa\#/testCase/LVV-T186}{Verify implementation of Summit to Base Network Reliability} \tabularnewline
    \hyperref[lvv-t187]{LVV-T187} &
    \href{https://jira.lsstcorp.org/secure/Tests.jspa\#/testCase/LVV-T187}{Verify implementation of Summit to Base Network Secondary Link} \tabularnewline
    \hyperref[lvv-t188]{LVV-T188} &
    \href{https://jira.lsstcorp.org/secure/Tests.jspa\#/testCase/LVV-T188}{Verify implementation of Summit to Base Network Ownership and Operation} \tabularnewline
    \hyperref[lvv-t189]{LVV-T189} &
    \href{https://jira.lsstcorp.org/secure/Tests.jspa\#/testCase/LVV-T189}{Verify implementation of Base Facility Infrastructure} \tabularnewline
    \hyperref[lvv-t190]{LVV-T190} &
    \href{https://jira.lsstcorp.org/secure/Tests.jspa\#/testCase/LVV-T190}{Verify implementation of Base Facility Co-Location with Existing
Facility} \tabularnewline
    \hyperref[lvv-t191]{LVV-T191} &
    \href{https://jira.lsstcorp.org/secure/Tests.jspa\#/testCase/LVV-T191}{Verify implementation of Commissioning Cluster} \tabularnewline
    \hyperref[lvv-t192]{LVV-T192} &
    \href{https://jira.lsstcorp.org/secure/Tests.jspa\#/testCase/LVV-T192}{Verify implementation of Base Wireless LAN (WiFi)} \tabularnewline
    \hyperref[lvv-t193]{LVV-T193} &
    \href{https://jira.lsstcorp.org/secure/Tests.jspa\#/testCase/LVV-T193}{Verify implementation of Base to Archive Network} \tabularnewline
    \hyperref[lvv-t194]{LVV-T194} &
    \href{https://jira.lsstcorp.org/secure/Tests.jspa\#/testCase/LVV-T194}{Verify implementation of Base to Archive Network Availability} \tabularnewline
    \hyperref[lvv-t195]{LVV-T195} &
    \href{https://jira.lsstcorp.org/secure/Tests.jspa\#/testCase/LVV-T195}{Verify implementation of Base to Archive Network Reliability} \tabularnewline
    \hyperref[lvv-t196]{LVV-T196} &
    \href{https://jira.lsstcorp.org/secure/Tests.jspa\#/testCase/LVV-T196}{Verify implementation of Base to Archive Network Secondary Link} \tabularnewline
    \hyperref[lvv-t197]{LVV-T197} &
    \href{https://jira.lsstcorp.org/secure/Tests.jspa\#/testCase/LVV-T197}{Verify implementation of Archive Center} \tabularnewline
    \hyperref[lvv-t198]{LVV-T198} &
    \href{https://jira.lsstcorp.org/secure/Tests.jspa\#/testCase/LVV-T198}{Verify implementation of Archive Center Disaster Recovery} \tabularnewline
    \hyperref[lvv-t199]{LVV-T199} &
    \href{https://jira.lsstcorp.org/secure/Tests.jspa\#/testCase/LVV-T199}{Verify implementation of Archive Center Co-Location with Existing
Facility} \tabularnewline
    \hyperref[lvv-t200]{LVV-T200} &
    \href{https://jira.lsstcorp.org/secure/Tests.jspa\#/testCase/LVV-T200}{Verify implementation of Archive to Data Access Center Network} \tabularnewline
    \hyperref[lvv-t201]{LVV-T201} &
    \href{https://jira.lsstcorp.org/secure/Tests.jspa\#/testCase/LVV-T201}{Verify implementation of Archive to Data Access Center Network
Availability} \tabularnewline
    \hyperref[lvv-t202]{LVV-T202} &
    \href{https://jira.lsstcorp.org/secure/Tests.jspa\#/testCase/LVV-T202}{Verify implementation of Archive to Data Access Center Network
Reliability} \tabularnewline
    \hyperref[lvv-t203]{LVV-T203} &
    \href{https://jira.lsstcorp.org/secure/Tests.jspa\#/testCase/LVV-T203}{Verify implementation of Archive to Data Access Center Network Secondary
Link} \tabularnewline
    \hyperref[lvv-t204]{LVV-T204} &
    \href{https://jira.lsstcorp.org/secure/Tests.jspa\#/testCase/LVV-T204}{Verify implementation of Access to catalogs for external Level 3
processing} \tabularnewline
    \hyperref[lvv-t205]{LVV-T205} &
    \href{https://jira.lsstcorp.org/secure/Tests.jspa\#/testCase/LVV-T205}{Verify implementation of Access to input catalogs for DAC-based Level 3
processing} \tabularnewline
    \hyperref[lvv-t206]{LVV-T206} &
    \href{https://jira.lsstcorp.org/secure/Tests.jspa\#/testCase/LVV-T206}{Verify implementation of Federation with external catalogs} \tabularnewline
    \hyperref[lvv-t207]{LVV-T207} &
    \href{https://jira.lsstcorp.org/secure/Tests.jspa\#/testCase/LVV-T207}{Verify implementation of Access to images for external Level 3
processing} \tabularnewline
    \hyperref[lvv-t208]{LVV-T208} &
    \href{https://jira.lsstcorp.org/secure/Tests.jspa\#/testCase/LVV-T208}{Verify implementation of Access to input images for DAC-based Level 3
processing} \tabularnewline
    \hyperref[lvv-t209]{LVV-T209} &
    \href{https://jira.lsstcorp.org/secure/Tests.jspa\#/testCase/LVV-T209}{Verify implementation of Data Access Centers} \tabularnewline
    \hyperref[lvv-t210]{LVV-T210} &
    \href{https://jira.lsstcorp.org/secure/Tests.jspa\#/testCase/LVV-T210}{Verify implementation of Data Access Center Simultaneous Connections} \tabularnewline
    \hyperref[lvv-t211]{LVV-T211} &
    \href{https://jira.lsstcorp.org/secure/Tests.jspa\#/testCase/LVV-T211}{Verify implementation of Data Access Center Geographical Distribution} \tabularnewline
    \hyperref[lvv-t212]{LVV-T212} &
    \href{https://jira.lsstcorp.org/secure/Tests.jspa\#/testCase/LVV-T212}{Verify implementation of No Limit on Data Access Centers} \tabularnewline
    \hyperref[lvv-t216]{LVV-T216} &
    \href{https://jira.lsstcorp.org/secure/Tests.jspa\#/testCase/LVV-T216}{Installation of the Alert Distribution payloads.} \tabularnewline
    \hyperref[lvv-t217]{LVV-T217} &
    \href{https://jira.lsstcorp.org/secure/Tests.jspa\#/testCase/LVV-T217}{Full Stream Alert Distribution} \tabularnewline
    \hyperref[lvv-t218]{LVV-T218} &
    \href{https://jira.lsstcorp.org/secure/Tests.jspa\#/testCase/LVV-T218}{Simple Filtering of the LSST Alert Stream} \tabularnewline
    \hyperref[lvv-t283]{LVV-T283} &
    \href{https://jira.lsstcorp.org/secure/Tests.jspa\#/testCase/LVV-T283}{RAS-00-00: Writing well-formed raw image} \tabularnewline
    \hyperref[lvv-t284]{LVV-T284} &
    \href{https://jira.lsstcorp.org/secure/Tests.jspa\#/testCase/LVV-T284}{RAS-00-05: (LDM-503-8b) Writing data from CCOB to the DBB for further
data processing} \tabularnewline
    \hyperref[lvv-t285]{LVV-T285} &
    \href{https://jira.lsstcorp.org/secure/Tests.jspa\#/testCase/LVV-T285}{RAS-00-10: Raw images in Observatory Operations Data Service} \tabularnewline
    \hyperref[lvv-t286]{LVV-T286} &
    \href{https://jira.lsstcorp.org/secure/Tests.jspa\#/testCase/LVV-T286}{RAS-00-20: Raw image are part of the permanent record of survey via DBB} \tabularnewline
    \hyperref[lvv-t287]{LVV-T287} &
    \href{https://jira.lsstcorp.org/secure/Tests.jspa\#/testCase/LVV-T287}{RAS-00-30: Raw Image Archiving Availability, Throughput, Reliability,
and Heterogeneity} \tabularnewline
    \hyperref[lvv-t362]{LVV-T362} &
    \href{https://jira.lsstcorp.org/secure/Tests.jspa\#/testCase/LVV-T362}{Installation of the LSST Science Pipelines Payloads} \tabularnewline
    \hyperref[lvv-t363]{LVV-T363} &
    \href{https://jira.lsstcorp.org/secure/Tests.jspa\#/testCase/LVV-T363}{Science Pipelines Release Documentation} \tabularnewline
    \hyperref[lvv-t368]{LVV-T368} &
    \href{https://jira.lsstcorp.org/secure/Tests.jspa\#/testCase/LVV-T368}{Loading and processing Camera test data} \tabularnewline
    \hyperref[lvv-t374]{LVV-T374} &
    \href{https://jira.lsstcorp.org/secure/Tests.jspa\#/testCase/LVV-T374}{Ingesting Camera test data} \tabularnewline
    \hyperref[lvv-t376]{LVV-T376} &
    \href{https://jira.lsstcorp.org/secure/Tests.jspa\#/testCase/LVV-T376}{Verify the Calculation of Ellipticity Residuals and Correlations} \tabularnewline
    \hyperref[lvv-t377]{LVV-T377} &
    \href{https://jira.lsstcorp.org/secure/Tests.jspa\#/testCase/LVV-T377}{Verify Calculation of Photometric Performance Metrics} \tabularnewline
    \hyperref[lvv-t378]{LVV-T378} &
    \href{https://jira.lsstcorp.org/secure/Tests.jspa\#/testCase/LVV-T378}{Verify Calculation of Astrometric Performance Metrics} \tabularnewline
    \hyperref[lvv-t385]{LVV-T385} &
    \href{https://jira.lsstcorp.org/secure/Tests.jspa\#/testCase/LVV-T385}{Verify implementation of minimum number of simultaneous retrievals of
CCD-sized coadd cutouts} \tabularnewline
    \hyperref[lvv-t454]{LVV-T454} &
    \href{https://jira.lsstcorp.org/secure/Tests.jspa\#/testCase/LVV-T454}{LDM-503-8 Enable LSP viewing of spectrograph data.} \tabularnewline
    \hyperref[lvv-t1085]{LVV-T1085} &
    \href{https://jira.lsstcorp.org/secure/Tests.jspa\#/testCase/LVV-T1085}{Short Queries Functional Test} \tabularnewline
    \hyperref[lvv-t1086]{LVV-T1086} &
    \href{https://jira.lsstcorp.org/secure/Tests.jspa\#/testCase/LVV-T1086}{Full Table Scans Functional Test} \tabularnewline
    \hyperref[lvv-t1087]{LVV-T1087} &
    \href{https://jira.lsstcorp.org/secure/Tests.jspa\#/testCase/LVV-T1087}{Full Table Joins Functional Test} \tabularnewline
    \hyperref[lvv-t1088]{LVV-T1088} &
    \href{https://jira.lsstcorp.org/secure/Tests.jspa\#/testCase/LVV-T1088}{Concurrent Scans Scaling Test} \tabularnewline
    \hyperref[lvv-t1089]{LVV-T1089} &
    \href{https://jira.lsstcorp.org/secure/Tests.jspa\#/testCase/LVV-T1089}{Load Test} \tabularnewline
    \hyperref[lvv-t1090]{LVV-T1090} &
    \href{https://jira.lsstcorp.org/secure/Tests.jspa\#/testCase/LVV-T1090}{Heavy Load Test} \tabularnewline
    \hyperref[lvv-t1097]{LVV-T1097} &
    \href{https://jira.lsstcorp.org/secure/Tests.jspa\#/testCase/LVV-T1097}{Verify Summit Facility Network Implementation} \tabularnewline
    \hyperref[lvv-t1168]{LVV-T1168} &
    \href{https://jira.lsstcorp.org/secure/Tests.jspa\#/testCase/LVV-T1168}{Verify Summit - Base Network Integration} \tabularnewline
    \hyperref[lvv-t1232]{LVV-T1232} &
    \href{https://jira.lsstcorp.org/secure/Tests.jspa\#/testCase/LVV-T1232}{Verify Implementation of Catalog Export Formats From the Portal Aspect} \tabularnewline
    \hyperref[lvv-t1240]{LVV-T1240} &
    \href{https://jira.lsstcorp.org/secure/Tests.jspa\#/testCase/LVV-T1240}{Verify implementation of minimum astrometric standards per CCD} \tabularnewline
    \hyperref[lvv-t1250]{LVV-T1250} &
    \href{https://jira.lsstcorp.org/secure/Tests.jspa\#/testCase/LVV-T1250}{Verify implementation of minimum number of simultaneous DM EFD query
users} \tabularnewline
    \hyperref[lvv-t1251]{LVV-T1251} &
    \href{https://jira.lsstcorp.org/secure/Tests.jspa\#/testCase/LVV-T1251}{Verify implementation of maximum time to retrieve DM EFD query results} \tabularnewline
    \hyperref[lvv-t1252]{LVV-T1252} &
    \href{https://jira.lsstcorp.org/secure/Tests.jspa\#/testCase/LVV-T1252}{Verify number of simultaneous alert filter users} \tabularnewline
    \hyperref[lvv-t1264]{LVV-T1264} &
    \href{https://jira.lsstcorp.org/secure/Tests.jspa\#/testCase/LVV-T1264}{Verify implementation of archiving camera test data} \tabularnewline
    \hyperref[lvv-t1276]{LVV-T1276} &
    \href{https://jira.lsstcorp.org/secure/Tests.jspa\#/testCase/LVV-T1276}{Verify implementation of latency of reporting optical transients} \tabularnewline
    \hyperref[lvv-t1277]{LVV-T1277} &
    \href{https://jira.lsstcorp.org/secure/Tests.jspa\#/testCase/LVV-T1277}{Verify processing of maximum number of calibration exposures} \tabularnewline
    \hyperref[lvv-t1332]{LVV-T1332} &
    \href{https://jira.lsstcorp.org/secure/Tests.jspa\#/testCase/LVV-T1332}{Verify implementation of maximum time for retrieval of CCD-sized coadd
cutouts} \tabularnewline
    \hyperref[lvv-t1524]{LVV-T1524} &
    \href{https://jira.lsstcorp.org/secure/Tests.jspa\#/testCase/LVV-T1524}{Verify Implementation of Exporting MOCs as FITS} \tabularnewline
    \hyperref[lvv-t1525]{LVV-T1525} &
    \href{https://jira.lsstcorp.org/secure/Tests.jspa\#/testCase/LVV-T1525}{Verify Implementation of Linkage Between HiPS Maps and Coadded Images} \tabularnewline
    \hyperref[lvv-t1526]{LVV-T1526} &
    \href{https://jira.lsstcorp.org/secure/Tests.jspa\#/testCase/LVV-T1526}{Verify Availability of Secure and Authenticated HiPS Service} \tabularnewline
    \hyperref[lvv-t1527]{LVV-T1527} &
    \href{https://jira.lsstcorp.org/secure/Tests.jspa\#/testCase/LVV-T1527}{Verify Support for HiPS Visualization} \tabularnewline
    \hyperref[lvv-t1528]{LVV-T1528} &
    \href{https://jira.lsstcorp.org/secure/Tests.jspa\#/testCase/LVV-T1528}{Verify Visualization of MOCs via Science Platform} \tabularnewline
    \hyperref[lvv-t1529]{LVV-T1529} &
    \href{https://jira.lsstcorp.org/secure/Tests.jspa\#/testCase/LVV-T1529}{Verify Production of All-Sky HiPS Map} \tabularnewline
    \hyperref[lvv-t1530]{LVV-T1530} &
    \href{https://jira.lsstcorp.org/secure/Tests.jspa\#/testCase/LVV-T1530}{Verify Production of Multi-Order Coverage Maps for Survey Data} \tabularnewline
    \hyperref[lvv-t1549]{LVV-T1549} &
    \href{https://jira.lsstcorp.org/secure/Tests.jspa\#/testCase/LVV-T1549}{LDM-503-6 Comcam verification readiness} \tabularnewline
    \hyperref[lvv-t1550]{LVV-T1550} &
    \href{https://jira.lsstcorp.org/secure/Tests.jspa\#/testCase/LVV-T1550}{LDM-503-10 DAQ Validation} \tabularnewline
    \hyperref[lvv-t1556]{LVV-T1556} &
    \href{https://jira.lsstcorp.org/secure/Tests.jspa\#/testCase/LVV-T1556}{LDM-503-10B Large Scale CCOB Data Access} \tabularnewline
    \hyperref[lvv-t1560]{LVV-T1560} &
    \href{https://jira.lsstcorp.org/secure/Tests.jspa\#/testCase/LVV-T1560}{Verify archiving of processing provenance} \tabularnewline
    \hyperref[lvv-t1561]{LVV-T1561} &
    \href{https://jira.lsstcorp.org/secure/Tests.jspa\#/testCase/LVV-T1561}{Verify provenance availability to science users} \tabularnewline
    \hyperref[lvv-t1562]{LVV-T1562} &
    \href{https://jira.lsstcorp.org/secure/Tests.jspa\#/testCase/LVV-T1562}{Verify availability of re-run tools} \tabularnewline
    \hyperref[lvv-t1563]{LVV-T1563} &
    \href{https://jira.lsstcorp.org/secure/Tests.jspa\#/testCase/LVV-T1563}{Verify re-run on different system produces the same results} \tabularnewline
    \hyperref[lvv-t1564]{LVV-T1564} &
    \href{https://jira.lsstcorp.org/secure/Tests.jspa\#/testCase/LVV-T1564}{Verify re-run on similar system produces the same results} \tabularnewline
    \hyperref[lvv-t1612]{LVV-T1612} &
    \href{https://jira.lsstcorp.org/secure/Tests.jspa\#/testCase/LVV-T1612}{Verify Summit - Base Network Integration (System Level)} \tabularnewline
    \hyperref[lvv-t1745]{LVV-T1745} &
    \href{https://jira.lsstcorp.org/secure/Tests.jspa\#/testCase/LVV-T1745}{Verify calculation of median relative astrometric measurement error on
20 arcminute scales} \tabularnewline
    \hyperref[lvv-t1746]{LVV-T1746} &
    \href{https://jira.lsstcorp.org/secure/Tests.jspa\#/testCase/LVV-T1746}{Verify calculation of fraction of relative astrometric measurement error
on 5 arcminute scales exceeding outlier limit} \tabularnewline
    \hyperref[lvv-t1747]{LVV-T1747} &
    \href{https://jira.lsstcorp.org/secure/Tests.jspa\#/testCase/LVV-T1747}{Verify calculation of relative astrometric measurement error on 5
arcminute scales} \tabularnewline
    \hyperref[lvv-t1748]{LVV-T1748} &
    \href{https://jira.lsstcorp.org/secure/Tests.jspa\#/testCase/LVV-T1748}{Verify calculation of median error in absolute position for RA, Dec axes} \tabularnewline
    \hyperref[lvv-t1749]{LVV-T1749} &
    \href{https://jira.lsstcorp.org/secure/Tests.jspa\#/testCase/LVV-T1749}{Verify calculation of fraction of relative astrometric measurement error
on 20 arcminute scales exceeding outlier limit} \tabularnewline
    \hyperref[lvv-t1750]{LVV-T1750} &
    \href{https://jira.lsstcorp.org/secure/Tests.jspa\#/testCase/LVV-T1750}{Verify calculation of separations relative to r-band exceeding color
difference outlier limit} \tabularnewline
    \hyperref[lvv-t1751]{LVV-T1751} &
    \href{https://jira.lsstcorp.org/secure/Tests.jspa\#/testCase/LVV-T1751}{Verify calculation of median relative astrometric measurement error on
200 arcminute scales} \tabularnewline
    \hyperref[lvv-t1752]{LVV-T1752} &
    \href{https://jira.lsstcorp.org/secure/Tests.jspa\#/testCase/LVV-T1752}{Verify calculation of fraction of relative astrometric measurement error
on 200 arcminute scales exceeding outlier limit} \tabularnewline
    \hyperref[lvv-t1753]{LVV-T1753} &
    \href{https://jira.lsstcorp.org/secure/Tests.jspa\#/testCase/LVV-T1753}{Verify calculation of RMS difference of separations relative to r-band} \tabularnewline
    \hyperref[lvv-t1754]{LVV-T1754} &
    \href{https://jira.lsstcorp.org/secure/Tests.jspa\#/testCase/LVV-T1754}{Verify calculation of residual PSF ellipticity correlations for
separations less than 5 arcmin} \tabularnewline
    \hyperref[lvv-t1755]{LVV-T1755} &
    \href{https://jira.lsstcorp.org/secure/Tests.jspa\#/testCase/LVV-T1755}{Verify calculation of residual PSF ellipticity correlations for
separations less than 1 arcmin} \tabularnewline
    \hyperref[lvv-t1756]{LVV-T1756} &
    \href{https://jira.lsstcorp.org/secure/Tests.jspa\#/testCase/LVV-T1756}{Verify calculation of photometric repeatability in uzy filters} \tabularnewline
    \hyperref[lvv-t1757]{LVV-T1757} &
    \href{https://jira.lsstcorp.org/secure/Tests.jspa\#/testCase/LVV-T1757}{Verify calculation of photometric repeatability in gri filters} \tabularnewline
    \hyperref[lvv-t1758]{LVV-T1758} &
    \href{https://jira.lsstcorp.org/secure/Tests.jspa\#/testCase/LVV-T1758}{Verify calculation of photometric outliers in uzy bands} \tabularnewline
    \hyperref[lvv-t1759]{LVV-T1759} &
    \href{https://jira.lsstcorp.org/secure/Tests.jspa\#/testCase/LVV-T1759}{Verify calculation of photometric outliers in gri bands} \tabularnewline
    \hyperref[lvv-t1830]{LVV-T1830} &
    \href{https://jira.lsstcorp.org/secure/Tests.jspa\#/testCase/LVV-T1830}{Verify Implementation of Scientific Visualization of Camera Image Data} \tabularnewline
    \hyperref[lvv-t1831]{LVV-T1831} &
    \href{https://jira.lsstcorp.org/secure/Tests.jspa\#/testCase/LVV-T1831}{Verify Implementation of Data Management Nightly Reporting} \tabularnewline
    \hyperref[lvv-t1836]{LVV-T1836} &
    \href{https://jira.lsstcorp.org/secure/Tests.jspa\#/testCase/LVV-T1836}{Verify calculation of resolved-to-unresolved flux ratio errors} \tabularnewline
    \hyperref[lvv-t1837]{LVV-T1837} &
    \href{https://jira.lsstcorp.org/secure/Tests.jspa\#/testCase/LVV-T1837}{Verify calculation of band-to-band color zero-point accuracy} \tabularnewline
    \hyperref[lvv-t1838]{LVV-T1838} &
    \href{https://jira.lsstcorp.org/secure/Tests.jspa\#/testCase/LVV-T1838}{Verify calculation of image fraction affected by ghosts} \tabularnewline
    \hyperref[lvv-t1839]{LVV-T1839} &
    \href{https://jira.lsstcorp.org/secure/Tests.jspa\#/testCase/LVV-T1839}{Verify calculation of RMS width of photometric zeropoint} \tabularnewline
    \hyperref[lvv-t1840]{LVV-T1840} &
    \href{https://jira.lsstcorp.org/secure/Tests.jspa\#/testCase/LVV-T1840}{Verify calculation of sky brightness precision} \tabularnewline
    \hyperref[lvv-t1841]{LVV-T1841} &
    \href{https://jira.lsstcorp.org/secure/Tests.jspa\#/testCase/LVV-T1841}{Verify calculation of scientifically unusable pixel fraction} \tabularnewline
    \hyperref[lvv-t1842]{LVV-T1842} &
    \href{https://jira.lsstcorp.org/secure/Tests.jspa\#/testCase/LVV-T1842}{Verify calculation of zeropoint error fraction exceeding the outlier
limit} \tabularnewline
    \hyperref[lvv-t1843]{LVV-T1843} &
    \href{https://jira.lsstcorp.org/secure/Tests.jspa\#/testCase/LVV-T1843}{Verify calculation of significance of imperfect crosstalk corrections} \tabularnewline
    \hyperref[lvv-t1844]{LVV-T1844} &
    \href{https://jira.lsstcorp.org/secure/Tests.jspa\#/testCase/LVV-T1844}{Verify calculation of u-band photometric zero-point RMS} \tabularnewline
    \hyperref[lvv-t1845]{LVV-T1845} &
    \href{https://jira.lsstcorp.org/secure/Tests.jspa\#/testCase/LVV-T1845}{Verify accuracy of photometric transformation to physical scale} \tabularnewline
    \hyperref[lvv-t1846]{LVV-T1846} &
    \href{https://jira.lsstcorp.org/secure/Tests.jspa\#/testCase/LVV-T1846}{Verify calculation of band-to-band color zero-point accuracy including
u-band} \tabularnewline
    \hyperref[lvv-t1847]{LVV-T1847} &
    \href{https://jira.lsstcorp.org/secure/Tests.jspa\#/testCase/LVV-T1847}{Verify calculation of sensor fraction with unusable pixels} \tabularnewline
    \hyperref[lvv-t1862]{LVV-T1862} &
    \href{https://jira.lsstcorp.org/secure/Tests.jspa\#/testCase/LVV-T1862}{Verify determining effectiveness of dark current frame} \tabularnewline
    \hyperref[lvv-t1863]{LVV-T1863} &
    \href{https://jira.lsstcorp.org/secure/Tests.jspa\#/testCase/LVV-T1863}{Verify ability to process Special Programs data alongside normal
processing} \tabularnewline
    \hyperref[lvv-t1865]{LVV-T1865} &
    \href{https://jira.lsstcorp.org/secure/Tests.jspa\#/testCase/LVV-T1865}{Verify implementation of time to L1 public release for Special Programs} \tabularnewline
    \hyperref[lvv-t1866]{LVV-T1866} &
    \href{https://jira.lsstcorp.org/secure/Tests.jspa\#/testCase/LVV-T1866}{Verify latency of reporting optical transients from Special Programs} \tabularnewline
    \hyperref[lvv-t1867]{LVV-T1867} &
    \href{https://jira.lsstcorp.org/secure/Tests.jspa\#/testCase/LVV-T1867}{Verify implementation of at least numStreams alert streams supported} \tabularnewline
    \hyperref[lvv-t1868]{LVV-T1868} &
    \href{https://jira.lsstcorp.org/secure/Tests.jspa\#/testCase/LVV-T1868}{Verify implementation of alert streams distributed within latency limit} \tabularnewline
    \hyperref[lvv-t1946]{LVV-T1946} &
    \href{https://jira.lsstcorp.org/secure/Tests.jspa\#/testCase/LVV-T1946}{Verify implementation of measurements in catalogs from coadds} \tabularnewline
    \hyperref[lvv-t1947]{LVV-T1947} &
    \href{https://jira.lsstcorp.org/secure/Tests.jspa\#/testCase/LVV-T1947}{Verify implementation of measurements in catalogs from difference images} \tabularnewline
\bottomrule
\end{longtable}

\newpage

\section{Active Test Cases}

This section documents all active test cases that have a status in the Jira/ATM system of Draft, Defined or Approved.



\subsection{LVV-T10 - DRP-00-00: Installation of the Data Release Production v14.0 science
payload}\label{lvv-t10}

\begin{longtable}[]{llllll}
\toprule
Version & Status & Priority & Verification Type & Owner
\\\midrule
1 & Approved & Normal &
Test & Jim Bosch
\\\bottomrule
\multicolumn{6}{c}{ Open \href{https://jira.lsstcorp.org/secure/Tests.jspa\#/testCase/LVV-T10}{LVV-T10} in Jira } \\
\end{longtable}

\subsubsection{Verification Elements}
\begin{itemize}
\item \href{https://jira.lsstcorp.org/browse/LVV-139}{LVV-139} - DMS-REQ-0308-V-01: Software Architecture to Enable Community Re-Use

\end{itemize}

\subsubsection{Test Items}
This test will check:

\begin{itemize}
\tightlist
\item
  That the Data Release Production science payload is available for
  distribution from documented channels;
\item
  That the Data Release Production science payload can be installed on
  LSST Data Facility-managed systems.
\end{itemize}


\subsubsection{Predecessors}

\subsubsection{Environment Needs}

\paragraph{Software}
All prerequisite packages listed at https://pipelines.lsst.io/install/
prereqs/centos.html must be available on the test system and on the
LSST-VC compute node.

\paragraph{Hardware}
This test case shall be executed on a developer system at the LSST Data
Facility which serves as the ``head node'' or otherwise provides access
to filesystems shared by the LSST Verification Cluster (LSST-VC). We
assume that this system will be lsst-dev01.ncsa.illinois.edu and the
filesystem will be a GPFS-based system mounted at /software. The test
also requires access to one LSST-VC compute node.

\subsubsection{Input Specification}
No input data is required for this test case.

\subsubsection{Output Specification}
The Data Release Production science payload will be made available on a
shared filesystem accessible from LSST-VC compute notes.

\subsubsection{Test Procedure}
    \begin{longtable}[]{p{1.3cm}p{2cm}p{13cm}}
    %\toprule
    Step & \multicolumn{2}{@{}l}{Description, Input Data and Expected Result} \\ \toprule
    \endhead

            \multirow{3}{*}{ 1 } & Description &
            \begin{minipage}[t]{13cm}{\footnotesize
            Release 14.0 of the LSST Science Pipelines will be installed into the
GPFS filesystem accessible at /software on lsst-dev01 following the
instructions at https://pipelines.lsst. io/install/newinstall.html.

            \vspace{\dp0}
            } \end{minipage} \\ \cline{2-3}
            & Test Data &
            \begin{minipage}[t]{13cm}{\footnotesize
                No data.
                \vspace{\dp0}
            } \end{minipage} \\ \cline{2-3}
            & Expected Result &
        \\ \midrule

            \multirow{3}{*}{ 2 } & Description &
            \begin{minipage}[t]{13cm}{\footnotesize
            The lsst\_distrib top level package will be enabled:\\
\hspace*{0.333em} ~ ~ ~source /software/lsstsw/stack3/loadLSST.bash\\
\hspace*{0.333em} ~ ~ ~setup lsst\_distrib

            \vspace{\dp0}
            } \end{minipage} \\ \cline{2-3}
            & Test Data &
            \begin{minipage}[t]{13cm}{\footnotesize
                No data.
                \vspace{\dp0}
            } \end{minipage} \\ \cline{2-3}
            & Expected Result &
        \\ \midrule

            \multirow{3}{*}{ 3 } & Description &
            \begin{minipage}[t]{13cm}{\footnotesize
            The ``LSST Stack Demo'' package will be downloaded onto the test system
from https: //github.com/lsst/lsst\_dm\_stack\_demo/releases/tag/14.0
and uncompressed.

            \vspace{\dp0}
            } \end{minipage} \\ \cline{2-3}
            & Test Data &
            \begin{minipage}[t]{13cm}{\footnotesize
                No data.
                \vspace{\dp0}
            } \end{minipage} \\ \cline{2-3}
            & Expected Result &
        \\ \midrule

            \multirow{3}{*}{ 4 } & Description &
            \begin{minipage}[t]{13cm}{\footnotesize
            The demo package will be executed by following the instructions in its
``README`` file. The string ``Ok.`` should be returned.

            \vspace{\dp0}
            } \end{minipage} \\ \cline{2-3}
            & Test Data &
            \begin{minipage}[t]{13cm}{\footnotesize
                No data.
                \vspace{\dp0}
            } \end{minipage} \\ \cline{2-3}
            & Expected Result &
        \\ \midrule

            \multirow{3}{*}{ 5 } & Description &
            \begin{minipage}[t]{13cm}{\footnotesize
            A shell on an LSST-VC compute node will now be obtained by executing:\\
\hspace*{0.333em} ~ ~ ~\$ srun -I --pty bash

            \vspace{\dp0}
            } \end{minipage} \\ \cline{2-3}
            & Test Data &
            \begin{minipage}[t]{13cm}{\footnotesize
                No data.
                \vspace{\dp0}
            } \end{minipage} \\ \cline{2-3}
            & Expected Result &
        \\ \midrule

            \multirow{3}{*}{ 6 } & Description &
            \begin{minipage}[t]{13cm}{\footnotesize
            The demo package will be executed on the compute node and the same
result obtained.

            \vspace{\dp0}
            } \end{minipage} \\ \cline{2-3}
            & Test Data &
            \begin{minipage}[t]{13cm}{\footnotesize
                No data.
                \vspace{\dp0}
            } \end{minipage} \\ \cline{2-3}
            & Expected Result &
        \\ \midrule
    \end{longtable}

\subsection{LVV-T11 - DRP-00-05: Execution of the DRP Science Payload by the Batch Production
Service}\label{lvv-t11}

\begin{longtable}[]{llllll}
\toprule
Version & Status & Priority & Verification Type & Owner
\\\midrule
1 & Approved & Normal &
Test & Jim Bosch
\\\bottomrule
\multicolumn{6}{c}{ Open \href{https://jira.lsstcorp.org/secure/Tests.jspa\#/testCase/LVV-T11}{LVV-T11} in Jira } \\
\end{longtable}

\subsubsection{Verification Elements}
\begin{itemize}
\item \href{https://jira.lsstcorp.org/browse/LVV-46}{LVV-46} - DMS-REQ-0106-V-01: Coadded Image Provenance

\item \href{https://jira.lsstcorp.org/browse/LVV-124}{LVV-124} - DMS-REQ-0293-V-01: Selection of Datasets

\item \href{https://jira.lsstcorp.org/browse/LVV-134}{LVV-134} - DMS-REQ-0303-V-01: Production Monitoring

\item \href{https://jira.lsstcorp.org/browse/LVV-133}{LVV-133} - DMS-REQ-0302-V-01: Production Orchestration

\item \href{https://jira.lsstcorp.org/browse/LVV-136}{LVV-136} - DMS-REQ-0305-V-01: Task Specification

\item \href{https://jira.lsstcorp.org/browse/LVV-137}{LVV-137} - DMS-REQ-0306-V-01: Task Configuration

\item \href{https://jira.lsstcorp.org/browse/LVV-62}{LVV-62} - DMS-REQ-0158-V-01: Provide Pipeline Construction Services

\end{itemize}

\subsubsection{Test Items}
This test will check that the DRP Science Payload can be executed using
a specific version of the Batch Production Service provided by the LSST
Data Facility. Since the outputs are stored in the Data Backbone, it too
is a component of this test.


\subsubsection{Predecessors}
LVV-T10 (DRP-00-00)

\subsubsection{Environment Needs}

\paragraph{Software}
All the necessary software will be pre-installed. The software includes
the science pipeline codes as well as the Data Management system codes
(Batch Processing Service, Data Backbone).\\
For LDM-503-2, Python 2 versions of software will be used. The science
pipeline codes will be provided via the LSST DM Software Stack, release
14.0. The Batch Processing Service and the Data Backbone are initial
versions using the DESDM Framework packages. LSST-specific plugins as
well as DRP pipeline integration codes are also pre-installed. All
python DESDM Framework packages, plugins and integration codes exist in
the lsst-dm github with tag 1.01. The DESDM prerequisites come from the
official DESDM eups package firstcut Y4N+5. They are installed using
DESDMâ\euro{}™s eups install process.\\
The ticket branch tickets/DM-12291 of the LSST Software Stack packages
meas\_base, pipe\_tasks, and obs\_subaru will be used to change the
patch ID naming convention. This is due to issues of having commas in
the filenames and data IDs, as discussed in RFC-361; the solution has
been agreed in RFC-365 for future implementation in DM-11874, DM-11875,
and DM-11876.\\
For LDM-503-2, the software will be installed into the GPFS space at
/project/production/ on LSST-VC. A single eups prototype package will be
defined to encompass the above mentioned software.

\paragraph{Hardware}
This test case shall be executed on a testbed at the LSST Data Facility.
For LDM-503-2, this testbed includes:

\begin{itemize}
\tightlist
\item
  LSST Verification Cluster (LSST-VC) with Slurm Job Scheduler
\item
  Submit and compute nodes with read/write access to various GPFS shared
  filesystems:

  \begin{enumerate}
  \tightlist
  \item
    Filesystem containing the software stack
  \item
    Filesystem for the submit side temporary outputs
  \item
    Filesystem being used by the prototype DataBackbone.(This means that
    the frame-work can use a cp transfer protocol between the job and
    the Data Backbone and does not require additional transfer services
    to be running.)
  \item
    Filesystem for the individual job scratch directories.
  \end{enumerate}
\item
  Single node Oracle database (version 12c)
\item
  Submit node (lsst-dev01.ncsa.illinois.edu) running the HTCondor
  Central Manager (ver- sion 8.7.3).
\end{itemize}

\subsubsection{Input Specification}
A small number of selected tracts of the Hyper Suprime-Cam dataset will
be used along with appropriate calibration datasets.\\
For LDM-503-2, the three tracts of the Hyper Suprime-Cam ``RC1''
dataset, as defined Appendix A.1.1, will be used. The calibration
dataset will be the 20170105 version, defined as per DMTR- 31. Raw files
known to fail processCcd will be blacklisted.\\[2\baselineskip]

\subsubsection{Output Specification}
The output data products will be available from the Data Backbone.\\
For LDM-503-2, the output data products will be available on the LSST-VC
via a shared filesystem and advanced data discovery is done via SQL
queries against the Oracle database.

\subsubsection{Test Procedure}
    \begin{longtable}[]{p{1.3cm}p{2cm}p{13cm}}
    %\toprule
    Step & \multicolumn{2}{@{}l}{Description, Input Data and Expected Result} \\ \toprule
    \endhead

            \multirow{3}{*}{ 1 } & Description &
            \begin{minipage}[t]{13cm}{\footnotesize
            Setup. The LSST Science Pipelines and the DESDM Framework, plugins, and
integration codes as described in Environment - Software paragraph have
already been installed. The Operator merely sets up the expanded stack
using eups.

            \vspace{\dp0}
            } \end{minipage} \\ \cline{2-3}
            & Test Data &
            \begin{minipage}[t]{13cm}{\footnotesize
                No data.
                \vspace{\dp0}
            } \end{minipage} \\ \cline{2-3}
            & Expected Result &
        \\ \midrule

            \multirow{3}{*}{ 2 } & Description &
            \begin{minipage}[t]{13cm}{\footnotesize
            Input raw and calibration data must exist in the Data Backbone. If not,
the data will be ingested into Data Backbone.

            \vspace{\dp0}
            } \end{minipage} \\ \cline{2-3}
            & Test Data &
            \begin{minipage}[t]{13cm}{\footnotesize
                No data.
                \vspace{\dp0}
            } \end{minipage} \\ \cline{2-3}
            & Expected Result &
        \\ \midrule

            \multirow{3}{*}{ 3 } & Description &
            \begin{minipage}[t]{13cm}{\footnotesize
            The operator tags and blacklists input data as appropriate for test (see
Input Specifications ยง4.2.5).

            \vspace{\dp0}
            } \end{minipage} \\ \cline{2-3}
            & Test Data &
            \begin{minipage}[t]{13cm}{\footnotesize
                No data.
                \vspace{\dp0}
            } \end{minipage} \\ \cline{2-3}
            & Expected Result &
        \\ \midrule

            \multirow{3}{*}{ 4 } & Description &
            \begin{minipage}[t]{13cm}{\footnotesize
            Given the LSST Science Pipelines version, the operator will generate the
full config files and schema files (Test Configuration ยง4.2.7), which
are then ingested into the Data Backbone.

            \vspace{\dp0}
            } \end{minipage} \\ \cline{2-3}
            & Test Data &
            \begin{minipage}[t]{13cm}{\footnotesize
                No data.
                \vspace{\dp0}
            } \end{minipage} \\ \cline{2-3}
            & Expected Result &
        \\ \midrule

            \multirow{3}{*}{ 5 } & Description &
            \begin{minipage}[t]{13cm}{\footnotesize
            Write a DRP pipeline workflow definition file from scratch or modify an
existing file from github making its operations- and dataset-specific
inputs match this test.

\begin{itemize}
\tightlist
\item
  (a) For LDM-503-2, the pipeline workflow definition file is written in
  a workflow control language (wcl) format as used by the DESDM
  Framework.
\end{itemize}

            \vspace{\dp0}
            } \end{minipage} \\ \cline{2-3}
            & Test Data &
            \begin{minipage}[t]{13cm}{\footnotesize
                No data.
                \vspace{\dp0}
            } \end{minipage} \\ \cline{2-3}
            & Expected Result &
        \\ \midrule

            \multirow{3}{*}{ 6 } & Description &
            \begin{minipage}[t]{13cm}{\footnotesize
            Make special hardware requests (e.g., disk or compute node reservations)
if needed.

            \vspace{\dp0}
            } \end{minipage} \\ \cline{2-3}
            & Test Data &
            \begin{minipage}[t]{13cm}{\footnotesize
                No data.
                \vspace{\dp0}
            } \end{minipage} \\ \cline{2-3}
            & Expected Result &
        \\ \midrule

            \multirow{3}{*}{ 7 } & Description &
            \begin{minipage}[t]{13cm}{\footnotesize
            Execution starts. If HTCondor processes are not already running, start
HTCondor processes on compute nodes. This step makes the compute nodes
join the HTCondor Central Manager to cre- ate a working HTCondor Pool.

            \vspace{\dp0}
            } \end{minipage} \\ \cline{2-3}
            & Test Data &
            \begin{minipage}[t]{13cm}{\footnotesize
                No data.
                \vspace{\dp0}
            } \end{minipage} \\ \cline{2-3}
            & Expected Result &
        \\ \midrule

            \multirow{3}{*}{ 8 } & Description &
            \begin{minipage}[t]{13cm}{\footnotesize
            The execution for each tract of the input data in the Input
specification section (ยง4.2.5) will be submitted to the hardware
specified in ``Environmental Needs - Hardware'' section (ยง4.2.4.1)
using the configuration specified in ``Test configuration'' section
(ยง4.2.7).

            \vspace{\dp0}
            } \end{minipage} \\ \cline{2-3}
            & Test Data &
            \begin{minipage}[t]{13cm}{\footnotesize
                No data.
                \vspace{\dp0}
            } \end{minipage} \\ \cline{2-3}
            & Expected Result &
        \\ \midrule

            \multirow{3}{*}{ 9 } & Description &
            \begin{minipage}[t]{13cm}{\footnotesize
            During execution, the operator will use software to demonstrate the
ability to check the processing status.

\begin{itemize}
\tightlist
\item
  (a) ForLDM-503-2,the available Batch Production Service monitoring
  software consists of two commands: one to summarize currently
  submitted processing, one to get more details of single submission.
\end{itemize}

            \vspace{\dp0}
            } \end{minipage} \\ \cline{2-3}
            & Test Data &
            \begin{minipage}[t]{13cm}{\footnotesize
                No data.
                \vspace{\dp0}
            } \end{minipage} \\ \cline{2-3}
            & Expected Result &
        \\ \midrule

            \multirow{3}{*}{ 10 } & Description &
            \begin{minipage}[t]{13cm}{\footnotesize
            If the processing attempt completes, the attempt is marked as completed
and tagged as potential for the next test steps. These campaign tags are
used to make pre-release QA queries simpler.

            \vspace{\dp0}
            } \end{minipage} \\ \cline{2-3}
            & Test Data &
            \begin{minipage}[t]{13cm}{\footnotesize
                No data.
                \vspace{\dp0}
            } \end{minipage} \\ \cline{2-3}
            & Expected Result &
        \\ \midrule

            \multirow{3}{*}{ 11 } & Description &
            \begin{minipage}[t]{13cm}{\footnotesize
            If the processing attempt fails, the attempt is marked as failed.

            \vspace{\dp0}
            } \end{minipage} \\ \cline{2-3}
            & Test Data &
            \begin{minipage}[t]{13cm}{\footnotesize
                No data.
                \vspace{\dp0}
            } \end{minipage} \\ \cline{2-3}
            & Expected Result &
        \\ \midrule

            \multirow{3}{*}{ 12 } & Description &
            \begin{minipage}[t]{13cm}{\footnotesize
            If the processing attempt fails due to certain infrastructure
configuration or transient instability (e.g., network blips), the
processing of the tract can be tried again after the problem is
resolved.

            \vspace{\dp0}
            } \end{minipage} \\ \cline{2-3}
            & Test Data &
            \begin{minipage}[t]{13cm}{\footnotesize
                No data.
                \vspace{\dp0}
            } \end{minipage} \\ \cline{2-3}
            & Expected Result &
        \\ \midrule

            \multirow{3}{*}{ 13 } & Description &
            \begin{minipage}[t]{13cm}{\footnotesize
            Checks.~When the execution finishes, the success of the execution will
be verified by checking the existence of the expected output data.

\begin{itemize}
\tightlist
\item
  (a) For each of the expected data products types (listed in ยง4.3.2)
  and each of the expected units (visits, patches, etc), verify the data
  product is in the Data Backbone and has filesize greater than zero via
  DB queries.
\item
  (b) Verify the physical and location information in Data Backbone DB
  matches theData Backbone filesystem and vice-versa.
\end{itemize}

            \vspace{\dp0}
            } \end{minipage} \\ \cline{2-3}
            & Test Data &
            \begin{minipage}[t]{13cm}{\footnotesize
                No data.
                \vspace{\dp0}
            } \end{minipage} \\ \cline{2-3}
            & Expected Result &
        \\ \midrule

            \multirow{3}{*}{ 14 } & Description &
            \begin{minipage}[t]{13cm}{\footnotesize
            Check that for each data product type has appropriate metadata saved for
each file as defined in ``Test Configuration'' section (ยง4.2.7).

            \vspace{\dp0}
            } \end{minipage} \\ \cline{2-3}
            & Test Data &
            \begin{minipage}[t]{13cm}{\footnotesize
                No data.
                \vspace{\dp0}
            } \end{minipage} \\ \cline{2-3}
            & Expected Result &
        \\ \midrule

            \multirow{3}{*}{ 15 } & Description &
            \begin{minipage}[t]{13cm}{\footnotesize
            Check provenance

\begin{itemize}
\tightlist
\item
  (a) Verify that each file can be linked with the step and processing
  attempt that created it via the Data Backbone.
\item
  (b) Verify that the information linking input files to each step was
  saved to the Oracle database.
\end{itemize}

            \vspace{\dp0}
            } \end{minipage} \\ \cline{2-3}
            & Test Data &
            \begin{minipage}[t]{13cm}{\footnotesize
                No data.
                \vspace{\dp0}
            } \end{minipage} \\ \cline{2-3}
            & Expected Result &
        \\ \midrule

            \multirow{3}{*}{ 16 } & Description &
            \begin{minipage}[t]{13cm}{\footnotesize
            Check runtime metrics, such as the number of executions of each code,
wallclock per step, wallclock per tract, etc.

            \vspace{\dp0}
            } \end{minipage} \\ \cline{2-3}
            & Test Data &
            \begin{minipage}[t]{13cm}{\footnotesize
                No data.
                \vspace{\dp0}
            } \end{minipage} \\ \cline{2-3}
            & Expected Result &
        \\ \midrule
    \end{longtable}

\subsection{LVV-T12 - DRP-00-10: Data Release Includes Required Data Products}\label{lvv-t12}

\begin{longtable}[]{llllll}
\toprule
Version & Status & Priority & Verification Type & Owner
\\\midrule
1 & Approved & Normal &
Test & Jim Bosch
\\\bottomrule
\multicolumn{6}{c}{ Open \href{https://jira.lsstcorp.org/secure/Tests.jspa\#/testCase/LVV-T12}{LVV-T12} in Jira } \\
\end{longtable}

\subsubsection{Verification Elements}
\begin{itemize}
\item \href{https://jira.lsstcorp.org/browse/LVV-165}{LVV-165} - DMS-REQ-0334-V-01: Persisting Data Products

\item \href{https://jira.lsstcorp.org/browse/LVV-98}{LVV-98} - DMS-REQ-0267-V-01: Source Catalog

\item \href{https://jira.lsstcorp.org/browse/LVV-99}{LVV-99} - DMS-REQ-0268-V-01: Forced-Source Catalog

\item \href{https://jira.lsstcorp.org/browse/LVV-106}{LVV-106} - DMS-REQ-0275-V-01: Object Catalog

\item \href{https://jira.lsstcorp.org/browse/LVV-110}{LVV-110} - DMS-REQ-0279-V-01: Deep Detection Coadds

\item \href{https://jira.lsstcorp.org/browse/LVV-125}{LVV-125} - DMS-REQ-0294-V-01: Processing of Datasets

\end{itemize}

\subsubsection{Test Items}
This test will check that the basic data products which should be in an
data release are generated by execution of the science payload.\\
These products will include:

\begin{itemize}
\tightlist
\item
  Source catalogs, derived from PVIs and coadded images (DMS-REQ-0267 \&
  DMS-REQ-0277);
\item
  Forced source catalogs (DMS-REQ-0268);
\item
  Object catalogs (DMS-REQ-0275);
\item
  Processed visit images (PVIs; DMS-REQ-0069);
\item
  Coadded images (DMS-REQ-0279);
\end{itemize}


\subsubsection{Predecessors}
LVV-T10 (DRP-00-00)

\subsubsection{Environment Needs}

\paragraph{Software}
Release 14.0 of the DM Software Stack will be pre-installed (following
the procedure described in DRP-00-00)

\paragraph{Hardware}
The test shall be carried out on a machine with at least 16 GB of RAM
and multiple CPU cores which has access to the /datasets shared (GPFS)
filesystem at the LSST Data Facility.

\subsubsection{Input Specification}
A complete processing of the Hyper Suprime-Cam ``RC1'' dataset (Appendix
A.1.1 through the DRP Science Payload.\\
This dataset shall be made available in a standard LSST data repository,
accessible via the ``Data Butler''.\\
It is not required that all combinations of visit and CCD have been
processed successfully: a number of failures are expected. However,
documentation to describe processing failures should be provided.

\subsubsection{Output Specification}
None.

\subsubsection{Test Procedure}
    \begin{longtable}[]{p{1.3cm}p{2cm}p{13cm}}
    %\toprule
    Step & \multicolumn{2}{@{}l}{Description, Input Data and Expected Result} \\ \toprule
    \endhead

            \multirow{3}{*}{ 1 } & Description &
            \begin{minipage}[t]{13cm}{\footnotesize
            The DM Stack shall be initialized using the loadLSST script (as
described in DRP-00-00).

            \vspace{\dp0}
            } \end{minipage} \\ \cline{2-3}
            & Test Data &
            \begin{minipage}[t]{13cm}{\footnotesize
                No data.
                \vspace{\dp0}
            } \end{minipage} \\ \cline{2-3}
            & Expected Result &
        \\ \midrule

            \multirow{3}{*}{ 2 } & Description &
            \begin{minipage}[t]{13cm}{\footnotesize
            A ``Data Butler'' will be initialized to access the repository.

            \vspace{\dp0}
            } \end{minipage} \\ \cline{2-3}
            & Test Data &
            \begin{minipage}[t]{13cm}{\footnotesize
                No data.
                \vspace{\dp0}
            } \end{minipage} \\ \cline{2-3}
            & Expected Result &
        \\ \midrule

            \multirow{3}{*}{ 3 } & Description &
            \begin{minipage}[t]{13cm}{\footnotesize
            For each of the expected data products types (listed in Test Items
section ยง4.3.2) and each of the expected units (PVIs, coadds, etc), the
data product will be retrieved from the Butler and verified to be
non-empty.

            \vspace{\dp0}
            } \end{minipage} \\ \cline{2-3}
            & Test Data &
            \begin{minipage}[t]{13cm}{\footnotesize
                No data.
                \vspace{\dp0}
            } \end{minipage} \\ \cline{2-3}
            & Expected Result &
        \\ \midrule
    \end{longtable}

\subsection{LVV-T13 - DRP-00-15: Scientific Verification of Source Catalog}\label{lvv-t13}

\begin{longtable}[]{llllll}
\toprule
Version & Status & Priority & Verification Type & Owner
\\\midrule
1 & Approved & Normal &
Test & Jim Bosch
\\\bottomrule
\multicolumn{6}{c}{ Open \href{https://jira.lsstcorp.org/secure/Tests.jspa\#/testCase/LVV-T13}{LVV-T13} in Jira } \\
\end{longtable}

\subsubsection{Verification Elements}
\begin{itemize}
\item \href{https://jira.lsstcorp.org/browse/LVV-165}{LVV-165} - DMS-REQ-0334-V-01: Persisting Data Products

\item \href{https://jira.lsstcorp.org/browse/LVV-98}{LVV-98} - DMS-REQ-0267-V-01: Source Catalog

\item \href{https://jira.lsstcorp.org/browse/LVV-178}{LVV-178} - DMS-REQ-0347-V-01: Measurements in catalogs

\item \href{https://jira.lsstcorp.org/browse/LVV-162}{LVV-162} - DMS-REQ-0331-V-01: Computing Derived Quantities

\end{itemize}

\subsubsection{Test Items}
This test will check that the source catalogs delivered by the DRP
science payload meet the requirements laid down by \citeds{LSE-61}.\\
Specifically, this will demonstrate that:

\begin{itemize}
\tightlist
\item
  Measurements in the catalog are presented in flux units
  (DMS-REQ-0347);
\item
  Derived quantities are provided in pre-computed columns
  (DMS-REQ-0331);
\item
  Aperture corrections for different photometry algorithms are
  consistent.
\item
  Photometry measurements are consistent with reference catalog
  photometry (including sources not used in photometric calibration).
\item
  Astrometry measurements are consistent with reference catalog
  positions (including sources not used in astrometric calibration).
\end{itemize}

This test does not include quantitative targets for the science quality
criteria; we instead require for each test that we be able to quickly
construct a plot in which such a target can be visualized.


\subsubsection{Predecessors}
lvv-t10 (DRP-00-00), lvv-t12 (DRP-00-10)

\subsubsection{Environment Needs}

\paragraph{Software}
Release 14.0 of the DM Software Stack will be pre-installed (following
the procedure described in DRP-00-00).

\paragraph{Hardware}
The test shall be carried out on a machine with at least 16 GB of RAM
and multiple CPU cores which has access to the /datasets shared (GPFS)
filesystem at the LSST Data Facility.

\subsubsection{Input Specification}
A complete processing of the Hyper Suprime-Cam ``RC1'' dataset (Appendix
A.1.1 through the DRP Science Payload.\\
This dataset shall be made available in a standard LSST data repository,
accessible via the ``Data Butler''.\\
It is not required that all combinations of visit and CCD have been
processed successfully: a number of failures are expected. However,
documentation to describe processing failures should be provided.

\subsubsection{Output Specification}
None.

\subsubsection{Test Procedure}
    \begin{longtable}[]{p{1.3cm}p{2cm}p{13cm}}
    %\toprule
    Step & \multicolumn{2}{@{}l}{Description, Input Data and Expected Result} \\ \toprule
    \endhead

            \multirow{3}{*}{ 1 } & Description &
            \begin{minipage}[t]{13cm}{\footnotesize
            The DM Stack shall be initialized using the loadLSST script (as
described in LVV-T10 - DRP-00-00).

            \vspace{\dp0}
            } \end{minipage} \\ \cline{2-3}
            & Test Data &
            \begin{minipage}[t]{13cm}{\footnotesize
                No data.
                \vspace{\dp0}
            } \end{minipage} \\ \cline{2-3}
            & Expected Result &
        \\ \midrule

            \multirow{3}{*}{ 2 } & Description &
            \begin{minipage}[t]{13cm}{\footnotesize
            A ``Data Butler'' will be initialized to access the repository.

            \vspace{\dp0}
            } \end{minipage} \\ \cline{2-3}
            & Test Data &
            \begin{minipage}[t]{13cm}{\footnotesize
                No data.
                \vspace{\dp0}
            } \end{minipage} \\ \cline{2-3}
            & Expected Result &
        \\ \midrule

            \multirow{3}{*}{ 3 } & Description &
            \begin{minipage}[t]{13cm}{\footnotesize
            Scripts from the pipe\_analysis package will be run on every visit to
check for the presence of data products and make plots.

            \vspace{\dp0}
            } \end{minipage} \\ \cline{2-3}
            & Test Data &
            \begin{minipage}[t]{13cm}{\footnotesize
                No data.
                \vspace{\dp0}
            } \end{minipage} \\ \cline{2-3}
            & Expected Result &
        \\ \midrule
    \end{longtable}

\subsection{LVV-T14 - DRP-00-25: Scientific Verification of Object Catalog}\label{lvv-t14}

\begin{longtable}[]{llllll}
\toprule
Version & Status & Priority & Verification Type & Owner
\\\midrule
1 & Approved & Normal &
Test & Jim Bosch
\\\bottomrule
\multicolumn{6}{c}{ Open \href{https://jira.lsstcorp.org/secure/Tests.jspa\#/testCase/LVV-T14}{LVV-T14} in Jira } \\
\end{longtable}

\subsubsection{Verification Elements}
\begin{itemize}
\item \href{https://jira.lsstcorp.org/browse/LVV-165}{LVV-165} - DMS-REQ-0334-V-01: Persisting Data Products

\item \href{https://jira.lsstcorp.org/browse/LVV-106}{LVV-106} - DMS-REQ-0275-V-01: Object Catalog

\item \href{https://jira.lsstcorp.org/browse/LVV-178}{LVV-178} - DMS-REQ-0347-V-01: Measurements in catalogs

\item \href{https://jira.lsstcorp.org/browse/LVV-162}{LVV-162} - DMS-REQ-0331-V-01: Computing Derived Quantities

\end{itemize}

\subsubsection{Test Items}
This test will check that the object catalogs delivered by the DRP
science payload meet the requirements laid down by \citeds{LSE-61}.\\
Specifically, this will demonstrate that:

\begin{itemize}
\tightlist
\item
  Measurements in the catalog are presented in flux units
  (DMS-REQ-0347);
\item
  Derived quantities are provided in pre-computed columns
  (DMS-REQ-0331);
\item
  Aperture corrections for different photometry algorithms are
  consistent.
\item
  PSF models correctly predict the ellipticities of stars over each
  tract.
\item
  Photometry measurements are consistent with reference catalog
  photometry (including sources not used in photometric calibration).
\item
  Astrometry measurements are consistent with reference catalog
  positions (including sources not used in astrometric calibration).
\item
  Forced and unforced photometry measurements are consistent.
\item
  The slope of the stellar locus in color-color space is not a function
  of position on the sky.
\end{itemize}

This test does not include quantitative targets for the science quality
criteria; we instead re- quire for each test that we be able to quickly
construct a plot in which such a target can be visualized.\\
All science quality tests in this section shall distinguish between
blended and isolated objects.


\subsubsection{Predecessors}
LVV-T10 (DRP-00-00)\\
LVV-T12 (DRP-00-10)

\subsubsection{Environment Needs}

\paragraph{Software}
Release 14.0 of the DM Software Stack will be pre-installed (following
the procedure described in DRP-00-00).

\paragraph{Hardware}
The test shall be carried out on a machine with at least 16 GB of RAM
and multiple CPU cores which has access to the /datasets shared (GPFS)
filesystem at the LSST Data Facility.

\subsubsection{Input Specification}
A complete processing of the Hyper Suprime-Cam ``RC1'' dataset (Appendix
A.1.1 through the DRP Science Payload.\\
This dataset shall be made available in a standard LSST data repository,
accessible via the ``Data Butler''.\\
It is not required that all combinations of visit and CCD have been
processed successfully: a number of failures are expected. However,
documentation to describe processing failures should be provided.

\subsubsection{Output Specification}
None.

\subsubsection{Test Procedure}
    \begin{longtable}[]{p{1.3cm}p{2cm}p{13cm}}
    %\toprule
    Step & \multicolumn{2}{@{}l}{Description, Input Data and Expected Result} \\ \toprule
    \endhead

            \multirow{3}{*}{ 1 } & Description &
            \begin{minipage}[t]{13cm}{\footnotesize
            The DM Stack shall be initialized using the loadLSST script (as
described in LVV-T10 - DRP-00-00).

            \vspace{\dp0}
            } \end{minipage} \\ \cline{2-3}
            & Test Data &
            \begin{minipage}[t]{13cm}{\footnotesize
                No data.
                \vspace{\dp0}
            } \end{minipage} \\ \cline{2-3}
            & Expected Result &
        \\ \midrule

            \multirow{3}{*}{ 2 } & Description &
            \begin{minipage}[t]{13cm}{\footnotesize
            A ``Data Butler'' will be initialized to access the repository.

            \vspace{\dp0}
            } \end{minipage} \\ \cline{2-3}
            & Test Data &
            \begin{minipage}[t]{13cm}{\footnotesize
                No data.
                \vspace{\dp0}
            } \end{minipage} \\ \cline{2-3}
            & Expected Result &
        \\ \midrule

            \multirow{3}{*}{ 3 } & Description &
            \begin{minipage}[t]{13cm}{\footnotesize
            Scripts from the pipe\_analysis package will be run on every tract to
check for the presence of data products and make plots

            \vspace{\dp0}
            } \end{minipage} \\ \cline{2-3}
            & Test Data &
            \begin{minipage}[t]{13cm}{\footnotesize
                No data.
                \vspace{\dp0}
            } \end{minipage} \\ \cline{2-3}
            & Expected Result &
        \\ \midrule
    \end{longtable}

\subsection{LVV-T15 - DRP-00-30: Scientific Verification of Processed Visit Images}\label{lvv-t15}

\begin{longtable}[]{llllll}
\toprule
Version & Status & Priority & Verification Type & Owner
\\\midrule
1 & Approved & Normal &
Test & Jim Bosch
\\\bottomrule
\multicolumn{6}{c}{ Open \href{https://jira.lsstcorp.org/secure/Tests.jspa\#/testCase/LVV-T15}{LVV-T15} in Jira } \\
\end{longtable}

\subsubsection{Verification Elements}
\begin{itemize}
\item \href{https://jira.lsstcorp.org/browse/LVV-165}{LVV-165} - DMS-REQ-0334-V-01: Persisting Data Products

\item \href{https://jira.lsstcorp.org/browse/LVV-29}{LVV-29} - DMS-REQ-0069-V-01: Processed Visit Images

\item \href{https://jira.lsstcorp.org/browse/LVV-158}{LVV-158} - DMS-REQ-0327-V-01: Background Model Calculation

\item \href{https://jira.lsstcorp.org/browse/LVV-12}{LVV-12} - DMS-REQ-0029-V-01: Generate Photometric Zeropoint for Visit Image

\item \href{https://jira.lsstcorp.org/browse/LVV-30}{LVV-30} - DMS-REQ-0070-V-01: Generate PSF for Visit Images

\item \href{https://jira.lsstcorp.org/browse/LVV-13}{LVV-13} - DMS-REQ-0030-V-01: Absolute accuracy of WCS

\item \href{https://jira.lsstcorp.org/browse/LVV-31}{LVV-31} - DMS-REQ-0072-V-01: Processed Visit Image Content

\end{itemize}

\subsubsection{Test Items}
This test will check that the Processed Visit Images (PVIs) delivered by
the DRP science payload meet the requirements laid down by \citeds{LSE-61}.\\
Specifically, this will demonstrate that:

\begin{itemize}
\tightlist
\item
  Processed visit images have been generated and persisted during
  payload execution;
\item
  Each PVI includes a background model (DMS-REQ-0327), photometric
  zero-point (DMS- REQ-0029), spatially-varying PSF (DMS-REQ-0070) and
  WCS (DMS-REQ-0030).
\item
  Saturated pixels are correctly masked.
\item
  Pixels affected by cosmic rays are correctly masked.
\item
  The background is not oversubtracted around bright objects.
\end{itemize}

This test does not include quantitative targets for the science quality
criteria; we instead re- quire for each test that we be able to quickly
construct a plot or display summary images that allow such a target can
be visualized.


\subsubsection{Predecessors}
LVV-T10\\
LVV-T12

\subsubsection{Environment Needs}

\paragraph{Software}
Release 14.0 of the DM Software Stack will be pre-installed (following
the procedure described in DRP-00-00).\\[2\baselineskip]

\paragraph{Hardware}
The test shall be carried out on a machine with at least 16 GB of RAM
and multiple CPU cores which has access to the /datasets shared (GPFS)
filesystem at the LSST Data Facility.

\subsubsection{Input Specification}
A complete processing of the Hyper Suprime-Cam ``RC1'' dataset (Appendix
A.1.1 through the DRP Science Payload.\\
This dataset shall be made available in a standard LSST data repository,
accessible via the ``Data Butler''.\\
It is not required that all combinations of visit and CCD have been
processed successfully: a number of failures are expected. However,
documentation to describe processing failures should be provided.

\subsubsection{Output Specification}
None.

\subsubsection{Test Procedure}
    \begin{longtable}[]{p{1.3cm}p{2cm}p{13cm}}
    %\toprule
    Step & \multicolumn{2}{@{}l}{Description, Input Data and Expected Result} \\ \toprule
    \endhead

            \multirow{3}{*}{ 1 } & Description &
            \begin{minipage}[t]{13cm}{\footnotesize
            The DM Stack shall be initialized using the loadLSST script (as
described in LVV-T10 - DRP-00-00).

            \vspace{\dp0}
            } \end{minipage} \\ \cline{2-3}
            & Test Data &
            \begin{minipage}[t]{13cm}{\footnotesize
                No data.
                \vspace{\dp0}
            } \end{minipage} \\ \cline{2-3}
            & Expected Result &
        \\ \midrule

            \multirow{3}{*}{ 2 } & Description &
            \begin{minipage}[t]{13cm}{\footnotesize
            A ``Data Butler'' will be initialized to access the repository.

            \vspace{\dp0}
            } \end{minipage} \\ \cline{2-3}
            & Test Data &
            \begin{minipage}[t]{13cm}{\footnotesize
                No data.
                \vspace{\dp0}
            } \end{minipage} \\ \cline{2-3}
            & Expected Result &
        \\ \midrule

            \multirow{3}{*}{ 3 } & Description &
            \begin{minipage}[t]{13cm}{\footnotesize
            For each processed CCD, the PVI will be retrieved from the Butler, and
the existence of all components described in section Test Items
(ยง4.6.2) will be verified.

            \vspace{\dp0}
            } \end{minipage} \\ \cline{2-3}
            & Test Data &
            \begin{minipage}[t]{13cm}{\footnotesize
                No data.
                \vspace{\dp0}
            } \end{minipage} \\ \cline{2-3}
            & Expected Result &
        \\ \midrule

            \multirow{3}{*}{ 4 } & Description &
            \begin{minipage}[t]{13cm}{\footnotesize
            Scripts from the pipe\_analysis package will be run on every visit to
check for the presence of data products and make plots

            \vspace{\dp0}
            } \end{minipage} \\ \cline{2-3}
            & Test Data &
            \begin{minipage}[t]{13cm}{\footnotesize
                No data.
                \vspace{\dp0}
            } \end{minipage} \\ \cline{2-3}
            & Expected Result &
        \\ \midrule

            \multirow{3}{*}{ 5 } & Description &
            \begin{minipage}[t]{13cm}{\footnotesize
            Five sensors will be chosen at random from each of two visits and
inspected by eye for unmasked artifacts.

            \vspace{\dp0}
            } \end{minipage} \\ \cline{2-3}
            & Test Data &
            \begin{minipage}[t]{13cm}{\footnotesize
                No data.
                \vspace{\dp0}
            } \end{minipage} \\ \cline{2-3}
            & Expected Result &
        \\ \midrule
    \end{longtable}

\subsection{LVV-T16 - DRP-00-35: Scientific Verification of Coadd Images}\label{lvv-t16}

\begin{longtable}[]{llllll}
\toprule
Version & Status & Priority & Verification Type & Owner
\\\midrule
1 & Approved & Normal &
Test & Jim Bosch
\\\bottomrule
\multicolumn{6}{c}{ Open \href{https://jira.lsstcorp.org/secure/Tests.jspa\#/testCase/LVV-T16}{LVV-T16} in Jira } \\
\end{longtable}

\subsubsection{Verification Elements}
\begin{itemize}
\item \href{https://jira.lsstcorp.org/browse/LVV-165}{LVV-165} - DMS-REQ-0334-V-01: Persisting Data Products

\item \href{https://jira.lsstcorp.org/browse/LVV-110}{LVV-110} - DMS-REQ-0279-V-01: Deep Detection Coadds

\item \href{https://jira.lsstcorp.org/browse/LVV-109}{LVV-109} - DMS-REQ-0278-V-01: Coadd Image Method Constraints

\item \href{https://jira.lsstcorp.org/browse/LVV-20}{LVV-20} - DMS-REQ-0047-V-01: Provide PSF for Coadded Images

\end{itemize}

\subsubsection{Test Items}
This test will check that the coadded images delivered by the DRP
science payload meet the requirements laid down by \citeds{LSE-61}.\\
Specifically, this will demonstrate that:

\begin{itemize}
\tightlist
\item
  Coadds have been generated and persisted during payload execution;~
\item
  Each coadd provides a spatially varying PSF model (DMS-REQ-0047).
\item
  Saturated pixels are correctly masked.
\item
  Pixels affected by satellite trails and ghosts are rejected from the
  coadd.
\item
  The background is not oversubtracted around bright objects.
\end{itemize}

This test does not include quantitative targets for the science quality
criteria; we instead require for each test that we be able to quickly
construct a plot or display summary images that allow such a target can
be visualized.\\[2\baselineskip]


\subsubsection{Predecessors}
LVV-T10 (DRP-00-00)\\
LVV-T12 (DRP-00-10)

\subsubsection{Environment Needs}

\paragraph{Software}
Release 14.0 of the DM Software Stack will be pre-installed (following
the procedure described in DRP-00-00).

\paragraph{Hardware}
The test shall be carried out on a machine with at least 16 GB of RAM
and multiple CPU cores which has access to the /datasets shared (GPFS)
filesystem at the LSST Data Facility.

\subsubsection{Input Specification}
A complete processing of the Hyper Suprime-Cam ``RC1'' dataset (Appendix
A.1.1 through the DRP Science Payload.\\
This dataset shall be made available in a standard LSST data repository,
accessible via the ``Data Butler''.\\
It is not required that all combinations of visit and CCD have been
processed successfully: a number of failures are expected. However,
documentation to describe processing failures should be provided.

\subsubsection{Output Specification}
None.

\subsubsection{Test Procedure}
    \begin{longtable}[]{p{1.3cm}p{2cm}p{13cm}}
    %\toprule
    Step & \multicolumn{2}{@{}l}{Description, Input Data and Expected Result} \\ \toprule
    \endhead

            \multirow{3}{*}{ 1 } & Description &
            \begin{minipage}[t]{13cm}{\footnotesize
            The DM Stack shall be initialized using the loadLSST script (as
described in LVV-T10 - DRP-00-00)

            \vspace{\dp0}
            } \end{minipage} \\ \cline{2-3}
            & Test Data &
            \begin{minipage}[t]{13cm}{\footnotesize
                No data.
                \vspace{\dp0}
            } \end{minipage} \\ \cline{2-3}
            & Expected Result &
        \\ \midrule

            \multirow{3}{*}{ 2 } & Description &
            \begin{minipage}[t]{13cm}{\footnotesize
            A ``Data Butler'' will be initialized to access the repository.

            \vspace{\dp0}
            } \end{minipage} \\ \cline{2-3}
            & Test Data &
            \begin{minipage}[t]{13cm}{\footnotesize
                No data.
                \vspace{\dp0}
            } \end{minipage} \\ \cline{2-3}
            & Expected Result &
        \\ \midrule

            \multirow{3}{*}{ 3 } & Description &
            \begin{minipage}[t]{13cm}{\footnotesize
            For each combination of tract/patch/filter, the PVI will be retrieved
from the Butler, and the existence of all components described in Test
items section ยง4.6.2 will be verified.

            \vspace{\dp0}
            } \end{minipage} \\ \cline{2-3}
            & Test Data &
            \begin{minipage}[t]{13cm}{\footnotesize
                No data.
                \vspace{\dp0}
            } \end{minipage} \\ \cline{2-3}
            & Expected Result &
        \\ \midrule

            \multirow{3}{*}{ 4 } & Description &
            \begin{minipage}[t]{13cm}{\footnotesize
            Scripts from the pipe\_analysis package will be run on every visit to
check for the presence of data products and make plots

            \vspace{\dp0}
            } \end{minipage} \\ \cline{2-3}
            & Test Data &
            \begin{minipage}[t]{13cm}{\footnotesize
                No data.
                \vspace{\dp0}
            } \end{minipage} \\ \cline{2-3}
            & Expected Result &
        \\ \midrule

            \multirow{3}{*}{ 5 } & Description &
            \begin{minipage}[t]{13cm}{\footnotesize
            Ten patches will be chosen at random and inspected by eye for unmasked
artifacts.

            \vspace{\dp0}
            } \end{minipage} \\ \cline{2-3}
            & Test Data &
            \begin{minipage}[t]{13cm}{\footnotesize
                No data.
                \vspace{\dp0}
            } \end{minipage} \\ \cline{2-3}
            & Expected Result &
        \\ \midrule
    \end{longtable}

\subsection{LVV-T17 - AG-00-00: Installation of the Alert Generation v16.0 science payload.}\label{lvv-t17}

\begin{longtable}[]{llllll}
\toprule
Version & Status & Priority & Verification Type & Owner
\\\midrule
1 & Approved & Normal &
Test & Eric Bellm
\\\bottomrule
\multicolumn{6}{c}{ Open \href{https://jira.lsstcorp.org/secure/Tests.jspa\#/testCase/LVV-T17}{LVV-T17} in Jira } \\
\end{longtable}

\subsubsection{Verification Elements}
\begin{itemize}
\item \href{https://jira.lsstcorp.org/browse/LVV-139}{LVV-139} - DMS-REQ-0308-V-01: Software Architecture to Enable Community Re-Use

\end{itemize}

\subsubsection{Test Items}
This test will check:

\begin{itemize}
\tightlist
\item
  That the Alert Generation science payload is available for
  distribution from documented channels;
\item
  That the Alert Generation science payload can be installed on LSST
  Data Facility-managed systems.
\end{itemize}


\subsubsection{Predecessors}
None.

\subsubsection{Environment Needs}

\paragraph{Software}
All prerequisite packages listed at
https://pipelines.lsst.io/install/prereqs/centos.html must be available
on the test system and on the LSST-VC compute node.

\paragraph{Hardware}
This test case shall be executed on a developer system at NCSA which
serves as the ``head node'' or otherwise provides access to filesystems
shared by the LSST Verification Cluster (LSST-VC). We assume that this
system will be lsst-dev01.ncsa.illinois.edu and the filesystem will be a
GPFS-based system mounted at /software.\\
The test also requires access to one LSST-VC compute node.

\subsubsection{Input Specification}
No input data is required for this test case.

\subsubsection{Output Specification}
The Alert Generation science payload will be made available on a shared
filesystem accessible from LSST-VC compute notes.

\subsubsection{Test Procedure}
    \begin{longtable}[]{p{1.3cm}p{2cm}p{13cm}}
    %\toprule
    Step & \multicolumn{2}{@{}l}{Description, Input Data and Expected Result} \\ \toprule
    \endhead

            \multirow{3}{*}{ 1 } & Description &
            \begin{minipage}[t]{13cm}{\footnotesize
            Release 16.0 of the LSST Science Pipelines will be installed into the
GPFS filesystem accessible at /software on lsst-dev01 following the
instructions at \url{https://pipelines.lsst.io/install/newinstall.html}
.

            \vspace{\dp0}
            } \end{minipage} \\ \cline{2-3}
            & Test Data &
            \begin{minipage}[t]{13cm}{\footnotesize
                No data.
                \vspace{\dp0}
            } \end{minipage} \\ \cline{2-3}
            & Expected Result &
        \\ \midrule

            \multirow{3}{*}{ 2 } & Description &
            \begin{minipage}[t]{13cm}{\footnotesize
            The lsst\_distrib top level package will be
enabled:\\[2\baselineskip]\hspace*{0.333em} ~ ~ ~source
/software/lsstsw/stack3/loadLSST.bash\\
\hspace*{0.333em} ~ ~ ~setup lsst\_distrib

            \vspace{\dp0}
            } \end{minipage} \\ \cline{2-3}
            & Test Data &
            \begin{minipage}[t]{13cm}{\footnotesize
                No data.
                \vspace{\dp0}
            } \end{minipage} \\ \cline{2-3}
            & Expected Result &
        \\ \midrule

            \multirow{3}{*}{ 3 } & Description &
            \begin{minipage}[t]{13cm}{\footnotesize
            The ``LSST Stack Demo'' package will be downloaded onto the test system
from
\href{https://github.com/lsst/lsst_dm_stack_demo/releases/tag/14.0}{https://github.com/lsst/lsst\_dm\_stack\_demo/releases/tag/16.0}
and uncompressed.

            \vspace{\dp0}
            } \end{minipage} \\ \cline{2-3}
            & Test Data &
            \begin{minipage}[t]{13cm}{\footnotesize
                No data.
                \vspace{\dp0}
            } \end{minipage} \\ \cline{2-3}
            & Expected Result &
        \\ \midrule

            \multirow{3}{*}{ 4 } & Description &
            \begin{minipage}[t]{13cm}{\footnotesize
            The demo package will be executed by following the instructions in its
``README`` file. The string ``Ok.`` should be returned. Specifically, we
execute:\\
\hspace*{0.333em} ~ ~ ~setup obs\_sdss\\
\hspace*{0.333em} ~ ~ ~./bin/demo.sh\\
\hspace*{0.333em} ~ ~ ~python bin/compare
expected/Linux64/detected-sources.txt

            \vspace{\dp0}
            } \end{minipage} \\ \cline{2-3}
            & Test Data &
            \begin{minipage}[t]{13cm}{\footnotesize
                No data.
                \vspace{\dp0}
            } \end{minipage} \\ \cline{2-3}
            & Expected Result &
        \\ \midrule

            \multirow{3}{*}{ 5 } & Description &
            \begin{minipage}[t]{13cm}{\footnotesize
            A shell on an LSST-VC compute node will now be obtained by executing:\\
\hspace*{0.333em} ~ ~\$ srun -I --pty bash

            \vspace{\dp0}
            } \end{minipage} \\ \cline{2-3}
            & Test Data &
            \begin{minipage}[t]{13cm}{\footnotesize
                No data.
                \vspace{\dp0}
            } \end{minipage} \\ \cline{2-3}
            & Expected Result &
        \\ \midrule

            \multirow{3}{*}{ 6 } & Description &
            \begin{minipage}[t]{13cm}{\footnotesize
            The demo package will be executed on the compute node and the same
result obtained.

            \vspace{\dp0}
            } \end{minipage} \\ \cline{2-3}
            & Test Data &
            \begin{minipage}[t]{13cm}{\footnotesize
                No data.
                \vspace{\dp0}
            } \end{minipage} \\ \cline{2-3}
            & Expected Result &
        \\ \midrule

            \multirow{3}{*}{ 7 } & Description &
            \begin{minipage}[t]{13cm}{\footnotesize
            The Alert Production datasets and packages are not yet part of
lsst\_distrib and so must be installed separately. They will be
installed as follows on the GPFS
filesystem:\\[2\baselineskip]\hspace*{0.333em} ~ setup git\_lfs\\
\hspace*{0.333em} ~ git clone
https://github.com/lsst/ap\_verify\_hits2015.git\\[2\baselineskip]\hspace*{0.333em}
~ export AP\_VERIFY\_HITS2015\_DIR=\$PWD/ap\_verify\_hits2015 cd
\$AP\_VERIFY\_HITS2015\_DIR\\
\hspace*{0.333em} ~ setup -r .\\
\hspace*{0.333em} ~ cd-\\
\hspace*{0.333em} ~\\
\hspace*{0.333em} ~ setup obs\_decam\\
\hspace*{0.333em} ~ git clone
https://github.com/lsst-dm/ap\_association\\
\hspace*{0.333em} ~ cd ap\_association\\
\hspace*{0.333em} ~ setup -k -r .\\
\hspace*{0.333em} ~ scons\\
\hspace*{0.333em} ~ cd-\\
\hspace*{0.333em} ~\\
\hspace*{0.333em} ~ git clone https://github.com/lsst-dm/ap\_pipe\\
\hspace*{0.333em} ~ cd ap\_pipe\\
\hspace*{0.333em} ~ setup -k -r .\\
\hspace*{0.333em} ~ scons\\
\hspace*{0.333em} ~ cd-\\
\hspace*{0.333em} ~\\
\hspace*{0.333em} ~ git clone https://github.com/lsst-dm/ap\_verify\\
\hspace*{0.333em} ~ cd ap\_verify\\
\hspace*{0.333em} ~ setup -k -r .\\
\hspace*{0.333em} ~ scons\\
\hspace*{0.333em} ~ cd-\\[2\baselineskip]and any errors or failures
reported.

            \vspace{\dp0}
            } \end{minipage} \\ \cline{2-3}
            & Test Data &
            \begin{minipage}[t]{13cm}{\footnotesize
                No data.
                \vspace{\dp0}
            } \end{minipage} \\ \cline{2-3}
            & Expected Result &
        \\ \midrule
    \end{longtable}

\subsection{LVV-T18 - AG-00-05: Alert Generation Produces Required Data Products}\label{lvv-t18}

\begin{longtable}[]{llllll}
\toprule
Version & Status & Priority & Verification Type & Owner
\\\midrule
1 & Approved & Normal &
Test & Eric Bellm
\\\bottomrule
\multicolumn{6}{c}{ Open \href{https://jira.lsstcorp.org/secure/Tests.jspa\#/testCase/LVV-T18}{LVV-T18} in Jira } \\
\end{longtable}

\subsubsection{Verification Elements}
\begin{itemize}
\item \href{https://jira.lsstcorp.org/browse/LVV-29}{LVV-29} - DMS-REQ-0069-V-01: Processed Visit Images

\item \href{https://jira.lsstcorp.org/browse/LVV-7}{LVV-7} - DMS-REQ-0010-V-01: Difference Exposures

\item \href{https://jira.lsstcorp.org/browse/LVV-100}{LVV-100} - DMS-REQ-0269-V-01: DIASource Catalog

\item \href{https://jira.lsstcorp.org/browse/LVV-102}{LVV-102} - DMS-REQ-0271-V-01: Max nearby galaxies associated with DIASource

\end{itemize}

\subsubsection{Test Items}
This test will check that the basic data products produced by Alert
Generation are generated by execution of the science payload.\\
These products will include:

\begin{itemize}
\tightlist
\item
  Processed visit images (PVIs; DMS-REQ-0069);
\item
  Difference Exposures (DMS-REQ-0010);
\item
  DIASource catalogs (DMS-REQ-0269);
\item
  DIAObject catalogs (DMS-REQ-0271);
\end{itemize}


\subsubsection{Predecessors}
LVV-T17 (AG-00-00)

\subsubsection{Environment Needs}

\paragraph{Software}
Release 16.0 of the DM Software Stack will be pre-installed (following
the procedure described in AG-00-00).

\paragraph{Hardware}
The test shall be carried out on a machine with at least 16 GB of RAM
and multiple CPU cores which has access to the /datasets shared (GPFS)
filesystem at the LSST Data Facility.

\subsubsection{Input Specification}
A complete processing of the DECam ``HiTS'' dataset, as defined at
https://dmtn-039.lsst.io/ and
https://github.com/lsst/ap\_verify\_hits2015, through the Alert
Generation science payload.\\
This dataset shall be made available in a standard LSST data repository,
accessible via the ``Data Butler''.\\
It is not required that all combinations of visit and CCD have been
processed successfully: a number of failures are expected. However,
documentation to describe processing failures should be provided.

\subsubsection{Output Specification}
None.

\subsubsection{Test Procedure}
    \begin{longtable}[]{p{1.3cm}p{2cm}p{13cm}}
    %\toprule
    Step & \multicolumn{2}{@{}l}{Description, Input Data and Expected Result} \\ \toprule
    \endhead

            \multirow{3}{*}{ 1 } & Description &
            \begin{minipage}[t]{13cm}{\footnotesize
            The DM Stack and Alert Processing packaged shall be initialized as
described in LVT-T17 (AG-00-00).

            \vspace{\dp0}
            } \end{minipage} \\ \cline{2-3}
            & Test Data &
            \begin{minipage}[t]{13cm}{\footnotesize
                No data.
                \vspace{\dp0}
            } \end{minipage} \\ \cline{2-3}
            & Expected Result &
        \\ \midrule

            \multirow{3}{*}{ 2 } & Description &
            \begin{minipage}[t]{13cm}{\footnotesize
            The alert generation processing will be executed using the verification
cluster:\\[2\baselineskip]```bash\\
python ap\_verify/bin/prepare\_demo\_slurm\_files.py\\
\# At present we must run a single ccd+visit to handle ingestion
before\\
\# parallel processing can begin\\
./ap\_verify/bin/exec\_demo\_run\_1ccd.sh 410915 25\\
ln -s ap\_verify/bin/demo\_run.sl\\
ln -s ap\_verify/bin/demo\_cmds.conf\\
sbatch demo\_run.sl\\
```\\[2\baselineskip]and any errors or failures reported.

            \vspace{\dp0}
            } \end{minipage} \\ \cline{2-3}
            & Test Data &
            \begin{minipage}[t]{13cm}{\footnotesize
                No data.
                \vspace{\dp0}
            } \end{minipage} \\ \cline{2-3}
            & Expected Result &
        \\ \midrule

            \multirow{3}{*}{ 3 } & Description &
            \begin{minipage}[t]{13cm}{\footnotesize
            A ``Data Butler'' will be initialized to access the repository.

            \vspace{\dp0}
            } \end{minipage} \\ \cline{2-3}
            & Test Data &
            \begin{minipage}[t]{13cm}{\footnotesize
                No data.
                \vspace{\dp0}
            } \end{minipage} \\ \cline{2-3}
            & Expected Result &
        \\ \midrule

            \multirow{3}{*}{ 4 } & Description &
            \begin{minipage}[t]{13cm}{\footnotesize
            For each of the expected data products types (listed in ยง4.2.2) and
each of the expected units (PVIs, catalogs, etc.), the data product will
be retrieved from the Butler and verified to be non-empty.

            \vspace{\dp0}
            } \end{minipage} \\ \cline{2-3}
            & Test Data &
            \begin{minipage}[t]{13cm}{\footnotesize
                No data.
                \vspace{\dp0}
            } \end{minipage} \\ \cline{2-3}
            & Expected Result &
        \\ \midrule

            \multirow{3}{*}{ 5 } & Description &
            \begin{minipage}[t]{13cm}{\footnotesize
            DIAObjects are currently only stored in a database, without shims to the
Butler, so the existence of the database table and its non-empty
contents will be verified by directly accessing it using sqlite3 and
executing appropriate SQL queries.

            \vspace{\dp0}
            } \end{minipage} \\ \cline{2-3}
            & Test Data &
            \begin{minipage}[t]{13cm}{\footnotesize
                No data.
                \vspace{\dp0}
            } \end{minipage} \\ \cline{2-3}
            & Expected Result &
        \\ \midrule
    \end{longtable}

\subsection{LVV-T19 - AG-00-10: Scientific Verification of Processed Visit Images}\label{lvv-t19}

\begin{longtable}[]{llllll}
\toprule
Version & Status & Priority & Verification Type & Owner
\\\midrule
1 & Approved & Normal &
Test & Eric Bellm
\\\bottomrule
\multicolumn{6}{c}{ Open \href{https://jira.lsstcorp.org/secure/Tests.jspa\#/testCase/LVV-T19}{LVV-T19} in Jira } \\
\end{longtable}

\subsubsection{Verification Elements}
\begin{itemize}
\item \href{https://jira.lsstcorp.org/browse/LVV-29}{LVV-29} - DMS-REQ-0069-V-01: Processed Visit Images

\item \href{https://jira.lsstcorp.org/browse/LVV-158}{LVV-158} - DMS-REQ-0327-V-01: Background Model Calculation

\item \href{https://jira.lsstcorp.org/browse/LVV-12}{LVV-12} - DMS-REQ-0029-V-01: Generate Photometric Zeropoint for Visit Image

\item \href{https://jira.lsstcorp.org/browse/LVV-30}{LVV-30} - DMS-REQ-0070-V-01: Generate PSF for Visit Images

\item \href{https://jira.lsstcorp.org/browse/LVV-13}{LVV-13} - DMS-REQ-0030-V-01: Absolute accuracy of WCS

\item \href{https://jira.lsstcorp.org/browse/LVV-31}{LVV-31} - DMS-REQ-0072-V-01: Processed Visit Image Content

\end{itemize}

\subsubsection{Test Items}
This test will check that the Processed Visit Images (PVIs) delivered by
the alert generation science payload meet the requirements laid down by
\citeds{LSE-61}.\\
Specifically, this will demonstrate that:

\begin{itemize}
\tightlist
\item
  Processed visit images have been generated and persisted during
  payload execution;
\item
  Each PVI includes a science pixel array, a mask array, and a variance
  array. (DMS-REQ-0072).
\item
  Each PVI includes a background model (DMS-REQ-0327), photometric
  zero-point (DMS- REQ-0029), spatially-varying PSF (DMS-REQ-0070) and
  WCS (DMS-REQ-0030).
\item
  Saturated pixels are correctly masked.
\item
  Pixels affected by cosmic rays are correctly masked.
\item
  The background is not oversubtracted around bright objects.
\end{itemize}

This test does not include quantitative targets for the science quality
criteria.


\subsubsection{Predecessors}
LVT-T17 (AG-00-00)\\
LVT-T18 (AG-00-05)

\subsubsection{Environment Needs}

\paragraph{Software}
Release 14.0 of the DM Software Stack will be pre-installed (following
the procedure described in AG-00-00).

\paragraph{Hardware}
The test shall be carried out on a machine with at least 16 GB of RAM
and multiple CPU cores which has access to the /datasets shared (GPFS)
filesystem at the LSST Data Facility.

\subsubsection{Input Specification}
A complete processing of the DECam ``HiTS'' dataset, as defined at
https://dmtn-039.lsst.io/ and
https://github.com/lsst/ap\_verify\_hits2015, through the Alert
Generation science payload.\\
This dataset shall be made available in a standard LSST data repository,
accessible via the ``Data Butler''.\\
It is not required that all combinations of visit and CCD have been
processed successfully: a number of failures are expected. However,
documentation to describe processing failures should be provided.

\subsubsection{Output Specification}
None.

\subsubsection{Test Procedure}
    \begin{longtable}[]{p{1.3cm}p{2cm}p{13cm}}
    %\toprule
    Step & \multicolumn{2}{@{}l}{Description, Input Data and Expected Result} \\ \toprule
    \endhead

            \multirow{3}{*}{ 1 } & Description &
            \begin{minipage}[t]{13cm}{\footnotesize
            The DM Stack shall be initialized using the loadLSST script (as
described in LVV-T17 - AG-00-00).

            \vspace{\dp0}
            } \end{minipage} \\ \cline{2-3}
            & Test Data &
            \begin{minipage}[t]{13cm}{\footnotesize
                No data.
                \vspace{\dp0}
            } \end{minipage} \\ \cline{2-3}
            & Expected Result &
        \\ \midrule

            \multirow{3}{*}{ 2 } & Description &
            \begin{minipage}[t]{13cm}{\footnotesize
            A ``Data Butler'' will be initialized to access the repository.

            \vspace{\dp0}
            } \end{minipage} \\ \cline{2-3}
            & Test Data &
            \begin{minipage}[t]{13cm}{\footnotesize
                No data.
                \vspace{\dp0}
            } \end{minipage} \\ \cline{2-3}
            & Expected Result &
        \\ \midrule

            \multirow{3}{*}{ 3 } & Description &
            \begin{minipage}[t]{13cm}{\footnotesize
            For each processed CCD, the PVI will be retrieved from the Butler, and
the existence of all components described in ยง4.3.2 will be verified.

            \vspace{\dp0}
            } \end{minipage} \\ \cline{2-3}
            & Test Data &
            \begin{minipage}[t]{13cm}{\footnotesize
                No data.
                \vspace{\dp0}
            } \end{minipage} \\ \cline{2-3}
            & Expected Result &
        \\ \midrule

            \multirow{3}{*}{ 4 } & Description &
            \begin{minipage}[t]{13cm}{\footnotesize
            Five sensors will be chosen at random from each of two visits and
inspected by eye for unmasked artifacts.

            \vspace{\dp0}
            } \end{minipage} \\ \cline{2-3}
            & Test Data &
            \begin{minipage}[t]{13cm}{\footnotesize
                No data.
                \vspace{\dp0}
            } \end{minipage} \\ \cline{2-3}
            & Expected Result &
        \\ \midrule
    \end{longtable}

\subsection{LVV-T20 - AG-00-15: Scientific Verification of Difference Images}\label{lvv-t20}

\begin{longtable}[]{llllll}
\toprule
Version & Status & Priority & Verification Type & Owner
\\\midrule
1 & Approved & Normal &
Test & Eric Bellm
\\\bottomrule
\multicolumn{6}{c}{ Open \href{https://jira.lsstcorp.org/secure/Tests.jspa\#/testCase/LVV-T20}{LVV-T20} in Jira } \\
\end{longtable}

\subsubsection{Verification Elements}
\begin{itemize}
\item \href{https://jira.lsstcorp.org/browse/LVV-7}{LVV-7} - DMS-REQ-0010-V-01: Difference Exposures

\item \href{https://jira.lsstcorp.org/browse/LVV-32}{LVV-32} - DMS-REQ-0074-V-01: Difference Exposure Attributes

\end{itemize}

\subsubsection{Test Items}
This test will check that the difference images delivered by the Alert
Generation science pay- load meet the requirements laid down by
\citeds{LSE-61}.\\
Specifically, this will demonstrate that:

\begin{itemize}
\tightlist
\item
  Difference images have been generated and persisted during payload
  execution;
\item
  Each difference image includes information about the identity of the
  input exposures, and metadata such as a representation of the PSF
  matching kernel (DMS-REQ-0074);
\item
  Masks are correctly propagated from the input images.
\end{itemize}

This test does not include quantitative targets for the science quality
criteria.


\subsubsection{Predecessors}
LVV-T17 (AG-00-00)\\
LVV-T18 (AG-00-05)

\subsubsection{Environment Needs}

\paragraph{Software}
Release 14.0 of the DM Software Stack will be pre-installed (following
the procedure described in AG-00-00).

\paragraph{Hardware}
The test shall be carried out on a machine with at least 16 GB of RAM
and multiple CPU cores which has access to the /datasets shared (GPFS)
filesystem at the LSST Data Facility.

\subsubsection{Input Specification}
A complete processing of the DECam ``HiTS'' dataset, as defined at
https://dmtn-039.lsst. io/ and
https://github.com/lsst/ap\_verify\_hits2015, through the Alert
Generation science payload.\\
This dataset shall be made available in a standard LSST data repository,
accessible via the ``Data Butler''.\\
It is not required that all combinations of visit and CCD have been
processed successfully: a number of failures are expected. However,
documentation to describe processing failures should be provided.

\subsubsection{Output Specification}
None.

\subsubsection{Test Procedure}
    \begin{longtable}[]{p{1.3cm}p{2cm}p{13cm}}
    %\toprule
    Step & \multicolumn{2}{@{}l}{Description, Input Data and Expected Result} \\ \toprule
    \endhead

            \multirow{3}{*}{ 1 } & Description &
            \begin{minipage}[t]{13cm}{\footnotesize
            The DM Stack shall be initialized using the loadLSST script (as
described in LVV-T-17 AG-00-00).

            \vspace{\dp0}
            } \end{minipage} \\ \cline{2-3}
            & Test Data &
            \begin{minipage}[t]{13cm}{\footnotesize
                No data.
                \vspace{\dp0}
            } \end{minipage} \\ \cline{2-3}
            & Expected Result &
        \\ \midrule

            \multirow{3}{*}{ 2 } & Description &
            \begin{minipage}[t]{13cm}{\footnotesize
            A ``Data Butler'' will be initialized to access the repository.

            \vspace{\dp0}
            } \end{minipage} \\ \cline{2-3}
            & Test Data &
            \begin{minipage}[t]{13cm}{\footnotesize
                No data.
                \vspace{\dp0}
            } \end{minipage} \\ \cline{2-3}
            & Expected Result &
        \\ \midrule

            \multirow{3}{*}{ 3 } & Description &
            \begin{minipage}[t]{13cm}{\footnotesize
            For each processed CCD, the difference image will be retrieved from the
Butler, and the existence of all components described in ยง4.4.2 will be
verified.

            \vspace{\dp0}
            } \end{minipage} \\ \cline{2-3}
            & Test Data &
            \begin{minipage}[t]{13cm}{\footnotesize
                No data.
                \vspace{\dp0}
            } \end{minipage} \\ \cline{2-3}
            & Expected Result &
        \\ \midrule

            \multirow{3}{*}{ 4 } & Description &
            \begin{minipage}[t]{13cm}{\footnotesize
            Five sensors will be chosen at random from each of two visits and the
masks of the input and difference images compared by eye.

            \vspace{\dp0}
            } \end{minipage} \\ \cline{2-3}
            & Test Data &
            \begin{minipage}[t]{13cm}{\footnotesize
                No data.
                \vspace{\dp0}
            } \end{minipage} \\ \cline{2-3}
            & Expected Result &
        \\ \midrule
    \end{longtable}

\subsection{LVV-T21 - AG-00-20: Scientific Verification of DIASource Catalog}\label{lvv-t21}

\begin{longtable}[]{llllll}
\toprule
Version & Status & Priority & Verification Type & Owner
\\\midrule
1 & Approved & Normal &
Test & Eric Bellm
\\\bottomrule
\multicolumn{6}{c}{ Open \href{https://jira.lsstcorp.org/secure/Tests.jspa\#/testCase/LVV-T21}{LVV-T21} in Jira } \\
\end{longtable}

\subsubsection{Verification Elements}
\begin{itemize}
\item \href{https://jira.lsstcorp.org/browse/LVV-100}{LVV-100} - DMS-REQ-0269-V-01: DIASource Catalog

\item \href{https://jira.lsstcorp.org/browse/LVV-101}{LVV-101} - DMS-REQ-0270-V-01: Faint DIASource Measurements

\item \href{https://jira.lsstcorp.org/browse/LVV-178}{LVV-178} - DMS-REQ-0347-V-01: Measurements in catalogs

\item \href{https://jira.lsstcorp.org/browse/LVV-162}{LVV-162} - DMS-REQ-0331-V-01: Computing Derived Quantities

\item \href{https://jira.lsstcorp.org/browse/LVV-18}{LVV-18} - DMS-REQ-0043-V-01: Provide Calibrated Photometry

\end{itemize}

\subsubsection{Test Items}
This test will check that the difference image source catalogs delivered
by the Alert Generation science payload meet the requirements laid down
by \citeds{LSE-61}.

\begin{itemize}
\tightlist
\item
  Specifically, this will demonstrate that:
\item
  Measurements in the catalog are presented in flux units
  (DMS-REQ-0347);
\item
  Each DIASource record contains an appropriate subset of the attributes
  required by DMS-REQ-0269. In particular, the LDM-503-3-era pipeline is
  expected to provide DIASource positions (sky and focal plane), fluxes,
  and flags indicative of issues encountered during processing.
\item
  Faint DIASources satisfying additional criteria are stored
  (DMS-REQ-0270).
\item
  Derived quantities are provided in pre-computed columns
  (DMS-REQ-0331);
\end{itemize}

This test does not include quantitative targets for the science quality
criteria.\\[2\baselineskip]


\subsubsection{Predecessors}
LVT-T17 (AG-00-00)\\
LVT-T18 (AG-00-05)

\subsubsection{Environment Needs}

\paragraph{Software}
Release 14.0 of the DM Software Stack will be pre-installed (following
the procedure described in AG-00-00).

\paragraph{Hardware}
The test shall be carried out on a machine with at least 16 GB of RAM
and multiple CPU cores which has access to the /datasets shared (GPFS)
filesystem at the LSST Data Facility.

\subsubsection{Input Specification}
A complete processing of the DECam ``HiTS'' dataset, as defined at
https://dmtn-039.lsst. io/ and
https://github.com/lsst/ap\_verify\_hits2015, through the Alert
Generation science payload.\\
This dataset shall be made available in a standard LSST data repository,
accessible via the ``Data Butler''.\\
It is not required that all combinations of visit and CCD have been
processed successfully: a number of failures are expected. However,
documentation to describe processing failures should be provided.

\subsubsection{Output Specification}
None.

\subsubsection{Test Procedure}
    \begin{longtable}[]{p{1.3cm}p{2cm}p{13cm}}
    %\toprule
    Step & \multicolumn{2}{@{}l}{Description, Input Data and Expected Result} \\ \toprule
    \endhead

            \multirow{3}{*}{ 1 } & Description &
            \begin{minipage}[t]{13cm}{\footnotesize
            The DM Stack shall be initialized using the loadLSST script (as
described in LVV-T17 - AG-00-00).

            \vspace{\dp0}
            } \end{minipage} \\ \cline{2-3}
            & Test Data &
            \begin{minipage}[t]{13cm}{\footnotesize
                No data.
                \vspace{\dp0}
            } \end{minipage} \\ \cline{2-3}
            & Expected Result &
        \\ \midrule

            \multirow{3}{*}{ 2 } & Description &
            \begin{minipage}[t]{13cm}{\footnotesize
            A ``Data Butler'' will be initialized to access the repository.

            \vspace{\dp0}
            } \end{minipage} \\ \cline{2-3}
            & Test Data &
            \begin{minipage}[t]{13cm}{\footnotesize
                No data.
                \vspace{\dp0}
            } \end{minipage} \\ \cline{2-3}
            & Expected Result &
        \\ \midrule

            \multirow{3}{*}{ 3 } & Description &
            \begin{minipage}[t]{13cm}{\footnotesize
            DIASource records will be accessed by querying the Butler, then examined
interactively at a Python prompt.

            \vspace{\dp0}
            } \end{minipage} \\ \cline{2-3}
            & Test Data &
            \begin{minipage}[t]{13cm}{\footnotesize
                No data.
                \vspace{\dp0}
            } \end{minipage} \\ \cline{2-3}
            & Expected Result &
        \\ \midrule
    \end{longtable}

\subsection{LVV-T22 - AG-00-25: Scientific Verification of DIAObject Catalog}\label{lvv-t22}

\begin{longtable}[]{llllll}
\toprule
Version & Status & Priority & Verification Type & Owner
\\\midrule
1 & Approved & Normal &
Test & Eric Bellm
\\\bottomrule
\multicolumn{6}{c}{ Open \href{https://jira.lsstcorp.org/secure/Tests.jspa\#/testCase/LVV-T22}{LVV-T22} in Jira } \\
\end{longtable}

\subsubsection{Verification Elements}
\begin{itemize}
\item \href{https://jira.lsstcorp.org/browse/LVV-116}{LVV-116} - DMS-REQ-0285-V-01: Level 1 Source Association

\item \href{https://jira.lsstcorp.org/browse/LVV-102}{LVV-102} - DMS-REQ-0271-V-01: Max nearby galaxies associated with DIASource

\item \href{https://jira.lsstcorp.org/browse/LVV-103}{LVV-103} - DMS-REQ-0272-V-01: DIAObject Attributes

\item \href{https://jira.lsstcorp.org/browse/LVV-178}{LVV-178} - DMS-REQ-0347-V-01: Measurements in catalogs

\item \href{https://jira.lsstcorp.org/browse/LVV-162}{LVV-162} - DMS-REQ-0331-V-01: Computing Derived Quantities

\item \href{https://jira.lsstcorp.org/browse/LVV-18}{LVV-18} - DMS-REQ-0043-V-01: Provide Calibrated Photometry

\end{itemize}

\subsubsection{Test Items}
This test will check that the DIAObject catalogs delivered by the Alert
Generation science pay- load meet the requirements laid down by
\citeds{LSE-61}.\\
Specifically, this will demonstrate that:

\begin{itemize}
\tightlist
\item
  DIAObjects are recorded with unique identifiers (DMS-REQ-0271);
\item
  Measurements in the catalog are presented in flux units
  (DMS-REQ-0347);
\item
  EachDIAObjectrecordcontainscontainsanappropriatesetofsummaryattributes(DMS-
  REQ-0271 and DMS-REQ-0272). Note:

  \begin{itemize}
  \tightlist
  \item
    This test is executed independently of the Data Release Production
    system. Hence, DIAObjects are not associated to Objects, and the
    association metadata specified by DMS-REQ-0271 is not expected to be
    available.
  \item
    TheLDM-503-3erapipelineisnotexpectedtocalculateorpersistallattributesspec-
    ified by DMS-REQ-0272 requirement.
  \end{itemize}
\item
  Relevant derived quantities are provided in pre-computed columns
  (DMS-REQ-0331);~
\end{itemize}

This test does not include quantitative targets for the science quality
criteria.


\subsubsection{Predecessors}
LVT-T17 (AG-00-00)\\
LVT-T18 (AG-00-05)

\subsubsection{Environment Needs}

\paragraph{Software}
Release 14.0 of the DM Software Stack will be pre-installed (following
the procedure described in AG-00-00).

\paragraph{Hardware}
The test shall be carried out on a machine with at least 16 GB of RAM
and multiple CPU cores which has access to the /datasets shared (GPFS)
filesystem at the LSST Data Facility.

\subsubsection{Input Specification}
A complete processing of the DECam ``HiTS'' dataset, as defined at
https://dmtn-039.lsst. io/ and
https://github.com/lsst/ap\_verify\_hits2015, through the Alert
Generation science payload.\\
This dataset shall be made available in a standard LSST data repository,
accessible via the ``Data Butler''.\\
It is not required that all combinations of visit and CCD have been
processed successfully: a number of failures are expected. However,
documentation to describe processing failures should be provided.

\subsubsection{Output Specification}
None.

\subsubsection{Test Procedure}
    \begin{longtable}[]{p{1.3cm}p{2cm}p{13cm}}
    %\toprule
    Step & \multicolumn{2}{@{}l}{Description, Input Data and Expected Result} \\ \toprule
    \endhead

            \multirow{3}{*}{ 1 } & Description &
            \begin{minipage}[t]{13cm}{\footnotesize
            The DM Stack shall be initialized using the loadLSST script (as
described in LVV-T17 - AG-00-00).

            \vspace{\dp0}
            } \end{minipage} \\ \cline{2-3}
            & Test Data &
            \begin{minipage}[t]{13cm}{\footnotesize
                No data.
                \vspace{\dp0}
            } \end{minipage} \\ \cline{2-3}
            & Expected Result &
        \\ \midrule

            \multirow{3}{*}{ 2 } & Description &
            \begin{minipage}[t]{13cm}{\footnotesize
            sqlite3 or Pythonâ\euro{}™s sqlalchemy module will be used to access the
Level 1 database.

            \vspace{\dp0}
            } \end{minipage} \\ \cline{2-3}
            & Test Data &
            \begin{minipage}[t]{13cm}{\footnotesize
                No data.
                \vspace{\dp0}
            } \end{minipage} \\ \cline{2-3}
            & Expected Result &
        \\ \midrule
    \end{longtable}

\subsection{LVV-T23 - Verify implementation of Storing Approximations of Per-pixel Metadata}\label{lvv-t23}

\begin{longtable}[]{llllll}
\toprule
Version & Status & Priority & Verification Type & Owner
\\\midrule
1 & Draft & Normal &
Test & Simon Krughoff
\\\bottomrule
\multicolumn{6}{c}{ Open \href{https://jira.lsstcorp.org/secure/Tests.jspa\#/testCase/LVV-T23}{LVV-T23} in Jira } \\
\end{longtable}

\subsubsection{Verification Elements}
\begin{itemize}
\item \href{https://jira.lsstcorp.org/browse/LVV-157}{LVV-157} - DMS-REQ-0326-V-01: Storing Approximations of Per-pixel Metadata

\end{itemize}

\subsubsection{Test Items}
\textbf{Test Items}\\[2\baselineskip]Show that the compressed form depth
and mask maps adequately represents the exact version of the same
information.


\subsubsection{Predecessors}

\subsubsection{Environment Needs}

\paragraph{Software}

\paragraph{Hardware}

\subsubsection{Input Specification}
Test data: A data repository containing a full DRP data reduction of the
HSC PDR dataset.

\subsubsection{Output Specification}

\subsubsection{Test Procedure}
    \begin{longtable}[]{p{1.3cm}p{2cm}p{13cm}}
    %\toprule
    Step & \multicolumn{2}{@{}l}{Description, Input Data and Expected Result} \\ \toprule
    \endhead

                \multirow{3}{*}{\parbox{1.3cm}{ 1-1
                {\scriptsize from \hyperref[lvv-t860]
                {LVV-T860} } } }

                & {\small Description} &
                \begin{minipage}[t]{13cm}{\scriptsize
                The `path` that you will use depends on where you are running the
science pipelines. Options:\\[2\baselineskip]

\begin{itemize}
\tightlist
\item
  local (newinstall.sh - based
  install):{[}path\_to\_installation{]}/loadLSST.bash
\item
  development cluster (``lsst-dev''):
  /software/lsstsw/stack/loadLSST.bash
\item
  LSP Notebook aspect (from a terminal):
  /opt/lsst/software/stack/loadLSST.bash
\end{itemize}

From the command line, execute the commands below in the example
code:\\[2\baselineskip]

                \vspace{\dp0}
                } \end{minipage} \\ \cdashline{2-3}
                & {\small Test Data} &
                \begin{minipage}[t]{13cm}{\scriptsize
                } \end{minipage} \\ \cdashline{2-3}
                & {\small Expected Result} &
                    \begin{minipage}[t]{13cm}{\scriptsize
                    Science pipeline software is available for use. If additional packages
are needed (for example, `obs' packages such as `obs\_subaru`), then
additional `setup` commands will be necessary.\\[2\baselineskip]To check
versions in use, type:\\
eups list -s

                    \vspace{\dp0}
                    } \end{minipage}
                \\ \hdashline


        \\ \midrule

                \multirow{3}{*}{\parbox{1.3cm}{ 2-1
                {\scriptsize from \hyperref[lvv-t987]
                {LVV-T987} } } }

                & {\small Description} &
                \begin{minipage}[t]{13cm}{\scriptsize
                Identify the path to the data repository, which we will refer to as
`DATA/path', then execute the following:

                \vspace{\dp0}
                } \end{minipage} \\ \cdashline{2-3}
                & {\small Test Data} &
                \begin{minipage}[t]{13cm}{\scriptsize
                } \end{minipage} \\ \cdashline{2-3}
                & {\small Expected Result} &
                    \begin{minipage}[t]{13cm}{\scriptsize
                    Butler repo available for reading.

                    \vspace{\dp0}
                    } \end{minipage}
                \\ \hdashline


        \\ \midrule

            \multirow{3}{*}{ 3 } & Description &
            \begin{minipage}[t]{13cm}{\footnotesize
            For each of the expected data products types (listed in Test Items
section ยง4.3.2) and each of the expected units (PVIs, coadds, etc),
retrieve the data product from the Butler and verify that it is
non-empty.

            \vspace{\dp0}
            } \end{minipage} \\ \cline{2-3}
            & Test Data &
            \begin{minipage}[t]{13cm}{\footnotesize
                No data.
                \vspace{\dp0}
            } \end{minipage} \\ \cline{2-3}
            & Expected Result &
        \\ \midrule

            \multirow{3}{*}{ 4 } & Description &
            \begin{minipage}[t]{13cm}{\footnotesize
            Create the coadd pixel level depth map for the HSC PDR dataset.

            \vspace{\dp0}
            } \end{minipage} \\ \cline{2-3}
            & Test Data &
            \begin{minipage}[t]{13cm}{\footnotesize
                No data.
                \vspace{\dp0}
            } \end{minipage} \\ \cline{2-3}
            & Expected Result &
        \\ \midrule

            \multirow{3}{*}{ 5 } & Description &
            \begin{minipage}[t]{13cm}{\footnotesize
            Generate compressed representation of the pixel level depth map.

            \vspace{\dp0}
            } \end{minipage} \\ \cline{2-3}
            & Test Data &
            \begin{minipage}[t]{13cm}{\footnotesize
                No data.
                \vspace{\dp0}
            } \end{minipage} \\ \cline{2-3}
            & Expected Result &
        \\ \midrule

            \multirow{3}{*}{ 6 } & Description &
            \begin{minipage}[t]{13cm}{\footnotesize
            Create the coadd pixel level mask map for the HSC PDR dataset.

            \vspace{\dp0}
            } \end{minipage} \\ \cline{2-3}
            & Test Data &
            \begin{minipage}[t]{13cm}{\footnotesize
                No data.
                \vspace{\dp0}
            } \end{minipage} \\ \cline{2-3}
            & Expected Result &
        \\ \midrule

            \multirow{3}{*}{ 7 } & Description &
            \begin{minipage}[t]{13cm}{\footnotesize
            Generate compressed representation of the mask map.

            \vspace{\dp0}
            } \end{minipage} \\ \cline{2-3}
            & Test Data &
            \begin{minipage}[t]{13cm}{\footnotesize
                No data.
                \vspace{\dp0}
            } \end{minipage} \\ \cline{2-3}
            & Expected Result &
        \\ \midrule

            \multirow{3}{*}{ 8 } & Description &
            \begin{minipage}[t]{13cm}{\footnotesize
            Sample randomly from both the pixel level and compressed depth maps.
~Compare the distribution of depths sampled from the pixel level depth
map to that sampled from the compressed representation.

            \vspace{\dp0}
            } \end{minipage} \\ \cline{2-3}
            & Test Data &
            \begin{minipage}[t]{13cm}{\footnotesize
                No data.
                \vspace{\dp0}
            } \end{minipage} \\ \cline{2-3}
            & Expected Result &
        \\ \midrule

            \multirow{3}{*}{ 9 } & Description &
            \begin{minipage}[t]{13cm}{\footnotesize
            Divide the mask planes into two groups: INFO and BAD. ~BAD flags are any
that would cause a particular pixel to be excluded from processing: e.g.
EDGE, SAT, BAD. ~Sample masks from both the pixel level mask map and the
compressed mask map.\\[2\baselineskip]For each sample, compute
sum(mask\_pixel xor mask\_compressed). ~Produce the distribution of the
number of bits that differ between the samples.\\[2\baselineskip]Repeat
for both the INFO flags and the BAD flags.

            \vspace{\dp0}
            } \end{minipage} \\ \cline{2-3}
            & Test Data &
            \begin{minipage}[t]{13cm}{\footnotesize
                No data.
                \vspace{\dp0}
            } \end{minipage} \\ \cline{2-3}
            & Expected Result &
        \\ \midrule
    \end{longtable}

\subsection{LVV-T24 - Verify implementation of Computing Derived Quantities}\label{lvv-t24}

\begin{longtable}[]{llllll}
\toprule
Version & Status & Priority & Verification Type & Owner
\\\midrule
1 & Draft & Normal &
Test & Melissa Graham
\\\bottomrule
\multicolumn{6}{c}{ Open \href{https://jira.lsstcorp.org/secure/Tests.jspa\#/testCase/LVV-T24}{LVV-T24} in Jira } \\
\end{longtable}

\subsubsection{Verification Elements}
\begin{itemize}
\item \href{https://jira.lsstcorp.org/browse/LVV-162}{LVV-162} - DMS-REQ-0331-V-01: Computing Derived Quantities

\end{itemize}

\subsubsection{Test Items}
To confirm that common derived quantities (apparent magnitude, FWHM in
arcsec, ellipticity) are available to an end-user by, e.g., ensuring a
color-color diagram is easy to construction, fitting functions to
derived data, or generating other common scientific derivatives.


\subsubsection{Predecessors}

\subsubsection{Environment Needs}

\paragraph{Software}

\paragraph{Hardware}

\subsubsection{Input Specification}
Example data set (e.g., non-LSST or LSST commissioning) loaded into the
Science Platform in a format consistent with the DPDD.

\subsubsection{Output Specification}

\subsubsection{Test Procedure}
    \begin{longtable}[]{p{1.3cm}p{2cm}p{13cm}}
    %\toprule
    Step & \multicolumn{2}{@{}l}{Description, Input Data and Expected Result} \\ \toprule
    \endhead

                \multirow{3}{*}{\parbox{1.3cm}{ 1-1
                {\scriptsize from \hyperref[lvv-t860]
                {LVV-T860} } } }

                & {\small Description} &
                \begin{minipage}[t]{13cm}{\scriptsize
                The `path` that you will use depends on where you are running the
science pipelines. Options:\\[2\baselineskip]

\begin{itemize}
\tightlist
\item
  local (newinstall.sh - based
  install):{[}path\_to\_installation{]}/loadLSST.bash
\item
  development cluster (``lsst-dev''):
  /software/lsstsw/stack/loadLSST.bash
\item
  LSP Notebook aspect (from a terminal):
  /opt/lsst/software/stack/loadLSST.bash
\end{itemize}

From the command line, execute the commands below in the example
code:\\[2\baselineskip]

                \vspace{\dp0}
                } \end{minipage} \\ \cdashline{2-3}
                & {\small Test Data} &
                \begin{minipage}[t]{13cm}{\scriptsize
                } \end{minipage} \\ \cdashline{2-3}
                & {\small Expected Result} &
                    \begin{minipage}[t]{13cm}{\scriptsize
                    Science pipeline software is available for use. If additional packages
are needed (for example, `obs' packages such as `obs\_subaru`), then
additional `setup` commands will be necessary.\\[2\baselineskip]To check
versions in use, type:\\
eups list -s

                    \vspace{\dp0}
                    } \end{minipage}
                \\ \hdashline


        \\ \midrule

                \multirow{3}{*}{\parbox{1.3cm}{ 2-1
                {\scriptsize from \hyperref[lvv-t987]
                {LVV-T987} } } }

                & {\small Description} &
                \begin{minipage}[t]{13cm}{\scriptsize
                Identify the path to the data repository, which we will refer to as
`DATA/path', then execute the following:

                \vspace{\dp0}
                } \end{minipage} \\ \cdashline{2-3}
                & {\small Test Data} &
                \begin{minipage}[t]{13cm}{\scriptsize
                } \end{minipage} \\ \cdashline{2-3}
                & {\small Expected Result} &
                    \begin{minipage}[t]{13cm}{\scriptsize
                    Butler repo available for reading.

                    \vspace{\dp0}
                    } \end{minipage}
                \\ \hdashline


        \\ \midrule

            \multirow{3}{*}{ 3 } & Description &
            \begin{minipage}[t]{13cm}{\footnotesize
            For each of the expected data product types (listed in Test Items
section ยง4.3.2) and each of the expected units (PVIs, coadds, etc),
retrieve the data product from the Butler and verify it to be non-empty.

            \vspace{\dp0}
            } \end{minipage} \\ \cline{2-3}
            & Test Data &
            \begin{minipage}[t]{13cm}{\footnotesize
                No data.
                \vspace{\dp0}
            } \end{minipage} \\ \cline{2-3}
            & Expected Result &
        \\ \midrule

            \multirow{3}{*}{ 4 } & Description &
            \begin{minipage}[t]{13cm}{\footnotesize
            Load into DPDD+Science Platform

            \vspace{\dp0}
            } \end{minipage} \\ \cline{2-3}
            & Test Data &
            \begin{minipage}[t]{13cm}{\footnotesize
                No data.
                \vspace{\dp0}
            } \end{minipage} \\ \cline{2-3}
            & Expected Result &
        \\ \midrule

            \multirow{3}{*}{ 5 } & Description &
            \begin{minipage}[t]{13cm}{\footnotesize
            Constructing color-color diagram and fitting stellar locus in Science
Platform.

            \vspace{\dp0}
            } \end{minipage} \\ \cline{2-3}
            & Test Data &
            \begin{minipage}[t]{13cm}{\footnotesize
                No data.
                \vspace{\dp0}
            } \end{minipage} \\ \cline{2-3}
            & Expected Result &
        \\ \midrule

            \multirow{3}{*}{ 6 } & Description &
            \begin{minipage}[t]{13cm}{\footnotesize
            Invite three members of commissioning team to create color-color diagram
from coadd catalogs based on merged coadd reference catalog.

            \vspace{\dp0}
            } \end{minipage} \\ \cline{2-3}
            & Test Data &
            \begin{minipage}[t]{13cm}{\footnotesize
                No data.
                \vspace{\dp0}
            } \end{minipage} \\ \cline{2-3}
            & Expected Result &
        \\ \midrule
    \end{longtable}

\subsection{LVV-T25 - Verify implementation of Denormalizing Database Tables}\label{lvv-t25}

\begin{longtable}[]{llllll}
\toprule
Version & Status & Priority & Verification Type & Owner
\\\midrule
1 & Draft & Normal &
Test & Colin Slater
\\\bottomrule
\multicolumn{6}{c}{ Open \href{https://jira.lsstcorp.org/secure/Tests.jspa\#/testCase/LVV-T25}{LVV-T25} in Jira } \\
\end{longtable}

\subsubsection{Verification Elements}
\begin{itemize}
\item \href{https://jira.lsstcorp.org/browse/LVV-163}{LVV-163} - DMS-REQ-0332-V-01: Denormalizing Database Tables

\end{itemize}

\subsubsection{Test Items}
Verify that commonly useful views of data are easy to obtain through the
Science Platform.


\subsubsection{Predecessors}

\subsubsection{Environment Needs}

\paragraph{Software}

\paragraph{Hardware}

\subsubsection{Input Specification}

\subsubsection{Output Specification}

\subsubsection{Test Procedure}
    \begin{longtable}[]{p{1.3cm}p{2cm}p{13cm}}
    %\toprule
    Step & \multicolumn{2}{@{}l}{Description, Input Data and Expected Result} \\ \toprule
    \endhead

            \multirow{3}{*}{ 1 } & Description &
            \begin{minipage}[t]{13cm}{\footnotesize
            Connect to the Science Platform's portal query interface.

            \vspace{\dp0}
            } \end{minipage} \\ \cline{2-3}
            & Test Data &
            \begin{minipage}[t]{13cm}{\footnotesize
                No data.
                \vspace{\dp0}
            } \end{minipage} \\ \cline{2-3}
            & Expected Result &
        \\ \midrule

            \multirow{3}{*}{ 2 } & Description &
            \begin{minipage}[t]{13cm}{\footnotesize
            List the available views in the database.

            \vspace{\dp0}
            } \end{minipage} \\ \cline{2-3}
            & Test Data &
            \begin{minipage}[t]{13cm}{\footnotesize
                No data.
                \vspace{\dp0}
            } \end{minipage} \\ \cline{2-3}
            & Expected Result &
        \\ \midrule

            \multirow{3}{*}{ 3 } & Description &
            \begin{minipage}[t]{13cm}{\footnotesize
            {Take 20 sampled queries and determine which are easily done on views
and which require complicated joins. Discuss the complicated ones and
determine if any could be simplified by adding additional views.}

            \vspace{\dp0}
            } \end{minipage} \\ \cline{2-3}
            & Test Data &
            \begin{minipage}[t]{13cm}{\footnotesize
                No data.
                \vspace{\dp0}
            } \end{minipage} \\ \cline{2-3}
            & Expected Result &
        \\ \midrule
    \end{longtable}

\subsection{LVV-T26 - Verify implementation of Maximum Likelihood Values and Covariances}\label{lvv-t26}

\begin{longtable}[]{llllll}
\toprule
Version & Status & Priority & Verification Type & Owner
\\\midrule
1 & Draft & Normal &
Test & Jim Bosch
\\\bottomrule
\multicolumn{6}{c}{ Open \href{https://jira.lsstcorp.org/secure/Tests.jspa\#/testCase/LVV-T26}{LVV-T26} in Jira } \\
\end{longtable}

\subsubsection{Verification Elements}
\begin{itemize}
\item \href{https://jira.lsstcorp.org/browse/LVV-164}{LVV-164} - DMS-REQ-0333-V-01: Maximum Likelihood Values and Covariances

\end{itemize}

\subsubsection{Test Items}
\begin{itemize}
\tightlist
\item
  Check that all measurements in source and object schemas include
  columns containing uncertainties, including covariances between
  jointly-measured quantities.
\item
  Check that all model-fit measurements in source and object schemas
  include columns that report goodness-of-fit.
\item
  Check that most sources and objects with successful measurements
  report finite uncertainty values for those measurements.
\item
  Check that most sources and objects with successful model-fit
  measurements report finite goodness-of-fit values.
\end{itemize}


\subsubsection{Predecessors}

\subsubsection{Environment Needs}

\paragraph{Software}

\paragraph{Hardware}

\subsubsection{Input Specification}

\subsubsection{Output Specification}

\subsubsection{Test Procedure}
    \begin{longtable}[]{p{1.3cm}p{2cm}p{13cm}}
    %\toprule
    Step & \multicolumn{2}{@{}l}{Description, Input Data and Expected Result} \\ \toprule
    \endhead

                \multirow{3}{*}{\parbox{1.3cm}{ 1-1
                {\scriptsize from \hyperref[lvv-t860]
                {LVV-T860} } } }

                & {\small Description} &
                \begin{minipage}[t]{13cm}{\scriptsize
                The `path` that you will use depends on where you are running the
science pipelines. Options:\\[2\baselineskip]

\begin{itemize}
\tightlist
\item
  local (newinstall.sh - based
  install):{[}path\_to\_installation{]}/loadLSST.bash
\item
  development cluster (``lsst-dev''):
  /software/lsstsw/stack/loadLSST.bash
\item
  LSP Notebook aspect (from a terminal):
  /opt/lsst/software/stack/loadLSST.bash
\end{itemize}

From the command line, execute the commands below in the example
code:\\[2\baselineskip]

                \vspace{\dp0}
                } \end{minipage} \\ \cdashline{2-3}
                & {\small Test Data} &
                \begin{minipage}[t]{13cm}{\scriptsize
                } \end{minipage} \\ \cdashline{2-3}
                & {\small Expected Result} &
                    \begin{minipage}[t]{13cm}{\scriptsize
                    Science pipeline software is available for use. If additional packages
are needed (for example, `obs' packages such as `obs\_subaru`), then
additional `setup` commands will be necessary.\\[2\baselineskip]To check
versions in use, type:\\
eups list -s

                    \vspace{\dp0}
                    } \end{minipage}
                \\ \hdashline


        \\ \midrule

                \multirow{3}{*}{\parbox{1.3cm}{ 2-1
                {\scriptsize from \hyperref[lvv-t987]
                {LVV-T987} } } }

                & {\small Description} &
                \begin{minipage}[t]{13cm}{\scriptsize
                Identify the path to the data repository, which we will refer to as
`DATA/path', then execute the following:

                \vspace{\dp0}
                } \end{minipage} \\ \cdashline{2-3}
                & {\small Test Data} &
                \begin{minipage}[t]{13cm}{\scriptsize
                } \end{minipage} \\ \cdashline{2-3}
                & {\small Expected Result} &
                    \begin{minipage}[t]{13cm}{\scriptsize
                    Butler repo available for reading.

                    \vspace{\dp0}
                    } \end{minipage}
                \\ \hdashline


        \\ \midrule

            \multirow{3}{*}{ 3 } & Description &
            \begin{minipage}[t]{13cm}{\footnotesize
            For each of the expected data product types (listed in Test Items
section ยง4.3.2) and each of the expected units (PVIs, coadds, etc),
retrieve the data product from the Butler and verify it to be non-empty.

            \vspace{\dp0}
            } \end{minipage} \\ \cline{2-3}
            & Test Data &
            \begin{minipage}[t]{13cm}{\footnotesize
                No data.
                \vspace{\dp0}
            } \end{minipage} \\ \cline{2-3}
            & Expected Result &
        \\ \midrule

            \multirow{3}{*}{ 4 } & Description &
            \begin{minipage}[t]{13cm}{\footnotesize
            Verify that maximum likelihood and covariant quantities are provided.
~Test and manually inspect that they are reasonable (finite,
appropriately normed).

            \vspace{\dp0}
            } \end{minipage} \\ \cline{2-3}
            & Test Data &
            \begin{minipage}[t]{13cm}{\footnotesize
                No data.
                \vspace{\dp0}
            } \end{minipage} \\ \cline{2-3}
            & Expected Result &
        \\ \midrule
    \end{longtable}

\subsection{LVV-T27 - Verify implementation of Data Availability}\label{lvv-t27}

\begin{longtable}[]{llllll}
\toprule
Version & Status & Priority & Verification Type & Owner
\\\midrule
1 & Draft & Normal &
Test & Gregory Dubois-Felsmann
\\\bottomrule
\multicolumn{6}{c}{ Open \href{https://jira.lsstcorp.org/secure/Tests.jspa\#/testCase/LVV-T27}{LVV-T27} in Jira } \\
\end{longtable}

\subsubsection{Verification Elements}
\begin{itemize}
\item \href{https://jira.lsstcorp.org/browse/LVV-177}{LVV-177} - DMS-REQ-0346-V-01: Data Availability

\end{itemize}

\subsubsection{Test Items}
Determine if all required categories of raw data (specifically
enumerated: raw exposures, calibration frames, telemetry, configuration
metadata) can be located through the Science Platform and are available
for download. ~Verify through (1) administrative review; (2) checking
with precursor data; (3) checking on early data feeds from the Summit
such as from AuxTel and ComCam.


\subsubsection{Predecessors}

\subsubsection{Environment Needs}

\paragraph{Software}

\paragraph{Hardware}

\subsubsection{Input Specification}

\subsubsection{Output Specification}

\subsubsection{Test Procedure}
    \begin{longtable}[]{p{1.3cm}p{2cm}p{13cm}}
    %\toprule
    Step & \multicolumn{2}{@{}l}{Description, Input Data and Expected Result} \\ \toprule
    \endhead

            \multirow{3}{*}{ 1 } & Description &
            \begin{minipage}[t]{13cm}{\footnotesize
            {Invite two reviewers to review that plan that seems reasonable to
expect the archiving and provision of raw data}

            \vspace{\dp0}
            } \end{minipage} \\ \cline{2-3}
            & Test Data &
            \begin{minipage}[t]{13cm}{\footnotesize
                No data.
                \vspace{\dp0}
            } \end{minipage} \\ \cline{2-3}
            & Expected Result &
        \\ \midrule

            \multirow{3}{*}{ 2 } & Description &
            \begin{minipage}[t]{13cm}{\footnotesize
            Pass a set of HSC data through (equal in size to the first public data
release) the data backbone through ingest and provide interface

            \vspace{\dp0}
            } \end{minipage} \\ \cline{2-3}
            & Test Data &
            \begin{minipage}[t]{13cm}{\footnotesize
                No data.
                \vspace{\dp0}
            } \end{minipage} \\ \cline{2-3}
            & Expected Result &
        \\ \midrule

            \multirow{3}{*}{ 3 } & Description &
            \begin{minipage}[t]{13cm}{\footnotesize
            Track the ingestion of AuxTel data during one month in 2018-2019 and
verify delivery and test download.

            \vspace{\dp0}
            } \end{minipage} \\ \cline{2-3}
            & Test Data &
            \begin{minipage}[t]{13cm}{\footnotesize
                No data.
                \vspace{\dp0}
            } \end{minipage} \\ \cline{2-3}
            & Expected Result &
        \\ \midrule
    \end{longtable}

\subsection{LVV-T28 - Verify implementation of measurements in catalogs from PVIs}\label{lvv-t28}

\begin{longtable}[]{llllll}
\toprule
Version & Status & Priority & Verification Type & Owner
\\\midrule
1 & Approved & Normal &
Test & Colin Slater
\\\bottomrule
\multicolumn{6}{c}{ Open \href{https://jira.lsstcorp.org/secure/Tests.jspa\#/testCase/LVV-T28}{LVV-T28} in Jira } \\
\end{longtable}

\subsubsection{Verification Elements}
\begin{itemize}
\item \href{https://jira.lsstcorp.org/browse/LVV-178}{LVV-178} - DMS-REQ-0347-V-01: Measurements in catalogs

\end{itemize}

\subsubsection{Test Items}
Verify that source measurements in catalogs containing measurements from
processed visit images are in flux units.


\subsubsection{Predecessors}

\subsubsection{Environment Needs}

\paragraph{Software}

\paragraph{Hardware}

\subsubsection{Input Specification}

\subsubsection{Output Specification}

\subsubsection{Test Procedure}
    \begin{longtable}[]{p{1.3cm}p{2cm}p{13cm}}
    %\toprule
    Step & \multicolumn{2}{@{}l}{Description, Input Data and Expected Result} \\ \toprule
    \endhead

                \multirow{3}{*}{\parbox{1.3cm}{ 1-1
                {\scriptsize from \hyperref[lvv-t987]
                {LVV-T987} } } }

                & {\small Description} &
                \begin{minipage}[t]{13cm}{\scriptsize
                Identify the path to the data repository, which we will refer to as
`DATA/path', then execute the following:

                \vspace{\dp0}
                } \end{minipage} \\ \cdashline{2-3}
                & {\small Test Data} &
                \begin{minipage}[t]{13cm}{\scriptsize
                } \end{minipage} \\ \cdashline{2-3}
                & {\small Expected Result} &
                    \begin{minipage}[t]{13cm}{\scriptsize
                    Butler repo available for reading.

                    \vspace{\dp0}
                    } \end{minipage}
                \\ \hdashline


        \\ \midrule

            \multirow{3}{*}{ 2 } & Description &
            \begin{minipage}[t]{13cm}{\footnotesize
            Identify and read an appropriate processed precursor dataset containing
coadds with the Butler.~

            \vspace{\dp0}
            } \end{minipage} \\ \cline{2-3}
            & Test Data &
            \begin{minipage}[t]{13cm}{\footnotesize
                No data.
                \vspace{\dp0}
            } \end{minipage} \\ \cline{2-3}
            & Expected Result &
        \\ \midrule

            \multirow{3}{*}{ 3 } & Description &
            \begin{minipage}[t]{13cm}{\footnotesize
            Verify that the single-visit catalog provides measurements in flux
units.

            \vspace{\dp0}
            } \end{minipage} \\ \cline{2-3}
            & Test Data &
            \begin{minipage}[t]{13cm}{\footnotesize
                No data.
                \vspace{\dp0}
            } \end{minipage} \\ \cline{2-3}
            & Expected Result &
                \begin{minipage}[t]{13cm}{\footnotesize
                Confirmation of measurements in catalogs encoded in flux units.

                \vspace{\dp0}
                } \end{minipage}
        \\ \midrule
    \end{longtable}

\subsection{LVV-T29 - Verify implementation of Raw Science Image Data Acquisition}\label{lvv-t29}

\begin{longtable}[]{llllll}
\toprule
Version & Status & Priority & Verification Type & Owner
\\\midrule
1 & Defined & Normal &
Test & Kian-Tat Lim
\\\bottomrule
\multicolumn{6}{c}{ Open \href{https://jira.lsstcorp.org/secure/Tests.jspa\#/testCase/LVV-T29}{LVV-T29} in Jira } \\
\end{longtable}

\subsubsection{Verification Elements}
\begin{itemize}
\item \href{https://jira.lsstcorp.org/browse/LVV-8}{LVV-8} - DMS-REQ-0018-V-01: Raw Science Image Data Acquisition

\end{itemize}

\subsubsection{Test Items}
Verify acquisition of raw data from L1 Test Stand DAQ while simulating
all modes


\subsubsection{Predecessors}

\subsubsection{Environment Needs}

\paragraph{Software}

\paragraph{Hardware}

\subsubsection{Input Specification}

\subsubsection{Output Specification}

\subsubsection{Test Procedure}
    \begin{longtable}[]{p{1.3cm}p{2cm}p{13cm}}
    %\toprule
    Step & \multicolumn{2}{@{}l}{Description, Input Data and Expected Result} \\ \toprule
    \endhead

            \multirow{3}{*}{ 1 } & Description &
            \begin{minipage}[t]{13cm}{\footnotesize
            {Ingest raw data from L1 Test Stand DAQ, simulating each observing
mode\\
}

            \vspace{\dp0}
            } \end{minipage} \\ \cline{2-3}
            & Test Data &
            \begin{minipage}[t]{13cm}{\footnotesize
                No data.
                \vspace{\dp0}
            } \end{minipage} \\ \cline{2-3}
            & Expected Result &
        \\ \midrule

            \multirow{3}{*}{ 2 } & Description &
            \begin{minipage}[t]{13cm}{\footnotesize
            O{bserve image and its metadata is present and queryable in the Data
Backbone.}

            \vspace{\dp0}
            } \end{minipage} \\ \cline{2-3}
            & Test Data &
            \begin{minipage}[t]{13cm}{\footnotesize
                No data.
                \vspace{\dp0}
            } \end{minipage} \\ \cline{2-3}
            & Expected Result &
                \begin{minipage}[t]{13cm}{\footnotesize
                Well-formed image data with appropriate associated metadata.

                \vspace{\dp0}
                } \end{minipage}
        \\ \midrule
    \end{longtable}

\subsection{LVV-T30 - Verify implementation of Wavefront Sensor Data Acquisition}\label{lvv-t30}

\begin{longtable}[]{llllll}
\toprule
Version & Status & Priority & Verification Type & Owner
\\\midrule
1 & Defined & Normal &
Test & Kian-Tat Lim
\\\bottomrule
\multicolumn{6}{c}{ Open \href{https://jira.lsstcorp.org/secure/Tests.jspa\#/testCase/LVV-T30}{LVV-T30} in Jira } \\
\end{longtable}

\subsubsection{Verification Elements}
\begin{itemize}
\item \href{https://jira.lsstcorp.org/browse/LVV-9}{LVV-9} - DMS-REQ-0020-V-01: Wavefront Sensor Data Acquisition

\end{itemize}

\subsubsection{Test Items}
Verify successful ingestion of wavefront sensor data from L1 Test Stand
DAQ while simulating all modes.


\subsubsection{Predecessors}

\subsubsection{Environment Needs}

\paragraph{Software}

\paragraph{Hardware}

\subsubsection{Input Specification}

\subsubsection{Output Specification}

\subsubsection{Test Procedure}
    \begin{longtable}[]{p{1.3cm}p{2cm}p{13cm}}
    %\toprule
    Step & \multicolumn{2}{@{}l}{Description, Input Data and Expected Result} \\ \toprule
    \endhead

            \multirow{3}{*}{ 1 } & Description &
            \begin{minipage}[t]{13cm}{\footnotesize
            {Ingest wavefront sensor data from L1 Test Stand DAQ while simulating
all modes}

            \vspace{\dp0}
            } \end{minipage} \\ \cline{2-3}
            & Test Data &
            \begin{minipage}[t]{13cm}{\footnotesize
                No data.
                \vspace{\dp0}
            } \end{minipage} \\ \cline{2-3}
            & Expected Result &
        \\ \midrule

            \multirow{3}{*}{ 2 } & Description &
            \begin{minipage}[t]{13cm}{\footnotesize
            Observe wavefront sensor data and metadata archived in the Data
Backbone.

            \vspace{\dp0}
            } \end{minipage} \\ \cline{2-3}
            & Test Data &
            \begin{minipage}[t]{13cm}{\footnotesize
                No data.
                \vspace{\dp0}
            } \end{minipage} \\ \cline{2-3}
            & Expected Result &
                \begin{minipage}[t]{13cm}{\footnotesize
                Well-formed wavefront sensor image data with appropriate associated
metadata.

                \vspace{\dp0}
                } \end{minipage}
        \\ \midrule
    \end{longtable}

\subsection{LVV-T31 - Verify implementation of Crosstalk Corrected Science Image Data
Acquisition}\label{lvv-t31}

\begin{longtable}[]{llllll}
\toprule
Version & Status & Priority & Verification Type & Owner
\\\midrule
1 & Draft & Normal &
Test & Kian-Tat Lim
\\\bottomrule
\multicolumn{6}{c}{ Open \href{https://jira.lsstcorp.org/secure/Tests.jspa\#/testCase/LVV-T31}{LVV-T31} in Jira } \\
\end{longtable}

\subsubsection{Verification Elements}
\begin{itemize}
\item \href{https://jira.lsstcorp.org/browse/LVV-10}{LVV-10} - DMS-REQ-0022-V-01: Crosstalk Corrected Science Image Data Acquisition

\end{itemize}

\subsubsection{Test Items}
Verify successful ingestion of crosstalk corrected data from L1 Test
Stand DAQ while simulating all modes.


\subsubsection{Predecessors}

\subsubsection{Environment Needs}

\paragraph{Software}

\paragraph{Hardware}

\subsubsection{Input Specification}

\subsubsection{Output Specification}

\subsubsection{Test Procedure}
    \begin{longtable}[]{p{1.3cm}p{2cm}p{13cm}}
    %\toprule
    Step & \multicolumn{2}{@{}l}{Description, Input Data and Expected Result} \\ \toprule
    \endhead

            \multirow{3}{*}{ 1 } & Description &
            \begin{minipage}[t]{13cm}{\footnotesize
            Inject signals of different relative strength

            \vspace{\dp0}
            } \end{minipage} \\ \cline{2-3}
            & Test Data &
            \begin{minipage}[t]{13cm}{\footnotesize
                No data.
                \vspace{\dp0}
            } \end{minipage} \\ \cline{2-3}
            & Expected Result &
        \\ \midrule

            \multirow{3}{*}{ 2 } & Description &
            \begin{minipage}[t]{13cm}{\footnotesize
            Apply Camera cross-talk correction

            \vspace{\dp0}
            } \end{minipage} \\ \cline{2-3}
            & Test Data &
            \begin{minipage}[t]{13cm}{\footnotesize
                No data.
                \vspace{\dp0}
            } \end{minipage} \\ \cline{2-3}
            & Expected Result &
        \\ \midrule

            \multirow{3}{*}{ 3 } & Description &
            \begin{minipage}[t]{13cm}{\footnotesize
            Verify that DMS sytem can import the cross-talk corrected images

            \vspace{\dp0}
            } \end{minipage} \\ \cline{2-3}
            & Test Data &
            \begin{minipage}[t]{13cm}{\footnotesize
                No data.
                \vspace{\dp0}
            } \end{minipage} \\ \cline{2-3}
            & Expected Result &
        \\ \midrule

            \multirow{3}{*}{ 4 } & Description &
            \begin{minipage}[t]{13cm}{\footnotesize
            Verify that images are corrected for crosstalk

            \vspace{\dp0}
            } \end{minipage} \\ \cline{2-3}
            & Test Data &
            \begin{minipage}[t]{13cm}{\footnotesize
                No data.
                \vspace{\dp0}
            } \end{minipage} \\ \cline{2-3}
            & Expected Result &
        \\ \midrule
    \end{longtable}

\subsection{LVV-T32 - Verify implementation of Raw Image Assembly}\label{lvv-t32}

\begin{longtable}[]{llllll}
\toprule
Version & Status & Priority & Verification Type & Owner
\\\midrule
1 & Defined & Normal &
Test & Kian-Tat Lim
\\\bottomrule
\multicolumn{6}{c}{ Open \href{https://jira.lsstcorp.org/secure/Tests.jspa\#/testCase/LVV-T32}{LVV-T32} in Jira } \\
\end{longtable}

\subsubsection{Verification Elements}
\begin{itemize}
\item \href{https://jira.lsstcorp.org/browse/LVV-11}{LVV-11} - DMS-REQ-0024-V-01: Raw Image Assembly

\end{itemize}

\subsubsection{Test Items}
Verify that the raw exposure data from all readout channels in a sensor
can be assembled into a single image, and that all required/relevant
metadata are associated with the image data.~


\subsubsection{Predecessors}

\subsubsection{Environment Needs}

\paragraph{Software}

\paragraph{Hardware}

\subsubsection{Input Specification}

\subsubsection{Output Specification}

\subsubsection{Test Procedure}
    \begin{longtable}[]{p{1.3cm}p{2cm}p{13cm}}
    %\toprule
    Step & \multicolumn{2}{@{}l}{Description, Input Data and Expected Result} \\ \toprule
    \endhead

            \multirow{3}{*}{ 1 } & Description &
            \begin{minipage}[t]{13cm}{\footnotesize
            Ingest data from the L1 Camera Test Stand DAQ.

            \vspace{\dp0}
            } \end{minipage} \\ \cline{2-3}
            & Test Data &
            \begin{minipage}[t]{13cm}{\footnotesize
                No data.
                \vspace{\dp0}
            } \end{minipage} \\ \cline{2-3}
            & Expected Result &
        \\ \midrule

            \multirow{3}{*}{ 2 } & Description &
            \begin{minipage}[t]{13cm}{\footnotesize
            Simulate all different modes of data gathering.

            \vspace{\dp0}
            } \end{minipage} \\ \cline{2-3}
            & Test Data &
            \begin{minipage}[t]{13cm}{\footnotesize
                No data.
                \vspace{\dp0}
            } \end{minipage} \\ \cline{2-3}
            & Expected Result &
        \\ \midrule

            \multirow{3}{*}{ 3 } & Description &
            \begin{minipage}[t]{13cm}{\footnotesize
            Verify that a raw image is constructed in correct format.

            \vspace{\dp0}
            } \end{minipage} \\ \cline{2-3}
            & Test Data &
            \begin{minipage}[t]{13cm}{\footnotesize
                No data.
                \vspace{\dp0}
            } \end{minipage} \\ \cline{2-3}
            & Expected Result &
                \begin{minipage}[t]{13cm}{\footnotesize
                A single raw image combining data from all readout channels for a given
sensor.~

                \vspace{\dp0}
                } \end{minipage}
        \\ \midrule

            \multirow{3}{*}{ 4 } & Description &
            \begin{minipage}[t]{13cm}{\footnotesize
            Verify that a raw image is constructed with correct metadata.

            \vspace{\dp0}
            } \end{minipage} \\ \cline{2-3}
            & Test Data &
            \begin{minipage}[t]{13cm}{\footnotesize
                No data.
                \vspace{\dp0}
            } \end{minipage} \\ \cline{2-3}
            & Expected Result &
                \begin{minipage}[t]{13cm}{\footnotesize
                Image header or ancillary table contains the required metadata about the
observing context in which data were gathered.

                \vspace{\dp0}
                } \end{minipage}
        \\ \midrule
    \end{longtable}

\subsection{LVV-T33 - Verify implementation of Raw Science Image Metadata}\label{lvv-t33}

\begin{longtable}[]{llllll}
\toprule
Version & Status & Priority & Verification Type & Owner
\\\midrule
1 & Defined & Normal &
Test & Kian-Tat Lim
\\\bottomrule
\multicolumn{6}{c}{ Open \href{https://jira.lsstcorp.org/secure/Tests.jspa\#/testCase/LVV-T33}{LVV-T33} in Jira } \\
\end{longtable}

\subsubsection{Verification Elements}
\begin{itemize}
\item \href{https://jira.lsstcorp.org/browse/LVV-28}{LVV-28} - DMS-REQ-0068-V-01: Raw Science Image Metadata

\item \href{https://jira.lsstcorp.org/browse/LVV-1234}{LVV-1234} - OSS-REQ-0122-V-01: Provenance

\end{itemize}

\subsubsection{Test Items}
Verify successful ingestion of raw data from L1 Test Stand DAQ and that
image metadata is present and queryable.


\subsubsection{Predecessors}
\href{https://jira.lsstcorp.org/secure/Tests.jspa\#/testCase/LVV-T29}{LVV-T29},
​\href{https://jira.lsstcorp.org/secure/Tests.jspa\#/testCase/LVV-T32}{LVV-T32}​​​

\subsubsection{Environment Needs}

\paragraph{Software}

\paragraph{Hardware}

\subsubsection{Input Specification}

\subsubsection{Output Specification}

\subsubsection{Test Procedure}
    \begin{longtable}[]{p{1.3cm}p{2cm}p{13cm}}
    %\toprule
    Step & \multicolumn{2}{@{}l}{Description, Input Data and Expected Result} \\ \toprule
    \endhead

            \multirow{3}{*}{ 1 } & Description &
            \begin{minipage}[t]{13cm}{\footnotesize
            Identify (or gather) a dataset of raw science images.

            \vspace{\dp0}
            } \end{minipage} \\ \cline{2-3}
            & Test Data &
            \begin{minipage}[t]{13cm}{\footnotesize
                No data.
                \vspace{\dp0}
            } \end{minipage} \\ \cline{2-3}
            & Expected Result &
        \\ \midrule

            \multirow{3}{*}{ 2 } & Description &
            \begin{minipage}[t]{13cm}{\footnotesize
            Verify that time of exposure start/end, site metadata, telescope
metadata, and camera metadata are stored in DMS
system.\\[2\baselineskip]

            \vspace{\dp0}
            } \end{minipage} \\ \cline{2-3}
            & Test Data &
            \begin{minipage}[t]{13cm}{\footnotesize
                No data.
                \vspace{\dp0}
            } \end{minipage} \\ \cline{2-3}
            & Expected Result &
                \begin{minipage}[t]{13cm}{\footnotesize
                Raw image data contain the required metadata.

                \vspace{\dp0}
                } \end{minipage}
        \\ \midrule
    \end{longtable}

\subsection{LVV-T34 - Verify implementation of Guider Calibration Data Acquisition}\label{lvv-t34}

\begin{longtable}[]{llllll}
\toprule
Version & Status & Priority & Verification Type & Owner
\\\midrule
1 & Defined & Normal &
Test & Kian-Tat Lim
\\\bottomrule
\multicolumn{6}{c}{ Open \href{https://jira.lsstcorp.org/secure/Tests.jspa\#/testCase/LVV-T34}{LVV-T34} in Jira } \\
\end{longtable}

\subsubsection{Verification Elements}
\begin{itemize}
\item \href{https://jira.lsstcorp.org/browse/LVV-96}{LVV-96} - DMS-REQ-0265-V-01: Guider Calibration Data Acquisition

\end{itemize}

\subsubsection{Test Items}
{Verify successful}\\
{~1. Ingestion of calibration frames from L1 Test Stand DAQ}\\
{~2. Execution of CPP payloads}\\
{~3. Availability of observed guider calibration products}


\subsubsection{Predecessors}

\subsubsection{Environment Needs}

\paragraph{Software}

\paragraph{Hardware}

\subsubsection{Input Specification}

\subsubsection{Output Specification}

\subsubsection{Test Procedure}
    \begin{longtable}[]{p{1.3cm}p{2cm}p{13cm}}
    %\toprule
    Step & \multicolumn{2}{@{}l}{Description, Input Data and Expected Result} \\ \toprule
    \endhead

            \multirow{3}{*}{ 1 } & Description &
            \begin{minipage}[t]{13cm}{\footnotesize
            {Ingest calibration frames for the guider sensors from L1 Test Stand
DAQ}

            \vspace{\dp0}
            } \end{minipage} \\ \cline{2-3}
            & Test Data &
            \begin{minipage}[t]{13cm}{\footnotesize
                No data.
                \vspace{\dp0}
            } \end{minipage} \\ \cline{2-3}
            & Expected Result &
        \\ \midrule

                \multirow{3}{*}{\parbox{1.3cm}{ 2-1
                {\scriptsize from \hyperref[lvv-t1060]
                {LVV-T1060} } } }

                & {\small Description} &
                \begin{minipage}[t]{13cm}{\scriptsize
                Execute the Calibration Products Production payload. The payload uses
raw calibration images and information from the Transformed EFD to
generate a subset of Master Calibration Images and Calibration Database
entries in the Data Backbone.

                \vspace{\dp0}
                } \end{minipage} \\ \cdashline{2-3}
                & {\small Test Data} &
                \begin{minipage}[t]{13cm}{\scriptsize
                } \end{minipage} \\ \cdashline{2-3}
                & {\small Expected Result} &
                \\ \hdashline


                \multirow{3}{*}{\parbox{1.3cm}{ 2-2
                {\scriptsize from \hyperref[lvv-t1060]
                {LVV-T1060} } } }

                & {\small Description} &
                \begin{minipage}[t]{13cm}{\scriptsize
                Confirm that the expected Master Calibration images and Calibration
Database entries are present and well-formed.

                \vspace{\dp0}
                } \end{minipage} \\ \cdashline{2-3}
                & {\small Test Data} &
                \begin{minipage}[t]{13cm}{\scriptsize
                } \end{minipage} \\ \cdashline{2-3}
                & {\small Expected Result} &
                \\ \hdashline


        \\ \midrule

            \multirow{3}{*}{ 3 } & Description &
            \begin{minipage}[t]{13cm}{\footnotesize
            Observe that guider calibration products have been produced.

            \vspace{\dp0}
            } \end{minipage} \\ \cline{2-3}
            & Test Data &
            \begin{minipage}[t]{13cm}{\footnotesize
                No data.
                \vspace{\dp0}
            } \end{minipage} \\ \cline{2-3}
            & Expected Result &
                \begin{minipage}[t]{13cm}{\footnotesize
                Well-formed calibration frames for the guider sensors.

                \vspace{\dp0}
                } \end{minipage}
        \\ \midrule
    \end{longtable}

\subsection{LVV-T35 - Verify implementation of Nightly Data Accessible Within 24 hrs}\label{lvv-t35}

\begin{longtable}[]{llllll}
\toprule
Version & Status & Priority & Verification Type & Owner
\\\midrule
1 & Draft & Normal &
Test & Eric Bellm
\\\bottomrule
\multicolumn{6}{c}{ Open \href{https://jira.lsstcorp.org/secure/Tests.jspa\#/testCase/LVV-T35}{LVV-T35} in Jira } \\
\end{longtable}

\subsubsection{Verification Elements}
\begin{itemize}
\item \href{https://jira.lsstcorp.org/browse/LVV-175}{LVV-175} - DMS-REQ-0004-V-01: Time to L1 public release

\end{itemize}

\subsubsection{Test Items}
\textbf{Test Items}\\[2\baselineskip]Verify that\\
1. Alerts are available within OTT1\\
2. Level 1 Data Products are available within L1PublicT\\
3. Solar System Object orbits are available within L1PublicT of the
updated calculations completion on the following night.


\subsubsection{Predecessors}

\subsubsection{Environment Needs}

\paragraph{Software}

\paragraph{Hardware}

\subsubsection{Input Specification}

\subsubsection{Output Specification}

\subsubsection{Test Procedure}
    \begin{longtable}[]{p{1.3cm}p{2cm}p{13cm}}
    %\toprule
    Step & \multicolumn{2}{@{}l}{Description, Input Data and Expected Result} \\ \toprule
    \endhead

                \multirow{3}{*}{\parbox{1.3cm}{ 1-1
                {\scriptsize from \hyperref[lvv-t860]
                {LVV-T860} } } }

                & {\small Description} &
                \begin{minipage}[t]{13cm}{\scriptsize
                The `path` that you will use depends on where you are running the
science pipelines. Options:\\[2\baselineskip]

\begin{itemize}
\tightlist
\item
  local (newinstall.sh - based
  install):{[}path\_to\_installation{]}/loadLSST.bash
\item
  development cluster (``lsst-dev''):
  /software/lsstsw/stack/loadLSST.bash
\item
  LSP Notebook aspect (from a terminal):
  /opt/lsst/software/stack/loadLSST.bash
\end{itemize}

From the command line, execute the commands below in the example
code:\\[2\baselineskip]

                \vspace{\dp0}
                } \end{minipage} \\ \cdashline{2-3}
                & {\small Test Data} &
                \begin{minipage}[t]{13cm}{\scriptsize
                } \end{minipage} \\ \cdashline{2-3}
                & {\small Expected Result} &
                    \begin{minipage}[t]{13cm}{\scriptsize
                    Science pipeline software is available for use. If additional packages
are needed (for example, `obs' packages such as `obs\_subaru`), then
additional `setup` commands will be necessary.\\[2\baselineskip]To check
versions in use, type:\\
eups list -s

                    \vspace{\dp0}
                    } \end{minipage}
                \\ \hdashline


        \\ \midrule

                \multirow{3}{*}{\parbox{1.3cm}{ 2-1
                {\scriptsize from \hyperref[lvv-t866]
                {LVV-T866} } } }

                & {\small Description} &
                \begin{minipage}[t]{13cm}{\scriptsize
                Perform the steps of Alert Production (including, but not necessarily
limited to, single frame processing, ISR, source detection/measurement,
PSF estimation, photometric and astrometric calibration, difference
imaging, DIASource detection/measurement, source association). During
Operations, it is presumed that these are automated for a given
dataset.~

                \vspace{\dp0}
                } \end{minipage} \\ \cdashline{2-3}
                & {\small Test Data} &
                \begin{minipage}[t]{13cm}{\scriptsize
                } \end{minipage} \\ \cdashline{2-3}
                & {\small Expected Result} &
                    \begin{minipage}[t]{13cm}{\scriptsize
                    An output dataset including difference images and DIASource and
DIAObject measurements.

                    \vspace{\dp0}
                    } \end{minipage}
                \\ \hdashline


                \multirow{3}{*}{\parbox{1.3cm}{ 2-2
                {\scriptsize from \hyperref[lvv-t866]
                {LVV-T866} } } }

                & {\small Description} &
                \begin{minipage}[t]{13cm}{\scriptsize
                Verify that the expected data products have been produced, and that
catalogs contain reasonable values for measured quantities of interest.

                \vspace{\dp0}
                } \end{minipage} \\ \cdashline{2-3}
                & {\small Test Data} &
                \begin{minipage}[t]{13cm}{\scriptsize
                } \end{minipage} \\ \cdashline{2-3}
                & {\small Expected Result} &
                \\ \hdashline


        \\ \midrule

            \multirow{3}{*}{ 3 } & Description &
            \begin{minipage}[t]{13cm}{\footnotesize
            Time processing of data starting from (pre-ingested) raw files until an
alert is available for distribution; verify that this time is less than
OTT1.

            \vspace{\dp0}
            } \end{minipage} \\ \cline{2-3}
            & Test Data &
            \begin{minipage}[t]{13cm}{\footnotesize
                No data.
                \vspace{\dp0}
            } \end{minipage} \\ \cline{2-3}
            & Expected Result &
        \\ \midrule

            \multirow{3}{*}{ 4 } & Description &
            \begin{minipage}[t]{13cm}{\footnotesize
            Time processing of data starting from (pre-ingested) raw files until the
required data products are available in the Science Platform. Verify
that this time is less than L1PublicT.

            \vspace{\dp0}
            } \end{minipage} \\ \cline{2-3}
            & Test Data &
            \begin{minipage}[t]{13cm}{\footnotesize
                No data.
                \vspace{\dp0}
            } \end{minipage} \\ \cline{2-3}
            & Expected Result &
        \\ \midrule

            \multirow{3}{*}{ 5 } & Description &
            \begin{minipage}[t]{13cm}{\footnotesize
            Run MOPS on 1 night equivalent of LSST observing worth of precursor data
and verify that Solar System Object orbits can be updated within 24
hours.

            \vspace{\dp0}
            } \end{minipage} \\ \cline{2-3}
            & Test Data &
            \begin{minipage}[t]{13cm}{\footnotesize
                No data.
                \vspace{\dp0}
            } \end{minipage} \\ \cline{2-3}
            & Expected Result &
        \\ \midrule

            \multirow{3}{*}{ 6 } & Description &
            \begin{minipage}[t]{13cm}{\footnotesize
            Record time between completion of MOPS processing and availability of
the updated SSObject catalogue through the Science Platform; verify this
time is less than L1PublicT.

            \vspace{\dp0}
            } \end{minipage} \\ \cline{2-3}
            & Test Data &
            \begin{minipage}[t]{13cm}{\footnotesize
                No data.
                \vspace{\dp0}
            } \end{minipage} \\ \cline{2-3}
            & Expected Result &
        \\ \midrule
    \end{longtable}

\subsection{LVV-T36 - Verify implementation of Difference Exposures}\label{lvv-t36}

\begin{longtable}[]{llllll}
\toprule
Version & Status & Priority & Verification Type & Owner
\\\midrule
1 & Draft & Normal &
Test & Eric Bellm
\\\bottomrule
\multicolumn{6}{c}{ Open \href{https://jira.lsstcorp.org/secure/Tests.jspa\#/testCase/LVV-T36}{LVV-T36} in Jira } \\
\end{longtable}

\subsubsection{Verification Elements}
\begin{itemize}
\item \href{https://jira.lsstcorp.org/browse/LVV-7}{LVV-7} - DMS-REQ-0010-V-01: Difference Exposures

\end{itemize}

\subsubsection{Test Items}
Verify successful creation of a\\
1. PSF-matched template image for a given Processed Visit Image\\
2. Difference Exposure from each Processed Visit Image


\subsubsection{Predecessors}

\subsubsection{Environment Needs}

\paragraph{Software}

\paragraph{Hardware}

\subsubsection{Input Specification}

\subsubsection{Output Specification}

\subsubsection{Test Procedure}
    \begin{longtable}[]{p{1.3cm}p{2cm}p{13cm}}
    %\toprule
    Step & \multicolumn{2}{@{}l}{Description, Input Data and Expected Result} \\ \toprule
    \endhead

                \multirow{3}{*}{\parbox{1.3cm}{ 1-1
                {\scriptsize from \hyperref[lvv-t860]
                {LVV-T860} } } }

                & {\small Description} &
                \begin{minipage}[t]{13cm}{\scriptsize
                The `path` that you will use depends on where you are running the
science pipelines. Options:\\[2\baselineskip]

\begin{itemize}
\tightlist
\item
  local (newinstall.sh - based
  install):{[}path\_to\_installation{]}/loadLSST.bash
\item
  development cluster (``lsst-dev''):
  /software/lsstsw/stack/loadLSST.bash
\item
  LSP Notebook aspect (from a terminal):
  /opt/lsst/software/stack/loadLSST.bash
\end{itemize}

From the command line, execute the commands below in the example
code:\\[2\baselineskip]

                \vspace{\dp0}
                } \end{minipage} \\ \cdashline{2-3}
                & {\small Test Data} &
                \begin{minipage}[t]{13cm}{\scriptsize
                } \end{minipage} \\ \cdashline{2-3}
                & {\small Expected Result} &
                    \begin{minipage}[t]{13cm}{\scriptsize
                    Science pipeline software is available for use. If additional packages
are needed (for example, `obs' packages such as `obs\_subaru`), then
additional `setup` commands will be necessary.\\[2\baselineskip]To check
versions in use, type:\\
eups list -s

                    \vspace{\dp0}
                    } \end{minipage}
                \\ \hdashline


        \\ \midrule

                \multirow{3}{*}{\parbox{1.3cm}{ 2-1
                {\scriptsize from \hyperref[lvv-t866]
                {LVV-T866} } } }

                & {\small Description} &
                \begin{minipage}[t]{13cm}{\scriptsize
                Perform the steps of Alert Production (including, but not necessarily
limited to, single frame processing, ISR, source detection/measurement,
PSF estimation, photometric and astrometric calibration, difference
imaging, DIASource detection/measurement, source association). During
Operations, it is presumed that these are automated for a given
dataset.~

                \vspace{\dp0}
                } \end{minipage} \\ \cdashline{2-3}
                & {\small Test Data} &
                \begin{minipage}[t]{13cm}{\scriptsize
                } \end{minipage} \\ \cdashline{2-3}
                & {\small Expected Result} &
                    \begin{minipage}[t]{13cm}{\scriptsize
                    An output dataset including difference images and DIASource and
DIAObject measurements.

                    \vspace{\dp0}
                    } \end{minipage}
                \\ \hdashline


                \multirow{3}{*}{\parbox{1.3cm}{ 2-2
                {\scriptsize from \hyperref[lvv-t866]
                {LVV-T866} } } }

                & {\small Description} &
                \begin{minipage}[t]{13cm}{\scriptsize
                Verify that the expected data products have been produced, and that
catalogs contain reasonable values for measured quantities of interest.

                \vspace{\dp0}
                } \end{minipage} \\ \cdashline{2-3}
                & {\small Test Data} &
                \begin{minipage}[t]{13cm}{\scriptsize
                } \end{minipage} \\ \cdashline{2-3}
                & {\small Expected Result} &
                \\ \hdashline


        \\ \midrule

            \multirow{3}{*}{ 3 } & Description &
            \begin{minipage}[t]{13cm}{\footnotesize
            Demonstrate successful creation of a template image from HSC PDF and
DECAM HiTS data. ~Demonstrate successful creation of a Difference
Exposure for at least 10 other images from survey, ideally at a range of
arimass. ~In particular, HiTS has 2013A u-band data. ~While the Blanco
4-m does have an ADC, there are still some chromatic effects and we
should demonstrate that we can successfully produce Difference Exposures
and templates for diferent airmass bins.

            \vspace{\dp0}
            } \end{minipage} \\ \cline{2-3}
            & Test Data &
            \begin{minipage}[t]{13cm}{\footnotesize
                No data.
                \vspace{\dp0}
            } \end{minipage} \\ \cline{2-3}
            & Expected Result &
        \\ \midrule
    \end{longtable}

\subsection{LVV-T37 - Verify implementation of Difference Exposure Attributes}\label{lvv-t37}

\begin{longtable}[]{llllll}
\toprule
Version & Status & Priority & Verification Type & Owner
\\\midrule
1 & Draft & Normal &
Test & Eric Bellm
\\\bottomrule
\multicolumn{6}{c}{ Open \href{https://jira.lsstcorp.org/secure/Tests.jspa\#/testCase/LVV-T37}{LVV-T37} in Jira } \\
\end{longtable}

\subsubsection{Verification Elements}
\begin{itemize}
\item \href{https://jira.lsstcorp.org/browse/LVV-32}{LVV-32} - DMS-REQ-0074-V-01: Difference Exposure Attributes

\item \href{https://jira.lsstcorp.org/browse/LVV-1234}{LVV-1234} - OSS-REQ-0122-V-01: Provenance

\end{itemize}

\subsubsection{Test Items}
Verify that for each Difference Exposure the DMS stores\\
1. The identify of the input exposures and related provenance
information\\
2. Metadata attributes of the subtraction, including the PSF-matching
kernel used.


\subsubsection{Predecessors}

\subsubsection{Environment Needs}

\paragraph{Software}

\paragraph{Hardware}

\subsubsection{Input Specification}

\subsubsection{Output Specification}

\subsubsection{Test Procedure}
    \begin{longtable}[]{p{1.3cm}p{2cm}p{13cm}}
    %\toprule
    Step & \multicolumn{2}{@{}l}{Description, Input Data and Expected Result} \\ \toprule
    \endhead

                \multirow{3}{*}{\parbox{1.3cm}{ 1-1
                {\scriptsize from \hyperref[lvv-t860]
                {LVV-T860} } } }

                & {\small Description} &
                \begin{minipage}[t]{13cm}{\scriptsize
                The `path` that you will use depends on where you are running the
science pipelines. Options:\\[2\baselineskip]

\begin{itemize}
\tightlist
\item
  local (newinstall.sh - based
  install):{[}path\_to\_installation{]}/loadLSST.bash
\item
  development cluster (``lsst-dev''):
  /software/lsstsw/stack/loadLSST.bash
\item
  LSP Notebook aspect (from a terminal):
  /opt/lsst/software/stack/loadLSST.bash
\end{itemize}

From the command line, execute the commands below in the example
code:\\[2\baselineskip]

                \vspace{\dp0}
                } \end{minipage} \\ \cdashline{2-3}
                & {\small Test Data} &
                \begin{minipage}[t]{13cm}{\scriptsize
                } \end{minipage} \\ \cdashline{2-3}
                & {\small Expected Result} &
                    \begin{minipage}[t]{13cm}{\scriptsize
                    Science pipeline software is available for use. If additional packages
are needed (for example, `obs' packages such as `obs\_subaru`), then
additional `setup` commands will be necessary.\\[2\baselineskip]To check
versions in use, type:\\
eups list -s

                    \vspace{\dp0}
                    } \end{minipage}
                \\ \hdashline


        \\ \midrule

                \multirow{3}{*}{\parbox{1.3cm}{ 2-1
                {\scriptsize from \hyperref[lvv-t866]
                {LVV-T866} } } }

                & {\small Description} &
                \begin{minipage}[t]{13cm}{\scriptsize
                Perform the steps of Alert Production (including, but not necessarily
limited to, single frame processing, ISR, source detection/measurement,
PSF estimation, photometric and astrometric calibration, difference
imaging, DIASource detection/measurement, source association). During
Operations, it is presumed that these are automated for a given
dataset.~

                \vspace{\dp0}
                } \end{minipage} \\ \cdashline{2-3}
                & {\small Test Data} &
                \begin{minipage}[t]{13cm}{\scriptsize
                } \end{minipage} \\ \cdashline{2-3}
                & {\small Expected Result} &
                    \begin{minipage}[t]{13cm}{\scriptsize
                    An output dataset including difference images and DIASource and
DIAObject measurements.

                    \vspace{\dp0}
                    } \end{minipage}
                \\ \hdashline


                \multirow{3}{*}{\parbox{1.3cm}{ 2-2
                {\scriptsize from \hyperref[lvv-t866]
                {LVV-T866} } } }

                & {\small Description} &
                \begin{minipage}[t]{13cm}{\scriptsize
                Verify that the expected data products have been produced, and that
catalogs contain reasonable values for measured quantities of interest.

                \vspace{\dp0}
                } \end{minipage} \\ \cdashline{2-3}
                & {\small Test Data} &
                \begin{minipage}[t]{13cm}{\scriptsize
                } \end{minipage} \\ \cdashline{2-3}
                & {\small Expected Result} &
                \\ \hdashline


        \\ \midrule

            \multirow{3}{*}{ 3 } & Description &
            \begin{minipage}[t]{13cm}{\footnotesize
            For each of HSC PDR and DECAM HiTS data: set up three different
templates and run subtractions on 10 different images from at least two
different filters. ~Verify that we can recover the provenance
information about which template was used for each subtraction, which
input images were used for that template, and that we can successfull
extract the PSF matching kernel.

            \vspace{\dp0}
            } \end{minipage} \\ \cline{2-3}
            & Test Data &
            \begin{minipage}[t]{13cm}{\footnotesize
                No data.
                \vspace{\dp0}
            } \end{minipage} \\ \cline{2-3}
            & Expected Result &
        \\ \midrule
    \end{longtable}

\subsection{LVV-T38 - Verify implementation of Processed Visit Images}\label{lvv-t38}

\begin{longtable}[]{llllll}
\toprule
Version & Status & Priority & Verification Type & Owner
\\\midrule
1 & Defined & Normal &
Test & Eric Bellm
\\\bottomrule
\multicolumn{6}{c}{ Open \href{https://jira.lsstcorp.org/secure/Tests.jspa\#/testCase/LVV-T38}{LVV-T38} in Jira } \\
\end{longtable}

\subsubsection{Verification Elements}
\begin{itemize}
\item \href{https://jira.lsstcorp.org/browse/LVV-29}{LVV-29} - DMS-REQ-0069-V-01: Processed Visit Images

\end{itemize}

\subsubsection{Test Items}
Verify that the DMS\\
1. Successfully produces Processed Visit Images, where the instrument
signature has been removed.\\
2. Successfully combines images obtained during a standard visit.


\subsubsection{Predecessors}

\subsubsection{Environment Needs}

\paragraph{Software}

\paragraph{Hardware}

\subsubsection{Input Specification}

\subsubsection{Output Specification}

\subsubsection{Test Procedure}
    \begin{longtable}[]{p{1.3cm}p{2cm}p{13cm}}
    %\toprule
    Step & \multicolumn{2}{@{}l}{Description, Input Data and Expected Result} \\ \toprule
    \endhead

            \multirow{3}{*}{ 1 } & Description &
            \begin{minipage}[t]{13cm}{\footnotesize
            Identify suitable precursor datasets containing unprocessed raw images.

            \vspace{\dp0}
            } \end{minipage} \\ \cline{2-3}
            & Test Data &
            \begin{minipage}[t]{13cm}{\footnotesize
                No data.
                \vspace{\dp0}
            } \end{minipage} \\ \cline{2-3}
            & Expected Result &
        \\ \midrule

            \multirow{3}{*}{ 2 } & Description &
            \begin{minipage}[t]{13cm}{\footnotesize
            Run the Prompt Processing payload on these data. ~Verify that Processed
Visit Images are generated at correct size and with significant
instrumental artifacts removed.

            \vspace{\dp0}
            } \end{minipage} \\ \cline{2-3}
            & Test Data &
            \begin{minipage}[t]{13cm}{\footnotesize
                No data.
                \vspace{\dp0}
            } \end{minipage} \\ \cline{2-3}
            & Expected Result &
                \begin{minipage}[t]{13cm}{\footnotesize
                Raw precursor dataset images have been processed into Processed Visit
Images, with instrumental artifacts corrected.

                \vspace{\dp0}
                } \end{minipage}
        \\ \midrule

            \multirow{3}{*}{ 3 } & Description &
            \begin{minipage}[t]{13cm}{\footnotesize
            Run camera test stand data through full acquisition+backbone+ISR.

            \vspace{\dp0}
            } \end{minipage} \\ \cline{2-3}
            & Test Data &
            \begin{minipage}[t]{13cm}{\footnotesize
                No data.
                \vspace{\dp0}
            } \end{minipage} \\ \cline{2-3}
            & Expected Result &
        \\ \midrule

            \multirow{3}{*}{ 4 } & Description &
            \begin{minipage}[t]{13cm}{\footnotesize
            Run simulated LSST data with calibrations through prompt processing
system and inspect Processed Visit images to verify that they have been
cleaned of significant artifacts and are of the correct, shape, and
described orientation.

            \vspace{\dp0}
            } \end{minipage} \\ \cline{2-3}
            & Test Data &
            \begin{minipage}[t]{13cm}{\footnotesize
                No data.
                \vspace{\dp0}
            } \end{minipage} \\ \cline{2-3}
            & Expected Result &
                \begin{minipage}[t]{13cm}{\footnotesize
                Raw images have been processed into Processed Visit Images, with
instrumental artifacts corrected.

                \vspace{\dp0}
                } \end{minipage}
        \\ \midrule
    \end{longtable}

\subsection{LVV-T39 - Verify implementation of Generate Photometric Zeropoint for Visit Image}\label{lvv-t39}

\begin{longtable}[]{llllll}
\toprule
Version & Status & Priority & Verification Type & Owner
\\\midrule
1 & Approved & Normal &
Test & Jim Bosch
\\\bottomrule
\multicolumn{6}{c}{ Open \href{https://jira.lsstcorp.org/secure/Tests.jspa\#/testCase/LVV-T39}{LVV-T39} in Jira } \\
\end{longtable}

\subsubsection{Verification Elements}
\begin{itemize}
\item \href{https://jira.lsstcorp.org/browse/LVV-12}{LVV-12} - DMS-REQ-0029-V-01: Generate Photometric Zeropoint for Visit Image

\end{itemize}

\subsubsection{Test Items}
Verify that Processed Visit Image data products produced by the DRP and
AP pipelines include the parameters of a model that relates the observed
flux on the image to physical flux units.


\subsubsection{Predecessors}

\subsubsection{Environment Needs}

\paragraph{Software}

\paragraph{Hardware}

\subsubsection{Input Specification}

\subsubsection{Output Specification}

\subsubsection{Test Procedure}
    \begin{longtable}[]{p{1.3cm}p{2cm}p{13cm}}
    %\toprule
    Step & \multicolumn{2}{@{}l}{Description, Input Data and Expected Result} \\ \toprule
    \endhead

            \multirow{3}{*}{ 1 } & Description &
            \begin{minipage}[t]{13cm}{\footnotesize
            Identify a dataset with processed visit images in multiple filters.

            \vspace{\dp0}
            } \end{minipage} \\ \cline{2-3}
            & Test Data &
            \begin{minipage}[t]{13cm}{\footnotesize
                No data.
                \vspace{\dp0}
            } \end{minipage} \\ \cline{2-3}
            & Expected Result &
        \\ \midrule

                \multirow{3}{*}{\parbox{1.3cm}{ 2-1
                {\scriptsize from \hyperref[lvv-t987]
                {LVV-T987} } } }

                & {\small Description} &
                \begin{minipage}[t]{13cm}{\scriptsize
                Identify the path to the data repository, which we will refer to as
`DATA/path', then execute the following:

                \vspace{\dp0}
                } \end{minipage} \\ \cdashline{2-3}
                & {\small Test Data} &
                \begin{minipage}[t]{13cm}{\scriptsize
                } \end{minipage} \\ \cdashline{2-3}
                & {\small Expected Result} &
                    \begin{minipage}[t]{13cm}{\scriptsize
                    Butler repo available for reading.

                    \vspace{\dp0}
                    } \end{minipage}
                \\ \hdashline


        \\ \midrule

            \multirow{3}{*}{ 3 } & Description &
            \begin{minipage}[t]{13cm}{\footnotesize
            Extract the photometric zeropoint from the source catalog associated
with a visit image. Repeat this for all available filters, and confirm
that the zeropoint has been set, and has a reasonable value.

            \vspace{\dp0}
            } \end{minipage} \\ \cline{2-3}
            & Test Data &
            \begin{minipage}[t]{13cm}{\footnotesize
                No data.
                \vspace{\dp0}
            } \end{minipage} \\ \cline{2-3}
            & Expected Result &
                \begin{minipage}[t]{13cm}{\footnotesize
                A zeropoint that enables one to convert the measured fluxes to
magnitudes.

                \vspace{\dp0}
                } \end{minipage}
        \\ \midrule

            \multirow{3}{*}{ 4 } & Description &
            \begin{minipage}[t]{13cm}{\footnotesize
            Extract fluxes for some sources, and convert them to magnitudes. Confirm
that the distribution spans a reasonable range.

            \vspace{\dp0}
            } \end{minipage} \\ \cline{2-3}
            & Test Data &
            \begin{minipage}[t]{13cm}{\footnotesize
                No data.
                \vspace{\dp0}
            } \end{minipage} \\ \cline{2-3}
            & Expected Result &
                \begin{minipage}[t]{13cm}{\footnotesize
                In most cases, well-measured magnitudes (i.e., for high S/N
measurements) should be between 12 to 28 for all bands.

                \vspace{\dp0}
                } \end{minipage}
        \\ \midrule
    \end{longtable}

\subsection{LVV-T40 - Verify implementation of Generate WCS for Visit Images}\label{lvv-t40}

\begin{longtable}[]{llllll}
\toprule
Version & Status & Priority & Verification Type & Owner
\\\midrule
1 & Approved & Normal &
Test & Jim Bosch
\\\bottomrule
\multicolumn{6}{c}{ Open \href{https://jira.lsstcorp.org/secure/Tests.jspa\#/testCase/LVV-T40}{LVV-T40} in Jira } \\
\end{longtable}

\subsubsection{Verification Elements}
\begin{itemize}
\item \href{https://jira.lsstcorp.org/browse/LVV-13}{LVV-13} - DMS-REQ-0030-V-01: Absolute accuracy of WCS

\end{itemize}

\subsubsection{Test Items}
Verify that Processed Visit Images produced by the AP and DRP pipelines
include FITS WCS accurate to specified \textbf{astrometricAccuracy} over
the bounds of the image.


\subsubsection{Predecessors}

\subsubsection{Environment Needs}

\paragraph{Software}

\paragraph{Hardware}

\subsubsection{Input Specification}

\subsubsection{Output Specification}

\subsubsection{Test Procedure}
    \begin{longtable}[]{p{1.3cm}p{2cm}p{13cm}}
    %\toprule
    Step & \multicolumn{2}{@{}l}{Description, Input Data and Expected Result} \\ \toprule
    \endhead

            \multirow{3}{*}{ 1 } & Description &
            \begin{minipage}[t]{13cm}{\footnotesize
            Identify an appropriate processed dataset for this test.

            \vspace{\dp0}
            } \end{minipage} \\ \cline{2-3}
            & Test Data &
            \begin{minipage}[t]{13cm}{\footnotesize
                No data.
                \vspace{\dp0}
            } \end{minipage} \\ \cline{2-3}
            & Expected Result &
                \begin{minipage}[t]{13cm}{\footnotesize
                A dataset with Processed Visit Images available.

                \vspace{\dp0}
                } \end{minipage}
        \\ \midrule

                \multirow{3}{*}{\parbox{1.3cm}{ 2-1
                {\scriptsize from \hyperref[lvv-t987]
                {LVV-T987} } } }

                & {\small Description} &
                \begin{minipage}[t]{13cm}{\scriptsize
                Identify the path to the data repository, which we will refer to as
`DATA/path', then execute the following:

                \vspace{\dp0}
                } \end{minipage} \\ \cdashline{2-3}
                & {\small Test Data} &
                \begin{minipage}[t]{13cm}{\scriptsize
                } \end{minipage} \\ \cdashline{2-3}
                & {\small Expected Result} &
                    \begin{minipage}[t]{13cm}{\scriptsize
                    Butler repo available for reading.

                    \vspace{\dp0}
                    } \end{minipage}
                \\ \hdashline


        \\ \midrule

            \multirow{3}{*}{ 3 } & Description &
            \begin{minipage}[t]{13cm}{\footnotesize
            Select a single visit from the dataset, and extract its WCS object and
the source list.

            \vspace{\dp0}
            } \end{minipage} \\ \cline{2-3}
            & Test Data &
            \begin{minipage}[t]{13cm}{\footnotesize
                No data.
                \vspace{\dp0}
            } \end{minipage} \\ \cline{2-3}
            & Expected Result &
                \begin{minipage}[t]{13cm}{\footnotesize
                A table containing detected sources, and a WCS object associated with
that catalog.

                \vspace{\dp0}
                } \end{minipage}
        \\ \midrule

            \multirow{3}{*}{ 4 } & Description &
            \begin{minipage}[t]{13cm}{\footnotesize
            Confirm that each CCD within the visit image contains at
least~\textbf{astrometricMinStandards~}astrometric standards that were
used in deriving the astrometric solution.

            \vspace{\dp0}
            } \end{minipage} \\ \cline{2-3}
            & Test Data &
            \begin{minipage}[t]{13cm}{\footnotesize
                No data.
                \vspace{\dp0}
            } \end{minipage} \\ \cline{2-3}
            & Expected Result &
                \begin{minipage}[t]{13cm}{\footnotesize
                At least \textbf{astrometricMinStandards} from each CCD\textbf{~}were
used in determining the WCS solution.

                \vspace{\dp0}
                } \end{minipage}
        \\ \midrule

            \multirow{3}{*}{ 5 } & Description &
            \begin{minipage}[t]{13cm}{\footnotesize
            Starting from the XY pixel coordinates of the sources, apply the WCS to
obtain RA, Dec coordinates.\\[2\baselineskip]

            \vspace{\dp0}
            } \end{minipage} \\ \cline{2-3}
            & Test Data &
            \begin{minipage}[t]{13cm}{\footnotesize
                No data.
                \vspace{\dp0}
            } \end{minipage} \\ \cline{2-3}
            & Expected Result &
                \begin{minipage}[t]{13cm}{\footnotesize
                A list of RA, Dec coordinates for all sources in the catalog.

                \vspace{\dp0}
                } \end{minipage}
        \\ \midrule

            \multirow{3}{*}{ 6 } & Description &
            \begin{minipage}[t]{13cm}{\footnotesize
            We will assume that Gaia provides a source of ``truth.'' Match the
source list to Gaia DR2, and calculate the positional offset between the
test data and the Gaia catalog.

            \vspace{\dp0}
            } \end{minipage} \\ \cline{2-3}
            & Test Data &
            \begin{minipage}[t]{13cm}{\footnotesize
                No data.
                \vspace{\dp0}
            } \end{minipage} \\ \cline{2-3}
            & Expected Result &
                \begin{minipage}[t]{13cm}{\footnotesize
                A matched catalog of sources in common between the test source list and
Gaia DR2.

                \vspace{\dp0}
                } \end{minipage}
        \\ \midrule

            \multirow{3}{*}{ 7 } & Description &
            \begin{minipage}[t]{13cm}{\footnotesize
            Apply appropriate cuts to extract the optimal dataset for comparison,
then calculate statistics (median, 1-sigma range, etc.; also plot a
histogram) of the offsets in milliarcseconds. Confirm that the offset is
less than \textbf{astrometricAccuracy}.

            \vspace{\dp0}
            } \end{minipage} \\ \cline{2-3}
            & Test Data &
            \begin{minipage}[t]{13cm}{\footnotesize
                No data.
                \vspace{\dp0}
            } \end{minipage} \\ \cline{2-3}
            & Expected Result &
                \begin{minipage}[t]{13cm}{\footnotesize
                Histogram and relevant statistics needed to confirm that the WCS
transformation is accurate.

                \vspace{\dp0}
                } \end{minipage}
        \\ \midrule

            \multirow{3}{*}{ 8 } & Description &
            \begin{minipage}[t]{13cm}{\footnotesize
            Repeat Step 5, but for subregions of the image, to confirm that the
accuracy criterion is met at all positions.

            \vspace{\dp0}
            } \end{minipage} \\ \cline{2-3}
            & Test Data &
            \begin{minipage}[t]{13cm}{\footnotesize
                No data.
                \vspace{\dp0}
            } \end{minipage} \\ \cline{2-3}
            & Expected Result &
                \begin{minipage}[t]{13cm}{\footnotesize
                \textbf{astrometricAccuracy~}requirement is met over the entire image.

                \vspace{\dp0}
                } \end{minipage}
        \\ \midrule
    \end{longtable}

\subsection{LVV-T41 - Verify implementation of Generate PSF for Visit Images}\label{lvv-t41}

\begin{longtable}[]{llllll}
\toprule
Version & Status & Priority & Verification Type & Owner
\\\midrule
1 & Approved & Normal &
Test & Jim Bosch
\\\bottomrule
\multicolumn{6}{c}{ Open \href{https://jira.lsstcorp.org/secure/Tests.jspa\#/testCase/LVV-T41}{LVV-T41} in Jira } \\
\end{longtable}

\subsubsection{Verification Elements}
\begin{itemize}
\item \href{https://jira.lsstcorp.org/browse/LVV-30}{LVV-30} - DMS-REQ-0070-V-01: Generate PSF for Visit Images

\end{itemize}

\subsubsection{Test Items}
Verify that Processed Visit Images produced by the DRP and AP pipelines
are associated with a model from which one can obtain an image of the
PSF given a point on the image.


\subsubsection{Predecessors}

\subsubsection{Environment Needs}

\paragraph{Software}

\paragraph{Hardware}

\subsubsection{Input Specification}

\subsubsection{Output Specification}

\subsubsection{Test Procedure}
    \begin{longtable}[]{p{1.3cm}p{2cm}p{13cm}}
    %\toprule
    Step & \multicolumn{2}{@{}l}{Description, Input Data and Expected Result} \\ \toprule
    \endhead

            \multirow{3}{*}{ 1 } & Description &
            \begin{minipage}[t]{13cm}{\footnotesize
            Identify a dataset with processed visit images in multiple filters.

            \vspace{\dp0}
            } \end{minipage} \\ \cline{2-3}
            & Test Data &
            \begin{minipage}[t]{13cm}{\footnotesize
                No data.
                \vspace{\dp0}
            } \end{minipage} \\ \cline{2-3}
            & Expected Result &
        \\ \midrule

                \multirow{3}{*}{\parbox{1.3cm}{ 2-1
                {\scriptsize from \hyperref[lvv-t987]
                {LVV-T987} } } }

                & {\small Description} &
                \begin{minipage}[t]{13cm}{\scriptsize
                Identify the path to the data repository, which we will refer to as
`DATA/path', then execute the following:

                \vspace{\dp0}
                } \end{minipage} \\ \cdashline{2-3}
                & {\small Test Data} &
                \begin{minipage}[t]{13cm}{\scriptsize
                } \end{minipage} \\ \cdashline{2-3}
                & {\small Expected Result} &
                    \begin{minipage}[t]{13cm}{\scriptsize
                    Butler repo available for reading.

                    \vspace{\dp0}
                    } \end{minipage}
                \\ \hdashline


        \\ \midrule

            \multirow{3}{*}{ 3 } & Description &
            \begin{minipage}[t]{13cm}{\footnotesize
            Select Objects classified as point sources on at least 10 different
processed visit images (including all bands). ~Evaluate the PSF model at
the positions of these Objects, and verify that subtracting a scaled
version of the PSF model from the processed visit image yields residuals
consistent with pure noise.

            \vspace{\dp0}
            } \end{minipage} \\ \cline{2-3}
            & Test Data &
            \begin{minipage}[t]{13cm}{\footnotesize
                No data.
                \vspace{\dp0}
            } \end{minipage} \\ \cline{2-3}
            & Expected Result &
                \begin{minipage}[t]{13cm}{\footnotesize
                Images with the PSF model subtracted, leaving only residuals that are
consistent with being noise.

                \vspace{\dp0}
                } \end{minipage}
        \\ \midrule
    \end{longtable}

\subsection{LVV-T42 - Verify implementation of Processed Visit Image Content}\label{lvv-t42}

\begin{longtable}[]{llllll}
\toprule
Version & Status & Priority & Verification Type & Owner
\\\midrule
1 & Defined & Normal &
Test & Jim Bosch
\\\bottomrule
\multicolumn{6}{c}{ Open \href{https://jira.lsstcorp.org/secure/Tests.jspa\#/testCase/LVV-T42}{LVV-T42} in Jira } \\
\end{longtable}

\subsubsection{Verification Elements}
\begin{itemize}
\item \href{https://jira.lsstcorp.org/browse/LVV-31}{LVV-31} - DMS-REQ-0072-V-01: Processed Visit Image Content

\end{itemize}

\subsubsection{Test Items}
Verify that Processed Visit Images produced by the DRP and AP pipelines
include the observed data, a mask array, a variance array, a PSF model,
and a WCS model.


\subsubsection{Predecessors}

\subsubsection{Environment Needs}

\paragraph{Software}

\paragraph{Hardware}

\subsubsection{Input Specification}

\subsubsection{Output Specification}

\subsubsection{Test Procedure}
    \begin{longtable}[]{p{1.3cm}p{2cm}p{13cm}}
    %\toprule
    Step & \multicolumn{2}{@{}l}{Description, Input Data and Expected Result} \\ \toprule
    \endhead

                \multirow{3}{*}{\parbox{1.3cm}{ 1-1
                {\scriptsize from \hyperref[lvv-t987]
                {LVV-T987} } } }

                & {\small Description} &
                \begin{minipage}[t]{13cm}{\scriptsize
                Identify the path to the data repository, which we will refer to as
`DATA/path', then execute the following:

                \vspace{\dp0}
                } \end{minipage} \\ \cdashline{2-3}
                & {\small Test Data} &
                \begin{minipage}[t]{13cm}{\scriptsize
                } \end{minipage} \\ \cdashline{2-3}
                & {\small Expected Result} &
                    \begin{minipage}[t]{13cm}{\scriptsize
                    Butler repo available for reading.

                    \vspace{\dp0}
                    } \end{minipage}
                \\ \hdashline


        \\ \midrule

            \multirow{3}{*}{ 2 } & Description &
            \begin{minipage}[t]{13cm}{\footnotesize
            Ingest the data from an appropriate processed dataset.

            \vspace{\dp0}
            } \end{minipage} \\ \cline{2-3}
            & Test Data &
            \begin{minipage}[t]{13cm}{\footnotesize
                No data.
                \vspace{\dp0}
            } \end{minipage} \\ \cline{2-3}
            & Expected Result &
        \\ \midrule

            \multirow{3}{*}{ 3 } & Description &
            \begin{minipage}[t]{13cm}{\footnotesize
            Select a single visit from the dataset, and extract its WCS object,
calexp image, psf model, and source list.

            \vspace{\dp0}
            } \end{minipage} \\ \cline{2-3}
            & Test Data &
            \begin{minipage}[t]{13cm}{\footnotesize
                No data.
                \vspace{\dp0}
            } \end{minipage} \\ \cline{2-3}
            & Expected Result &
        \\ \midrule

            \multirow{3}{*}{ 4 } & Description &
            \begin{minipage}[t]{13cm}{\footnotesize
            Inspect the calexp image to ensure that

\begin{enumerate}
\tightlist
\item
  A well-formed image is present,
\item
  The variance plane is present and well-behaved,
\item
  Mask planes are present and contain information about defects.
\end{enumerate}

            \vspace{\dp0}
            } \end{minipage} \\ \cline{2-3}
            & Test Data &
            \begin{minipage}[t]{13cm}{\footnotesize
                No data.
                \vspace{\dp0}
            } \end{minipage} \\ \cline{2-3}
            & Expected Result &
                \begin{minipage}[t]{13cm}{\footnotesize
                An astronomical image with mask and variance planes. This can be readily
visualized using Firefly, which displays mask planes by default.

                \vspace{\dp0}
                } \end{minipage}
        \\ \midrule

            \multirow{3}{*}{ 5 } & Description &
            \begin{minipage}[t]{13cm}{\footnotesize
            Plot images of the PSF model at various points, and verify that the PSF
differs with position.

            \vspace{\dp0}
            } \end{minipage} \\ \cline{2-3}
            & Test Data &
            \begin{minipage}[t]{13cm}{\footnotesize
                No data.
                \vspace{\dp0}
            } \end{minipage} \\ \cline{2-3}
            & Expected Result &
                \begin{minipage}[t]{13cm}{\footnotesize
                A ``star-like'' image of the PSF evaluated at various positions. The PSF
should vary slightly with position (this could be readily visualized by
taking a difference of PSFs at two positions).

                \vspace{\dp0}
                } \end{minipage}
        \\ \midrule

            \multirow{3}{*}{ 6 } & Description &
            \begin{minipage}[t]{13cm}{\footnotesize
            Starting from the XY pixel coordinates of the sources, apply the WCS to
obtain RA, Dec coordinates. Plot these positions and confirm that they
match the expected values from the WCS object.

            \vspace{\dp0}
            } \end{minipage} \\ \cline{2-3}
            & Test Data &
            \begin{minipage}[t]{13cm}{\footnotesize
                No data.
                \vspace{\dp0}
            } \end{minipage} \\ \cline{2-3}
            & Expected Result &
                \begin{minipage}[t]{13cm}{\footnotesize
                RA, Dec coordinates that are returned should be near the central
position of the visit coordinate as given in either the calexp metadata
or the WCS.

                \vspace{\dp0}
                } \end{minipage}
        \\ \midrule

            \multirow{3}{*}{ 7 } & Description &
            \begin{minipage}[t]{13cm}{\footnotesize
            Repeat steps 2-6, but now with difference images created by the Alert
Production pipeline (for example, in the `ap\_verify` test data
processing).

            \vspace{\dp0}
            } \end{minipage} \\ \cline{2-3}
            & Test Data &
            \begin{minipage}[t]{13cm}{\footnotesize
                No data.
                \vspace{\dp0}
            } \end{minipage} \\ \cline{2-3}
            & Expected Result &
        \\ \midrule
    \end{longtable}

\subsection{LVV-T43 - Verify implementation of Background Model Calculation}\label{lvv-t43}

\begin{longtable}[]{llllll}
\toprule
Version & Status & Priority & Verification Type & Owner
\\\midrule
1 & Approved & Normal &
Test & Jim Bosch
\\\bottomrule
\multicolumn{6}{c}{ Open \href{https://jira.lsstcorp.org/secure/Tests.jspa\#/testCase/LVV-T43}{LVV-T43} in Jira } \\
\end{longtable}

\subsubsection{Verification Elements}
\begin{itemize}
\item \href{https://jira.lsstcorp.org/browse/LVV-158}{LVV-158} - DMS-REQ-0327-V-01: Background Model Calculation

\end{itemize}

\subsubsection{Test Items}
Verify that Processed Visit Images produced by the DRP and AP pipelines
have had a model of the background subtracted, and that this model is
persisted in a way that permits the background subtracted from any CCD
to be retrieved along with the image for that CCD.


\subsubsection{Predecessors}
\href{https://jira.lsstcorp.org/secure/Tests.jspa\#/testCase/127}{LVV-T15}\\
\href{https://jira.lsstcorp.org/secure/Tests.jspa\#/testCase/131}{LVV-T19}

\subsubsection{Environment Needs}

\paragraph{Software}

\paragraph{Hardware}

\subsubsection{Input Specification}

\subsubsection{Output Specification}

\subsubsection{Test Procedure}
    \begin{longtable}[]{p{1.3cm}p{2cm}p{13cm}}
    %\toprule
    Step & \multicolumn{2}{@{}l}{Description, Input Data and Expected Result} \\ \toprule
    \endhead

            \multirow{3}{*}{ 1 } & Description &
            \begin{minipage}[t]{13cm}{\footnotesize
            Identify a dataset with processed visit images in multiple filters.

            \vspace{\dp0}
            } \end{minipage} \\ \cline{2-3}
            & Test Data &
            \begin{minipage}[t]{13cm}{\footnotesize
                No data.
                \vspace{\dp0}
            } \end{minipage} \\ \cline{2-3}
            & Expected Result &
        \\ \midrule

                \multirow{3}{*}{\parbox{1.3cm}{ 2-1
                {\scriptsize from \hyperref[lvv-t987]
                {LVV-T987} } } }

                & {\small Description} &
                \begin{minipage}[t]{13cm}{\scriptsize
                Identify the path to the data repository, which we will refer to as
`DATA/path', then execute the following:

                \vspace{\dp0}
                } \end{minipage} \\ \cdashline{2-3}
                & {\small Test Data} &
                \begin{minipage}[t]{13cm}{\scriptsize
                } \end{minipage} \\ \cdashline{2-3}
                & {\small Expected Result} &
                    \begin{minipage}[t]{13cm}{\scriptsize
                    Butler repo available for reading.

                    \vspace{\dp0}
                    } \end{minipage}
                \\ \hdashline


        \\ \midrule

            \multirow{3}{*}{ 3 } & Description &
            \begin{minipage}[t]{13cm}{\footnotesize
            Display an image of the background model for a full CCD. Repeat this for
all available filters, and confirm that the background is smoothly
varying and defined over the full CCD.

            \vspace{\dp0}
            } \end{minipage} \\ \cline{2-3}
            & Test Data &
            \begin{minipage}[t]{13cm}{\footnotesize
                No data.
                \vspace{\dp0}
            } \end{minipage} \\ \cline{2-3}
            & Expected Result &
                \begin{minipage}[t]{13cm}{\footnotesize
                Well-formed background covering the entire CCD for all CCDs in all
filters.

                \vspace{\dp0}
                } \end{minipage}
        \\ \midrule
    \end{longtable}

\subsection{LVV-T44 - Verify implementation of Documenting Image Characterization}\label{lvv-t44}

\begin{longtable}[]{llllll}
\toprule
Version & Status & Priority & Verification Type & Owner
\\\midrule
1 & Draft & Normal &
Test & Jim Bosch
\\\bottomrule
\multicolumn{6}{c}{ Open \href{https://jira.lsstcorp.org/secure/Tests.jspa\#/testCase/LVV-T44}{LVV-T44} in Jira } \\
\end{longtable}

\subsubsection{Verification Elements}
\begin{itemize}
\item \href{https://jira.lsstcorp.org/browse/LVV-159}{LVV-159} - DMS-REQ-0328-V-01: Documenting Image Characterization

\end{itemize}

\subsubsection{Test Items}
Verify that the persisted format for Processed Visit Images and
associated instrument-signature-removal data products is documented.


\subsubsection{Predecessors}

\subsubsection{Environment Needs}

\paragraph{Software}

\paragraph{Hardware}

\subsubsection{Input Specification}

\subsubsection{Output Specification}

\subsubsection{Test Procedure}
    \begin{longtable}[]{p{1.3cm}p{2cm}p{13cm}}
    %\toprule
    Step & \multicolumn{2}{@{}l}{Description, Input Data and Expected Result} \\ \toprule
    \endhead

            \multirow{3}{*}{ 1 } & Description &
            \begin{minipage}[t]{13cm}{\footnotesize
            Delegate to Alert Production

            \vspace{\dp0}
            } \end{minipage} \\ \cline{2-3}
            & Test Data &
            \begin{minipage}[t]{13cm}{\footnotesize
                No data.
                \vspace{\dp0}
            } \end{minipage} \\ \cline{2-3}
            & Expected Result &
        \\ \midrule
    \end{longtable}

\subsection{LVV-T45 - Verify implementation of Prompt Processing Data Quality Report
Definition}\label{lvv-t45}

\begin{longtable}[]{llllll}
\toprule
Version & Status & Priority & Verification Type & Owner
\\\midrule
1 & Defined & Normal &
Test & Eric Bellm
\\\bottomrule
\multicolumn{6}{c}{ Open \href{https://jira.lsstcorp.org/secure/Tests.jspa\#/testCase/LVV-T45}{LVV-T45} in Jira } \\
\end{longtable}

\subsubsection{Verification Elements}
\begin{itemize}
\item \href{https://jira.lsstcorp.org/browse/LVV-39}{LVV-39} - DMS-REQ-0097-V-01: Level 1 Data Quality Report Definition

\end{itemize}

\subsubsection{Test Items}
Verify that the DMS produces a Prompt Processing Data Quality Report.
~Specifically check absolute value and temporal variation of\\
1. Photometric zeropoint\\
2. Sky brightness\\
3. Seeing\\
4. PSF\\
5. Detection efficiency


\subsubsection{Predecessors}

\subsubsection{Environment Needs}

\paragraph{Software}

\paragraph{Hardware}

\subsubsection{Input Specification}

\subsubsection{Output Specification}

\subsubsection{Test Procedure}
    \begin{longtable}[]{p{1.3cm}p{2cm}p{13cm}}
    %\toprule
    Step & \multicolumn{2}{@{}l}{Description, Input Data and Expected Result} \\ \toprule
    \endhead

            \multirow{3}{*}{ 1 } & Description &
            \begin{minipage}[t]{13cm}{\footnotesize
            Ingest raw data from L1 Test Stand DAQ.

            \vspace{\dp0}
            } \end{minipage} \\ \cline{2-3}
            & Test Data &
            \begin{minipage}[t]{13cm}{\footnotesize
                No data.
                \vspace{\dp0}
            } \end{minipage} \\ \cline{2-3}
            & Expected Result &
        \\ \midrule

                \multirow{3}{*}{\parbox{1.3cm}{ 2-1
                {\scriptsize from \hyperref[lvv-t866]
                {LVV-T866} } } }

                & {\small Description} &
                \begin{minipage}[t]{13cm}{\scriptsize
                Perform the steps of Alert Production (including, but not necessarily
limited to, single frame processing, ISR, source detection/measurement,
PSF estimation, photometric and astrometric calibration, difference
imaging, DIASource detection/measurement, source association). During
Operations, it is presumed that these are automated for a given
dataset.~

                \vspace{\dp0}
                } \end{minipage} \\ \cdashline{2-3}
                & {\small Test Data} &
                \begin{minipage}[t]{13cm}{\scriptsize
                } \end{minipage} \\ \cdashline{2-3}
                & {\small Expected Result} &
                    \begin{minipage}[t]{13cm}{\scriptsize
                    An output dataset including difference images and DIASource and
DIAObject measurements.

                    \vspace{\dp0}
                    } \end{minipage}
                \\ \hdashline


                \multirow{3}{*}{\parbox{1.3cm}{ 2-2
                {\scriptsize from \hyperref[lvv-t866]
                {LVV-T866} } } }

                & {\small Description} &
                \begin{minipage}[t]{13cm}{\scriptsize
                Verify that the expected data products have been produced, and that
catalogs contain reasonable values for measured quantities of interest.

                \vspace{\dp0}
                } \end{minipage} \\ \cdashline{2-3}
                & {\small Test Data} &
                \begin{minipage}[t]{13cm}{\scriptsize
                } \end{minipage} \\ \cdashline{2-3}
                & {\small Expected Result} &
                \\ \hdashline


        \\ \midrule

            \multirow{3}{*}{ 3 } & Description &
            \begin{minipage}[t]{13cm}{\footnotesize
            Load the Prompt Processing QC reports, and observe that a dynamically
updated Data Quality Report has become available at the relevant UI.

            \vspace{\dp0}
            } \end{minipage} \\ \cline{2-3}
            & Test Data &
            \begin{minipage}[t]{13cm}{\footnotesize
                No data.
                \vspace{\dp0}
            } \end{minipage} \\ \cline{2-3}
            & Expected Result &
                \begin{minipage}[t]{13cm}{\footnotesize
                A Prompt Processing QC report is available via a UI, and contains
information about the photometric zeropoint, sky brightness, seeing,
PSF, and detection efficiency, and possibly other relevant quantities.

                \vspace{\dp0}
                } \end{minipage}
        \\ \midrule

            \multirow{3}{*}{ 4 } & Description &
            \begin{minipage}[t]{13cm}{\footnotesize
            Check that a static report is created and archived in a
readily-accessible location.

            \vspace{\dp0}
            } \end{minipage} \\ \cline{2-3}
            & Test Data &
            \begin{minipage}[t]{13cm}{\footnotesize
                No data.
                \vspace{\dp0}
            } \end{minipage} \\ \cline{2-3}
            & Expected Result &
                \begin{minipage}[t]{13cm}{\footnotesize
                Persistence of a static QC report in an accessible location, containing
the same information as in the report from Step 3.

                \vspace{\dp0}
                } \end{minipage}
        \\ \midrule
    \end{longtable}

\subsection{LVV-T46 - Verify implementation of Prompt Processing Performance Report Definition}\label{lvv-t46}

\begin{longtable}[]{llllll}
\toprule
Version & Status & Priority & Verification Type & Owner
\\\midrule
1 & Draft & Normal &
Test & Eric Bellm
\\\bottomrule
\multicolumn{6}{c}{ Open \href{https://jira.lsstcorp.org/secure/Tests.jspa\#/testCase/LVV-T46}{LVV-T46} in Jira } \\
\end{longtable}

\subsubsection{Verification Elements}
\begin{itemize}
\item \href{https://jira.lsstcorp.org/browse/LVV-41}{LVV-41} - DMS-REQ-0099-V-01: Level 1 Performance Report Definition

\end{itemize}

\subsubsection{Test Items}
Verify that the DMS produces a Prompt Processing Performance Report.
~Specifically check that the number of observations that describe each
of the following:\\
1. Successfully processed, recoverable failures, unrecoverable
failures.\\
2. Archived\\
3. Result in science.\\[2\baselineskip]This is testing more the
processing rather than the observatory system.


\subsubsection{Predecessors}

\subsubsection{Environment Needs}

\paragraph{Software}

\paragraph{Hardware}

\subsubsection{Input Specification}

\subsubsection{Output Specification}

\subsubsection{Test Procedure}
    \begin{longtable}[]{p{1.3cm}p{2cm}p{13cm}}
    %\toprule
    Step & \multicolumn{2}{@{}l}{Description, Input Data and Expected Result} \\ \toprule
    \endhead

            \multirow{3}{*}{ 1 } & Description &
            \begin{minipage}[t]{13cm}{\footnotesize
            Execute single-day operations rehearsal, observe report

            \vspace{\dp0}
            } \end{minipage} \\ \cline{2-3}
            & Test Data &
            \begin{minipage}[t]{13cm}{\footnotesize
                No data.
                \vspace{\dp0}
            } \end{minipage} \\ \cline{2-3}
            & Expected Result &
        \\ \midrule
    \end{longtable}

\subsection{LVV-T47 - Verify implementation of Prompt Processing Calibration Report Definition}\label{lvv-t47}

\begin{longtable}[]{llllll}
\toprule
Version & Status & Priority & Verification Type & Owner
\\\midrule
1 & Defined & Normal &
Test & Eric Bellm
\\\bottomrule
\multicolumn{6}{c}{ Open \href{https://jira.lsstcorp.org/secure/Tests.jspa\#/testCase/LVV-T47}{LVV-T47} in Jira } \\
\end{longtable}

\subsubsection{Verification Elements}
\begin{itemize}
\item \href{https://jira.lsstcorp.org/browse/LVV-43}{LVV-43} - DMS-REQ-0101-V-01: Level 1 Calibration Report Definition

\end{itemize}

\subsubsection{Test Items}
Verify that the DMS produces a Prompt Processing Calibration Report.
~Specifically check that this report is capable of identifying when
aspects of the telescope or camera are changing with time.


\subsubsection{Predecessors}

\subsubsection{Environment Needs}

\paragraph{Software}

\paragraph{Hardware}

\subsubsection{Input Specification}

\subsubsection{Output Specification}

\subsubsection{Test Procedure}
    \begin{longtable}[]{p{1.3cm}p{2cm}p{13cm}}
    %\toprule
    Step & \multicolumn{2}{@{}l}{Description, Input Data and Expected Result} \\ \toprule
    \endhead

            \multirow{3}{*}{ 1 } & Description &
            \begin{minipage}[t]{13cm}{\footnotesize
            Identify precursor and simulated calibration datasets on which to run
the L1 calibration pipeline.

            \vspace{\dp0}
            } \end{minipage} \\ \cline{2-3}
            & Test Data &
            \begin{minipage}[t]{13cm}{\footnotesize
                No data.
                \vspace{\dp0}
            } \end{minipage} \\ \cline{2-3}
            & Expected Result &
        \\ \midrule

                \multirow{3}{*}{\parbox{1.3cm}{ 2-1
                {\scriptsize from \hyperref[lvv-t1059]
                {LVV-T1059} } } }

                & {\small Description} &
                \begin{minipage}[t]{13cm}{\scriptsize
                Execute the Daily Calibration Products Update payload. The payload uses
raw calibration images and information from the Transformed EFD to
generate a subset of Master Calibration Images and Calibration Database
entries in the Data Backbone.

                \vspace{\dp0}
                } \end{minipage} \\ \cdashline{2-3}
                & {\small Test Data} &
                \begin{minipage}[t]{13cm}{\scriptsize
                } \end{minipage} \\ \cdashline{2-3}
                & {\small Expected Result} &
                \\ \hdashline


                \multirow{3}{*}{\parbox{1.3cm}{ 2-2
                {\scriptsize from \hyperref[lvv-t1059]
                {LVV-T1059} } } }

                & {\small Description} &
                \begin{minipage}[t]{13cm}{\scriptsize
                Confirm that the expected Master Calibration images and Calibration
Database entries are present and well-formed.

                \vspace{\dp0}
                } \end{minipage} \\ \cdashline{2-3}
                & {\small Test Data} &
                \begin{minipage}[t]{13cm}{\scriptsize
                } \end{minipage} \\ \cdashline{2-3}
                & {\small Expected Result} &
                \\ \hdashline


        \\ \midrule

            \multirow{3}{*}{ 3 } & Description &
            \begin{minipage}[t]{13cm}{\footnotesize
            Check that a dynamic report is created that triggers alerts if
calibrations go out of range.~

            \vspace{\dp0}
            } \end{minipage} \\ \cline{2-3}
            & Test Data &
            \begin{minipage}[t]{13cm}{\footnotesize
                No data.
                \vspace{\dp0}
            } \end{minipage} \\ \cline{2-3}
            & Expected Result &
                \begin{minipage}[t]{13cm}{\footnotesize
                A dynamic report is available via UI to users, and if any out-of-spec
changes have occurred, alerts have been issued.

                \vspace{\dp0}
                } \end{minipage}
        \\ \midrule

            \multirow{3}{*}{ 4 } & Description &
            \begin{minipage}[t]{13cm}{\footnotesize
            Check that a static report is created and archived in a
readily-accessible location.

            \vspace{\dp0}
            } \end{minipage} \\ \cline{2-3}
            & Test Data &
            \begin{minipage}[t]{13cm}{\footnotesize
                No data.
                \vspace{\dp0}
            } \end{minipage} \\ \cline{2-3}
            & Expected Result &
                \begin{minipage}[t]{13cm}{\footnotesize
                An archived version of the calibration report is available and will be
retained in a static file format.

                \vspace{\dp0}
                } \end{minipage}
        \\ \midrule
    \end{longtable}

\subsection{LVV-T48 - Verify implementation of Exposure Catalog}\label{lvv-t48}

\begin{longtable}[]{llllll}
\toprule
Version & Status & Priority & Verification Type & Owner
\\\midrule
1 & Defined & Normal &
Test & Jim Bosch
\\\bottomrule
\multicolumn{6}{c}{ Open \href{https://jira.lsstcorp.org/secure/Tests.jspa\#/testCase/LVV-T48}{LVV-T48} in Jira } \\
\end{longtable}

\subsubsection{Verification Elements}
\begin{itemize}
\item \href{https://jira.lsstcorp.org/browse/LVV-97}{LVV-97} - DMS-REQ-0266-V-01: Exposure Catalog

\end{itemize}

\subsubsection{Test Items}
Verify that the DMS creates an Exposure Catalog that includes\\
1. Observation datetime, exposure time\\
2. Filter\\
3. Dome, telescope orientation and status\\
4. Calibration status\\
5. Airmass and zenith\\
6. Environmental information\\
7. Per-sensor information


\subsubsection{Predecessors}

\subsubsection{Environment Needs}

\paragraph{Software}

\paragraph{Hardware}

\subsubsection{Input Specification}

\subsubsection{Output Specification}

\subsubsection{Test Procedure}
    \begin{longtable}[]{p{1.3cm}p{2cm}p{13cm}}
    %\toprule
    Step & \multicolumn{2}{@{}l}{Description, Input Data and Expected Result} \\ \toprule
    \endhead

            \multirow{3}{*}{ 1 } & Description &
            \begin{minipage}[t]{13cm}{\footnotesize
            Verify that Exposure Catalogs contain the required elements. At present,
the form of the exposure catalog is not defined. This information can be
found for a given Butler repo from the metadata, but will ultimately be
aggregated into a database/table summarizing available exposures.

            \vspace{\dp0}
            } \end{minipage} \\ \cline{2-3}
            & Test Data &
            \begin{minipage}[t]{13cm}{\footnotesize
                No data.
                \vspace{\dp0}
            } \end{minipage} \\ \cline{2-3}
            & Expected Result &
                \begin{minipage}[t]{13cm}{\footnotesize
                A list of the required metadata for a set of exposures is returned and
both human- and machine-readable.

                \vspace{\dp0}
                } \end{minipage}
        \\ \midrule
    \end{longtable}

\subsection{LVV-T49 - Verify implementation of DIASource Catalog}\label{lvv-t49}

\begin{longtable}[]{llllll}
\toprule
Version & Status & Priority & Verification Type & Owner
\\\midrule
1 & Draft & Normal &
Test & Eric Bellm
\\\bottomrule
\multicolumn{6}{c}{ Open \href{https://jira.lsstcorp.org/secure/Tests.jspa\#/testCase/LVV-T49}{LVV-T49} in Jira } \\
\end{longtable}

\subsubsection{Verification Elements}
\begin{itemize}
\item \href{https://jira.lsstcorp.org/browse/LVV-100}{LVV-100} - DMS-REQ-0269-V-01: DIASource Catalog

\end{itemize}

\subsubsection{Test Items}
Verify that the DMS produces a Source catalog from Difference Exposures
with the required attributes.


\subsubsection{Predecessors}

\subsubsection{Environment Needs}

\paragraph{Software}

\paragraph{Hardware}

\subsubsection{Input Specification}

\subsubsection{Output Specification}

\subsubsection{Test Procedure}
    \begin{longtable}[]{p{1.3cm}p{2cm}p{13cm}}
    %\toprule
    Step & \multicolumn{2}{@{}l}{Description, Input Data and Expected Result} \\ \toprule
    \endhead

                \multirow{3}{*}{\parbox{1.3cm}{ 1-1
                {\scriptsize from \hyperref[lvv-t860]
                {LVV-T860} } } }

                & {\small Description} &
                \begin{minipage}[t]{13cm}{\scriptsize
                The `path` that you will use depends on where you are running the
science pipelines. Options:\\[2\baselineskip]

\begin{itemize}
\tightlist
\item
  local (newinstall.sh - based
  install):{[}path\_to\_installation{]}/loadLSST.bash
\item
  development cluster (``lsst-dev''):
  /software/lsstsw/stack/loadLSST.bash
\item
  LSP Notebook aspect (from a terminal):
  /opt/lsst/software/stack/loadLSST.bash
\end{itemize}

From the command line, execute the commands below in the example
code:\\[2\baselineskip]

                \vspace{\dp0}
                } \end{minipage} \\ \cdashline{2-3}
                & {\small Test Data} &
                \begin{minipage}[t]{13cm}{\scriptsize
                } \end{minipage} \\ \cdashline{2-3}
                & {\small Expected Result} &
                    \begin{minipage}[t]{13cm}{\scriptsize
                    Science pipeline software is available for use. If additional packages
are needed (for example, `obs' packages such as `obs\_subaru`), then
additional `setup` commands will be necessary.\\[2\baselineskip]To check
versions in use, type:\\
eups list -s

                    \vspace{\dp0}
                    } \end{minipage}
                \\ \hdashline


        \\ \midrule

                \multirow{3}{*}{\parbox{1.3cm}{ 2-1
                {\scriptsize from \hyperref[lvv-t866]
                {LVV-T866} } } }

                & {\small Description} &
                \begin{minipage}[t]{13cm}{\scriptsize
                Perform the steps of Alert Production (including, but not necessarily
limited to, single frame processing, ISR, source detection/measurement,
PSF estimation, photometric and astrometric calibration, difference
imaging, DIASource detection/measurement, source association). During
Operations, it is presumed that these are automated for a given
dataset.~

                \vspace{\dp0}
                } \end{minipage} \\ \cdashline{2-3}
                & {\small Test Data} &
                \begin{minipage}[t]{13cm}{\scriptsize
                } \end{minipage} \\ \cdashline{2-3}
                & {\small Expected Result} &
                    \begin{minipage}[t]{13cm}{\scriptsize
                    An output dataset including difference images and DIASource and
DIAObject measurements.

                    \vspace{\dp0}
                    } \end{minipage}
                \\ \hdashline


                \multirow{3}{*}{\parbox{1.3cm}{ 2-2
                {\scriptsize from \hyperref[lvv-t866]
                {LVV-T866} } } }

                & {\small Description} &
                \begin{minipage}[t]{13cm}{\scriptsize
                Verify that the expected data products have been produced, and that
catalogs contain reasonable values for measured quantities of interest.

                \vspace{\dp0}
                } \end{minipage} \\ \cdashline{2-3}
                & {\small Test Data} &
                \begin{minipage}[t]{13cm}{\scriptsize
                } \end{minipage} \\ \cdashline{2-3}
                & {\small Expected Result} &
                \\ \hdashline


        \\ \midrule

                \multirow{3}{*}{\parbox{1.3cm}{ 3-1
                {\scriptsize from \hyperref[lvv-t987]
                {LVV-T987} } } }

                & {\small Description} &
                \begin{minipage}[t]{13cm}{\scriptsize
                Identify the path to the data repository, which we will refer to as
`DATA/path', then execute the following:

                \vspace{\dp0}
                } \end{minipage} \\ \cdashline{2-3}
                & {\small Test Data} &
                \begin{minipage}[t]{13cm}{\scriptsize
                } \end{minipage} \\ \cdashline{2-3}
                & {\small Expected Result} &
                    \begin{minipage}[t]{13cm}{\scriptsize
                    Butler repo available for reading.

                    \vspace{\dp0}
                    } \end{minipage}
                \\ \hdashline


        \\ \midrule

            \multirow{3}{*}{ 4 } & Description &
            \begin{minipage}[t]{13cm}{\footnotesize
            Verify that products are produced for DIASource catalog

            \vspace{\dp0}
            } \end{minipage} \\ \cline{2-3}
            & Test Data &
            \begin{minipage}[t]{13cm}{\footnotesize
                No data.
                \vspace{\dp0}
            } \end{minipage} \\ \cline{2-3}
            & Expected Result &
        \\ \midrule
    \end{longtable}

\subsection{LVV-T50 - Verify implementation of Faint DIASource Measurements}\label{lvv-t50}

\begin{longtable}[]{llllll}
\toprule
Version & Status & Priority & Verification Type & Owner
\\\midrule
1 & Draft & Normal &
Test & Eric Bellm
\\\bottomrule
\multicolumn{6}{c}{ Open \href{https://jira.lsstcorp.org/secure/Tests.jspa\#/testCase/LVV-T50}{LVV-T50} in Jira } \\
\end{longtable}

\subsubsection{Verification Elements}
\begin{itemize}
\item \href{https://jira.lsstcorp.org/browse/LVV-101}{LVV-101} - DMS-REQ-0270-V-01: Faint DIASource Measurements

\end{itemize}

\subsubsection{Test Items}
Verify that the DMS can produces DIASources measurements for sources
below the nominal S/N cutoff that satisfy additional criteria.


\subsubsection{Predecessors}

\subsubsection{Environment Needs}

\paragraph{Software}

\paragraph{Hardware}

\subsubsection{Input Specification}
Input Data\\
\hspace*{0.333em}DECam HiTS data.

\subsubsection{Output Specification}

\subsubsection{Test Procedure}
    \begin{longtable}[]{p{1.3cm}p{2cm}p{13cm}}
    %\toprule
    Step & \multicolumn{2}{@{}l}{Description, Input Data and Expected Result} \\ \toprule
    \endhead

                \multirow{3}{*}{\parbox{1.3cm}{ 1-1
                {\scriptsize from \hyperref[lvv-t860]
                {LVV-T860} } } }

                & {\small Description} &
                \begin{minipage}[t]{13cm}{\scriptsize
                The `path` that you will use depends on where you are running the
science pipelines. Options:\\[2\baselineskip]

\begin{itemize}
\tightlist
\item
  local (newinstall.sh - based
  install):{[}path\_to\_installation{]}/loadLSST.bash
\item
  development cluster (``lsst-dev''):
  /software/lsstsw/stack/loadLSST.bash
\item
  LSP Notebook aspect (from a terminal):
  /opt/lsst/software/stack/loadLSST.bash
\end{itemize}

From the command line, execute the commands below in the example
code:\\[2\baselineskip]

                \vspace{\dp0}
                } \end{minipage} \\ \cdashline{2-3}
                & {\small Test Data} &
                \begin{minipage}[t]{13cm}{\scriptsize
                } \end{minipage} \\ \cdashline{2-3}
                & {\small Expected Result} &
                    \begin{minipage}[t]{13cm}{\scriptsize
                    Science pipeline software is available for use. If additional packages
are needed (for example, `obs' packages such as `obs\_subaru`), then
additional `setup` commands will be necessary.\\[2\baselineskip]To check
versions in use, type:\\
eups list -s

                    \vspace{\dp0}
                    } \end{minipage}
                \\ \hdashline


        \\ \midrule

                \multirow{3}{*}{\parbox{1.3cm}{ 2-1
                {\scriptsize from \hyperref[lvv-t866]
                {LVV-T866} } } }

                & {\small Description} &
                \begin{minipage}[t]{13cm}{\scriptsize
                Perform the steps of Alert Production (including, but not necessarily
limited to, single frame processing, ISR, source detection/measurement,
PSF estimation, photometric and astrometric calibration, difference
imaging, DIASource detection/measurement, source association). During
Operations, it is presumed that these are automated for a given
dataset.~

                \vspace{\dp0}
                } \end{minipage} \\ \cdashline{2-3}
                & {\small Test Data} &
                \begin{minipage}[t]{13cm}{\scriptsize
                } \end{minipage} \\ \cdashline{2-3}
                & {\small Expected Result} &
                    \begin{minipage}[t]{13cm}{\scriptsize
                    An output dataset including difference images and DIASource and
DIAObject measurements.

                    \vspace{\dp0}
                    } \end{minipage}
                \\ \hdashline


                \multirow{3}{*}{\parbox{1.3cm}{ 2-2
                {\scriptsize from \hyperref[lvv-t866]
                {LVV-T866} } } }

                & {\small Description} &
                \begin{minipage}[t]{13cm}{\scriptsize
                Verify that the expected data products have been produced, and that
catalogs contain reasonable values for measured quantities of interest.

                \vspace{\dp0}
                } \end{minipage} \\ \cdashline{2-3}
                & {\small Test Data} &
                \begin{minipage}[t]{13cm}{\scriptsize
                } \end{minipage} \\ \cdashline{2-3}
                & {\small Expected Result} &
                \\ \hdashline


        \\ \midrule

            \multirow{3}{*}{ 3 } & Description &
            \begin{minipage}[t]{13cm}{\footnotesize
            As an example of selecting with constrains, Re-run source detection as
an afterburner to select isolated sources (defined as more than 2
arcseconds away from any other objects in the single-image-depth
catalog) that are fainter than the fiducial transSNR cut.

            \vspace{\dp0}
            } \end{minipage} \\ \cline{2-3}
            & Test Data &
            \begin{minipage}[t]{13cm}{\footnotesize
                No data.
                \vspace{\dp0}
            } \end{minipage} \\ \cline{2-3}
            & Expected Result &
        \\ \midrule
    \end{longtable}

\subsection{LVV-T51 - Verify implementation of DIAObject Catalog}\label{lvv-t51}

\begin{longtable}[]{llllll}
\toprule
Version & Status & Priority & Verification Type & Owner
\\\midrule
1 & Draft & Normal &
Test & Eric Bellm
\\\bottomrule
\multicolumn{6}{c}{ Open \href{https://jira.lsstcorp.org/secure/Tests.jspa\#/testCase/LVV-T51}{LVV-T51} in Jira } \\
\end{longtable}

\subsubsection{Verification Elements}
\begin{itemize}
\item \href{https://jira.lsstcorp.org/browse/LVV-102}{LVV-102} - DMS-REQ-0271-V-01: Max nearby galaxies associated with DIASource

\end{itemize}

\subsubsection{Test Items}
Verify that the DIAObject includes a unique ID, identifiers for nearest
stars and nearest galaxies, and probability of matching to static
Object.


\subsubsection{Predecessors}

\subsubsection{Environment Needs}

\paragraph{Software}

\paragraph{Hardware}

\subsubsection{Input Specification}

\subsubsection{Output Specification}

\subsubsection{Test Procedure}
    \begin{longtable}[]{p{1.3cm}p{2cm}p{13cm}}
    %\toprule
    Step & \multicolumn{2}{@{}l}{Description, Input Data and Expected Result} \\ \toprule
    \endhead

                \multirow{3}{*}{\parbox{1.3cm}{ 1-1
                {\scriptsize from \hyperref[lvv-t866]
                {LVV-T866} } } }

                & {\small Description} &
                \begin{minipage}[t]{13cm}{\scriptsize
                Perform the steps of Alert Production (including, but not necessarily
limited to, single frame processing, ISR, source detection/measurement,
PSF estimation, photometric and astrometric calibration, difference
imaging, DIASource detection/measurement, source association). During
Operations, it is presumed that these are automated for a given
dataset.~

                \vspace{\dp0}
                } \end{minipage} \\ \cdashline{2-3}
                & {\small Test Data} &
                \begin{minipage}[t]{13cm}{\scriptsize
                } \end{minipage} \\ \cdashline{2-3}
                & {\small Expected Result} &
                    \begin{minipage}[t]{13cm}{\scriptsize
                    An output dataset including difference images and DIASource and
DIAObject measurements.

                    \vspace{\dp0}
                    } \end{minipage}
                \\ \hdashline


                \multirow{3}{*}{\parbox{1.3cm}{ 1-2
                {\scriptsize from \hyperref[lvv-t866]
                {LVV-T866} } } }

                & {\small Description} &
                \begin{minipage}[t]{13cm}{\scriptsize
                Verify that the expected data products have been produced, and that
catalogs contain reasonable values for measured quantities of interest.

                \vspace{\dp0}
                } \end{minipage} \\ \cdashline{2-3}
                & {\small Test Data} &
                \begin{minipage}[t]{13cm}{\scriptsize
                } \end{minipage} \\ \cdashline{2-3}
                & {\small Expected Result} &
                \\ \hdashline


        \\ \midrule

                \multirow{3}{*}{\parbox{1.3cm}{ 2-1
                {\scriptsize from \hyperref[lvv-t987]
                {LVV-T987} } } }

                & {\small Description} &
                \begin{minipage}[t]{13cm}{\scriptsize
                Identify the path to the data repository, which we will refer to as
`DATA/path', then execute the following:

                \vspace{\dp0}
                } \end{minipage} \\ \cdashline{2-3}
                & {\small Test Data} &
                \begin{minipage}[t]{13cm}{\scriptsize
                } \end{minipage} \\ \cdashline{2-3}
                & {\small Expected Result} &
                    \begin{minipage}[t]{13cm}{\scriptsize
                    Butler repo available for reading.

                    \vspace{\dp0}
                    } \end{minipage}
                \\ \hdashline


        \\ \midrule

            \multirow{3}{*}{ 3 } & Description &
            \begin{minipage}[t]{13cm}{\footnotesize
            Verify that DIAObjects have diaNearbyObjMaxStar and
diaNearbyObjMaxGalaxies that point to the Object catalog and are within
dianNearbyObjRadius; the probability of association; and the required
DIAObject properties.

            \vspace{\dp0}
            } \end{minipage} \\ \cline{2-3}
            & Test Data &
            \begin{minipage}[t]{13cm}{\footnotesize
                No data.
                \vspace{\dp0}
            } \end{minipage} \\ \cline{2-3}
            & Expected Result &
        \\ \midrule
    \end{longtable}

\subsection{LVV-T52 - Verify implementation of DIAObject Attributes}\label{lvv-t52}

\begin{longtable}[]{llllll}
\toprule
Version & Status & Priority & Verification Type & Owner
\\\midrule
1 & Draft & Normal &
Test & Eric Bellm
\\\bottomrule
\multicolumn{6}{c}{ Open \href{https://jira.lsstcorp.org/secure/Tests.jspa\#/testCase/LVV-T52}{LVV-T52} in Jira } \\
\end{longtable}

\subsubsection{Verification Elements}
\begin{itemize}
\item \href{https://jira.lsstcorp.org/browse/LVV-103}{LVV-103} - DMS-REQ-0272-V-01: DIAObject Attributes

\end{itemize}

\subsubsection{Test Items}
Verify that the DMS provides summary attributes for each DIAObject,
including periodicity measures.


\subsubsection{Predecessors}

\subsubsection{Environment Needs}

\paragraph{Software}

\paragraph{Hardware}

\subsubsection{Input Specification}

\subsubsection{Output Specification}

\subsubsection{Test Procedure}
    \begin{longtable}[]{p{1.3cm}p{2cm}p{13cm}}
    %\toprule
    Step & \multicolumn{2}{@{}l}{Description, Input Data and Expected Result} \\ \toprule
    \endhead

                \multirow{3}{*}{\parbox{1.3cm}{ 1-1
                {\scriptsize from \hyperref[lvv-t866]
                {LVV-T866} } } }

                & {\small Description} &
                \begin{minipage}[t]{13cm}{\scriptsize
                Perform the steps of Alert Production (including, but not necessarily
limited to, single frame processing, ISR, source detection/measurement,
PSF estimation, photometric and astrometric calibration, difference
imaging, DIASource detection/measurement, source association). During
Operations, it is presumed that these are automated for a given
dataset.~

                \vspace{\dp0}
                } \end{minipage} \\ \cdashline{2-3}
                & {\small Test Data} &
                \begin{minipage}[t]{13cm}{\scriptsize
                } \end{minipage} \\ \cdashline{2-3}
                & {\small Expected Result} &
                    \begin{minipage}[t]{13cm}{\scriptsize
                    An output dataset including difference images and DIASource and
DIAObject measurements.

                    \vspace{\dp0}
                    } \end{minipage}
                \\ \hdashline


                \multirow{3}{*}{\parbox{1.3cm}{ 1-2
                {\scriptsize from \hyperref[lvv-t866]
                {LVV-T866} } } }

                & {\small Description} &
                \begin{minipage}[t]{13cm}{\scriptsize
                Verify that the expected data products have been produced, and that
catalogs contain reasonable values for measured quantities of interest.

                \vspace{\dp0}
                } \end{minipage} \\ \cdashline{2-3}
                & {\small Test Data} &
                \begin{minipage}[t]{13cm}{\scriptsize
                } \end{minipage} \\ \cdashline{2-3}
                & {\small Expected Result} &
                \\ \hdashline


        \\ \midrule

                \multirow{3}{*}{\parbox{1.3cm}{ 2-1
                {\scriptsize from \hyperref[lvv-t987]
                {LVV-T987} } } }

                & {\small Description} &
                \begin{minipage}[t]{13cm}{\scriptsize
                Identify the path to the data repository, which we will refer to as
`DATA/path', then execute the following:

                \vspace{\dp0}
                } \end{minipage} \\ \cdashline{2-3}
                & {\small Test Data} &
                \begin{minipage}[t]{13cm}{\scriptsize
                } \end{minipage} \\ \cdashline{2-3}
                & {\small Expected Result} &
                    \begin{minipage}[t]{13cm}{\scriptsize
                    Butler repo available for reading.

                    \vspace{\dp0}
                    } \end{minipage}
                \\ \hdashline


        \\ \midrule

            \multirow{3}{*}{ 3 } & Description &
            \begin{minipage}[t]{13cm}{\footnotesize
            Confirm that the DIAObjects include summary attributes as specified.

            \vspace{\dp0}
            } \end{minipage} \\ \cline{2-3}
            & Test Data &
            \begin{minipage}[t]{13cm}{\footnotesize
                No data.
                \vspace{\dp0}
            } \end{minipage} \\ \cline{2-3}
            & Expected Result &
        \\ \midrule
    \end{longtable}

\subsection{LVV-T53 - Verify implementation of SSObject Catalog}\label{lvv-t53}

\begin{longtable}[]{llllll}
\toprule
Version & Status & Priority & Verification Type & Owner
\\\midrule
1 & Draft & Normal &
Test & Eric Bellm
\\\bottomrule
\multicolumn{6}{c}{ Open \href{https://jira.lsstcorp.org/secure/Tests.jspa\#/testCase/LVV-T53}{LVV-T53} in Jira } \\
\end{longtable}

\subsubsection{Verification Elements}
\begin{itemize}
\item \href{https://jira.lsstcorp.org/browse/LVV-104}{LVV-104} - DMS-REQ-0273-V-01: SSObject Catalog

\end{itemize}

\subsubsection{Test Items}
Verify that the DMS produces a catalog of Solar System Objects identify
from Moving Object Processing.\\
Verify that the SSObject catalog includes orbital elements and
additional related quanitites.


\subsubsection{Predecessors}

\subsubsection{Environment Needs}

\paragraph{Software}

\paragraph{Hardware}

\subsubsection{Input Specification}

\subsubsection{Output Specification}

\subsubsection{Test Procedure}
    \begin{longtable}[]{p{1.3cm}p{2cm}p{13cm}}
    %\toprule
    Step & \multicolumn{2}{@{}l}{Description, Input Data and Expected Result} \\ \toprule
    \endhead

                \multirow{3}{*}{\parbox{1.3cm}{ 1-1
                {\scriptsize from \hyperref[lvv-t866]
                {LVV-T866} } } }

                & {\small Description} &
                \begin{minipage}[t]{13cm}{\scriptsize
                Perform the steps of Alert Production (including, but not necessarily
limited to, single frame processing, ISR, source detection/measurement,
PSF estimation, photometric and astrometric calibration, difference
imaging, DIASource detection/measurement, source association). During
Operations, it is presumed that these are automated for a given
dataset.~

                \vspace{\dp0}
                } \end{minipage} \\ \cdashline{2-3}
                & {\small Test Data} &
                \begin{minipage}[t]{13cm}{\scriptsize
                } \end{minipage} \\ \cdashline{2-3}
                & {\small Expected Result} &
                    \begin{minipage}[t]{13cm}{\scriptsize
                    An output dataset including difference images and DIASource and
DIAObject measurements.

                    \vspace{\dp0}
                    } \end{minipage}
                \\ \hdashline


                \multirow{3}{*}{\parbox{1.3cm}{ 1-2
                {\scriptsize from \hyperref[lvv-t866]
                {LVV-T866} } } }

                & {\small Description} &
                \begin{minipage}[t]{13cm}{\scriptsize
                Verify that the expected data products have been produced, and that
catalogs contain reasonable values for measured quantities of interest.

                \vspace{\dp0}
                } \end{minipage} \\ \cdashline{2-3}
                & {\small Test Data} &
                \begin{minipage}[t]{13cm}{\scriptsize
                } \end{minipage} \\ \cdashline{2-3}
                & {\small Expected Result} &
                \\ \hdashline


        \\ \midrule

                \multirow{3}{*}{\parbox{1.3cm}{ 2-1
                {\scriptsize from \hyperref[lvv-t901]
                {LVV-T901} } } }

                & {\small Description} &
                \begin{minipage}[t]{13cm}{\scriptsize
                Perform the steps of Moving Object Pipeline (MOPS) processing on newly
detected DIASources, and generate Solar System data products including
Solar System objects with associated Keplerian orbits, errors, and
detected DIASources. This includes running processes to link DIASource
detections within a night (called tracklets), to link these tracklets
across multiple nights (into tracks), to fit the tracks with an orbital
model to identify those tracks that are consistent with an asteroid
orbit, to match these new orbits with existing SSObjects, and to update
the SSObject table. ~ ~ ~ ~ ~ ~ ~ ~ ~ ~ ~ ~ ~ ~ ~ ~ ~ ~ ~~

                \vspace{\dp0}
                } \end{minipage} \\ \cdashline{2-3}
                & {\small Test Data} &
                \begin{minipage}[t]{13cm}{\scriptsize
                } \end{minipage} \\ \cdashline{2-3}
                & {\small Expected Result} &
                    \begin{minipage}[t]{13cm}{\scriptsize
                    An output dataset consisting of an updated SSObject database with
SSObjects both added and pruned as the orbital fits have been refined,
and an updated DIASource database with DIASources assigned and
unassigned to SSObjects.

                    \vspace{\dp0}
                    } \end{minipage}
                \\ \hdashline


                \multirow{3}{*}{\parbox{1.3cm}{ 2-2
                {\scriptsize from \hyperref[lvv-t901]
                {LVV-T901} } } }

                & {\small Description} &
                \begin{minipage}[t]{13cm}{\scriptsize
                Verify that the expected data products have been produced, and that
catalogs contain reasonable values for measured quantities of interest.

                \vspace{\dp0}
                } \end{minipage} \\ \cdashline{2-3}
                & {\small Test Data} &
                \begin{minipage}[t]{13cm}{\scriptsize
                } \end{minipage} \\ \cdashline{2-3}
                & {\small Expected Result} &
                \\ \hdashline


        \\ \midrule

                \multirow{3}{*}{\parbox{1.3cm}{ 3-1
                {\scriptsize from \hyperref[lvv-t987]
                {LVV-T987} } } }

                & {\small Description} &
                \begin{minipage}[t]{13cm}{\scriptsize
                Identify the path to the data repository, which we will refer to as
`DATA/path', then execute the following:

                \vspace{\dp0}
                } \end{minipage} \\ \cdashline{2-3}
                & {\small Test Data} &
                \begin{minipage}[t]{13cm}{\scriptsize
                } \end{minipage} \\ \cdashline{2-3}
                & {\small Expected Result} &
                    \begin{minipage}[t]{13cm}{\scriptsize
                    Butler repo available for reading.

                    \vspace{\dp0}
                    } \end{minipage}
                \\ \hdashline


        \\ \midrule

            \multirow{3}{*}{ 4 } & Description &
            \begin{minipage}[t]{13cm}{\footnotesize
            Inspect SSObject catalog and verify the presence of the required
elements (​\href{https://jira.lsstcorp.org/browse/LVV-104}{LVV-104)}​​​.

            \vspace{\dp0}
            } \end{minipage} \\ \cline{2-3}
            & Test Data &
            \begin{minipage}[t]{13cm}{\footnotesize
                No data.
                \vspace{\dp0}
            } \end{minipage} \\ \cline{2-3}
            & Expected Result &
        \\ \midrule
    \end{longtable}

\subsection{LVV-T54 - Verify implementation of Alert Content}\label{lvv-t54}

\begin{longtable}[]{llllll}
\toprule
Version & Status & Priority & Verification Type & Owner
\\\midrule
1 & Draft & Normal &
Test & Eric Bellm
\\\bottomrule
\multicolumn{6}{c}{ Open \href{https://jira.lsstcorp.org/secure/Tests.jspa\#/testCase/LVV-T54}{LVV-T54} in Jira } \\
\end{longtable}

\subsubsection{Verification Elements}
\begin{itemize}
\item \href{https://jira.lsstcorp.org/browse/LVV-105}{LVV-105} - DMS-REQ-0274-V-01: Alert Content

\end{itemize}

\subsubsection{Test Items}
Verify that the DMS creates an Alert for each detected DIASource\\
Verify that this Alert is broadcasted using community protocols\\
Verify that the context of the Alert packet match requirements.


\subsubsection{Predecessors}

\subsubsection{Environment Needs}

\paragraph{Software}

\paragraph{Hardware}

\subsubsection{Input Specification}

\subsubsection{Output Specification}

\subsubsection{Test Procedure}
    \begin{longtable}[]{p{1.3cm}p{2cm}p{13cm}}
    %\toprule
    Step & \multicolumn{2}{@{}l}{Description, Input Data and Expected Result} \\ \toprule
    \endhead

                \multirow{3}{*}{\parbox{1.3cm}{ 1-1
                {\scriptsize from \hyperref[lvv-t866]
                {LVV-T866} } } }

                & {\small Description} &
                \begin{minipage}[t]{13cm}{\scriptsize
                Perform the steps of Alert Production (including, but not necessarily
limited to, single frame processing, ISR, source detection/measurement,
PSF estimation, photometric and astrometric calibration, difference
imaging, DIASource detection/measurement, source association). During
Operations, it is presumed that these are automated for a given
dataset.~

                \vspace{\dp0}
                } \end{minipage} \\ \cdashline{2-3}
                & {\small Test Data} &
                \begin{minipage}[t]{13cm}{\scriptsize
                } \end{minipage} \\ \cdashline{2-3}
                & {\small Expected Result} &
                    \begin{minipage}[t]{13cm}{\scriptsize
                    An output dataset including difference images and DIASource and
DIAObject measurements.

                    \vspace{\dp0}
                    } \end{minipage}
                \\ \hdashline


                \multirow{3}{*}{\parbox{1.3cm}{ 1-2
                {\scriptsize from \hyperref[lvv-t866]
                {LVV-T866} } } }

                & {\small Description} &
                \begin{minipage}[t]{13cm}{\scriptsize
                Verify that the expected data products have been produced, and that
catalogs contain reasonable values for measured quantities of interest.

                \vspace{\dp0}
                } \end{minipage} \\ \cdashline{2-3}
                & {\small Test Data} &
                \begin{minipage}[t]{13cm}{\scriptsize
                } \end{minipage} \\ \cdashline{2-3}
                & {\small Expected Result} &
                \\ \hdashline


        \\ \midrule

            \multirow{3}{*}{ 2 } & Description &
            \begin{minipage}[t]{13cm}{\footnotesize
            Examine the serialized alert packets to confirm the presence of the
required elements
(\href{https://jira.lsstcorp.org/browse/LVV-105}{LVV-105}).~ ~ ~ ~ ~ ~ ~
~ ~ ~ ~ ~ ~ ~ ~ ~ ~

            \vspace{\dp0}
            } \end{minipage} \\ \cline{2-3}
            & Test Data &
            \begin{minipage}[t]{13cm}{\footnotesize
                No data.
                \vspace{\dp0}
            } \end{minipage} \\ \cline{2-3}
            & Expected Result &
        \\ \midrule
    \end{longtable}

\subsection{LVV-T55 - Verify implementation of DIAForcedSource Catalog}\label{lvv-t55}

\begin{longtable}[]{llllll}
\toprule
Version & Status & Priority & Verification Type & Owner
\\\midrule
1 & Draft & Normal &
Test & Eric Bellm
\\\bottomrule
\multicolumn{6}{c}{ Open \href{https://jira.lsstcorp.org/secure/Tests.jspa\#/testCase/LVV-T55}{LVV-T55} in Jira } \\
\end{longtable}

\subsubsection{Verification Elements}
\begin{itemize}
\item \href{https://jira.lsstcorp.org/browse/LVV-148}{LVV-148} - DMS-REQ-0317-V-01: DIAForcedSource Catalog

\end{itemize}

\subsubsection{Test Items}
Verify that the DMS produces a DIAForcedSource Catalog and that the
catalog contains measured fluxes for DIAObjects.


\subsubsection{Predecessors}

\subsubsection{Environment Needs}

\paragraph{Software}

\paragraph{Hardware}

\subsubsection{Input Specification}

\subsubsection{Output Specification}

\subsubsection{Test Procedure}
    \begin{longtable}[]{p{1.3cm}p{2cm}p{13cm}}
    %\toprule
    Step & \multicolumn{2}{@{}l}{Description, Input Data and Expected Result} \\ \toprule
    \endhead

                \multirow{3}{*}{\parbox{1.3cm}{ 1-1
                {\scriptsize from \hyperref[lvv-t866]
                {LVV-T866} } } }

                & {\small Description} &
                \begin{minipage}[t]{13cm}{\scriptsize
                Perform the steps of Alert Production (including, but not necessarily
limited to, single frame processing, ISR, source detection/measurement,
PSF estimation, photometric and astrometric calibration, difference
imaging, DIASource detection/measurement, source association). During
Operations, it is presumed that these are automated for a given
dataset.~

                \vspace{\dp0}
                } \end{minipage} \\ \cdashline{2-3}
                & {\small Test Data} &
                \begin{minipage}[t]{13cm}{\scriptsize
                } \end{minipage} \\ \cdashline{2-3}
                & {\small Expected Result} &
                    \begin{minipage}[t]{13cm}{\scriptsize
                    An output dataset including difference images and DIASource and
DIAObject measurements.

                    \vspace{\dp0}
                    } \end{minipage}
                \\ \hdashline


                \multirow{3}{*}{\parbox{1.3cm}{ 1-2
                {\scriptsize from \hyperref[lvv-t866]
                {LVV-T866} } } }

                & {\small Description} &
                \begin{minipage}[t]{13cm}{\scriptsize
                Verify that the expected data products have been produced, and that
catalogs contain reasonable values for measured quantities of interest.

                \vspace{\dp0}
                } \end{minipage} \\ \cdashline{2-3}
                & {\small Test Data} &
                \begin{minipage}[t]{13cm}{\scriptsize
                } \end{minipage} \\ \cdashline{2-3}
                & {\small Expected Result} &
                \\ \hdashline


        \\ \midrule

                \multirow{3}{*}{\parbox{1.3cm}{ 2-1
                {\scriptsize from \hyperref[lvv-t987]
                {LVV-T987} } } }

                & {\small Description} &
                \begin{minipage}[t]{13cm}{\scriptsize
                Identify the path to the data repository, which we will refer to as
`DATA/path', then execute the following:

                \vspace{\dp0}
                } \end{minipage} \\ \cdashline{2-3}
                & {\small Test Data} &
                \begin{minipage}[t]{13cm}{\scriptsize
                } \end{minipage} \\ \cdashline{2-3}
                & {\small Expected Result} &
                    \begin{minipage}[t]{13cm}{\scriptsize
                    Butler repo available for reading.

                    \vspace{\dp0}
                    } \end{minipage}
                \\ \hdashline


        \\ \midrule

            \multirow{3}{*}{ 3 } & Description &
            \begin{minipage}[t]{13cm}{\footnotesize
            Confirm that the DIAForcedSource catalog contains measurements for each
source.

            \vspace{\dp0}
            } \end{minipage} \\ \cline{2-3}
            & Test Data &
            \begin{minipage}[t]{13cm}{\footnotesize
                No data.
                \vspace{\dp0}
            } \end{minipage} \\ \cline{2-3}
            & Expected Result &
        \\ \midrule
    \end{longtable}

\subsection{LVV-T56 - Verify implementation of Characterizing Variability}\label{lvv-t56}

\begin{longtable}[]{llllll}
\toprule
Version & Status & Priority & Verification Type & Owner
\\\midrule
1 & Draft & Normal &
Test & Eric Bellm
\\\bottomrule
\multicolumn{6}{c}{ Open \href{https://jira.lsstcorp.org/secure/Tests.jspa\#/testCase/LVV-T56}{LVV-T56} in Jira } \\
\end{longtable}

\subsubsection{Verification Elements}
\begin{itemize}
\item \href{https://jira.lsstcorp.org/browse/LVV-150}{LVV-150} - DMS-REQ-0319-V-01: Characterizing Variability

\end{itemize}

\subsubsection{Test Items}
Verify that the variability characterization in the DIAObject catalog
includes data collected within previous ``diaCharacterizationCutoff''
period of time.


\subsubsection{Predecessors}

\subsubsection{Environment Needs}

\paragraph{Software}

\paragraph{Hardware}

\subsubsection{Input Specification}

\subsubsection{Output Specification}

\subsubsection{Test Procedure}
    \begin{longtable}[]{p{1.3cm}p{2cm}p{13cm}}
    %\toprule
    Step & \multicolumn{2}{@{}l}{Description, Input Data and Expected Result} \\ \toprule
    \endhead

                \multirow{3}{*}{\parbox{1.3cm}{ 1-1
                {\scriptsize from \hyperref[lvv-t866]
                {LVV-T866} } } }

                & {\small Description} &
                \begin{minipage}[t]{13cm}{\scriptsize
                Perform the steps of Alert Production (including, but not necessarily
limited to, single frame processing, ISR, source detection/measurement,
PSF estimation, photometric and astrometric calibration, difference
imaging, DIASource detection/measurement, source association). During
Operations, it is presumed that these are automated for a given
dataset.~

                \vspace{\dp0}
                } \end{minipage} \\ \cdashline{2-3}
                & {\small Test Data} &
                \begin{minipage}[t]{13cm}{\scriptsize
                } \end{minipage} \\ \cdashline{2-3}
                & {\small Expected Result} &
                    \begin{minipage}[t]{13cm}{\scriptsize
                    An output dataset including difference images and DIASource and
DIAObject measurements.

                    \vspace{\dp0}
                    } \end{minipage}
                \\ \hdashline


                \multirow{3}{*}{\parbox{1.3cm}{ 1-2
                {\scriptsize from \hyperref[lvv-t866]
                {LVV-T866} } } }

                & {\small Description} &
                \begin{minipage}[t]{13cm}{\scriptsize
                Verify that the expected data products have been produced, and that
catalogs contain reasonable values for measured quantities of interest.

                \vspace{\dp0}
                } \end{minipage} \\ \cdashline{2-3}
                & {\small Test Data} &
                \begin{minipage}[t]{13cm}{\scriptsize
                } \end{minipage} \\ \cdashline{2-3}
                & {\small Expected Result} &
                \\ \hdashline


        \\ \midrule

            \multirow{3}{*}{ 2 } & Description &
            \begin{minipage}[t]{13cm}{\footnotesize
            Verify that the issued alerts contain measurements during the
diaCharacterizationCutoff.

            \vspace{\dp0}
            } \end{minipage} \\ \cline{2-3}
            & Test Data &
            \begin{minipage}[t]{13cm}{\footnotesize
                No data.
                \vspace{\dp0}
            } \end{minipage} \\ \cline{2-3}
            & Expected Result &
        \\ \midrule
    \end{longtable}

\subsection{LVV-T57 - Verify implementation of Calculating SSObject Parameters}\label{lvv-t57}

\begin{longtable}[]{llllll}
\toprule
Version & Status & Priority & Verification Type & Owner
\\\midrule
1 & Draft & Normal &
Test & Eric Bellm
\\\bottomrule
\multicolumn{6}{c}{ Open \href{https://jira.lsstcorp.org/secure/Tests.jspa\#/testCase/LVV-T57}{LVV-T57} in Jira } \\
\end{longtable}

\subsubsection{Verification Elements}
\begin{itemize}
\item \href{https://jira.lsstcorp.org/browse/LVV-154}{LVV-154} - DMS-REQ-0323-V-01: Calculating SSObject Parameters

\end{itemize}

\subsubsection{Test Items}
Verify that the DMS database provides functions to compute phase angles
and magnitudes in LSST bands for every SSObject.


\subsubsection{Predecessors}

\subsubsection{Environment Needs}

\paragraph{Software}

\paragraph{Hardware}

\subsubsection{Input Specification}

\subsubsection{Output Specification}

\subsubsection{Test Procedure}
    \begin{longtable}[]{p{1.3cm}p{2cm}p{13cm}}
    %\toprule
    Step & \multicolumn{2}{@{}l}{Description, Input Data and Expected Result} \\ \toprule
    \endhead

                \multirow{3}{*}{\parbox{1.3cm}{ 1-1
                {\scriptsize from \hyperref[lvv-t866]
                {LVV-T866} } } }

                & {\small Description} &
                \begin{minipage}[t]{13cm}{\scriptsize
                Perform the steps of Alert Production (including, but not necessarily
limited to, single frame processing, ISR, source detection/measurement,
PSF estimation, photometric and astrometric calibration, difference
imaging, DIASource detection/measurement, source association). During
Operations, it is presumed that these are automated for a given
dataset.~

                \vspace{\dp0}
                } \end{minipage} \\ \cdashline{2-3}
                & {\small Test Data} &
                \begin{minipage}[t]{13cm}{\scriptsize
                } \end{minipage} \\ \cdashline{2-3}
                & {\small Expected Result} &
                    \begin{minipage}[t]{13cm}{\scriptsize
                    An output dataset including difference images and DIASource and
DIAObject measurements.

                    \vspace{\dp0}
                    } \end{minipage}
                \\ \hdashline


                \multirow{3}{*}{\parbox{1.3cm}{ 1-2
                {\scriptsize from \hyperref[lvv-t866]
                {LVV-T866} } } }

                & {\small Description} &
                \begin{minipage}[t]{13cm}{\scriptsize
                Verify that the expected data products have been produced, and that
catalogs contain reasonable values for measured quantities of interest.

                \vspace{\dp0}
                } \end{minipage} \\ \cdashline{2-3}
                & {\small Test Data} &
                \begin{minipage}[t]{13cm}{\scriptsize
                } \end{minipage} \\ \cdashline{2-3}
                & {\small Expected Result} &
                \\ \hdashline


        \\ \midrule

                \multirow{3}{*}{\parbox{1.3cm}{ 2-1
                {\scriptsize from \hyperref[lvv-t901]
                {LVV-T901} } } }

                & {\small Description} &
                \begin{minipage}[t]{13cm}{\scriptsize
                Perform the steps of Moving Object Pipeline (MOPS) processing on newly
detected DIASources, and generate Solar System data products including
Solar System objects with associated Keplerian orbits, errors, and
detected DIASources. This includes running processes to link DIASource
detections within a night (called tracklets), to link these tracklets
across multiple nights (into tracks), to fit the tracks with an orbital
model to identify those tracks that are consistent with an asteroid
orbit, to match these new orbits with existing SSObjects, and to update
the SSObject table. ~ ~ ~ ~ ~ ~ ~ ~ ~ ~ ~ ~ ~ ~ ~ ~ ~ ~ ~~

                \vspace{\dp0}
                } \end{minipage} \\ \cdashline{2-3}
                & {\small Test Data} &
                \begin{minipage}[t]{13cm}{\scriptsize
                } \end{minipage} \\ \cdashline{2-3}
                & {\small Expected Result} &
                    \begin{minipage}[t]{13cm}{\scriptsize
                    An output dataset consisting of an updated SSObject database with
SSObjects both added and pruned as the orbital fits have been refined,
and an updated DIASource database with DIASources assigned and
unassigned to SSObjects.

                    \vspace{\dp0}
                    } \end{minipage}
                \\ \hdashline


                \multirow{3}{*}{\parbox{1.3cm}{ 2-2
                {\scriptsize from \hyperref[lvv-t901]
                {LVV-T901} } } }

                & {\small Description} &
                \begin{minipage}[t]{13cm}{\scriptsize
                Verify that the expected data products have been produced, and that
catalogs contain reasonable values for measured quantities of interest.

                \vspace{\dp0}
                } \end{minipage} \\ \cdashline{2-3}
                & {\small Test Data} &
                \begin{minipage}[t]{13cm}{\scriptsize
                } \end{minipage} \\ \cdashline{2-3}
                & {\small Expected Result} &
                \\ \hdashline


        \\ \midrule

            \multirow{3}{*}{ 3 } & Description &
            \begin{minipage}[t]{13cm}{\footnotesize
            Computer the phase angle, reduced and absolute asteroid magnitudes for
objects identified in SSObject Catalog

            \vspace{\dp0}
            } \end{minipage} \\ \cline{2-3}
            & Test Data &
            \begin{minipage}[t]{13cm}{\footnotesize
                No data.
                \vspace{\dp0}
            } \end{minipage} \\ \cline{2-3}
            & Expected Result &
        \\ \midrule
    \end{longtable}

\subsection{LVV-T58 - Verify implementation of Matching DIASources to Objects}\label{lvv-t58}

\begin{longtable}[]{llllll}
\toprule
Version & Status & Priority & Verification Type & Owner
\\\midrule
1 & Draft & Normal &
Test & Eric Bellm
\\\bottomrule
\multicolumn{6}{c}{ Open \href{https://jira.lsstcorp.org/secure/Tests.jspa\#/testCase/LVV-T58}{LVV-T58} in Jira } \\
\end{longtable}

\subsubsection{Verification Elements}
\begin{itemize}
\item \href{https://jira.lsstcorp.org/browse/LVV-155}{LVV-155} - DMS-REQ-0324-V-01: Matching DIASources to Objects

\end{itemize}

\subsubsection{Test Items}
Verify that a cross-match table is available between DIASources and
Objects.


\subsubsection{Predecessors}

\subsubsection{Environment Needs}

\paragraph{Software}

\paragraph{Hardware}

\subsubsection{Input Specification}

\subsubsection{Output Specification}

\subsubsection{Test Procedure}
    \begin{longtable}[]{p{1.3cm}p{2cm}p{13cm}}
    %\toprule
    Step & \multicolumn{2}{@{}l}{Description, Input Data and Expected Result} \\ \toprule
    \endhead

                \multirow{3}{*}{\parbox{1.3cm}{ 1-1
                {\scriptsize from \hyperref[lvv-t866]
                {LVV-T866} } } }

                & {\small Description} &
                \begin{minipage}[t]{13cm}{\scriptsize
                Perform the steps of Alert Production (including, but not necessarily
limited to, single frame processing, ISR, source detection/measurement,
PSF estimation, photometric and astrometric calibration, difference
imaging, DIASource detection/measurement, source association). During
Operations, it is presumed that these are automated for a given
dataset.~

                \vspace{\dp0}
                } \end{minipage} \\ \cdashline{2-3}
                & {\small Test Data} &
                \begin{minipage}[t]{13cm}{\scriptsize
                } \end{minipage} \\ \cdashline{2-3}
                & {\small Expected Result} &
                    \begin{minipage}[t]{13cm}{\scriptsize
                    An output dataset including difference images and DIASource and
DIAObject measurements.

                    \vspace{\dp0}
                    } \end{minipage}
                \\ \hdashline


                \multirow{3}{*}{\parbox{1.3cm}{ 1-2
                {\scriptsize from \hyperref[lvv-t866]
                {LVV-T866} } } }

                & {\small Description} &
                \begin{minipage}[t]{13cm}{\scriptsize
                Verify that the expected data products have been produced, and that
catalogs contain reasonable values for measured quantities of interest.

                \vspace{\dp0}
                } \end{minipage} \\ \cdashline{2-3}
                & {\small Test Data} &
                \begin{minipage}[t]{13cm}{\scriptsize
                } \end{minipage} \\ \cdashline{2-3}
                & {\small Expected Result} &
                \\ \hdashline


        \\ \midrule

                \multirow{3}{*}{\parbox{1.3cm}{ 2-1
                {\scriptsize from \hyperref[lvv-t987]
                {LVV-T987} } } }

                & {\small Description} &
                \begin{minipage}[t]{13cm}{\scriptsize
                Identify the path to the data repository, which we will refer to as
`DATA/path', then execute the following:

                \vspace{\dp0}
                } \end{minipage} \\ \cdashline{2-3}
                & {\small Test Data} &
                \begin{minipage}[t]{13cm}{\scriptsize
                } \end{minipage} \\ \cdashline{2-3}
                & {\small Expected Result} &
                    \begin{minipage}[t]{13cm}{\scriptsize
                    Butler repo available for reading.

                    \vspace{\dp0}
                    } \end{minipage}
                \\ \hdashline


        \\ \midrule

            \multirow{3}{*}{ 3 } & Description &
            \begin{minipage}[t]{13cm}{\footnotesize
            Verify that a cross-match table between the Prompt DIASources and DRP
Objects is available.

            \vspace{\dp0}
            } \end{minipage} \\ \cline{2-3}
            & Test Data &
            \begin{minipage}[t]{13cm}{\footnotesize
                No data.
                \vspace{\dp0}
            } \end{minipage} \\ \cline{2-3}
            & Expected Result &
        \\ \midrule
    \end{longtable}

\subsection{LVV-T59 - Verify implementation of Regenerating L1 Data Products During Data
Release Processing}\label{lvv-t59}

\begin{longtable}[]{llllll}
\toprule
Version & Status & Priority & Verification Type & Owner
\\\midrule
1 & Draft & Normal &
Test & Kian-Tat Lim
\\\bottomrule
\multicolumn{6}{c}{ Open \href{https://jira.lsstcorp.org/secure/Tests.jspa\#/testCase/LVV-T59}{LVV-T59} in Jira } \\
\end{longtable}

\subsubsection{Verification Elements}
\begin{itemize}
\item \href{https://jira.lsstcorp.org/browse/LVV-156}{LVV-156} - DMS-REQ-0325-V-01: Regenerating L1 Data Products During Data Release
Processing

\end{itemize}

\subsubsection{Test Items}
Verify that the Prompt Processing data products are regenerated during
DRP.


\subsubsection{Predecessors}

\subsubsection{Environment Needs}

\paragraph{Software}

\paragraph{Hardware}

\subsubsection{Input Specification}

\subsubsection{Output Specification}

\subsubsection{Test Procedure}
    \begin{longtable}[]{p{1.3cm}p{2cm}p{13cm}}
    %\toprule
    Step & \multicolumn{2}{@{}l}{Description, Input Data and Expected Result} \\ \toprule
    \endhead

            \multirow{3}{*}{ 1 } & Description &
            \begin{minipage}[t]{13cm}{\footnotesize
            Execute DRP

            \vspace{\dp0}
            } \end{minipage} \\ \cline{2-3}
            & Test Data &
            \begin{minipage}[t]{13cm}{\footnotesize
                No data.
                \vspace{\dp0}
            } \end{minipage} \\ \cline{2-3}
            & Expected Result &
        \\ \midrule

            \multirow{3}{*}{ 2 } & Description &
            \begin{minipage}[t]{13cm}{\footnotesize
            Observe production of difference image data products

            \vspace{\dp0}
            } \end{minipage} \\ \cline{2-3}
            & Test Data &
            \begin{minipage}[t]{13cm}{\footnotesize
                No data.
                \vspace{\dp0}
            } \end{minipage} \\ \cline{2-3}
            & Expected Result &
        \\ \midrule
    \end{longtable}

\subsection{LVV-T60 - Verify implementation of Publishing predicted visit schedule}\label{lvv-t60}

\begin{longtable}[]{llllll}
\toprule
Version & Status & Priority & Verification Type & Owner
\\\midrule
1 & Draft & Normal &
Test & Eric Bellm
\\\bottomrule
\multicolumn{6}{c}{ Open \href{https://jira.lsstcorp.org/secure/Tests.jspa\#/testCase/LVV-T60}{LVV-T60} in Jira } \\
\end{longtable}

\subsubsection{Verification Elements}
\begin{itemize}
\item \href{https://jira.lsstcorp.org/browse/LVV-184}{LVV-184} - DMS-REQ-0353-V-01: Publishing predicted visit schedule

\end{itemize}

\subsubsection{Test Items}
Verify that a predict-visit schedule can be published by the OCS.


\subsubsection{Predecessors}

\subsubsection{Environment Needs}

\paragraph{Software}

\paragraph{Hardware}

\subsubsection{Input Specification}

\subsubsection{Output Specification}

\subsubsection{Test Procedure}
    \begin{longtable}[]{p{1.3cm}p{2cm}p{13cm}}
    %\toprule
    Step & \multicolumn{2}{@{}l}{Description, Input Data and Expected Result} \\ \toprule
    \endhead

            \multirow{3}{*}{ 1 } & Description &
            \begin{minipage}[t]{13cm}{\footnotesize
            
            \vspace{\dp0}
            } \end{minipage} \\ \cline{2-3}
            & Test Data &
            \begin{minipage}[t]{13cm}{\footnotesize
                No data.
                \vspace{\dp0}
            } \end{minipage} \\ \cline{2-3}
            & Expected Result &
        \\ \midrule
    \end{longtable}

\subsection{LVV-T61 - Verify implementation of Associate Sources to Objects}\label{lvv-t61}

\begin{longtable}[]{llllll}
\toprule
Version & Status & Priority & Verification Type & Owner
\\\midrule
1 & Defined & Normal &
Test & Jim Bosch
\\\bottomrule
\multicolumn{6}{c}{ Open \href{https://jira.lsstcorp.org/secure/Tests.jspa\#/testCase/LVV-T61}{LVV-T61} in Jira } \\
\end{longtable}

\subsubsection{Verification Elements}
\begin{itemize}
\item \href{https://jira.lsstcorp.org/browse/LVV-16}{LVV-16} - DMS-REQ-0034-V-01: Associate Sources to Objects

\end{itemize}

\subsubsection{Test Items}
Verify that each Source record contains an ID that associates it with a
best guess at the Object it corresponds to.


\subsubsection{Predecessors}

\subsubsection{Environment Needs}

\paragraph{Software}

\paragraph{Hardware}

\subsubsection{Input Specification}

\subsubsection{Output Specification}

\subsubsection{Test Procedure}
    \begin{longtable}[]{p{1.3cm}p{2cm}p{13cm}}
    %\toprule
    Step & \multicolumn{2}{@{}l}{Description, Input Data and Expected Result} \\ \toprule
    \endhead

                \multirow{3}{*}{\parbox{1.3cm}{ 1-1
                {\scriptsize from \hyperref[lvv-t987]
                {LVV-T987} } } }

                & {\small Description} &
                \begin{minipage}[t]{13cm}{\scriptsize
                Identify the path to the data repository, which we will refer to as
`DATA/path', then execute the following:

                \vspace{\dp0}
                } \end{minipage} \\ \cdashline{2-3}
                & {\small Test Data} &
                \begin{minipage}[t]{13cm}{\scriptsize
                } \end{minipage} \\ \cdashline{2-3}
                & {\small Expected Result} &
                    \begin{minipage}[t]{13cm}{\scriptsize
                    Butler repo available for reading.

                    \vspace{\dp0}
                    } \end{minipage}
                \\ \hdashline


        \\ \midrule

            \multirow{3}{*}{ 2 } & Description &
            \begin{minipage}[t]{13cm}{\footnotesize
            Read a dataset via the Butler and extract its source and object
catalogs.

            \vspace{\dp0}
            } \end{minipage} \\ \cline{2-3}
            & Test Data &
            \begin{minipage}[t]{13cm}{\footnotesize
                No data.
                \vspace{\dp0}
            } \end{minipage} \\ \cline{2-3}
            & Expected Result &
        \\ \midrule

            \multirow{3}{*}{ 3 } & Description &
            \begin{minipage}[t]{13cm}{\footnotesize
            Verify that sources have objects

            \vspace{\dp0}
            } \end{minipage} \\ \cline{2-3}
            & Test Data &
            \begin{minipage}[t]{13cm}{\footnotesize
                No data.
                \vspace{\dp0}
            } \end{minipage} \\ \cline{2-3}
            & Expected Result &
        \\ \midrule

            \multirow{3}{*}{ 4 } & Description &
            \begin{minipage}[t]{13cm}{\footnotesize
            Verify that objects list sources that seem reasonably near them.

            \vspace{\dp0}
            } \end{minipage} \\ \cline{2-3}
            & Test Data &
            \begin{minipage}[t]{13cm}{\footnotesize
                No data.
                \vspace{\dp0}
            } \end{minipage} \\ \cline{2-3}
            & Expected Result &
        \\ \midrule
    \end{longtable}

\subsection{LVV-T62 - Verify implementation of Provide PSF for Coadded Images}\label{lvv-t62}

\begin{longtable}[]{llllll}
\toprule
Version & Status & Priority & Verification Type & Owner
\\\midrule
2 & Approved & Normal &
Test & Jim Bosch
\\\bottomrule
\multicolumn{6}{c}{ Open \href{https://jira.lsstcorp.org/secure/Tests.jspa\#/testCase/LVV-T62}{LVV-T62} in Jira } \\
\end{longtable}

\subsubsection{Verification Elements}
\begin{itemize}
\item \href{https://jira.lsstcorp.org/browse/LVV-20}{LVV-20} - DMS-REQ-0047-V-01: Provide PSF for Coadded Images

\end{itemize}

\subsubsection{Test Items}
Verify that all coadd images produced by the DRP pipelines include a
model from which an image of the PSF at any point on the coadd can be
obtained.


\subsubsection{Predecessors}

\subsubsection{Environment Needs}

\paragraph{Software}

\paragraph{Hardware}

\subsubsection{Input Specification}
Fully covered by preconditions for
\href{https://jira.lsstcorp.org/secure/Tests.jspa\#/testCase/LVV-T16}{LVV-T16}.

\subsubsection{Output Specification}

\subsubsection{Test Procedure}
    \begin{longtable}[]{p{1.3cm}p{2cm}p{13cm}}
    %\toprule
    Step & \multicolumn{2}{@{}l}{Description, Input Data and Expected Result} \\ \toprule
    \endhead

            \multirow{3}{*}{ 1 } & Description &
            \begin{minipage}[t]{13cm}{\footnotesize
            Identify a dataset with coadded images in multiple filters.

            \vspace{\dp0}
            } \end{minipage} \\ \cline{2-3}
            & Test Data &
            \begin{minipage}[t]{13cm}{\footnotesize
                No data.
                \vspace{\dp0}
            } \end{minipage} \\ \cline{2-3}
            & Expected Result &
                \begin{minipage}[t]{13cm}{\footnotesize
                Multi-band data that has been processed through the coaddition stage.

                \vspace{\dp0}
                } \end{minipage}
        \\ \midrule

                \multirow{3}{*}{\parbox{1.3cm}{ 2-1
                {\scriptsize from \hyperref[lvv-t987]
                {LVV-T987} } } }

                & {\small Description} &
                \begin{minipage}[t]{13cm}{\scriptsize
                Identify the path to the data repository, which we will refer to as
`DATA/path', then execute the following:

                \vspace{\dp0}
                } \end{minipage} \\ \cdashline{2-3}
                & {\small Test Data} &
                \begin{minipage}[t]{13cm}{\scriptsize
                } \end{minipage} \\ \cdashline{2-3}
                & {\small Expected Result} &
                    \begin{minipage}[t]{13cm}{\scriptsize
                    Butler repo available for reading.

                    \vspace{\dp0}
                    } \end{minipage}
                \\ \hdashline


        \\ \midrule

            \multirow{3}{*}{ 3 } & Description &
            \begin{minipage}[t]{13cm}{\footnotesize
            Load the exposures, then select Objects classified as point sources on
at least 10 different coadd images (including all bands). Evaluate the
PSF model at the positions of these Objects, and verify that subtracting
a scaled version of the PSF model from the processed visit image yields
residuals consistent with pure noise.

            \vspace{\dp0}
            } \end{minipage} \\ \cline{2-3}
            & Test Data &
            \begin{minipage}[t]{13cm}{\footnotesize
                No data.
                \vspace{\dp0}
            } \end{minipage} \\ \cline{2-3}
            & Expected Result &
                \begin{minipage}[t]{13cm}{\footnotesize
                Images with the PSF model subtracted, leaving only residuals that are
consistent with being noise.

                \vspace{\dp0}
                } \end{minipage}
        \\ \midrule
    \end{longtable}

\subsection{LVV-T63 - Verify implementation of Produce Images for EPO}\label{lvv-t63}

\begin{longtable}[]{llllll}
\toprule
Version & Status & Priority & Verification Type & Owner
\\\midrule
1 & Draft & Normal &
Test & Gregory Dubois-Felsmann
\\\bottomrule
\multicolumn{6}{c}{ Open \href{https://jira.lsstcorp.org/secure/Tests.jspa\#/testCase/LVV-T63}{LVV-T63} in Jira } \\
\end{longtable}

\subsubsection{Verification Elements}
\begin{itemize}
\item \href{https://jira.lsstcorp.org/browse/LVV-45}{LVV-45} - DMS-REQ-0103-V-01: Produce Images for EPO

\end{itemize}

\subsubsection{Test Items}
This test will verify that the DRP pipelines produce the image data
products called out in \citeds{LSE-131}. ~Currently this is limited to a color
all-sky HiPS map. ~This will be verified (1) by inspection of pipeline
configurations and (2) in operations rehearsals on precursor data. ~The
production of a usable HiPS map will be verified by browsing it with
community tools.


\subsubsection{Predecessors}

\subsubsection{Environment Needs}

\paragraph{Software}

\paragraph{Hardware}

\subsubsection{Input Specification}
In order for an operational test to be successful, as a precondition the
inputs to that production must exist. ~For the only currently mandated
image data production in \citeds{LSE-131}, a color all-sky HiPS map down to 1
arcsecond resolution, the prerequisite inputs to that are the
single-filter coadds in the bands required by the yet-to-be-specified
color prescription.

\subsubsection{Output Specification}

\subsubsection{Test Procedure}
    \begin{longtable}[]{p{1.3cm}p{2cm}p{13cm}}
    %\toprule
    Step & \multicolumn{2}{@{}l}{Description, Input Data and Expected Result} \\ \toprule
    \endhead

                \multirow{3}{*}{\parbox{1.3cm}{ 1-1
                {\scriptsize from \hyperref[lvv-t987]
                {LVV-T987} } } }

                & {\small Description} &
                \begin{minipage}[t]{13cm}{\scriptsize
                Identify the path to the data repository, which we will refer to as
`DATA/path', then execute the following:

                \vspace{\dp0}
                } \end{minipage} \\ \cdashline{2-3}
                & {\small Test Data} &
                \begin{minipage}[t]{13cm}{\scriptsize
                } \end{minipage} \\ \cdashline{2-3}
                & {\small Expected Result} &
                    \begin{minipage}[t]{13cm}{\scriptsize
                    Butler repo available for reading.

                    \vspace{\dp0}
                    } \end{minipage}
                \\ \hdashline


        \\ \midrule

            \multirow{3}{*}{ 2 } & Description &
            \begin{minipage}[t]{13cm}{\footnotesize
            For each of the expected data product types needed for creation of HiPS
images, retrieve the data product from the Butler and verify it to be
non-empty.

            \vspace{\dp0}
            } \end{minipage} \\ \cline{2-3}
            & Test Data &
            \begin{minipage}[t]{13cm}{\footnotesize
                No data.
                \vspace{\dp0}
            } \end{minipage} \\ \cline{2-3}
            & Expected Result &
        \\ \midrule

            \multirow{3}{*}{ 3 } & Description &
            \begin{minipage}[t]{13cm}{\footnotesize
            Verify that a HiPS image map covering the LSST survey area, with a
limiting depth yielding 1 arcsecond resolution, has been produced
matching the color prescriptions provided by EPO (in updates to LSE-131
which are expected to be made ``once ComCam data is available'').

            \vspace{\dp0}
            } \end{minipage} \\ \cline{2-3}
            & Test Data &
            \begin{minipage}[t]{13cm}{\footnotesize
                No data.
                \vspace{\dp0}
            } \end{minipage} \\ \cline{2-3}
            & Expected Result &
        \\ \midrule

            \multirow{3}{*}{ 4 } & Description &
            \begin{minipage}[t]{13cm}{\footnotesize
            Place the image map in a location accessible to a Firefly and an Aladin
Lite client, ideally with the client running in the EPO data systems
environment.

            \vspace{\dp0}
            } \end{minipage} \\ \cline{2-3}
            & Test Data &
            \begin{minipage}[t]{13cm}{\footnotesize
                No data.
                \vspace{\dp0}
            } \end{minipage} \\ \cline{2-3}
            & Expected Result &
        \\ \midrule

            \multirow{3}{*}{ 5 } & Description &
            \begin{minipage}[t]{13cm}{\footnotesize
            Use Firefly to manually explore the image map at the largest scales to
verify coverage of the entire sky. ~Sample in various locations to
confirm the 1 arcsecond maximum depth.\\
Confirm using Aladin Lite that the format of the image map is supported
by this common community tool.

            \vspace{\dp0}
            } \end{minipage} \\ \cline{2-3}
            & Test Data &
            \begin{minipage}[t]{13cm}{\footnotesize
                No data.
                \vspace{\dp0}
            } \end{minipage} \\ \cline{2-3}
            & Expected Result &
        \\ \midrule

            \multirow{3}{*}{ 6 } & Description &
            \begin{minipage}[t]{13cm}{\footnotesize
            Verify programmatically, perhaps both by sampling a variety of
locations, and by counting the tiles created at the
1-arcsecond-resolution depth, that the map is complete and meets its
specifications.

            \vspace{\dp0}
            } \end{minipage} \\ \cline{2-3}
            & Test Data &
            \begin{minipage}[t]{13cm}{\footnotesize
                No data.
                \vspace{\dp0}
            } \end{minipage} \\ \cline{2-3}
            & Expected Result &
        \\ \midrule

            \multirow{3}{*}{ 7 } & Description &
            \begin{minipage}[t]{13cm}{\footnotesize
            Apply an IVOA-community HiPS service validation tool, if available, to
the service location.

            \vspace{\dp0}
            } \end{minipage} \\ \cline{2-3}
            & Test Data &
            \begin{minipage}[t]{13cm}{\footnotesize
                No data.
                \vspace{\dp0}
            } \end{minipage} \\ \cline{2-3}
            & Expected Result &
        \\ \midrule

            \multirow{3}{*}{ 8 } & Description &
            \begin{minipage}[t]{13cm}{\footnotesize
            Verify that the HiPS map created is in a location accessible to the EPO
data systems.

            \vspace{\dp0}
            } \end{minipage} \\ \cline{2-3}
            & Test Data &
            \begin{minipage}[t]{13cm}{\footnotesize
                No data.
                \vspace{\dp0}
            } \end{minipage} \\ \cline{2-3}
            & Expected Result &
        \\ \midrule
    \end{longtable}

\subsection{LVV-T64 - Verify implementation of Coadded Image Provenance}\label{lvv-t64}

\begin{longtable}[]{llllll}
\toprule
Version & Status & Priority & Verification Type & Owner
\\\midrule
1 & Draft & Normal &
Test & Jim Bosch
\\\bottomrule
\multicolumn{6}{c}{ Open \href{https://jira.lsstcorp.org/secure/Tests.jspa\#/testCase/LVV-T64}{LVV-T64} in Jira } \\
\end{longtable}

\subsubsection{Verification Elements}
\begin{itemize}
\item \href{https://jira.lsstcorp.org/browse/LVV-46}{LVV-46} - DMS-REQ-0106-V-01: Coadded Image Provenance

\item \href{https://jira.lsstcorp.org/browse/LVV-1234}{LVV-1234} - OSS-REQ-0122-V-01: Provenance

\end{itemize}

\subsubsection{Test Items}
Verify that all coadd data products produced by the DRP pipelines are
associated with provenance information that includes the set of input
epochs contributing to that coadd as well as any additional information
needed to exactly produce that coadd.


\subsubsection{Predecessors}

\subsubsection{Environment Needs}

\paragraph{Software}

\paragraph{Hardware}

\subsubsection{Input Specification}

\subsubsection{Output Specification}

\subsubsection{Test Procedure}
    \begin{longtable}[]{p{1.3cm}p{2cm}p{13cm}}
    %\toprule
    Step & \multicolumn{2}{@{}l}{Description, Input Data and Expected Result} \\ \toprule
    \endhead

                \multirow{3}{*}{\parbox{1.3cm}{ 1-1
                {\scriptsize from \hyperref[lvv-t860]
                {LVV-T860} } } }

                & {\small Description} &
                \begin{minipage}[t]{13cm}{\scriptsize
                The `path` that you will use depends on where you are running the
science pipelines. Options:\\[2\baselineskip]

\begin{itemize}
\tightlist
\item
  local (newinstall.sh - based
  install):{[}path\_to\_installation{]}/loadLSST.bash
\item
  development cluster (``lsst-dev''):
  /software/lsstsw/stack/loadLSST.bash
\item
  LSP Notebook aspect (from a terminal):
  /opt/lsst/software/stack/loadLSST.bash
\end{itemize}

From the command line, execute the commands below in the example
code:\\[2\baselineskip]

                \vspace{\dp0}
                } \end{minipage} \\ \cdashline{2-3}
                & {\small Test Data} &
                \begin{minipage}[t]{13cm}{\scriptsize
                } \end{minipage} \\ \cdashline{2-3}
                & {\small Expected Result} &
                    \begin{minipage}[t]{13cm}{\scriptsize
                    Science pipeline software is available for use. If additional packages
are needed (for example, `obs' packages such as `obs\_subaru`), then
additional `setup` commands will be necessary.\\[2\baselineskip]To check
versions in use, type:\\
eups list -s

                    \vspace{\dp0}
                    } \end{minipage}
                \\ \hdashline


        \\ \midrule

                \multirow{3}{*}{\parbox{1.3cm}{ 2-1
                {\scriptsize from \hyperref[lvv-t987]
                {LVV-T987} } } }

                & {\small Description} &
                \begin{minipage}[t]{13cm}{\scriptsize
                Identify the path to the data repository, which we will refer to as
`DATA/path', then execute the following:

                \vspace{\dp0}
                } \end{minipage} \\ \cdashline{2-3}
                & {\small Test Data} &
                \begin{minipage}[t]{13cm}{\scriptsize
                } \end{minipage} \\ \cdashline{2-3}
                & {\small Expected Result} &
                    \begin{minipage}[t]{13cm}{\scriptsize
                    Butler repo available for reading.

                    \vspace{\dp0}
                    } \end{minipage}
                \\ \hdashline


        \\ \midrule

            \multirow{3}{*}{ 3 } & Description &
            \begin{minipage}[t]{13cm}{\footnotesize
            For each of the expected data product types and each of the expected
units (PVIs, coadds, etc), retrieve the data product from the Butler and
verify it to be non-empty.

            \vspace{\dp0}
            } \end{minipage} \\ \cline{2-3}
            & Test Data &
            \begin{minipage}[t]{13cm}{\footnotesize
                No data.
                \vspace{\dp0}
            } \end{minipage} \\ \cline{2-3}
            & Expected Result &
        \\ \midrule

            \multirow{3}{*}{ 4 } & Description &
            \begin{minipage}[t]{13cm}{\footnotesize
            Query and verify provenance of input images, and software versions that
went into producing stack.

            \vspace{\dp0}
            } \end{minipage} \\ \cline{2-3}
            & Test Data &
            \begin{minipage}[t]{13cm}{\footnotesize
                No data.
                \vspace{\dp0}
            } \end{minipage} \\ \cline{2-3}
            & Expected Result &
        \\ \midrule

            \multirow{3}{*}{ 5 } & Description &
            \begin{minipage}[t]{13cm}{\footnotesize
            Test re-generating 10 different coadds tract+patches based on the
provenance image given

            \vspace{\dp0}
            } \end{minipage} \\ \cline{2-3}
            & Test Data &
            \begin{minipage}[t]{13cm}{\footnotesize
                No data.
                \vspace{\dp0}
            } \end{minipage} \\ \cline{2-3}
            & Expected Result &
        \\ \midrule
    \end{longtable}

\subsection{LVV-T65 - Verify implementation of Source Catalog}\label{lvv-t65}

\begin{longtable}[]{llllll}
\toprule
Version & Status & Priority & Verification Type & Owner
\\\midrule
1 & Defined & Normal &
Test & Jim Bosch
\\\bottomrule
\multicolumn{6}{c}{ Open \href{https://jira.lsstcorp.org/secure/Tests.jspa\#/testCase/LVV-T65}{LVV-T65} in Jira } \\
\end{longtable}

\subsubsection{Verification Elements}
\begin{itemize}
\item \href{https://jira.lsstcorp.org/browse/LVV-98}{LVV-98} - DMS-REQ-0267-V-01: Source Catalog

\end{itemize}

\subsubsection{Test Items}
Verify that all Sources produced by the DRP pipelines contain the
entries listed in DMS-REQ-0267.


\subsubsection{Predecessors}

\subsubsection{Environment Needs}

\paragraph{Software}

\paragraph{Hardware}

\subsubsection{Input Specification}

\subsubsection{Output Specification}

\subsubsection{Test Procedure}
    \begin{longtable}[]{p{1.3cm}p{2cm}p{13cm}}
    %\toprule
    Step & \multicolumn{2}{@{}l}{Description, Input Data and Expected Result} \\ \toprule
    \endhead

            \multirow{3}{*}{ 1 } & Description &
            \begin{minipage}[t]{13cm}{\footnotesize
            Identify a suitable small dataset to process through the DRP.

            \vspace{\dp0}
            } \end{minipage} \\ \cline{2-3}
            & Test Data &
            \begin{minipage}[t]{13cm}{\footnotesize
                No data.
                \vspace{\dp0}
            } \end{minipage} \\ \cline{2-3}
            & Expected Result &
        \\ \midrule

                \multirow{3}{*}{\parbox{1.3cm}{ 2-1
                {\scriptsize from \hyperref[lvv-t1064]
                {LVV-T1064} } } }

                & {\small Description} &
                \begin{minipage}[t]{13cm}{\scriptsize
                Process data with the Data Release Production payload, starting from raw
science images and generating science data products, placing them in the
Data Backbone.

                \vspace{\dp0}
                } \end{minipage} \\ \cdashline{2-3}
                & {\small Test Data} &
                \begin{minipage}[t]{13cm}{\scriptsize
                } \end{minipage} \\ \cdashline{2-3}
                & {\small Expected Result} &
                \\ \hdashline


        \\ \midrule

            \multirow{3}{*}{ 3 } & Description &
            \begin{minipage}[t]{13cm}{\footnotesize
            Confirm that source catalogs have been produced for single visits and
coadds, and that it contains the required measurements.

            \vspace{\dp0}
            } \end{minipage} \\ \cline{2-3}
            & Test Data &
            \begin{minipage}[t]{13cm}{\footnotesize
                No data.
                \vspace{\dp0}
            } \end{minipage} \\ \cline{2-3}
            & Expected Result &
                \begin{minipage}[t]{13cm}{\footnotesize
                A source catalog containing the measured attributes (and associated
errors), including location on the focal plane; a static point-source
model fit to world coordinates and flux; a centroid and adaptive
moments; and surface brightnesses through elliptical multiple apertures
that are concentric, PSF-homogenized, and logarithmically spaced in
intensity.

                \vspace{\dp0}
                } \end{minipage}
        \\ \midrule
    \end{longtable}

\subsection{LVV-T66 - Verify implementation of Forced-Source Catalog}\label{lvv-t66}

\begin{longtable}[]{llllll}
\toprule
Version & Status & Priority & Verification Type & Owner
\\\midrule
1 & Draft & Normal &
Test & Jim Bosch
\\\bottomrule
\multicolumn{6}{c}{ Open \href{https://jira.lsstcorp.org/secure/Tests.jspa\#/testCase/LVV-T66}{LVV-T66} in Jira } \\
\end{longtable}

\subsubsection{Verification Elements}
\begin{itemize}
\item \href{https://jira.lsstcorp.org/browse/LVV-99}{LVV-99} - DMS-REQ-0268-V-01: Forced-Source Catalog

\end{itemize}

\subsubsection{Test Items}
Verify that all ForcedSources produced by the DRP pipelines contain
fluxes measured on difference and direct single-epoch images, associated
uncertainties, an Object ID, and a Visit ID.


\subsubsection{Predecessors}

\subsubsection{Environment Needs}

\paragraph{Software}

\paragraph{Hardware}

\subsubsection{Input Specification}

\subsubsection{Output Specification}

\subsubsection{Test Procedure}
    \begin{longtable}[]{p{1.3cm}p{2cm}p{13cm}}
    %\toprule
    Step & \multicolumn{2}{@{}l}{Description, Input Data and Expected Result} \\ \toprule
    \endhead

                \multirow{3}{*}{\parbox{1.3cm}{ 1-1
                {\scriptsize from \hyperref[lvv-t987]
                {LVV-T987} } } }

                & {\small Description} &
                \begin{minipage}[t]{13cm}{\scriptsize
                Identify the path to the data repository, which we will refer to as
`DATA/path', then execute the following:

                \vspace{\dp0}
                } \end{minipage} \\ \cdashline{2-3}
                & {\small Test Data} &
                \begin{minipage}[t]{13cm}{\scriptsize
                } \end{minipage} \\ \cdashline{2-3}
                & {\small Expected Result} &
                    \begin{minipage}[t]{13cm}{\scriptsize
                    Butler repo available for reading.

                    \vspace{\dp0}
                    } \end{minipage}
                \\ \hdashline


        \\ \midrule

            \multirow{3}{*}{ 2 } & Description &
            \begin{minipage}[t]{13cm}{\footnotesize
            Retrieve the forced-source catalog from the Butler and verify it to be
non-empty.

            \vspace{\dp0}
            } \end{minipage} \\ \cline{2-3}
            & Test Data &
            \begin{minipage}[t]{13cm}{\footnotesize
                No data.
                \vspace{\dp0}
            } \end{minipage} \\ \cline{2-3}
            & Expected Result &
        \\ \midrule

            \multirow{3}{*}{ 3 } & Description &
            \begin{minipage}[t]{13cm}{\footnotesize
            Verify that there exist entries in the forced-photometry table for all
coadd objects for the PVIs on which the object should appear.

            \vspace{\dp0}
            } \end{minipage} \\ \cline{2-3}
            & Test Data &
            \begin{minipage}[t]{13cm}{\footnotesize
                No data.
                \vspace{\dp0}
            } \end{minipage} \\ \cline{2-3}
            & Expected Result &
        \\ \midrule

            \multirow{3}{*}{ 4 } & Description &
            \begin{minipage}[t]{13cm}{\footnotesize
            Verify that there exist entries in a forced-photometry table for each
image for all DIAObjects.\\[2\baselineskip]

            \vspace{\dp0}
            } \end{minipage} \\ \cline{2-3}
            & Test Data &
            \begin{minipage}[t]{13cm}{\footnotesize
                No data.
                \vspace{\dp0}
            } \end{minipage} \\ \cline{2-3}
            & Expected Result &
        \\ \midrule
    \end{longtable}

\subsection{LVV-T67 - Verify implementation of Object Catalog}\label{lvv-t67}

\begin{longtable}[]{llllll}
\toprule
Version & Status & Priority & Verification Type & Owner
\\\midrule
1 & Draft & Normal &
Test & Jim Bosch
\\\bottomrule
\multicolumn{6}{c}{ Open \href{https://jira.lsstcorp.org/secure/Tests.jspa\#/testCase/LVV-T67}{LVV-T67} in Jira } \\
\end{longtable}

\subsubsection{Verification Elements}
\begin{itemize}
\item \href{https://jira.lsstcorp.org/browse/LVV-106}{LVV-106} - DMS-REQ-0275-V-01: Object Catalog

\end{itemize}

\subsubsection{Test Items}
Verify that the DRP pipelines produce an Object catalog derived from
detections made on both coadded images and difference images and
measurements performed on coadds and possibly overlapping single-epoch
images.


\subsubsection{Predecessors}

\subsubsection{Environment Needs}

\paragraph{Software}

\paragraph{Hardware}

\subsubsection{Input Specification}
Input Data\\[2\baselineskip]DECam HiTS data (raw science images and
master calibrations)\\
HSC ``RC2'' data (raw science images and master calibrations)

\subsubsection{Output Specification}

\subsubsection{Test Procedure}
    \begin{longtable}[]{p{1.3cm}p{2cm}p{13cm}}
    %\toprule
    Step & \multicolumn{2}{@{}l}{Description, Input Data and Expected Result} \\ \toprule
    \endhead

            \multirow{3}{*}{ 1 } & Description &
            \begin{minipage}[t]{13cm}{\footnotesize
            load LSST DM Stack

            \vspace{\dp0}
            } \end{minipage} \\ \cline{2-3}
            & Test Data &
            \begin{minipage}[t]{13cm}{\footnotesize
                No data.
                \vspace{\dp0}
            } \end{minipage} \\ \cline{2-3}
            & Expected Result &
        \\ \midrule

            \multirow{3}{*}{ 2 } & Description &
            \begin{minipage}[t]{13cm}{\footnotesize
            Run the single-frame processing and self-calibration steps of the DRP
pipeline.~

            \vspace{\dp0}
            } \end{minipage} \\ \cline{2-3}
            & Test Data &
            \begin{minipage}[t]{13cm}{\footnotesize
                No data.
                \vspace{\dp0}
            } \end{minipage} \\ \cline{2-3}
            & Expected Result &
        \\ \midrule

            \multirow{3}{*}{ 3 } & Description &
            \begin{minipage}[t]{13cm}{\footnotesize
            Insert simulated sources into all single-frame images, including:

\begin{itemize}
\tightlist
\item
  static objects (e.g. galaxies), including some too faint to be
  detectable in single-epoch images;
\item
  objects with static positions that are sufficiently bright and
  variable that they should be detectable in single-epoch difference
  images;
\item
  transient objects that appear in only a few epochs;
\item
  stars with significant proper motions and parallaxes, some below the
  single-epoch detection limit
\item
  simulated solar system objects with orbits that can be constrained
  from just the epochs in the test dataset
\end{itemize}

            \vspace{\dp0}
            } \end{minipage} \\ \cline{2-3}
            & Test Data &
            \begin{minipage}[t]{13cm}{\footnotesize
                No data.
                \vspace{\dp0}
            } \end{minipage} \\ \cline{2-3}
            & Expected Result &
        \\ \midrule

            \multirow{3}{*}{ 4 } & Description &
            \begin{minipage}[t]{13cm}{\footnotesize
            Run all remaining DRP pipeline steps.

            \vspace{\dp0}
            } \end{minipage} \\ \cline{2-3}
            & Test Data &
            \begin{minipage}[t]{13cm}{\footnotesize
                No data.
                \vspace{\dp0}
            } \end{minipage} \\ \cline{2-3}
            & Expected Result &
        \\ \midrule

            \multirow{3}{*}{ 5 } & Description &
            \begin{minipage}[t]{13cm}{\footnotesize
            Load data into DRP database

            \vspace{\dp0}
            } \end{minipage} \\ \cline{2-3}
            & Test Data &
            \begin{minipage}[t]{13cm}{\footnotesize
                No data.
                \vspace{\dp0}
            } \end{minipage} \\ \cline{2-3}
            & Expected Result &
        \\ \midrule

            \multirow{3}{*}{ 6 } & Description &
            \begin{minipage}[t]{13cm}{\footnotesize
            Verify that the injected simulated objects are recovered at a rate
consistent with their S/N \emph{when not blended with each other or real
objects}, and that flags indicating how each Object was detected are
consistent with their properties:

\begin{itemize}
\tightlist
\item
  static objects should be detected in coadds only (not difference
  images)
\item
  static-position/variable-flux objects should be detected in coadds and
  possibly difference images
\item
  transient objects should be detected in difference images only
\item
  stars with significant proper motions may be detected in either coadds
  or difference images
\item
  solar system objects should be detected in difference images only.
\end{itemize}

            \vspace{\dp0}
            } \end{minipage} \\ \cline{2-3}
            & Test Data &
            \begin{minipage}[t]{13cm}{\footnotesize
                No data.
                \vspace{\dp0}
            } \end{minipage} \\ \cline{2-3}
            & Expected Result &
        \\ \midrule
    \end{longtable}

\subsection{LVV-T68 - Verify implementation of Provide Photometric Redshifts of Galaxies}\label{lvv-t68}

\begin{longtable}[]{llllll}
\toprule
Version & Status & Priority & Verification Type & Owner
\\\midrule
1 & Draft & Normal &
Test & Jim Bosch
\\\bottomrule
\multicolumn{6}{c}{ Open \href{https://jira.lsstcorp.org/secure/Tests.jspa\#/testCase/LVV-T68}{LVV-T68} in Jira } \\
\end{longtable}

\subsubsection{Verification Elements}
\begin{itemize}
\item \href{https://jira.lsstcorp.org/browse/LVV-19}{LVV-19} - DMS-REQ-0046-V-01: Provide Photometric Redshifts of Galaxies

\end{itemize}

\subsubsection{Test Items}
Verify that Object catalogs produced by the DRP Pipeline include
photometric redshift information.


\subsubsection{Predecessors}

\subsubsection{Environment Needs}

\paragraph{Software}

\paragraph{Hardware}

\subsubsection{Input Specification}
Input Data\\
HSC Public Data Release (raw science images, master calibrations)\\
Assorted public spectroscopic catalogs and high-accuracy photometric
redshift catalogs in the HSC PDR footprint.

\subsubsection{Output Specification}

\subsubsection{Test Procedure}
    \begin{longtable}[]{p{1.3cm}p{2cm}p{13cm}}
    %\toprule
    Step & \multicolumn{2}{@{}l}{Description, Input Data and Expected Result} \\ \toprule
    \endhead

            \multirow{3}{*}{ 1 } & Description &
            \begin{minipage}[t]{13cm}{\footnotesize
            Run DRP processing steps through (at least) final galaxy photometry
measurements.

            \vspace{\dp0}
            } \end{minipage} \\ \cline{2-3}
            & Test Data &
            \begin{minipage}[t]{13cm}{\footnotesize
                No data.
                \vspace{\dp0}
            } \end{minipage} \\ \cline{2-3}
            & Expected Result &
        \\ \midrule

            \multirow{3}{*}{ 2 } & Description &
            \begin{minipage}[t]{13cm}{\footnotesize
            Train photometric redshift algorithm(s) on spectroscopic and
high-accuracy photometric redshift catalogs.

            \vspace{\dp0}
            } \end{minipage} \\ \cline{2-3}
            & Test Data &
            \begin{minipage}[t]{13cm}{\footnotesize
                No data.
                \vspace{\dp0}
            } \end{minipage} \\ \cline{2-3}
            & Expected Result &
        \\ \midrule

            \multirow{3}{*}{ 3 } & Description &
            \begin{minipage}[t]{13cm}{\footnotesize
            Estimate photometric redshifts for all Objects generated by DRP
processing.

            \vspace{\dp0}
            } \end{minipage} \\ \cline{2-3}
            & Test Data &
            \begin{minipage}[t]{13cm}{\footnotesize
                No data.
                \vspace{\dp0}
            } \end{minipage} \\ \cline{2-3}
            & Expected Result &
        \\ \midrule

            \multirow{3}{*}{ 4 } & Description &
            \begin{minipage}[t]{13cm}{\footnotesize
            Load into DRP Database

            \vspace{\dp0}
            } \end{minipage} \\ \cline{2-3}
            & Test Data &
            \begin{minipage}[t]{13cm}{\footnotesize
                No data.
                \vspace{\dp0}
            } \end{minipage} \\ \cline{2-3}
            & Expected Result &
        \\ \midrule

            \multirow{3}{*}{ 5 } & Description &
            \begin{minipage}[t]{13cm}{\footnotesize
            Inspect database to verify that photometric redshifts are present for
all objects

            \vspace{\dp0}
            } \end{minipage} \\ \cline{2-3}
            & Test Data &
            \begin{minipage}[t]{13cm}{\footnotesize
                No data.
                \vspace{\dp0}
            } \end{minipage} \\ \cline{2-3}
            & Expected Result &
        \\ \midrule
    \end{longtable}

\subsection{LVV-T69 - Verify implementation of Object Characterization}\label{lvv-t69}

\begin{longtable}[]{llllll}
\toprule
Version & Status & Priority & Verification Type & Owner
\\\midrule
1 & Draft & Normal &
Test & Jim Bosch
\\\bottomrule
\multicolumn{6}{c}{ Open \href{https://jira.lsstcorp.org/secure/Tests.jspa\#/testCase/LVV-T69}{LVV-T69} in Jira } \\
\end{longtable}

\subsubsection{Verification Elements}
\begin{itemize}
\item \href{https://jira.lsstcorp.org/browse/LVV-107}{LVV-107} - DMS-REQ-0276-V-01: Object Characterization

\end{itemize}

\subsubsection{Test Items}
Verify that Object catalogs produced by the DRP pipeline include all
measurements listed in DMS-REQ-0276: a point-source model fit, a
bulge-disk model fit, standard colors, a centroid, adap- tive moments,
Petrosian and Kron fluxes, surface brightness at multiple apertures,
proper motion and parallax, and a variability characterization.


\subsubsection{Predecessors}

\subsubsection{Environment Needs}

\paragraph{Software}

\paragraph{Hardware}

\subsubsection{Input Specification}

\subsubsection{Output Specification}

\subsubsection{Test Procedure}
    \begin{longtable}[]{p{1.3cm}p{2cm}p{13cm}}
    %\toprule
    Step & \multicolumn{2}{@{}l}{Description, Input Data and Expected Result} \\ \toprule
    \endhead

            \multirow{3}{*}{ 1 } & Description &
            \begin{minipage}[t]{13cm}{\footnotesize
            Precursor data, execute DRP, load results, observe catalog contents

            \vspace{\dp0}
            } \end{minipage} \\ \cline{2-3}
            & Test Data &
            \begin{minipage}[t]{13cm}{\footnotesize
                No data.
                \vspace{\dp0}
            } \end{minipage} \\ \cline{2-3}
            & Expected Result &
        \\ \midrule
    \end{longtable}

\subsection{LVV-T71 - Verify implementation of Detecting extended low surface brightness
objects}\label{lvv-t71}

\begin{longtable}[]{llllll}
\toprule
Version & Status & Priority & Verification Type & Owner
\\\midrule
1 & Draft & Normal &
Test & Jim Bosch
\\\bottomrule
\multicolumn{6}{c}{ Open \href{https://jira.lsstcorp.org/secure/Tests.jspa\#/testCase/LVV-T71}{LVV-T71} in Jira } \\
\end{longtable}

\subsubsection{Verification Elements}
\begin{itemize}
\item \href{https://jira.lsstcorp.org/browse/LVV-180}{LVV-180} - DMS-REQ-0349-V-01: Detecting extended low surface brightness objects

\end{itemize}

\subsubsection{Test Items}
Verify that low-surface brightness objects (including those whose PSF
S/N is lower than the detection threshold) are detected in coadds.


\subsubsection{Predecessors}

\subsubsection{Environment Needs}

\paragraph{Software}

\paragraph{Hardware}

\subsubsection{Input Specification}
Input Data\\
HSC ``RC2'' data (raw science images and master calibrations)

\subsubsection{Output Specification}

\subsubsection{Test Procedure}
    \begin{longtable}[]{p{1.3cm}p{2cm}p{13cm}}
    %\toprule
    Step & \multicolumn{2}{@{}l}{Description, Input Data and Expected Result} \\ \toprule
    \endhead

            \multirow{3}{*}{ 1 } & Description &
            \begin{minipage}[t]{13cm}{\footnotesize
            load LSST DM Stack

            \vspace{\dp0}
            } \end{minipage} \\ \cline{2-3}
            & Test Data &
            \begin{minipage}[t]{13cm}{\footnotesize
                No data.
                \vspace{\dp0}
            } \end{minipage} \\ \cline{2-3}
            & Expected Result &
        \\ \midrule

            \multirow{3}{*}{ 2 } & Description &
            \begin{minipage}[t]{13cm}{\footnotesize
            Run the single-frame processing and self-calibration steps of the DRP
pipeline.

            \vspace{\dp0}
            } \end{minipage} \\ \cline{2-3}
            & Test Data &
            \begin{minipage}[t]{13cm}{\footnotesize
                No data.
                \vspace{\dp0}
            } \end{minipage} \\ \cline{2-3}
            & Expected Result &
        \\ \midrule

            \multirow{3}{*}{ 3 } & Description &
            \begin{minipage}[t]{13cm}{\footnotesize
            Insert simulated low-surface-brightness galaxies (with exponential
profiles) consistently into all calibrated single-epoch images.

            \vspace{\dp0}
            } \end{minipage} \\ \cline{2-3}
            & Test Data &
            \begin{minipage}[t]{13cm}{\footnotesize
                No data.
                \vspace{\dp0}
            } \end{minipage} \\ \cline{2-3}
            & Expected Result &
        \\ \midrule

            \multirow{3}{*}{ 4 } & Description &
            \begin{minipage}[t]{13cm}{\footnotesize
            Run all remaining DRP pipeline steps.

            \vspace{\dp0}
            } \end{minipage} \\ \cline{2-3}
            & Test Data &
            \begin{minipage}[t]{13cm}{\footnotesize
                No data.
                \vspace{\dp0}
            } \end{minipage} \\ \cline{2-3}
            & Expected Result &
        \\ \midrule

            \multirow{3}{*}{ 5 } & Description &
            \begin{minipage}[t]{13cm}{\footnotesize
            ​​​​Load data into DRP database

            \vspace{\dp0}
            } \end{minipage} \\ \cline{2-3}
            & Test Data &
            \begin{minipage}[t]{13cm}{\footnotesize
                No data.
                \vspace{\dp0}
            } \end{minipage} \\ \cline{2-3}
            & Expected Result &
        \\ \midrule

            \multirow{3}{*}{ 6 } & Description &
            \begin{minipage}[t]{13cm}{\footnotesize
            Verify that the injected simulated objects are recovered at a rate
consistent with their S/N and true profile \emph{when not blended with
each other or real objects.}

            \vspace{\dp0}
            } \end{minipage} \\ \cline{2-3}
            & Test Data &
            \begin{minipage}[t]{13cm}{\footnotesize
                No data.
                \vspace{\dp0}
            } \end{minipage} \\ \cline{2-3}
            & Expected Result &
        \\ \midrule
    \end{longtable}

\subsection{LVV-T72 - Verify implementation of Coadd Image Method Constraints}\label{lvv-t72}

\begin{longtable}[]{llllll}
\toprule
Version & Status & Priority & Verification Type & Owner
\\\midrule
1 & Draft & Normal &
Test & Jim Bosch
\\\bottomrule
\multicolumn{6}{c}{ Open \href{https://jira.lsstcorp.org/secure/Tests.jspa\#/testCase/LVV-T72}{LVV-T72} in Jira } \\
\end{longtable}

\subsubsection{Verification Elements}
\begin{itemize}
\item \href{https://jira.lsstcorp.org/browse/LVV-109}{LVV-109} - DMS-REQ-0278-V-01: Coadd Image Method Constraints

\end{itemize}

\subsubsection{Test Items}
Verify the implementation of how Coadd images are created.


\subsubsection{Predecessors}

\subsubsection{Environment Needs}

\paragraph{Software}

\paragraph{Hardware}

\subsubsection{Input Specification}

\subsubsection{Output Specification}

\subsubsection{Test Procedure}
    \begin{longtable}[]{p{1.3cm}p{2cm}p{13cm}}
    %\toprule
    Step & \multicolumn{2}{@{}l}{Description, Input Data and Expected Result} \\ \toprule
    \endhead

            \multirow{3}{*}{ 1 } & Description &
            \begin{minipage}[t]{13cm}{\footnotesize
            Identify a dataset that has been processed to create coadd images.

            \vspace{\dp0}
            } \end{minipage} \\ \cline{2-3}
            & Test Data &
            \begin{minipage}[t]{13cm}{\footnotesize
                No data.
                \vspace{\dp0}
            } \end{minipage} \\ \cline{2-3}
            & Expected Result &
        \\ \midrule

                \multirow{3}{*}{\parbox{1.3cm}{ 2-1
                {\scriptsize from \hyperref[lvv-t987]
                {LVV-T987} } } }

                & {\small Description} &
                \begin{minipage}[t]{13cm}{\scriptsize
                Identify the path to the data repository, which we will refer to as
`DATA/path', then execute the following:

                \vspace{\dp0}
                } \end{minipage} \\ \cdashline{2-3}
                & {\small Test Data} &
                \begin{minipage}[t]{13cm}{\scriptsize
                } \end{minipage} \\ \cdashline{2-3}
                & {\small Expected Result} &
                    \begin{minipage}[t]{13cm}{\scriptsize
                    Butler repo available for reading.

                    \vspace{\dp0}
                    } \end{minipage}
                \\ \hdashline


        \\ \midrule

            \multirow{3}{*}{ 3 } & Description &
            \begin{minipage}[t]{13cm}{\footnotesize
            Retrieve the coadds in the dataset and verify that they are non-empty.

            \vspace{\dp0}
            } \end{minipage} \\ \cline{2-3}
            & Test Data &
            \begin{minipage}[t]{13cm}{\footnotesize
                No data.
                \vspace{\dp0}
            } \end{minipage} \\ \cline{2-3}
            & Expected Result &
        \\ \midrule

            \multirow{3}{*}{ 4 } & Description &
            \begin{minipage}[t]{13cm}{\footnotesize
            Verify that coadds were created following specification

            \vspace{\dp0}
            } \end{minipage} \\ \cline{2-3}
            & Test Data &
            \begin{minipage}[t]{13cm}{\footnotesize
                No data.
                \vspace{\dp0}
            } \end{minipage} \\ \cline{2-3}
            & Expected Result &
        \\ \midrule
    \end{longtable}

\subsection{LVV-T73 - Verify implementation of Deep Detection Coadds}\label{lvv-t73}

\begin{longtable}[]{llllll}
\toprule
Version & Status & Priority & Verification Type & Owner
\\\midrule
1 & Draft & Normal &
Test & Jim Bosch
\\\bottomrule
\multicolumn{6}{c}{ Open \href{https://jira.lsstcorp.org/secure/Tests.jspa\#/testCase/LVV-T73}{LVV-T73} in Jira } \\
\end{longtable}

\subsubsection{Verification Elements}
\begin{itemize}
\item \href{https://jira.lsstcorp.org/browse/LVV-110}{LVV-110} - DMS-REQ-0279-V-01: Deep Detection Coadds

\end{itemize}

\subsubsection{Test Items}
Verify that the DRP pipelines produce a suite of per-band coadded images
that are optimized for depth.


\subsubsection{Predecessors}

\subsubsection{Environment Needs}

\paragraph{Software}

\paragraph{Hardware}

\subsubsection{Input Specification}

\subsubsection{Output Specification}

\subsubsection{Test Procedure}
    \begin{longtable}[]{p{1.3cm}p{2cm}p{13cm}}
    %\toprule
    Step & \multicolumn{2}{@{}l}{Description, Input Data and Expected Result} \\ \toprule
    \endhead

                \multirow{3}{*}{\parbox{1.3cm}{ 1-1
                {\scriptsize from \hyperref[lvv-t987]
                {LVV-T987} } } }

                & {\small Description} &
                \begin{minipage}[t]{13cm}{\scriptsize
                Identify the path to the data repository, which we will refer to as
`DATA/path', then execute the following:

                \vspace{\dp0}
                } \end{minipage} \\ \cdashline{2-3}
                & {\small Test Data} &
                \begin{minipage}[t]{13cm}{\scriptsize
                } \end{minipage} \\ \cdashline{2-3}
                & {\small Expected Result} &
                    \begin{minipage}[t]{13cm}{\scriptsize
                    Butler repo available for reading.

                    \vspace{\dp0}
                    } \end{minipage}
                \\ \hdashline


        \\ \midrule

            \multirow{3}{*}{ 2 } & Description &
            \begin{minipage}[t]{13cm}{\footnotesize
            Verify through inspection that per-filter coadds exist for each
tract+patch possible

            \vspace{\dp0}
            } \end{minipage} \\ \cline{2-3}
            & Test Data &
            \begin{minipage}[t]{13cm}{\footnotesize
                No data.
                \vspace{\dp0}
            } \end{minipage} \\ \cline{2-3}
            & Expected Result &
        \\ \midrule

            \multirow{3}{*}{ 3 } & Description &
            \begin{minipage}[t]{13cm}{\footnotesize
            Verify through inspection that the images used to generate those coadds
met specified conditions

            \vspace{\dp0}
            } \end{minipage} \\ \cline{2-3}
            & Test Data &
            \begin{minipage}[t]{13cm}{\footnotesize
                No data.
                \vspace{\dp0}
            } \end{minipage} \\ \cline{2-3}
            & Expected Result &
        \\ \midrule

            \multirow{3}{*}{ 4 } & Description &
            \begin{minipage}[t]{13cm}{\footnotesize
            Visually inspect a subset of the coadds to verify that they visually
appear reasonable and to be from good quality data.

            \vspace{\dp0}
            } \end{minipage} \\ \cline{2-3}
            & Test Data &
            \begin{minipage}[t]{13cm}{\footnotesize
                No data.
                \vspace{\dp0}
            } \end{minipage} \\ \cline{2-3}
            & Expected Result &
        \\ \midrule
    \end{longtable}

\subsection{LVV-T74 - Verify implementation of Template Coadds}\label{lvv-t74}

\begin{longtable}[]{llllll}
\toprule
Version & Status & Priority & Verification Type & Owner
\\\midrule
1 & Draft & Normal &
Test & Eric Bellm
\\\bottomrule
\multicolumn{6}{c}{ Open \href{https://jira.lsstcorp.org/secure/Tests.jspa\#/testCase/LVV-T74}{LVV-T74} in Jira } \\
\end{longtable}

\subsubsection{Verification Elements}
\begin{itemize}
\item \href{https://jira.lsstcorp.org/browse/LVV-111}{LVV-111} - DMS-REQ-0280-V-01: Template Coadds

\end{itemize}

\subsubsection{Test Items}
Verify that the DMS can produce Template Coadds for DIA processing.


\subsubsection{Predecessors}

\subsubsection{Environment Needs}

\paragraph{Software}

\paragraph{Hardware}

\subsubsection{Input Specification}

\subsubsection{Output Specification}

\subsubsection{Test Procedure}
    \begin{longtable}[]{p{1.3cm}p{2cm}p{13cm}}
    %\toprule
    Step & \multicolumn{2}{@{}l}{Description, Input Data and Expected Result} \\ \toprule
    \endhead

                \multirow{3}{*}{\parbox{1.3cm}{ 1-1
                {\scriptsize from \hyperref[lvv-t866]
                {LVV-T866} } } }

                & {\small Description} &
                \begin{minipage}[t]{13cm}{\scriptsize
                Perform the steps of Alert Production (including, but not necessarily
limited to, single frame processing, ISR, source detection/measurement,
PSF estimation, photometric and astrometric calibration, difference
imaging, DIASource detection/measurement, source association). During
Operations, it is presumed that these are automated for a given
dataset.~

                \vspace{\dp0}
                } \end{minipage} \\ \cdashline{2-3}
                & {\small Test Data} &
                \begin{minipage}[t]{13cm}{\scriptsize
                } \end{minipage} \\ \cdashline{2-3}
                & {\small Expected Result} &
                    \begin{minipage}[t]{13cm}{\scriptsize
                    An output dataset including difference images and DIASource and
DIAObject measurements.

                    \vspace{\dp0}
                    } \end{minipage}
                \\ \hdashline


                \multirow{3}{*}{\parbox{1.3cm}{ 1-2
                {\scriptsize from \hyperref[lvv-t866]
                {LVV-T866} } } }

                & {\small Description} &
                \begin{minipage}[t]{13cm}{\scriptsize
                Verify that the expected data products have been produced, and that
catalogs contain reasonable values for measured quantities of interest.

                \vspace{\dp0}
                } \end{minipage} \\ \cdashline{2-3}
                & {\small Test Data} &
                \begin{minipage}[t]{13cm}{\scriptsize
                } \end{minipage} \\ \cdashline{2-3}
                & {\small Expected Result} &
                \\ \hdashline


        \\ \midrule

            \multirow{3}{*}{ 2 } & Description &
            \begin{minipage}[t]{13cm}{\footnotesize
            Confirm that the template coadds have been created and are well-formed.

            \vspace{\dp0}
            } \end{minipage} \\ \cline{2-3}
            & Test Data &
            \begin{minipage}[t]{13cm}{\footnotesize
                No data.
                \vspace{\dp0}
            } \end{minipage} \\ \cline{2-3}
            & Expected Result &
        \\ \midrule
    \end{longtable}

\subsection{LVV-T75 - Verify implementation of Multi-band Coadds}\label{lvv-t75}

\begin{longtable}[]{llllll}
\toprule
Version & Status & Priority & Verification Type & Owner
\\\midrule
1 & Draft & Normal &
Test & Jim Bosch
\\\bottomrule
\multicolumn{6}{c}{ Open \href{https://jira.lsstcorp.org/secure/Tests.jspa\#/testCase/LVV-T75}{LVV-T75} in Jira } \\
\end{longtable}

\subsubsection{Verification Elements}
\begin{itemize}
\item \href{https://jira.lsstcorp.org/browse/LVV-112}{LVV-112} - DMS-REQ-0281-V-01: Multi-band Coadds

\end{itemize}

\subsubsection{Test Items}
Verify that the DRP pipelines produce multi-band coadds for detection
purposes.


\subsubsection{Predecessors}

\subsubsection{Environment Needs}

\paragraph{Software}

\paragraph{Hardware}

\subsubsection{Input Specification}

\subsubsection{Output Specification}

\subsubsection{Test Procedure}
    \begin{longtable}[]{p{1.3cm}p{2cm}p{13cm}}
    %\toprule
    Step & \multicolumn{2}{@{}l}{Description, Input Data and Expected Result} \\ \toprule
    \endhead

                \multirow{3}{*}{\parbox{1.3cm}{ 1-1
                {\scriptsize from \hyperref[lvv-t987]
                {LVV-T987} } } }

                & {\small Description} &
                \begin{minipage}[t]{13cm}{\scriptsize
                Identify the path to the data repository, which we will refer to as
`DATA/path', then execute the following:

                \vspace{\dp0}
                } \end{minipage} \\ \cdashline{2-3}
                & {\small Test Data} &
                \begin{minipage}[t]{13cm}{\scriptsize
                } \end{minipage} \\ \cdashline{2-3}
                & {\small Expected Result} &
                    \begin{minipage}[t]{13cm}{\scriptsize
                    Butler repo available for reading.

                    \vspace{\dp0}
                    } \end{minipage}
                \\ \hdashline


        \\ \midrule

            \multirow{3}{*}{ 2 } & Description &
            \begin{minipage}[t]{13cm}{\footnotesize
            Verify that deep detection coadds exist based on all
filters.\\[2\baselineskip]

            \vspace{\dp0}
            } \end{minipage} \\ \cline{2-3}
            & Test Data &
            \begin{minipage}[t]{13cm}{\footnotesize
                No data.
                \vspace{\dp0}
            } \end{minipage} \\ \cline{2-3}
            & Expected Result &
        \\ \midrule
    \end{longtable}

\subsection{LVV-T76 - Verify implementation of All-Sky Visualization of Data Releases}\label{lvv-t76}

\begin{longtable}[]{llllll}
\toprule
Version & Status & Priority & Verification Type & Owner
\\\midrule
1 & Draft & Normal &
Test & Simon Krughoff
\\\bottomrule
\multicolumn{6}{c}{ Open \href{https://jira.lsstcorp.org/secure/Tests.jspa\#/testCase/LVV-T76}{LVV-T76} in Jira } \\
\end{longtable}

\subsubsection{Verification Elements}
\begin{itemize}
\item \href{https://jira.lsstcorp.org/browse/LVV-160}{LVV-160} - DMS-REQ-0329-V-01: All-Sky Visualization of Data Releases

\end{itemize}

\subsubsection{Test Items}
Show that it's possible to produce large area visualizations from Data
Release data products.


\subsubsection{Predecessors}

\subsubsection{Environment Needs}

\paragraph{Software}

\paragraph{Hardware}

\subsubsection{Input Specification}
Input Data\\
Dataset of perhaps \textasciitilde{}100 square degrees. ~The first HSC
Public Data Release will be used for this test. ~Larger (in sky area)
datasets should be identified for further testing.

\subsubsection{Output Specification}

\subsubsection{Test Procedure}
    \begin{longtable}[]{p{1.3cm}p{2cm}p{13cm}}
    %\toprule
    Step & \multicolumn{2}{@{}l}{Description, Input Data and Expected Result} \\ \toprule
    \endhead

                \multirow{3}{*}{\parbox{1.3cm}{ 1-1
                {\scriptsize from \hyperref[lvv-t987]
                {LVV-T987} } } }

                & {\small Description} &
                \begin{minipage}[t]{13cm}{\scriptsize
                Identify the path to the data repository, which we will refer to as
`DATA/path', then execute the following:

                \vspace{\dp0}
                } \end{minipage} \\ \cdashline{2-3}
                & {\small Test Data} &
                \begin{minipage}[t]{13cm}{\scriptsize
                } \end{minipage} \\ \cdashline{2-3}
                & {\small Expected Result} &
                    \begin{minipage}[t]{13cm}{\scriptsize
                    Butler repo available for reading.

                    \vspace{\dp0}
                    } \end{minipage}
                \\ \hdashline


        \\ \midrule

            \multirow{3}{*}{ 2 } & Description &
            \begin{minipage}[t]{13cm}{\footnotesize
            Run all sky tile generation task to produce the data products necessary
for serving the all sky visualization.

            \vspace{\dp0}
            } \end{minipage} \\ \cline{2-3}
            & Test Data &
            \begin{minipage}[t]{13cm}{\footnotesize
                No data.
                \vspace{\dp0}
            } \end{minipage} \\ \cline{2-3}
            & Expected Result &
        \\ \midrule

            \multirow{3}{*}{ 3 } & Description &
            \begin{minipage}[t]{13cm}{\footnotesize
            Manually perform, and log (including timing where applicable), the
following steps against that all sky visualization application. ~At all
steps take special care to note any missing or un-rendered image
tiles:\\[2\baselineskip]1. Navigate to the all sky viewer and log the
URL, browser and version.\\
2. Zoom to native pixel display (1 image pixel per display pixel)\\
3. Zoom to fit the full PDR footprint\\
4. Zoom to 1/4x native resolution\\
5. Pan to eastern edge of the footprint.\\
6. Pan to western edge of the footprint.\\
7. Navigate to the middle of the footprint.\\
8. Zoom to max magnification

            \vspace{\dp0}
            } \end{minipage} \\ \cline{2-3}
            & Test Data &
            \begin{minipage}[t]{13cm}{\footnotesize
                No data.
                \vspace{\dp0}
            } \end{minipage} \\ \cline{2-3}
            & Expected Result &
        \\ \midrule
    \end{longtable}

\subsection{LVV-T77 - Verify implementation of Best Seeing Coadds}\label{lvv-t77}

\begin{longtable}[]{llllll}
\toprule
Version & Status & Priority & Verification Type & Owner
\\\midrule
1 & Draft & Normal &
Test & Jim Bosch
\\\bottomrule
\multicolumn{6}{c}{ Open \href{https://jira.lsstcorp.org/secure/Tests.jspa\#/testCase/LVV-T77}{LVV-T77} in Jira } \\
\end{longtable}

\subsubsection{Verification Elements}
\begin{itemize}
\item \href{https://jira.lsstcorp.org/browse/LVV-161}{LVV-161} - DMS-REQ-0330-V-01: Best Seeing Coadds

\end{itemize}

\subsubsection{Test Items}
Verify that the DRP pipelines produce a suite of per-band coadds with
input images filtered to optimize the size of the effective PSF on the
coadd.


\subsubsection{Predecessors}

\subsubsection{Environment Needs}

\paragraph{Software}

\paragraph{Hardware}

\subsubsection{Input Specification}

\subsubsection{Output Specification}

\subsubsection{Test Procedure}
    \begin{longtable}[]{p{1.3cm}p{2cm}p{13cm}}
    %\toprule
    Step & \multicolumn{2}{@{}l}{Description, Input Data and Expected Result} \\ \toprule
    \endhead

                \multirow{3}{*}{\parbox{1.3cm}{ 1-1
                {\scriptsize from \hyperref[lvv-t860]
                {LVV-T860} } } }

                & {\small Description} &
                \begin{minipage}[t]{13cm}{\scriptsize
                The `path` that you will use depends on where you are running the
science pipelines. Options:\\[2\baselineskip]

\begin{itemize}
\tightlist
\item
  local (newinstall.sh - based
  install):{[}path\_to\_installation{]}/loadLSST.bash
\item
  development cluster (``lsst-dev''):
  /software/lsstsw/stack/loadLSST.bash
\item
  LSP Notebook aspect (from a terminal):
  /opt/lsst/software/stack/loadLSST.bash
\end{itemize}

From the command line, execute the commands below in the example
code:\\[2\baselineskip]

                \vspace{\dp0}
                } \end{minipage} \\ \cdashline{2-3}
                & {\small Test Data} &
                \begin{minipage}[t]{13cm}{\scriptsize
                } \end{minipage} \\ \cdashline{2-3}
                & {\small Expected Result} &
                    \begin{minipage}[t]{13cm}{\scriptsize
                    Science pipeline software is available for use. If additional packages
are needed (for example, `obs' packages such as `obs\_subaru`), then
additional `setup` commands will be necessary.\\[2\baselineskip]To check
versions in use, type:\\
eups list -s

                    \vspace{\dp0}
                    } \end{minipage}
                \\ \hdashline


        \\ \midrule

                \multirow{3}{*}{\parbox{1.3cm}{ 2-1
                {\scriptsize from \hyperref[lvv-t987]
                {LVV-T987} } } }

                & {\small Description} &
                \begin{minipage}[t]{13cm}{\scriptsize
                Identify the path to the data repository, which we will refer to as
`DATA/path', then execute the following:

                \vspace{\dp0}
                } \end{minipage} \\ \cdashline{2-3}
                & {\small Test Data} &
                \begin{minipage}[t]{13cm}{\scriptsize
                } \end{minipage} \\ \cdashline{2-3}
                & {\small Expected Result} &
                    \begin{minipage}[t]{13cm}{\scriptsize
                    Butler repo available for reading.

                    \vspace{\dp0}
                    } \end{minipage}
                \\ \hdashline


        \\ \midrule

            \multirow{3}{*}{ 3 } & Description &
            \begin{minipage}[t]{13cm}{\footnotesize
            Explicitly create a coadd for a specified seeing range in each filter.

            \vspace{\dp0}
            } \end{minipage} \\ \cline{2-3}
            & Test Data &
            \begin{minipage}[t]{13cm}{\footnotesize
                No data.
                \vspace{\dp0}
            } \end{minipage} \\ \cline{2-3}
            & Expected Result &
        \\ \midrule

            \multirow{3}{*}{ 4 } & Description &
            \begin{minipage}[t]{13cm}{\footnotesize
            Verify that these coadds exist.

            \vspace{\dp0}
            } \end{minipage} \\ \cline{2-3}
            & Test Data &
            \begin{minipage}[t]{13cm}{\footnotesize
                No data.
                \vspace{\dp0}
            } \end{minipage} \\ \cline{2-3}
            & Expected Result &
        \\ \midrule
    \end{longtable}

\subsection{LVV-T78 - Verify implementation of Persisting Data Products}\label{lvv-t78}

\begin{longtable}[]{llllll}
\toprule
Version & Status & Priority & Verification Type & Owner
\\\midrule
1 & Draft & Normal &
Test & Kian-Tat Lim
\\\bottomrule
\multicolumn{6}{c}{ Open \href{https://jira.lsstcorp.org/secure/Tests.jspa\#/testCase/LVV-T78}{LVV-T78} in Jira } \\
\end{longtable}

\subsubsection{Verification Elements}
\begin{itemize}
\item \href{https://jira.lsstcorp.org/browse/LVV-165}{LVV-165} - DMS-REQ-0334-V-01: Persisting Data Products

\end{itemize}

\subsubsection{Test Items}
Verify that per-band deep coadds and best-seeing coadds are present,
kept, and available.


\subsubsection{Predecessors}

\subsubsection{Environment Needs}

\paragraph{Software}

\paragraph{Hardware}

\subsubsection{Input Specification}
Precursor data from HSC PDR.

\subsubsection{Output Specification}

\subsubsection{Test Procedure}
    \begin{longtable}[]{p{1.3cm}p{2cm}p{13cm}}
    %\toprule
    Step & \multicolumn{2}{@{}l}{Description, Input Data and Expected Result} \\ \toprule
    \endhead

            \multirow{3}{*}{ 1 } & Description &
            \begin{minipage}[t]{13cm}{\footnotesize
            \hypertarget{description-val}{}
Produce some relevant coadds and store them in the Archive

            \vspace{\dp0}
            } \end{minipage} \\ \cline{2-3}
            & Test Data &
            \begin{minipage}[t]{13cm}{\footnotesize
                No data.
                \vspace{\dp0}
            } \end{minipage} \\ \cline{2-3}
            & Expected Result &
        \\ \midrule

            \multirow{3}{*}{ 2 } & Description &
            \begin{minipage}[t]{13cm}{\footnotesize
            Examine the data retention policies for those products

            \vspace{\dp0}
            } \end{minipage} \\ \cline{2-3}
            & Test Data &
            \begin{minipage}[t]{13cm}{\footnotesize
                No data.
                \vspace{\dp0}
            } \end{minipage} \\ \cline{2-3}
            & Expected Result &
        \\ \midrule
    \end{longtable}

\subsection{LVV-T79 - Verify implementation of PSF-Matched Coadds}\label{lvv-t79}

\begin{longtable}[]{llllll}
\toprule
Version & Status & Priority & Verification Type & Owner
\\\midrule
1 & Draft & Normal &
Test & Jim Bosch
\\\bottomrule
\multicolumn{6}{c}{ Open \href{https://jira.lsstcorp.org/secure/Tests.jspa\#/testCase/LVV-T79}{LVV-T79} in Jira } \\
\end{longtable}

\subsubsection{Verification Elements}
\begin{itemize}
\item \href{https://jira.lsstcorp.org/browse/LVV-166}{LVV-166} - DMS-REQ-0335-V-01: PSF-Matched Coadds

\end{itemize}

\subsubsection{Test Items}
Verify that the DRP pipelines produce PSF matched coadds.


\subsubsection{Predecessors}

\subsubsection{Environment Needs}

\paragraph{Software}

\paragraph{Hardware}

\subsubsection{Input Specification}

\subsubsection{Output Specification}

\subsubsection{Test Procedure}
    \begin{longtable}[]{p{1.3cm}p{2cm}p{13cm}}
    %\toprule
    Step & \multicolumn{2}{@{}l}{Description, Input Data and Expected Result} \\ \toprule
    \endhead

                \multirow{3}{*}{\parbox{1.3cm}{ 1-1
                {\scriptsize from \hyperref[lvv-t987]
                {LVV-T987} } } }

                & {\small Description} &
                \begin{minipage}[t]{13cm}{\scriptsize
                Identify the path to the data repository, which we will refer to as
`DATA/path', then execute the following:

                \vspace{\dp0}
                } \end{minipage} \\ \cdashline{2-3}
                & {\small Test Data} &
                \begin{minipage}[t]{13cm}{\scriptsize
                } \end{minipage} \\ \cdashline{2-3}
                & {\small Expected Result} &
                    \begin{minipage}[t]{13cm}{\scriptsize
                    Butler repo available for reading.

                    \vspace{\dp0}
                    } \end{minipage}
                \\ \hdashline


        \\ \midrule

            \multirow{3}{*}{ 2 } & Description &
            \begin{minipage}[t]{13cm}{\footnotesize
            Verify that PSF-matched coadds were created.

            \vspace{\dp0}
            } \end{minipage} \\ \cline{2-3}
            & Test Data &
            \begin{minipage}[t]{13cm}{\footnotesize
                No data.
                \vspace{\dp0}
            } \end{minipage} \\ \cline{2-3}
            & Expected Result &
        \\ \midrule
    \end{longtable}

\subsection{LVV-T80 - Verify implementation of Detecting faint variable objects}\label{lvv-t80}

\begin{longtable}[]{llllll}
\toprule
Version & Status & Priority & Verification Type & Owner
\\\midrule
1 & Draft & Normal &
Test & Melissa Graham
\\\bottomrule
\multicolumn{6}{c}{ Open \href{https://jira.lsstcorp.org/secure/Tests.jspa\#/testCase/LVV-T80}{LVV-T80} in Jira } \\
\end{longtable}

\subsubsection{Verification Elements}
\begin{itemize}
\item \href{https://jira.lsstcorp.org/browse/LVV-168}{LVV-168} - DMS-REQ-0337-V-01: Detecting faint variable objects

\end{itemize}

\subsubsection{Test Items}
To verify that the Data Release Production pipeline will be able to
detect faint sources with long-term variability (e.g., quasars, proper
motion stars) via, e.g., shorter timescale coadds (month to a few
months).


\subsubsection{Predecessors}

\subsubsection{Environment Needs}

\paragraph{Software}

\paragraph{Hardware}

\subsubsection{Input Specification}
Input Data such as:\\
DECam HiTS data.\\
Gaia catalog of faint moving objects.\\
Catalog of spectroscopically confirmed quasars.\\
(Alternative: input data injected with faint variable sources).

\subsubsection{Output Specification}

\subsubsection{Test Procedure}
    \begin{longtable}[]{p{1.3cm}p{2cm}p{13cm}}
    %\toprule
    Step & \multicolumn{2}{@{}l}{Description, Input Data and Expected Result} \\ \toprule
    \endhead

                \multirow{3}{*}{\parbox{1.3cm}{ 1-1
                {\scriptsize from \hyperref[lvv-t866]
                {LVV-T866} } } }

                & {\small Description} &
                \begin{minipage}[t]{13cm}{\scriptsize
                Perform the steps of Alert Production (including, but not necessarily
limited to, single frame processing, ISR, source detection/measurement,
PSF estimation, photometric and astrometric calibration, difference
imaging, DIASource detection/measurement, source association). During
Operations, it is presumed that these are automated for a given
dataset.~

                \vspace{\dp0}
                } \end{minipage} \\ \cdashline{2-3}
                & {\small Test Data} &
                \begin{minipage}[t]{13cm}{\scriptsize
                } \end{minipage} \\ \cdashline{2-3}
                & {\small Expected Result} &
                    \begin{minipage}[t]{13cm}{\scriptsize
                    An output dataset including difference images and DIASource and
DIAObject measurements.

                    \vspace{\dp0}
                    } \end{minipage}
                \\ \hdashline


                \multirow{3}{*}{\parbox{1.3cm}{ 1-2
                {\scriptsize from \hyperref[lvv-t866]
                {LVV-T866} } } }

                & {\small Description} &
                \begin{minipage}[t]{13cm}{\scriptsize
                Verify that the expected data products have been produced, and that
catalogs contain reasonable values for measured quantities of interest.

                \vspace{\dp0}
                } \end{minipage} \\ \cdashline{2-3}
                & {\small Test Data} &
                \begin{minipage}[t]{13cm}{\scriptsize
                } \end{minipage} \\ \cdashline{2-3}
                & {\small Expected Result} &
                \\ \hdashline


        \\ \midrule

                \multirow{3}{*}{\parbox{1.3cm}{ 2-1
                {\scriptsize from \hyperref[lvv-t987]
                {LVV-T987} } } }

                & {\small Description} &
                \begin{minipage}[t]{13cm}{\scriptsize
                Identify the path to the data repository, which we will refer to as
`DATA/path', then execute the following:

                \vspace{\dp0}
                } \end{minipage} \\ \cdashline{2-3}
                & {\small Test Data} &
                \begin{minipage}[t]{13cm}{\scriptsize
                } \end{minipage} \\ \cdashline{2-3}
                & {\small Expected Result} &
                    \begin{minipage}[t]{13cm}{\scriptsize
                    Butler repo available for reading.

                    \vspace{\dp0}
                    } \end{minipage}
                \\ \hdashline


        \\ \midrule

            \multirow{3}{*}{ 3 } & Description &
            \begin{minipage}[t]{13cm}{\footnotesize
            Identify 100 objects from Gaia with proper motions high enough to have
detectably moved during HSC observations.

            \vspace{\dp0}
            } \end{minipage} \\ \cline{2-3}
            & Test Data &
            \begin{minipage}[t]{13cm}{\footnotesize
                No data.
                \vspace{\dp0}
            } \end{minipage} \\ \cline{2-3}
            & Expected Result &
        \\ \midrule

            \multirow{3}{*}{ 4 } & Description &
            \begin{minipage}[t]{13cm}{\footnotesize
            Measure reported proper motion of these objects in DM Stack processing.
~Verify that it is consistent with Gaia objects.

            \vspace{\dp0}
            } \end{minipage} \\ \cline{2-3}
            & Test Data &
            \begin{minipage}[t]{13cm}{\footnotesize
                No data.
                \vspace{\dp0}
            } \end{minipage} \\ \cline{2-3}
            & Expected Result &
        \\ \midrule

            \multirow{3}{*}{ 5 } & Description &
            \begin{minipage}[t]{13cm}{\footnotesize
            Identify 100 quasars from color-space or existing extragalactic
spectroscopic catalog.

            \vspace{\dp0}
            } \end{minipage} \\ \cline{2-3}
            & Test Data &
            \begin{minipage}[t]{13cm}{\footnotesize
                No data.
                \vspace{\dp0}
            } \end{minipage} \\ \cline{2-3}
            & Expected Result &
        \\ \midrule

            \multirow{3}{*}{ 6 } & Description &
            \begin{minipage}[t]{13cm}{\footnotesize
            Measure lightcurves of these quasars. ~Determine if structure function
is reasonable (may require at least a year to determine if the structure
function of 100 quasars is ``reasonable'').

            \vspace{\dp0}
            } \end{minipage} \\ \cline{2-3}
            & Test Data &
            \begin{minipage}[t]{13cm}{\footnotesize
                No data.
                \vspace{\dp0}
            } \end{minipage} \\ \cline{2-3}
            & Expected Result &
        \\ \midrule

            \multirow{3}{*}{ 7 } & Description &
            \begin{minipage}[t]{13cm}{\footnotesize
            (Alternative: if faint variable source can be injected into the input
data, test to see if they are recovered).

            \vspace{\dp0}
            } \end{minipage} \\ \cline{2-3}
            & Test Data &
            \begin{minipage}[t]{13cm}{\footnotesize
                No data.
                \vspace{\dp0}
            } \end{minipage} \\ \cline{2-3}
            & Expected Result &
                \begin{minipage}[t]{13cm}{\footnotesize
                (This Alternative would enable us not only to tell if faint variable
objects are detected, but exactly which kinds, how faint, and with what
efficiency.)

                \vspace{\dp0}
                } \end{minipage}
        \\ \midrule
    \end{longtable}

\subsection{LVV-T81 - Verify implementation of Targeted Coadds}\label{lvv-t81}

\begin{longtable}[]{llllll}
\toprule
Version & Status & Priority & Verification Type & Owner
\\\midrule
1 & Draft & Normal &
Test & Jim Bosch
\\\bottomrule
\multicolumn{6}{c}{ Open \href{https://jira.lsstcorp.org/secure/Tests.jspa\#/testCase/LVV-T81}{LVV-T81} in Jira } \\
\end{longtable}

\subsubsection{Verification Elements}
\begin{itemize}
\item \href{https://jira.lsstcorp.org/browse/LVV-169}{LVV-169} - DMS-REQ-0338-V-01: Targeted Coadds

\end{itemize}

\subsubsection{Test Items}
Verify that small sections of any coadd produced by the DRP pipelines
can be retained, even if the full coadd is not.


\subsubsection{Predecessors}

\subsubsection{Environment Needs}

\paragraph{Software}

\paragraph{Hardware}

\subsubsection{Input Specification}

\subsubsection{Output Specification}

\subsubsection{Test Procedure}
    \begin{longtable}[]{p{1.3cm}p{2cm}p{13cm}}
    %\toprule
    Step & \multicolumn{2}{@{}l}{Description, Input Data and Expected Result} \\ \toprule
    \endhead

            \multirow{3}{*}{ 1 } & Description &
            \begin{minipage}[t]{13cm}{\footnotesize
            Remove DR from disk

            \vspace{\dp0}
            } \end{minipage} \\ \cline{2-3}
            & Test Data &
            \begin{minipage}[t]{13cm}{\footnotesize
                No data.
                \vspace{\dp0}
            } \end{minipage} \\ \cline{2-3}
            & Expected Result &
        \\ \midrule

            \multirow{3}{*}{ 2 } & Description &
            \begin{minipage}[t]{13cm}{\footnotesize
            Observe retention of designated coadd sections

            \vspace{\dp0}
            } \end{minipage} \\ \cline{2-3}
            & Test Data &
            \begin{minipage}[t]{13cm}{\footnotesize
                No data.
                \vspace{\dp0}
            } \end{minipage} \\ \cline{2-3}
            & Expected Result &
        \\ \midrule

            \multirow{3}{*}{ 3 } & Description &
            \begin{minipage}[t]{13cm}{\footnotesize
            Observe accessibility of designated coadd sections via simulated DAC LSP
instance

            \vspace{\dp0}
            } \end{minipage} \\ \cline{2-3}
            & Test Data &
            \begin{minipage}[t]{13cm}{\footnotesize
                No data.
                \vspace{\dp0}
            } \end{minipage} \\ \cline{2-3}
            & Expected Result &
        \\ \midrule
    \end{longtable}

\subsection{LVV-T82 - Verify implementation of Tracking Characterization Changes Between Data
Releases}\label{lvv-t82}

\begin{longtable}[]{llllll}
\toprule
Version & Status & Priority & Verification Type & Owner
\\\midrule
1 & Defined & Normal &
Test & Jim Bosch
\\\bottomrule
\multicolumn{6}{c}{ Open \href{https://jira.lsstcorp.org/secure/Tests.jspa\#/testCase/LVV-T82}{LVV-T82} in Jira } \\
\end{longtable}

\subsubsection{Verification Elements}
\begin{itemize}
\item \href{https://jira.lsstcorp.org/browse/LVV-170}{LVV-170} - DMS-REQ-0339-V-01: Tracking Characterization Changes Between Data
Releases

\end{itemize}

\subsubsection{Test Items}
Verify that small-area subsets of a DR can be retained when most of that
DR is retired, for comparison with future DRs.


\subsubsection{Predecessors}

\subsubsection{Environment Needs}

\paragraph{Software}

\paragraph{Hardware}

\subsubsection{Input Specification}

\subsubsection{Output Specification}

\subsubsection{Test Procedure}
    \begin{longtable}[]{p{1.3cm}p{2cm}p{13cm}}
    %\toprule
    Step & \multicolumn{2}{@{}l}{Description, Input Data and Expected Result} \\ \toprule
    \endhead

            \multirow{3}{*}{ 1 } & Description &
            \begin{minipage}[t]{13cm}{\footnotesize
            Prepare a second DRP run -\textgreater{} DPDD with different
configuration parameters for this second test Data Release.

            \vspace{\dp0}
            } \end{minipage} \\ \cline{2-3}
            & Test Data &
            \begin{minipage}[t]{13cm}{\footnotesize
                No data.
                \vspace{\dp0}
            } \end{minipage} \\ \cline{2-3}
            & Expected Result &
        \\ \midrule

                \multirow{3}{*}{\parbox{1.3cm}{ 2-1
                {\scriptsize from \hyperref[lvv-t1064]
                {LVV-T1064} } } }

                & {\small Description} &
                \begin{minipage}[t]{13cm}{\scriptsize
                Process data with the Data Release Production payload, starting from raw
science images and generating science data products, placing them in the
Data Backbone.

                \vspace{\dp0}
                } \end{minipage} \\ \cdashline{2-3}
                & {\small Test Data} &
                \begin{minipage}[t]{13cm}{\scriptsize
                } \end{minipage} \\ \cdashline{2-3}
                & {\small Expected Result} &
                \\ \hdashline


        \\ \midrule

            \multirow{3}{*}{ 3 } & Description &
            \begin{minipage}[t]{13cm}{\footnotesize
            Stage subset of products from first test Data Release to separate
storage.

            \vspace{\dp0}
            } \end{minipage} \\ \cline{2-3}
            & Test Data &
            \begin{minipage}[t]{13cm}{\footnotesize
                No data.
                \vspace{\dp0}
            } \end{minipage} \\ \cline{2-3}
            & Expected Result &
        \\ \midrule

            \multirow{3}{*}{ 4 } & Description &
            \begin{minipage}[t]{13cm}{\footnotesize
            Scientifically compare the results of the subset of that region of sky
to those in the second test Data Release comparing the results of the
DRP Scientific Verification tests.

            \vspace{\dp0}
            } \end{minipage} \\ \cline{2-3}
            & Test Data &
            \begin{minipage}[t]{13cm}{\footnotesize
                No data.
                \vspace{\dp0}
            } \end{minipage} \\ \cline{2-3}
            & Expected Result &
                \begin{minipage}[t]{13cm}{\footnotesize
                Diagnostic plots quantifying the differences between scientific outputs
between the first and second test datasets.

                \vspace{\dp0}
                } \end{minipage}
        \\ \midrule
    \end{longtable}

\subsection{LVV-T83 - Verify implementation of Bad Pixel Map}\label{lvv-t83}

\begin{longtable}[]{llllll}
\toprule
Version & Status & Priority & Verification Type & Owner
\\\midrule
1 & Defined & Normal &
Test & Robert Lupton
\\\bottomrule
\multicolumn{6}{c}{ Open \href{https://jira.lsstcorp.org/secure/Tests.jspa\#/testCase/LVV-T83}{LVV-T83} in Jira } \\
\end{longtable}

\subsubsection{Verification Elements}
\begin{itemize}
\item \href{https://jira.lsstcorp.org/browse/LVV-22}{LVV-22} - DMS-REQ-0059-V-01: Bad Pixel Map

\end{itemize}

\subsubsection{Test Items}
Verify that the DMS can produce a map of detector pixels that suffer
from pathologies, and that these pathologies are encoded in at least
32-bit values.


\subsubsection{Predecessors}

\subsubsection{Environment Needs}

\paragraph{Software}

\paragraph{Hardware}

\subsubsection{Input Specification}

\subsubsection{Output Specification}

\subsubsection{Test Procedure}
    \begin{longtable}[]{p{1.3cm}p{2cm}p{13cm}}
    %\toprule
    Step & \multicolumn{2}{@{}l}{Description, Input Data and Expected Result} \\ \toprule
    \endhead

            \multirow{3}{*}{ 1 } & Description &
            \begin{minipage}[t]{13cm}{\footnotesize
            Interrogate the calibRegistry for the metadata associated with a bad
pixel map, where the validity range contains the date of interest.

            \vspace{\dp0}
            } \end{minipage} \\ \cline{2-3}
            & Test Data &
            \begin{minipage}[t]{13cm}{\footnotesize
                No data.
                \vspace{\dp0}
            } \end{minipage} \\ \cline{2-3}
            & Expected Result &
                \begin{minipage}[t]{13cm}{\footnotesize
                A bad pixel map for the requested date has been returned.

                \vspace{\dp0}
                } \end{minipage}
        \\ \midrule

            \multirow{3}{*}{ 2 } & Description &
            \begin{minipage}[t]{13cm}{\footnotesize
            Check that the bad pixel pathologies are encoded as at least 32-bit
values, and that the various pathologies are represented by different
encoding.

            \vspace{\dp0}
            } \end{minipage} \\ \cline{2-3}
            & Test Data &
            \begin{minipage}[t]{13cm}{\footnotesize
                No data.
                \vspace{\dp0}
            } \end{minipage} \\ \cline{2-3}
            & Expected Result &
                \begin{minipage}[t]{13cm}{\footnotesize
                Bad pixel values can be decoded to determine their pathologies using
their 32-bit values.

                \vspace{\dp0}
                } \end{minipage}
        \\ \midrule
    \end{longtable}

\subsection{LVV-T84 - Verify implementation of Bias Residual Image}\label{lvv-t84}

\begin{longtable}[]{llllll}
\toprule
Version & Status & Priority & Verification Type & Owner
\\\midrule
1 & Defined & Normal &
Test & Robert Lupton
\\\bottomrule
\multicolumn{6}{c}{ Open \href{https://jira.lsstcorp.org/secure/Tests.jspa\#/testCase/LVV-T84}{LVV-T84} in Jira } \\
\end{longtable}

\subsubsection{Verification Elements}
\begin{itemize}
\item \href{https://jira.lsstcorp.org/browse/LVV-23}{LVV-23} - DMS-REQ-0060-V-01: Bias Residual Image

\end{itemize}

\subsubsection{Test Items}
Verify that DMS can construct a bias residual image that corrects for
temporally-stable bias structures.\\
Verify that DMS can do this on demand.


\subsubsection{Predecessors}

\subsubsection{Environment Needs}

\paragraph{Software}

\paragraph{Hardware}

\subsubsection{Input Specification}

\subsubsection{Output Specification}

\subsubsection{Test Procedure}
    \begin{longtable}[]{p{1.3cm}p{2cm}p{13cm}}
    %\toprule
    Step & \multicolumn{2}{@{}l}{Description, Input Data and Expected Result} \\ \toprule
    \endhead

            \multirow{3}{*}{ 1 } & Description &
            \begin{minipage}[t]{13cm}{\footnotesize
            Identify the location of an appropriate precursor dataset.

            \vspace{\dp0}
            } \end{minipage} \\ \cline{2-3}
            & Test Data &
            \begin{minipage}[t]{13cm}{\footnotesize
                No data.
                \vspace{\dp0}
            } \end{minipage} \\ \cline{2-3}
            & Expected Result &
        \\ \midrule

                \multirow{3}{*}{\parbox{1.3cm}{ 2-1
                {\scriptsize from \hyperref[lvv-t987]
                {LVV-T987} } } }

                & {\small Description} &
                \begin{minipage}[t]{13cm}{\scriptsize
                Identify the path to the data repository, which we will refer to as
`DATA/path', then execute the following:

                \vspace{\dp0}
                } \end{minipage} \\ \cdashline{2-3}
                & {\small Test Data} &
                \begin{minipage}[t]{13cm}{\scriptsize
                } \end{minipage} \\ \cdashline{2-3}
                & {\small Expected Result} &
                    \begin{minipage}[t]{13cm}{\scriptsize
                    Butler repo available for reading.

                    \vspace{\dp0}
                    } \end{minipage}
                \\ \hdashline


        \\ \midrule

            \multirow{3}{*}{ 3 } & Description &
            \begin{minipage}[t]{13cm}{\footnotesize
            Import the standard libraries required for the rest of this test:

            \vspace{\dp0}
            } \end{minipage} \\ \cline{2-3}
            & Test Data &
            \begin{minipage}[t]{13cm}{\footnotesize
                No data.
                \vspace{\dp0}
            } \end{minipage} \\ \cline{2-3}
                & Example Code &
                \begin{minipage}[t]{13cm}{\footnotesize
                import osimport lsst.afw.display as afwDisplay\\
from lsst.daf.persistence import Butler\\
from lsst.ip.isr import IsrTask\\
from firefly\_client import FireflyClient\\
from IPython.display import IFrame

                \vspace{\dp0}
                } \end{minipage} \\ \cline{2-3}
            & Expected Result &
        \\ \midrule

            \multirow{3}{*}{ 4 } & Description &
            \begin{minipage}[t]{13cm}{\footnotesize
            Ingest the dataset from step 1 using the Butler (e.g., following example
code below).

            \vspace{\dp0}
            } \end{minipage} \\ \cline{2-3}
            & Test Data &
            \begin{minipage}[t]{13cm}{\footnotesize
                No data.
                \vspace{\dp0}
            } \end{minipage} \\ \cline{2-3}
                & Example Code &
                \begin{minipage}[t]{13cm}{\footnotesize
                butler = Butler(\$REPOSITORY\_PATH)\\
raw = butler.get(``raw'', visit=\$VISIT\_ID, detector=2)\\
bias = butler.get(``bias'', visit=\$VISIT\_ID, detector=2)

                \vspace{\dp0}
                } \end{minipage} \\ \cline{2-3}
            & Expected Result &
        \\ \midrule

            \multirow{3}{*}{ 5 } & Description &
            \begin{minipage}[t]{13cm}{\footnotesize
            Display the bias image and inspect that its pixels contain unique
values.

            \vspace{\dp0}
            } \end{minipage} \\ \cline{2-3}
            & Test Data &
            \begin{minipage}[t]{13cm}{\footnotesize
                No data.
                \vspace{\dp0}
            } \end{minipage} \\ \cline{2-3}
            & Expected Result &
                \begin{minipage}[t]{13cm}{\footnotesize
                A relatively flat image showing the bias level with roughly Poisson
noise.

                \vspace{\dp0}
                } \end{minipage}
        \\ \midrule

            \multirow{3}{*}{ 6 } & Description &
            \begin{minipage}[t]{13cm}{\footnotesize
            Configure and run an Instrument Signature Removal (ISR) task on the raw
data. Most corrections are disabled for simplicity, but the bias frame
is applied.\\[2\baselineskip]

            \vspace{\dp0}
            } \end{minipage} \\ \cline{2-3}
            & Test Data &
            \begin{minipage}[t]{13cm}{\footnotesize
                No data.
                \vspace{\dp0}
            } \end{minipage} \\ \cline{2-3}
                & Example Code &
                \begin{minipage}[t]{13cm}{\footnotesize
                isr\_config = IsrTask.ConfigClass()\\
isr\_config.doDark=False\\
isr\_config.doFlat=False\\
isr\_config.doFringe=False\\
isr\_config.doDefect=False\\
isr\_config.doAddDistortionModel=False\\
isr\_config.doLinearize=False\\
isr = IsrTask(config=isr\_config)\\
result = isr.run(raw, bias=bias)

                \vspace{\dp0}
                } \end{minipage} \\ \cline{2-3}
            & Expected Result &
                \begin{minipage}[t]{13cm}{\footnotesize
                A trimmed, bias-corrected image in `result`.

                \vspace{\dp0}
                } \end{minipage}
        \\ \midrule

            \multirow{3}{*}{ 7 } & Description &
            \begin{minipage}[t]{13cm}{\footnotesize
            Display the `result` image and confirm that the bias correction has been
performed.

            \vspace{\dp0}
            } \end{minipage} \\ \cline{2-3}
            & Test Data &
            \begin{minipage}[t]{13cm}{\footnotesize
                No data.
                \vspace{\dp0}
            } \end{minipage} \\ \cline{2-3}
            & Expected Result &
                \begin{minipage}[t]{13cm}{\footnotesize
                A displayed image with bias removed (i.e., typical background counts
reduced relative to the raw frame).

                \vspace{\dp0}
                } \end{minipage}
        \\ \midrule
    \end{longtable}

\subsection{LVV-T85 - Verify implementation of Crosstalk Correction Matrix}\label{lvv-t85}

\begin{longtable}[]{llllll}
\toprule
Version & Status & Priority & Verification Type & Owner
\\\midrule
1 & Defined & Normal &
Test & Robert Lupton
\\\bottomrule
\multicolumn{6}{c}{ Open \href{https://jira.lsstcorp.org/secure/Tests.jspa\#/testCase/LVV-T85}{LVV-T85} in Jira } \\
\end{longtable}

\subsubsection{Verification Elements}
\begin{itemize}
\item \href{https://jira.lsstcorp.org/browse/LVV-24}{LVV-24} - DMS-REQ-0061-V-01: Crosstalk Correction Matrix

\end{itemize}

\subsubsection{Test Items}
Verify that the DMS can generate a cross-talk correction matrix from
appropriate calibration data.\\
Verify that the DMS can measure the effectiveness of the cross-talk
correction matrix.


\subsubsection{Predecessors}

\subsubsection{Environment Needs}

\paragraph{Software}

\paragraph{Hardware}

\subsubsection{Input Specification}

\subsubsection{Output Specification}

\subsubsection{Test Procedure}
    \begin{longtable}[]{p{1.3cm}p{2cm}p{13cm}}
    %\toprule
    Step & \multicolumn{2}{@{}l}{Description, Input Data and Expected Result} \\ \toprule
    \endhead

            \multirow{3}{*}{ 1 } & Description &
            \begin{minipage}[t]{13cm}{\footnotesize
            Identify an appropriate calibration dataset that can be used to derive
the crosstalk correction matrix.

            \vspace{\dp0}
            } \end{minipage} \\ \cline{2-3}
            & Test Data &
            \begin{minipage}[t]{13cm}{\footnotesize
                No data.
                \vspace{\dp0}
            } \end{minipage} \\ \cline{2-3}
            & Expected Result &
        \\ \midrule

                \multirow{3}{*}{\parbox{1.3cm}{ 2-1
                {\scriptsize from \hyperref[lvv-t1060]
                {LVV-T1060} } } }

                & {\small Description} &
                \begin{minipage}[t]{13cm}{\scriptsize
                Execute the Calibration Products Production payload. The payload uses
raw calibration images and information from the Transformed EFD to
generate a subset of Master Calibration Images and Calibration Database
entries in the Data Backbone.

                \vspace{\dp0}
                } \end{minipage} \\ \cdashline{2-3}
                & {\small Test Data} &
                \begin{minipage}[t]{13cm}{\scriptsize
                } \end{minipage} \\ \cdashline{2-3}
                & {\small Expected Result} &
                \\ \hdashline


                \multirow{3}{*}{\parbox{1.3cm}{ 2-2
                {\scriptsize from \hyperref[lvv-t1060]
                {LVV-T1060} } } }

                & {\small Description} &
                \begin{minipage}[t]{13cm}{\scriptsize
                Confirm that the expected Master Calibration images and Calibration
Database entries are present and well-formed.

                \vspace{\dp0}
                } \end{minipage} \\ \cdashline{2-3}
                & {\small Test Data} &
                \begin{minipage}[t]{13cm}{\scriptsize
                } \end{minipage} \\ \cdashline{2-3}
                & {\small Expected Result} &
                \\ \hdashline


        \\ \midrule

            \multirow{3}{*}{ 3 } & Description &
            \begin{minipage}[t]{13cm}{\footnotesize
            Confirm that the crosstalk correction matrix is produced and persisted.

            \vspace{\dp0}
            } \end{minipage} \\ \cline{2-3}
            & Test Data &
            \begin{minipage}[t]{13cm}{\footnotesize
                No data.
                \vspace{\dp0}
            } \end{minipage} \\ \cline{2-3}
            & Expected Result &
                \begin{minipage}[t]{13cm}{\footnotesize
                A correction matrix quantifying what fraction of the signal detected in
any given amplifier on each sensor in the focal plane appears in any
other amplifier.

                \vspace{\dp0}
                } \end{minipage}
        \\ \midrule

            \multirow{3}{*}{ 4 } & Description &
            \begin{minipage}[t]{13cm}{\footnotesize
            Apply the crosstalk correction to simulated images, and confirm that the
correction is performing as expected.

            \vspace{\dp0}
            } \end{minipage} \\ \cline{2-3}
            & Test Data &
            \begin{minipage}[t]{13cm}{\footnotesize
                No data.
                \vspace{\dp0}
            } \end{minipage} \\ \cline{2-3}
            & Expected Result &
                \begin{minipage}[t]{13cm}{\footnotesize
                A noticeable difference between images before and after applying the
correction.

                \vspace{\dp0}
                } \end{minipage}
        \\ \midrule
    \end{longtable}

\subsection{LVV-T86 - Verify implementation of Illumination Correction Frame}\label{lvv-t86}

\begin{longtable}[]{llllll}
\toprule
Version & Status & Priority & Verification Type & Owner
\\\midrule
1 & Draft & Normal &
Test & Robert Lupton
\\\bottomrule
\multicolumn{6}{c}{ Open \href{https://jira.lsstcorp.org/secure/Tests.jspa\#/testCase/LVV-T86}{LVV-T86} in Jira } \\
\end{longtable}

\subsubsection{Verification Elements}
\begin{itemize}
\item \href{https://jira.lsstcorp.org/browse/LVV-25}{LVV-25} - DMS-REQ-0062-V-01: Illumination Correction Frame

\end{itemize}

\subsubsection{Test Items}
Verify that the DMS can produce an illumination correction frame
calibration product.\\
Verify that the DMS can determine the effectiveness of an illumination
correction and determine how often it should be updated.


\subsubsection{Predecessors}

\subsubsection{Environment Needs}

\paragraph{Software}

\paragraph{Hardware}

\subsubsection{Input Specification}

\subsubsection{Output Specification}

\subsubsection{Test Procedure}
    \begin{longtable}[]{p{1.3cm}p{2cm}p{13cm}}
    %\toprule
    Step & \multicolumn{2}{@{}l}{Description, Input Data and Expected Result} \\ \toprule
    \endhead

            \multirow{3}{*}{ 1 } & Description &
            \begin{minipage}[t]{13cm}{\footnotesize
            Delegate to CPP

            \vspace{\dp0}
            } \end{minipage} \\ \cline{2-3}
            & Test Data &
            \begin{minipage}[t]{13cm}{\footnotesize
                No data.
                \vspace{\dp0}
            } \end{minipage} \\ \cline{2-3}
            & Expected Result &
        \\ \midrule
    \end{longtable}

\subsection{LVV-T87 - Verify implementation of Monochromatic Flatfield Data Cube}\label{lvv-t87}

\begin{longtable}[]{llllll}
\toprule
Version & Status & Priority & Verification Type & Owner
\\\midrule
1 & Draft & Normal &
Test & Robert Lupton
\\\bottomrule
\multicolumn{6}{c}{ Open \href{https://jira.lsstcorp.org/secure/Tests.jspa\#/testCase/LVV-T87}{LVV-T87} in Jira } \\
\end{longtable}

\subsubsection{Verification Elements}
\begin{itemize}
\item \href{https://jira.lsstcorp.org/browse/LVV-26}{LVV-26} - DMS-REQ-0063-V-01: Monochromatic Flatfield Data Cube

\end{itemize}

\subsubsection{Test Items}
Verify that the DMS can generate a calibration image/cube that corrects
for pixel-to-pixel wavelength-dependent detector response.\\
Verify that the DMS can measure the effectiveness of this monochromatic
flatfield data cube.


\subsubsection{Predecessors}

\subsubsection{Environment Needs}

\paragraph{Software}

\paragraph{Hardware}

\subsubsection{Input Specification}

\subsubsection{Output Specification}

\subsubsection{Test Procedure}
    \begin{longtable}[]{p{1.3cm}p{2cm}p{13cm}}
    %\toprule
    Step & \multicolumn{2}{@{}l}{Description, Input Data and Expected Result} \\ \toprule
    \endhead

            \multirow{3}{*}{ 1 } & Description &
            \begin{minipage}[t]{13cm}{\footnotesize
            Delegate to CPP

            \vspace{\dp0}
            } \end{minipage} \\ \cline{2-3}
            & Test Data &
            \begin{minipage}[t]{13cm}{\footnotesize
                No data.
                \vspace{\dp0}
            } \end{minipage} \\ \cline{2-3}
            & Expected Result &
        \\ \midrule
    \end{longtable}

\subsection{LVV-T88 - Verify implementation of Calibration Data Products}\label{lvv-t88}

\begin{longtable}[]{llllll}
\toprule
Version & Status & Priority & Verification Type & Owner
\\\midrule
1 & Defined & Normal &
Test & Robert Lupton
\\\bottomrule
\multicolumn{6}{c}{ Open \href{https://jira.lsstcorp.org/secure/Tests.jspa\#/testCase/LVV-T88}{LVV-T88} in Jira } \\
\end{longtable}

\subsubsection{Verification Elements}
\begin{itemize}
\item \href{https://jira.lsstcorp.org/browse/LVV-57}{LVV-57} - DMS-REQ-0130-V-01: Calibration Data Products

\end{itemize}

\subsubsection{Test Items}
Verify that the DMS can produce and archive the required Calibration
Data Products: cross talk correction, bias, dark, monochromatic dome
flats, broad-band flats, fringe correction, and illumination
corrections.


\subsubsection{Predecessors}

\subsubsection{Environment Needs}

\paragraph{Software}

\paragraph{Hardware}

\subsubsection{Input Specification}

\subsubsection{Output Specification}

\subsubsection{Test Procedure}
    \begin{longtable}[]{p{1.3cm}p{2cm}p{13cm}}
    %\toprule
    Step & \multicolumn{2}{@{}l}{Description, Input Data and Expected Result} \\ \toprule
    \endhead

            \multirow{3}{*}{ 1 } & Description &
            \begin{minipage}[t]{13cm}{\footnotesize
            Identify a suitable set of calibration frames, including biases, dark
frames, and flat-field frames.

            \vspace{\dp0}
            } \end{minipage} \\ \cline{2-3}
            & Test Data &
            \begin{minipage}[t]{13cm}{\footnotesize
                No data.
                \vspace{\dp0}
            } \end{minipage} \\ \cline{2-3}
            & Expected Result &
        \\ \midrule

                \multirow{3}{*}{\parbox{1.3cm}{ 2-1
                {\scriptsize from \hyperref[lvv-t1060]
                {LVV-T1060} } } }

                & {\small Description} &
                \begin{minipage}[t]{13cm}{\scriptsize
                Execute the Calibration Products Production payload. The payload uses
raw calibration images and information from the Transformed EFD to
generate a subset of Master Calibration Images and Calibration Database
entries in the Data Backbone.

                \vspace{\dp0}
                } \end{minipage} \\ \cdashline{2-3}
                & {\small Test Data} &
                \begin{minipage}[t]{13cm}{\scriptsize
                } \end{minipage} \\ \cdashline{2-3}
                & {\small Expected Result} &
                \\ \hdashline


                \multirow{3}{*}{\parbox{1.3cm}{ 2-2
                {\scriptsize from \hyperref[lvv-t1060]
                {LVV-T1060} } } }

                & {\small Description} &
                \begin{minipage}[t]{13cm}{\scriptsize
                Confirm that the expected Master Calibration images and Calibration
Database entries are present and well-formed.

                \vspace{\dp0}
                } \end{minipage} \\ \cdashline{2-3}
                & {\small Test Data} &
                \begin{minipage}[t]{13cm}{\scriptsize
                } \end{minipage} \\ \cdashline{2-3}
                & {\small Expected Result} &
                \\ \hdashline


        \\ \midrule

            \multirow{3}{*}{ 3 } & Description &
            \begin{minipage}[t]{13cm}{\footnotesize
            Confirm that the expected data products are created, and that they have
the expected properties.

            \vspace{\dp0}
            } \end{minipage} \\ \cline{2-3}
            & Test Data &
            \begin{minipage}[t]{13cm}{\footnotesize
                No data.
                \vspace{\dp0}
            } \end{minipage} \\ \cline{2-3}
            & Expected Result &
                \begin{minipage}[t]{13cm}{\footnotesize
                A full set of calibration data products has been created, and they are
well-formed.

                \vspace{\dp0}
                } \end{minipage}
        \\ \midrule

            \multirow{3}{*}{ 4 } & Description &
            \begin{minipage}[t]{13cm}{\footnotesize
            Test that the calibration products are archived, and can readily be
applied to science data to produce the desired corrections.

            \vspace{\dp0}
            } \end{minipage} \\ \cline{2-3}
            & Test Data &
            \begin{minipage}[t]{13cm}{\footnotesize
                No data.
                \vspace{\dp0}
            } \end{minipage} \\ \cline{2-3}
            & Expected Result &
                \begin{minipage}[t]{13cm}{\footnotesize
                Confirmation that application of the calibration products to processed
data has the desired effects.

                \vspace{\dp0}
                } \end{minipage}
        \\ \midrule
    \end{longtable}

\subsection{LVV-T89 - Verify implementation of Calibration Image Provenance}\label{lvv-t89}

\begin{longtable}[]{llllll}
\toprule
Version & Status & Priority & Verification Type & Owner
\\\midrule
1 & Defined & Normal &
Test & Robert Lupton
\\\bottomrule
\multicolumn{6}{c}{ Open \href{https://jira.lsstcorp.org/secure/Tests.jspa\#/testCase/LVV-T89}{LVV-T89} in Jira } \\
\end{longtable}

\subsubsection{Verification Elements}
\begin{itemize}
\item \href{https://jira.lsstcorp.org/browse/LVV-59}{LVV-59} - DMS-REQ-0132-V-01: Calibration Image Provenance

\item \href{https://jira.lsstcorp.org/browse/LVV-1234}{LVV-1234} - OSS-REQ-0122-V-01: Provenance

\end{itemize}

\subsubsection{Test Items}
Verify that the DMS records the required provenance information for the
Calibration Data Products.


\subsubsection{Predecessors}

\subsubsection{Environment Needs}

\paragraph{Software}

\paragraph{Hardware}

\subsubsection{Input Specification}

\subsubsection{Output Specification}

\subsubsection{Test Procedure}
    \begin{longtable}[]{p{1.3cm}p{2cm}p{13cm}}
    %\toprule
    Step & \multicolumn{2}{@{}l}{Description, Input Data and Expected Result} \\ \toprule
    \endhead

            \multirow{3}{*}{ 1 } & Description &
            \begin{minipage}[t]{13cm}{\footnotesize
            Ingest an appropriate precursor calibration dataset into a Butler repo.

            \vspace{\dp0}
            } \end{minipage} \\ \cline{2-3}
            & Test Data &
            \begin{minipage}[t]{13cm}{\footnotesize
                No data.
                \vspace{\dp0}
            } \end{minipage} \\ \cline{2-3}
            & Expected Result &
        \\ \midrule

                \multirow{3}{*}{\parbox{1.3cm}{ 2-1
                {\scriptsize from \hyperref[lvv-t1060]
                {LVV-T1060} } } }

                & {\small Description} &
                \begin{minipage}[t]{13cm}{\scriptsize
                Execute the Calibration Products Production payload. The payload uses
raw calibration images and information from the Transformed EFD to
generate a subset of Master Calibration Images and Calibration Database
entries in the Data Backbone.

                \vspace{\dp0}
                } \end{minipage} \\ \cdashline{2-3}
                & {\small Test Data} &
                \begin{minipage}[t]{13cm}{\scriptsize
                } \end{minipage} \\ \cdashline{2-3}
                & {\small Expected Result} &
                \\ \hdashline


                \multirow{3}{*}{\parbox{1.3cm}{ 2-2
                {\scriptsize from \hyperref[lvv-t1060]
                {LVV-T1060} } } }

                & {\small Description} &
                \begin{minipage}[t]{13cm}{\scriptsize
                Confirm that the expected Master Calibration images and Calibration
Database entries are present and well-formed.

                \vspace{\dp0}
                } \end{minipage} \\ \cdashline{2-3}
                & {\small Test Data} &
                \begin{minipage}[t]{13cm}{\scriptsize
                } \end{minipage} \\ \cdashline{2-3}
                & {\small Expected Result} &
                \\ \hdashline


        \\ \midrule

            \multirow{3}{*}{ 3 } & Description &
            \begin{minipage}[t]{13cm}{\footnotesize
            Load the relevant database/Butler data product, and observe that all
provenance information has been retained.

            \vspace{\dp0}
            } \end{minipage} \\ \cline{2-3}
            & Test Data &
            \begin{minipage}[t]{13cm}{\footnotesize
                No data.
                \vspace{\dp0}
            } \end{minipage} \\ \cline{2-3}
            & Expected Result &
                \begin{minipage}[t]{13cm}{\footnotesize
                A dataset consisting of calibration images, with provenance information
recorded and properly associated with the calibration images.

                \vspace{\dp0}
                } \end{minipage}
        \\ \midrule
    \end{longtable}

\subsection{LVV-T90 - Verify implementation of Dark Current Correction Frame}\label{lvv-t90}

\begin{longtable}[]{llllll}
\toprule
Version & Status & Priority & Verification Type & Owner
\\\midrule
1 & Defined & Normal &
Test & Robert Lupton
\\\bottomrule
\multicolumn{6}{c}{ Open \href{https://jira.lsstcorp.org/secure/Tests.jspa\#/testCase/LVV-T90}{LVV-T90} in Jira } \\
\end{longtable}

\subsubsection{Verification Elements}
\begin{itemize}
\item \href{https://jira.lsstcorp.org/browse/LVV-113}{LVV-113} - DMS-REQ-0282-V-01: Dark Current Correction Frame Creation

\end{itemize}

\subsubsection{Test Items}
Verify that the DMS can produce a dark correction frame calibration
product.


\subsubsection{Predecessors}

\subsubsection{Environment Needs}

\paragraph{Software}

\paragraph{Hardware}

\subsubsection{Input Specification}

\subsubsection{Output Specification}

\subsubsection{Test Procedure}
    \begin{longtable}[]{p{1.3cm}p{2cm}p{13cm}}
    %\toprule
    Step & \multicolumn{2}{@{}l}{Description, Input Data and Expected Result} \\ \toprule
    \endhead

            \multirow{3}{*}{ 1 } & Description &
            \begin{minipage}[t]{13cm}{\footnotesize
            Identify the path to a dataset containing dark frames (i.e., exposures
taken with the shutter closed).

            \vspace{\dp0}
            } \end{minipage} \\ \cline{2-3}
            & Test Data &
            \begin{minipage}[t]{13cm}{\footnotesize
                No data.
                \vspace{\dp0}
            } \end{minipage} \\ \cline{2-3}
            & Expected Result &
        \\ \midrule

            \multirow{3}{*}{ 2 } & Description &
            \begin{minipage}[t]{13cm}{\footnotesize
            Execute the relevant steps from `cp\_pipe` (the calibration pipeline) to
produce dark correction frames.

            \vspace{\dp0}
            } \end{minipage} \\ \cline{2-3}
            & Test Data &
            \begin{minipage}[t]{13cm}{\footnotesize
                No data.
                \vspace{\dp0}
            } \end{minipage} \\ \cline{2-3}
            & Expected Result &
        \\ \midrule

            \multirow{3}{*}{ 3 } & Description &
            \begin{minipage}[t]{13cm}{\footnotesize
            Inspect the resulting dark correction frame to confirm that it appears
as expected.

            \vspace{\dp0}
            } \end{minipage} \\ \cline{2-3}
            & Test Data &
            \begin{minipage}[t]{13cm}{\footnotesize
                No data.
                \vspace{\dp0}
            } \end{minipage} \\ \cline{2-3}
            & Expected Result &
                \begin{minipage}[t]{13cm}{\footnotesize
                A well-formed dark correction frame is present and accessible via the
Data Butler.

                \vspace{\dp0}
                } \end{minipage}
        \\ \midrule
    \end{longtable}

\subsection{LVV-T91 - Verify implementation of Fringe Correction Frame}\label{lvv-t91}

\begin{longtable}[]{llllll}
\toprule
Version & Status & Priority & Verification Type & Owner
\\\midrule
1 & Draft & Normal &
Test & Robert Lupton
\\\bottomrule
\multicolumn{6}{c}{ Open \href{https://jira.lsstcorp.org/secure/Tests.jspa\#/testCase/LVV-T91}{LVV-T91} in Jira } \\
\end{longtable}

\subsubsection{Verification Elements}
\begin{itemize}
\item \href{https://jira.lsstcorp.org/browse/LVV-114}{LVV-114} - DMS-REQ-0283-V-01: Fringe Correction Frame

\end{itemize}

\subsubsection{Test Items}
Verify that the DMS can produce an fringe-correction frame calibration
product.\\
Verify that the DMS can determine the effectiveness of the
fringe-correction frame and determine how often it should be updated.


\subsubsection{Predecessors}

\subsubsection{Environment Needs}

\paragraph{Software}

\paragraph{Hardware}

\subsubsection{Input Specification}

\subsubsection{Output Specification}

\subsubsection{Test Procedure}
    \begin{longtable}[]{p{1.3cm}p{2cm}p{13cm}}
    %\toprule
    Step & \multicolumn{2}{@{}l}{Description, Input Data and Expected Result} \\ \toprule
    \endhead

            \multirow{3}{*}{ 1 } & Description &
            \begin{minipage}[t]{13cm}{\footnotesize
            Delegate to CPP

            \vspace{\dp0}
            } \end{minipage} \\ \cline{2-3}
            & Test Data &
            \begin{minipage}[t]{13cm}{\footnotesize
                No data.
                \vspace{\dp0}
            } \end{minipage} \\ \cline{2-3}
            & Expected Result &
        \\ \midrule
    \end{longtable}

\subsection{LVV-T92 - Verify implementation of Processing of Data From Special Programs}\label{lvv-t92}

\begin{longtable}[]{llllll}
\toprule
Version & Status & Priority & Verification Type & Owner
\\\midrule
1 & Draft & Normal &
Test & Melissa Graham
\\\bottomrule
\multicolumn{6}{c}{ Open \href{https://jira.lsstcorp.org/secure/Tests.jspa\#/testCase/LVV-T92}{LVV-T92} in Jira } \\
\end{longtable}

\subsubsection{Verification Elements}
\begin{itemize}
\item \href{https://jira.lsstcorp.org/browse/LVV-151}{LVV-151} - DMS-REQ-0320-V-01: Processing of Data From Special Programs

\end{itemize}

\subsubsection{Test Items}
For a simulated night of observing that includes some special program
observations, show that the SP observations are reduced using their
designated reconfigured pipelines (i.e., that the image metadata is
sufficient to trigger the processing and include all other relevant
images in the processing).


\subsubsection{Predecessors}

\subsubsection{Environment Needs}

\paragraph{Software}

\paragraph{Hardware}

\subsubsection{Input Specification}
A variety of imaging data from Special Programs, including these
scenarios:\\
(1) Special Programs data that can be processed by the Prompt pipeline
(i.e., standard visits)\\
(2) Special Programs data that requires `real-time'
(\textasciitilde{}24) processing with a reconfigured pipeline (e.g., DDF
imaging sequence)\\
(3) Special Programs data that can (should) be processed by the Data
Release pipeline (e.g., North Ecliptic Spur standard visits)

\subsubsection{Output Specification}

\subsubsection{Test Procedure}
    \begin{longtable}[]{p{1.3cm}p{2cm}p{13cm}}
    %\toprule
    Step & \multicolumn{2}{@{}l}{Description, Input Data and Expected Result} \\ \toprule
    \endhead

            \multirow{3}{*}{ 1 } & Description &
            \begin{minipage}[t]{13cm}{\footnotesize
            (1) Special Programs data that can be processed by the Prompt pipeline
(i.e., standard visits).\\
Check that all images with the header keyword for SP were processed by
the Prompt pipeline. Check that the Prompt pipeline's data products --
DIASource, DIAObject catalogs and the Alerts -- contain items flagged
with their origin as that SP.

            \vspace{\dp0}
            } \end{minipage} \\ \cline{2-3}
            & Test Data &
            \begin{minipage}[t]{13cm}{\footnotesize
                No data.
                \vspace{\dp0}
            } \end{minipage} \\ \cline{2-3}
            & Expected Result &
        \\ \midrule

            \multirow{3}{*}{ 2 } & Description &
            \begin{minipage}[t]{13cm}{\footnotesize
            (2) Special Programs data that requires `real-time'
(\textasciitilde{}24) processing with a reconfigured pipeline (e.g., DDF
imaging sequence)\\
Check that all images with the header keywords for a given SP were
processed by their reconfigured pipeline. Check that the pipeline's data
products have been updated, and passed their QA.

            \vspace{\dp0}
            } \end{minipage} \\ \cline{2-3}
            & Test Data &
            \begin{minipage}[t]{13cm}{\footnotesize
                No data.
                \vspace{\dp0}
            } \end{minipage} \\ \cline{2-3}
            & Expected Result &
        \\ \midrule

            \multirow{3}{*}{ 3 } & Description &
            \begin{minipage}[t]{13cm}{\footnotesize
            (3) Special Programs data that can (should) be processed by the Data
Release pipeline (e.g., North Ecliptic Spur standard visits).\\
SP data would be added manually to the DRP processing. Check that the
DRP's data products -- Source, Object, CoAdds -- contain items flagged
as originating in that SP.

            \vspace{\dp0}
            } \end{minipage} \\ \cline{2-3}
            & Test Data &
            \begin{minipage}[t]{13cm}{\footnotesize
                No data.
                \vspace{\dp0}
            } \end{minipage} \\ \cline{2-3}
            & Expected Result &
        \\ \midrule
    \end{longtable}

\subsection{LVV-T93 - Verify implementation of Level 1 Processing of Special Programs Data}\label{lvv-t93}

\begin{longtable}[]{llllll}
\toprule
Version & Status & Priority & Verification Type & Owner
\\\midrule
1 & Draft & Normal &
Test & Melissa Graham
\\\bottomrule
\multicolumn{6}{c}{ Open \href{https://jira.lsstcorp.org/secure/Tests.jspa\#/testCase/LVV-T93}{LVV-T93} in Jira } \\
\end{longtable}

\subsubsection{Verification Elements}
\begin{itemize}
\item \href{https://jira.lsstcorp.org/browse/LVV-152}{LVV-152} - DMS-REQ-0321-V-01: Level 1 Processing of Special Programs Data

\end{itemize}

\subsubsection{Test Items}
Execute multi-day operations rehearsal. Observe whether Prompt
Processing data products generated in time and confirm whether
processing has completed before the start of the next simulated night.~


\subsubsection{Predecessors}

\subsubsection{Environment Needs}

\paragraph{Software}

\paragraph{Hardware}

\subsubsection{Input Specification}
Imaging data obtained under a Special Program: for example, a sequence
of consecutive images of a deep drilling field.

\subsubsection{Output Specification}

\subsubsection{Test Procedure}
    \begin{longtable}[]{p{1.3cm}p{2cm}p{13cm}}
    %\toprule
    Step & \multicolumn{2}{@{}l}{Description, Input Data and Expected Result} \\ \toprule
    \endhead

            \multirow{3}{*}{ 1 } & Description &
            \begin{minipage}[t]{13cm}{\footnotesize
            If imaging data for a Special Program that requires processing with the
Prompt pipeline was obtained the previous night, check that there exist
DIASources/Objects/Alerts with flags that they originated from the
Special Program.

            \vspace{\dp0}
            } \end{minipage} \\ \cline{2-3}
            & Test Data &
            \begin{minipage}[t]{13cm}{\footnotesize
                No data.
                \vspace{\dp0}
            } \end{minipage} \\ \cline{2-3}
            & Expected Result &
        \\ \midrule

            \multirow{3}{*}{ 2 } & Description &
            \begin{minipage}[t]{13cm}{\footnotesize
            If imaging data for a Special Program that requires prompt processing
with a reconfigured pipeline was obtained the previous night, check that
the relevant data products have been updated.

            \vspace{\dp0}
            } \end{minipage} \\ \cline{2-3}
            & Test Data &
            \begin{minipage}[t]{13cm}{\footnotesize
                No data.
                \vspace{\dp0}
            } \end{minipage} \\ \cline{2-3}
            & Expected Result &
        \\ \midrule
    \end{longtable}

\subsection{LVV-T94 - Verify implementation of Special Programs Database}\label{lvv-t94}

\begin{longtable}[]{llllll}
\toprule
Version & Status & Priority & Verification Type & Owner
\\\midrule
1 & Draft & Normal &
Test & Melissa Graham
\\\bottomrule
\multicolumn{6}{c}{ Open \href{https://jira.lsstcorp.org/secure/Tests.jspa\#/testCase/LVV-T94}{LVV-T94} in Jira } \\
\end{longtable}

\subsubsection{Verification Elements}
\begin{itemize}
\item \href{https://jira.lsstcorp.org/browse/LVV-153}{LVV-153} - DMS-REQ-0322-V-01: Special Programs Database

\end{itemize}

\subsubsection{Test Items}
To confirm that data products from Special Programs are based solely on
images obtained as part of SP via, e.g., metadata queries. To confirm
that the SP data products can be joined to Prompt and DRP products by
attempting to do so via, e.g., coordinate table joins, and attempting to
e.g., find the faint counterparts in a Deep Drilling stack to variables
with no Object detections in the DRP coadds.


\subsubsection{Predecessors}

\subsubsection{Environment Needs}

\paragraph{Software}

\paragraph{Hardware}

\subsubsection{Input Specification}
Databases created by reconfigured pipelines for processing Special
Programs data (e.g., DIAObject/DIASource catalogs for a Deep Drilling
Field).

\subsubsection{Output Specification}

\subsubsection{Test Procedure}
    \begin{longtable}[]{p{1.3cm}p{2cm}p{13cm}}
    %\toprule
    Step & \multicolumn{2}{@{}l}{Description, Input Data and Expected Result} \\ \toprule
    \endhead

            \multirow{3}{*}{ 1 } & Description &
            \begin{minipage}[t]{13cm}{\footnotesize
            SP data product: DDF DIAObjects catalog\\
Non-SP data product: WFD DIAObjects catalog\\
Test: join the two catalogs by coordinate (e.g., to get a longer time
baseline for variable stars in the DDF)

            \vspace{\dp0}
            } \end{minipage} \\ \cline{2-3}
            & Test Data &
            \begin{minipage}[t]{13cm}{\footnotesize
                No data.
                \vspace{\dp0}
            } \end{minipage} \\ \cline{2-3}
            & Expected Result &
        \\ \midrule

            \multirow{3}{*}{ 2 } & Description &
            \begin{minipage}[t]{13cm}{\footnotesize
            SP data product: DDF Objects catalog\\
Non-SP data product: WFD DIAObjects catalog\\
Test: join the two catalogs by coordinate to identify faint host
galaxies of transients found in WFD

            \vspace{\dp0}
            } \end{minipage} \\ \cline{2-3}
            & Test Data &
            \begin{minipage}[t]{13cm}{\footnotesize
                No data.
                \vspace{\dp0}
            } \end{minipage} \\ \cline{2-3}
            & Expected Result &
        \\ \midrule
    \end{longtable}

\subsection{LVV-T95 - Verify implementation of Constraints on Level 1 Special Program Products
Generation}\label{lvv-t95}

\begin{longtable}[]{llllll}
\toprule
Version & Status & Priority & Verification Type & Owner
\\\midrule
1 & Draft & Normal &
Test & Melissa Graham
\\\bottomrule
\multicolumn{6}{c}{ Open \href{https://jira.lsstcorp.org/secure/Tests.jspa\#/testCase/LVV-T95}{LVV-T95} in Jira } \\
\end{longtable}

\subsubsection{Verification Elements}
\begin{itemize}
\item \href{https://jira.lsstcorp.org/browse/LVV-175}{LVV-175} - DMS-REQ-0004-V-01: Time to L1 public release

\item \href{https://jira.lsstcorp.org/browse/LVV-1276}{LVV-1276} - OSS-REQ-0127-V-01: Level 1 Data Product Availability

\end{itemize}

\subsubsection{Test Items}
Execute single-day operations rehearsal. Observe Prompt Processing data
products generated in time. Confirm that data from Special Programs is
processed with the same latency as required for main survey data:
release of public data within L1publicT and Alerts within OTT1.


\subsubsection{Predecessors}

\subsubsection{Environment Needs}

\paragraph{Software}

\paragraph{Hardware}

\subsubsection{Input Specification}
Data from a Special Program that is appropriate for the Prompt pipeline
(i.e., a Deep Drilling type series of standard visits from a non-crowded
field).

\subsubsection{Output Specification}

\subsubsection{Test Procedure}
    \begin{longtable}[]{p{1.3cm}p{2cm}p{13cm}}
    %\toprule
    Step & \multicolumn{2}{@{}l}{Description, Input Data and Expected Result} \\ \toprule
    \endhead

                \multirow{3}{*}{\parbox{1.3cm}{ 1-1
                {\scriptsize from \hyperref[lvv-t866]
                {LVV-T866} } } }

                & {\small Description} &
                \begin{minipage}[t]{13cm}{\scriptsize
                Perform the steps of Alert Production (including, but not necessarily
limited to, single frame processing, ISR, source detection/measurement,
PSF estimation, photometric and astrometric calibration, difference
imaging, DIASource detection/measurement, source association). During
Operations, it is presumed that these are automated for a given
dataset.~

                \vspace{\dp0}
                } \end{minipage} \\ \cdashline{2-3}
                & {\small Test Data} &
                \begin{minipage}[t]{13cm}{\scriptsize
                } \end{minipage} \\ \cdashline{2-3}
                & {\small Expected Result} &
                    \begin{minipage}[t]{13cm}{\scriptsize
                    An output dataset including difference images and DIASource and
DIAObject measurements.

                    \vspace{\dp0}
                    } \end{minipage}
                \\ \hdashline


                \multirow{3}{*}{\parbox{1.3cm}{ 1-2
                {\scriptsize from \hyperref[lvv-t866]
                {LVV-T866} } } }

                & {\small Description} &
                \begin{minipage}[t]{13cm}{\scriptsize
                Verify that the expected data products have been produced, and that
catalogs contain reasonable values for measured quantities of interest.

                \vspace{\dp0}
                } \end{minipage} \\ \cdashline{2-3}
                & {\small Test Data} &
                \begin{minipage}[t]{13cm}{\scriptsize
                } \end{minipage} \\ \cdashline{2-3}
                & {\small Expected Result} &
                \\ \hdashline


        \\ \midrule

            \multirow{3}{*}{ 2 } & Description &
            \begin{minipage}[t]{13cm}{\footnotesize
            Confirm that Special Program prompt data products have been generated
within 24 hours.

            \vspace{\dp0}
            } \end{minipage} \\ \cline{2-3}
            & Test Data &
            \begin{minipage}[t]{13cm}{\footnotesize
                No data.
                \vspace{\dp0}
            } \end{minipage} \\ \cline{2-3}
            & Expected Result &
        \\ \midrule
    \end{longtable}

\subsection{LVV-T96 - Verify implementation of Query Repeatability}\label{lvv-t96}

\begin{longtable}[]{llllll}
\toprule
Version & Status & Priority & Verification Type & Owner
\\\midrule
1 & Draft & Normal &
Test & Colin Slater
\\\bottomrule
\multicolumn{6}{c}{ Open \href{https://jira.lsstcorp.org/secure/Tests.jspa\#/testCase/LVV-T96}{LVV-T96} in Jira } \\
\end{longtable}

\subsubsection{Verification Elements}
\begin{itemize}
\item \href{https://jira.lsstcorp.org/browse/LVV-122}{LVV-122} - DMS-REQ-0291-V-01: Query Repeatability

\end{itemize}

\subsubsection{Test Items}
Verify that prior queries can be rerun with identical results, or with
new additional data for live (Alert Production) databases.


\subsubsection{Predecessors}

\subsubsection{Environment Needs}

\paragraph{Software}

\paragraph{Hardware}

\subsubsection{Input Specification}

\subsubsection{Output Specification}

\subsubsection{Test Procedure}
    \begin{longtable}[]{p{1.3cm}p{2cm}p{13cm}}
    %\toprule
    Step & \multicolumn{2}{@{}l}{Description, Input Data and Expected Result} \\ \toprule
    \endhead

            \multirow{3}{*}{ 1 } & Description &
            \begin{minipage}[t]{13cm}{\footnotesize
            Select and download (deterministic) random subsample of records from
Data Release Object and Source tables.

            \vspace{\dp0}
            } \end{minipage} \\ \cline{2-3}
            & Test Data &
            \begin{minipage}[t]{13cm}{\footnotesize
                No data.
                \vspace{\dp0}
            } \end{minipage} \\ \cline{2-3}
            & Expected Result &
        \\ \midrule

            \multirow{3}{*}{ 2 } & Description &
            \begin{minipage}[t]{13cm}{\footnotesize
            Select and download random subsample of PPDB DIAObject and DIASource
tables.

            \vspace{\dp0}
            } \end{minipage} \\ \cline{2-3}
            & Test Data &
            \begin{minipage}[t]{13cm}{\footnotesize
                No data.
                \vspace{\dp0}
            } \end{minipage} \\ \cline{2-3}
            & Expected Result &
        \\ \midrule

            \multirow{3}{*}{ 3 } & Description &
            \begin{minipage}[t]{13cm}{\footnotesize
            As appropriate, wait for some amount of non-trivial database usage to
occur, such as Prompt Processing ingestion or ingestion of other DRP
database tables.

            \vspace{\dp0}
            } \end{minipage} \\ \cline{2-3}
            & Test Data &
            \begin{minipage}[t]{13cm}{\footnotesize
                No data.
                \vspace{\dp0}
            } \end{minipage} \\ \cline{2-3}
            & Expected Result &
        \\ \midrule

            \multirow{3}{*}{ 4 } & Description &
            \begin{minipage}[t]{13cm}{\footnotesize
            Re-run the queries in steps 1 and 2 and verify that the resulting data
are identical.

            \vspace{\dp0}
            } \end{minipage} \\ \cline{2-3}
            & Test Data &
            \begin{minipage}[t]{13cm}{\footnotesize
                No data.
                \vspace{\dp0}
            } \end{minipage} \\ \cline{2-3}
            & Expected Result &
        \\ \midrule
    \end{longtable}

\subsection{LVV-T97 - Verify implementation of Uniqueness of IDs Across Data Releases}\label{lvv-t97}

\begin{longtable}[]{llllll}
\toprule
Version & Status & Priority & Verification Type & Owner
\\\midrule
1 & Defined & Normal &
Test & Kian-Tat Lim
\\\bottomrule
\multicolumn{6}{c}{ Open \href{https://jira.lsstcorp.org/secure/Tests.jspa\#/testCase/LVV-T97}{LVV-T97} in Jira } \\
\end{longtable}

\subsubsection{Verification Elements}
\begin{itemize}
\item \href{https://jira.lsstcorp.org/browse/LVV-123}{LVV-123} - DMS-REQ-0292-V-01: Uniqueness of IDs Across Data Releases

\end{itemize}

\subsubsection{Test Items}
Verify that the IDs of Objects, Sources, DIAObjects, and DIASources from
different Data Releases are unique.


\subsubsection{Predecessors}

\subsubsection{Environment Needs}

\paragraph{Software}

\paragraph{Hardware}

\subsubsection{Input Specification}

\subsubsection{Output Specification}

\subsubsection{Test Procedure}
    \begin{longtable}[]{p{1.3cm}p{2cm}p{13cm}}
    %\toprule
    Step & \multicolumn{2}{@{}l}{Description, Input Data and Expected Result} \\ \toprule
    \endhead

            \multirow{3}{*}{ 1 } & Description &
            \begin{minipage}[t]{13cm}{\footnotesize
            Identify an appropriate precursor dataset to be processed through Data
Release Production.

            \vspace{\dp0}
            } \end{minipage} \\ \cline{2-3}
            & Test Data &
            \begin{minipage}[t]{13cm}{\footnotesize
                No data.
                \vspace{\dp0}
            } \end{minipage} \\ \cline{2-3}
            & Expected Result &
        \\ \midrule

                \multirow{3}{*}{\parbox{1.3cm}{ 2-1
                {\scriptsize from \hyperref[lvv-t1064]
                {LVV-T1064} } } }

                & {\small Description} &
                \begin{minipage}[t]{13cm}{\scriptsize
                Process data with the Data Release Production payload, starting from raw
science images and generating science data products, placing them in the
Data Backbone.

                \vspace{\dp0}
                } \end{minipage} \\ \cdashline{2-3}
                & {\small Test Data} &
                \begin{minipage}[t]{13cm}{\scriptsize
                } \end{minipage} \\ \cdashline{2-3}
                & {\small Expected Result} &
                \\ \hdashline


        \\ \midrule

                \multirow{3}{*}{\parbox{1.3cm}{ 3-1
                {\scriptsize from \hyperref[lvv-t987]
                {LVV-T987} } } }

                & {\small Description} &
                \begin{minipage}[t]{13cm}{\scriptsize
                Identify the path to the data repository, which we will refer to as
`DATA/path', then execute the following:

                \vspace{\dp0}
                } \end{minipage} \\ \cdashline{2-3}
                & {\small Test Data} &
                \begin{minipage}[t]{13cm}{\scriptsize
                } \end{minipage} \\ \cdashline{2-3}
                & {\small Expected Result} &
                    \begin{minipage}[t]{13cm}{\scriptsize
                    Butler repo available for reading.

                    \vspace{\dp0}
                    } \end{minipage}
                \\ \hdashline


        \\ \midrule

            \multirow{3}{*}{ 4 } & Description &
            \begin{minipage}[t]{13cm}{\footnotesize
            After running the DRP payload multiple times, load the resulting data
products (both data release and prompt products) using the Butler.

            \vspace{\dp0}
            } \end{minipage} \\ \cline{2-3}
            & Test Data &
            \begin{minipage}[t]{13cm}{\footnotesize
                No data.
                \vspace{\dp0}
            } \end{minipage} \\ \cline{2-3}
            & Expected Result &
                \begin{minipage}[t]{13cm}{\footnotesize
                Multiple datasets resulting from processing of the same input data.

                \vspace{\dp0}
                } \end{minipage}
        \\ \midrule

            \multirow{3}{*}{ 5 } & Description &
            \begin{minipage}[t]{13cm}{\footnotesize
            Inspect the IDs in the multiple data products and confirm that all IDs
are unique.

            \vspace{\dp0}
            } \end{minipage} \\ \cline{2-3}
            & Test Data &
            \begin{minipage}[t]{13cm}{\footnotesize
                No data.
                \vspace{\dp0}
            } \end{minipage} \\ \cline{2-3}
            & Expected Result &
                \begin{minipage}[t]{13cm}{\footnotesize
                No IDs are repeated between multiple processings of the identical input
dataset.

                \vspace{\dp0}
                } \end{minipage}
        \\ \midrule
    \end{longtable}

\subsection{LVV-T98 - Verify implementation of Selection of Datasets}\label{lvv-t98}

\begin{longtable}[]{llllll}
\toprule
Version & Status & Priority & Verification Type & Owner
\\\midrule
1 & Defined & Normal &
Test & Kian-Tat Lim
\\\bottomrule
\multicolumn{6}{c}{ Open \href{https://jira.lsstcorp.org/secure/Tests.jspa\#/testCase/LVV-T98}{LVV-T98} in Jira } \\
\end{longtable}

\subsubsection{Verification Elements}
\begin{itemize}
\item \href{https://jira.lsstcorp.org/browse/LVV-124}{LVV-124} - DMS-REQ-0293-V-01: Selection of Datasets

\end{itemize}

\subsubsection{Test Items}
Verify that the DMS can identify and retrieve datasets consisting of
logical groupings of Exposures, metadata, provenance, etc., or other
groupings that are processed or produced as a logical unit.


\subsubsection{Predecessors}

\subsubsection{Environment Needs}

\paragraph{Software}

\paragraph{Hardware}

\subsubsection{Input Specification}

\subsubsection{Output Specification}

\subsubsection{Test Procedure}
    \begin{longtable}[]{p{1.3cm}p{2cm}p{13cm}}
    %\toprule
    Step & \multicolumn{2}{@{}l}{Description, Input Data and Expected Result} \\ \toprule
    \endhead

                \multirow{3}{*}{\parbox{1.3cm}{ 1-1
                {\scriptsize from \hyperref[lvv-t987]
                {LVV-T987} } } }

                & {\small Description} &
                \begin{minipage}[t]{13cm}{\scriptsize
                Identify the path to the data repository, which we will refer to as
`DATA/path', then execute the following:

                \vspace{\dp0}
                } \end{minipage} \\ \cdashline{2-3}
                & {\small Test Data} &
                \begin{minipage}[t]{13cm}{\scriptsize
                } \end{minipage} \\ \cdashline{2-3}
                & {\small Expected Result} &
                    \begin{minipage}[t]{13cm}{\scriptsize
                    Butler repo available for reading.

                    \vspace{\dp0}
                    } \end{minipage}
                \\ \hdashline


        \\ \midrule

            \multirow{3}{*}{ 2 } & Description &
            \begin{minipage}[t]{13cm}{\footnotesize
            Ingest data from an appropriate processed dataset.

            \vspace{\dp0}
            } \end{minipage} \\ \cline{2-3}
            & Test Data &
            \begin{minipage}[t]{13cm}{\footnotesize
                No data.
                \vspace{\dp0}
            } \end{minipage} \\ \cline{2-3}
            & Expected Result &
        \\ \midrule

            \multirow{3}{*}{ 3 } & Description &
            \begin{minipage}[t]{13cm}{\footnotesize
            Observe retrieval of single Processed Visit Image (PVI) with metadata.

            \vspace{\dp0}
            } \end{minipage} \\ \cline{2-3}
            & Test Data &
            \begin{minipage}[t]{13cm}{\footnotesize
                No data.
                \vspace{\dp0}
            } \end{minipage} \\ \cline{2-3}
            & Expected Result &
                \begin{minipage}[t]{13cm}{\footnotesize
                A PVI and its associated metadata.

                \vspace{\dp0}
                } \end{minipage}
        \\ \midrule

            \multirow{3}{*}{ 4 } & Description &
            \begin{minipage}[t]{13cm}{\footnotesize
            Observe retrieval of multiple PVIs with metadata.

            \vspace{\dp0}
            } \end{minipage} \\ \cline{2-3}
            & Test Data &
            \begin{minipage}[t]{13cm}{\footnotesize
                No data.
                \vspace{\dp0}
            } \end{minipage} \\ \cline{2-3}
            & Expected Result &
                \begin{minipage}[t]{13cm}{\footnotesize
                A set of PVIs and their associated metadata.

                \vspace{\dp0}
                } \end{minipage}
        \\ \midrule

            \multirow{3}{*}{ 5 } & Description &
            \begin{minipage}[t]{13cm}{\footnotesize
            Observe retrieval of coadd patch with metadata and provenance
information.

            \vspace{\dp0}
            } \end{minipage} \\ \cline{2-3}
            & Test Data &
            \begin{minipage}[t]{13cm}{\footnotesize
                No data.
                \vspace{\dp0}
            } \end{minipage} \\ \cline{2-3}
            & Expected Result &
                \begin{minipage}[t]{13cm}{\footnotesize
                An image of coadded data in a patch, along with its metadata and
information describing the provenance of the patch constituents.

                \vspace{\dp0}
                } \end{minipage}
        \\ \midrule

            \multirow{3}{*}{ 6 } & Description &
            \begin{minipage}[t]{13cm}{\footnotesize
            Observe retrieval of subset of rows in each of the above catalogs.

            \vspace{\dp0}
            } \end{minipage} \\ \cline{2-3}
            & Test Data &
            \begin{minipage}[t]{13cm}{\footnotesize
                No data.
                \vspace{\dp0}
            } \end{minipage} \\ \cline{2-3}
            & Expected Result &
        \\ \midrule
    \end{longtable}

\subsection{LVV-T99 - Verify implementation of Processing of Datasets}\label{lvv-t99}

\begin{longtable}[]{llllll}
\toprule
Version & Status & Priority & Verification Type & Owner
\\\midrule
1 & Draft & Normal &
Test & Kian-Tat Lim
\\\bottomrule
\multicolumn{6}{c}{ Open \href{https://jira.lsstcorp.org/secure/Tests.jspa\#/testCase/LVV-T99}{LVV-T99} in Jira } \\
\end{longtable}

\subsubsection{Verification Elements}
\begin{itemize}
\item \href{https://jira.lsstcorp.org/browse/LVV-125}{LVV-125} - DMS-REQ-0294-V-01: Processing of Datasets

\end{itemize}

\subsubsection{Test Items}
Execute AP and DRP, simulate failures, observe correct processing


\subsubsection{Predecessors}

\subsubsection{Environment Needs}

\paragraph{Software}

\paragraph{Hardware}

\subsubsection{Input Specification}

\subsubsection{Output Specification}

\subsubsection{Test Procedure}
    \begin{longtable}[]{p{1.3cm}p{2cm}p{13cm}}
    %\toprule
    Step & \multicolumn{2}{@{}l}{Description, Input Data and Expected Result} \\ \toprule
    \endhead

            \multirow{3}{*}{ 1 } & Description &
            \begin{minipage}[t]{13cm}{\footnotesize
            Execute AP and DRP

            \vspace{\dp0}
            } \end{minipage} \\ \cline{2-3}
            & Test Data &
            \begin{minipage}[t]{13cm}{\footnotesize
                No data.
                \vspace{\dp0}
            } \end{minipage} \\ \cline{2-3}
            & Expected Result &
        \\ \midrule

            \multirow{3}{*}{ 2 } & Description &
            \begin{minipage}[t]{13cm}{\footnotesize
            ~Simulate failures

            \vspace{\dp0}
            } \end{minipage} \\ \cline{2-3}
            & Test Data &
            \begin{minipage}[t]{13cm}{\footnotesize
                No data.
                \vspace{\dp0}
            } \end{minipage} \\ \cline{2-3}
            & Expected Result &
        \\ \midrule

            \multirow{3}{*}{ 3 } & Description &
            \begin{minipage}[t]{13cm}{\footnotesize
            Observe correct processing

            \vspace{\dp0}
            } \end{minipage} \\ \cline{2-3}
            & Test Data &
            \begin{minipage}[t]{13cm}{\footnotesize
                No data.
                \vspace{\dp0}
            } \end{minipage} \\ \cline{2-3}
            & Expected Result &
        \\ \midrule
    \end{longtable}

\subsection{LVV-T100 - Verify implementation of Transparent Data Access}\label{lvv-t100}

\begin{longtable}[]{llllll}
\toprule
Version & Status & Priority & Verification Type & Owner
\\\midrule
1 & Draft & Normal &
Test & Kian-Tat Lim
\\\bottomrule
\multicolumn{6}{c}{ Open \href{https://jira.lsstcorp.org/secure/Tests.jspa\#/testCase/LVV-T100}{LVV-T100} in Jira } \\
\end{longtable}

\subsubsection{Verification Elements}
\begin{itemize}
\item \href{https://jira.lsstcorp.org/browse/LVV-126}{LVV-126} - DMS-REQ-0295-V-01: Transparent Data Access

\end{itemize}

\subsubsection{Test Items}
\textbf{Test Items\\
}\\
Observe dataset retrieval from multiple LSP instances


\subsubsection{Predecessors}

\subsubsection{Environment Needs}

\paragraph{Software}

\paragraph{Hardware}

\subsubsection{Input Specification}

\subsubsection{Output Specification}

\subsubsection{Test Procedure}
    \begin{longtable}[]{p{1.3cm}p{2cm}p{13cm}}
    %\toprule
    Step & \multicolumn{2}{@{}l}{Description, Input Data and Expected Result} \\ \toprule
    \endhead

            \multirow{3}{*}{ 1 } & Description &
            \begin{minipage}[t]{13cm}{\footnotesize
            Observe dataset retrieval from multiple LSP instances

            \vspace{\dp0}
            } \end{minipage} \\ \cline{2-3}
            & Test Data &
            \begin{minipage}[t]{13cm}{\footnotesize
                No data.
                \vspace{\dp0}
            } \end{minipage} \\ \cline{2-3}
            & Expected Result &
        \\ \midrule
    \end{longtable}

\subsection{LVV-T101 - Verify implementation of Transient Alert Distribution}\label{lvv-t101}

\begin{longtable}[]{llllll}
\toprule
Version & Status & Priority & Verification Type & Owner
\\\midrule
1 & Draft & Normal &
Test & Kian-Tat Lim
\\\bottomrule
\multicolumn{6}{c}{ Open \href{https://jira.lsstcorp.org/secure/Tests.jspa\#/testCase/LVV-T101}{LVV-T101} in Jira } \\
\end{longtable}

\subsubsection{Verification Elements}
\begin{itemize}
\item \href{https://jira.lsstcorp.org/browse/LVV-3}{LVV-3} - DMS-REQ-0002-V-01: Transient Alert Distribution

\end{itemize}

\subsubsection{Test Items}
Precursor or simulated data, execute AP, observe distribution to
simulated clients using standard protocols


\subsubsection{Predecessors}

\subsubsection{Environment Needs}

\paragraph{Software}

\paragraph{Hardware}

\subsubsection{Input Specification}
Obtain precursor or simulated data; duplicated by
\href{https://jira.lsstcorp.org/secure/Tests.jspa\#/testCase/LVV-T217}{LVV-T217}
-- delete?

\subsubsection{Output Specification}

\subsubsection{Test Procedure}
    \begin{longtable}[]{p{1.3cm}p{2cm}p{13cm}}
    %\toprule
    Step & \multicolumn{2}{@{}l}{Description, Input Data and Expected Result} \\ \toprule
    \endhead

            \multirow{3}{*}{ 1 } & Description &
            \begin{minipage}[t]{13cm}{\footnotesize
            Execute AP

            \vspace{\dp0}
            } \end{minipage} \\ \cline{2-3}
            & Test Data &
            \begin{minipage}[t]{13cm}{\footnotesize
                No data.
                \vspace{\dp0}
            } \end{minipage} \\ \cline{2-3}
            & Expected Result &
        \\ \midrule

            \multirow{3}{*}{ 2 } & Description &
            \begin{minipage}[t]{13cm}{\footnotesize
            Observe distribution to simulated clients using standard protocols

            \vspace{\dp0}
            } \end{minipage} \\ \cline{2-3}
            & Test Data &
            \begin{minipage}[t]{13cm}{\footnotesize
                No data.
                \vspace{\dp0}
            } \end{minipage} \\ \cline{2-3}
            & Expected Result &
        \\ \midrule
    \end{longtable}

\subsection{LVV-T102 - Verify implementation of Solar System Objects Available Within Specified
Time}\label{lvv-t102}

\begin{longtable}[]{llllll}
\toprule
Version & Status & Priority & Verification Type & Owner
\\\midrule
1 & Draft & Normal &
Test & Kian-Tat Lim
\\\bottomrule
\multicolumn{6}{c}{ Open \href{https://jira.lsstcorp.org/secure/Tests.jspa\#/testCase/LVV-T102}{LVV-T102} in Jira } \\
\end{longtable}

\subsubsection{Verification Elements}
\begin{itemize}
\item \href{https://jira.lsstcorp.org/browse/LVV-36}{LVV-36} - DMS-REQ-0089-V-01: Solar System Objects Available Within Specified Time

\item \href{https://jira.lsstcorp.org/browse/LVV-1276}{LVV-1276} - OSS-REQ-0127-V-01: Level 1 Data Product Availability

\item \href{https://jira.lsstcorp.org/browse/LVV-9803}{LVV-9803} - DMS-REQ-0004-V-03: Time to availability of Solar System Object orbits

\end{itemize}

\subsubsection{Test Items}
Execute single-day operations rehearsal, observe data products generated
in time


\subsubsection{Predecessors}

\subsubsection{Environment Needs}

\paragraph{Software}

\paragraph{Hardware}

\subsubsection{Input Specification}

\subsubsection{Output Specification}

\subsubsection{Test Procedure}
    \begin{longtable}[]{p{1.3cm}p{2cm}p{13cm}}
    %\toprule
    Step & \multicolumn{2}{@{}l}{Description, Input Data and Expected Result} \\ \toprule
    \endhead

            \multirow{3}{*}{ 1 } & Description &
            \begin{minipage}[t]{13cm}{\footnotesize
            Execute single-day operations rehearsal

            \vspace{\dp0}
            } \end{minipage} \\ \cline{2-3}
            & Test Data &
            \begin{minipage}[t]{13cm}{\footnotesize
                No data.
                \vspace{\dp0}
            } \end{minipage} \\ \cline{2-3}
            & Expected Result &
        \\ \midrule

            \multirow{3}{*}{ 2 } & Description &
            \begin{minipage}[t]{13cm}{\footnotesize
            ~Observe data products generated in time

            \vspace{\dp0}
            } \end{minipage} \\ \cline{2-3}
            & Test Data &
            \begin{minipage}[t]{13cm}{\footnotesize
                No data.
                \vspace{\dp0}
            } \end{minipage} \\ \cline{2-3}
            & Expected Result &
        \\ \midrule
    \end{longtable}

\subsection{LVV-T103 - Verify implementation of Generate Data Quality Report Within Specified
Time}\label{lvv-t103}

\begin{longtable}[]{llllll}
\toprule
Version & Status & Priority & Verification Type & Owner
\\\midrule
1 & Defined & Normal &
Test & Kian-Tat Lim
\\\bottomrule
\multicolumn{6}{c}{ Open \href{https://jira.lsstcorp.org/secure/Tests.jspa\#/testCase/LVV-T103}{LVV-T103} in Jira } \\
\end{longtable}

\subsubsection{Verification Elements}
\begin{itemize}
\item \href{https://jira.lsstcorp.org/browse/LVV-38}{LVV-38} - DMS-REQ-0096-V-01: Generate Data Quality Report Within Specified Time

\end{itemize}

\subsubsection{Test Items}
Verify that the DMS can generate a nightly L1 Data Quality Report within
\textbf{dqReportComplTime = 4{[}hour{]}}, in both human- and
machine-readable formats.


\subsubsection{Predecessors}

\subsubsection{Environment Needs}

\paragraph{Software}

\paragraph{Hardware}

\subsubsection{Input Specification}

\subsubsection{Output Specification}

\subsubsection{Test Procedure}
    \begin{longtable}[]{p{1.3cm}p{2cm}p{13cm}}
    %\toprule
    Step & \multicolumn{2}{@{}l}{Description, Input Data and Expected Result} \\ \toprule
    \endhead

            \multirow{3}{*}{ 1 } & Description &
            \begin{minipage}[t]{13cm}{\footnotesize
            Execute single-day operations rehearsal

            \vspace{\dp0}
            } \end{minipage} \\ \cline{2-3}
            & Test Data &
            \begin{minipage}[t]{13cm}{\footnotesize
                No data.
                \vspace{\dp0}
            } \end{minipage} \\ \cline{2-3}
            & Expected Result &
        \\ \midrule

            \multirow{3}{*}{ 2 } & Description &
            \begin{minipage}[t]{13cm}{\footnotesize
            After \textbf{dqReportComplTime = 4{[}hour{]}~}has passed, confirm (via
timestamps) that the data quality report has been generated within
\textbf{dqReportComplTime = 4{[}hour{]},} and that it contains the
correct contents.

            \vspace{\dp0}
            } \end{minipage} \\ \cline{2-3}
            & Test Data &
            \begin{minipage}[t]{13cm}{\footnotesize
                No data.
                \vspace{\dp0}
            } \end{minipage} \\ \cline{2-3}
            & Expected Result &
                \begin{minipage}[t]{13cm}{\footnotesize
                Both human- and machine-readable versions of the L1 Data Quality Report
are available with dqReportComplTime.

                \vspace{\dp0}
                } \end{minipage}
        \\ \midrule
    \end{longtable}

\subsection{LVV-T104 - Verify implementation of Generate DMS Performance Report Within
Specified Time}\label{lvv-t104}

\begin{longtable}[]{llllll}
\toprule
Version & Status & Priority & Verification Type & Owner
\\\midrule
1 & Draft & Normal &
Test & Kian-Tat Lim
\\\bottomrule
\multicolumn{6}{c}{ Open \href{https://jira.lsstcorp.org/secure/Tests.jspa\#/testCase/LVV-T104}{LVV-T104} in Jira } \\
\end{longtable}

\subsubsection{Verification Elements}
\begin{itemize}
\item \href{https://jira.lsstcorp.org/browse/LVV-40}{LVV-40} - DMS-REQ-0098-V-01: Generate DMS Performance Report Within Specified Time

\end{itemize}

\subsubsection{Test Items}
Verify that the DMS can generate a nightly Perfomance Report within
perfReportComplTime


\subsubsection{Predecessors}

\subsubsection{Environment Needs}

\paragraph{Software}

\paragraph{Hardware}

\subsubsection{Input Specification}

\subsubsection{Output Specification}

\subsubsection{Test Procedure}
    \begin{longtable}[]{p{1.3cm}p{2cm}p{13cm}}
    %\toprule
    Step & \multicolumn{2}{@{}l}{Description, Input Data and Expected Result} \\ \toprule
    \endhead

            \multirow{3}{*}{ 1 } & Description &
            \begin{minipage}[t]{13cm}{\footnotesize
            Execute single-day operations rehearsal

            \vspace{\dp0}
            } \end{minipage} \\ \cline{2-3}
            & Test Data &
            \begin{minipage}[t]{13cm}{\footnotesize
                No data.
                \vspace{\dp0}
            } \end{minipage} \\ \cline{2-3}
            & Expected Result &
        \\ \midrule

            \multirow{3}{*}{ 2 } & Description &
            \begin{minipage}[t]{13cm}{\footnotesize
            Observe performance report is generated on time and with correct
contents

            \vspace{\dp0}
            } \end{minipage} \\ \cline{2-3}
            & Test Data &
            \begin{minipage}[t]{13cm}{\footnotesize
                No data.
                \vspace{\dp0}
            } \end{minipage} \\ \cline{2-3}
            & Expected Result &
        \\ \midrule
    \end{longtable}

\subsection{LVV-T105 - Verify implementation of Generate Calibration Report Within Specified
Time}\label{lvv-t105}

\begin{longtable}[]{llllll}
\toprule
Version & Status & Priority & Verification Type & Owner
\\\midrule
1 & Draft & Normal &
Test & Kian-Tat Lim
\\\bottomrule
\multicolumn{6}{c}{ Open \href{https://jira.lsstcorp.org/secure/Tests.jspa\#/testCase/LVV-T105}{LVV-T105} in Jira } \\
\end{longtable}

\subsubsection{Verification Elements}
\begin{itemize}
\item \href{https://jira.lsstcorp.org/browse/LVV-42}{LVV-42} - DMS-REQ-0100-V-01: Generate Calibration Report Within Specified Time

\end{itemize}

\subsubsection{Test Items}
Verify that the DMS can generate a night Calibration Report ~in both
human-readable and machine-parseable forms.


\subsubsection{Predecessors}

\subsubsection{Environment Needs}

\paragraph{Software}

\paragraph{Hardware}

\subsubsection{Input Specification}

\subsubsection{Output Specification}

\subsubsection{Test Procedure}
    \begin{longtable}[]{p{1.3cm}p{2cm}p{13cm}}
    %\toprule
    Step & \multicolumn{2}{@{}l}{Description, Input Data and Expected Result} \\ \toprule
    \endhead

            \multirow{3}{*}{ 1 } & Description &
            \begin{minipage}[t]{13cm}{\footnotesize
            Execute single-day operations rehearsal

            \vspace{\dp0}
            } \end{minipage} \\ \cline{2-3}
            & Test Data &
            \begin{minipage}[t]{13cm}{\footnotesize
                No data.
                \vspace{\dp0}
            } \end{minipage} \\ \cline{2-3}
            & Expected Result &
        \\ \midrule

            \multirow{3}{*}{ 2 } & Description &
            \begin{minipage}[t]{13cm}{\footnotesize
            Observe calibration report is generated on time and with correct
contents

            \vspace{\dp0}
            } \end{minipage} \\ \cline{2-3}
            & Test Data &
            \begin{minipage}[t]{13cm}{\footnotesize
                No data.
                \vspace{\dp0}
            } \end{minipage} \\ \cline{2-3}
            & Expected Result &
        \\ \midrule
    \end{longtable}

\subsection{LVV-T106 - Verify implementation of Calibration Images Available Within Specified
Time}\label{lvv-t106}

\begin{longtable}[]{llllll}
\toprule
Version & Status & Priority & Verification Type & Owner
\\\midrule
1 & Draft & Normal &
Test & Kian-Tat Lim
\\\bottomrule
\multicolumn{6}{c}{ Open \href{https://jira.lsstcorp.org/secure/Tests.jspa\#/testCase/LVV-T106}{LVV-T106} in Jira } \\
\end{longtable}

\subsubsection{Verification Elements}
\begin{itemize}
\item \href{https://jira.lsstcorp.org/browse/LVV-58}{LVV-58} - DMS-REQ-0131-V-01: Time allowed to process calibs

\end{itemize}

\subsubsection{Test Items}
Execute single-day operations rehearsal, observe data products generated


\subsubsection{Predecessors}

\subsubsection{Environment Needs}

\paragraph{Software}

\paragraph{Hardware}

\subsubsection{Input Specification}

\subsubsection{Output Specification}

\subsubsection{Test Procedure}
    \begin{longtable}[]{p{1.3cm}p{2cm}p{13cm}}
    %\toprule
    Step & \multicolumn{2}{@{}l}{Description, Input Data and Expected Result} \\ \toprule
    \endhead

            \multirow{3}{*}{ 1 } & Description &
            \begin{minipage}[t]{13cm}{\footnotesize
            Identify a dataset of raw calibration exposures containing at least
\textbf{nCalExpProc = 25~}exposures. (If it contains more than 25
exposures, use only 25 for the test.)

            \vspace{\dp0}
            } \end{minipage} \\ \cline{2-3}
            & Test Data &
            \begin{minipage}[t]{13cm}{\footnotesize
                No data.
                \vspace{\dp0}
            } \end{minipage} \\ \cline{2-3}
            & Expected Result &
        \\ \midrule

                \multirow{3}{*}{\parbox{1.3cm}{ 2-1
                {\scriptsize from \hyperref[lvv-t1059]
                {LVV-T1059} } } }

                & {\small Description} &
                \begin{minipage}[t]{13cm}{\scriptsize
                Execute the Daily Calibration Products Update payload. The payload uses
raw calibration images and information from the Transformed EFD to
generate a subset of Master Calibration Images and Calibration Database
entries in the Data Backbone.

                \vspace{\dp0}
                } \end{minipage} \\ \cdashline{2-3}
                & {\small Test Data} &
                \begin{minipage}[t]{13cm}{\scriptsize
                } \end{minipage} \\ \cdashline{2-3}
                & {\small Expected Result} &
                \\ \hdashline


                \multirow{3}{*}{\parbox{1.3cm}{ 2-2
                {\scriptsize from \hyperref[lvv-t1059]
                {LVV-T1059} } } }

                & {\small Description} &
                \begin{minipage}[t]{13cm}{\scriptsize
                Confirm that the expected Master Calibration images and Calibration
Database entries are present and well-formed.

                \vspace{\dp0}
                } \end{minipage} \\ \cdashline{2-3}
                & {\small Test Data} &
                \begin{minipage}[t]{13cm}{\scriptsize
                } \end{minipage} \\ \cdashline{2-3}
                & {\small Expected Result} &
                \\ \hdashline


        \\ \midrule

            \multirow{3}{*}{ 3 } & Description &
            \begin{minipage}[t]{13cm}{\footnotesize
            Confirm that the processing completed successfully within
\textbf{calProcTime = 1200 seconds.}

            \vspace{\dp0}
            } \end{minipage} \\ \cline{2-3}
            & Test Data &
            \begin{minipage}[t]{13cm}{\footnotesize
                No data.
                \vspace{\dp0}
            } \end{minipage} \\ \cline{2-3}
            & Expected Result &
                \begin{minipage}[t]{13cm}{\footnotesize
                Calibration products resulting from processed raw calibration exposures
are present within calProcTime, and are well-formed images.

                \vspace{\dp0}
                } \end{minipage}
        \\ \midrule
    \end{longtable}

\subsection{LVV-T107 - Verify implementation of Level-1 Production Completeness}\label{lvv-t107}

\begin{longtable}[]{llllll}
\toprule
Version & Status & Priority & Verification Type & Owner
\\\midrule
1 & Draft & Normal &
Test & Eric Bellm
\\\bottomrule
\multicolumn{6}{c}{ Open \href{https://jira.lsstcorp.org/secure/Tests.jspa\#/testCase/LVV-T107}{LVV-T107} in Jira } \\
\end{longtable}

\subsubsection{Verification Elements}
\begin{itemize}
\item \href{https://jira.lsstcorp.org/browse/LVV-115}{LVV-115} - DMS-REQ-0284-V-01: Level-1 Production Completeness

\end{itemize}

\subsubsection{Test Items}
Verify that the DMS successfully processes all images of sufficiently
quality for processing are eventually processed even after connectivity
failures.


\subsubsection{Predecessors}
\href{https://jira.lsstcorp.org/secure/Tests.jspa\#/testCase/LVV-T284}{LVV-T284}

\subsubsection{Environment Needs}

\paragraph{Software}

\paragraph{Hardware}

\subsubsection{Input Specification}

\subsubsection{Output Specification}

\subsubsection{Test Procedure}
    \begin{longtable}[]{p{1.3cm}p{2cm}p{13cm}}
    %\toprule
    Step & \multicolumn{2}{@{}l}{Description, Input Data and Expected Result} \\ \toprule
    \endhead

            \multirow{3}{*}{ 1 } & Description &
            \begin{minipage}[t]{13cm}{\footnotesize
            Ingest raw data while simulating failures and outages, observe eventual
recovery

            \vspace{\dp0}
            } \end{minipage} \\ \cline{2-3}
            & Test Data &
            \begin{minipage}[t]{13cm}{\footnotesize
                No data.
                \vspace{\dp0}
            } \end{minipage} \\ \cline{2-3}
            & Expected Result &
        \\ \midrule
    \end{longtable}

\subsection{LVV-T108 - Verify implementation of Level 1 Source Association}\label{lvv-t108}

\begin{longtable}[]{llllll}
\toprule
Version & Status & Priority & Verification Type & Owner
\\\midrule
1 & Draft & Normal &
Test & Eric Bellm
\\\bottomrule
\multicolumn{6}{c}{ Open \href{https://jira.lsstcorp.org/secure/Tests.jspa\#/testCase/LVV-T108}{LVV-T108} in Jira } \\
\end{longtable}

\subsubsection{Verification Elements}
\begin{itemize}
\item \href{https://jira.lsstcorp.org/browse/LVV-116}{LVV-116} - DMS-REQ-0285-V-01: Level 1 Source Association

\end{itemize}

\subsubsection{Test Items}
Verify that the DMS associates DIASources into a DIAObject or SSObject.


\subsubsection{Predecessors}

\subsubsection{Environment Needs}

\paragraph{Software}

\paragraph{Hardware}

\subsubsection{Input Specification}

\subsubsection{Output Specification}

\subsubsection{Test Procedure}
    \begin{longtable}[]{p{1.3cm}p{2cm}p{13cm}}
    %\toprule
    Step & \multicolumn{2}{@{}l}{Description, Input Data and Expected Result} \\ \toprule
    \endhead

            \multirow{3}{*}{ 1 } & Description &
            \begin{minipage}[t]{13cm}{\footnotesize
            Delegate to AP

            \vspace{\dp0}
            } \end{minipage} \\ \cline{2-3}
            & Test Data &
            \begin{minipage}[t]{13cm}{\footnotesize
                No data.
                \vspace{\dp0}
            } \end{minipage} \\ \cline{2-3}
            & Expected Result &
        \\ \midrule
    \end{longtable}

\subsection{LVV-T109 - Verify implementation of SSObject Precovery}\label{lvv-t109}

\begin{longtable}[]{llllll}
\toprule
Version & Status & Priority & Verification Type & Owner
\\\midrule
1 & Draft & Normal &
Test & Eric Bellm
\\\bottomrule
\multicolumn{6}{c}{ Open \href{https://jira.lsstcorp.org/secure/Tests.jspa\#/testCase/LVV-T109}{LVV-T109} in Jira } \\
\end{longtable}

\subsubsection{Verification Elements}
\begin{itemize}
\item \href{https://jira.lsstcorp.org/browse/LVV-117}{LVV-117} - DMS-REQ-0286-V-01: SSObject Precovery

\end{itemize}

\subsubsection{Test Items}
Verify that the DMS associates additional DIAObjects (both forward and
back in time) with objects classified as SSObjects.


\subsubsection{Predecessors}

\subsubsection{Environment Needs}

\paragraph{Software}

\paragraph{Hardware}

\subsubsection{Input Specification}

\subsubsection{Output Specification}

\subsubsection{Test Procedure}
    \begin{longtable}[]{p{1.3cm}p{2cm}p{13cm}}
    %\toprule
    Step & \multicolumn{2}{@{}l}{Description, Input Data and Expected Result} \\ \toprule
    \endhead

            \multirow{3}{*}{ 1 } & Description &
            \begin{minipage}[t]{13cm}{\footnotesize
            Delegate to AP

            \vspace{\dp0}
            } \end{minipage} \\ \cline{2-3}
            & Test Data &
            \begin{minipage}[t]{13cm}{\footnotesize
                No data.
                \vspace{\dp0}
            } \end{minipage} \\ \cline{2-3}
            & Expected Result &
        \\ \midrule
    \end{longtable}

\subsection{LVV-T110 - Verify implementation of DIASource Precovery}\label{lvv-t110}

\begin{longtable}[]{llllll}
\toprule
Version & Status & Priority & Verification Type & Owner
\\\midrule
1 & Draft & Normal &
Test & Eric Bellm
\\\bottomrule
\multicolumn{6}{c}{ Open \href{https://jira.lsstcorp.org/secure/Tests.jspa\#/testCase/LVV-T110}{LVV-T110} in Jira } \\
\end{longtable}

\subsubsection{Verification Elements}
\begin{itemize}
\item \href{https://jira.lsstcorp.org/browse/LVV-118}{LVV-118} - DMS-REQ-0287-V-01: Max look-back time for precovery

\end{itemize}

\subsubsection{Test Items}
Verify that DMS performs forced photometry for new DIAObjects at all
available images within the precoveryWindow.


\subsubsection{Predecessors}

\subsubsection{Environment Needs}

\paragraph{Software}

\paragraph{Hardware}

\subsubsection{Input Specification}

\subsubsection{Output Specification}

\subsubsection{Test Procedure}
    \begin{longtable}[]{p{1.3cm}p{2cm}p{13cm}}
    %\toprule
    Step & \multicolumn{2}{@{}l}{Description, Input Data and Expected Result} \\ \toprule
    \endhead

            \multirow{3}{*}{ 1 } & Description &
            \begin{minipage}[t]{13cm}{\footnotesize
            Execute single-day operations rehearsal, observe data products generated
in time

            \vspace{\dp0}
            } \end{minipage} \\ \cline{2-3}
            & Test Data &
            \begin{minipage}[t]{13cm}{\footnotesize
                No data.
                \vspace{\dp0}
            } \end{minipage} \\ \cline{2-3}
            & Expected Result &
        \\ \midrule
    \end{longtable}

\subsection{LVV-T111 - Verify implementation of Use of External Orbit Catalogs}\label{lvv-t111}

\begin{longtable}[]{llllll}
\toprule
Version & Status & Priority & Verification Type & Owner
\\\midrule
1 & Draft & Normal &
Test & Eric Bellm
\\\bottomrule
\multicolumn{6}{c}{ Open \href{https://jira.lsstcorp.org/secure/Tests.jspa\#/testCase/LVV-T111}{LVV-T111} in Jira } \\
\end{longtable}

\subsubsection{Verification Elements}
\begin{itemize}
\item \href{https://jira.lsstcorp.org/browse/LVV-119}{LVV-119} - DMS-REQ-0288-V-01: Use of External Orbit Catalogs

\end{itemize}

\subsubsection{Test Items}
Verify that the DMS can make use of external catalogs to improve
identification of SSObjects.


\subsubsection{Predecessors}

\subsubsection{Environment Needs}

\paragraph{Software}

\paragraph{Hardware}

\subsubsection{Input Specification}

\subsubsection{Output Specification}

\subsubsection{Test Procedure}
    \begin{longtable}[]{p{1.3cm}p{2cm}p{13cm}}
    %\toprule
    Step & \multicolumn{2}{@{}l}{Description, Input Data and Expected Result} \\ \toprule
    \endhead

            \multirow{3}{*}{ 1 } & Description &
            \begin{minipage}[t]{13cm}{\footnotesize
            Delegate to AP

            \vspace{\dp0}
            } \end{minipage} \\ \cline{2-3}
            & Test Data &
            \begin{minipage}[t]{13cm}{\footnotesize
                No data.
                \vspace{\dp0}
            } \end{minipage} \\ \cline{2-3}
            & Expected Result &
        \\ \midrule
    \end{longtable}

\subsection{LVV-T112 - Verify implementation of Alert Filtering Service}\label{lvv-t112}

\begin{longtable}[]{llllll}
\toprule
Version & Status & Priority & Verification Type & Owner
\\\midrule
1 & Defined & Normal &
Test & Eric Bellm
\\\bottomrule
\multicolumn{6}{c}{ Open \href{https://jira.lsstcorp.org/secure/Tests.jspa\#/testCase/LVV-T112}{LVV-T112} in Jira } \\
\end{longtable}

\subsubsection{Verification Elements}
\begin{itemize}
\item \href{https://jira.lsstcorp.org/browse/LVV-173}{LVV-173} - DMS-REQ-0342-V-01: Alert Filtering Service

\end{itemize}

\subsubsection{Test Items}
Verify that user-defined filters can be used to generate a basic alert
filtering service.


\subsubsection{Predecessors}

\subsubsection{Environment Needs}

\paragraph{Software}

\paragraph{Hardware}

\subsubsection{Input Specification}

\subsubsection{Output Specification}

\subsubsection{Test Procedure}
    \begin{longtable}[]{p{1.3cm}p{2cm}p{13cm}}
    %\toprule
    Step & \multicolumn{2}{@{}l}{Description, Input Data and Expected Result} \\ \toprule
    \endhead

            \multirow{3}{*}{ 1 } & Description &
            \begin{minipage}[t]{13cm}{\footnotesize
            Identify a suitable precursor dataset for processing through the Alert
Production pipeline.

            \vspace{\dp0}
            } \end{minipage} \\ \cline{2-3}
            & Test Data &
            \begin{minipage}[t]{13cm}{\footnotesize
                No data.
                \vspace{\dp0}
            } \end{minipage} \\ \cline{2-3}
            & Expected Result &
        \\ \midrule

                \multirow{3}{*}{\parbox{1.3cm}{ 2-1
                {\scriptsize from \hyperref[lvv-t866]
                {LVV-T866} } } }

                & {\small Description} &
                \begin{minipage}[t]{13cm}{\scriptsize
                Perform the steps of Alert Production (including, but not necessarily
limited to, single frame processing, ISR, source detection/measurement,
PSF estimation, photometric and astrometric calibration, difference
imaging, DIASource detection/measurement, source association). During
Operations, it is presumed that these are automated for a given
dataset.~

                \vspace{\dp0}
                } \end{minipage} \\ \cdashline{2-3}
                & {\small Test Data} &
                \begin{minipage}[t]{13cm}{\scriptsize
                } \end{minipage} \\ \cdashline{2-3}
                & {\small Expected Result} &
                    \begin{minipage}[t]{13cm}{\scriptsize
                    An output dataset including difference images and DIASource and
DIAObject measurements.

                    \vspace{\dp0}
                    } \end{minipage}
                \\ \hdashline


                \multirow{3}{*}{\parbox{1.3cm}{ 2-2
                {\scriptsize from \hyperref[lvv-t866]
                {LVV-T866} } } }

                & {\small Description} &
                \begin{minipage}[t]{13cm}{\scriptsize
                Verify that the expected data products have been produced, and that
catalogs contain reasonable values for measured quantities of interest.

                \vspace{\dp0}
                } \end{minipage} \\ \cdashline{2-3}
                & {\small Test Data} &
                \begin{minipage}[t]{13cm}{\scriptsize
                } \end{minipage} \\ \cdashline{2-3}
                & {\small Expected Result} &
                \\ \hdashline


        \\ \midrule

            \multirow{3}{*}{ 3 } & Description &
            \begin{minipage}[t]{13cm}{\footnotesize
            Confirm that alerts are generated, and that an Alert Distribution
service is making them available via a stream.

            \vspace{\dp0}
            } \end{minipage} \\ \cline{2-3}
            & Test Data &
            \begin{minipage}[t]{13cm}{\footnotesize
                No data.
                \vspace{\dp0}
            } \end{minipage} \\ \cline{2-3}
            & Expected Result &
                \begin{minipage}[t]{13cm}{\footnotesize
                Via either a UI or API, confirmation that a stream of alerts are
available.

                \vspace{\dp0}
                } \end{minipage}
        \\ \midrule

            \multirow{3}{*}{ 4 } & Description &
            \begin{minipage}[t]{13cm}{\footnotesize
            Confirm that a UI (or API) exists that allows users to define simple
filters. Define a filter, and observe both the full and the filtered
alert streams to confirm that the filter has reduced the volume of
alerts.

            \vspace{\dp0}
            } \end{minipage} \\ \cline{2-3}
            & Test Data &
            \begin{minipage}[t]{13cm}{\footnotesize
                No data.
                \vspace{\dp0}
            } \end{minipage} \\ \cline{2-3}
            & Expected Result &
                \begin{minipage}[t]{13cm}{\footnotesize
                The user-defined filter has reduced the number of alerts being received
relative to the full stream.

                \vspace{\dp0}
                } \end{minipage}
        \\ \midrule
    \end{longtable}

\subsection{LVV-T113 - Verify implementation of Performance Requirements for LSST Alert
Filtering Service}\label{lvv-t113}

\begin{longtable}[]{llllll}
\toprule
Version & Status & Priority & Verification Type & Owner
\\\midrule
1 & Defined & Normal &
Test & Eric Bellm
\\\bottomrule
\multicolumn{6}{c}{ Open \href{https://jira.lsstcorp.org/secure/Tests.jspa\#/testCase/LVV-T113}{LVV-T113} in Jira } \\
\end{longtable}

\subsubsection{Verification Elements}
\begin{itemize}
\item \href{https://jira.lsstcorp.org/browse/LVV-174}{LVV-174} - DMS-REQ-0343-V-01: Number of full-size alerts

\end{itemize}

\subsubsection{Test Items}
Verify that the DMS alert filter service provides sufficient bandwidth
for \textbf{numBrokerUsers = 100} simultaneously-operating brokers to
receive up to \textbf{numBrokerAlerts = 20} alerts per visit.


\subsubsection{Predecessors}

\subsubsection{Environment Needs}

\paragraph{Software}

\paragraph{Hardware}

\subsubsection{Input Specification}

\subsubsection{Output Specification}

\subsubsection{Test Procedure}
    \begin{longtable}[]{p{1.3cm}p{2cm}p{13cm}}
    %\toprule
    Step & \multicolumn{2}{@{}l}{Description, Input Data and Expected Result} \\ \toprule
    \endhead

            \multirow{3}{*}{ 1 } & Description &
            \begin{minipage}[t]{13cm}{\footnotesize
            Create a simulated alert stream.

            \vspace{\dp0}
            } \end{minipage} \\ \cline{2-3}
            & Test Data &
            \begin{minipage}[t]{13cm}{\footnotesize
                No data.
                \vspace{\dp0}
            } \end{minipage} \\ \cline{2-3}
            & Expected Result &
        \\ \midrule

            \multirow{3}{*}{ 2 } & Description &
            \begin{minipage}[t]{13cm}{\footnotesize
            Simultaneously execute user-defined alert filters for at least
\textbf{numBrokerUsers = 100} users, and confirm that the system
successfully filters the stream as requested. Confirm that the bandwidth
requirement of \textbf{numBrokerAlerts = 20} per user was met.

            \vspace{\dp0}
            } \end{minipage} \\ \cline{2-3}
            & Test Data &
            \begin{minipage}[t]{13cm}{\footnotesize
                No data.
                \vspace{\dp0}
            } \end{minipage} \\ \cline{2-3}
            & Expected Result &
                \begin{minipage}[t]{13cm}{\footnotesize
                All of the (simulated) users successfully receive their requested
filtered alerts, with \textbf{numBrokerAlerts = 20~}per user.

                \vspace{\dp0}
                } \end{minipage}
        \\ \midrule
    \end{longtable}

\subsection{LVV-T114 - Verify implementation of Pre-defined alert filters}\label{lvv-t114}

\begin{longtable}[]{llllll}
\toprule
Version & Status & Priority & Verification Type & Owner
\\\midrule
1 & Defined & Normal &
Test & Eric Bellm
\\\bottomrule
\multicolumn{6}{c}{ Open \href{https://jira.lsstcorp.org/secure/Tests.jspa\#/testCase/LVV-T114}{LVV-T114} in Jira } \\
\end{longtable}

\subsubsection{Verification Elements}
\begin{itemize}
\item \href{https://jira.lsstcorp.org/browse/LVV-179}{LVV-179} - DMS-REQ-0348-V-01: Pre-defined alert filters

\end{itemize}

\subsubsection{Test Items}
Verify that users of the Alert Filtering service can use a predefined
set of filters.


\subsubsection{Predecessors}

\subsubsection{Environment Needs}

\paragraph{Software}

\paragraph{Hardware}

\subsubsection{Input Specification}

\subsubsection{Output Specification}

\subsubsection{Test Procedure}
    \begin{longtable}[]{p{1.3cm}p{2cm}p{13cm}}
    %\toprule
    Step & \multicolumn{2}{@{}l}{Description, Input Data and Expected Result} \\ \toprule
    \endhead

            \multirow{3}{*}{ 1 } & Description &
            \begin{minipage}[t]{13cm}{\footnotesize
            Create a simulated alert stream. Confirm that alerts are generated, and
that an Alert Distribution service is making them available.

            \vspace{\dp0}
            } \end{minipage} \\ \cline{2-3}
            & Test Data &
            \begin{minipage}[t]{13cm}{\footnotesize
                No data.
                \vspace{\dp0}
            } \end{minipage} \\ \cline{2-3}
            & Expected Result &
                \begin{minipage}[t]{13cm}{\footnotesize
                A stream of alerts that is confirmed to be generated and distributed.

                \vspace{\dp0}
                } \end{minipage}
        \\ \midrule

            \multirow{3}{*}{ 2 } & Description &
            \begin{minipage}[t]{13cm}{\footnotesize
            Confirm that a UI (or API) exists that presents users some pre-defined
filters.

            \vspace{\dp0}
            } \end{minipage} \\ \cline{2-3}
            & Test Data &
            \begin{minipage}[t]{13cm}{\footnotesize
                No data.
                \vspace{\dp0}
            } \end{minipage} \\ \cline{2-3}
            & Expected Result &
                \begin{minipage}[t]{13cm}{\footnotesize
                The UI (or API) for accessing alert streams has some pre-defined filters
available for users.

                \vspace{\dp0}
                } \end{minipage}
        \\ \midrule

            \multirow{3}{*}{ 3 } & Description &
            \begin{minipage}[t]{13cm}{\footnotesize
            Select one of the pre-defined filters, and confirm that the results have
been properly filtered.

            \vspace{\dp0}
            } \end{minipage} \\ \cline{2-3}
            & Test Data &
            \begin{minipage}[t]{13cm}{\footnotesize
                No data.
                \vspace{\dp0}
            } \end{minipage} \\ \cline{2-3}
            & Expected Result &
                \begin{minipage}[t]{13cm}{\footnotesize
                After applying the pre-defined filter, the number of alerts has
decreased relative to the raw stream.~

                \vspace{\dp0}
                } \end{minipage}
        \\ \midrule
    \end{longtable}

\subsection{LVV-T115 - Verify implementation of Calibration Production Processing}\label{lvv-t115}

\begin{longtable}[]{llllll}
\toprule
Version & Status & Priority & Verification Type & Owner
\\\midrule
1 & Defined & Normal &
Test & Kian-Tat Lim
\\\bottomrule
\multicolumn{6}{c}{ Open \href{https://jira.lsstcorp.org/secure/Tests.jspa\#/testCase/LVV-T115}{LVV-T115} in Jira } \\
\end{longtable}

\subsubsection{Verification Elements}
\begin{itemize}
\item \href{https://jira.lsstcorp.org/browse/LVV-120}{LVV-120} - DMS-REQ-0289-V-01: Calibration Production Processing

\end{itemize}

\subsubsection{Test Items}
Execute CPP on a variety of representative cadences, and verify that the
calibration pipeline correctly produces necessary calibration products.


\subsubsection{Predecessors}

\subsubsection{Environment Needs}

\paragraph{Software}

\paragraph{Hardware}

\subsubsection{Input Specification}

\subsubsection{Output Specification}

\subsubsection{Test Procedure}
    \begin{longtable}[]{p{1.3cm}p{2cm}p{13cm}}
    %\toprule
    Step & \multicolumn{2}{@{}l}{Description, Input Data and Expected Result} \\ \toprule
    \endhead

            \multirow{3}{*}{ 1 } & Description &
            \begin{minipage}[t]{13cm}{\footnotesize
            Identify a suitable set of calibration frames, including biases, dark
frames, and flat-field frames.

            \vspace{\dp0}
            } \end{minipage} \\ \cline{2-3}
            & Test Data &
            \begin{minipage}[t]{13cm}{\footnotesize
                No data.
                \vspace{\dp0}
            } \end{minipage} \\ \cline{2-3}
            & Expected Result &
        \\ \midrule

                \multirow{3}{*}{\parbox{1.3cm}{ 2-1
                {\scriptsize from \hyperref[lvv-t1060]
                {LVV-T1060} } } }

                & {\small Description} &
                \begin{minipage}[t]{13cm}{\scriptsize
                Execute the Calibration Products Production payload. The payload uses
raw calibration images and information from the Transformed EFD to
generate a subset of Master Calibration Images and Calibration Database
entries in the Data Backbone.

                \vspace{\dp0}
                } \end{minipage} \\ \cdashline{2-3}
                & {\small Test Data} &
                \begin{minipage}[t]{13cm}{\scriptsize
                } \end{minipage} \\ \cdashline{2-3}
                & {\small Expected Result} &
                \\ \hdashline


                \multirow{3}{*}{\parbox{1.3cm}{ 2-2
                {\scriptsize from \hyperref[lvv-t1060]
                {LVV-T1060} } } }

                & {\small Description} &
                \begin{minipage}[t]{13cm}{\scriptsize
                Confirm that the expected Master Calibration images and Calibration
Database entries are present and well-formed.

                \vspace{\dp0}
                } \end{minipage} \\ \cdashline{2-3}
                & {\small Test Data} &
                \begin{minipage}[t]{13cm}{\scriptsize
                } \end{minipage} \\ \cdashline{2-3}
                & {\small Expected Result} &
                \\ \hdashline


        \\ \midrule

            \multirow{3}{*}{ 3 } & Description &
            \begin{minipage}[t]{13cm}{\footnotesize
            Confirm that the expected data products are created, and that they have
the expected properties.

            \vspace{\dp0}
            } \end{minipage} \\ \cline{2-3}
            & Test Data &
            \begin{minipage}[t]{13cm}{\footnotesize
                No data.
                \vspace{\dp0}
            } \end{minipage} \\ \cline{2-3}
            & Expected Result &
                \begin{minipage}[t]{13cm}{\footnotesize
                Repos containing valid calibration products that are well-formed and
ready to be applied to processed datasets.

                \vspace{\dp0}
                } \end{minipage}
        \\ \midrule
    \end{longtable}

\subsection{LVV-T116 - Verify implementation of Associating Objects across data releases}\label{lvv-t116}

\begin{longtable}[]{llllll}
\toprule
Version & Status & Priority & Verification Type & Owner
\\\midrule
1 & Draft & Normal &
Test & Kian-Tat Lim
\\\bottomrule
\multicolumn{6}{c}{ Open \href{https://jira.lsstcorp.org/secure/Tests.jspa\#/testCase/LVV-T116}{LVV-T116} in Jira } \\
\end{longtable}

\subsubsection{Verification Elements}
\begin{itemize}
\item \href{https://jira.lsstcorp.org/browse/LVV-181}{LVV-181} - DMS-REQ-0350-V-01: Associating Objects across data releases

\end{itemize}

\subsubsection{Test Items}
Load DR, observe queryable association


\subsubsection{Predecessors}

\subsubsection{Environment Needs}

\paragraph{Software}

\paragraph{Hardware}

\subsubsection{Input Specification}

\subsubsection{Output Specification}

\subsubsection{Test Procedure}
    \begin{longtable}[]{p{1.3cm}p{2cm}p{13cm}}
    %\toprule
    Step & \multicolumn{2}{@{}l}{Description, Input Data and Expected Result} \\ \toprule
    \endhead

            \multirow{3}{*}{ 1 } & Description &
            \begin{minipage}[t]{13cm}{\footnotesize
            Load DR

            \vspace{\dp0}
            } \end{minipage} \\ \cline{2-3}
            & Test Data &
            \begin{minipage}[t]{13cm}{\footnotesize
                No data.
                \vspace{\dp0}
            } \end{minipage} \\ \cline{2-3}
            & Expected Result &
        \\ \midrule

            \multirow{3}{*}{ 2 } & Description &
            \begin{minipage}[t]{13cm}{\footnotesize
            Observe queryable association

            \vspace{\dp0}
            } \end{minipage} \\ \cline{2-3}
            & Test Data &
            \begin{minipage}[t]{13cm}{\footnotesize
                No data.
                \vspace{\dp0}
            } \end{minipage} \\ \cline{2-3}
            & Expected Result &
        \\ \midrule
    \end{longtable}

\subsection{LVV-T117 - Verify implementation of DAC resource allocation for Level 3 processing}\label{lvv-t117}

\begin{longtable}[]{llllll}
\toprule
Version & Status & Priority & Verification Type & Owner
\\\midrule
1 & Draft & Normal &
Test & Colin Slater
\\\bottomrule
\multicolumn{6}{c}{ Open \href{https://jira.lsstcorp.org/secure/Tests.jspa\#/testCase/LVV-T117}{LVV-T117} in Jira } \\
\end{longtable}

\subsubsection{Verification Elements}
\begin{itemize}
\item \href{https://jira.lsstcorp.org/browse/LVV-47}{LVV-47} - DMS-REQ-0119-V-01: DAC resource allocation for Level 3 processing

\end{itemize}

\subsubsection{Test Items}
Verify that compute time and storage space allocations can be granted to
science users.


\subsubsection{Predecessors}

\subsubsection{Environment Needs}

\paragraph{Software}

\paragraph{Hardware}

\subsubsection{Input Specification}

\subsubsection{Output Specification}

\subsubsection{Test Procedure}
    \begin{longtable}[]{p{1.3cm}p{2cm}p{13cm}}
    %\toprule
    Step & \multicolumn{2}{@{}l}{Description, Input Data and Expected Result} \\ \toprule
    \endhead

            \multirow{3}{*}{ 1 } & Description &
            \begin{minipage}[t]{13cm}{\footnotesize
            Create a test user account for the Science Platform.

            \vspace{\dp0}
            } \end{minipage} \\ \cline{2-3}
            & Test Data &
            \begin{minipage}[t]{13cm}{\footnotesize
                No data.
                \vspace{\dp0}
            } \end{minipage} \\ \cline{2-3}
            & Expected Result &
        \\ \midrule

            \multirow{3}{*}{ 2 } & Description &
            \begin{minipage}[t]{13cm}{\footnotesize
            Set the LSP resource allocations for the test user to very low values.

            \vspace{\dp0}
            } \end{minipage} \\ \cline{2-3}
            & Test Data &
            \begin{minipage}[t]{13cm}{\footnotesize
                No data.
                \vspace{\dp0}
            } \end{minipage} \\ \cline{2-3}
            & Expected Result &
        \\ \midrule

            \multirow{3}{*}{ 3 } & Description &
            \begin{minipage}[t]{13cm}{\footnotesize
            Initiate example batch jobs and notebook sessions that will exceed the
specified resource limits.

            \vspace{\dp0}
            } \end{minipage} \\ \cline{2-3}
            & Test Data &
            \begin{minipage}[t]{13cm}{\footnotesize
                No data.
                \vspace{\dp0}
            } \end{minipage} \\ \cline{2-3}
            & Expected Result &
                \begin{minipage}[t]{13cm}{\footnotesize
                Quota error.

                \vspace{\dp0}
                } \end{minipage}
        \\ \midrule

            \multirow{3}{*}{ 4 } & Description &
            \begin{minipage}[t]{13cm}{\footnotesize
            Transfer sufficient data volumes into the user workspace and MyDB tables
that would exceed the resource quotas.

            \vspace{\dp0}
            } \end{minipage} \\ \cline{2-3}
            & Test Data &
            \begin{minipage}[t]{13cm}{\footnotesize
                No data.
                \vspace{\dp0}
            } \end{minipage} \\ \cline{2-3}
            & Expected Result &
                \begin{minipage}[t]{13cm}{\footnotesize
                Quota error.

                \vspace{\dp0}
                } \end{minipage}
        \\ \midrule

            \multirow{3}{*}{ 5 } & Description &
            \begin{minipage}[t]{13cm}{\footnotesize
            Reset the user resource quotas to normal values.

            \vspace{\dp0}
            } \end{minipage} \\ \cline{2-3}
            & Test Data &
            \begin{minipage}[t]{13cm}{\footnotesize
                No data.
                \vspace{\dp0}
            } \end{minipage} \\ \cline{2-3}
            & Expected Result &
        \\ \midrule

            \multirow{3}{*}{ 6 } & Description &
            \begin{minipage}[t]{13cm}{\footnotesize
            Initiate the same example batch jobs and notebook sessions that
previously caused an error.

            \vspace{\dp0}
            } \end{minipage} \\ \cline{2-3}
            & Test Data &
            \begin{minipage}[t]{13cm}{\footnotesize
                No data.
                \vspace{\dp0}
            } \end{minipage} \\ \cline{2-3}
            & Expected Result &
                \begin{minipage}[t]{13cm}{\footnotesize
                Successful notebook and batch job execution.

                \vspace{\dp0}
                } \end{minipage}
        \\ \midrule

            \multirow{3}{*}{ 7 } & Description &
            \begin{minipage}[t]{13cm}{\footnotesize
            Transfer the same data volumes into the user workspace and MyDB tables
that previously caused an error.

            \vspace{\dp0}
            } \end{minipage} \\ \cline{2-3}
            & Test Data &
            \begin{minipage}[t]{13cm}{\footnotesize
                No data.
                \vspace{\dp0}
            } \end{minipage} \\ \cline{2-3}
            & Expected Result &
                \begin{minipage}[t]{13cm}{\footnotesize
                Successful data transfer.

                \vspace{\dp0}
                } \end{minipage}
        \\ \midrule
    \end{longtable}

\subsection{LVV-T118 - Verify implementation of Level 3 Data Product Self Consistency}\label{lvv-t118}

\begin{longtable}[]{llllll}
\toprule
Version & Status & Priority & Verification Type & Owner
\\\midrule
1 & Draft & Normal &
Test & Colin Slater
\\\bottomrule
\multicolumn{6}{c}{ Open \href{https://jira.lsstcorp.org/secure/Tests.jspa\#/testCase/LVV-T118}{LVV-T118} in Jira } \\
\end{longtable}

\subsubsection{Verification Elements}
\begin{itemize}
\item \href{https://jira.lsstcorp.org/browse/LVV-48}{LVV-48} - DMS-REQ-0120-V-01: Level 3 Data Product Self Consistency

\end{itemize}

\subsubsection{Test Items}
Verify that user-driven Level 3 processing is conducted on consistent
sets of input data.


\subsubsection{Predecessors}

\subsubsection{Environment Needs}

\paragraph{Software}

\paragraph{Hardware}

\subsubsection{Input Specification}

\subsubsection{Output Specification}

\subsubsection{Test Procedure}
    \begin{longtable}[]{p{1.3cm}p{2cm}p{13cm}}
    %\toprule
    Step & \multicolumn{2}{@{}l}{Description, Input Data and Expected Result} \\ \toprule
    \endhead

            \multirow{3}{*}{ 1 } & Description &
            \begin{minipage}[t]{13cm}{\footnotesize
            Execute representative processing on DR in PDAC, observe consistency

            \vspace{\dp0}
            } \end{minipage} \\ \cline{2-3}
            & Test Data &
            \begin{minipage}[t]{13cm}{\footnotesize
                No data.
                \vspace{\dp0}
            } \end{minipage} \\ \cline{2-3}
            & Expected Result &
        \\ \midrule
    \end{longtable}

\subsection{LVV-T119 - Verify implementation of Provenance for Level 3 processing at DACs}\label{lvv-t119}

\begin{longtable}[]{llllll}
\toprule
Version & Status & Priority & Verification Type & Owner
\\\midrule
1 & Draft & Normal &
Test & Colin Slater
\\\bottomrule
\multicolumn{6}{c}{ Open \href{https://jira.lsstcorp.org/secure/Tests.jspa\#/testCase/LVV-T119}{LVV-T119} in Jira } \\
\end{longtable}

\subsubsection{Verification Elements}
\begin{itemize}
\item \href{https://jira.lsstcorp.org/browse/LVV-49}{LVV-49} - DMS-REQ-0121-V-01: Provenance for Level 3 processing at DACs

\item \href{https://jira.lsstcorp.org/browse/LVV-1234}{LVV-1234} - OSS-REQ-0122-V-01: Provenance

\end{itemize}

\subsubsection{Test Items}
Verify that provenance information is recorded and accessible for
user-generated Level 3 products.


\subsubsection{Predecessors}

\subsubsection{Environment Needs}

\paragraph{Software}

\paragraph{Hardware}

\subsubsection{Input Specification}

\subsubsection{Output Specification}

\subsubsection{Test Procedure}
    \begin{longtable}[]{p{1.3cm}p{2cm}p{13cm}}
    %\toprule
    Step & \multicolumn{2}{@{}l}{Description, Input Data and Expected Result} \\ \toprule
    \endhead

            \multirow{3}{*}{ 1 } & Description &
            \begin{minipage}[t]{13cm}{\footnotesize
            Execute representative processing on DR in PDAC, observe provenance
recording

            \vspace{\dp0}
            } \end{minipage} \\ \cline{2-3}
            & Test Data &
            \begin{minipage}[t]{13cm}{\footnotesize
                No data.
                \vspace{\dp0}
            } \end{minipage} \\ \cline{2-3}
            & Expected Result &
        \\ \midrule
    \end{longtable}

\subsection{LVV-T120 - Verify implementation of Software framework for Level 3 catalog
processing}\label{lvv-t120}

\begin{longtable}[]{llllll}
\toprule
Version & Status & Priority & Verification Type & Owner
\\\midrule
1 & Draft & Normal &
Test & Colin Slater
\\\bottomrule
\multicolumn{6}{c}{ Open \href{https://jira.lsstcorp.org/secure/Tests.jspa\#/testCase/LVV-T120}{LVV-T120} in Jira } \\
\end{longtable}

\subsubsection{Verification Elements}
\begin{itemize}
\item \href{https://jira.lsstcorp.org/browse/LVV-53}{LVV-53} - DMS-REQ-0125-V-01: Software framework for Level 3 catalog processing

\end{itemize}

\subsubsection{Test Items}
Verify that user-driven Level 3 processing can be consistently applied
to all records in a catalog.


\subsubsection{Predecessors}

\subsubsection{Environment Needs}

\paragraph{Software}

\paragraph{Hardware}

\subsubsection{Input Specification}

\subsubsection{Output Specification}

\subsubsection{Test Procedure}
    \begin{longtable}[]{p{1.3cm}p{2cm}p{13cm}}
    %\toprule
    Step & \multicolumn{2}{@{}l}{Description, Input Data and Expected Result} \\ \toprule
    \endhead

            \multirow{3}{*}{ 1 } & Description &
            \begin{minipage}[t]{13cm}{\footnotesize
            Execute representative processing on DR in PDAC, observe recognition of
and recovery from failures

            \vspace{\dp0}
            } \end{minipage} \\ \cline{2-3}
            & Test Data &
            \begin{minipage}[t]{13cm}{\footnotesize
                No data.
                \vspace{\dp0}
            } \end{minipage} \\ \cline{2-3}
            & Expected Result &
        \\ \midrule
    \end{longtable}

\subsection{LVV-T121 - Verify implementation of Software framework for Level 3 image processing}\label{lvv-t121}

\begin{longtable}[]{llllll}
\toprule
Version & Status & Priority & Verification Type & Owner
\\\midrule
1 & Draft & Normal &
Test & Colin Slater
\\\bottomrule
\multicolumn{6}{c}{ Open \href{https://jira.lsstcorp.org/secure/Tests.jspa\#/testCase/LVV-T121}{LVV-T121} in Jira } \\
\end{longtable}

\subsubsection{Verification Elements}
\begin{itemize}
\item \href{https://jira.lsstcorp.org/browse/LVV-56}{LVV-56} - DMS-REQ-0128-V-01: Software framework for Level 3 image processing

\end{itemize}

\subsubsection{Test Items}
Verify that user-specified Level 3 processing can be applied to the
desired set of images.


\subsubsection{Predecessors}

\subsubsection{Environment Needs}

\paragraph{Software}

\paragraph{Hardware}

\subsubsection{Input Specification}

\subsubsection{Output Specification}

\subsubsection{Test Procedure}
    \begin{longtable}[]{p{1.3cm}p{2cm}p{13cm}}
    %\toprule
    Step & \multicolumn{2}{@{}l}{Description, Input Data and Expected Result} \\ \toprule
    \endhead

            \multirow{3}{*}{ 1 } & Description &
            \begin{minipage}[t]{13cm}{\footnotesize
            Execute representative processing on DR in PDAC, observe recognition of
and recovery from failures

            \vspace{\dp0}
            } \end{minipage} \\ \cline{2-3}
            & Test Data &
            \begin{minipage}[t]{13cm}{\footnotesize
                No data.
                \vspace{\dp0}
            } \end{minipage} \\ \cline{2-3}
            & Expected Result &
        \\ \midrule
    \end{longtable}

\subsection{LVV-T122 - Verify implementation of Level 3 Data Import}\label{lvv-t122}

\begin{longtable}[]{llllll}
\toprule
Version & Status & Priority & Verification Type & Owner
\\\midrule
1 & Draft & Normal &
Test & Colin Slater
\\\bottomrule
\multicolumn{6}{c}{ Open \href{https://jira.lsstcorp.org/secure/Tests.jspa\#/testCase/LVV-T122}{LVV-T122} in Jira } \\
\end{longtable}

\subsubsection{Verification Elements}
\begin{itemize}
\item \href{https://jira.lsstcorp.org/browse/LVV-121}{LVV-121} - DMS-REQ-0290-V-01: Level 3 Data Import

\end{itemize}

\subsubsection{Test Items}
Verify that the Science Platform can ingest data from community-standard
file formats.


\subsubsection{Predecessors}

\subsubsection{Environment Needs}

\paragraph{Software}

\paragraph{Hardware}

\subsubsection{Input Specification}

\subsubsection{Output Specification}

\subsubsection{Test Procedure}
    \begin{longtable}[]{p{1.3cm}p{2cm}p{13cm}}
    %\toprule
    Step & \multicolumn{2}{@{}l}{Description, Input Data and Expected Result} \\ \toprule
    \endhead

            \multirow{3}{*}{ 1 } & Description &
            \begin{minipage}[t]{13cm}{\footnotesize
            Use the Science Platform catalog upload tool to ingest a small example
FITS table.

            \vspace{\dp0}
            } \end{minipage} \\ \cline{2-3}
            & Test Data &
            \begin{minipage}[t]{13cm}{\footnotesize
                No data.
                \vspace{\dp0}
            } \end{minipage} \\ \cline{2-3}
            & Expected Result &
        \\ \midrule

            \multirow{3}{*}{ 2 } & Description &
            \begin{minipage}[t]{13cm}{\footnotesize
            Use the Science Platform catalog upload tool to ingest a small example
CSV table.

            \vspace{\dp0}
            } \end{minipage} \\ \cline{2-3}
            & Test Data &
            \begin{minipage}[t]{13cm}{\footnotesize
                No data.
                \vspace{\dp0}
            } \end{minipage} \\ \cline{2-3}
            & Expected Result &
        \\ \midrule

            \multirow{3}{*}{ 3 } & Description &
            \begin{minipage}[t]{13cm}{\footnotesize
            Use the Science Platform catalog upload tool to ingest a large FITS
table that needs to be spatially-sharded in the database.

            \vspace{\dp0}
            } \end{minipage} \\ \cline{2-3}
            & Test Data &
            \begin{minipage}[t]{13cm}{\footnotesize
                No data.
                \vspace{\dp0}
            } \end{minipage} \\ \cline{2-3}
            & Expected Result &
        \\ \midrule

            \multirow{3}{*}{ 4 } & Description &
            \begin{minipage}[t]{13cm}{\footnotesize
            Perform example queries on each of the three tables to verify that all
data is present.

            \vspace{\dp0}
            } \end{minipage} \\ \cline{2-3}
            & Test Data &
            \begin{minipage}[t]{13cm}{\footnotesize
                No data.
                \vspace{\dp0}
            } \end{minipage} \\ \cline{2-3}
            & Expected Result &
                \begin{minipage}[t]{13cm}{\footnotesize
                Data returned in the queries is identical to the data uploaded.

                \vspace{\dp0}
                } \end{minipage}
        \\ \midrule
    \end{longtable}

\subsection{LVV-T123 - Verify implementation of Access Controls of Level 3 Data Products}\label{lvv-t123}

\begin{longtable}[]{llllll}
\toprule
Version & Status & Priority & Verification Type & Owner
\\\midrule
1 & Draft & Normal &
Test & Robert Gruendl
\\\bottomrule
\multicolumn{6}{c}{ Open \href{https://jira.lsstcorp.org/secure/Tests.jspa\#/testCase/LVV-T123}{LVV-T123} in Jira } \\
\end{longtable}

\subsubsection{Verification Elements}
\begin{itemize}
\item \href{https://jira.lsstcorp.org/browse/LVV-171}{LVV-171} - DMS-REQ-0340-V-01: Access Controls of Level 3 Data Products

\end{itemize}

\subsubsection{Test Items}
This test touches upon the interface between the following areas: IT
Security, Identity Management, LSP Portal, and Parallel Distributed
Database. ~The purpose is to show that access to user generated data
products (previously Level 3) can have a variety of access restrictions
varying from single-user, a list, a named group, or open access.


\subsubsection{Predecessors}

\subsubsection{Environment Needs}

\paragraph{Software}

\paragraph{Hardware}

\subsubsection{Input Specification}

\subsubsection{Output Specification}

\subsubsection{Test Procedure}
    \begin{longtable}[]{p{1.3cm}p{2cm}p{13cm}}
    %\toprule
    Step & \multicolumn{2}{@{}l}{Description, Input Data and Expected Result} \\ \toprule
    \endhead

            \multirow{3}{*}{ 1 } & Description &
            \begin{minipage}[t]{13cm}{\footnotesize
            Configure representative access controls in PDAC, observe proper
restrictions

            \vspace{\dp0}
            } \end{minipage} \\ \cline{2-3}
            & Test Data &
            \begin{minipage}[t]{13cm}{\footnotesize
                No data.
                \vspace{\dp0}
            } \end{minipage} \\ \cline{2-3}
            & Expected Result &
        \\ \midrule
    \end{longtable}

\subsection{LVV-T124 - Verify implementation of Software Architecture to Enable Community
Re-Use}\label{lvv-t124}

\begin{longtable}[]{llllll}
\toprule
Version & Status & Priority & Verification Type & Owner
\\\midrule
1 & Defined & Normal &
Test & Simon Krughoff
\\\bottomrule
\multicolumn{6}{c}{ Open \href{https://jira.lsstcorp.org/secure/Tests.jspa\#/testCase/LVV-T124}{LVV-T124} in Jira } \\
\end{longtable}

\subsubsection{Verification Elements}
\begin{itemize}
\item \href{https://jira.lsstcorp.org/browse/LVV-139}{LVV-139} - DMS-REQ-0308-V-01: Software Architecture to Enable Community Re-Use

\end{itemize}

\subsubsection{Test Items}
Show that the LSST software is capable of being executed in multiple
contexts: single user instance, batch processing, continuous
integration.\\
Also show that the algorithms can be reconfigured and, if desired,
completely replaced at run time.


\subsubsection{Predecessors}

\subsubsection{Environment Needs}

\paragraph{Software}

\paragraph{Hardware}

\subsubsection{Input Specification}

\subsubsection{Output Specification}

\subsubsection{Test Procedure}
    \begin{longtable}[]{p{1.3cm}p{2cm}p{13cm}}
    %\toprule
    Step & \multicolumn{2}{@{}l}{Description, Input Data and Expected Result} \\ \toprule
    \endhead

                \multirow{3}{*}{\parbox{1.3cm}{ 1-1
                {\scriptsize from \hyperref[lvv-t860]
                {LVV-T860} } } }

                & {\small Description} &
                \begin{minipage}[t]{13cm}{\scriptsize
                The `path` that you will use depends on where you are running the
science pipelines. Options:\\[2\baselineskip]

\begin{itemize}
\tightlist
\item
  local (newinstall.sh - based
  install):{[}path\_to\_installation{]}/loadLSST.bash
\item
  development cluster (``lsst-dev''):
  /software/lsstsw/stack/loadLSST.bash
\item
  LSP Notebook aspect (from a terminal):
  /opt/lsst/software/stack/loadLSST.bash
\end{itemize}

From the command line, execute the commands below in the example
code:\\[2\baselineskip]

                \vspace{\dp0}
                } \end{minipage} \\ \cdashline{2-3}
                & {\small Test Data} &
                \begin{minipage}[t]{13cm}{\scriptsize
                } \end{minipage} \\ \cdashline{2-3}
                & {\small Expected Result} &
                    \begin{minipage}[t]{13cm}{\scriptsize
                    Science pipeline software is available for use. If additional packages
are needed (for example, `obs' packages such as `obs\_subaru`), then
additional `setup` commands will be necessary.\\[2\baselineskip]To check
versions in use, type:\\
eups list -s

                    \vspace{\dp0}
                    } \end{minipage}
                \\ \hdashline


        \\ \midrule

            \multirow{3}{*}{ 2 } & Description &
            \begin{minipage}[t]{13cm}{\footnotesize
            Using curated test datasets for multiple precursor instruments, verify
and log that the prototype DRP pipelines execute successfully in three
contexts:\\
1. The CI system\\
2. On a single user system: laptop, desktop, or notebook running in the
Notebook aspect of the LSP.\\
3. Project workflow system.

            \vspace{\dp0}
            } \end{minipage} \\ \cline{2-3}
            & Test Data &
            \begin{minipage}[t]{13cm}{\footnotesize
                No data.
                \vspace{\dp0}
            } \end{minipage} \\ \cline{2-3}
            & Expected Result &
        \\ \midrule

            \multirow{3}{*}{ 3 } & Description &
            \begin{minipage}[t]{13cm}{\footnotesize
            Using a template testing notebook in the Notebook aspect of the LSP,
verify and log the following:\\
1. Individual pipeline steps (tasks) are importable and executable on
their own. ~this is not comprehensive, but demonstrative.\\
2. Individual pipeline steps may be overridden by configuration.\\
3. Users can implement a custom pipeline step and insert i into the
processing flow via configuration.

            \vspace{\dp0}
            } \end{minipage} \\ \cline{2-3}
            & Test Data &
            \begin{minipage}[t]{13cm}{\footnotesize
                No data.
                \vspace{\dp0}
            } \end{minipage} \\ \cline{2-3}
            & Expected Result &
        \\ \midrule

                \multirow{3}{*}{\parbox{1.3cm}{ 4-1
                {\scriptsize from \hyperref[lvv-t987]
                {LVV-T987} } } }

                & {\small Description} &
                \begin{minipage}[t]{13cm}{\scriptsize
                Identify the path to the data repository, which we will refer to as
`DATA/path', then execute the following:

                \vspace{\dp0}
                } \end{minipage} \\ \cdashline{2-3}
                & {\small Test Data} &
                \begin{minipage}[t]{13cm}{\scriptsize
                } \end{minipage} \\ \cdashline{2-3}
                & {\small Expected Result} &
                    \begin{minipage}[t]{13cm}{\scriptsize
                    Butler repo available for reading.

                    \vspace{\dp0}
                    } \end{minipage}
                \\ \hdashline


        \\ \midrule

            \multirow{3}{*}{ 5 } & Description &
            \begin{minipage}[t]{13cm}{\footnotesize
            Read the resulting dataset using the Bulter, and confirm that it
produced the desired data products.

            \vspace{\dp0}
            } \end{minipage} \\ \cline{2-3}
            & Test Data &
            \begin{minipage}[t]{13cm}{\footnotesize
                No data.
                \vspace{\dp0}
            } \end{minipage} \\ \cline{2-3}
            & Expected Result &
        \\ \midrule

            \multirow{3}{*}{ 6 } & Description &
            \begin{minipage}[t]{13cm}{\footnotesize
            Run subset of full DRP from previous step on an individual node. ~Was
this organizationally easy? ~Did the performance scale appropriately?

            \vspace{\dp0}
            } \end{minipage} \\ \cline{2-3}
            & Test Data &
            \begin{minipage}[t]{13cm}{\footnotesize
                No data.
                \vspace{\dp0}
            } \end{minipage} \\ \cline{2-3}
            & Expected Result &
        \\ \midrule

            \multirow{3}{*}{ 7 } & Description &
            \begin{minipage}[t]{13cm}{\footnotesize
            Re-run aperture correction on subset. ~Verify that same results as DRP
run are achieved.

            \vspace{\dp0}
            } \end{minipage} \\ \cline{2-3}
            & Test Data &
            \begin{minipage}[t]{13cm}{\footnotesize
                No data.
                \vspace{\dp0}
            } \end{minipage} \\ \cline{2-3}
            & Expected Result &
        \\ \midrule

            \multirow{3}{*}{ 8 } & Description &
            \begin{minipage}[t]{13cm}{\footnotesize
            Re-run photometric redshift estimation algorithm on subset coadd
catalogs. ~Verify that same results are achieved as from full DRP.

            \vspace{\dp0}
            } \end{minipage} \\ \cline{2-3}
            & Test Data &
            \begin{minipage}[t]{13cm}{\footnotesize
                No data.
                \vspace{\dp0}
            } \end{minipage} \\ \cline{2-3}
            & Expected Result &
        \\ \midrule
    \end{longtable}

\subsection{LVV-T125 - Verify implementation of Simulated Data}\label{lvv-t125}

\begin{longtable}[]{llllll}
\toprule
Version & Status & Priority & Verification Type & Owner
\\\midrule
1 & Approved & Normal &
Test & Robert Lupton
\\\bottomrule
\multicolumn{6}{c}{ Open \href{https://jira.lsstcorp.org/secure/Tests.jspa\#/testCase/LVV-T125}{LVV-T125} in Jira } \\
\end{longtable}

\subsubsection{Verification Elements}
\begin{itemize}
\item \href{https://jira.lsstcorp.org/browse/LVV-6}{LVV-6} - DMS-REQ-0009-V-01: Simulated Data

\end{itemize}

\subsubsection{Test Items}
Verify that the DMS can inject simulated data into data products for
testing.


\subsubsection{Predecessors}

\subsubsection{Environment Needs}

\paragraph{Software}

\paragraph{Hardware}

\subsubsection{Input Specification}

\subsubsection{Output Specification}

\subsubsection{Test Procedure}
    \begin{longtable}[]{p{1.3cm}p{2cm}p{13cm}}
    %\toprule
    Step & \multicolumn{2}{@{}l}{Description, Input Data and Expected Result} \\ \toprule
    \endhead

            \multirow{3}{*}{ 1 } & Description &
            \begin{minipage}[t]{13cm}{\footnotesize
            Identify a dataset that has been (or can be readily) processed through
single-frame processing and coaddition.

            \vspace{\dp0}
            } \end{minipage} \\ \cline{2-3}
            & Test Data &
            \begin{minipage}[t]{13cm}{\footnotesize
                No data.
                \vspace{\dp0}
            } \end{minipage} \\ \cline{2-3}
            & Expected Result &
                \begin{minipage}[t]{13cm}{\footnotesize
                The `calexp` and `deepCoadd\_calexp` images and their associated source
catalogs are created.

                \vspace{\dp0}
                } \end{minipage}
        \\ \midrule

            \multirow{3}{*}{ 2 } & Description &
            \begin{minipage}[t]{13cm}{\footnotesize
            Roughly determine the coordinates of a bounding box that is contained
within the images that were processed.

            \vspace{\dp0}
            } \end{minipage} \\ \cline{2-3}
            & Test Data &
            \begin{minipage}[t]{13cm}{\footnotesize
                No data.
                \vspace{\dp0}
            } \end{minipage} \\ \cline{2-3}
            & Expected Result &
                \begin{minipage}[t]{13cm}{\footnotesize
                RA, Dec boundaries of a region in which to generate fake sources.

                \vspace{\dp0}
                } \end{minipage}
        \\ \midrule

            \multirow{3}{*}{ 3 } & Description &
            \begin{minipage}[t]{13cm}{\footnotesize
            Generate a catalog in the correct format for `insertFakes` to accept.
The catalog should specify positions and magnitudes of stars (and
optionally, parameters specifying galaxy shape, if galaxies are also
being inserted).

            \vspace{\dp0}
            } \end{minipage} \\ \cline{2-3}
            & Test Data &
            \begin{minipage}[t]{13cm}{\footnotesize
                No data.
                \vspace{\dp0}
            } \end{minipage} \\ \cline{2-3}
            & Expected Result &
                \begin{minipage}[t]{13cm}{\footnotesize
                An input catalog of fake source positions and magnitudes to be inserted
into the images.

                \vspace{\dp0}
                } \end{minipage}
        \\ \midrule

            \multirow{3}{*}{ 4 } & Description &
            \begin{minipage}[t]{13cm}{\footnotesize
            Execute `insertFakes.py` on the repository, specifying the input catalog
from the previous step.

            \vspace{\dp0}
            } \end{minipage} \\ \cline{2-3}
            & Test Data &
            \begin{minipage}[t]{13cm}{\footnotesize
                No data.
                \vspace{\dp0}
            } \end{minipage} \\ \cline{2-3}
            & Expected Result &
                \begin{minipage}[t]{13cm}{\footnotesize
                A repository with images that have fake sources inserted.

                \vspace{\dp0}
                } \end{minipage}
        \\ \midrule

            \multirow{3}{*}{ 5 } & Description &
            \begin{minipage}[t]{13cm}{\footnotesize
            Run `multiBandDriver.py` on the repository, specifying the fake-source
repository as the input.

            \vspace{\dp0}
            } \end{minipage} \\ \cline{2-3}
            & Test Data &
            \begin{minipage}[t]{13cm}{\footnotesize
                No data.
                \vspace{\dp0}
            } \end{minipage} \\ \cline{2-3}
            & Expected Result &
                \begin{minipage}[t]{13cm}{\footnotesize
                `calexp` and coadd images containing the artificial sources and sources
catalogs that contain their measurements along with the sources detected
in the original run.

                \vspace{\dp0}
                } \end{minipage}
        \\ \midrule

            \multirow{3}{*}{ 6 } & Description &
            \begin{minipage}[t]{13cm}{\footnotesize
            Confirm that the injected sources appear in the images and the catalogs.

            \vspace{\dp0}
            } \end{minipage} \\ \cline{2-3}
            & Test Data &
            \begin{minipage}[t]{13cm}{\footnotesize
                No data.
                \vspace{\dp0}
            } \end{minipage} \\ \cline{2-3}
            & Expected Result &
                \begin{minipage}[t]{13cm}{\footnotesize
                Fake sources and their measured properties are recoverable.

                \vspace{\dp0}
                } \end{minipage}
        \\ \midrule
    \end{longtable}

\subsection{LVV-T126 - Verify implementation of Image Differencing}\label{lvv-t126}

\begin{longtable}[]{llllll}
\toprule
Version & Status & Priority & Verification Type & Owner
\\\midrule
1 & Defined & Normal &
Test & Eric Bellm
\\\bottomrule
\multicolumn{6}{c}{ Open \href{https://jira.lsstcorp.org/secure/Tests.jspa\#/testCase/LVV-T126}{LVV-T126} in Jira } \\
\end{longtable}

\subsubsection{Verification Elements}
\begin{itemize}
\item \href{https://jira.lsstcorp.org/browse/LVV-14}{LVV-14} - DMS-REQ-0032-V-01: Image Differencing

\end{itemize}

\subsubsection{Test Items}
Verify that the DMS can perform image differencing from single exposures
and coadds.


\subsubsection{Predecessors}

\subsubsection{Environment Needs}

\paragraph{Software}

\paragraph{Hardware}

\subsubsection{Input Specification}

\subsubsection{Output Specification}

\subsubsection{Test Procedure}
    \begin{longtable}[]{p{1.3cm}p{2cm}p{13cm}}
    %\toprule
    Step & \multicolumn{2}{@{}l}{Description, Input Data and Expected Result} \\ \toprule
    \endhead

            \multirow{3}{*}{ 1 } & Description &
            \begin{minipage}[t]{13cm}{\footnotesize
            Identify a repository containing data that have been processed through
the difference imaging pipeline. (e.g., the HiTS 2015 data that are
processed monthly for testing)

            \vspace{\dp0}
            } \end{minipage} \\ \cline{2-3}
            & Test Data &
            \begin{minipage}[t]{13cm}{\footnotesize
                No data.
                \vspace{\dp0}
            } \end{minipage} \\ \cline{2-3}
            & Expected Result &
                \begin{minipage}[t]{13cm}{\footnotesize
                A dataset containing calexps, difference images, and source catalogs (of
diaSrcs).

                \vspace{\dp0}
                } \end{minipage}
        \\ \midrule

                \multirow{3}{*}{\parbox{1.3cm}{ 2-1
                {\scriptsize from \hyperref[lvv-t987]
                {LVV-T987} } } }

                & {\small Description} &
                \begin{minipage}[t]{13cm}{\scriptsize
                Identify the path to the data repository, which we will refer to as
`DATA/path', then execute the following:

                \vspace{\dp0}
                } \end{minipage} \\ \cdashline{2-3}
                & {\small Test Data} &
                \begin{minipage}[t]{13cm}{\scriptsize
                } \end{minipage} \\ \cdashline{2-3}
                & {\small Expected Result} &
                    \begin{minipage}[t]{13cm}{\scriptsize
                    Butler repo available for reading.

                    \vspace{\dp0}
                    } \end{minipage}
                \\ \hdashline


        \\ \midrule

            \multirow{3}{*}{ 3 } & Description &
            \begin{minipage}[t]{13cm}{\footnotesize
            Extract a `calexp`, a `deepDiff\_differenceExp`, and the
`deepDiff\_diaSrc` catalog of measurements.

            \vspace{\dp0}
            } \end{minipage} \\ \cline{2-3}
            & Test Data &
            \begin{minipage}[t]{13cm}{\footnotesize
                No data.
                \vspace{\dp0}
            } \end{minipage} \\ \cline{2-3}
            & Expected Result &
                \begin{minipage}[t]{13cm}{\footnotesize
                Well-formed images and catalogs containing the calexp from the visit
image and the difference image, and measurements of sources from the
difference image.

                \vspace{\dp0}
                } \end{minipage}
        \\ \midrule

            \multirow{3}{*}{ 4 } & Description &
            \begin{minipage}[t]{13cm}{\footnotesize
            Confirm (by visual inspection) that the difference image is mostly blank
sky (i.e., has had a template of the same field subtracted), and that
the source catalog contains sources with photometric and astrometric
measurements.

            \vspace{\dp0}
            } \end{minipage} \\ \cline{2-3}
            & Test Data &
            \begin{minipage}[t]{13cm}{\footnotesize
                No data.
                \vspace{\dp0}
            } \end{minipage} \\ \cline{2-3}
            & Expected Result &
                \begin{minipage}[t]{13cm}{\footnotesize
                A mostly blank image (with perhaps some artifacts due to imperfect
subtraction) and a catalog of sources detected/measured from that image.

                \vspace{\dp0}
                } \end{minipage}
        \\ \midrule
    \end{longtable}

\subsection{LVV-T127 - Verify implementation of Provide Source Detection Software}\label{lvv-t127}

\begin{longtable}[]{llllll}
\toprule
Version & Status & Priority & Verification Type & Owner
\\\midrule
1 & Defined & Normal &
Test & Robert Lupton
\\\bottomrule
\multicolumn{6}{c}{ Open \href{https://jira.lsstcorp.org/secure/Tests.jspa\#/testCase/LVV-T127}{LVV-T127} in Jira } \\
\end{longtable}

\subsubsection{Verification Elements}
\begin{itemize}
\item \href{https://jira.lsstcorp.org/browse/LVV-15}{LVV-15} - DMS-REQ-0033-V-01: Provide Source Detection Software

\end{itemize}

\subsubsection{Test Items}
Verify that the DMS provides source detection software that can be
applied to calibrated images, including both difference images and
coadds. This will be verified using simulated data, but could also be
done by inserting artificial sources into existing datasets.


\subsubsection{Predecessors}

\subsubsection{Environment Needs}

\paragraph{Software}

\paragraph{Hardware}

\subsubsection{Input Specification}

\subsubsection{Output Specification}

\subsubsection{Test Procedure}
    \begin{longtable}[]{p{1.3cm}p{2cm}p{13cm}}
    %\toprule
    Step & \multicolumn{2}{@{}l}{Description, Input Data and Expected Result} \\ \toprule
    \endhead

            \multirow{3}{*}{ 1 } & Description &
            \begin{minipage}[t]{13cm}{\footnotesize
            Run DRP and AP processing, including source detection and measurement
algorithms, on a small portion of the data from a simulated dataset.

            \vspace{\dp0}
            } \end{minipage} \\ \cline{2-3}
            & Test Data &
            \begin{minipage}[t]{13cm}{\footnotesize
                No data.
                \vspace{\dp0}
            } \end{minipage} \\ \cline{2-3}
            & Expected Result &
                \begin{minipage}[t]{13cm}{\footnotesize
                Source catalogs containing measurements of all sources detected in the
input images.

                \vspace{\dp0}
                } \end{minipage}
        \\ \midrule

            \multirow{3}{*}{ 2 } & Description &
            \begin{minipage}[t]{13cm}{\footnotesize
            Confirm that the output repos contain catalogs of source detections.
Compare these output catalogs to the original simulated source catalogs,
and confirm that a large fraction of the sources within a reasonable
signal-to-noise range were recovered.

            \vspace{\dp0}
            } \end{minipage} \\ \cline{2-3}
            & Test Data &
            \begin{minipage}[t]{13cm}{\footnotesize
                No data.
                \vspace{\dp0}
            } \end{minipage} \\ \cline{2-3}
            & Expected Result &
                \begin{minipage}[t]{13cm}{\footnotesize
                Most sources above a reasonable S/N threshold were detected, and their
measured fluxes are reasonably close to the simulated inputs.

                \vspace{\dp0}
                } \end{minipage}
        \\ \midrule
    \end{longtable}

\subsection{LVV-T128 - Verify implementation Provide Astrometric Model}\label{lvv-t128}

\begin{longtable}[]{llllll}
\toprule
Version & Status & Priority & Verification Type & Owner
\\\midrule
1 & Draft & Normal &
Test & Colin Slater
\\\bottomrule
\multicolumn{6}{c}{ Open \href{https://jira.lsstcorp.org/secure/Tests.jspa\#/testCase/LVV-T128}{LVV-T128} in Jira } \\
\end{longtable}

\subsubsection{Verification Elements}
\begin{itemize}
\item \href{https://jira.lsstcorp.org/browse/LVV-17}{LVV-17} - DMS-REQ-0042-V-01: Provide Astrometric Model

\end{itemize}

\subsubsection{Test Items}
Verify that an astrometric model is available for Objects and
DIAObjects.


\subsubsection{Predecessors}

\subsubsection{Environment Needs}

\paragraph{Software}

\paragraph{Hardware}

\subsubsection{Input Specification}

\subsubsection{Output Specification}

\subsubsection{Test Procedure}
    \begin{longtable}[]{p{1.3cm}p{2cm}p{13cm}}
    %\toprule
    Step & \multicolumn{2}{@{}l}{Description, Input Data and Expected Result} \\ \toprule
    \endhead

            \multirow{3}{*}{ 1 } & Description &
            \begin{minipage}[t]{13cm}{\footnotesize
            Delegate to AP and DRP

            \vspace{\dp0}
            } \end{minipage} \\ \cline{2-3}
            & Test Data &
            \begin{minipage}[t]{13cm}{\footnotesize
                No data.
                \vspace{\dp0}
            } \end{minipage} \\ \cline{2-3}
            & Expected Result &
        \\ \midrule
    \end{longtable}

\subsection{LVV-T129 - Verify implementation of Provide Calibrated Photometry}\label{lvv-t129}

\begin{longtable}[]{llllll}
\toprule
Version & Status & Priority & Verification Type & Owner
\\\midrule
1 & Defined & Normal &
Test & Robert Lupton
\\\bottomrule
\multicolumn{6}{c}{ Open \href{https://jira.lsstcorp.org/secure/Tests.jspa\#/testCase/LVV-T129}{LVV-T129} in Jira } \\
\end{longtable}

\subsubsection{Verification Elements}
\begin{itemize}
\item \href{https://jira.lsstcorp.org/browse/LVV-18}{LVV-18} - DMS-REQ-0043-V-01: Provide Calibrated Photometry

\end{itemize}

\subsubsection{Test Items}
Verify that the DMS provides photometry calibrated in AB mags and fluxes
(in nJy) for all measured objects and sources. Must be tested for both
DRP and AP products.


\subsubsection{Predecessors}

\subsubsection{Environment Needs}

\paragraph{Software}

\paragraph{Hardware}

\subsubsection{Input Specification}

\subsubsection{Output Specification}

\subsubsection{Test Procedure}
    \begin{longtable}[]{p{1.3cm}p{2cm}p{13cm}}
    %\toprule
    Step & \multicolumn{2}{@{}l}{Description, Input Data and Expected Result} \\ \toprule
    \endhead

                \multirow{3}{*}{\parbox{1.3cm}{ 1-1
                {\scriptsize from \hyperref[lvv-t987]
                {LVV-T987} } } }

                & {\small Description} &
                \begin{minipage}[t]{13cm}{\scriptsize
                Identify the path to the data repository, which we will refer to as
`DATA/path', then execute the following:

                \vspace{\dp0}
                } \end{minipage} \\ \cdashline{2-3}
                & {\small Test Data} &
                \begin{minipage}[t]{13cm}{\scriptsize
                } \end{minipage} \\ \cdashline{2-3}
                & {\small Expected Result} &
                    \begin{minipage}[t]{13cm}{\scriptsize
                    Butler repo available for reading.

                    \vspace{\dp0}
                    } \end{minipage}
                \\ \hdashline


        \\ \midrule

            \multirow{3}{*}{ 2 } & Description &
            \begin{minipage}[t]{13cm}{\footnotesize
            Ingest the data products from an appropriate DRP-processed dataset.

            \vspace{\dp0}
            } \end{minipage} \\ \cline{2-3}
            & Test Data &
            \begin{minipage}[t]{13cm}{\footnotesize
                No data.
                \vspace{\dp0}
            } \end{minipage} \\ \cline{2-3}
            & Expected Result &
        \\ \midrule

            \multirow{3}{*}{ 3 } & Description &
            \begin{minipage}[t]{13cm}{\footnotesize
            Confirm that AB-calibrated magnitudes and fluxes are available for all
measured Sources and Objects. {[}An enhanced verification could include
matching the sources to an external source catalog and comparing the
magnitudes to show that they are well-calibrated.{]}

            \vspace{\dp0}
            } \end{minipage} \\ \cline{2-3}
            & Test Data &
            \begin{minipage}[t]{13cm}{\footnotesize
                No data.
                \vspace{\dp0}
            } \end{minipage} \\ \cline{2-3}
            & Expected Result &
                \begin{minipage}[t]{13cm}{\footnotesize
                Calibrated fluxes and magnitudes are available for all sources, as well
as tools to convert measured fluxes to magnitudes (and vice-versa).

                \vspace{\dp0}
                } \end{minipage}
        \\ \midrule

            \multirow{3}{*}{ 4 } & Description &
            \begin{minipage}[t]{13cm}{\footnotesize
            Ingest the data products from an appropriate AP processing dataset.

            \vspace{\dp0}
            } \end{minipage} \\ \cline{2-3}
            & Test Data &
            \begin{minipage}[t]{13cm}{\footnotesize
                No data.
                \vspace{\dp0}
            } \end{minipage} \\ \cline{2-3}
            & Expected Result &
        \\ \midrule

            \multirow{3}{*}{ 5 } & Description &
            \begin{minipage}[t]{13cm}{\footnotesize
            Confirm that AB-calibrated magnitudes and fluxes are available for all
measured Sources, DIASources, and Objects. {[}An enhanced verification
could include matching the sources to an external source catalog and
comparing the magnitudes to show that they are well-calibrated.{]}

            \vspace{\dp0}
            } \end{minipage} \\ \cline{2-3}
            & Test Data &
            \begin{minipage}[t]{13cm}{\footnotesize
                No data.
                \vspace{\dp0}
            } \end{minipage} \\ \cline{2-3}
            & Expected Result &
                \begin{minipage}[t]{13cm}{\footnotesize
                Calibrated fluxes and magnitudes are available for all Sources,
DIASources, and Objects, as well as tools to convert measured fluxes to
magnitudes (and vice-versa).

                \vspace{\dp0}
                } \end{minipage}
        \\ \midrule
    \end{longtable}

\subsection{LVV-T130 - Verify implementation of Enable a Range of Shape Measurement Approaches}\label{lvv-t130}

\begin{longtable}[]{llllll}
\toprule
Version & Status & Priority & Verification Type & Owner
\\\midrule
1 & Draft & Normal &
Test & Colin Slater
\\\bottomrule
\multicolumn{6}{c}{ Open \href{https://jira.lsstcorp.org/secure/Tests.jspa\#/testCase/LVV-T130}{LVV-T130} in Jira } \\
\end{longtable}

\subsubsection{Verification Elements}
\begin{itemize}
\item \href{https://jira.lsstcorp.org/browse/LVV-21}{LVV-21} - DMS-REQ-0052-V-01: Enable a Range of Shape Measurement Approaches

\end{itemize}

\subsubsection{Test Items}
Verify that multiple shape measurement algorithms can be used.


\subsubsection{Predecessors}

\subsubsection{Environment Needs}

\paragraph{Software}

\paragraph{Hardware}

\subsubsection{Input Specification}

\subsubsection{Output Specification}

\subsubsection{Test Procedure}
    \begin{longtable}[]{p{1.3cm}p{2cm}p{13cm}}
    %\toprule
    Step & \multicolumn{2}{@{}l}{Description, Input Data and Expected Result} \\ \toprule
    \endhead

            \multirow{3}{*}{ 1 } & Description &
            \begin{minipage}[t]{13cm}{\footnotesize
            Delegate to AP and DRP

            \vspace{\dp0}
            } \end{minipage} \\ \cline{2-3}
            & Test Data &
            \begin{minipage}[t]{13cm}{\footnotesize
                No data.
                \vspace{\dp0}
            } \end{minipage} \\ \cline{2-3}
            & Expected Result &
        \\ \midrule
    \end{longtable}

\subsection{LVV-T131 - Verify implementation of Provide User Interface Services}\label{lvv-t131}

\begin{longtable}[]{llllll}
\toprule
Version & Status & Priority & Verification Type & Owner
\\\midrule
1 & Defined & Normal &
Test & Gregory Dubois-Felsmann
\\\bottomrule
\multicolumn{6}{c}{ Open \href{https://jira.lsstcorp.org/secure/Tests.jspa\#/testCase/LVV-T131}{LVV-T131} in Jira } \\
\end{longtable}

\subsubsection{Verification Elements}
\begin{itemize}
\item \href{https://jira.lsstcorp.org/browse/LVV-63}{LVV-63} - DMS-REQ-0160-V-01: Provide User Interface Services

\end{itemize}

\subsubsection{Test Items}
Verify the availability and functionality of the broad range of user
interface services called for in the requirement, as applied to both
Nightly and DRP data. ~This will primarily be done by verifications
performed at the LSST Science Platform level, based on the requirements
in \citeds{LDM-554}; however, a high-level set of tests corresponding to the
DMS-REQ-0160 requirement are defined below.


\subsubsection{Predecessors}

\subsubsection{Environment Needs}

\paragraph{Software}

\paragraph{Hardware}
As noted in Verification Configuration, the systems required to carry
out the tests include both an ``inside'' test execution platform - the
ability to execute test notebooks within the Science Platform Notebook
Aspect - and an ``outside'' test execution platform with connectivity to
the Science Platform instance under test that is comparable to that
available to offsite science users.

\subsubsection{Input Specification}
\begin{enumerate}
\tightlist
\item
  Testing this requirement relies on a set of data products meeting the
  data model implied by the DPDD existing in a deployment of the Science
  Platform and its underlying database and file services.

  \begin{enumerate}
  \tightlist
  \item
    In particular, both image and catalog data products are required.
  \item
    From the specific language of the underlying requirement, it appears
    clear that coadded data products are required, but in practice
    single-epoch data products should be included in the test as well.
  \end{enumerate}
\item
  Depending on when this requirement is tested, the tests may involve
  either or both of precursor data and LSST commissioning data. ~The use
  of the latter is ultimately essential to ensure that the tests are
  performed with as LSST-like a dataset as possible.
\end{enumerate}

\subsubsection{Output Specification}

\subsubsection{Test Procedure}
    \begin{longtable}[]{p{1.3cm}p{2cm}p{13cm}}
    %\toprule
    Step & \multicolumn{2}{@{}l}{Description, Input Data and Expected Result} \\ \toprule
    \endhead

            \multirow{3}{*}{ 1 } & Description &
            \begin{minipage}[t]{13cm}{\footnotesize
            \textbf{Establishment of test coordinates:}\\
Establish sky positions and surrounding regions (e.g., cones or
polygons), field sizes, filter bands, and temporal epochs for the tests
that are consistent with the known content of the test dataset, whether
precursor or LSST commissioning data.\\
Establishing sky positions should include pre-determining the
corresponding LSST ``tract and patch'' identifiers.\\
If the plan to not keep all calibrated single-epoch images on disk is
still in place at the time of the test, identify for use in the test
both images that are, and are not, on disk.\\
Establish target image boundaries, projections, and pixel scales to be
used for resampling tests. ~Ensure that at least some of these test
conditions include coadded image boundaries that cross tract and patch
boundaries, and single-epoch image boundaries that cross focal plane
raft boundaries.

            \vspace{\dp0}
            } \end{minipage} \\ \cline{2-3}
            & Test Data &
            \begin{minipage}[t]{13cm}{\footnotesize
                No data.
                \vspace{\dp0}
            } \end{minipage} \\ \cline{2-3}
            & Expected Result &
        \\ \midrule

            \multirow{3}{*}{ 2 } & Description &
            \begin{minipage}[t]{13cm}{\footnotesize
            \textbf{Butler image access:}\\
From within the Notebook Aspect, verify that coadded images for the
identified regions of sky and filter bands are accessible via the
Butler. ~Verify that the same images are available whether obtained by
direct reference to the previous established tract/patch identifiers or
by the use of LSST stack code for retrieving images based on sky
coordinates.\\
From within the Notebook Aspect, verify that single-epoch raw images for
the selected locations and times are available. ~Verify that calibrated
images (PVIs) for the selected locations and times are available;
depending on the details of the test dataset, verify that PVIs still on
disk can be retrieved immediately.\\
Verify that lists or tables of image metadata, not just individual
images, can be retrieved. ~E.g., a list of all the single-epoch images
covering a selected sky location.

            \vspace{\dp0}
            } \end{minipage} \\ \cline{2-3}
            & Test Data &
            \begin{minipage}[t]{13cm}{\footnotesize
                No data.
                \vspace{\dp0}
            } \end{minipage} \\ \cline{2-3}
            & Expected Result &
        \\ \midrule

            \multirow{3}{*}{ 3 } & Description &
            \begin{minipage}[t]{13cm}{\footnotesize
            \textbf{Programmatic PVI re-creation:}\\
From within the Notebook Aspect, verify that the recreation on demand of
a PVI can be performed. ~Ideally, this should be done as follows:

\begin{itemize}
\tightlist
\item
  Verify that recreation of a PVI that \emph{is} still available works
  and that it reproduces the original PVI exactly (except for provenance
  metadata that must be different) or within the reasonable ability of
  processing systems to do so (e.g., taking into account that the
  original calibration and the recreation may have run on different CPU
  architectures).
\item
  The test conditions should ensure the verification that a recreation
  was actually performed, i.e., that the still-available PVI was not
  returned instead.
\item
  Note that it does not appear to be a requirement that \emph{at Butler
  level} recreation on demand of PVIs is a completely transparent
  process. ~If this \emph{is} decided to be a requirement, the test must
  also verify that it has been satisfied. ~If it is \emph{not} a
  requirement, verify that adequate documentation on the PVI-recreation
  process (e.g., the SuperTasks and configuration to be used) is
  available.
\end{itemize}

            \vspace{\dp0}
            } \end{minipage} \\ \cline{2-3}
            & Test Data &
            \begin{minipage}[t]{13cm}{\footnotesize
                No data.
                \vspace{\dp0}
            } \end{minipage} \\ \cline{2-3}
            & Expected Result &
        \\ \midrule

            \multirow{3}{*}{ 4 } & Description &
            \begin{minipage}[t]{13cm}{\footnotesize
            \textbf{Butler catalog access:}\\
From within the Notebook Aspect, verify that all the catalog data
products described in the DPDD can be retrieved for the coordinates
selected above via the Butler. (This test should include access to
SSObject data, but the details of how such a test would depend on the
coordinate selections require additional thought.)

            \vspace{\dp0}
            } \end{minipage} \\ \cline{2-3}
            & Test Data &
            \begin{minipage}[t]{13cm}{\footnotesize
                No data.
                \vspace{\dp0}
            } \end{minipage} \\ \cline{2-3}
            & Expected Result &
        \\ \midrule

            \multirow{3}{*}{ 5 } & Description &
            \begin{minipage}[t]{13cm}{\footnotesize
            \textbf{LSST-stack-based resampling/reprojection:}\\
Verify the availability of software in the LSST stack, and associated
documentation, that permits the resampling of LSST images to different
pixel grids and projections.\\
Exercise this capability for the test conditions selected in Step 1
above.\\
Perform photometric and astrometric tests on the resulting resampled
images to provide evidence that the transformations performed were
correct to the accuracy supported by the data.

            \vspace{\dp0}
            } \end{minipage} \\ \cline{2-3}
            & Test Data &
            \begin{minipage}[t]{13cm}{\footnotesize
                No data.
                \vspace{\dp0}
            } \end{minipage} \\ \cline{2-3}
            & Expected Result &
        \\ \midrule

            \multirow{3}{*}{ 6 } & Description &
            \begin{minipage}[t]{13cm}{\footnotesize
            \textbf{Comment:}\\
The following API Aspect test steps should be carried out on the
required ``offsite-like'' test platform, to ensure that their success
does not reflect any privileged access given to processes inside the
Data Access Center or other Science Platform instance. ~However, at
least a small sampling of them should \emph{also} be carried out
\emph{within} the Science Platform environment, i.e., in the Notebook
Aspect, and the results compared.

            \vspace{\dp0}
            } \end{minipage} \\ \cline{2-3}
            & Test Data &
            \begin{minipage}[t]{13cm}{\footnotesize
                No data.
                \vspace{\dp0}
            } \end{minipage} \\ \cline{2-3}
            & Expected Result &
        \\ \midrule

            \multirow{3}{*}{ 7 } & Description &
            \begin{minipage}[t]{13cm}{\footnotesize
            \textbf{API Aspect image access:}\\
Using IVOA services such as the Registry and ObsTAP, from the
``offsite-like'' test platform, verify that the existence of the classes
of image data products foreseen in the DPDD can be determined.\\
Verify that ObsTAP and/or SIAv2 can be used to find the same images and
lists of images for the established test coordinates that were retrieved
via the Butler in Step 2 above.\\
Verify that the selected images are retrievable from the Web services.\\
Verify that the retrieved images are identical in their pixel content
and metadata.\\
The tests must include both coadded and single-epoch images.

            \vspace{\dp0}
            } \end{minipage} \\ \cline{2-3}
            & Test Data &
            \begin{minipage}[t]{13cm}{\footnotesize
                No data.
                \vspace{\dp0}
            } \end{minipage} \\ \cline{2-3}
            & Expected Result &
        \\ \midrule

            \multirow{3}{*}{ 8 } & Description &
            \begin{minipage}[t]{13cm}{\footnotesize
            \textbf{API Aspect image transformations:}\\
Verify that image cutouts and resamplings can be performed via the IVOA
SODA service, and that the results are identical to those obtained for
the same parameters from the LSST-stack-based tests in Step 5.\\
(The requirements for supported reprojections, if any, in the SODA
service have not been established at the time of writing.)

            \vspace{\dp0}
            } \end{minipage} \\ \cline{2-3}
            & Test Data &
            \begin{minipage}[t]{13cm}{\footnotesize
                No data.
                \vspace{\dp0}
            } \end{minipage} \\ \cline{2-3}
            & Expected Result &
        \\ \midrule

            \multirow{3}{*}{ 9 } & Description &
            \begin{minipage}[t]{13cm}{\footnotesize
            \textbf{API Aspect catalog data access:}\\
Verify that the IVOA Registry, RegTAP, TAP\_SCHEMA, and other relevant
mechanisms can be used to discover the existence of all the catalog data
products foreseen in the DPDD.\\
Using the IVOA TAP service, verify that all the catalog data products
foreseen in the DPDD can be retrieved for the coordinates determined in
Step 1. ~Verify that their scientific content is the same as when they
are retrieved via the Butler.

            \vspace{\dp0}
            } \end{minipage} \\ \cline{2-3}
            & Test Data &
            \begin{minipage}[t]{13cm}{\footnotesize
                No data.
                \vspace{\dp0}
            } \end{minipage} \\ \cline{2-3}
            & Expected Result &
        \\ \midrule

            \multirow{3}{*}{ 10 } & Description &
            \begin{minipage}[t]{13cm}{\footnotesize
            \textbf{Comment:}\\
The Portal Aspect tests below should be carried out from a web browser
on an ``offsite-like'' test platform, to ensure that no privileged
access provided to intra-data-center clients is relied upon.

            \vspace{\dp0}
            } \end{minipage} \\ \cline{2-3}
            & Test Data &
            \begin{minipage}[t]{13cm}{\footnotesize
                No data.
                \vspace{\dp0}
            } \end{minipage} \\ \cline{2-3}
            & Expected Result &
        \\ \midrule

            \multirow{3}{*}{ 11 } & Description &
            \begin{minipage}[t]{13cm}{\footnotesize
            \textbf{Portal Aspect data browsing:}\\
Verify that the Portal Aspect can be used to discover the existence of
all the data products foreseen in the DPDD. ~Verify that the UI permits
locating the data for the coordinates selected in Step 1 by visual
means, e.g., by zooming and panning in from an all-sky view.\\
Verify that the UI permits locating the data by typing in coordinates as
well.

            \vspace{\dp0}
            } \end{minipage} \\ \cline{2-3}
            & Test Data &
            \begin{minipage}[t]{13cm}{\footnotesize
                No data.
                \vspace{\dp0}
            } \end{minipage} \\ \cline{2-3}
            & Expected Result &
        \\ \midrule

            \multirow{3}{*}{ 12 } & Description &
            \begin{minipage}[t]{13cm}{\footnotesize
            \textbf{Portal Aspect image access:}\\
Verify that the Portal Aspect allows both the retrieval of ``original''
image data, i.e., in its native LSST pixel projection and with full
metadata, as well as retrieval of on-demand UI cutouts of coadded image
data for selected locations.

            \vspace{\dp0}
            } \end{minipage} \\ \cline{2-3}
            & Test Data &
            \begin{minipage}[t]{13cm}{\footnotesize
                No data.
                \vspace{\dp0}
            } \end{minipage} \\ \cline{2-3}
            & Expected Result &
        \\ \midrule

            \multirow{3}{*}{ 13 } & Description &
            \begin{minipage}[t]{13cm}{\footnotesize
            \textbf{Portal Aspect catalog query and visualization:}\\
Verify that the Portal Aspect allows graphical querying of DPDD catalog
data, both coadded and single-epoch, for selected regions of sky and/or
with selected properties, and supports the visualization of the results
(including histogramming, scatterplots, time series, table
manipulations, and overplotting on image data).\\
(Note that the Science Platform requirements, LDM-554, lay out a
detailed set of requirements on the selection and visualization of
catalog data.)

            \vspace{\dp0}
            } \end{minipage} \\ \cline{2-3}
            & Test Data &
            \begin{minipage}[t]{13cm}{\footnotesize
                No data.
                \vspace{\dp0}
            } \end{minipage} \\ \cline{2-3}
            & Expected Result &
        \\ \midrule

            \multirow{3}{*}{ 14 } & Description &
            \begin{minipage}[t]{13cm}{\footnotesize
            \textbf{Portal Aspect data download:}\\
Verify that data identified and/or visualized in the Portal Aspect can
be downloaded to the remote system running the web browser in which the
Portal is displayed, as well as to the User Workspace.

            \vspace{\dp0}
            } \end{minipage} \\ \cline{2-3}
            & Test Data &
            \begin{minipage}[t]{13cm}{\footnotesize
                No data.
                \vspace{\dp0}
            } \end{minipage} \\ \cline{2-3}
            & Expected Result &
        \\ \midrule
    \end{longtable}

\subsection{LVV-T132 - Verify implementation of Pre-cursor and Real Data}\label{lvv-t132}

\begin{longtable}[]{llllll}
\toprule
Version & Status & Priority & Verification Type & Owner
\\\midrule
1 & Approved & Normal &
Test & Robert Gruendl
\\\bottomrule
\multicolumn{6}{c}{ Open \href{https://jira.lsstcorp.org/secure/Tests.jspa\#/testCase/LVV-T132}{LVV-T132} in Jira } \\
\end{longtable}

\subsubsection{Verification Elements}
\begin{itemize}
\item \href{https://jira.lsstcorp.org/browse/LVV-127}{LVV-127} - DMS-REQ-0296-V-01: Pre-cursor, and Real Data

\end{itemize}

\subsubsection{Test Items}
Demonstrate that pixel-oriented data from astronomical imaging cameras
(precursor or otherwise) can be processed using LSST Science Algorithms
and organized for access through the Data Butler Access Client. ~


\subsubsection{Predecessors}

\subsubsection{Environment Needs}

\paragraph{Software}

\paragraph{Hardware}

\subsubsection{Input Specification}

\subsubsection{Output Specification}

\subsubsection{Test Procedure}
    \begin{longtable}[]{p{1.3cm}p{2cm}p{13cm}}
    %\toprule
    Step & \multicolumn{2}{@{}l}{Description, Input Data and Expected Result} \\ \toprule
    \endhead

            \multirow{3}{*}{ 1 } & Description &
            \begin{minipage}[t]{13cm}{\footnotesize
            Confirm that the CI jobs used to test DRP processing successfully run.
These jobs use precursor datasets from cameras other than LSST.

            \vspace{\dp0}
            } \end{minipage} \\ \cline{2-3}
            & Test Data &
            \begin{minipage}[t]{13cm}{\footnotesize
                No data.
                \vspace{\dp0}
            } \end{minipage} \\ \cline{2-3}
            & Expected Result &
        \\ \midrule

            \multirow{3}{*}{ 2 } & Description &
            \begin{minipage}[t]{13cm}{\footnotesize
            For the precursor dataset, instantiate the Butler, load the data
products, and confirm that they exist as expected.

            \vspace{\dp0}
            } \end{minipage} \\ \cline{2-3}
            & Test Data &
            \begin{minipage}[t]{13cm}{\footnotesize
                No data.
                \vspace{\dp0}
            } \end{minipage} \\ \cline{2-3}
            & Expected Result &
                \begin{minipage}[t]{13cm}{\footnotesize
                Processed images, catalogs, calibration information, and other related
data products are present and accessible via the Butler.

                \vspace{\dp0}
                } \end{minipage}
        \\ \midrule
    \end{longtable}

\subsection{LVV-T133 - Verify implementation of Provide Beam Projector Coordinate Calculation
Software}\label{lvv-t133}

\begin{longtable}[]{llllll}
\toprule
Version & Status & Priority & Verification Type & Owner
\\\midrule
1 & Defined & Normal &
Test & Robert Lupton
\\\bottomrule
\multicolumn{6}{c}{ Open \href{https://jira.lsstcorp.org/secure/Tests.jspa\#/testCase/LVV-T133}{LVV-T133} in Jira } \\
\end{longtable}

\subsubsection{Verification Elements}
\begin{itemize}
\item \href{https://jira.lsstcorp.org/browse/LVV-182}{LVV-182} - DMS-REQ-0351-V-01: Provide Beam Projector Coordinate Calculation
Software

\end{itemize}

\subsubsection{Test Items}
Verify that the DMS provides software to calculate coordinates relating
the collimated beam projector position and telescope pupil position to
the illumination position on the telescope optical elements and focal
plane.


\subsubsection{Predecessors}

\subsubsection{Environment Needs}

\paragraph{Software}

\paragraph{Hardware}

\subsubsection{Input Specification}

\subsubsection{Output Specification}

\subsubsection{Test Procedure}
    \begin{longtable}[]{p{1.3cm}p{2cm}p{13cm}}
    %\toprule
    Step & \multicolumn{2}{@{}l}{Description, Input Data and Expected Result} \\ \toprule
    \endhead

            \multirow{3}{*}{ 1 } & Description &
            \begin{minipage}[t]{13cm}{\footnotesize
            On the LSST development cluster or notebook aspect, git clone the repo
containing the CBP package: \url{https://github.com/lsst/cbp}

            \vspace{\dp0}
            } \end{minipage} \\ \cline{2-3}
            & Test Data &
            \begin{minipage}[t]{13cm}{\footnotesize
                No data.
                \vspace{\dp0}
            } \end{minipage} \\ \cline{2-3}
            & Expected Result &
        \\ \midrule

            \multirow{3}{*}{ 2 } & Description &
            \begin{minipage}[t]{13cm}{\footnotesize
            Follow the steps in the package README to install the package.

            \vspace{\dp0}
            } \end{minipage} \\ \cline{2-3}
            & Test Data &
            \begin{minipage}[t]{13cm}{\footnotesize
                No data.
                \vspace{\dp0}
            } \end{minipage} \\ \cline{2-3}
            & Expected Result &
        \\ \midrule

            \multirow{3}{*}{ 3 } & Description &
            \begin{minipage}[t]{13cm}{\footnotesize
            Confirm that the package can be loaded in python, and that some of the
tests in the `tests/` folder will execute.

            \vspace{\dp0}
            } \end{minipage} \\ \cline{2-3}
            & Test Data &
            \begin{minipage}[t]{13cm}{\footnotesize
                No data.
                \vspace{\dp0}
            } \end{minipage} \\ \cline{2-3}
            & Expected Result &
                \begin{minipage}[t]{13cm}{\footnotesize
                Successful execution of test scripts, which demonstrate the calculation
of beam projector coordinates.

                \vspace{\dp0}
                } \end{minipage}
        \\ \midrule
    \end{longtable}

\subsection{LVV-T134 - Verify implementation of Provide Image Access Services}\label{lvv-t134}

\begin{longtable}[]{llllll}
\toprule
Version & Status & Priority & Verification Type & Owner
\\\midrule
1 & Draft & Normal &
Inspection & Gregory Dubois-Felsmann
\\\bottomrule
\multicolumn{6}{c}{ Open \href{https://jira.lsstcorp.org/secure/Tests.jspa\#/testCase/LVV-T134}{LVV-T134} in Jira } \\
\end{longtable}

\subsubsection{Verification Elements}
\begin{itemize}
\item \href{https://jira.lsstcorp.org/browse/LVV-27}{LVV-27} - DMS-REQ-0065-V-01: Provide Image Access Services

\end{itemize}

\subsubsection{Test Items}
Verify that images can be identified and that images and image cut-outs
can be retrieved using the network interfaces - primarily IVOA
standards-based - and Python APIs provided for image access by science
users.


\subsubsection{Predecessors}

\subsubsection{Environment Needs}

\paragraph{Software}

\paragraph{Hardware}

\subsubsection{Input Specification}
Testing requires the establishment of running services such as SIAv2 and
SODA to which the tests can be applied.

\subsubsection{Output Specification}

\subsubsection{Test Procedure}
    \begin{longtable}[]{p{1.3cm}p{2cm}p{13cm}}
    %\toprule
    Step & \multicolumn{2}{@{}l}{Description, Input Data and Expected Result} \\ \toprule
    \endhead

            \multirow{3}{*}{ 1 } & Description &
            \begin{minipage}[t]{13cm}{\footnotesize
            Inspect that the following test cases have been executed and passed:
\href{https://jira.lsstcorp.org/secure/Tests.jspa\#/testCase/LVV-T803}{LVV-T803},
\href{https://jira.lsstcorp.org/secure/Tests.jspa\#/testCase/LVV-T810}{LVV-T810},
\href{https://jira.lsstcorp.org/secure/Tests.jspa\#/testCase/LVV-T811}{LVV-T811},
\href{https://jira.lsstcorp.org/secure/Tests.jspa\#/testCase/LVV-T812}{LVV-T812}.\\[2\baselineskip]The
requirement is fully satisfied by lower-level LSP test cases.

            \vspace{\dp0}
            } \end{minipage} \\ \cline{2-3}
            & Test Data &
            \begin{minipage}[t]{13cm}{\footnotesize
                No data.
                \vspace{\dp0}
            } \end{minipage} \\ \cline{2-3}
            & Expected Result &
                \begin{minipage}[t]{13cm}{\footnotesize
                Test cases
\href{https://jira.lsstcorp.org/secure/Tests.jspa\#/testCase/LVV-T803}{LVV-T803},
\href{https://jira.lsstcorp.org/secure/Tests.jspa\#/testCase/LVV-T810}{LVV-T810},
\href{https://jira.lsstcorp.org/secure/Tests.jspa\#/testCase/LVV-T811}{LVV-T811},
\href{https://jira.lsstcorp.org/secure/Tests.jspa\#/testCase/LVV-T812}{LVV-T812}
passed without blocking issues.

                \vspace{\dp0}
                } \end{minipage}
        \\ \midrule
    \end{longtable}

\subsection{LVV-T136 - Verify implementation of Data Product and Raw Data Access}\label{lvv-t136}

\begin{longtable}[]{llllll}
\toprule
Version & Status & Priority & Verification Type & Owner
\\\midrule
1 & Defined & Normal &
Test & Colin Slater
\\\bottomrule
\multicolumn{6}{c}{ Open \href{https://jira.lsstcorp.org/secure/Tests.jspa\#/testCase/LVV-T136}{LVV-T136} in Jira } \\
\end{longtable}

\subsubsection{Verification Elements}
\begin{itemize}
\item \href{https://jira.lsstcorp.org/browse/LVV-129}{LVV-129} - DMS-REQ-0298-V-01: Data Product and Raw Data Access

\end{itemize}

\subsubsection{Test Items}
Verify that available image, file, and catalog data products, and their
metadata and provenance information, can be listed and retrieved.


\subsubsection{Predecessors}

\subsubsection{Environment Needs}

\paragraph{Software}

\paragraph{Hardware}

\subsubsection{Input Specification}

\subsubsection{Output Specification}

\subsubsection{Test Procedure}
    \begin{longtable}[]{p{1.3cm}p{2cm}p{13cm}}
    %\toprule
    Step & \multicolumn{2}{@{}l}{Description, Input Data and Expected Result} \\ \toprule
    \endhead

            \multirow{3}{*}{ 1 } & Description &
            \begin{minipage}[t]{13cm}{\footnotesize
            Details of the Gen3 Butler and ObsTAP tables are still being worked out.
The general overview of this test will be to use some combination of the
Gen3 Butler and TAP access to the ObsTAP tables to test that the
required access is provided.

            \vspace{\dp0}
            } \end{minipage} \\ \cline{2-3}
            & Test Data &
            \begin{minipage}[t]{13cm}{\footnotesize
                No data.
                \vspace{\dp0}
            } \end{minipage} \\ \cline{2-3}
            & Expected Result &
                \begin{minipage}[t]{13cm}{\footnotesize
                Verification that the relevant data products and their related tables,
metadata, and provenance information are available and readily
accessible.

                \vspace{\dp0}
                } \end{minipage}
        \\ \midrule
    \end{longtable}

\subsection{LVV-T137 - Verify implementation of Data Product Ingest}\label{lvv-t137}

\begin{longtable}[]{llllll}
\toprule
Version & Status & Priority & Verification Type & Owner
\\\midrule
1 & Defined & Normal &
Test & Colin Slater
\\\bottomrule
\multicolumn{6}{c}{ Open \href{https://jira.lsstcorp.org/secure/Tests.jspa\#/testCase/LVV-T137}{LVV-T137} in Jira } \\
\end{longtable}

\subsubsection{Verification Elements}
\begin{itemize}
\item \href{https://jira.lsstcorp.org/browse/LVV-130}{LVV-130} - DMS-REQ-0299-V-01: Data Product Ingest

\end{itemize}

\subsubsection{Test Items}
Verify that data products can be ingested.


\subsubsection{Predecessors}

\subsubsection{Environment Needs}

\paragraph{Software}

\paragraph{Hardware}

\subsubsection{Input Specification}

\subsubsection{Output Specification}

\subsubsection{Test Procedure}
    \begin{longtable}[]{p{1.3cm}p{2cm}p{13cm}}
    %\toprule
    Step & \multicolumn{2}{@{}l}{Description, Input Data and Expected Result} \\ \toprule
    \endhead

            \multirow{3}{*}{ 1 } & Description &
            \begin{minipage}[t]{13cm}{\footnotesize
            Identify a suitable set of raw data to be run through ``mini-DRP''
processing.

            \vspace{\dp0}
            } \end{minipage} \\ \cline{2-3}
            & Test Data &
            \begin{minipage}[t]{13cm}{\footnotesize
                No data.
                \vspace{\dp0}
            } \end{minipage} \\ \cline{2-3}
            & Expected Result &
        \\ \midrule

                \multirow{3}{*}{\parbox{1.3cm}{ 2-1
                {\scriptsize from \hyperref[lvv-t1064]
                {LVV-T1064} } } }

                & {\small Description} &
                \begin{minipage}[t]{13cm}{\scriptsize
                Process data with the Data Release Production payload, starting from raw
science images and generating science data products, placing them in the
Data Backbone.

                \vspace{\dp0}
                } \end{minipage} \\ \cdashline{2-3}
                & {\small Test Data} &
                \begin{minipage}[t]{13cm}{\scriptsize
                } \end{minipage} \\ \cdashline{2-3}
                & {\small Expected Result} &
                \\ \hdashline


        \\ \midrule

                \multirow{3}{*}{\parbox{1.3cm}{ 3-1
                {\scriptsize from \hyperref[lvv-t987]
                {LVV-T987} } } }

                & {\small Description} &
                \begin{minipage}[t]{13cm}{\scriptsize
                Identify the path to the data repository, which we will refer to as
`DATA/path', then execute the following:

                \vspace{\dp0}
                } \end{minipage} \\ \cdashline{2-3}
                & {\small Test Data} &
                \begin{minipage}[t]{13cm}{\scriptsize
                } \end{minipage} \\ \cdashline{2-3}
                & {\small Expected Result} &
                    \begin{minipage}[t]{13cm}{\scriptsize
                    Butler repo available for reading.

                    \vspace{\dp0}
                    } \end{minipage}
                \\ \hdashline


        \\ \midrule

            \multirow{3}{*}{ 4 } & Description &
            \begin{minipage}[t]{13cm}{\footnotesize
            Confirm that the data products from the DRP processing have been
ingested into the Data Backbone.

            \vspace{\dp0}
            } \end{minipage} \\ \cline{2-3}
            & Test Data &
            \begin{minipage}[t]{13cm}{\footnotesize
                No data.
                \vspace{\dp0}
            } \end{minipage} \\ \cline{2-3}
            & Expected Result &
                \begin{minipage}[t]{13cm}{\footnotesize
                Processed images, catalogs, calibration information, and other related
data products are present and accessible via the Butler.

                \vspace{\dp0}
                } \end{minipage}
        \\ \midrule
    \end{longtable}

\subsection{LVV-T138 - Verify implementation of Bulk Download Service}\label{lvv-t138}

\begin{longtable}[]{llllll}
\toprule
Version & Status & Priority & Verification Type & Owner
\\\midrule
1 & Draft & Normal &
Test & Robert Gruendl
\\\bottomrule
\multicolumn{6}{c}{ Open \href{https://jira.lsstcorp.org/secure/Tests.jspa\#/testCase/LVV-T138}{LVV-T138} in Jira } \\
\end{longtable}

\subsubsection{Verification Elements}
\begin{itemize}
\item \href{https://jira.lsstcorp.org/browse/LVV-131}{LVV-131} - DMS-REQ-0300-V-01: Bulk Download Service

\end{itemize}

\subsubsection{Test Items}
Bulk Download


\subsubsection{Predecessors}

\subsubsection{Environment Needs}

\paragraph{Software}

\paragraph{Hardware}

\subsubsection{Input Specification}
A large dataset (at least a few TB) must be available.\\
Requires identity management to confirm bulk download use.\\
While this can be tested and shown to work using LSST DAC, Chilean DAC,
and IN2P3 endpoints, this should also be tested to demonstrate expected
throughput for outside users (e.g. FNAL, NERSC sites could be tested).

\subsubsection{Output Specification}

\subsubsection{Test Procedure}
    \begin{longtable}[]{p{1.3cm}p{2cm}p{13cm}}
    %\toprule
    Step & \multicolumn{2}{@{}l}{Description, Input Data and Expected Result} \\ \toprule
    \endhead

            \multirow{3}{*}{ 1 } & Description &
            \begin{minipage}[t]{13cm}{\footnotesize
            Setup large transfer request and examine the data transfer rates
achieved.

            \vspace{\dp0}
            } \end{minipage} \\ \cline{2-3}
            & Test Data &
            \begin{minipage}[t]{13cm}{\footnotesize
                No data.
                \vspace{\dp0}
            } \end{minipage} \\ \cline{2-3}
            & Expected Result &
        \\ \midrule

            \multirow{3}{*}{ 2 } & Description &
            \begin{minipage}[t]{13cm}{\footnotesize
            Test should be repeated while observing in firehose mode (with LSSTCam)
during science verification to ensure that bulk transfer does not
compromise normal nightly operations.

            \vspace{\dp0}
            } \end{minipage} \\ \cline{2-3}
            & Test Data &
            \begin{minipage}[t]{13cm}{\footnotesize
                No data.
                \vspace{\dp0}
            } \end{minipage} \\ \cline{2-3}
            & Expected Result &
        \\ \midrule
    \end{longtable}

\subsection{LVV-T140 - Verify implementation of Production Orchestration}\label{lvv-t140}

\begin{longtable}[]{llllll}
\toprule
Version & Status & Priority & Verification Type & Owner
\\\midrule
1 & Defined & Normal &
Test & Robert Gruendl
\\\bottomrule
\multicolumn{6}{c}{ Open \href{https://jira.lsstcorp.org/secure/Tests.jspa\#/testCase/LVV-T140}{LVV-T140} in Jira } \\
\end{longtable}

\subsubsection{Verification Elements}
\begin{itemize}
\item \href{https://jira.lsstcorp.org/browse/LVV-133}{LVV-133} - DMS-REQ-0302-V-01: Production Orchestration

\end{itemize}

\subsubsection{Test Items}
Demonstrate use to orchestration software to perform real-time and batch
production on LSST compute platform(s).


\subsubsection{Predecessors}

\subsubsection{Environment Needs}

\paragraph{Software}

\paragraph{Hardware}

\subsubsection{Input Specification}

\subsubsection{Output Specification}

\subsubsection{Test Procedure}
    \begin{longtable}[]{p{1.3cm}p{2cm}p{13cm}}
    %\toprule
    Step & \multicolumn{2}{@{}l}{Description, Input Data and Expected Result} \\ \toprule
    \endhead

            \multirow{3}{*}{ 1 } & Description &
            \begin{minipage}[t]{13cm}{\footnotesize
            Identify an appropriate precursor dataset.

            \vspace{\dp0}
            } \end{minipage} \\ \cline{2-3}
            & Test Data &
            \begin{minipage}[t]{13cm}{\footnotesize
                No data.
                \vspace{\dp0}
            } \end{minipage} \\ \cline{2-3}
            & Expected Result &
        \\ \midrule

            \multirow{3}{*}{ 2 } & Description &
            \begin{minipage}[t]{13cm}{\footnotesize
            Execute a batch processing job using the orchestration system, and
confirm (manually and/or via QA tools typically used for HSC
reprocessing) that the pipeline executed and produced all expected
products (or error logs in cases of failure).

            \vspace{\dp0}
            } \end{minipage} \\ \cline{2-3}
            & Test Data &
            \begin{minipage}[t]{13cm}{\footnotesize
                No data.
                \vspace{\dp0}
            } \end{minipage} \\ \cline{2-3}
            & Expected Result &
                \begin{minipage}[t]{13cm}{\footnotesize
                Calexp single-visit and coadd images, and associated catalogs, are
present in a Butler repository. Logs of the processing are available to
be inspected for identification of problems in the processing.

                \vspace{\dp0}
                } \end{minipage}
        \\ \midrule
    \end{longtable}

\subsection{LVV-T141 - Verify implementation of Production Monitoring}\label{lvv-t141}

\begin{longtable}[]{llllll}
\toprule
Version & Status & Priority & Verification Type & Owner
\\\midrule
1 & Defined & Normal &
Test & Robert Gruendl
\\\bottomrule
\multicolumn{6}{c}{ Open \href{https://jira.lsstcorp.org/secure/Tests.jspa\#/testCase/LVV-T141}{LVV-T141} in Jira } \\
\end{longtable}

\subsubsection{Verification Elements}
\begin{itemize}
\item \href{https://jira.lsstcorp.org/browse/LVV-134}{LVV-134} - DMS-REQ-0303-V-01: Production Monitoring

\end{itemize}

\subsubsection{Test Items}
Demonstrate monitoring capabilities that give real-time view of pipeline
execution and production systems usage/load.


\subsubsection{Predecessors}
\href{https://jira.lsstcorp.org/secure/Tests.jspa\#/testCase/LVV-T140}{LVV-T140}​​​​

\subsubsection{Environment Needs}

\paragraph{Software}

\paragraph{Hardware}

\subsubsection{Input Specification}
Data set and mechanism for Production Orchestration as outlined in
\href{https://jira.lsstcorp.org/secure/Tests.jspa\#/testCase/LVV-T140}{LVV-T140}.

\subsubsection{Output Specification}

\subsubsection{Test Procedure}
    \begin{longtable}[]{p{1.3cm}p{2cm}p{13cm}}
    %\toprule
    Step & \multicolumn{2}{@{}l}{Description, Input Data and Expected Result} \\ \toprule
    \endhead

                \multirow{3}{*}{\parbox{1.3cm}{ 1-1
                {\scriptsize from \hyperref[lvv-t1064]
                {LVV-T1064} } } }

                & {\small Description} &
                \begin{minipage}[t]{13cm}{\scriptsize
                Process data with the Data Release Production payload, starting from raw
science images and generating science data products, placing them in the
Data Backbone.

                \vspace{\dp0}
                } \end{minipage} \\ \cdashline{2-3}
                & {\small Test Data} &
                \begin{minipage}[t]{13cm}{\scriptsize
                } \end{minipage} \\ \cdashline{2-3}
                & {\small Expected Result} &
                \\ \hdashline


        \\ \midrule

            \multirow{3}{*}{ 2 } & Description &
            \begin{minipage}[t]{13cm}{\footnotesize
            While DRP processing is executing, monitor the progress and resource
usage of processing.

            \vspace{\dp0}
            } \end{minipage} \\ \cline{2-3}
            & Test Data &
            \begin{minipage}[t]{13cm}{\footnotesize
                No data.
                \vspace{\dp0}
            } \end{minipage} \\ \cline{2-3}
            & Expected Result &
                \begin{minipage}[t]{13cm}{\footnotesize
                Ability to monitor in real-time the orchestrated production processing,
including resource usage.

                \vspace{\dp0}
                } \end{minipage}
        \\ \midrule
    \end{longtable}

\subsection{LVV-T142 - Verify implementation of Production Fault Tolerance}\label{lvv-t142}

\begin{longtable}[]{llllll}
\toprule
Version & Status & Priority & Verification Type & Owner
\\\midrule
1 & Draft & Normal &
Test & Robert Gruendl
\\\bottomrule
\multicolumn{6}{c}{ Open \href{https://jira.lsstcorp.org/secure/Tests.jspa\#/testCase/LVV-T142}{LVV-T142} in Jira } \\
\end{longtable}

\subsubsection{Verification Elements}
\begin{itemize}
\item \href{https://jira.lsstcorp.org/browse/LVV-135}{LVV-135} - DMS-REQ-0304-V-01: Production Fault Tolerance

\end{itemize}

\subsubsection{Test Items}
Demonstrate production systems report faults in pipeline executions and
that system is able to recover. ~Where recovery can mean the ability to
provide production artifacts for examination, return production elements
ready for subsequent use, and/or reset and repeat production attempts.


\subsubsection{Predecessors}

\subsubsection{Environment Needs}

\paragraph{Software}

\paragraph{Hardware}

\subsubsection{Input Specification}

\subsubsection{Output Specification}

\subsubsection{Test Procedure}
    \begin{longtable}[]{p{1.3cm}p{2cm}p{13cm}}
    %\toprule
    Step & \multicolumn{2}{@{}l}{Description, Input Data and Expected Result} \\ \toprule
    \endhead

            \multirow{3}{*}{ 1 } & Description &
            \begin{minipage}[t]{13cm}{\footnotesize
            Execute AP and DRP, simulate failures, observe correct processing

            \vspace{\dp0}
            } \end{minipage} \\ \cline{2-3}
            & Test Data &
            \begin{minipage}[t]{13cm}{\footnotesize
                No data.
                \vspace{\dp0}
            } \end{minipage} \\ \cline{2-3}
            & Expected Result &
        \\ \midrule
    \end{longtable}

\subsection{LVV-T144 - Verify implementation of Task Specification}\label{lvv-t144}

\begin{longtable}[]{llllll}
\toprule
Version & Status & Priority & Verification Type & Owner
\\\midrule
1 & Approved & Normal &
Test & Kian-Tat Lim
\\\bottomrule
\multicolumn{6}{c}{ Open \href{https://jira.lsstcorp.org/secure/Tests.jspa\#/testCase/LVV-T144}{LVV-T144} in Jira } \\
\end{longtable}

\subsubsection{Verification Elements}
\begin{itemize}
\item \href{https://jira.lsstcorp.org/browse/LVV-136}{LVV-136} - DMS-REQ-0305-V-01: Task Specification

\end{itemize}

\subsubsection{Test Items}
Verify that the DMS provides the ability to define a new or modified
pipeline task without recompilation.


\subsubsection{Predecessors}

\subsubsection{Environment Needs}

\paragraph{Software}

\paragraph{Hardware}

\subsubsection{Input Specification}

\subsubsection{Output Specification}

\subsubsection{Test Procedure}
    \begin{longtable}[]{p{1.3cm}p{2cm}p{13cm}}
    %\toprule
    Step & \multicolumn{2}{@{}l}{Description, Input Data and Expected Result} \\ \toprule
    \endhead

            \multirow{3}{*}{ 1 } & Description &
            \begin{minipage}[t]{13cm}{\footnotesize
            Inspect software architecture. ~Verify that there exist Tasks that can
be run and configured without re-compilation.

            \vspace{\dp0}
            } \end{minipage} \\ \cline{2-3}
            & Test Data &
            \begin{minipage}[t]{13cm}{\footnotesize
                No data.
                \vspace{\dp0}
            } \end{minipage} \\ \cline{2-3}
            & Expected Result &
                \begin{minipage}[t]{13cm}{\footnotesize
                Confirmation that the software architecture has allowed for
reconfiguring and running Tasks without recompilation.

                \vspace{\dp0}
                } \end{minipage}
        \\ \midrule

            \multirow{3}{*}{ 2 } & Description &
            \begin{minipage}[t]{13cm}{\footnotesize
            Verify that an example science algorithm can be run through one of these
Tasks.~ Three examples from different areas: source measurement, image
subtraction, and photometric-redshift estimation.

            \vspace{\dp0}
            } \end{minipage} \\ \cline{2-3}
            & Test Data &
            \begin{minipage}[t]{13cm}{\footnotesize
                No data.
                \vspace{\dp0}
            } \end{minipage} \\ \cline{2-3}
            & Expected Result &
                \begin{minipage}[t]{13cm}{\footnotesize
                Successful Task execution with different configurations, including
confirmation that the outputs are different from tasks with altered
configurations.

                \vspace{\dp0}
                } \end{minipage}
        \\ \midrule
    \end{longtable}

\subsection{LVV-T145 - Verify implementation of Task Configuration}\label{lvv-t145}

\begin{longtable}[]{llllll}
\toprule
Version & Status & Priority & Verification Type & Owner
\\\midrule
1 & Approved & Normal &
Test & Robert Lupton
\\\bottomrule
\multicolumn{6}{c}{ Open \href{https://jira.lsstcorp.org/secure/Tests.jspa\#/testCase/LVV-T145}{LVV-T145} in Jira } \\
\end{longtable}

\subsubsection{Verification Elements}
\begin{itemize}
\item \href{https://jira.lsstcorp.org/browse/LVV-137}{LVV-137} - DMS-REQ-0306-V-01: Task Configuration

\end{itemize}

\subsubsection{Test Items}
Verify that the DMS software provides configuration control to define,
override, and verify the configuration for a DMS Task.


\subsubsection{Predecessors}

\subsubsection{Environment Needs}

\paragraph{Software}

\paragraph{Hardware}

\subsubsection{Input Specification}

\subsubsection{Output Specification}

\subsubsection{Test Procedure}
    \begin{longtable}[]{p{1.3cm}p{2cm}p{13cm}}
    %\toprule
    Step & \multicolumn{2}{@{}l}{Description, Input Data and Expected Result} \\ \toprule
    \endhead

            \multirow{3}{*}{ 1 } & Description &
            \begin{minipage}[t]{13cm}{\footnotesize
            Inspect software design to verify that one can define the configuration
for a Task.

            \vspace{\dp0}
            } \end{minipage} \\ \cline{2-3}
            & Test Data &
            \begin{minipage}[t]{13cm}{\footnotesize
                No data.
                \vspace{\dp0}
            } \end{minipage} \\ \cline{2-3}
            & Expected Result &
        \\ \midrule

            \multirow{3}{*}{ 2 } & Description &
            \begin{minipage}[t]{13cm}{\footnotesize
            Run a Task with a known invalid configuration. ~Verify that the error is
caught before the science algorithm executes.

            \vspace{\dp0}
            } \end{minipage} \\ \cline{2-3}
            & Test Data &
            \begin{minipage}[t]{13cm}{\footnotesize
                No data.
                \vspace{\dp0}
            } \end{minipage} \\ \cline{2-3}
            & Expected Result &
        \\ \midrule

            \multirow{3}{*}{ 3 } & Description &
            \begin{minipage}[t]{13cm}{\footnotesize
            Run a simple task with two different configurations that make a material
difference for a Task. ~E.g., specify a different source detection
threshold. ~Verify that the configuration is different between the two
runs through difference in recorded provenance and in results.

            \vspace{\dp0}
            } \end{minipage} \\ \cline{2-3}
            & Test Data &
            \begin{minipage}[t]{13cm}{\footnotesize
                No data.
                \vspace{\dp0}
            } \end{minipage} \\ \cline{2-3}
            & Expected Result &
        \\ \midrule
    \end{longtable}

\subsection{LVV-T146 - Verify implementation of DMS Initialization Component}\label{lvv-t146}

\begin{longtable}[]{llllll}
\toprule
Version & Status & Priority & Verification Type & Owner
\\\midrule
1 & Approved & Normal &
Test & Robert Gruendl
\\\bottomrule
\multicolumn{6}{c}{ Open \href{https://jira.lsstcorp.org/secure/Tests.jspa\#/testCase/LVV-T146}{LVV-T146} in Jira } \\
\end{longtable}

\subsubsection{Verification Elements}
\begin{itemize}
\item \href{https://jira.lsstcorp.org/browse/LVV-128}{LVV-128} - DMS-REQ-0297-V-01: DMS Initialization Component

\end{itemize}

\subsubsection{Test Items}
Demonstrate that the DMS can be initialized in a safe state that will
not allow data corruption/loss.


\subsubsection{Predecessors}

\subsubsection{Environment Needs}

\paragraph{Software}

\paragraph{Hardware}

\subsubsection{Input Specification}

\subsubsection{Output Specification}

\subsubsection{Test Procedure}
    \begin{longtable}[]{p{1.3cm}p{2cm}p{13cm}}
    %\toprule
    Step & \multicolumn{2}{@{}l}{Description, Input Data and Expected Result} \\ \toprule
    \endhead

            \multirow{3}{*}{ 1 } & Description &
            \begin{minipage}[t]{13cm}{\footnotesize
            Power-cycle all of the DM systems at each Facility.

            \vspace{\dp0}
            } \end{minipage} \\ \cline{2-3}
            & Test Data &
            \begin{minipage}[t]{13cm}{\footnotesize
                No data.
                \vspace{\dp0}
            } \end{minipage} \\ \cline{2-3}
            & Expected Result &
                \begin{minipage}[t]{13cm}{\footnotesize
                Restart of all DM systems.

                \vspace{\dp0}
                } \end{minipage}
        \\ \midrule

            \multirow{3}{*}{ 2 } & Description &
            \begin{minipage}[t]{13cm}{\footnotesize
            Observe each system and ensure that it has recovered in a properly
initialized state.

            \vspace{\dp0}
            } \end{minipage} \\ \cline{2-3}
            & Test Data &
            \begin{minipage}[t]{13cm}{\footnotesize
                No data.
                \vspace{\dp0}
            } \end{minipage} \\ \cline{2-3}
            & Expected Result &
                \begin{minipage}[t]{13cm}{\footnotesize
                Systems are all active and initialized for their designated purpose.

                \vspace{\dp0}
                } \end{minipage}
        \\ \midrule
    \end{longtable}

\subsection{LVV-T147 - Verify implementation of Control of Level-1 Production}\label{lvv-t147}

\begin{longtable}[]{llllll}
\toprule
Version & Status & Priority & Verification Type & Owner
\\\midrule
1 & Draft & Normal &
Test & Robert Gruendl
\\\bottomrule
\multicolumn{6}{c}{ Open \href{https://jira.lsstcorp.org/secure/Tests.jspa\#/testCase/LVV-T147}{LVV-T147} in Jira } \\
\end{longtable}

\subsubsection{Verification Elements}
\begin{itemize}
\item \href{https://jira.lsstcorp.org/browse/LVV-132}{LVV-132} - DMS-REQ-0301-V-01: Control of Level-1 Production

\end{itemize}

\subsubsection{Test Items}
Demonstrate that the DMS can control all Prompt Processing across DMS
facilities.


\subsubsection{Predecessors}

\subsubsection{Environment Needs}

\paragraph{Software}

\paragraph{Hardware}

\subsubsection{Input Specification}

\subsubsection{Output Specification}

\subsubsection{Test Procedure}
    \begin{longtable}[]{p{1.3cm}p{2cm}p{13cm}}
    %\toprule
    Step & \multicolumn{2}{@{}l}{Description, Input Data and Expected Result} \\ \toprule
    \endhead

            \multirow{3}{*}{ 1 } & Description &
            \begin{minipage}[t]{13cm}{\footnotesize
            Observe existence and capability of Prompt DMCS

            \vspace{\dp0}
            } \end{minipage} \\ \cline{2-3}
            & Test Data &
            \begin{minipage}[t]{13cm}{\footnotesize
                No data.
                \vspace{\dp0}
            } \end{minipage} \\ \cline{2-3}
            & Expected Result &
        \\ \midrule
    \end{longtable}

\subsection{LVV-T148 - Verify implementation of Unique Processing Coverage}\label{lvv-t148}

\begin{longtable}[]{llllll}
\toprule
Version & Status & Priority & Verification Type & Owner
\\\midrule
1 & Draft & Normal &
Test & Colin Slater
\\\bottomrule
\multicolumn{6}{c}{ Open \href{https://jira.lsstcorp.org/secure/Tests.jspa\#/testCase/LVV-T148}{LVV-T148} in Jira } \\
\end{longtable}

\subsubsection{Verification Elements}
\begin{itemize}
\item \href{https://jira.lsstcorp.org/browse/LVV-138}{LVV-138} - DMS-REQ-0307-V-01: Unique Processing Coverage

\end{itemize}

\subsubsection{Test Items}
Verify that a user-specified criterion can be used to process each
record in a table exactly once.


\subsubsection{Predecessors}

\subsubsection{Environment Needs}

\paragraph{Software}

\paragraph{Hardware}

\subsubsection{Input Specification}

\subsubsection{Output Specification}

\subsubsection{Test Procedure}
    \begin{longtable}[]{p{1.3cm}p{2cm}p{13cm}}
    %\toprule
    Step & \multicolumn{2}{@{}l}{Description, Input Data and Expected Result} \\ \toprule
    \endhead

            \multirow{3}{*}{ 1 } & Description &
            \begin{minipage}[t]{13cm}{\footnotesize
            Execute representative processing, observe lack of duplicates or missing
rows even in the presence of failures

            \vspace{\dp0}
            } \end{minipage} \\ \cline{2-3}
            & Test Data &
            \begin{minipage}[t]{13cm}{\footnotesize
                No data.
                \vspace{\dp0}
            } \end{minipage} \\ \cline{2-3}
            & Expected Result &
        \\ \midrule
    \end{longtable}

\subsection{LVV-T149 - Verify implementation of Catalog Queries}\label{lvv-t149}

\begin{longtable}[]{llllll}
\toprule
Version & Status & Priority & Verification Type & Owner
\\\midrule
1 & Approved & Normal &
Test & Colin Slater
\\\bottomrule
\multicolumn{6}{c}{ Open \href{https://jira.lsstcorp.org/secure/Tests.jspa\#/testCase/LVV-T149}{LVV-T149} in Jira } \\
\end{longtable}

\subsubsection{Verification Elements}
\begin{itemize}
\item \href{https://jira.lsstcorp.org/browse/LVV-33}{LVV-33} - DMS-REQ-0075-V-01: Catalog Queries

\end{itemize}

\subsubsection{Test Items}
Verify that SQL, or a similar structured language, can be used to query
catalogs.


\subsubsection{Predecessors}

\subsubsection{Environment Needs}

\paragraph{Software}

\paragraph{Hardware}

\subsubsection{Input Specification}
An operational QSERV database that has been verified via
\href{https://jira.lsstcorp.org/secure/Tests.jspa\#/testCase/LVV-T1085}{LVV-T1085}
and
\href{https://jira.lsstcorp.org/secure/Tests.jspa\#/testCase/LVV-T1086}{LVV-T1086}
and
\href{https://jira.lsstcorp.org/secure/Tests.jspa\#/testCase/LVV-T1087}{LVV-T1087}.

\subsubsection{Output Specification}

\subsubsection{Test Procedure}
    \begin{longtable}[]{p{1.3cm}p{2cm}p{13cm}}
    %\toprule
    Step & \multicolumn{2}{@{}l}{Description, Input Data and Expected Result} \\ \toprule
    \endhead

            \multirow{3}{*}{ 1 } & Description &
            \begin{minipage}[t]{13cm}{\footnotesize
            Execute a simple query (for example, the one below) and confirm that it
returns the expected result.

            \vspace{\dp0}
            } \end{minipage} \\ \cline{2-3}
            & Test Data &
            \begin{minipage}[t]{13cm}{\footnotesize
                No data.
                \vspace{\dp0}
            } \end{minipage} \\ \cline{2-3}
                & Example Code &
                \begin{minipage}[t]{13cm}{\footnotesize
                SELECT * FROM Object WHERE qserv\_areaspec\_box(316.582327, −6.839078,
316.653938, −6.781822)

                \vspace{\dp0}
                } \end{minipage} \\ \cline{2-3}
            & Expected Result &
                \begin{minipage}[t]{13cm}{\footnotesize
                A catalog of objects satisfying the specified constraints.~

                \vspace{\dp0}
                } \end{minipage}
        \\ \midrule

            \multirow{3}{*}{ 2 } & Description &
            \begin{minipage}[t]{13cm}{\footnotesize
            Repeat the query from all available access routes (e.g., an external VO
client, internal DM tools on the development cluster, the Science
Platform query tool, and from within the Notebook Aspect), confirming in
each case that the results are as expected.

            \vspace{\dp0}
            } \end{minipage} \\ \cline{2-3}
            & Test Data &
            \begin{minipage}[t]{13cm}{\footnotesize
                No data.
                \vspace{\dp0}
            } \end{minipage} \\ \cline{2-3}
            & Expected Result &
        \\ \midrule
    \end{longtable}

\subsection{LVV-T150 - Verify implementation of Maintain Archive Publicly Accessible}\label{lvv-t150}

\begin{longtable}[]{llllll}
\toprule
Version & Status & Priority & Verification Type & Owner
\\\midrule
1 & Defined & Normal &
Test & Colin Slater
\\\bottomrule
\multicolumn{6}{c}{ Open \href{https://jira.lsstcorp.org/secure/Tests.jspa\#/testCase/LVV-T150}{LVV-T150} in Jira } \\
\end{longtable}

\subsubsection{Verification Elements}
\begin{itemize}
\item \href{https://jira.lsstcorp.org/browse/LVV-34}{LVV-34} - DMS-REQ-0077-V-01: Maintain Archive Publicly Accessible

\end{itemize}

\subsubsection{Test Items}
Verify that prior data releases remain accessible.


\subsubsection{Predecessors}

\subsubsection{Environment Needs}

\paragraph{Software}

\paragraph{Hardware}

\subsubsection{Input Specification}
Availability of at least three (3) data releases, of which at least one
of them must be archived outside the QSERV database. These can be
precursor datasets, if needed.

\subsubsection{Output Specification}

\subsubsection{Test Procedure}
    \begin{longtable}[]{p{1.3cm}p{2cm}p{13cm}}
    %\toprule
    Step & \multicolumn{2}{@{}l}{Description, Input Data and Expected Result} \\ \toprule
    \endhead

            \multirow{3}{*}{ 1 } & Description &
            \begin{minipage}[t]{13cm}{\footnotesize
            Confirm that at least two data releases (the most recent, and one
previous) are accessible to users (and can be queried) from the standard
channels.~~

            \vspace{\dp0}
            } \end{minipage} \\ \cline{2-3}
            & Test Data &
            \begin{minipage}[t]{13cm}{\footnotesize
                No data.
                \vspace{\dp0}
            } \end{minipage} \\ \cline{2-3}
            & Expected Result &
                \begin{minipage}[t]{13cm}{\footnotesize
                Simple queries return catalog data from the data releases that are
available in QSERV.

                \vspace{\dp0}
                } \end{minipage}
        \\ \midrule

            \multirow{3}{*}{ 2 } & Description &
            \begin{minipage}[t]{13cm}{\footnotesize
            Confirm that previous data releases are accessible for bulk download
(perhaps with significant latency) from tape or other bulk store, and
that the downloaded tables contain the expected data products.

            \vspace{\dp0}
            } \end{minipage} \\ \cline{2-3}
            & Test Data &
            \begin{minipage}[t]{13cm}{\footnotesize
                No data.
                \vspace{\dp0}
            } \end{minipage} \\ \cline{2-3}
            & Expected Result &
                \begin{minipage}[t]{13cm}{\footnotesize
                A download of an entire previous data release from its bulk store.

                \vspace{\dp0}
                } \end{minipage}
        \\ \midrule
    \end{longtable}

\subsection{LVV-T151 - Verify Implementation of Catalog Export Formats From the Notebook Aspect}\label{lvv-t151}

\begin{longtable}[]{llllll}
\toprule
Version & Status & Priority & Verification Type & Owner
\\\midrule
1 & Approved & Normal &
Test & Colin Slater
\\\bottomrule
\multicolumn{6}{c}{ Open \href{https://jira.lsstcorp.org/secure/Tests.jspa\#/testCase/LVV-T151}{LVV-T151} in Jira } \\
\end{longtable}

\subsubsection{Verification Elements}
\begin{itemize}
\item \href{https://jira.lsstcorp.org/browse/LVV-35}{LVV-35} - DMS-REQ-0078-V-01: Catalog Export Formats

\end{itemize}

\subsubsection{Test Items}
Verify that catalog data is exportable from the notebook aspect in a
variety of community-standard formats.


\subsubsection{Predecessors}

\subsubsection{Environment Needs}

\paragraph{Software}

\paragraph{Hardware}

\subsubsection{Input Specification}

\subsubsection{Output Specification}

\subsubsection{Test Procedure}
    \begin{longtable}[]{p{1.3cm}p{2cm}p{13cm}}
    %\toprule
    Step & \multicolumn{2}{@{}l}{Description, Input Data and Expected Result} \\ \toprule
    \endhead

                \multirow{3}{*}{\parbox{1.3cm}{ 1-1
                {\scriptsize from \hyperref[lvv-t837]
                {LVV-T837} } } }

                & {\small Description} &
                \begin{minipage}[t]{13cm}{\scriptsize
                Authenticate to the notebook aspect of the LSST Science Platform
(NB-LSP). ~This is currently at
https://lsst-lsp-stable.ncsa.illinois.edu/nb.

                \vspace{\dp0}
                } \end{minipage} \\ \cdashline{2-3}
                & {\small Test Data} &
                \begin{minipage}[t]{13cm}{\scriptsize
                } \end{minipage} \\ \cdashline{2-3}
                & {\small Expected Result} &
                    \begin{minipage}[t]{13cm}{\scriptsize
                    Redirection to the spawner page of the NB-LSP allowing selection of the
containerized stack version and machine flavor.

                    \vspace{\dp0}
                    } \end{minipage}
                \\ \hdashline


                \multirow{3}{*}{\parbox{1.3cm}{ 1-2
                {\scriptsize from \hyperref[lvv-t837]
                {LVV-T837} } } }

                & {\small Description} &
                \begin{minipage}[t]{13cm}{\scriptsize
                Spawn a container by:\\
1) choosing an appropriate stack version: e.g. the latest weekly.\\
2) choosing an appropriate machine flavor: e.g. medium\\
3) click ``Spawn''

                \vspace{\dp0}
                } \end{minipage} \\ \cdashline{2-3}
                & {\small Test Data} &
                \begin{minipage}[t]{13cm}{\scriptsize
                } \end{minipage} \\ \cdashline{2-3}
                & {\small Expected Result} &
                    \begin{minipage}[t]{13cm}{\scriptsize
                    Redirection to the JupyterLab environment served from the chosen
container containing the correct stack version.

                    \vspace{\dp0}
                    } \end{minipage}
                \\ \hdashline


        \\ \midrule

                \multirow{3}{*}{\parbox{1.3cm}{ 2-1
                {\scriptsize from \hyperref[lvv-t838]
                {LVV-T838} } } }

                & {\small Description} &
                \begin{minipage}[t]{13cm}{\scriptsize
                Open a new launcher by navigating in the top menu bar ``File''
-\textgreater{} ``New Launcher''

                \vspace{\dp0}
                } \end{minipage} \\ \cdashline{2-3}
                & {\small Test Data} &
                \begin{minipage}[t]{13cm}{\scriptsize
                } \end{minipage} \\ \cdashline{2-3}
                & {\small Expected Result} &
                    \begin{minipage}[t]{13cm}{\scriptsize
                    A launcher window with several sections, potentially with several kernel
versions for each.

                    \vspace{\dp0}
                    } \end{minipage}
                \\ \hdashline


                \multirow{3}{*}{\parbox{1.3cm}{ 2-2
                {\scriptsize from \hyperref[lvv-t838]
                {LVV-T838} } } }

                & {\small Description} &
                \begin{minipage}[t]{13cm}{\scriptsize
                Select the option under ``Notebook'' labeled ``LSST'' by clicking on the
icon.

                \vspace{\dp0}
                } \end{minipage} \\ \cdashline{2-3}
                & {\small Test Data} &
                \begin{minipage}[t]{13cm}{\scriptsize
                } \end{minipage} \\ \cdashline{2-3}
                & {\small Expected Result} &
                    \begin{minipage}[t]{13cm}{\scriptsize
                    An empty notebook with a single empty cell. ~The kernel show up as
``LSST'' in the top right of the notebook.

                    \vspace{\dp0}
                    } \end{minipage}
                \\ \hdashline


        \\ \midrule

                \multirow{3}{*}{\parbox{1.3cm}{ 3-1
                {\scriptsize from \hyperref[lvv-t1207]
                {LVV-T1207} } } }

                & {\small Description} &
                \begin{minipage}[t]{13cm}{\scriptsize
                Execute a query in a notebook to select a small number of stars. In the
example code below, we query the WISE catalog, then extract the results
to an Astropy table.

                \vspace{\dp0}
                } \end{minipage} \\ \cdashline{2-3}
                & {\small Test Data} &
                \begin{minipage}[t]{13cm}{\scriptsize
                } \end{minipage} \\ \cdashline{2-3}
                & {\small Expected Result} &
                \\ \hdashline


        \\ \midrule

            \multirow{3}{*}{ 4 } & Description &
            \begin{minipage}[t]{13cm}{\footnotesize
            Using the example code below, save the files to your storage space on
the LSP Notebook Aspect.\\[2\baselineskip]Confirm that non-empty output
files appear on disk.

            \vspace{\dp0}
            } \end{minipage} \\ \cline{2-3}
            & Test Data &
            \begin{minipage}[t]{13cm}{\footnotesize
                No data.
                \vspace{\dp0}
            } \end{minipage} \\ \cline{2-3}
                & Example Code &
                \begin{minipage}[t]{13cm}{\footnotesize
                tab.write('test.csv', format='ascii.csv')\\
tab.write('test.vot', format='votable')\\
tab.write('test.fits', format='fits')

                \vspace{\dp0}
                } \end{minipage} \\ \cline{2-3}
            & Expected Result &
                \begin{minipage}[t]{13cm}{\footnotesize
                For the example given here, there should be the following files with the
file size as listed:

\begin{itemize}
\tightlist
\item
  test.csv 5.7M
\item
  test.vot 16M
\item
  test.fits 4.5M
\end{itemize}

                \vspace{\dp0}
                } \end{minipage}
        \\ \midrule

            \multirow{3}{*}{ 5 } & Description &
            \begin{minipage}[t]{13cm}{\footnotesize
            Check that these files contain the same number of rows:

            \vspace{\dp0}
            } \end{minipage} \\ \cline{2-3}
            & Test Data &
            \begin{minipage}[t]{13cm}{\footnotesize
                No data.
                \vspace{\dp0}
            } \end{minipage} \\ \cline{2-3}
                & Example Code &
                \begin{minipage}[t]{13cm}{\footnotesize
                from astropy.table import Table\\
dat\_csv = Table.read('test.csv', format='ascii.csv')\\
dat\_vot = Table.read('test.vot', format='votable')\\
dat\_fits = Table.read('test.fits',
format='fits')\\[2\baselineskip]import numpy as np\\
print(np.size(dat\_csv), np.size(dat\_vot), np.size(dat\_fits))

                \vspace{\dp0}
                } \end{minipage} \\ \cline{2-3}
            & Expected Result &
                \begin{minipage}[t]{13cm}{\footnotesize
                Print statement produces output ``97058 97058 97058''.

                \vspace{\dp0}
                } \end{minipage}
        \\ \midrule

                \multirow{3}{*}{\parbox{1.3cm}{ 6-1
                {\scriptsize from \hyperref[lvv-t1208]
                {LVV-T1208} } } }

                & {\small Description} &
                \begin{minipage}[t]{13cm}{\scriptsize
                Under the `File' menu at the top of your Jupyter notebook session,
select one of the following:\\[2\baselineskip]

\begin{itemize}
\tightlist
\item
  Save All, Exit, and Log Out
\item
  Exit and Log Out Without Saving
\end{itemize}

                \vspace{\dp0}
                } \end{minipage} \\ \cdashline{2-3}
                & {\small Test Data} &
                \begin{minipage}[t]{13cm}{\scriptsize
                } \end{minipage} \\ \cdashline{2-3}
                & {\small Expected Result} &
                    \begin{minipage}[t]{13cm}{\scriptsize
                    You will be returned to the LSP landing page:
\url{https://lsst-lsp-stable.ncsa.illinois.edu/} It is now safe to close
the browser window.~

                    \vspace{\dp0}
                    } \end{minipage}
                \\ \hdashline


        \\ \midrule
    \end{longtable}

\subsection{LVV-T152 - Verify implementation of Keep Historical Alert Archive}\label{lvv-t152}

\begin{longtable}[]{llllll}
\toprule
Version & Status & Priority & Verification Type & Owner
\\\midrule
1 & Draft & Normal &
Test & Eric Bellm
\\\bottomrule
\multicolumn{6}{c}{ Open \href{https://jira.lsstcorp.org/secure/Tests.jspa\#/testCase/LVV-T152}{LVV-T152} in Jira } \\
\end{longtable}

\subsubsection{Verification Elements}
\begin{itemize}
\item \href{https://jira.lsstcorp.org/browse/LVV-37}{LVV-37} - DMS-REQ-0094-V-01: Keep Historical Alert Archive

\end{itemize}

\subsubsection{Test Items}
Verify that the DMS preserves and makes accessible an Alert Archive for
reference and for false alert analyses


\subsubsection{Predecessors}

\subsubsection{Environment Needs}

\paragraph{Software}

\paragraph{Hardware}

\subsubsection{Input Specification}

\subsubsection{Output Specification}

\subsubsection{Test Procedure}
    \begin{longtable}[]{p{1.3cm}p{2cm}p{13cm}}
    %\toprule
    Step & \multicolumn{2}{@{}l}{Description, Input Data and Expected Result} \\ \toprule
    \endhead

            \multirow{3}{*}{ 1 } & Description &
            \begin{minipage}[t]{13cm}{\footnotesize
            Simulated alert stream, load Alert DB, observe access to Alert DB

            \vspace{\dp0}
            } \end{minipage} \\ \cline{2-3}
            & Test Data &
            \begin{minipage}[t]{13cm}{\footnotesize
                No data.
                \vspace{\dp0}
            } \end{minipage} \\ \cline{2-3}
            & Expected Result &
        \\ \midrule
    \end{longtable}

\subsection{LVV-T153 - Verify implementation of Provide Engineering and Facility Database
Archive}\label{lvv-t153}

\begin{longtable}[]{llllll}
\toprule
Version & Status & Priority & Verification Type & Owner
\\\midrule
1 & Defined & Normal &
Test & Robert Gruendl
\\\bottomrule
\multicolumn{6}{c}{ Open \href{https://jira.lsstcorp.org/secure/Tests.jspa\#/testCase/LVV-T153}{LVV-T153} in Jira } \\
\end{longtable}

\subsubsection{Verification Elements}
\begin{itemize}
\item \href{https://jira.lsstcorp.org/browse/LVV-44}{LVV-44} - DMS-REQ-0102-V-01: Provide Engineering \& Facility Database Archive

\end{itemize}

\subsubsection{Test Items}
Demonstrate Engineering and Facilities Data (images, associated
metadata, and observatory environment and control data) are archived and
available for public access within \textbf{L1PublicT (24 hours)}.~


\subsubsection{Predecessors}

\subsubsection{Environment Needs}

\paragraph{Software}

\paragraph{Hardware}

\subsubsection{Input Specification}

\subsubsection{Output Specification}

\subsubsection{Test Procedure}
    \begin{longtable}[]{p{1.3cm}p{2cm}p{13cm}}
    %\toprule
    Step & \multicolumn{2}{@{}l}{Description, Input Data and Expected Result} \\ \toprule
    \endhead

            \multirow{3}{*}{ 1 } & Description &
            \begin{minipage}[t]{13cm}{\footnotesize
            Execute a single-day operations rehearsal, ingesting (simulated) OCS
commands into the EFD.~

            \vspace{\dp0}
            } \end{minipage} \\ \cline{2-3}
            & Test Data &
            \begin{minipage}[t]{13cm}{\footnotesize
                No data.
                \vspace{\dp0}
            } \end{minipage} \\ \cline{2-3}
            & Expected Result &
        \\ \midrule

            \multirow{3}{*}{ 2 } & Description &
            \begin{minipage}[t]{13cm}{\footnotesize
            Wait at least \textbf{L1PublicT=24} hours, then access the archived EFD.
Confirm that the data products are present in the archived EFD
after~\textbf{L1PublicT=24} hours have elapsed.

            \vspace{\dp0}
            } \end{minipage} \\ \cline{2-3}
            & Test Data &
            \begin{minipage}[t]{13cm}{\footnotesize
                No data.
                \vspace{\dp0}
            } \end{minipage} \\ \cline{2-3}
            & Expected Result &
                \begin{minipage}[t]{13cm}{\footnotesize
                The EFD contains the simulated OCS commands, and they were ingested
within~\textbf{L1PublicT=24} hours of the operations rehearsal.

                \vspace{\dp0}
                } \end{minipage}
        \\ \midrule

            \multirow{3}{*}{ 3 } & Description &
            \begin{minipage}[t]{13cm}{\footnotesize
            From the public access portal to the EFD, execute a query and
demonstrate that the data are publicly available.

            \vspace{\dp0}
            } \end{minipage} \\ \cline{2-3}
            & Test Data &
            \begin{minipage}[t]{13cm}{\footnotesize
                No data.
                \vspace{\dp0}
            } \end{minipage} \\ \cline{2-3}
            & Expected Result &
                \begin{minipage}[t]{13cm}{\footnotesize
                A query at the public interface to the EFD successfully executes and
returns EFD data.

                \vspace{\dp0}
                } \end{minipage}
        \\ \midrule
    \end{longtable}

\subsection{LVV-T154 - Verify implementation of Raw Data Archiving Reliability}\label{lvv-t154}

\begin{longtable}[]{llllll}
\toprule
Version & Status & Priority & Verification Type & Owner
\\\midrule
1 & Draft & Normal &
Test & Colin Slater
\\\bottomrule
\multicolumn{6}{c}{ Open \href{https://jira.lsstcorp.org/secure/Tests.jspa\#/testCase/LVV-T154}{LVV-T154} in Jira } \\
\end{longtable}

\subsubsection{Verification Elements}
\begin{itemize}
\item \href{https://jira.lsstcorp.org/browse/LVV-140}{LVV-140} - DMS-REQ-0309-V-01: Raw Data Archiving Reliability

\end{itemize}

\subsubsection{Test Items}
Verify that raw images are reliably archived.


\subsubsection{Predecessors}

\subsubsection{Environment Needs}

\paragraph{Software}

\paragraph{Hardware}

\subsubsection{Input Specification}

\subsubsection{Output Specification}

\subsubsection{Test Procedure}
    \begin{longtable}[]{p{1.3cm}p{2cm}p{13cm}}
    %\toprule
    Step & \multicolumn{2}{@{}l}{Description, Input Data and Expected Result} \\ \toprule
    \endhead

            \multirow{3}{*}{ 1 } & Description &
            \begin{minipage}[t]{13cm}{\footnotesize
            Analyze sources of loss or corruption after mitigation to compute
estimated reliability

            \vspace{\dp0}
            } \end{minipage} \\ \cline{2-3}
            & Test Data &
            \begin{minipage}[t]{13cm}{\footnotesize
                No data.
                \vspace{\dp0}
            } \end{minipage} \\ \cline{2-3}
            & Expected Result &
        \\ \midrule
    \end{longtable}

\subsection{LVV-T155 - Verify implementation of Un-Archived Data Product Cache}\label{lvv-t155}

\begin{longtable}[]{llllll}
\toprule
Version & Status & Priority & Verification Type & Owner
\\\midrule
1 & Draft & Normal &
Test & Robert Gruendl
\\\bottomrule
\multicolumn{6}{c}{ Open \href{https://jira.lsstcorp.org/secure/Tests.jspa\#/testCase/LVV-T155}{LVV-T155} in Jira } \\
\end{longtable}

\subsubsection{Verification Elements}
\begin{itemize}
\item \href{https://jira.lsstcorp.org/browse/LVV-141}{LVV-141} - DMS-REQ-0310-V-01: Un-Archived Data Product Cache

\end{itemize}

\subsubsection{Test Items}
Demonstrate that the DMS provides low-latency storage for at least
I1CacheLifetime (30 days) to keep prompt processing pre-covery images on
hand.


\subsubsection{Predecessors}

\subsubsection{Environment Needs}

\paragraph{Software}

\paragraph{Hardware}

\subsubsection{Input Specification}

\subsubsection{Output Specification}

\subsubsection{Test Procedure}
    \begin{longtable}[]{p{1.3cm}p{2cm}p{13cm}}
    %\toprule
    Step & \multicolumn{2}{@{}l}{Description, Input Data and Expected Result} \\ \toprule
    \endhead

            \multirow{3}{*}{ 1 } & Description &
            \begin{minipage}[t]{13cm}{\footnotesize
            Delegate to DBB

            \vspace{\dp0}
            } \end{minipage} \\ \cline{2-3}
            & Test Data &
            \begin{minipage}[t]{13cm}{\footnotesize
                No data.
                \vspace{\dp0}
            } \end{minipage} \\ \cline{2-3}
            & Expected Result &
        \\ \midrule
    \end{longtable}

\subsection{LVV-T156 - Verify implementation of Regenerate Un-archived Data Products}\label{lvv-t156}

\begin{longtable}[]{llllll}
\toprule
Version & Status & Priority & Verification Type & Owner
\\\midrule
1 & Draft & Normal &
Test & Simon Krughoff
\\\bottomrule
\multicolumn{6}{c}{ Open \href{https://jira.lsstcorp.org/secure/Tests.jspa\#/testCase/LVV-T156}{LVV-T156} in Jira } \\
\end{longtable}

\subsubsection{Verification Elements}
\begin{itemize}
\item \href{https://jira.lsstcorp.org/browse/LVV-142}{LVV-142} - DMS-REQ-0311-V-01: Regenerate Un-archived Data Products

\end{itemize}

\subsubsection{Test Items}
Not all of the ancillary data products produced by a data release will
be archived permanently. ~These ancillary products have been promised as
accessible to the community.~ Show that these products can be produced
from an archived data release after the fact.


\subsubsection{Predecessors}

\subsubsection{Environment Needs}

\paragraph{Software}

\paragraph{Hardware}

\subsubsection{Input Specification}

\subsubsection{Output Specification}

\subsubsection{Test Procedure}
    \begin{longtable}[]{p{1.3cm}p{2cm}p{13cm}}
    %\toprule
    Step & \multicolumn{2}{@{}l}{Description, Input Data and Expected Result} \\ \toprule
    \endhead

            \multirow{3}{*}{ 1 } & Description &
            \begin{minipage}[t]{13cm}{\footnotesize
            Run a small DRP processing job and download unarchived data products.

            \vspace{\dp0}
            } \end{minipage} \\ \cline{2-3}
            & Test Data &
            \begin{minipage}[t]{13cm}{\footnotesize
                No data.
                \vspace{\dp0}
            } \end{minipage} \\ \cline{2-3}
            & Expected Result &
        \\ \midrule

            \multirow{3}{*}{ 2 } & Description &
            \begin{minipage}[t]{13cm}{\footnotesize
            Wait for (or force) a processing stack change so that the subsequent
re-processing will be forced to use an older software build.

            \vspace{\dp0}
            } \end{minipage} \\ \cline{2-3}
            & Test Data &
            \begin{minipage}[t]{13cm}{\footnotesize
                No data.
                \vspace{\dp0}
            } \end{minipage} \\ \cline{2-3}
            & Expected Result &
        \\ \midrule

            \multirow{3}{*}{ 3 } & Description &
            \begin{minipage}[t]{13cm}{\footnotesize
            Using provenance information from the products in Step 1, request a
re-processing and compare results with previously unarchived products.

            \vspace{\dp0}
            } \end{minipage} \\ \cline{2-3}
            & Test Data &
            \begin{minipage}[t]{13cm}{\footnotesize
                No data.
                \vspace{\dp0}
            } \end{minipage} \\ \cline{2-3}
            & Expected Result &
        \\ \midrule
    \end{longtable}

\subsection{LVV-T157 - Verify implementation Level 1 Data Product Access}\label{lvv-t157}

\begin{longtable}[]{llllll}
\toprule
Version & Status & Priority & Verification Type & Owner
\\\midrule
1 & Draft & Normal &
Test & Colin Slater
\\\bottomrule
\multicolumn{6}{c}{ Open \href{https://jira.lsstcorp.org/secure/Tests.jspa\#/testCase/LVV-T157}{LVV-T157} in Jira } \\
\end{longtable}

\subsubsection{Verification Elements}
\begin{itemize}
\item \href{https://jira.lsstcorp.org/browse/LVV-143}{LVV-143} - DMS-REQ-0312-V-01: Level 1 Data Product Access

\end{itemize}

\subsubsection{Test Items}
Verify that Level 1 Data Products are accessible by science users.


\subsubsection{Predecessors}

\subsubsection{Environment Needs}

\paragraph{Software}

\paragraph{Hardware}

\subsubsection{Input Specification}

\subsubsection{Output Specification}

\subsubsection{Test Procedure}
    \begin{longtable}[]{p{1.3cm}p{2cm}p{13cm}}
    %\toprule
    Step & \multicolumn{2}{@{}l}{Description, Input Data and Expected Result} \\ \toprule
    \endhead

            \multirow{3}{*}{ 1 } & Description &
            \begin{minipage}[t]{13cm}{\footnotesize
            Delegate to LSP

            \vspace{\dp0}
            } \end{minipage} \\ \cline{2-3}
            & Test Data &
            \begin{minipage}[t]{13cm}{\footnotesize
                No data.
                \vspace{\dp0}
            } \end{minipage} \\ \cline{2-3}
            & Expected Result &
        \\ \midrule
    \end{longtable}

\subsection{LVV-T158 - Verify implementation Level 1 and 2 Catalog Access}\label{lvv-t158}

\begin{longtable}[]{llllll}
\toprule
Version & Status & Priority & Verification Type & Owner
\\\midrule
1 & Draft & Normal &
Test & Colin Slater
\\\bottomrule
\multicolumn{6}{c}{ Open \href{https://jira.lsstcorp.org/secure/Tests.jspa\#/testCase/LVV-T158}{LVV-T158} in Jira } \\
\end{longtable}

\subsubsection{Verification Elements}
\begin{itemize}
\item \href{https://jira.lsstcorp.org/browse/LVV-144}{LVV-144} - DMS-REQ-0313-V-01: Level 1 \& 2 Catalog Access

\end{itemize}

\subsubsection{Test Items}
Verify that Data Release Products are accessible by science users.


\subsubsection{Predecessors}

\subsubsection{Environment Needs}

\paragraph{Software}

\paragraph{Hardware}

\subsubsection{Input Specification}

\subsubsection{Output Specification}

\subsubsection{Test Procedure}
    \begin{longtable}[]{p{1.3cm}p{2cm}p{13cm}}
    %\toprule
    Step & \multicolumn{2}{@{}l}{Description, Input Data and Expected Result} \\ \toprule
    \endhead

            \multirow{3}{*}{ 1 } & Description &
            \begin{minipage}[t]{13cm}{\footnotesize
            Delegate to LSP

            \vspace{\dp0}
            } \end{minipage} \\ \cline{2-3}
            & Test Data &
            \begin{minipage}[t]{13cm}{\footnotesize
                No data.
                \vspace{\dp0}
            } \end{minipage} \\ \cline{2-3}
            & Expected Result &
        \\ \midrule
    \end{longtable}

\subsection{LVV-T159 - Verify implementation of Regenerating Data Products from Previous Data
Releases}\label{lvv-t159}

\begin{longtable}[]{llllll}
\toprule
Version & Status & Priority & Verification Type & Owner
\\\midrule
1 & Draft & Normal &
Test & Simon Krughoff
\\\bottomrule
\multicolumn{6}{c}{ Open \href{https://jira.lsstcorp.org/secure/Tests.jspa\#/testCase/LVV-T159}{LVV-T159} in Jira } \\
\end{longtable}

\subsubsection{Verification Elements}
\begin{itemize}
\item \href{https://jira.lsstcorp.org/browse/LVV-167}{LVV-167} - DMS-REQ-0336-V-01: Regenerating Data Products from Previous Data
Releases

\end{itemize}

\subsubsection{Test Items}
Show that un-archived data products from previous data releases can be
generated using through the LSST Science Platform.


\subsubsection{Predecessors}

\subsubsection{Environment Needs}

\paragraph{Software}

\paragraph{Hardware}

\subsubsection{Input Specification}

\subsubsection{Output Specification}

\subsubsection{Test Procedure}
    \begin{longtable}[]{p{1.3cm}p{2cm}p{13cm}}
    %\toprule
    Step & \multicolumn{2}{@{}l}{Description, Input Data and Expected Result} \\ \toprule
    \endhead

            \multirow{3}{*}{ 1 } & Description &
            \begin{minipage}[t]{13cm}{\footnotesize
            Delegate to LSP

            \vspace{\dp0}
            } \end{minipage} \\ \cline{2-3}
            & Test Data &
            \begin{minipage}[t]{13cm}{\footnotesize
                No data.
                \vspace{\dp0}
            } \end{minipage} \\ \cline{2-3}
            & Expected Result &
        \\ \midrule
    \end{longtable}

\subsection{LVV-T160 - Verify implementation of Providing a Precovery Service}\label{lvv-t160}

\begin{longtable}[]{llllll}
\toprule
Version & Status & Priority & Verification Type & Owner
\\\midrule
1 & Draft & Normal &
Test & Gregory Dubois-Felsmann
\\\bottomrule
\multicolumn{6}{c}{ Open \href{https://jira.lsstcorp.org/secure/Tests.jspa\#/testCase/LVV-T160}{LVV-T160} in Jira } \\
\end{longtable}

\subsubsection{Verification Elements}
\begin{itemize}
\item \href{https://jira.lsstcorp.org/browse/LVV-172}{LVV-172} - DMS-REQ-0341-V-01: Max elapsed time for precovery results

\end{itemize}

\subsubsection{Test Items}
Verify that a technical capability to perform user-directed precovery
analyses on difference images exists and that it is exposed through the
LSST Science Platform. ~Verified by testing against precursor
datasets.\\
(Involves: LSP Portal, MOPS and Forced Photometry)


\subsubsection{Predecessors}

\subsubsection{Environment Needs}

\paragraph{Software}

\paragraph{Hardware}

\subsubsection{Input Specification}
\begin{enumerate}
\tightlist
\item
  DECam HiTS data could be an appropriate set for this activity.
\item
  Precovery pipelines for follow-on to alert processing must exist and
  be made available as a containerized version within the Science
  Platform.
\item
  Determine limitations over which general precovery is supported. ~I
  would suggest that precovery services be limited to current (or last
  two) DRP campaigns with the possible addition of including non-DRP
  products to encompass observations over the preceding year (does this
  then require means to re-generate PVIs from Alert Production in
  addition to DRP?)
\item
  Could re-use elements of
  \href{https://jira.lsstcorp.org/secure/Tests.jspa\#/testCase/LVV-T80}{LVV-T80}
  where quasars are used to test faint object detection.
\end{enumerate}

\subsubsection{Output Specification}

\subsubsection{Test Procedure}
    \begin{longtable}[]{p{1.3cm}p{2cm}p{13cm}}
    %\toprule
    Step & \multicolumn{2}{@{}l}{Description, Input Data and Expected Result} \\ \toprule
    \endhead

            \multirow{3}{*}{ 1 } & Description &
            \begin{minipage}[t]{13cm}{\footnotesize
            Run Precovery within follow-on Alert Production (i.e. daily
post-processing on 30 day store).

            \vspace{\dp0}
            } \end{minipage} \\ \cline{2-3}
            & Test Data &
            \begin{minipage}[t]{13cm}{\footnotesize
                No data.
                \vspace{\dp0}
            } \end{minipage} \\ \cline{2-3}
            & Expected Result &
        \\ \midrule

            \multirow{3}{*}{ 2 } & Description &
            \begin{minipage}[t]{13cm}{\footnotesize
            Within Science Platform, initiate request to perform precovery for a
list of sources over same period (and longer). ~Include among the
sources for precovery quasars from
\href{https://jira.lsstcorp.org/secure/Tests.jspa\#/testCase/LVV-T80}{LVV-T80}.~

            \vspace{\dp0}
            } \end{minipage} \\ \cline{2-3}
            & Test Data &
            \begin{minipage}[t]{13cm}{\footnotesize
                No data.
                \vspace{\dp0}
            } \end{minipage} \\ \cline{2-3}
            & Expected Result &
        \\ \midrule

            \multirow{3}{*}{ 3 } & Description &
            \begin{minipage}[t]{13cm}{\footnotesize
            Examine the results. ~Compare the results for the period where there is
overlap with precovery run\ldots{} and quasar photometry with those from
\href{https://jira.lsstcorp.org/secure/Tests.jspa\#/testCase/LVV-T80}{LVV-T80}
to verify user service performs as production services.

            \vspace{\dp0}
            } \end{minipage} \\ \cline{2-3}
            & Test Data &
            \begin{minipage}[t]{13cm}{\footnotesize
                No data.
                \vspace{\dp0}
            } \end{minipage} \\ \cline{2-3}
            & Expected Result &
        \\ \midrule
    \end{longtable}

\subsection{LVV-T161 - Verify implementation of Logging of catalog queries}\label{lvv-t161}

\begin{longtable}[]{llllll}
\toprule
Version & Status & Priority & Verification Type & Owner
\\\midrule
1 & Draft & Normal &
Test & Robert Gruendl
\\\bottomrule
\multicolumn{6}{c}{ Open \href{https://jira.lsstcorp.org/secure/Tests.jspa\#/testCase/LVV-T161}{LVV-T161} in Jira } \\
\end{longtable}

\subsubsection{Verification Elements}
\begin{itemize}
\item \href{https://jira.lsstcorp.org/browse/LVV-176}{LVV-176} - DMS-REQ-0345-V-01: Logging of catalog queries

\end{itemize}

\subsubsection{Test Items}
Demonstrate logging of queries of LSST databases. ~Logged queries are
globally available to DB administrators but otherwise private excepting
the user that made the query.


\subsubsection{Predecessors}

\subsubsection{Environment Needs}

\paragraph{Software}

\paragraph{Hardware}

\subsubsection{Input Specification}

\subsubsection{Output Specification}

\subsubsection{Test Procedure}
    \begin{longtable}[]{p{1.3cm}p{2cm}p{13cm}}
    %\toprule
    Step & \multicolumn{2}{@{}l}{Description, Input Data and Expected Result} \\ \toprule
    \endhead

            \multirow{3}{*}{ 1 } & Description &
            \begin{minipage}[t]{13cm}{\footnotesize
            Delegate to LSP

            \vspace{\dp0}
            } \end{minipage} \\ \cline{2-3}
            & Test Data &
            \begin{minipage}[t]{13cm}{\footnotesize
                No data.
                \vspace{\dp0}
            } \end{minipage} \\ \cline{2-3}
            & Expected Result &
        \\ \midrule
    \end{longtable}

\subsection{LVV-T162 - Verify implementation of Access to Previous Data Releases}\label{lvv-t162}

\begin{longtable}[]{llllll}
\toprule
Version & Status & Priority & Verification Type & Owner
\\\midrule
1 & Draft & Normal &
Test & Gregory Dubois-Felsmann
\\\bottomrule
\multicolumn{6}{c}{ Open \href{https://jira.lsstcorp.org/secure/Tests.jspa\#/testCase/LVV-T162}{LVV-T162} in Jira } \\
\end{longtable}

\subsubsection{Verification Elements}
\begin{itemize}
\item \href{https://jira.lsstcorp.org/browse/LVV-189}{LVV-189} - DMS-REQ-0363-V-01: Access to Previous Data Releases

\end{itemize}

\subsubsection{Test Items}
Verify this high-level requirement, which states that the other data
access requirements, for images and catalogs, all must be satisfied for
multiple data releases. ~Verified by inspection, i.e., by determining
that the data access system components, from middleware through APIs to
user interfaces, are designed to support data from multiple releases, as
well as by direct testing using a synthetic test environment containing
multiple releases.\\
(Involves: Data Backbone, Managed Database, LSP Portal, LSP JupyterLab,
LSP Web APIs, Parallel Distributed Database)


\subsubsection{Predecessors}

\subsubsection{Environment Needs}

\paragraph{Software}

\paragraph{Hardware}

\subsubsection{Input Specification}
Requires two or more (fake) releases within DAC (or PDAC) with common
area/observations (preferably with some differing results but could use
metadata identifying provenance).~

\subsubsection{Output Specification}

\subsubsection{Test Procedure}
    \begin{longtable}[]{p{1.3cm}p{2cm}p{13cm}}
    %\toprule
    Step & \multicolumn{2}{@{}l}{Description, Input Data and Expected Result} \\ \toprule
    \endhead

            \multirow{3}{*}{ 1 } & Description &
            \begin{minipage}[t]{13cm}{\footnotesize
            From Science Platform initiate request for image and catalog products
from one of the two release sets.

            \vspace{\dp0}
            } \end{minipage} \\ \cline{2-3}
            & Test Data &
            \begin{minipage}[t]{13cm}{\footnotesize
                No data.
                \vspace{\dp0}
            } \end{minipage} \\ \cline{2-3}
            & Expected Result &
        \\ \midrule

            \multirow{3}{*}{ 2 } & Description &
            \begin{minipage}[t]{13cm}{\footnotesize
            From Science Platform re-issue the same request but specifying the
alternate/earlier release set.

            \vspace{\dp0}
            } \end{minipage} \\ \cline{2-3}
            & Test Data &
            \begin{minipage}[t]{13cm}{\footnotesize
                No data.
                \vspace{\dp0}
            } \end{minipage} \\ \cline{2-3}
            & Expected Result &
        \\ \midrule

            \multirow{3}{*}{ 3 } & Description &
            \begin{minipage}[t]{13cm}{\footnotesize
            Compare results and identify differences that are germaine to the
relevant Data Release Sets are found.

            \vspace{\dp0}
            } \end{minipage} \\ \cline{2-3}
            & Test Data &
            \begin{minipage}[t]{13cm}{\footnotesize
                No data.
                \vspace{\dp0}
            } \end{minipage} \\ \cline{2-3}
            & Expected Result &
        \\ \midrule
    \end{longtable}

\subsection{LVV-T163 - Verify implementation of Data Access Services}\label{lvv-t163}

\begin{longtable}[]{llllll}
\toprule
Version & Status & Priority & Verification Type & Owner
\\\midrule
1 & Draft & Normal &
Test & Robert Gruendl
\\\bottomrule
\multicolumn{6}{c}{ Open \href{https://jira.lsstcorp.org/secure/Tests.jspa\#/testCase/LVV-T163}{LVV-T163} in Jira } \\
\end{longtable}

\subsubsection{Verification Elements}
\begin{itemize}
\item \href{https://jira.lsstcorp.org/browse/LVV-190}{LVV-190} - DMS-REQ-0364-V-01: Total number of data releases

\end{itemize}

\subsubsection{Test Items}
Demonstrate that Data Access Services are capable of scaling to serve
data from nDRTot (11) data releases over a surveyYears (10) year survey.


\subsubsection{Predecessors}

\subsubsection{Environment Needs}

\paragraph{Software}

\paragraph{Hardware}

\subsubsection{Input Specification}

\subsubsection{Output Specification}

\subsubsection{Test Procedure}
    \begin{longtable}[]{p{1.3cm}p{2cm}p{13cm}}
    %\toprule
    Step & \multicolumn{2}{@{}l}{Description, Input Data and Expected Result} \\ \toprule
    \endhead

            \multirow{3}{*}{ 1 } & Description &
            \begin{minipage}[t]{13cm}{\footnotesize
            Delegate to LSP

            \vspace{\dp0}
            } \end{minipage} \\ \cline{2-3}
            & Test Data &
            \begin{minipage}[t]{13cm}{\footnotesize
                No data.
                \vspace{\dp0}
            } \end{minipage} \\ \cline{2-3}
            & Expected Result &
        \\ \midrule
    \end{longtable}

\subsection{LVV-T164 - Verify implementation of Operations Subsets}\label{lvv-t164}

\begin{longtable}[]{llllll}
\toprule
Version & Status & Priority & Verification Type & Owner
\\\midrule
1 & Draft & Normal &
Test & Robert Gruendl
\\\bottomrule
\multicolumn{6}{c}{ Open \href{https://jira.lsstcorp.org/secure/Tests.jspa\#/testCase/LVV-T164}{LVV-T164} in Jira } \\
\end{longtable}

\subsubsection{Verification Elements}
\begin{itemize}
\item \href{https://jira.lsstcorp.org/browse/LVV-191}{LVV-191} - DMS-REQ-0365-V-01: Operations Subsets

\end{itemize}

\subsubsection{Test Items}
Demonstrate that Data Access Services are designed such that subsets of
a Data Release may be retained and served (made available) after a Data
Release has been superseded. ~ (Data Backbone, Managed Database, LSP
Portal, LSP JupyterLab, LSP Web APIs, Parallel Distributed Database)


\subsubsection{Predecessors}

\subsubsection{Environment Needs}

\paragraph{Software}

\paragraph{Hardware}

\subsubsection{Input Specification}

\subsubsection{Output Specification}

\subsubsection{Test Procedure}
    \begin{longtable}[]{p{1.3cm}p{2cm}p{13cm}}
    %\toprule
    Step & \multicolumn{2}{@{}l}{Description, Input Data and Expected Result} \\ \toprule
    \endhead

            \multirow{3}{*}{ 1 } & Description &
            \begin{minipage}[t]{13cm}{\footnotesize
            Delegate to LSP

            \vspace{\dp0}
            } \end{minipage} \\ \cline{2-3}
            & Test Data &
            \begin{minipage}[t]{13cm}{\footnotesize
                No data.
                \vspace{\dp0}
            } \end{minipage} \\ \cline{2-3}
            & Expected Result &
        \\ \midrule
    \end{longtable}

\subsection{LVV-T165 - Verify implementation of Subsets Support}\label{lvv-t165}

\begin{longtable}[]{llllll}
\toprule
Version & Status & Priority & Verification Type & Owner
\\\midrule
1 & Draft & Normal &
Test & Robert Lupton
\\\bottomrule
\multicolumn{6}{c}{ Open \href{https://jira.lsstcorp.org/secure/Tests.jspa\#/testCase/LVV-T165}{LVV-T165} in Jira } \\
\end{longtable}

\subsubsection{Verification Elements}
\begin{itemize}
\item \href{https://jira.lsstcorp.org/browse/LVV-192}{LVV-192} - DMS-REQ-0366-V-01: Subsets Support

\end{itemize}

\subsubsection{Test Items}
Verify that the DMS can provide designated subsets of previous Data
Releases.


\subsubsection{Predecessors}

\subsubsection{Environment Needs}

\paragraph{Software}

\paragraph{Hardware}

\subsubsection{Input Specification}

\subsubsection{Output Specification}

\subsubsection{Test Procedure}
    \begin{longtable}[]{p{1.3cm}p{2cm}p{13cm}}
    %\toprule
    Step & \multicolumn{2}{@{}l}{Description, Input Data and Expected Result} \\ \toprule
    \endhead

            \multirow{3}{*}{ 1 } & Description &
            \begin{minipage}[t]{13cm}{\footnotesize
            Delegate to LSP

            \vspace{\dp0}
            } \end{minipage} \\ \cline{2-3}
            & Test Data &
            \begin{minipage}[t]{13cm}{\footnotesize
                No data.
                \vspace{\dp0}
            } \end{minipage} \\ \cline{2-3}
            & Expected Result &
        \\ \midrule
    \end{longtable}

\subsection{LVV-T166 - Verify implementation of Access Services Performance}\label{lvv-t166}

\begin{longtable}[]{llllll}
\toprule
Version & Status & Priority & Verification Type & Owner
\\\midrule
1 & Draft & Normal &
Test & Robert Gruendl
\\\bottomrule
\multicolumn{6}{c}{ Open \href{https://jira.lsstcorp.org/secure/Tests.jspa\#/testCase/LVV-T166}{LVV-T166} in Jira } \\
\end{longtable}

\subsubsection{Verification Elements}
\begin{itemize}
\item \href{https://jira.lsstcorp.org/browse/LVV-193}{LVV-193} - DMS-REQ-0367-V-01: Access Services Performance

\end{itemize}

\subsubsection{Test Items}
Demonstrate monitoring of Data Access Services that give real and
long-time views of system performance and usage.


\subsubsection{Predecessors}

\subsubsection{Environment Needs}

\paragraph{Software}

\paragraph{Hardware}

\subsubsection{Input Specification}

\subsubsection{Output Specification}

\subsubsection{Test Procedure}
    \begin{longtable}[]{p{1.3cm}p{2cm}p{13cm}}
    %\toprule
    Step & \multicolumn{2}{@{}l}{Description, Input Data and Expected Result} \\ \toprule
    \endhead

            \multirow{3}{*}{ 1 } & Description &
            \begin{minipage}[t]{13cm}{\footnotesize
            Delegate to LSP

            \vspace{\dp0}
            } \end{minipage} \\ \cline{2-3}
            & Test Data &
            \begin{minipage}[t]{13cm}{\footnotesize
                No data.
                \vspace{\dp0}
            } \end{minipage} \\ \cline{2-3}
            & Expected Result &
        \\ \midrule
    \end{longtable}

\subsection{LVV-T167 - Verify Capability to serve older Data Releases at Full Performance}\label{lvv-t167}

\begin{longtable}[]{llllll}
\toprule
Version & Status & Priority & Verification Type & Owner
\\\midrule
1 & Draft & Normal &
Test & Robert Gruendl
\\\bottomrule
\multicolumn{6}{c}{ Open \href{https://jira.lsstcorp.org/secure/Tests.jspa\#/testCase/LVV-T167}{LVV-T167} in Jira } \\
\end{longtable}

\subsubsection{Verification Elements}
\begin{itemize}
\item \href{https://jira.lsstcorp.org/browse/LVV-194}{LVV-194} - DMS-REQ-0368-V-01: Implementation Provisions

\end{itemize}

\subsubsection{Test Items}
Verify that implementation of the data access services do not preclude
serving all older Data Releases with the same performance requirements
as current Data Releases. ~Note that it is an operational consideration
whether sufficient compute and storage resources would actually be
provisioned to meet those requirements.


\subsubsection{Predecessors}

\subsubsection{Environment Needs}

\paragraph{Software}

\paragraph{Hardware}

\subsubsection{Input Specification}

\subsubsection{Output Specification}

\subsubsection{Test Procedure}
    \begin{longtable}[]{p{1.3cm}p{2cm}p{13cm}}
    %\toprule
    Step & \multicolumn{2}{@{}l}{Description, Input Data and Expected Result} \\ \toprule
    \endhead

            \multirow{3}{*}{ 1 } & Description &
            \begin{minipage}[t]{13cm}{\footnotesize
            Delegate to LSP

            \vspace{\dp0}
            } \end{minipage} \\ \cline{2-3}
            & Test Data &
            \begin{minipage}[t]{13cm}{\footnotesize
                No data.
                \vspace{\dp0}
            } \end{minipage} \\ \cline{2-3}
            & Expected Result &
        \\ \midrule
    \end{longtable}

\subsection{LVV-T168 - Verify design of Data Access Services allows Evolution of the LSST Data
Model}\label{lvv-t168}

\begin{longtable}[]{llllll}
\toprule
Version & Status & Priority & Verification Type & Owner
\\\midrule
1 & Draft & Normal &
Test & Robert Gruendl
\\\bottomrule
\multicolumn{6}{c}{ Open \href{https://jira.lsstcorp.org/secure/Tests.jspa\#/testCase/LVV-T168}{LVV-T168} in Jira } \\
\end{longtable}

\subsubsection{Verification Elements}
\begin{itemize}
\item \href{https://jira.lsstcorp.org/browse/LVV-195}{LVV-195} - DMS-REQ-0369-V-01: Evolution

\end{itemize}

\subsubsection{Test Items}
Verify that the design of the Data Access Services are able to
accommodate changes/evolution of the LSST data model from one release to
another.


\subsubsection{Predecessors}

\subsubsection{Environment Needs}

\paragraph{Software}

\paragraph{Hardware}

\subsubsection{Input Specification}

\subsubsection{Output Specification}

\subsubsection{Test Procedure}
    \begin{longtable}[]{p{1.3cm}p{2cm}p{13cm}}
    %\toprule
    Step & \multicolumn{2}{@{}l}{Description, Input Data and Expected Result} \\ \toprule
    \endhead

            \multirow{3}{*}{ 1 } & Description &
            \begin{minipage}[t]{13cm}{\footnotesize
            Delegate to LSP

            \vspace{\dp0}
            } \end{minipage} \\ \cline{2-3}
            & Test Data &
            \begin{minipage}[t]{13cm}{\footnotesize
                No data.
                \vspace{\dp0}
            } \end{minipage} \\ \cline{2-3}
            & Expected Result &
        \\ \midrule
    \end{longtable}

\subsection{LVV-T169 - Verify implementation of Older Release Behavior}\label{lvv-t169}

\begin{longtable}[]{llllll}
\toprule
Version & Status & Priority & Verification Type & Owner
\\\midrule
1 & Draft & Normal &
Test & Gregory Dubois-Felsmann
\\\bottomrule
\multicolumn{6}{c}{ Open \href{https://jira.lsstcorp.org/secure/Tests.jspa\#/testCase/LVV-T169}{LVV-T169} in Jira } \\
\end{longtable}

\subsubsection{Verification Elements}
\begin{itemize}
\item \href{https://jira.lsstcorp.org/browse/LVV-196}{LVV-196} - DMS-REQ-0370-V-01: Older Release Behavior

\end{itemize}

\subsubsection{Test Items}
Verify that the components of the data access system are technically
capable of handling data releases beyond the two for which full services
are required. ~DMS-REQ-0364 requires that up to 11 be supported.
~Verified by inspection, i.e., by determination that the system design
and implementation contain the necessary features to support this number
of releases, and by direct test in a synthetic test environment with
multiple releases.\\
(Involves: Data Backbone, Managed Database, LSP Portal, LSP JupyterLab,
LSP Web APIs, Parallel Distributed Database)


\subsubsection{Predecessors}

\subsubsection{Environment Needs}

\paragraph{Software}

\paragraph{Hardware}

\subsubsection{Input Specification}

\subsubsection{Output Specification}

\subsubsection{Test Procedure}
    \begin{longtable}[]{p{1.3cm}p{2cm}p{13cm}}
    %\toprule
    Step & \multicolumn{2}{@{}l}{Description, Input Data and Expected Result} \\ \toprule
    \endhead

            \multirow{3}{*}{ 1 } & Description &
            \begin{minipage}[t]{13cm}{\footnotesize
            Delegate to LSP

            \vspace{\dp0}
            } \end{minipage} \\ \cline{2-3}
            & Test Data &
            \begin{minipage}[t]{13cm}{\footnotesize
                No data.
                \vspace{\dp0}
            } \end{minipage} \\ \cline{2-3}
            & Expected Result &
        \\ \midrule
    \end{longtable}

\subsection{LVV-T170 - Verify implementation of Query Availability}\label{lvv-t170}

\begin{longtable}[]{llllll}
\toprule
Version & Status & Priority & Verification Type & Owner
\\\midrule
1 & Draft & Normal &
Test & Colin Slater
\\\bottomrule
\multicolumn{6}{c}{ Open \href{https://jira.lsstcorp.org/secure/Tests.jspa\#/testCase/LVV-T170}{LVV-T170} in Jira } \\
\end{longtable}

\subsubsection{Verification Elements}
\begin{itemize}
\item \href{https://jira.lsstcorp.org/browse/LVV-197}{LVV-197} - DMS-REQ-0371-V-01: Query Availability

\end{itemize}

\subsubsection{Test Items}
Verify that queries continue to be successfully executable over time.


\subsubsection{Predecessors}

\subsubsection{Environment Needs}

\paragraph{Software}

\paragraph{Hardware}

\subsubsection{Input Specification}

\subsubsection{Output Specification}

\subsubsection{Test Procedure}
    \begin{longtable}[]{p{1.3cm}p{2cm}p{13cm}}
    %\toprule
    Step & \multicolumn{2}{@{}l}{Description, Input Data and Expected Result} \\ \toprule
    \endhead

            \multirow{3}{*}{ 1 } & Description &
            \begin{minipage}[t]{13cm}{\footnotesize
            Delegate to LSP

            \vspace{\dp0}
            } \end{minipage} \\ \cline{2-3}
            & Test Data &
            \begin{minipage}[t]{13cm}{\footnotesize
                No data.
                \vspace{\dp0}
            } \end{minipage} \\ \cline{2-3}
            & Expected Result &
        \\ \midrule
    \end{longtable}

\subsection{LVV-T171 - Verify implementation of Pipeline Availability}\label{lvv-t171}

\begin{longtable}[]{llllll}
\toprule
Version & Status & Priority & Verification Type & Owner
\\\midrule
1 & Draft & Normal &
Test & Robert Gruendl
\\\bottomrule
\multicolumn{6}{c}{ Open \href{https://jira.lsstcorp.org/secure/Tests.jspa\#/testCase/LVV-T171}{LVV-T171} in Jira } \\
\end{longtable}

\subsubsection{Verification Elements}
\begin{itemize}
\item \href{https://jira.lsstcorp.org/browse/LVV-5}{LVV-5} - DMS-REQ-0008-V-01: Pipeline Availability

\end{itemize}

\subsubsection{Test Items}
Demonstrate that Data Management System pipelines are available for use
without disruptions of greater than productionMaxDowntime (24 hours). ~
This requires a regimented change control process and testing
infrastructure for all pipelines and their underlying software services,
and regimented management and monitoring of compute and networking
resources. ~The list of services covered by this test include: Image and
EFD Archiving, Prompt Processing, OCS Driven Batch, Telemetry Gateway,
Alert Distribution, Alert Filtering, Batch Production, Data Backbone,
Compute/Storage/LAN, Inter-Site Networks, and Service Management and
Monitoring.


\subsubsection{Predecessors}

\subsubsection{Environment Needs}

\paragraph{Software}

\paragraph{Hardware}

\subsubsection{Input Specification}

\subsubsection{Output Specification}

\subsubsection{Test Procedure}
    \begin{longtable}[]{p{1.3cm}p{2cm}p{13cm}}
    %\toprule
    Step & \multicolumn{2}{@{}l}{Description, Input Data and Expected Result} \\ \toprule
    \endhead

            \multirow{3}{*}{ 1 } & Description &
            \begin{minipage}[t]{13cm}{\footnotesize
            Analyze sources of downtime after mitigation to compute estimated
reliability; observe unscheduled downtime of developer, integration, and
pre-production systems

            \vspace{\dp0}
            } \end{minipage} \\ \cline{2-3}
            & Test Data &
            \begin{minipage}[t]{13cm}{\footnotesize
                No data.
                \vspace{\dp0}
            } \end{minipage} \\ \cline{2-3}
            & Expected Result &
        \\ \midrule
    \end{longtable}

\subsection{LVV-T172 - Verify implementation of Optimization of Cost, Reliability and
Availability}\label{lvv-t172}

\begin{longtable}[]{llllll}
\toprule
Version & Status & Priority & Verification Type & Owner
\\\midrule
1 & Draft & Normal &
Test & Robert Gruendl
\\\bottomrule
\multicolumn{6}{c}{ Open \href{https://jira.lsstcorp.org/secure/Tests.jspa\#/testCase/LVV-T172}{LVV-T172} in Jira } \\
\end{longtable}

\subsubsection{Verification Elements}
\begin{itemize}
\item \href{https://jira.lsstcorp.org/browse/LVV-64}{LVV-64} - DMS-REQ-0161-V-01: Optimization of Cost, Reliability and Availability in
Order

\end{itemize}

\subsubsection{Test Items}
In matters of cost, system reliability (functioning properly at a given
time) has precedence over system availability (ability to use the system
at a given time). ~ The optimization may be outside the realm of direct
testing as it is more of a system provisioning guideline but on its face
it demands that the Data Management System include failure reporting,
regimented change control, acceptance testing, maintenance and
monitoring.


\subsubsection{Predecessors}

\subsubsection{Environment Needs}

\paragraph{Software}

\paragraph{Hardware}

\subsubsection{Input Specification}

\subsubsection{Output Specification}

\subsubsection{Test Procedure}
    \begin{longtable}[]{p{1.3cm}p{2cm}p{13cm}}
    %\toprule
    Step & \multicolumn{2}{@{}l}{Description, Input Data and Expected Result} \\ \toprule
    \endhead

            \multirow{3}{*}{ 1 } & Description &
            \begin{minipage}[t]{13cm}{\footnotesize
            Analyze resource management policy

            \vspace{\dp0}
            } \end{minipage} \\ \cline{2-3}
            & Test Data &
            \begin{minipage}[t]{13cm}{\footnotesize
                No data.
                \vspace{\dp0}
            } \end{minipage} \\ \cline{2-3}
            & Expected Result &
        \\ \midrule
    \end{longtable}

\subsection{LVV-T173 - Verify implementation of Pipeline Throughput}\label{lvv-t173}

\begin{longtable}[]{llllll}
\toprule
Version & Status & Priority & Verification Type & Owner
\\\midrule
1 & Draft & Normal &
Test & Robert Gruendl
\\\bottomrule
\multicolumn{6}{c}{ Open \href{https://jira.lsstcorp.org/secure/Tests.jspa\#/testCase/LVV-T173}{LVV-T173} in Jira } \\
\end{longtable}

\subsubsection{Verification Elements}
\begin{itemize}
\item \href{https://jira.lsstcorp.org/browse/LVV-65}{LVV-65} - DMS-REQ-0162-V-01: Pipeline Throughput

\end{itemize}

\subsubsection{Test Items}
Demonstrate that the Alert Production Pipeline is capable of processing
nRawExpNightMax (2800) science exposures within a (24-nightDurationMax)
12 hour period and issue alerts in offline batch mode.~


\subsubsection{Predecessors}

\subsubsection{Environment Needs}

\paragraph{Software}

\paragraph{Hardware}

\subsubsection{Input Specification}

\subsubsection{Output Specification}

\subsubsection{Test Procedure}
    \begin{longtable}[]{p{1.3cm}p{2cm}p{13cm}}
    %\toprule
    Step & \multicolumn{2}{@{}l}{Description, Input Data and Expected Result} \\ \toprule
    \endhead

            \multirow{3}{*}{ 1 } & Description &
            \begin{minipage}[t]{13cm}{\footnotesize
            Execute single-day operations rehearsal, observe data products generated
in time

            \vspace{\dp0}
            } \end{minipage} \\ \cline{2-3}
            & Test Data &
            \begin{minipage}[t]{13cm}{\footnotesize
                No data.
                \vspace{\dp0}
            } \end{minipage} \\ \cline{2-3}
            & Expected Result &
        \\ \midrule
    \end{longtable}

\subsection{LVV-T174 - Verify implementation of Re-processing Capacity}\label{lvv-t174}

\begin{longtable}[]{llllll}
\toprule
Version & Status & Priority & Verification Type & Owner
\\\midrule
1 & Draft & Normal &
Test & Robert Gruendl
\\\bottomrule
\multicolumn{6}{c}{ Open \href{https://jira.lsstcorp.org/secure/Tests.jspa\#/testCase/LVV-T174}{LVV-T174} in Jira } \\
\end{longtable}

\subsubsection{Verification Elements}
\begin{itemize}
\item \href{https://jira.lsstcorp.org/browse/LVV-66}{LVV-66} - DMS-REQ-0163-V-01: Re-processing Capacity

\end{itemize}

\subsubsection{Test Items}
Verify that the DMS has sufficient processing, storage, and network to
reprocess all data within ``drProcessingPeriod'' (1 year) while
maintaining full Prompt Processing capability.


\subsubsection{Predecessors}

\subsubsection{Environment Needs}

\paragraph{Software}

\paragraph{Hardware}

\subsubsection{Input Specification}

\subsubsection{Output Specification}

\subsubsection{Test Procedure}
    \begin{longtable}[]{p{1.3cm}p{2cm}p{13cm}}
    %\toprule
    Step & \multicolumn{2}{@{}l}{Description, Input Data and Expected Result} \\ \toprule
    \endhead

            \multirow{3}{*}{ 1 } & Description &
            \begin{minipage}[t]{13cm}{\footnotesize
            ~Analyze sizing model; execute DRP, observe scaling

            \vspace{\dp0}
            } \end{minipage} \\ \cline{2-3}
            & Test Data &
            \begin{minipage}[t]{13cm}{\footnotesize
                No data.
                \vspace{\dp0}
            } \end{minipage} \\ \cline{2-3}
            & Expected Result &
        \\ \midrule
    \end{longtable}

\subsection{LVV-T175 - Verify implementation of Temporary Storage for Communications Links}\label{lvv-t175}

\begin{longtable}[]{llllll}
\toprule
Version & Status & Priority & Verification Type & Owner
\\\midrule
1 & Draft & Normal &
Test & Robert Gruendl
\\\bottomrule
\multicolumn{6}{c}{ Open \href{https://jira.lsstcorp.org/secure/Tests.jspa\#/testCase/LVV-T175}{LVV-T175} in Jira } \\
\end{longtable}

\subsubsection{Verification Elements}
\begin{itemize}
\item \href{https://jira.lsstcorp.org/browse/LVV-67}{LVV-67} - DMS-REQ-0164-V-01: Temporary Storage for Communications Links

\end{itemize}

\subsubsection{Test Items}
Demonstrate that storage capacity is present and usable to prevent data
loss if networking is interrupted between summit and base, base and
archive, or archive and DAC. ~The requirement is to have storage
necessary to hold tempStorageReIMTTR (200\%) of the expected raw data
that would arrive during the Mean Time to Repair (summToBaseNetMTTR = 24
hours, baseToArchNetMTTR = 48 hours, ~archToDacNetMTTR = 48 hours).
~This scale is further set by nCalibExpDay + nRawExpNightMax = 450 +
2800 = ~3250 exposures/day.


\subsubsection{Predecessors}

\subsubsection{Environment Needs}

\paragraph{Software}

\paragraph{Hardware}

\subsubsection{Input Specification}

\subsubsection{Output Specification}

\subsubsection{Test Procedure}
    \begin{longtable}[]{p{1.3cm}p{2cm}p{13cm}}
    %\toprule
    Step & \multicolumn{2}{@{}l}{Description, Input Data and Expected Result} \\ \toprule
    \endhead

            \multirow{3}{*}{ 1 } & Description &
            \begin{minipage}[t]{13cm}{\footnotesize
            Analyze sizing model and network/storage design

            \vspace{\dp0}
            } \end{minipage} \\ \cline{2-3}
            & Test Data &
            \begin{minipage}[t]{13cm}{\footnotesize
                No data.
                \vspace{\dp0}
            } \end{minipage} \\ \cline{2-3}
            & Expected Result &
        \\ \midrule
    \end{longtable}

\subsection{LVV-T176 - Verify implementation of Infrastructure Sizing for ``catching up''}\label{lvv-t176}

\begin{longtable}[]{llllll}
\toprule
Version & Status & Priority & Verification Type & Owner
\\\midrule
1 & Draft & Normal &
Test & Robert Gruendl
\\\bottomrule
\multicolumn{6}{c}{ Open \href{https://jira.lsstcorp.org/secure/Tests.jspa\#/testCase/LVV-T176}{LVV-T176} in Jira } \\
\end{longtable}

\subsubsection{Verification Elements}
\begin{itemize}
\item \href{https://jira.lsstcorp.org/browse/LVV-68}{LVV-68} - DMS-REQ-0165-V-01: Infrastructure Sizing for ``catching up''

\item \href{https://jira.lsstcorp.org/browse/LVV-994}{LVV-994} - OSS-REQ-0051-V-01: Summit-Base Connectivity Loss

\end{itemize}

\subsubsection{Test Items}
Demonstrate Data Management System has sufficient excess capacity
(compute infrastructure) to process one night's data (2800 exposures)
within 24 hours while also maintaining nightly Alert Production (note
this is very similar to
\href{https://jira.lsstcorp.org/secure/Tests.jspa\#/testCase/LVV-T173}{LVV-T173}).~


\subsubsection{Predecessors}

\subsubsection{Environment Needs}

\paragraph{Software}

\paragraph{Hardware}

\subsubsection{Input Specification}

\subsubsection{Output Specification}

\subsubsection{Test Procedure}
    \begin{longtable}[]{p{1.3cm}p{2cm}p{13cm}}
    %\toprule
    Step & \multicolumn{2}{@{}l}{Description, Input Data and Expected Result} \\ \toprule
    \endhead

            \multirow{3}{*}{ 1 } & Description &
            \begin{minipage}[t]{13cm}{\footnotesize
            Execute single-day operations rehearsal including catch-up after
failure, observe data products generated in time

            \vspace{\dp0}
            } \end{minipage} \\ \cline{2-3}
            & Test Data &
            \begin{minipage}[t]{13cm}{\footnotesize
                No data.
                \vspace{\dp0}
            } \end{minipage} \\ \cline{2-3}
            & Expected Result &
        \\ \midrule
    \end{longtable}

\subsection{LVV-T177 - Verify implementation of Incorporate Fault-Tolerance}\label{lvv-t177}

\begin{longtable}[]{llllll}
\toprule
Version & Status & Priority & Verification Type & Owner
\\\midrule
1 & Draft & Normal &
Test & Robert Gruendl
\\\bottomrule
\multicolumn{6}{c}{ Open \href{https://jira.lsstcorp.org/secure/Tests.jspa\#/testCase/LVV-T177}{LVV-T177} in Jira } \\
\end{longtable}

\subsubsection{Verification Elements}
\begin{itemize}
\item \href{https://jira.lsstcorp.org/browse/LVV-69}{LVV-69} - DMS-REQ-0166-V-01: Incorporate Fault-Tolerance

\end{itemize}

\subsubsection{Test Items}
Demonstrate that Data Management Systems have features that prevent data
loss. ~Includes: MD5SUM/checksum verification for data transfer; RAID to
eliminate single-point disk failures; multi-site and tape for disaster
recovery of raw data; multiple site (and tape?) for backup/recovery of
Data Release products; DB transaction logging and backup to maintain DB
integrity. ~ (Note: storage to prevent loss in case of networking
failures is covered in
\href{https://jira.lsstcorp.org/secure/Tests.jspa\#/testCase/LVV-T175}{LVV-T175}
). ~~


\subsubsection{Predecessors}

\subsubsection{Environment Needs}

\paragraph{Software}

\paragraph{Hardware}

\subsubsection{Input Specification}

\subsubsection{Output Specification}

\subsubsection{Test Procedure}
    \begin{longtable}[]{p{1.3cm}p{2cm}p{13cm}}
    %\toprule
    Step & \multicolumn{2}{@{}l}{Description, Input Data and Expected Result} \\ \toprule
    \endhead

            \multirow{3}{*}{ 1 } & Description &
            \begin{minipage}[t]{13cm}{\footnotesize
            Analyze design; execute single-day operations rehearsal including
failures, observe recovery without loss of data

            \vspace{\dp0}
            } \end{minipage} \\ \cline{2-3}
            & Test Data &
            \begin{minipage}[t]{13cm}{\footnotesize
                No data.
                \vspace{\dp0}
            } \end{minipage} \\ \cline{2-3}
            & Expected Result &
        \\ \midrule
    \end{longtable}

\subsection{LVV-T178 - Verify implementation of Incorporate Autonomics}\label{lvv-t178}

\begin{longtable}[]{llllll}
\toprule
Version & Status & Priority & Verification Type & Owner
\\\midrule
1 & Draft & Normal &
Test & Robert Gruendl
\\\bottomrule
\multicolumn{6}{c}{ Open \href{https://jira.lsstcorp.org/secure/Tests.jspa\#/testCase/LVV-T178}{LVV-T178} in Jira } \\
\end{longtable}

\subsubsection{Verification Elements}
\begin{itemize}
\item \href{https://jira.lsstcorp.org/browse/LVV-70}{LVV-70} - DMS-REQ-0167-V-01: Incorporate Autonomics

\end{itemize}

\subsubsection{Test Items}
Demonstrate that production systems monitor and report faults. ~Where
possible fault mitigation can include re-start, re-submission, or return
of partial products for triage.


\subsubsection{Predecessors}

\subsubsection{Environment Needs}

\paragraph{Software}

\paragraph{Hardware}

\subsubsection{Input Specification}

\subsubsection{Output Specification}

\subsubsection{Test Procedure}
    \begin{longtable}[]{p{1.3cm}p{2cm}p{13cm}}
    %\toprule
    Step & \multicolumn{2}{@{}l}{Description, Input Data and Expected Result} \\ \toprule
    \endhead

            \multirow{3}{*}{ 1 } & Description &
            \begin{minipage}[t]{13cm}{\footnotesize
            Analyze design; execute single-day operations rehearsal including
failures, observe automated recovery and continuation of processing

            \vspace{\dp0}
            } \end{minipage} \\ \cline{2-3}
            & Test Data &
            \begin{minipage}[t]{13cm}{\footnotesize
                No data.
                \vspace{\dp0}
            } \end{minipage} \\ \cline{2-3}
            & Expected Result &
        \\ \midrule
    \end{longtable}

\subsection{LVV-T179 - Verify implementation of Compute Platform Heterogeneity}\label{lvv-t179}

\begin{longtable}[]{llllll}
\toprule
Version & Status & Priority & Verification Type & Owner
\\\midrule
1 & Draft & Normal &
Test & Robert Gruendl
\\\bottomrule
\multicolumn{6}{c}{ Open \href{https://jira.lsstcorp.org/secure/Tests.jspa\#/testCase/LVV-T179}{LVV-T179} in Jira } \\
\end{longtable}

\subsubsection{Verification Elements}
\begin{itemize}
\item \href{https://jira.lsstcorp.org/browse/LVV-145}{LVV-145} - DMS-REQ-0314-V-01: Compute Platform Heterogeneity

\end{itemize}

\subsubsection{Test Items}
Demonstrate that production results are the same (within machine
accuracy) when production occurs on different platforms (OS, kernel,
hardware provisioning).


\subsubsection{Predecessors}

\subsubsection{Environment Needs}

\paragraph{Software}

\paragraph{Hardware}

\subsubsection{Input Specification}

\subsubsection{Output Specification}

\subsubsection{Test Procedure}
    \begin{longtable}[]{p{1.3cm}p{2cm}p{13cm}}
    %\toprule
    Step & \multicolumn{2}{@{}l}{Description, Input Data and Expected Result} \\ \toprule
    \endhead

            \multirow{3}{*}{ 1 } & Description &
            \begin{minipage}[t]{13cm}{\footnotesize
            Configure heterogeneous cluster, execute AP+DRP+LSP, observe correct
functioning

            \vspace{\dp0}
            } \end{minipage} \\ \cline{2-3}
            & Test Data &
            \begin{minipage}[t]{13cm}{\footnotesize
                No data.
                \vspace{\dp0}
            } \end{minipage} \\ \cline{2-3}
            & Expected Result &
        \\ \midrule
    \end{longtable}

\subsection{LVV-T180 - Verify implementation of Data Management Unscheduled Downtime}\label{lvv-t180}

\begin{longtable}[]{llllll}
\toprule
Version & Status & Priority & Verification Type & Owner
\\\midrule
1 & Draft & Normal &
Test & Robert Gruendl
\\\bottomrule
\multicolumn{6}{c}{ Open \href{https://jira.lsstcorp.org/secure/Tests.jspa\#/testCase/LVV-T180}{LVV-T180} in Jira } \\
\end{longtable}

\subsubsection{Verification Elements}
\begin{itemize}
\item \href{https://jira.lsstcorp.org/browse/LVV-149}{LVV-149} - DMS-REQ-0318-V-01: Data Management Unscheduled Downtime

\end{itemize}

\subsubsection{Test Items}
This applies only to downtime that would prevent the collection of
survey data. ~Verification means that analysis has occurred to identify
likely hardware failures that would prevent survey operations and that
mitigations that minimize the downtime to less than DMDowntime (1
day/year) are in place. ~Known systems that fall in this category
include: Image and EFD Archiving, Observatory Operations Data, Telemetry
Gateway, Data Backbone, Managed Database, Inter-Site Networks, and
Service Management and Monitoring.~


\subsubsection{Predecessors}

\subsubsection{Environment Needs}

\paragraph{Software}

\paragraph{Hardware}

\subsubsection{Input Specification}

\subsubsection{Output Specification}

\subsubsection{Test Procedure}
    \begin{longtable}[]{p{1.3cm}p{2cm}p{13cm}}
    %\toprule
    Step & \multicolumn{2}{@{}l}{Description, Input Data and Expected Result} \\ \toprule
    \endhead

            \multirow{3}{*}{ 1 } & Description &
            \begin{minipage}[t]{13cm}{\footnotesize
            Analyze likely hardware failures with mitigations to compute estimated
unplanned downtime

            \vspace{\dp0}
            } \end{minipage} \\ \cline{2-3}
            & Test Data &
            \begin{minipage}[t]{13cm}{\footnotesize
                No data.
                \vspace{\dp0}
            } \end{minipage} \\ \cline{2-3}
            & Expected Result &
        \\ \midrule
    \end{longtable}

\subsection{LVV-T181 - Verify Base Voice Over IP (VOIP)}\label{lvv-t181}

\begin{longtable}[]{llllll}
\toprule
Version & Status & Priority & Verification Type & Owner
\\\midrule
1 & Draft & Normal &
Test & Jeff Kantor
\\\bottomrule
\multicolumn{6}{c}{ Open \href{https://jira.lsstcorp.org/secure/Tests.jspa\#/testCase/LVV-T181}{LVV-T181} in Jira } \\
\end{longtable}

\subsubsection{Verification Elements}
\begin{itemize}
\item \href{https://jira.lsstcorp.org/browse/LVV-18491}{LVV-18491} - DMS-REQ-0352-V-02: Base Voice Over IP (VOIP)

\end{itemize}

\subsubsection{Test Items}
Verify as-built VOIP at the Base Facility is operational and performs as
expected (i.e. sufficient number of extensions allocated properly, no
frequent drop-outs, no frequent jaggies on video, etc.) on both voice
calls and videoconferening.


\subsubsection{Predecessors}
PMCS DLP-465 Complete\\
PMCS IT-702 Complete

\subsubsection{Environment Needs}

\paragraph{Software}
See pre-conditions.

\paragraph{Hardware}
See pre-conditions.

\subsubsection{Input Specification}
\begin{enumerate}
\tightlist
\item
  Base VOIP is installed/configured and Test Personnel have phone sets.
  ~Base Videoconference system is installed/configured. ~Summit,
  Headquarters, and/or LDF Videconference system is
  installed/configured.
\item
  As-built documentation for all of the above is available.
\end{enumerate}

\subsubsection{Output Specification}

\subsubsection{Test Procedure}
    \begin{longtable}[]{p{1.3cm}p{2cm}p{13cm}}
    %\toprule
    Step & \multicolumn{2}{@{}l}{Description, Input Data and Expected Result} \\ \toprule
    \endhead

            \multirow{3}{*}{ 1 } & Description &
            \begin{minipage}[t]{13cm}{\footnotesize
            Test voice calls over VOIP system from Base Facility to locations in
~Base and to other Rubin Observatory facilities.

            \vspace{\dp0}
            } \end{minipage} \\ \cline{2-3}
            & Test Data &
            \begin{minipage}[t]{13cm}{\footnotesize
                No data.
                \vspace{\dp0}
            } \end{minipage} \\ \cline{2-3}
            & Expected Result &
                \begin{minipage}[t]{13cm}{\footnotesize
                As-built VOIP at the Base Facility is operational and performs as
expected (i.e. sufficient number of extensions allocated properly, no
frequent drop-outs, etc.).

                \vspace{\dp0}
                } \end{minipage}
        \\ \midrule

            \multirow{3}{*}{ 2 } & Description &
            \begin{minipage}[t]{13cm}{\footnotesize
            Test video conferences over ~system from Base Facility to locations in
Base and to other Rubin Observatory facilities.

            \vspace{\dp0}
            } \end{minipage} \\ \cline{2-3}
            & Test Data &
            \begin{minipage}[t]{13cm}{\footnotesize
                No data.
                \vspace{\dp0}
            } \end{minipage} \\ \cline{2-3}
            & Expected Result &
                \begin{minipage}[t]{13cm}{\footnotesize
                Verify (a) plannned and (b) as-built VOIP at the Base Facility is
operational and performs as expected (i.e. no frequent drop-outs, no
frequent audio glitches, no frequent jaggies on video, etc.).

                \vspace{\dp0}
                } \end{minipage}
        \\ \midrule
    \end{longtable}

\subsection{LVV-T182 - Verify implementation of Prefer Computing and Storage Down}\label{lvv-t182}

\begin{longtable}[]{llllll}
\toprule
Version & Status & Priority & Verification Type & Owner
\\\midrule
1 & Draft & Normal &
Test & Robert Gruendl
\\\bottomrule
\multicolumn{6}{c}{ Open \href{https://jira.lsstcorp.org/secure/Tests.jspa\#/testCase/LVV-T182}{LVV-T182} in Jira } \\
\end{longtable}

\subsubsection{Verification Elements}
\begin{itemize}
\item \href{https://jira.lsstcorp.org/browse/LVV-72}{LVV-72} - DMS-REQ-0170-V-01: Prefer Computing and Storage Down

\end{itemize}

\subsubsection{Test Items}
Only build compute or storage facilities at the summit that are
justified by operational need or to prevent loss of data during
networking downtimes.


\subsubsection{Predecessors}

\subsubsection{Environment Needs}

\paragraph{Software}

\paragraph{Hardware}

\subsubsection{Input Specification}

\subsubsection{Output Specification}

\subsubsection{Test Procedure}
    \begin{longtable}[]{p{1.3cm}p{2cm}p{13cm}}
    %\toprule
    Step & \multicolumn{2}{@{}l}{Description, Input Data and Expected Result} \\ \toprule
    \endhead

            \multirow{3}{*}{ 1 } & Description &
            \begin{minipage}[t]{13cm}{\footnotesize
            Analyze design

            \vspace{\dp0}
            } \end{minipage} \\ \cline{2-3}
            & Test Data &
            \begin{minipage}[t]{13cm}{\footnotesize
                No data.
                \vspace{\dp0}
            } \end{minipage} \\ \cline{2-3}
            & Expected Result &
        \\ \midrule
    \end{longtable}

\subsection{LVV-T183 - Verify implementation of DMS Communication with OCS}\label{lvv-t183}

\begin{longtable}[]{llllll}
\toprule
Version & Status & Priority & Verification Type & Owner
\\\midrule
1 & Defined & Normal &
Test & Gregory Dubois-Felsmann
\\\bottomrule
\multicolumn{6}{c}{ Open \href{https://jira.lsstcorp.org/secure/Tests.jspa\#/testCase/LVV-T183}{LVV-T183} in Jira } \\
\end{longtable}

\subsubsection{Verification Elements}
\begin{itemize}
\item \href{https://jira.lsstcorp.org/browse/LVV-146}{LVV-146} - DMS-REQ-0315-V-01: DMS Communication with OCS

\end{itemize}

\subsubsection{Test Items}
Verify that the DMS at the Base Facility can receive commands from the
OCS and send command responses, events, and telemetry back. ~Verified by
Early Integration activities and during AuxTel commissioning.


\subsubsection{Predecessors}

\subsubsection{Environment Needs}

\paragraph{Software}

\paragraph{Hardware}

\subsubsection{Input Specification}

\subsubsection{Output Specification}

\subsubsection{Test Procedure}
    \begin{longtable}[]{p{1.3cm}p{2cm}p{13cm}}
    %\toprule
    Step & \multicolumn{2}{@{}l}{Description, Input Data and Expected Result} \\ \toprule
    \endhead

            \multirow{3}{*}{ 1 } & Description &
            \begin{minipage}[t]{13cm}{\footnotesize
            From the Base Site, connect to the (simulated) OCS telemetry stream.

            \vspace{\dp0}
            } \end{minipage} \\ \cline{2-3}
            & Test Data &
            \begin{minipage}[t]{13cm}{\footnotesize
                No data.
                \vspace{\dp0}
            } \end{minipage} \\ \cline{2-3}
            & Expected Result &
        \\ \midrule

            \multirow{3}{*}{ 2 } & Description &
            \begin{minipage}[t]{13cm}{\footnotesize
            Send a command to the OCS, and observe that the command has been
executed.

            \vspace{\dp0}
            } \end{minipage} \\ \cline{2-3}
            & Test Data &
            \begin{minipage}[t]{13cm}{\footnotesize
                No data.
                \vspace{\dp0}
            } \end{minipage} \\ \cline{2-3}
            & Expected Result &
                \begin{minipage}[t]{13cm}{\footnotesize
                Confirmation that the OCS command successfully executed.

                \vspace{\dp0}
                } \end{minipage}
        \\ \midrule

            \multirow{3}{*}{ 3 } & Description &
            \begin{minipage}[t]{13cm}{\footnotesize
            Extract information from the telemetry being broadcast by the OCS, and
ensure that these data are readable.

            \vspace{\dp0}
            } \end{minipage} \\ \cline{2-3}
            & Test Data &
            \begin{minipage}[t]{13cm}{\footnotesize
                No data.
                \vspace{\dp0}
            } \end{minipage} \\ \cline{2-3}
            & Expected Result &
                \begin{minipage}[t]{13cm}{\footnotesize
                A readable extract from the OCS telemetry stream.

                \vspace{\dp0}
                } \end{minipage}
        \\ \midrule
    \end{longtable}

\subsection{LVV-T185 - Verify implementation of Summit to Base Network Availability}\label{lvv-t185}

\begin{longtable}[]{llllll}
\toprule
Version & Status & Priority & Verification Type & Owner
\\\midrule
1 & Draft & Normal &
Inspection & Jeff Kantor
\\\bottomrule
\multicolumn{6}{c}{ Open \href{https://jira.lsstcorp.org/secure/Tests.jspa\#/testCase/LVV-T185}{LVV-T185} in Jira } \\
\end{longtable}

\subsubsection{Verification Elements}
\begin{itemize}
\item \href{https://jira.lsstcorp.org/browse/LVV-74}{LVV-74} - DMS-REQ-0172-V-01: Summit to Base Network Availability

\end{itemize}

\subsubsection{Test Items}
Verify the availability of Summit to Base Network by demonstrating that
the mean time between failures is less than summToBaseNetMTBF (90 days)
over 1 year.


\subsubsection{Predecessors}
See pre-conditions.

\subsubsection{Environment Needs}

\paragraph{Software}
See pre-conditions.

\paragraph{Hardware}
See pre-conditions.

\subsubsection{Input Specification}
\begin{enumerate}
\tightlist
\item
  PMCS DMTC-7400-2400 Complete.
\item
  6 months of historical availability data for this link is available.
\item
  perSonar installed in Summit and publishing statistics to MadDash.
\item
  As-built documentation for all of the above is available.
\end{enumerate}

NOE: After the initial test, the corresponding verification elements
will be flagged as ``Requires Monitoring'' such that those requirements
will be closed out as having been verified but will continue to be
monitored throughout commissioning to ensure they do not drop out of
compliance. This will also be monitored for end to end Summit - Data
Facility transfers during Commissioning.

\subsubsection{Output Specification}

\subsubsection{Test Procedure}
    \begin{longtable}[]{p{1.3cm}p{2cm}p{13cm}}
    %\toprule
    Step & \multicolumn{2}{@{}l}{Description, Input Data and Expected Result} \\ \toprule
    \endhead

            \multirow{3}{*}{ 1 } & Description &
            \begin{minipage}[t]{13cm}{\footnotesize
            Monitor summit to base networking for at least 1 week

            \vspace{\dp0}
            } \end{minipage} \\ \cline{2-3}
            & Test Data &
            \begin{minipage}[t]{13cm}{\footnotesize
                LATISS, ComCAM, and/or Full Camera data.

                \vspace{\dp0}
            } \end{minipage} \\ \cline{2-3}
            & Expected Result &
                \begin{minipage}[t]{13cm}{\footnotesize
                Summit - base network is operational for 1 week and monitoring data is
collected.

                \vspace{\dp0}
                } \end{minipage}
        \\ \midrule

            \multirow{3}{*}{ 2 } & Description &
            \begin{minipage}[t]{13cm}{\footnotesize
            Extrapolate annual availability, compare with at least 6 months of
historical data on the link.

            \vspace{\dp0}
            } \end{minipage} \\ \cline{2-3}
            & Test Data &
            \begin{minipage}[t]{13cm}{\footnotesize
                Historical and current logs

                \vspace{\dp0}
            } \end{minipage} \\ \cline{2-3}
            & Expected Result &
                \begin{minipage}[t]{13cm}{\footnotesize
                The mean time between failures (MTBF) is projected to be less than
summToBaseNetMTBF (90 days) over 1 year.

                \vspace{\dp0}
                } \end{minipage}
        \\ \midrule
    \end{longtable}

\subsection{LVV-T186 - Verify implementation of Summit to Base Network Reliability}\label{lvv-t186}

\begin{longtable}[]{llllll}
\toprule
Version & Status & Priority & Verification Type & Owner
\\\midrule
1 & Draft & Normal &
Demonstration & Jeff Kantor
\\\bottomrule
\multicolumn{6}{c}{ Open \href{https://jira.lsstcorp.org/secure/Tests.jspa\#/testCase/LVV-T186}{LVV-T186} in Jira } \\
\end{longtable}

\subsubsection{Verification Elements}
\begin{itemize}
\item \href{https://jira.lsstcorp.org/browse/LVV-75}{LVV-75} - DMS-REQ-0173-V-01: Summit to Base Network Reliability

\end{itemize}

\subsubsection{Test Items}
Verify the reliability of the summit to base network by demonstrating
reconnection and recovery to transfer of data at or exceeding rates
specified in \citeds{LDM-142} following a cut in network connection, within MTTR
specification. The network operator will provide MTTR data on links
during commissioning and operations.\\[2\baselineskip]


\subsubsection{Predecessors}
See pre-conditions.

\subsubsection{Environment Needs}

\paragraph{Software}
See pre-conditions.

\paragraph{Hardware}
See pre-conditions.

\subsubsection{Input Specification}
\begin{enumerate}
\tightlist
\item
  PMCS DMTC-7400-2400 Complete
\item
  As-built documentation for Summit - Base Network is available.
\end{enumerate}

NOTE: After the initial test, the corresponding verification elements
will be flagged as ``Requires Monitoring'' such that those requirements
will be closed out as having been verified but will continue to be
monitored throughout commissioning to ensure they do not drop out of
compliance. This will also be monitored for end to end Summit - Data
Facility transfers during Commissioning.

\subsubsection{Output Specification}

\subsubsection{Test Procedure}
    \begin{longtable}[]{p{1.3cm}p{2cm}p{13cm}}
    %\toprule
    Step & \multicolumn{2}{@{}l}{Description, Input Data and Expected Result} \\ \toprule
    \endhead

            \multirow{3}{*}{ 1 } & Description &
            \begin{minipage}[t]{13cm}{\footnotesize
            Disconnect fiber cable at an endpoint location on the base side of the
Summit - Base fiber.

            \vspace{\dp0}
            } \end{minipage} \\ \cline{2-3}
            & Test Data &
            \begin{minipage}[t]{13cm}{\footnotesize
                LATISS, ComCAM, or FullCam data

                \vspace{\dp0}
            } \end{minipage} \\ \cline{2-3}
            & Expected Result &
                \begin{minipage}[t]{13cm}{\footnotesize
                Fiber is disconnected and the fault is detected by the network
monitoring system.

                \vspace{\dp0}
                } \end{minipage}
        \\ \midrule

            \multirow{3}{*}{ 2 } & Description &
            \begin{minipage}[t]{13cm}{\footnotesize
            Measure the cable with the OTDR to locate the distance from the end
point. Diagnose that it is a break.

            \vspace{\dp0}
            } \end{minipage} \\ \cline{2-3}
            & Test Data &
            \begin{minipage}[t]{13cm}{\footnotesize
                NA

                \vspace{\dp0}
            } \end{minipage} \\ \cline{2-3}
            & Expected Result &
                \begin{minipage}[t]{13cm}{\footnotesize
                OTDR shows the fiber is disconnected (break).

                \vspace{\dp0}
                } \end{minipage}
        \\ \midrule

            \multirow{3}{*}{ 3 } & Description &
            \begin{minipage}[t]{13cm}{\footnotesize
            Elapse time to simulate the following:

\begin{itemize}
\tightlist
\item
  Go to the most inaccessible place which would mean carrying all the
  tools/splicer/generator/tent equipment some ~metres.
\item
  Erect a tent to make the splice
\item
  Start the generator
\item
  Do a splice on some random piece of cable
\item
  At an end point measure the cable again to ensure it is break free.
\item
  Take down and reinstall an isolated pole (not in the actual fiber
  path)
\item
  Put the cable on the pole.
\end{itemize}

            \vspace{\dp0}
            } \end{minipage} \\ \cline{2-3}
            & Test Data &
            \begin{minipage}[t]{13cm}{\footnotesize
                NA

                \vspace{\dp0}
            } \end{minipage} \\ \cline{2-3}
            & Expected Result &
                \begin{minipage}[t]{13cm}{\footnotesize
                Wall clock advances by 24 hours.

                \vspace{\dp0}
                } \end{minipage}
        \\ \midrule

            \multirow{3}{*}{ 4 } & Description &
            \begin{minipage}[t]{13cm}{\footnotesize
            Clean fiber connections. ~Restore connection (e.g. reconnect cable).
~Cycle equipment as necessary to confirm fiber is connected.

            \vspace{\dp0}
            } \end{minipage} \\ \cline{2-3}
            & Test Data &
            \begin{minipage}[t]{13cm}{\footnotesize
                NA

                \vspace{\dp0}
            } \end{minipage} \\ \cline{2-3}
            & Expected Result &
                \begin{minipage}[t]{13cm}{\footnotesize
                Network recovers and resumes sending data.

                \vspace{\dp0}
                } \end{minipage}
        \\ \midrule

            \multirow{3}{*}{ 5 } & Description &
            \begin{minipage}[t]{13cm}{\footnotesize
            Measure with OTDR to ensure back to normal state.

            \vspace{\dp0}
            } \end{minipage} \\ \cline{2-3}
            & Test Data &
            \begin{minipage}[t]{13cm}{\footnotesize
                NA

                \vspace{\dp0}
            } \end{minipage} \\ \cline{2-3}
            & Expected Result &
                \begin{minipage}[t]{13cm}{\footnotesize
                OTDR indicates normal state.

                \vspace{\dp0}
                } \end{minipage}
        \\ \midrule
    \end{longtable}

\subsection{LVV-T187 - Verify implementation of Summit to Base Network Secondary Link}\label{lvv-t187}

\begin{longtable}[]{llllll}
\toprule
Version & Status & Priority & Verification Type & Owner
\\\midrule
1 & Draft & Normal &
Test & Jeff Kantor
\\\bottomrule
\multicolumn{6}{c}{ Open \href{https://jira.lsstcorp.org/secure/Tests.jspa\#/testCase/LVV-T187}{LVV-T187} in Jira } \\
\end{longtable}

\subsubsection{Verification Elements}
\begin{itemize}
\item \href{https://jira.lsstcorp.org/browse/LVV-76}{LVV-76} - DMS-REQ-0174-V-01: Summit to Base Network Secondary Link

\end{itemize}

\subsubsection{Test Items}
Verify automated fail-over from primary to secondary equipment in Rubin
Observatory DWDM on simulated failure of primary. ~Verify bandwidth
sufficiency on secondary. ~Verify automated recovery to primary
equipment on simulated restoration of primary. ~Repeat for failure of
Rubin Observatory fiber and fail-over to AURA fiber and DWDM.
~Demonstrate use of secondary in ``catch-up'' mode.


\subsubsection{Predecessors}
See pre-conditions.

\subsubsection{Environment Needs}

\paragraph{Software}
See pre-conditions.

\paragraph{Hardware}
See pre-conditions.

\subsubsection{Input Specification}
\begin{enumerate}
\tightlist
\item
  PMCS DMTC-7400-2400 complete.
\item
  As-built documentation for Summit - Base Network is available.
\end{enumerate}

NOTE: After the initial test, the corresponding verification elements
will be flagged as ``Requires Monitoring'' such that those requirements
will be closed out as having been verified but will continue to be
monitored throughout commissioning to ensure they do not drop out of
compliance. This will also be monitored for end to end Summit - Data
Facility transfers during Commissioning.

\subsubsection{Output Specification}

\subsubsection{Test Procedure}
    \begin{longtable}[]{p{1.3cm}p{2cm}p{13cm}}
    %\toprule
    Step & \multicolumn{2}{@{}l}{Description, Input Data and Expected Result} \\ \toprule
    \endhead

            \multirow{3}{*}{ 1 } & Description &
            \begin{minipage}[t]{13cm}{\footnotesize
            Transfer data between summit and base on primary equipment (LSST Summit
- Base) over uninterrupted 1 day period. ~

            \vspace{\dp0}
            } \end{minipage} \\ \cline{2-3}
            & Test Data &
            \begin{minipage}[t]{13cm}{\footnotesize
                LATISS, ComCAM, or FullCAM data.

                \vspace{\dp0}
            } \end{minipage} \\ \cline{2-3}
            & Expected Result &
                \begin{minipage}[t]{13cm}{\footnotesize
                Normal operations.

                \vspace{\dp0}
                } \end{minipage}
        \\ \midrule

            \multirow{3}{*}{ 2 } & Description &
            \begin{minipage}[t]{13cm}{\footnotesize
            Simulate equipment outage by disconnecting power card from primary DWDM
equipment on base side of Summit - Base Fiber.

            \vspace{\dp0}
            } \end{minipage} \\ \cline{2-3}
            & Test Data &
            \begin{minipage}[t]{13cm}{\footnotesize
                NA

                \vspace{\dp0}
            } \end{minipage} \\ \cline{2-3}
            & Expected Result &
                \begin{minipage}[t]{13cm}{\footnotesize
                Network fails over to secondary equipment in \textless{}=60s.

                \vspace{\dp0}
                } \end{minipage}
        \\ \midrule

            \multirow{3}{*}{ 3 } & Description &
            \begin{minipage}[t]{13cm}{\footnotesize
            Transfer data between summit and base over secondary equipment
uninterrupted 1 day period while monitoring network.

            \vspace{\dp0}
            } \end{minipage} \\ \cline{2-3}
            & Test Data &
            \begin{minipage}[t]{13cm}{\footnotesize
                NA

                \vspace{\dp0}
            } \end{minipage} \\ \cline{2-3}
            & Expected Result &
                \begin{minipage}[t]{13cm}{\footnotesize
                Verify that secondary equipment is capable of transferring 1 night of
raw data (nCalibExpDay + nRawExpNightMax = 450 + 2800 = ~3250 exposures)
within summToBaseNet2TransMax (72 hours), i.e. at or exceeding rates
specified in LDM-142.

                \vspace{\dp0}
                } \end{minipage}
        \\ \midrule

            \multirow{3}{*}{ 4 } & Description &
            \begin{minipage}[t]{13cm}{\footnotesize
            Restore ~primary equipment (i.e. reconnect power card to primary
equipment.)

            \vspace{\dp0}
            } \end{minipage} \\ \cline{2-3}
            & Test Data &
            \begin{minipage}[t]{13cm}{\footnotesize
                NA

                \vspace{\dp0}
            } \end{minipage} \\ \cline{2-3}
            & Expected Result &
                \begin{minipage}[t]{13cm}{\footnotesize
                Network recovers to primary in \textless{}= 60s.

                \vspace{\dp0}
                } \end{minipage}
        \\ \midrule

            \multirow{3}{*}{ 5 } & Description &
            \begin{minipage}[t]{13cm}{\footnotesize
            Simulate fiber outage by disconnecting fiber from primary DWDM equipment
on base side of Summit - Base Fiber.

            \vspace{\dp0}
            } \end{minipage} \\ \cline{2-3}
            & Test Data &
            \begin{minipage}[t]{13cm}{\footnotesize
                NA

                \vspace{\dp0}
            } \end{minipage} \\ \cline{2-3}
            & Expected Result &
                \begin{minipage}[t]{13cm}{\footnotesize
                Network fails over to AURA DWDM and fiber.

                \vspace{\dp0}
                } \end{minipage}
        \\ \midrule

            \multirow{3}{*}{ 6 } & Description &
            \begin{minipage}[t]{13cm}{\footnotesize
            Transfer data between summit and base over AURA fiber and equipment
uninterrupted 1 day period while monitoring network.

            \vspace{\dp0}
            } \end{minipage} \\ \cline{2-3}
            & Test Data &
            \begin{minipage}[t]{13cm}{\footnotesize
                LATISS, ComCAM, or FullCAM data.

                \vspace{\dp0}
            } \end{minipage} \\ \cline{2-3}
            & Expected Result &
                \begin{minipage}[t]{13cm}{\footnotesize
                Verify that AURA fiber and equipment is capable of transferring 1 night
of raw data (nCalibExpDay + nRawExpNightMax = 450 + 2800 = ~3250
exposures) within summToBaseNet2TransMax (72 hours), i.e. at or
exceeding rates specified in LDM-142.

                \vspace{\dp0}
                } \end{minipage}
        \\ \midrule

            \multirow{3}{*}{ 7 } & Description &
            \begin{minipage}[t]{13cm}{\footnotesize
            Restore ~primary fiber (i.e. reconnect fiber to Rubin Observatory DWDM
equipment.)

            \vspace{\dp0}
            } \end{minipage} \\ \cline{2-3}
            & Test Data &
            \begin{minipage}[t]{13cm}{\footnotesize
                No data.
                \vspace{\dp0}
            } \end{minipage} \\ \cline{2-3}
            & Expected Result &
                \begin{minipage}[t]{13cm}{\footnotesize
                Network recovers to Rubin Observatory fiber and DWDM.

                \vspace{\dp0}
                } \end{minipage}
        \\ \midrule

            \multirow{3}{*}{ 8 } & Description &
            \begin{minipage}[t]{13cm}{\footnotesize
            Demonstrate use of secondary in ``catch-up'' mode.

            \vspace{\dp0}
            } \end{minipage} \\ \cline{2-3}
            & Test Data &
            \begin{minipage}[t]{13cm}{\footnotesize
                DAQ data buffer full of images and associated meta-data

                \vspace{\dp0}
            } \end{minipage} \\ \cline{2-3}
            & Expected Result &
                \begin{minipage}[t]{13cm}{\footnotesize
                Images from DAQ buffer and associated metadata are retrievable over
secondary path while current observing data is being transferred over
primary path.

                \vspace{\dp0}
                } \end{minipage}
        \\ \midrule
    \end{longtable}

\subsection{LVV-T188 - Verify implementation of Summit to Base Network Ownership and Operation}\label{lvv-t188}

\begin{longtable}[]{llllll}
\toprule
Version & Status & Priority & Verification Type & Owner
\\\midrule
1 & Draft & Normal &
Inspection & Jeff Kantor
\\\bottomrule
\multicolumn{6}{c}{ Open \href{https://jira.lsstcorp.org/secure/Tests.jspa\#/testCase/LVV-T188}{LVV-T188} in Jira } \\
\end{longtable}

\subsubsection{Verification Elements}
\begin{itemize}
\item \href{https://jira.lsstcorp.org/browse/LVV-77}{LVV-77} - DMS-REQ-0175-V-01: Summit to Base Network Ownership and Operation

\end{itemize}

\subsubsection{Test Items}
Verify Summit to Base Network Ownership and Operation by LSST and/or the
operations entity by inspection of construction and operations contracts
and Indefeasible Rights.


\subsubsection{Predecessors}
PMCS DMTC-7400-2140, -2240, -2330 Complete

\subsubsection{Environment Needs}

\paragraph{Software}
None

\paragraph{Hardware}
None

\subsubsection{Input Specification}
\begin{enumerate}
\tightlist
\item
  As-built documentation for all of the above contracts and IRUs is
  available.
\end{enumerate}

\subsubsection{Output Specification}

\subsubsection{Test Procedure}
    \begin{longtable}[]{p{1.3cm}p{2cm}p{13cm}}
    %\toprule
    Step & \multicolumn{2}{@{}l}{Description, Input Data and Expected Result} \\ \toprule
    \endhead

            \multirow{3}{*}{ 1 } & Description &
            \begin{minipage}[t]{13cm}{\footnotesize
            Examine contracts with REUNA and telefonica for fiber ownership and
maintenance terms.

            \vspace{\dp0}
            } \end{minipage} \\ \cline{2-3}
            & Test Data &
            \begin{minipage}[t]{13cm}{\footnotesize
                No data.
                \vspace{\dp0}
            } \end{minipage} \\ \cline{2-3}
            & Expected Result &
                \begin{minipage}[t]{13cm}{\footnotesize
                Rubin Observatory is owner of fibers on AURA property and Summit - Base
DWDM ~and has 15-year IRU for use of fibers on all segments. ~REUNA is
owner of LS - SCL DWDM on AURA property and in Santiago, and is operator
on all fibers and DWDM. ~Telefonica is contracted to maintain fibers not
on AURA property.

                \vspace{\dp0}
                } \end{minipage}
        \\ \midrule
    \end{longtable}

\subsection{LVV-T189 - Verify implementation of Base Facility Infrastructure}\label{lvv-t189}

\begin{longtable}[]{llllll}
\toprule
Version & Status & Priority & Verification Type & Owner
\\\midrule
1 & Draft & Normal &
Test & Robert Gruendl
\\\bottomrule
\multicolumn{6}{c}{ Open \href{https://jira.lsstcorp.org/secure/Tests.jspa\#/testCase/LVV-T189}{LVV-T189} in Jira } \\
\end{longtable}

\subsubsection{Verification Elements}
\begin{itemize}
\item \href{https://jira.lsstcorp.org/browse/LVV-78}{LVV-78} - DMS-REQ-0176-V-01: Base Facility Infrastructure

\end{itemize}

\subsubsection{Test Items}
Verify that the (a) planned infrastructure and (b) as-built
infrastructure for the Base Facility satisfies the needs for data
transfer and buffering, a copy of the Archive Facility, and support for
Commissioning.


\subsubsection{Predecessors}

\subsubsection{Environment Needs}

\paragraph{Software}

\paragraph{Hardware}

\subsubsection{Input Specification}

\subsubsection{Output Specification}

\subsubsection{Test Procedure}
    \begin{longtable}[]{p{1.3cm}p{2cm}p{13cm}}
    %\toprule
    Step & \multicolumn{2}{@{}l}{Description, Input Data and Expected Result} \\ \toprule
    \endhead

            \multirow{3}{*}{ 1 } & Description &
            \begin{minipage}[t]{13cm}{\footnotesize
            Analyze design and sizing model

            \vspace{\dp0}
            } \end{minipage} \\ \cline{2-3}
            & Test Data &
            \begin{minipage}[t]{13cm}{\footnotesize
                No data.
                \vspace{\dp0}
            } \end{minipage} \\ \cline{2-3}
            & Expected Result &
        \\ \midrule
    \end{longtable}

\subsection{LVV-T190 - Verify implementation of Base Facility Co-Location with Existing
Facility}\label{lvv-t190}

\begin{longtable}[]{llllll}
\toprule
Version & Status & Priority & Verification Type & Owner
\\\midrule
1 & Draft & Normal &
Test & Robert Gruendl
\\\bottomrule
\multicolumn{6}{c}{ Open \href{https://jira.lsstcorp.org/secure/Tests.jspa\#/testCase/LVV-T190}{LVV-T190} in Jira } \\
\end{longtable}

\subsubsection{Verification Elements}
\begin{itemize}
\item \href{https://jira.lsstcorp.org/browse/LVV-80}{LVV-80} - DMS-REQ-0178-V-01: Base Facility Co-Location with Existing Facility

\end{itemize}

\subsubsection{Test Items}
Verify that the Base Facility is located at an existing known supported
facility.


\subsubsection{Predecessors}

\subsubsection{Environment Needs}

\paragraph{Software}

\paragraph{Hardware}

\subsubsection{Input Specification}

\subsubsection{Output Specification}

\subsubsection{Test Procedure}
    \begin{longtable}[]{p{1.3cm}p{2cm}p{13cm}}
    %\toprule
    Step & \multicolumn{2}{@{}l}{Description, Input Data and Expected Result} \\ \toprule
    \endhead

            \multirow{3}{*}{ 1 } & Description &
            \begin{minipage}[t]{13cm}{\footnotesize
            Analyze design

            \vspace{\dp0}
            } \end{minipage} \\ \cline{2-3}
            & Test Data &
            \begin{minipage}[t]{13cm}{\footnotesize
                No data.
                \vspace{\dp0}
            } \end{minipage} \\ \cline{2-3}
            & Expected Result &
        \\ \midrule
    \end{longtable}

\subsection{LVV-T191 - Verify implementation of Commissioning Cluster}\label{lvv-t191}

\begin{longtable}[]{llllll}
\toprule
Version & Status & Priority & Verification Type & Owner
\\\midrule
1 & Draft & Normal &
Test & Robert Gruendl
\\\bottomrule
\multicolumn{6}{c}{ Open \href{https://jira.lsstcorp.org/secure/Tests.jspa\#/testCase/LVV-T191}{LVV-T191} in Jira } \\
\end{longtable}

\subsubsection{Verification Elements}
\begin{itemize}
\item \href{https://jira.lsstcorp.org/browse/LVV-147}{LVV-147} - DMS-REQ-0316-V-01: Commissioning Cluster

\end{itemize}

\subsubsection{Test Items}
Verify that the Commissioning Cluster has sufficient Compute/Storage/LAN
at the Base Facility to support Commissioning.


\subsubsection{Predecessors}

\subsubsection{Environment Needs}

\paragraph{Software}

\paragraph{Hardware}

\subsubsection{Input Specification}

\subsubsection{Output Specification}

\subsubsection{Test Procedure}
    \begin{longtable}[]{p{1.3cm}p{2cm}p{13cm}}
    %\toprule
    Step & \multicolumn{2}{@{}l}{Description, Input Data and Expected Result} \\ \toprule
    \endhead

            \multirow{3}{*}{ 1 } & Description &
            \begin{minipage}[t]{13cm}{\footnotesize
            Analyze design and budget

            \vspace{\dp0}
            } \end{minipage} \\ \cline{2-3}
            & Test Data &
            \begin{minipage}[t]{13cm}{\footnotesize
                No data.
                \vspace{\dp0}
            } \end{minipage} \\ \cline{2-3}
            & Expected Result &
        \\ \midrule
    \end{longtable}

\subsection{LVV-T192 - Verify implementation of Base Wireless LAN (WiFi)}\label{lvv-t192}

\begin{longtable}[]{llllll}
\toprule
Version & Status & Priority & Verification Type & Owner
\\\midrule
1 & Draft & Normal &
Test & Jeff Kantor
\\\bottomrule
\multicolumn{6}{c}{ Open \href{https://jira.lsstcorp.org/secure/Tests.jspa\#/testCase/LVV-T192}{LVV-T192} in Jira } \\
\end{longtable}

\subsubsection{Verification Elements}
\begin{itemize}
\item \href{https://jira.lsstcorp.org/browse/LVV-183}{LVV-183} - DMS-REQ-0352-V-01: Base Wireless LAN (WiFi)

\end{itemize}

\subsubsection{Test Items}
Verify as-built wireless network at the Base Facility supports
minBaseWiFi bandwidth (1000 Mbs).


\subsubsection{Predecessors}
PMCS DLP-465 Complete.

\subsubsection{Environment Needs}

\paragraph{Software}
See pre-conditions.

\paragraph{Hardware}
Desktop with WiFi NIC, email reader, internet browser.

\subsubsection{Input Specification}
\begin{enumerate}
\tightlist
\item
  Base Wireless LAN is installed/configured and Test Personnel have
  accounts for email, internet access.
\item
  As-built documentation for all of the above is available.
\end{enumerate}

\subsubsection{Output Specification}

\subsubsection{Test Procedure}
    \begin{longtable}[]{p{1.3cm}p{2cm}p{13cm}}
    %\toprule
    Step & \multicolumn{2}{@{}l}{Description, Input Data and Expected Result} \\ \toprule
    \endhead

            \multirow{3}{*}{ 1 } & Description &
            \begin{minipage}[t]{13cm}{\footnotesize
            Test internet web browsing and file download, email at summit and base
over wireless.

            \vspace{\dp0}
            } \end{minipage} \\ \cline{2-3}
            & Test Data &
            \begin{minipage}[t]{13cm}{\footnotesize
                NA

                \vspace{\dp0}
            } \end{minipage} \\ \cline{2-3}
            & Expected Result &
                \begin{minipage}[t]{13cm}{\footnotesize
                Verify as-built wireless network at the Base Facility supports
minBaseWiFi bandwidth (1000 Mbs). Verify wireless signal strength meets
or exceeds typical, and average and peak bandwidths meet or exceed
minBaseWiFI bandwidth.

                \vspace{\dp0}
                } \end{minipage}
        \\ \midrule
    \end{longtable}

\subsection{LVV-T193 - Verify implementation of Base to Archive Network}\label{lvv-t193}

\begin{longtable}[]{llllll}
\toprule
Version & Status & Priority & Verification Type & Owner
\\\midrule
1 & Draft & Normal &
Test & Jeff Kantor
\\\bottomrule
\multicolumn{6}{c}{ Open \href{https://jira.lsstcorp.org/secure/Tests.jspa\#/testCase/LVV-T193}{LVV-T193} in Jira } \\
\end{longtable}

\subsubsection{Verification Elements}
\begin{itemize}
\item \href{https://jira.lsstcorp.org/browse/LVV-81}{LVV-81} - DMS-REQ-0180-V-01: Base to Archive Network

\end{itemize}

\subsubsection{Test Items}
Verify that the data acquired by a DAQ can be transferred within the
required time, i.e. verify that link is capable of transferring image
for prompt processing in oArchiveMaxTransferTime = 5{[}second{]}, i.e.
at or exceeding rates specified in \citeds{LDM-142}.


\subsubsection{Predecessors}
PMCS DM-Net-5 Complete

\subsubsection{Environment Needs}

\paragraph{Software}
See pre-conditions.

\paragraph{Hardware}
See pre-conditions.

\subsubsection{Input Specification}
\begin{enumerate}
\tightlist
\item
  Archiver/Forwarders are configured at Base, connected to REUNA DWDM,
  loaded with simulated or pre-cursor data, running on end node
  computers that are the production hardware or equivalent to it.
\item
  Archiver/Forwarder receivers or other capability is on configured at
  LDF, connected to Base - Archive Network, running on end node
  computers that are the production hardware or equivalent to it.
\item
  As-built documentation for all of the above is available.
\end{enumerate}

NOTE: This test will be repeated at increasing data volumes as
additional observatory capabilities (e.g. ComCAM, FullCam) become
available. ~Final verification will be tested at full operational
volume. After the initial test, the corresponding verification elements
will be flagged as ``Requires Monitoring'' such that those requirements
will be closed out as having been verified but will continue to be
monitored throughout commissioning to ensure they do not drop out of
compliance. This will also be monitored for end to end Summit - Data
Facility transfers during Commissioning.

\subsubsection{Output Specification}

\subsubsection{Test Procedure}
    \begin{longtable}[]{p{1.3cm}p{2cm}p{13cm}}
    %\toprule
    Step & \multicolumn{2}{@{}l}{Description, Input Data and Expected Result} \\ \toprule
    \endhead

            \multirow{3}{*}{ 1 } & Description &
            \begin{minipage}[t]{13cm}{\footnotesize
            Transfer data between base and archive while monitoring the network over
uninterrupted 1 day period (with repeated transfers on normal observing
cadence).

            \vspace{\dp0}
            } \end{minipage} \\ \cline{2-3}
            & Test Data &
            \begin{minipage}[t]{13cm}{\footnotesize
                LATISS, ComCAM, or FullCAM data.

                \vspace{\dp0}
            } \end{minipage} \\ \cline{2-3}
            & Expected Result &
                \begin{minipage}[t]{13cm}{\footnotesize
                Data transfers occur without significant delay or frequent latency
spikes.

                \vspace{\dp0}
                } \end{minipage}
        \\ \midrule

            \multirow{3}{*}{ 2 } & Description &
            \begin{minipage}[t]{13cm}{\footnotesize
            ~Analyze the network logs and monitoring system to determine average and
peak latency and packet loss statistics.

            \vspace{\dp0}
            } \end{minipage} \\ \cline{2-3}
            & Test Data &
            \begin{minipage}[t]{13cm}{\footnotesize
                No data.
                \vspace{\dp0}
            } \end{minipage} \\ \cline{2-3}
            & Expected Result &
                \begin{minipage}[t]{13cm}{\footnotesize
                Data can be transferred within the required time, i.e. verify that link
is capable of transferring image for prompt processing in
oArchiveMaxTransferTime = 5{[}second{]}. Verify transfer of data at or
exceeding rates specified in LDM-142 at least 98\% of the time.

                \vspace{\dp0}
                } \end{minipage}
        \\ \midrule
    \end{longtable}

\subsection{LVV-T194 - Verify implementation of Base to Archive Network Availability}\label{lvv-t194}

\begin{longtable}[]{llllll}
\toprule
Version & Status & Priority & Verification Type & Owner
\\\midrule
1 & Draft & Normal &
Test & Jeff Kantor
\\\bottomrule
\multicolumn{6}{c}{ Open \href{https://jira.lsstcorp.org/secure/Tests.jspa\#/testCase/LVV-T194}{LVV-T194} in Jira } \\
\end{longtable}

\subsubsection{Verification Elements}
\begin{itemize}
\item \href{https://jira.lsstcorp.org/browse/LVV-82}{LVV-82} - DMS-REQ-0181-V-01: Base to Archive Network Availability

\end{itemize}

\subsubsection{Test Items}
Verify the availability of the Base to Archive Network communications by
demonstrating that it meets or exceeds a mean time between failures,
measured over a 1-yr period of MTBF \textgreater{} baseToArchNetMTBF
(180{[}day{]})


\subsubsection{Predecessors}
~PMCS DMTC-7400-2130 Complete

\subsubsection{Environment Needs}

\paragraph{Software}

\paragraph{Hardware}

\subsubsection{Input Specification}
\begin{enumerate}
\tightlist
\item
  Archiver/Forwarders are configured at Base, connected to REUNA DWDM,
  loaded with simulated or pre-cursor data, running on end node
  computers that are the production hardware or equivalent to it.
\item
  Archiver/Forwarder receivers or other capability is on configured at
  LDF, connected to Base - Archive Network, running on end node
  computers that are the production hardware or equivalent to it.
\item
  At least 6 months of historical monitoring data on this link is
  available.
\item
  As-built documentation for all of the above is available.
\end{enumerate}

NOTE: This test will be repeated at increasing data volumes as
additional observatory capabilities (e.g. ComCAM, FullCam) become
available. Final verification will be tested at full operational volume.
After the initial test, the corresponding verification elements will be
flagged as ``Requires Monitoring'' such that those requirements will be
closed out as having been verified but will continue to be monitored
throughout commissioning to ensure they do not drop out of compliance.
This will also be monitored for end to end Summit - Data Facility
transfers during Commissioning.

\subsubsection{Output Specification}

\subsubsection{Test Procedure}
    \begin{longtable}[]{p{1.3cm}p{2cm}p{13cm}}
    %\toprule
    Step & \multicolumn{2}{@{}l}{Description, Input Data and Expected Result} \\ \toprule
    \endhead

            \multirow{3}{*}{ 1 } & Description &
            \begin{minipage}[t]{13cm}{\footnotesize
            Transfer data between base and archive over uninterrupted 1 week period.

            \vspace{\dp0}
            } \end{minipage} \\ \cline{2-3}
            & Test Data &
            \begin{minipage}[t]{13cm}{\footnotesize
                LATISS, ComCAM, or FullCAM data.

                \vspace{\dp0}
            } \end{minipage} \\ \cline{2-3}
            & Expected Result &
                \begin{minipage}[t]{13cm}{\footnotesize
                Data is successfully transferred during the entire week.

                \vspace{\dp0}
                } \end{minipage}
        \\ \midrule

            \multirow{3}{*}{ 2 } & Description &
            \begin{minipage}[t]{13cm}{\footnotesize
            Analyze monitoring/performance data, compare to historical data, and
extrapolate to a full year, average and peak throughput and latency.

            \vspace{\dp0}
            } \end{minipage} \\ \cline{2-3}
            & Test Data &
            \begin{minipage}[t]{13cm}{\footnotesize
                NA

                \vspace{\dp0}
            } \end{minipage} \\ \cline{2-3}
            & Expected Result &
                \begin{minipage}[t]{13cm}{\footnotesize
                Extrapolated network availability meets baseToArchNetMTBF =
180{[}day{]}. ~Note that this is for complete loss of transfer service
(all paths), not a single path failure with successful fail-over.

                \vspace{\dp0}
                } \end{minipage}
        \\ \midrule
    \end{longtable}

\subsection{LVV-T195 - Verify implementation of Base to Archive Network Reliability}\label{lvv-t195}

\begin{longtable}[]{llllll}
\toprule
Version & Status & Priority & Verification Type & Owner
\\\midrule
1 & Draft & Normal &
Test & Jeff Kantor
\\\bottomrule
\multicolumn{6}{c}{ Open \href{https://jira.lsstcorp.org/secure/Tests.jspa\#/testCase/LVV-T195}{LVV-T195} in Jira } \\
\end{longtable}

\subsubsection{Verification Elements}
\begin{itemize}
\item \href{https://jira.lsstcorp.org/browse/LVV-83}{LVV-83} - DMS-REQ-0182-V-01: Base to Archive Network Reliability

\end{itemize}

\subsubsection{Test Items}
Verify Base to Archive Network Reliability by demonstrating that the
network can recover from outages within baseToArchNetMTTR =
48{[}hour{]}.


\subsubsection{Predecessors}
PMCS DM-NET-5 Complete

\subsubsection{Environment Needs}

\paragraph{Software}
See pre-conditions.

\paragraph{Hardware}
See pre-conditions.

\subsubsection{Input Specification}
\begin{enumerate}
\tightlist
\item
  Archiver/Forwarders are configured at Base, connected to REUNA DWDM,
  loaded with simulated or pre-cursor data, running on end node
  computers that are the production hardware or equivalent to it.
\item
  Archiver/Forwarder receivers or other capability is on configured at
  LDF, connected to Base - Archive Network, running on end node
  computers that are the production hardware or equivalent to it.
\item
  At least 6 months of monitoring data for this link is available.
\item
  As-built documentation for all of the above is available.
\end{enumerate}

NOTE: This test will be repeated at increasing data volumes as
additional observatory capabilities (e.g. ComCAM, FullCam) become
available. Final verification will be tested at full operational volume.
After the initial test, the corresponding verification elements will be
flagged as ``Requires Monitoring'' such that those requirements will be
closed out as having been verified but will continue to be monitored
throughout commissioning to ensure they do not drop out of compliance.
This will also be monitored for end to end Summit - Data Facility
transfers during Commissioning.\\[2\baselineskip]

\subsubsection{Output Specification}

\subsubsection{Test Procedure}
    \begin{longtable}[]{p{1.3cm}p{2cm}p{13cm}}
    %\toprule
    Step & \multicolumn{2}{@{}l}{Description, Input Data and Expected Result} \\ \toprule
    \endhead

            \multirow{3}{*}{ 1 } & Description &
            \begin{minipage}[t]{13cm}{\footnotesize
            Disconnect primary fiber on base side of Base - ~Archive network.

            \vspace{\dp0}
            } \end{minipage} \\ \cline{2-3}
            & Test Data &
            \begin{minipage}[t]{13cm}{\footnotesize
                LATISS, ComCAM, or FullCAM data.

                \vspace{\dp0}
            } \end{minipage} \\ \cline{2-3}
            & Expected Result &
                \begin{minipage}[t]{13cm}{\footnotesize
                Network fails over to secondary path.

                \vspace{\dp0}
                } \end{minipage}
        \\ \midrule

            \multirow{3}{*}{ 2 } & Description &
            \begin{minipage}[t]{13cm}{\footnotesize
            Simulate diagnosis and repair by elapsed time.\\[2\baselineskip]

            \vspace{\dp0}
            } \end{minipage} \\ \cline{2-3}
            & Test Data &
            \begin{minipage}[t]{13cm}{\footnotesize
                NA

                \vspace{\dp0}
            } \end{minipage} \\ \cline{2-3}
            & Expected Result &
                \begin{minipage}[t]{13cm}{\footnotesize
                Wall clock advances by 48 hours. ~Data is successfully transferred over
secondary path.

                \vspace{\dp0}
                } \end{minipage}
        \\ \midrule

            \multirow{3}{*}{ 3 } & Description &
            \begin{minipage}[t]{13cm}{\footnotesize
            Reconnect primary fiber on base side of Base - Archive network.

            \vspace{\dp0}
            } \end{minipage} \\ \cline{2-3}
            & Test Data &
            \begin{minipage}[t]{13cm}{\footnotesize
                NA

                \vspace{\dp0}
            } \end{minipage} \\ \cline{2-3}
            & Expected Result &
                \begin{minipage}[t]{13cm}{\footnotesize
                Network recovers to primary path.~

                \vspace{\dp0}
                } \end{minipage}
        \\ \midrule

            \multirow{3}{*}{ 4 } & Description &
            \begin{minipage}[t]{13cm}{\footnotesize
            Analyze fail-over and recovery times. ~Compare to historical data and
extrapolate to MTTR.

            \vspace{\dp0}
            } \end{minipage} \\ \cline{2-3}
            & Test Data &
            \begin{minipage}[t]{13cm}{\footnotesize
                No data.
                \vspace{\dp0}
            } \end{minipage} \\ \cline{2-3}
            & Expected Result &
                \begin{minipage}[t]{13cm}{\footnotesize
                Verify recovery can occur within baseToArchNetMTTR = 48{[}hour{]}.
Demonstrate reconnection and recovery to transfer of data at or
exceeding rates specified in LDM-142.

                \vspace{\dp0}
                } \end{minipage}
        \\ \midrule
    \end{longtable}

\subsection{LVV-T196 - Verify implementation of Base to Archive Network Secondary Link}\label{lvv-t196}

\begin{longtable}[]{llllll}
\toprule
Version & Status & Priority & Verification Type & Owner
\\\midrule
1 & Draft & Normal &
Test & Jeff Kantor
\\\bottomrule
\multicolumn{6}{c}{ Open \href{https://jira.lsstcorp.org/secure/Tests.jspa\#/testCase/LVV-T196}{LVV-T196} in Jira } \\
\end{longtable}

\subsubsection{Verification Elements}
\begin{itemize}
\item \href{https://jira.lsstcorp.org/browse/LVV-84}{LVV-84} - DMS-REQ-0183-V-01: Base to Archive Network Secondary Link

\end{itemize}

\subsubsection{Test Items}
Verify Base to Archive Network Secondary Link failover and capacity, and
subsequent recovery primary. Demonstrate the use of the secondary path
in ``catch-up'' mode.


\subsubsection{Predecessors}
PMCS DM-NET-5 Complete\\
PMCS DMTC-8000-0990 Complete\\
PMCS DMTC-8100-2130 Complete\\
PMCS DMTC-8100-2530 Complete\\
PMCS DMTC-8200-0600 Complete

\subsubsection{Environment Needs}

\paragraph{Software}
See pre-conditions.

\paragraph{Hardware}
See pre-conditions.

\subsubsection{Input Specification}
\begin{enumerate}
\tightlist
\item
  Archiver/Forwarders are configured at Base, connected to REUNA DWDM,
  loaded with simulated or pre-cursor data, running on end node
  computers that are the production hardware or equivalent to it.
\item
  Archiver/Forwarder receivers or other capability is on configured at
  LDF, connected to Base - Archive Network, running on end node
  computers that are the production hardware or equivalent to it.
\item
  As-built documentation for all of the above is available.
\end{enumerate}

NOTE: This test will be repeated at increasing data volumes as
additional observatory capabilities (e.g. ComCAM, FullCam) become
available. Final verification will be tested at full operational volume.
After the initial test, the corresponding verification elements will be
flagged as ``Requires Monitoring'' such that those requirements will be
closed out as having been verified but will continue to be monitored
throughout commissioning to ensure they do not drop out of compliance.
This will also be monitored for end to end Summit - Data Facility
transfers during Commissioning.\\[2\baselineskip]

\subsubsection{Output Specification}

\subsubsection{Test Procedure}
    \begin{longtable}[]{p{1.3cm}p{2cm}p{13cm}}
    %\toprule
    Step & \multicolumn{2}{@{}l}{Description, Input Data and Expected Result} \\ \toprule
    \endhead

            \multirow{3}{*}{ 1 } & Description &
            \begin{minipage}[t]{13cm}{\footnotesize
            Transfer data between base and archive on primary links over
uninterrupted 1 day period.

            \vspace{\dp0}
            } \end{minipage} \\ \cline{2-3}
            & Test Data &
            \begin{minipage}[t]{13cm}{\footnotesize
                LATISS, ComCAM, or FullCAM data.

                \vspace{\dp0}
            } \end{minipage} \\ \cline{2-3}
            & Expected Result &
                \begin{minipage}[t]{13cm}{\footnotesize
                Data is successfully transferred over primary link at or exceeding rates
specified in LDM-142 throughout period.

                \vspace{\dp0}
                } \end{minipage}
        \\ \midrule

            \multirow{3}{*}{ 2 } & Description &
            \begin{minipage}[t]{13cm}{\footnotesize
            Simulate outage by disconnecting fiber on primary fiber on Base side of
Base - Archive Network.

            \vspace{\dp0}
            } \end{minipage} \\ \cline{2-3}
            & Test Data &
            \begin{minipage}[t]{13cm}{\footnotesize
                NA

                \vspace{\dp0}
            } \end{minipage} \\ \cline{2-3}
            & Expected Result &
                \begin{minipage}[t]{13cm}{\footnotesize
                Network fails over to secondary links in \textless{}=60s

                \vspace{\dp0}
                } \end{minipage}
        \\ \midrule

            \multirow{3}{*}{ 3 } & Description &
            \begin{minipage}[t]{13cm}{\footnotesize
            Transfer data between base and archive over secondary equipment
uninterrupted 1 day period.

            \vspace{\dp0}
            } \end{minipage} \\ \cline{2-3}
            & Test Data &
            \begin{minipage}[t]{13cm}{\footnotesize
                LATISS, ComCAM, or FullCAM data.

                \vspace{\dp0}
            } \end{minipage} \\ \cline{2-3}
            & Expected Result &
                \begin{minipage}[t]{13cm}{\footnotesize
                Data is successfully transferred over secondary link ~at or exceeding
rates specified in LDM-142 throughout period.

                \vspace{\dp0}
                } \end{minipage}
        \\ \midrule

            \multirow{3}{*}{ 4 } & Description &
            \begin{minipage}[t]{13cm}{\footnotesize
            Restore connection on primary link by reconnecting
fiber.\\[2\baselineskip]

            \vspace{\dp0}
            } \end{minipage} \\ \cline{2-3}
            & Test Data &
            \begin{minipage}[t]{13cm}{\footnotesize
                NA

                \vspace{\dp0}
            } \end{minipage} \\ \cline{2-3}
            & Expected Result &
                \begin{minipage}[t]{13cm}{\footnotesize
                Network recovers to primary.

                \vspace{\dp0}
                } \end{minipage}
        \\ \midrule

            \multirow{3}{*}{ 5 } & Description &
            \begin{minipage}[t]{13cm}{\footnotesize
            Demonstrate use of secondary in catch-up mode.

            \vspace{\dp0}
            } \end{minipage} \\ \cline{2-3}
            & Test Data &
            \begin{minipage}[t]{13cm}{\footnotesize
                DAQ buffer full of images and associated metadata.

                \vspace{\dp0}
            } \end{minipage} \\ \cline{2-3}
            & Expected Result &
                \begin{minipage}[t]{13cm}{\footnotesize
                Images from DAQ buffer and associated metadata are retrievable over
secondary path while current observing data is being transferred over
primary path.

                \vspace{\dp0}
                } \end{minipage}
        \\ \midrule
    \end{longtable}

\subsection{LVV-T197 - Verify implementation of Archive Center}\label{lvv-t197}

\begin{longtable}[]{llllll}
\toprule
Version & Status & Priority & Verification Type & Owner
\\\midrule
1 & Draft & Normal &
Test & Robert Gruendl
\\\bottomrule
\multicolumn{6}{c}{ Open \href{https://jira.lsstcorp.org/secure/Tests.jspa\#/testCase/LVV-T197}{LVV-T197} in Jira } \\
\end{longtable}

\subsubsection{Verification Elements}
\begin{itemize}
\item \href{https://jira.lsstcorp.org/browse/LVV-85}{LVV-85} - DMS-REQ-0185-V-01: Archive Center

\end{itemize}

\subsubsection{Test Items}
Verify that the Archive Center is sufficiently provisioned to support
prompt processing, DRP, and data access needs.


\subsubsection{Predecessors}

\subsubsection{Environment Needs}

\paragraph{Software}

\paragraph{Hardware}

\subsubsection{Input Specification}

\subsubsection{Output Specification}

\subsubsection{Test Procedure}
    \begin{longtable}[]{p{1.3cm}p{2cm}p{13cm}}
    %\toprule
    Step & \multicolumn{2}{@{}l}{Description, Input Data and Expected Result} \\ \toprule
    \endhead

            \multirow{3}{*}{ 1 } & Description &
            \begin{minipage}[t]{13cm}{\footnotesize
            Analyze design and sizing model

            \vspace{\dp0}
            } \end{minipage} \\ \cline{2-3}
            & Test Data &
            \begin{minipage}[t]{13cm}{\footnotesize
                No data.
                \vspace{\dp0}
            } \end{minipage} \\ \cline{2-3}
            & Expected Result &
        \\ \midrule
    \end{longtable}

\subsection{LVV-T198 - Verify implementation of Archive Center Disaster Recovery}\label{lvv-t198}

\begin{longtable}[]{llllll}
\toprule
Version & Status & Priority & Verification Type & Owner
\\\midrule
1 & Draft & Normal &
Test & Robert Gruendl
\\\bottomrule
\multicolumn{6}{c}{ Open \href{https://jira.lsstcorp.org/secure/Tests.jspa\#/testCase/LVV-T198}{LVV-T198} in Jira } \\
\end{longtable}

\subsubsection{Verification Elements}
\begin{itemize}
\item \href{https://jira.lsstcorp.org/browse/LVV-86}{LVV-86} - DMS-REQ-0186-V-01: Archive Center Disaster Recovery

\end{itemize}

\subsubsection{Test Items}
Verify disaster recovery plan for Archive Center.


\subsubsection{Predecessors}

\subsubsection{Environment Needs}

\paragraph{Software}

\paragraph{Hardware}

\subsubsection{Input Specification}

\subsubsection{Output Specification}

\subsubsection{Test Procedure}
    \begin{longtable}[]{p{1.3cm}p{2cm}p{13cm}}
    %\toprule
    Step & \multicolumn{2}{@{}l}{Description, Input Data and Expected Result} \\ \toprule
    \endhead

            \multirow{3}{*}{ 1 } & Description &
            \begin{minipage}[t]{13cm}{\footnotesize
            Analyze design; simulate storage failure, observe restore from disaster
recovery

            \vspace{\dp0}
            } \end{minipage} \\ \cline{2-3}
            & Test Data &
            \begin{minipage}[t]{13cm}{\footnotesize
                No data.
                \vspace{\dp0}
            } \end{minipage} \\ \cline{2-3}
            & Expected Result &
        \\ \midrule
    \end{longtable}

\subsection{LVV-T199 - Verify implementation of Archive Center Co-Location with Existing
Facility}\label{lvv-t199}

\begin{longtable}[]{llllll}
\toprule
Version & Status & Priority & Verification Type & Owner
\\\midrule
1 & Draft & Normal &
Test & Robert Gruendl
\\\bottomrule
\multicolumn{6}{c}{ Open \href{https://jira.lsstcorp.org/secure/Tests.jspa\#/testCase/LVV-T199}{LVV-T199} in Jira } \\
\end{longtable}

\subsubsection{Verification Elements}
\begin{itemize}
\item \href{https://jira.lsstcorp.org/browse/LVV-87}{LVV-87} - DMS-REQ-0187-V-01: Archive Center Co-Location with Existing Facility

\end{itemize}

\subsubsection{Test Items}
Verify the Archive Center is located at an existing supported facility.


\subsubsection{Predecessors}

\subsubsection{Environment Needs}

\paragraph{Software}

\paragraph{Hardware}

\subsubsection{Input Specification}

\subsubsection{Output Specification}

\subsubsection{Test Procedure}
    \begin{longtable}[]{p{1.3cm}p{2cm}p{13cm}}
    %\toprule
    Step & \multicolumn{2}{@{}l}{Description, Input Data and Expected Result} \\ \toprule
    \endhead

            \multirow{3}{*}{ 1 } & Description &
            \begin{minipage}[t]{13cm}{\footnotesize
            Analyze design

            \vspace{\dp0}
            } \end{minipage} \\ \cline{2-3}
            & Test Data &
            \begin{minipage}[t]{13cm}{\footnotesize
                No data.
                \vspace{\dp0}
            } \end{minipage} \\ \cline{2-3}
            & Expected Result &
        \\ \midrule
    \end{longtable}

\subsection{LVV-T200 - Verify implementation of Archive to Data Access Center Network}\label{lvv-t200}

\begin{longtable}[]{llllll}
\toprule
Version & Status & Priority & Verification Type & Owner
\\\midrule
1 & Draft & Normal &
Test & Jeff Kantor
\\\bottomrule
\multicolumn{6}{c}{ Open \href{https://jira.lsstcorp.org/secure/Tests.jspa\#/testCase/LVV-T200}{LVV-T200} in Jira } \\
\end{longtable}

\subsubsection{Verification Elements}
\begin{itemize}
\item \href{https://jira.lsstcorp.org/browse/LVV-88}{LVV-88} - DMS-REQ-0188-V-01: Archive to Data Access Center Network

\end{itemize}

\subsubsection{Test Items}
Verify archiving of data to Data Access Center Network at or exceeding
rates specified in \citeds{LDM-142}, i.e at archToDacBandwidth = 10000{[}megabit
per second{]}.


\subsubsection{Predecessors}
PMCS DMTC-8100-2550 Complete

\subsubsection{Environment Needs}

\paragraph{Software}
See pre-conditions.

\paragraph{Hardware}
See pre-conditions.

\subsubsection{Input Specification}
\begin{enumerate}
\tightlist
\item
  Data is staged in LDF and data transfer capabilities to US DAC and
  Chilean DAC are in place, running on end node computers that are the
  production hardware or equivalent to it.
\item
  At least 6 months of historical monitoring data is available on these
  network links.
\item
  As-built documentation for all of the above is available.
\end{enumerate}

NOTE: This test will be repeated at increasing data volumes as
additional observatory capabilities (e.g. ComCAM, FullCam) become
available. Final verification will be tested at full operational volume.
After the initial test, the corresponding verification elements will be
flagged as ``Requires Monitoring'' such that those requirements will be
closed out as having been verified but will continue to be monitored
throughout commissioning to ensure they do not drop out of compliance.
This will also be monitored for end to end Summit - Data Facility
transfers during Commissioning.\\[2\baselineskip]

\subsubsection{Output Specification}

\subsubsection{Test Procedure}
    \begin{longtable}[]{p{1.3cm}p{2cm}p{13cm}}
    %\toprule
    Step & \multicolumn{2}{@{}l}{Description, Input Data and Expected Result} \\ \toprule
    \endhead

            \multirow{3}{*}{ 1 } & Description &
            \begin{minipage}[t]{13cm}{\footnotesize
            Transfer data from Data Facility to US and Chilean DACs over an
uninterrupted 1 week period.\\[2\baselineskip]

            \vspace{\dp0}
            } \end{minipage} \\ \cline{2-3}
            & Test Data &
            \begin{minipage}[t]{13cm}{\footnotesize
                Data Release

                \vspace{\dp0}
            } \end{minipage} \\ \cline{2-3}
            & Expected Result &
                \begin{minipage}[t]{13cm}{\footnotesize
                Data transfers without significant failures or extended latency spikes

                \vspace{\dp0}
                } \end{minipage}
        \\ \midrule

            \multirow{3}{*}{ 2 } & Description &
            \begin{minipage}[t]{13cm}{\footnotesize
            Analyze network logs and compare with historical data on the links.

            \vspace{\dp0}
            } \end{minipage} \\ \cline{2-3}
            & Test Data &
            \begin{minipage}[t]{13cm}{\footnotesize
                NA

                \vspace{\dp0}
            } \end{minipage} \\ \cline{2-3}
            & Expected Result &
                \begin{minipage}[t]{13cm}{\footnotesize
                The networks can transfer data at archToDacBandwidth = 10000{[}megabit
per second{]}, i.e. at or exceeding rates specified in LDM-142.

                \vspace{\dp0}
                } \end{minipage}
        \\ \midrule
    \end{longtable}

\subsection{LVV-T201 - Verify implementation of Archive to Data Access Center Network
Availability}\label{lvv-t201}

\begin{longtable}[]{llllll}
\toprule
Version & Status & Priority & Verification Type & Owner
\\\midrule
1 & Draft & Normal &
Test & Jeff Kantor
\\\bottomrule
\multicolumn{6}{c}{ Open \href{https://jira.lsstcorp.org/secure/Tests.jspa\#/testCase/LVV-T201}{LVV-T201} in Jira } \\
\end{longtable}

\subsubsection{Verification Elements}
\begin{itemize}
\item \href{https://jira.lsstcorp.org/browse/LVV-89}{LVV-89} - DMS-REQ-0189-V-01: Archive to Data Access Center Network Availability

\end{itemize}

\subsubsection{Test Items}
Verify availability of archiving to Data Access Center Network using
test and historical data of or exceeding archToDacNetMTBF= 180{[}day{]}.


\subsubsection{Predecessors}
PMCS DMTC-8100-2550 Complete

\subsubsection{Environment Needs}

\paragraph{Software}
See pre-conditions.

\paragraph{Hardware}
See pre-conditions.

\subsubsection{Input Specification}
\begin{enumerate}
\tightlist
\item
  Data is staged in LDF and data transfer capabilities to US DAC and
  Chilean DAC are in place, ~running on end node computers that are the
  production hardware or equivalent to it.
\item
  At least 6 months of historical monitoring data is available on these
  network links, ~running on end node computers that are the production
  hardware or equivalent to it.
\item
  As-built documentation for all of the above is available.
\end{enumerate}

NOTE: This test will be repeated at increasing data volumes as
additional observatory capabilities (e.g. ComCAM, FullCam) become
available. Final verification will be tested at full operational volume.
After the initial test, the corresponding verification elements will be
flagged as ``Requires Monitoring'' such that those requirements will be
closed out as having been verified but will continue to be monitored
throughout commissioning to ensure they do not drop out of compliance.
This will also be monitored for end to end Summit - Data Facility
transfers during Commissioning.\\[2\baselineskip]

\subsubsection{Output Specification}

\subsubsection{Test Procedure}
    \begin{longtable}[]{p{1.3cm}p{2cm}p{13cm}}
    %\toprule
    Step & \multicolumn{2}{@{}l}{Description, Input Data and Expected Result} \\ \toprule
    \endhead

            \multirow{3}{*}{ 1 } & Description &
            \begin{minipage}[t]{13cm}{\footnotesize
            Transfer data between archive and DACs over uninterrupted 1 week period.

            \vspace{\dp0}
            } \end{minipage} \\ \cline{2-3}
            & Test Data &
            \begin{minipage}[t]{13cm}{\footnotesize
                Data Release or petabyte-scale test data set

                \vspace{\dp0}
            } \end{minipage} \\ \cline{2-3}
            & Expected Result &
                \begin{minipage}[t]{13cm}{\footnotesize
                Data transfers without failures or extended latency spikes

                \vspace{\dp0}
                } \end{minipage}
        \\ \midrule

            \multirow{3}{*}{ 2 } & Description &
            \begin{minipage}[t]{13cm}{\footnotesize
            Analyze test data and compare to historical data. Extrapolate to 1 year
testimate of MTBF.

            \vspace{\dp0}
            } \end{minipage} \\ \cline{2-3}
            & Test Data &
            \begin{minipage}[t]{13cm}{\footnotesize
                NA

                \vspace{\dp0}
            } \end{minipage} \\ \cline{2-3}
            & Expected Result &
                \begin{minipage}[t]{13cm}{\footnotesize
                Networks can meet archToDacNetMTBF = 180{[}day{]} at or exceeding rates
specified in LDM-142.

                \vspace{\dp0}
                } \end{minipage}
        \\ \midrule
    \end{longtable}

\subsection{LVV-T202 - Verify implementation of Archive to Data Access Center Network
Reliability}\label{lvv-t202}

\begin{longtable}[]{llllll}
\toprule
Version & Status & Priority & Verification Type & Owner
\\\midrule
1 & Draft & Normal &
Test & Jeff Kantor
\\\bottomrule
\multicolumn{6}{c}{ Open \href{https://jira.lsstcorp.org/secure/Tests.jspa\#/testCase/LVV-T202}{LVV-T202} in Jira } \\
\end{longtable}

\subsubsection{Verification Elements}
\begin{itemize}
\item \href{https://jira.lsstcorp.org/browse/LVV-90}{LVV-90} - DMS-REQ-0190-V-01: Archive to Data Access Center Network Reliability

\end{itemize}

\subsubsection{Test Items}
Verify the reliability of Archive to Data Access Center Network by
demonstrating successful failover and capacity to the secondary part and
subsequent recovery to primary within or exceeding chToDacNetMTTR =
48{[}hour{]}.


\subsubsection{Predecessors}
PMCS DMTC-8100-2550 Complete

\subsubsection{Environment Needs}

\paragraph{Software}
See pre-conditions.

\paragraph{Hardware}
See pre-conditions.

\subsubsection{Input Specification}
\begin{enumerate}
\tightlist
\item
  Data is staged in LDF and data transfer capabilities to US DAC and
  Chilean DAC are in place, running on end node computers that are the
  production hardware or equivalent to it.
\item
  As-built documentation for all of the above is available.
\item
  NOTE: This test will be repeated at increasing data volumes as
  additional observatory capabilities (e.g. ComCAM, FullCam) become
  available. ~Final verification will be tested at full operational
  volume. ~After the initial test, the corresponding verification
  elements will be flagged as ``Requires Monitoring'' such that those
  requirements will be closed out as having been verified but will
  continue to be monitored throughout commissioning to ensure they do
  not drop out of compliance. This will also be monitored for end to end
  Summit - Data Facility transfers during Commissioning.
\end{enumerate}

\subsubsection{Output Specification}

\subsubsection{Test Procedure}
    \begin{longtable}[]{p{1.3cm}p{2cm}p{13cm}}
    %\toprule
    Step & \multicolumn{2}{@{}l}{Description, Input Data and Expected Result} \\ \toprule
    \endhead

            \multirow{3}{*}{ 1 } & Description &
            \begin{minipage}[t]{13cm}{\footnotesize
            Simulate failure on primary paths by disconnecting fiber at an endpoint
location in the archive on the Archive - ~DACs network.

            \vspace{\dp0}
            } \end{minipage} \\ \cline{2-3}
            & Test Data &
            \begin{minipage}[t]{13cm}{\footnotesize
                NA

                \vspace{\dp0}
            } \end{minipage} \\ \cline{2-3}
            & Expected Result &
                \begin{minipage}[t]{13cm}{\footnotesize
                Networks fail over to secondary paths.

                \vspace{\dp0}
                } \end{minipage}
        \\ \midrule

            \multirow{3}{*}{ 2 } & Description &
            \begin{minipage}[t]{13cm}{\footnotesize
            Monitor transfers on secondary paths for 1 day.

            \vspace{\dp0}
            } \end{minipage} \\ \cline{2-3}
            & Test Data &
            \begin{minipage}[t]{13cm}{\footnotesize
                No data.
                \vspace{\dp0}
            } \end{minipage} \\ \cline{2-3}
            & Expected Result &
                \begin{minipage}[t]{13cm}{\footnotesize
                Transfers occur without extended failures or extended latency spikes.
~Data transfers on secondary at rates at or above those specified in
LDM-142.

                \vspace{\dp0}
                } \end{minipage}
        \\ \midrule

            \multirow{3}{*}{ 3 } & Description &
            \begin{minipage}[t]{13cm}{\footnotesize
            Simulate repair and recovery period by leaving primary fiber
disconnected for at least 1 day, then reconnecting primary fiber.

            \vspace{\dp0}
            } \end{minipage} \\ \cline{2-3}
            & Test Data &
            \begin{minipage}[t]{13cm}{\footnotesize
                NA

                \vspace{\dp0}
            } \end{minipage} \\ \cline{2-3}
            & Expected Result &
                \begin{minipage}[t]{13cm}{\footnotesize
                Wall clock advances by 1 day. ~Network recovers to primary path. ~Verify
entire process meets chToDacNetMTTR = 48{[}hour{]}.

                \vspace{\dp0}
                } \end{minipage}
        \\ \midrule
    \end{longtable}

\subsection{LVV-T203 - Verify implementation of Archive to Data Access Center Network Secondary
Link}\label{lvv-t203}

\begin{longtable}[]{llllll}
\toprule
Version & Status & Priority & Verification Type & Owner
\\\midrule
1 & Draft & Normal &
Test & Kian-Tat Lim
\\\bottomrule
\multicolumn{6}{c}{ Open \href{https://jira.lsstcorp.org/secure/Tests.jspa\#/testCase/LVV-T203}{LVV-T203} in Jira } \\
\end{longtable}

\subsubsection{Verification Elements}
\begin{itemize}
\item \href{https://jira.lsstcorp.org/browse/LVV-91}{LVV-91} - DMS-REQ-0191-V-01: Archive to Data Access Center Network Secondary Link

\end{itemize}

\subsubsection{Test Items}
Verify the Archive to Data Access Center Network via Secondary Link by
simulating a failure on the primary path and capacity on the secondary
path.


\subsubsection{Predecessors}
PMCS DMTC-8100-2550 Complete

\subsubsection{Environment Needs}

\paragraph{Software}
See pre-conditions.

\paragraph{Hardware}
See pre-conditions.

\subsubsection{Input Specification}
\begin{enumerate}
\tightlist
\item
  Data is staged in LDF and data transfer capabilities to US DAC and
  Chilean DAC are in place, running on end node computers that are the
  production hardware or equivalent to it.
\item
  As-built documentation for all of the above is available.
\end{enumerate}

NOTE: This test will be repeated at increasing data volumes as
additional observatory capabilities (e.g. ComCAM, FullCam) become
available. Final verification will be tested at full operational volume.
After the initial test, the corresponding verification elements will be
flagged as ``Requires Monitoring'' such that those requirements will be
closed out as having been verified but will continue to be monitored
throughout commissioning to ensure they do not drop out of compliance.
This will also be monitored for end to end Summit - Data Facility
transfers during Commissioning.\\[2\baselineskip]

\subsubsection{Output Specification}

\subsubsection{Test Procedure}
    \begin{longtable}[]{p{1.3cm}p{2cm}p{13cm}}
    %\toprule
    Step & \multicolumn{2}{@{}l}{Description, Input Data and Expected Result} \\ \toprule
    \endhead

            \multirow{3}{*}{ 1 } & Description &
            \begin{minipage}[t]{13cm}{\footnotesize
            Transfer data between Archive and DACs on primary path over
uninterrupted 1 week period.

            \vspace{\dp0}
            } \end{minipage} \\ \cline{2-3}
            & Test Data &
            \begin{minipage}[t]{13cm}{\footnotesize
                Data Release or other petabyte-scale test data set.

                \vspace{\dp0}
            } \end{minipage} \\ \cline{2-3}
            & Expected Result &
                \begin{minipage}[t]{13cm}{\footnotesize
                Data transfers without failures or extended latency spikes, at or
exceeding rates specified in LDM-142 throughout fail-over period.

                \vspace{\dp0}
                } \end{minipage}
        \\ \midrule

            \multirow{3}{*}{ 2 } & Description &
            \begin{minipage}[t]{13cm}{\footnotesize
            Simulate outage on primary path by disconnecting fiber on primary on
Archive side of Archive - DACs networks.

            \vspace{\dp0}
            } \end{minipage} \\ \cline{2-3}
            & Test Data &
            \begin{minipage}[t]{13cm}{\footnotesize
                NA

                \vspace{\dp0}
            } \end{minipage} \\ \cline{2-3}
            & Expected Result &
                \begin{minipage}[t]{13cm}{\footnotesize
                Network fails over to secondary links in \textless{}=
60s.\\[2\baselineskip]

                \vspace{\dp0}
                } \end{minipage}
        \\ \midrule

            \multirow{3}{*}{ 3 } & Description &
            \begin{minipage}[t]{13cm}{\footnotesize
            Transfer data between base and archive over secondary equipment
uninterrupted 1 day period.

            \vspace{\dp0}
            } \end{minipage} \\ \cline{2-3}
            & Test Data &
            \begin{minipage}[t]{13cm}{\footnotesize
                Data Release or other petabyte-scale test data set.

                \vspace{\dp0}
            } \end{minipage} \\ \cline{2-3}
            & Expected Result &
                \begin{minipage}[t]{13cm}{\footnotesize
                Data transfers without failures or extended latency spikes, ~at or
exceeding rates specified in LDM-142 throughout fail-over period.

                \vspace{\dp0}
                } \end{minipage}
        \\ \midrule

            \multirow{3}{*}{ 4 } & Description &
            \begin{minipage}[t]{13cm}{\footnotesize
            Restore connection on primary link (reconnect fiber).

            \vspace{\dp0}
            } \end{minipage} \\ \cline{2-3}
            & Test Data &
            \begin{minipage}[t]{13cm}{\footnotesize
                NA

                \vspace{\dp0}
            } \end{minipage} \\ \cline{2-3}
            & Expected Result &
                \begin{minipage}[t]{13cm}{\footnotesize
                Network recovers to primary in \textless{}= 60s.

                \vspace{\dp0}
                } \end{minipage}
        \\ \midrule
    \end{longtable}

\subsection{LVV-T204 - Verify implementation of Access to catalogs for external Level 3
processing}\label{lvv-t204}

\begin{longtable}[]{llllll}
\toprule
Version & Status & Priority & Verification Type & Owner
\\\midrule
1 & Draft & Normal &
Test & Kian-Tat Lim
\\\bottomrule
\multicolumn{6}{c}{ Open \href{https://jira.lsstcorp.org/secure/Tests.jspa\#/testCase/LVV-T204}{LVV-T204} in Jira } \\
\end{longtable}

\subsubsection{Verification Elements}
\begin{itemize}
\item \href{https://jira.lsstcorp.org/browse/LVV-50}{LVV-50} - DMS-REQ-0122-V-01: Access to catalogs for external Level 3 processing

\end{itemize}

\subsubsection{Test Items}
Verify that catalog export, and maintenance/validation tools for Level 3
products to outside of the Data Access Centers.


\subsubsection{Predecessors}

\subsubsection{Environment Needs}

\paragraph{Software}

\paragraph{Hardware}

\subsubsection{Input Specification}

\subsubsection{Output Specification}

\subsubsection{Test Procedure}
    \begin{longtable}[]{p{1.3cm}p{2cm}p{13cm}}
    %\toprule
    Step & \multicolumn{2}{@{}l}{Description, Input Data and Expected Result} \\ \toprule
    \endhead

            \multirow{3}{*}{ 1 } & Description &
            \begin{minipage}[t]{13cm}{\footnotesize
            Execute bulk distribution of DRP catalogs

            \vspace{\dp0}
            } \end{minipage} \\ \cline{2-3}
            & Test Data &
            \begin{minipage}[t]{13cm}{\footnotesize
                No data.
                \vspace{\dp0}
            } \end{minipage} \\ \cline{2-3}
            & Expected Result &
        \\ \midrule

            \multirow{3}{*}{ 2 } & Description &
            \begin{minipage}[t]{13cm}{\footnotesize
            Observe correct transfer and use of maintenance/validation tools

            \vspace{\dp0}
            } \end{minipage} \\ \cline{2-3}
            & Test Data &
            \begin{minipage}[t]{13cm}{\footnotesize
                No data.
                \vspace{\dp0}
            } \end{minipage} \\ \cline{2-3}
            & Expected Result &
        \\ \midrule
    \end{longtable}

\subsection{LVV-T205 - Verify implementation of Access to input catalogs for DAC-based Level 3
processing}\label{lvv-t205}

\begin{longtable}[]{llllll}
\toprule
Version & Status & Priority & Verification Type & Owner
\\\midrule
1 & Draft & Normal &
Test & Robert Gruendl
\\\bottomrule
\multicolumn{6}{c}{ Open \href{https://jira.lsstcorp.org/secure/Tests.jspa\#/testCase/LVV-T205}{LVV-T205} in Jira } \\
\end{longtable}

\subsubsection{Verification Elements}
\begin{itemize}
\item \href{https://jira.lsstcorp.org/browse/LVV-51}{LVV-51} - DMS-REQ-0123-V-01: Access to input catalogs for DAC-based Level 3
processing

\end{itemize}

\subsubsection{Test Items}
Verify that data products are available at the Data Access Centers for
use in Level 3 processing.


\subsubsection{Predecessors}

\subsubsection{Environment Needs}

\paragraph{Software}

\paragraph{Hardware}

\subsubsection{Input Specification}

\subsubsection{Output Specification}

\subsubsection{Test Procedure}
    \begin{longtable}[]{p{1.3cm}p{2cm}p{13cm}}
    %\toprule
    Step & \multicolumn{2}{@{}l}{Description, Input Data and Expected Result} \\ \toprule
    \endhead

            \multirow{3}{*}{ 1 } & Description &
            \begin{minipage}[t]{13cm}{\footnotesize
            Load Prompt and DR catalogs into PDAC, observe access via LSP

            \vspace{\dp0}
            } \end{minipage} \\ \cline{2-3}
            & Test Data &
            \begin{minipage}[t]{13cm}{\footnotesize
                No data.
                \vspace{\dp0}
            } \end{minipage} \\ \cline{2-3}
            & Expected Result &
        \\ \midrule
    \end{longtable}

\subsection{LVV-T206 - Verify implementation of Federation with external catalogs}\label{lvv-t206}

\begin{longtable}[]{llllll}
\toprule
Version & Status & Priority & Verification Type & Owner
\\\midrule
1 & Draft & Normal &
Test & Colin Slater
\\\bottomrule
\multicolumn{6}{c}{ Open \href{https://jira.lsstcorp.org/secure/Tests.jspa\#/testCase/LVV-T206}{LVV-T206} in Jira } \\
\end{longtable}

\subsubsection{Verification Elements}
\begin{itemize}
\item \href{https://jira.lsstcorp.org/browse/LVV-52}{LVV-52} - DMS-REQ-0124-V-01: Federation with external catalogs

\end{itemize}

\subsubsection{Test Items}
Verify that LSST-produced data can be combined with external datasets.


\subsubsection{Predecessors}

\subsubsection{Environment Needs}

\paragraph{Software}

\paragraph{Hardware}

\subsubsection{Input Specification}

\subsubsection{Output Specification}

\subsubsection{Test Procedure}
    \begin{longtable}[]{p{1.3cm}p{2cm}p{13cm}}
    %\toprule
    Step & \multicolumn{2}{@{}l}{Description, Input Data and Expected Result} \\ \toprule
    \endhead

            \multirow{3}{*}{ 1 } & Description &
            \begin{minipage}[t]{13cm}{\footnotesize
            Load external catalog into PDAC (using VO if possible), observe
federation with other catalogs via LSP

            \vspace{\dp0}
            } \end{minipage} \\ \cline{2-3}
            & Test Data &
            \begin{minipage}[t]{13cm}{\footnotesize
                No data.
                \vspace{\dp0}
            } \end{minipage} \\ \cline{2-3}
            & Expected Result &
        \\ \midrule
    \end{longtable}

\subsection{LVV-T207 - Verify implementation of Access to images for external Level 3
processing}\label{lvv-t207}

\begin{longtable}[]{llllll}
\toprule
Version & Status & Priority & Verification Type & Owner
\\\midrule
1 & Draft & Normal &
Test & Kian-Tat Lim
\\\bottomrule
\multicolumn{6}{c}{ Open \href{https://jira.lsstcorp.org/secure/Tests.jspa\#/testCase/LVV-T207}{LVV-T207} in Jira } \\
\end{longtable}

\subsubsection{Verification Elements}
\begin{itemize}
\item \href{https://jira.lsstcorp.org/browse/LVV-54}{LVV-54} - DMS-REQ-0126-V-01: Access to images for external Level 3 processing

\end{itemize}

\subsubsection{Test Items}
Verify that bulk distribution of images, and accompanying
maintenance/validation tools for Level 3 image products to outside of
the Data Access Centers.


\subsubsection{Predecessors}

\subsubsection{Environment Needs}

\paragraph{Software}

\paragraph{Hardware}

\subsubsection{Input Specification}

\subsubsection{Output Specification}

\subsubsection{Test Procedure}
    \begin{longtable}[]{p{1.3cm}p{2cm}p{13cm}}
    %\toprule
    Step & \multicolumn{2}{@{}l}{Description, Input Data and Expected Result} \\ \toprule
    \endhead

            \multirow{3}{*}{ 1 } & Description &
            \begin{minipage}[t]{13cm}{\footnotesize
            Execute bulk distribution of DRP images

            \vspace{\dp0}
            } \end{minipage} \\ \cline{2-3}
            & Test Data &
            \begin{minipage}[t]{13cm}{\footnotesize
                No data.
                \vspace{\dp0}
            } \end{minipage} \\ \cline{2-3}
            & Expected Result &
        \\ \midrule

            \multirow{3}{*}{ 2 } & Description &
            \begin{minipage}[t]{13cm}{\footnotesize
            Observe correct transfer and use of maintenance/validation tools

            \vspace{\dp0}
            } \end{minipage} \\ \cline{2-3}
            & Test Data &
            \begin{minipage}[t]{13cm}{\footnotesize
                No data.
                \vspace{\dp0}
            } \end{minipage} \\ \cline{2-3}
            & Expected Result &
        \\ \midrule
    \end{longtable}

\subsection{LVV-T208 - Verify implementation of Access to input images for DAC-based Level 3
processing}\label{lvv-t208}

\begin{longtable}[]{llllll}
\toprule
Version & Status & Priority & Verification Type & Owner
\\\midrule
1 & Draft & Normal &
Test & Kian-Tat Lim
\\\bottomrule
\multicolumn{6}{c}{ Open \href{https://jira.lsstcorp.org/secure/Tests.jspa\#/testCase/LVV-T208}{LVV-T208} in Jira } \\
\end{longtable}

\subsubsection{Verification Elements}
\begin{itemize}
\item \href{https://jira.lsstcorp.org/browse/LVV-55}{LVV-55} - DMS-REQ-0127-V-01: Access to input images for DAC-based Level 3
processing

\end{itemize}

\subsubsection{Test Items}
Verify that prompt processing and DRP products are available at the DACs
for Level 3 processing at the DACs.


\subsubsection{Predecessors}

\subsubsection{Environment Needs}

\paragraph{Software}

\paragraph{Hardware}

\subsubsection{Input Specification}

\subsubsection{Output Specification}

\subsubsection{Test Procedure}
    \begin{longtable}[]{p{1.3cm}p{2cm}p{13cm}}
    %\toprule
    Step & \multicolumn{2}{@{}l}{Description, Input Data and Expected Result} \\ \toprule
    \endhead

            \multirow{3}{*}{ 1 } & Description &
            \begin{minipage}[t]{13cm}{\footnotesize
            Load Prompt and DR images into PDAC

            \vspace{\dp0}
            } \end{minipage} \\ \cline{2-3}
            & Test Data &
            \begin{minipage}[t]{13cm}{\footnotesize
                No data.
                \vspace{\dp0}
            } \end{minipage} \\ \cline{2-3}
            & Expected Result &
        \\ \midrule

            \multirow{3}{*}{ 2 } & Description &
            \begin{minipage}[t]{13cm}{\footnotesize
            Observe access via LSP

            \vspace{\dp0}
            } \end{minipage} \\ \cline{2-3}
            & Test Data &
            \begin{minipage}[t]{13cm}{\footnotesize
                No data.
                \vspace{\dp0}
            } \end{minipage} \\ \cline{2-3}
            & Expected Result &
        \\ \midrule
    \end{longtable}

\subsection{LVV-T209 - Verify implementation of Data Access Centers}\label{lvv-t209}

\begin{longtable}[]{llllll}
\toprule
Version & Status & Priority & Verification Type & Owner
\\\midrule
1 & Draft & Normal &
Analysis & Kian-Tat Lim
\\\bottomrule
\multicolumn{6}{c}{ Open \href{https://jira.lsstcorp.org/secure/Tests.jspa\#/testCase/LVV-T209}{LVV-T209} in Jira } \\
\end{longtable}

\subsubsection{Verification Elements}
\begin{itemize}
\item \href{https://jira.lsstcorp.org/browse/LVV-92}{LVV-92} - DMS-REQ-0193-V-01: Data Access Centers

\end{itemize}

\subsubsection{Test Items}
Verify that the Data Access Centers are provisioned with computing
resources necessary to support end-user access to LSST Data Products.


\subsubsection{Predecessors}

\subsubsection{Environment Needs}

\paragraph{Software}

\paragraph{Hardware}

\subsubsection{Input Specification}

\subsubsection{Output Specification}

\subsubsection{Test Procedure}
    \begin{longtable}[]{p{1.3cm}p{2cm}p{13cm}}
    %\toprule
    Step & \multicolumn{2}{@{}l}{Description, Input Data and Expected Result} \\ \toprule
    \endhead

            \multirow{3}{*}{ 1 } & Description &
            \begin{minipage}[t]{13cm}{\footnotesize
            Analyze design

            \vspace{\dp0}
            } \end{minipage} \\ \cline{2-3}
            & Test Data &
            \begin{minipage}[t]{13cm}{\footnotesize
                No data.
                \vspace{\dp0}
            } \end{minipage} \\ \cline{2-3}
            & Expected Result &
        \\ \midrule
    \end{longtable}

\subsection{LVV-T210 - Verify implementation of Data Access Center Simultaneous Connections}\label{lvv-t210}

\begin{longtable}[]{llllll}
\toprule
Version & Status & Priority & Verification Type & Owner
\\\midrule
1 & Draft & Normal &
Test & Kian-Tat Lim
\\\bottomrule
\multicolumn{6}{c}{ Open \href{https://jira.lsstcorp.org/secure/Tests.jspa\#/testCase/LVV-T210}{LVV-T210} in Jira } \\
\end{longtable}

\subsubsection{Verification Elements}
\begin{itemize}
\item \href{https://jira.lsstcorp.org/browse/LVV-93}{LVV-93} - DMS-REQ-0194-V-01: Data Access Center Simultaneous Connections

\end{itemize}

\subsubsection{Test Items}
Verify that the each DAC can support at least dacMinConnections
simultaneously


\subsubsection{Predecessors}

\subsubsection{Environment Needs}

\paragraph{Software}

\paragraph{Hardware}

\subsubsection{Input Specification}

\subsubsection{Output Specification}

\subsubsection{Test Procedure}
    \begin{longtable}[]{p{1.3cm}p{2cm}p{13cm}}
    %\toprule
    Step & \multicolumn{2}{@{}l}{Description, Input Data and Expected Result} \\ \toprule
    \endhead

            \multirow{3}{*}{ 1 } & Description &
            \begin{minipage}[t]{13cm}{\footnotesize
            Simulate data access to PDAC

            \vspace{\dp0}
            } \end{minipage} \\ \cline{2-3}
            & Test Data &
            \begin{minipage}[t]{13cm}{\footnotesize
                No data.
                \vspace{\dp0}
            } \end{minipage} \\ \cline{2-3}
            & Expected Result &
        \\ \midrule

            \multirow{3}{*}{ 2 } & Description &
            \begin{minipage}[t]{13cm}{\footnotesize
            Observe scaling

            \vspace{\dp0}
            } \end{minipage} \\ \cline{2-3}
            & Test Data &
            \begin{minipage}[t]{13cm}{\footnotesize
                No data.
                \vspace{\dp0}
            } \end{minipage} \\ \cline{2-3}
            & Expected Result &
        \\ \midrule
    \end{longtable}

\subsection{LVV-T211 - Verify implementation of Data Access Center Geographical Distribution}\label{lvv-t211}

\begin{longtable}[]{llllll}
\toprule
Version & Status & Priority & Verification Type & Owner
\\\midrule
1 & Draft & Normal &
Analysis & Kian-Tat Lim
\\\bottomrule
\multicolumn{6}{c}{ Open \href{https://jira.lsstcorp.org/secure/Tests.jspa\#/testCase/LVV-T211}{LVV-T211} in Jira } \\
\end{longtable}

\subsubsection{Verification Elements}
\begin{itemize}
\item \href{https://jira.lsstcorp.org/browse/LVV-94}{LVV-94} - DMS-REQ-0196-V-01: Data Access Center Geographical Distribution

\end{itemize}

\subsubsection{Test Items}
Verify that the DACs are geographically distributed to provide
low-latency access to data-rights community.


\subsubsection{Predecessors}

\subsubsection{Environment Needs}

\paragraph{Software}

\paragraph{Hardware}

\subsubsection{Input Specification}

\subsubsection{Output Specification}

\subsubsection{Test Procedure}
    \begin{longtable}[]{p{1.3cm}p{2cm}p{13cm}}
    %\toprule
    Step & \multicolumn{2}{@{}l}{Description, Input Data and Expected Result} \\ \toprule
    \endhead

            \multirow{3}{*}{ 1 } & Description &
            \begin{minipage}[t]{13cm}{\footnotesize
            Analyze design

            \vspace{\dp0}
            } \end{minipage} \\ \cline{2-3}
            & Test Data &
            \begin{minipage}[t]{13cm}{\footnotesize
                No data.
                \vspace{\dp0}
            } \end{minipage} \\ \cline{2-3}
            & Expected Result &
        \\ \midrule
    \end{longtable}

\subsection{LVV-T212 - Verify implementation of No Limit on Data Access Centers}\label{lvv-t212}

\begin{longtable}[]{llllll}
\toprule
Version & Status & Priority & Verification Type & Owner
\\\midrule
1 & Draft & Normal &
Test & Colin Slater
\\\bottomrule
\multicolumn{6}{c}{ Open \href{https://jira.lsstcorp.org/secure/Tests.jspa\#/testCase/LVV-T212}{LVV-T212} in Jira } \\
\end{longtable}

\subsubsection{Verification Elements}
\begin{itemize}
\item \href{https://jira.lsstcorp.org/browse/LVV-95}{LVV-95} - DMS-REQ-0197-V-01: No Limit on Data Access Centers

\end{itemize}

\subsubsection{Test Items}
Verify that additional Data Access Centers can be set up.


\subsubsection{Predecessors}

\subsubsection{Environment Needs}

\paragraph{Software}

\paragraph{Hardware}

\subsubsection{Input Specification}

\subsubsection{Output Specification}

\subsubsection{Test Procedure}
    \begin{longtable}[]{p{1.3cm}p{2cm}p{13cm}}
    %\toprule
    Step & \multicolumn{2}{@{}l}{Description, Input Data and Expected Result} \\ \toprule
    \endhead

            \multirow{3}{*}{ 1 } & Description &
            \begin{minipage}[t]{13cm}{\footnotesize
            Analyze design; instantiate and load simulated DAC, observe correct
functioning

            \vspace{\dp0}
            } \end{minipage} \\ \cline{2-3}
            & Test Data &
            \begin{minipage}[t]{13cm}{\footnotesize
                No data.
                \vspace{\dp0}
            } \end{minipage} \\ \cline{2-3}
            & Expected Result &
        \\ \midrule
    \end{longtable}

\subsection{LVV-T216 - Installation of the Alert Distribution payloads.}\label{lvv-t216}

\begin{longtable}[]{llllll}
\toprule
Version & Status & Priority & Verification Type & Owner
\\\midrule
1 & Approved & Normal &
Test & Eric Bellm
\\\bottomrule
\multicolumn{6}{c}{ Open \href{https://jira.lsstcorp.org/secure/Tests.jspa\#/testCase/LVV-T216}{LVV-T216} in Jira } \\
\end{longtable}

\subsubsection{Verification Elements}
\begin{itemize}
\item \href{https://jira.lsstcorp.org/browse/LVV-139}{LVV-139} - DMS-REQ-0308-V-01: Software Architecture to Enable Community Re-Use

\end{itemize}

\subsubsection{Test Items}
This test will check:\\

\begin{itemize}
\tightlist
\item
  That the Alert Distribution payloads are available from documented
  channels.
\item
  That the Alert Distribution payloads can be installed on LSST Data
  Facility-managed systems.
\item
  That the Alert Distribution payloads can be executed by LSST Data
  Facility-managed systems.
\end{itemize}


\subsubsection{Predecessors}

\subsubsection{Environment Needs}

\paragraph{Software}

\paragraph{Hardware}
This test case shall be executed on the Kubernetes Commons at the LDF.\\
As discussed in https://dmtn-028.lsst.io/ and https://dmtn-081.lsst.io/,
the test machine should have at least 16 cores, 64 GB of memory and
access to at least 1.5 TB of shared storage.

\subsubsection{Input Specification}

\subsubsection{Output Specification}

\subsubsection{Test Procedure}
    \begin{longtable}[]{p{1.3cm}p{2cm}p{13cm}}
    %\toprule
    Step & \multicolumn{2}{@{}l}{Description, Input Data and Expected Result} \\ \toprule
    \endhead

            \multirow{3}{*}{ 1 } & Description &
            \begin{minipage}[t]{13cm}{\footnotesize
            Download Kafka Docker image from
https://github.com/lsst-dm/alert\_stream.

            \vspace{\dp0}
            } \end{minipage} \\ \cline{2-3}
            & Test Data &
            \begin{minipage}[t]{13cm}{\footnotesize
                No data.
                \vspace{\dp0}
            } \end{minipage} \\ \cline{2-3}
            & Expected Result &
                \begin{minipage}[t]{13cm}{\footnotesize
                Runs without error

                \vspace{\dp0}
                } \end{minipage}
        \\ \midrule

            \multirow{3}{*}{ 2 } & Description &
            \begin{minipage}[t]{13cm}{\footnotesize
            Change to the alert\_stream directory and build the docker image.\\

\begin{verbatim}
docker build -t "lsst-kub001:5000/alert_stream"
\end{verbatim}

            \vspace{\dp0}
            } \end{minipage} \\ \cline{2-3}
            & Test Data &
            \begin{minipage}[t]{13cm}{\footnotesize
                No data.
                \vspace{\dp0}
            } \end{minipage} \\ \cline{2-3}
            & Expected Result &
                \begin{minipage}[t]{13cm}{\footnotesize
                Runs without error

                \vspace{\dp0}
                } \end{minipage}
        \\ \midrule

            \multirow{3}{*}{ 3 } & Description &
            \begin{minipage}[t]{13cm}{\footnotesize
            Register it with Kubernetes\\[2\baselineskip]docker push
lsst-kub001:5000/alert\_stream

            \vspace{\dp0}
            } \end{minipage} \\ \cline{2-3}
            & Test Data &
            \begin{minipage}[t]{13cm}{\footnotesize
                No data.
                \vspace{\dp0}
            } \end{minipage} \\ \cline{2-3}
            & Expected Result &
                \begin{minipage}[t]{13cm}{\footnotesize
                Runs without error

                \vspace{\dp0}
                } \end{minipage}
        \\ \midrule

            \multirow{3}{*}{ 4 } & Description &
            \begin{minipage}[t]{13cm}{\footnotesize
            From the alert\_stream/kubernetes directory, start Kafka and
Zookeeper:\\[2\baselineskip]

\begin{verbatim}
kubectl create -f zookeeper-service.yaml
kubectl create -f zookeeper-deployment.yaml
kubectl create -f kafka-deployment.yaml
kubectl create -f kafka-service.yaml
\end{verbatim}

(use kubectl get pods/services between each command to check status;
wait until each is ``Running'' before starting the next
command)\\[2\baselineskip]

            \vspace{\dp0}
            } \end{minipage} \\ \cline{2-3}
            & Test Data &
            \begin{minipage}[t]{13cm}{\footnotesize
                No data.
                \vspace{\dp0}
            } \end{minipage} \\ \cline{2-3}
            & Expected Result &
                \begin{minipage}[t]{13cm}{\footnotesize
                Runs without error

                \vspace{\dp0}
                } \end{minipage}
        \\ \midrule

            \multirow{3}{*}{ 5 } & Description &
            \begin{minipage}[t]{13cm}{\footnotesize
            Confirm Kafka and Zookeeper are listed when
running\\[2\baselineskip]kubectl get
pods\\[2\baselineskip]and\\[2\baselineskip]kubectl get services

            \vspace{\dp0}
            } \end{minipage} \\ \cline{2-3}
            & Test Data &
            \begin{minipage}[t]{13cm}{\footnotesize
                No data.
                \vspace{\dp0}
            } \end{minipage} \\ \cline{2-3}
            & Expected Result &
                \begin{minipage}[t]{13cm}{\footnotesize
                Output should be similar to:\\[2\baselineskip]kubectl get pods\\
NAME ~ ~ ~ ~ ~ ~ ~ ~ ~ ~ ~ ~READY ~ ~ STATUS ~ ~RESTARTS ~ AGE\\
kafka-768ddf5564-xwgvh ~ ~ ~1/1 ~ ~ ~ Running ~ 0 ~ ~ ~ ~ ~31s\\
zookeeper-f798cc548-mgkpn ~ 1/1 ~ ~ ~ Running ~ 0 ~ ~ ~ ~
~1m\\[2\baselineskip]kubectl get services\\
NAME ~ ~ ~ ~TYPE ~ ~ ~ ~CLUSTER-IP ~ ~ ~EXTERNAL-IP ~ PORT(S) ~ ~ AGE\\
kafka ~ ~ ~ ClusterIP ~ 10.105.19.124 ~ \textless{}none\textgreater{} ~
~ ~ ~9092/TCP ~ ~6s\\
zookeeper ~ ClusterIP ~ 10.97.110.124 ~ \textless{}none\textgreater{} ~
~ ~ ~32181/TCP ~ 2m

                \vspace{\dp0}
                } \end{minipage}
        \\ \midrule
    \end{longtable}

\subsection{LVV-T217 - Full Stream Alert Distribution}\label{lvv-t217}

\begin{longtable}[]{llllll}
\toprule
Version & Status & Priority & Verification Type & Owner
\\\midrule
1 & Approved & Normal &
Test & Eric Bellm
\\\bottomrule
\multicolumn{6}{c}{ Open \href{https://jira.lsstcorp.org/secure/Tests.jspa\#/testCase/LVV-T217}{LVV-T217} in Jira } \\
\end{longtable}

\subsubsection{Verification Elements}
\begin{itemize}
\item \href{https://jira.lsstcorp.org/browse/LVV-3}{LVV-3} - DMS-REQ-0002-V-01: Transient Alert Distribution

\end{itemize}

\subsubsection{Test Items}
This test will check that the full stream of LSST alerts can be
distributed to end users.\\[2\baselineskip]Specifically, this will
demonstrate that:

\begin{itemize}
\tightlist
\item
  Serialized alert packets can be loaded into the alert distribution
  system at LSST-relevant scales (10,000 alerts every 39 seconds);
\item
  Alert packets can be retrieved from the queue system at LSST-relevant
  scales.
\end{itemize}


\subsubsection{Predecessors}
\href{https://jira.lsstcorp.org/secure/Tests.jspa\#/testCase/LVV-T216}{LVV-T216}

\subsubsection{Environment Needs}

\paragraph{Software}
The Kafka cluster and Zookeeper shall be instantiated according to the
procedure described in
\href{https://jira.lsstcorp.org/secure/Tests.jspa\#/testCase/LVV-T216}{LVV-T216}.

\paragraph{Hardware}
This test case shall be executed on the Kubernetes Commons at the LDF.\\
As discussed in https://dmtn-028.lsst.io/ and https://dmtn-081.lsst.io/,
the test machine should have at least 16 cores, 64 GB of memory and
access to at least 1.5 TB of shared storage.

\subsubsection{Input Specification}
Input data: A sample of Avro-formatted alert packets.

\subsubsection{Output Specification}
Multiple Kafka consumers will run and write log files to disk.\\
The logs will include printing every \emph{Nth} alert to to the log as
well as a log summarizing the queue offset.

\subsubsection{Test Procedure}
    \begin{longtable}[]{p{1.3cm}p{2cm}p{13cm}}
    %\toprule
    Step & \multicolumn{2}{@{}l}{Description, Input Data and Expected Result} \\ \toprule
    \endhead

                \multirow{3}{*}{\parbox{1.3cm}{ 1-1
                {\scriptsize from \hyperref[lvv-t866]
                {LVV-T866} } } }

                & {\small Description} &
                \begin{minipage}[t]{13cm}{\scriptsize
                Perform the steps of Alert Production (including, but not necessarily
limited to, single frame processing, ISR, source detection/measurement,
PSF estimation, photometric and astrometric calibration, difference
imaging, DIASource detection/measurement, source association). During
Operations, it is presumed that these are automated for a given
dataset.~

                \vspace{\dp0}
                } \end{minipage} \\ \cdashline{2-3}
                & {\small Test Data} &
                \begin{minipage}[t]{13cm}{\scriptsize
                } \end{minipage} \\ \cdashline{2-3}
                & {\small Expected Result} &
                    \begin{minipage}[t]{13cm}{\scriptsize
                    An output dataset including difference images and DIASource and
DIAObject measurements.

                    \vspace{\dp0}
                    } \end{minipage}
                \\ \hdashline


                \multirow{3}{*}{\parbox{1.3cm}{ 1-2
                {\scriptsize from \hyperref[lvv-t866]
                {LVV-T866} } } }

                & {\small Description} &
                \begin{minipage}[t]{13cm}{\scriptsize
                Verify that the expected data products have been produced, and that
catalogs contain reasonable values for measured quantities of interest.

                \vspace{\dp0}
                } \end{minipage} \\ \cdashline{2-3}
                & {\small Test Data} &
                \begin{minipage}[t]{13cm}{\scriptsize
                } \end{minipage} \\ \cdashline{2-3}
                & {\small Expected Result} &
                \\ \hdashline


        \\ \midrule

            \multirow{3}{*}{ 2 } & Description &
            \begin{minipage}[t]{13cm}{\footnotesize
            Start a consumer that monitors the full stream and logs a deserialized
version of every Nth packet:\\

\begin{verbatim}
kubectl create -f consumerall-deployment.yaml
\end{verbatim}

            \vspace{\dp0}
            } \end{minipage} \\ \cline{2-3}
            & Test Data &
            \begin{minipage}[t]{13cm}{\footnotesize
                No data.
                \vspace{\dp0}
            } \end{minipage} \\ \cline{2-3}
            & Expected Result &
                \begin{minipage}[t]{13cm}{\footnotesize
                Runs without error

                \vspace{\dp0}
                } \end{minipage}
        \\ \midrule

            \multirow{3}{*}{ 3 } & Description &
            \begin{minipage}[t]{13cm}{\footnotesize
            \begin{verbatim}
Start a producer that reads alert packets from disk and loads them into the Kafka queue:
\end{verbatim}

\begin{verbatim}
kubectl create -f sender-deployment.yaml
\end{verbatim}

            \vspace{\dp0}
            } \end{minipage} \\ \cline{2-3}
            & Test Data &
            \begin{minipage}[t]{13cm}{\footnotesize
                No data.
                \vspace{\dp0}
            } \end{minipage} \\ \cline{2-3}
            & Expected Result &
                \begin{minipage}[t]{13cm}{\footnotesize
                Runs without error

                \vspace{\dp0}
                } \end{minipage}
        \\ \midrule

            \multirow{3}{*}{ 4 } & Description &
            \begin{minipage}[t]{13cm}{\footnotesize
            Determine the name of the alert sender pod with\\[2\baselineskip]kubectl
get pods\\[2\baselineskip]Examine output log
files.\\[2\baselineskip]kubectl logs \textless{}pod
name\textgreater{}\\[2\baselineskip]Verify that alerts are being sent
within 40 seconds by subtracting the timing measurements.

            \vspace{\dp0}
            } \end{minipage} \\ \cline{2-3}
            & Test Data &
            \begin{minipage}[t]{13cm}{\footnotesize
                No data.
                \vspace{\dp0}
            } \end{minipage} \\ \cline{2-3}
            & Expected Result &
                \begin{minipage}[t]{13cm}{\footnotesize
                Similar to\\[2\baselineskip]kubectl logs sender-7d6f98586f-nhwfj\\
visit: 1570. ~ ~ time: 1530588618.0313473\\
visits finished: 1 ~ ~ ~time: 1530588653.5614944\\
visit: 1571. ~ ~ time: 1530588657.0087624\\
visits finished: 2 ~ ~ ~time: 1530588692.506188\\
visit: 1572. ~ ~ time: 1530588696.0051727\\
visits finished: 3 ~ ~ ~time: 1530588731.5900314\\[3\baselineskip]

                \vspace{\dp0}
                } \end{minipage}
        \\ \midrule

            \multirow{3}{*}{ 5 } & Description &
            \begin{minipage}[t]{13cm}{\footnotesize
            Determine the name of the consumer pod with\\[2\baselineskip]kubectl get
pods\\[2\baselineskip]Examine output log files.\\[2\baselineskip]kubectl
logs \textless{}pod name\textgreater{}\\[2\baselineskip]The packet log
should show deserialized alert packets with contents matching the input
packets.\\[2\baselineskip]

            \vspace{\dp0}
            } \end{minipage} \\ \cline{2-3}
            & Test Data &
            \begin{minipage}[t]{13cm}{\footnotesize
                No data.
                \vspace{\dp0}
            } \end{minipage} \\ \cline{2-3}
            & Expected Result &
                \begin{minipage}[t]{13cm}{\footnotesize
                Similar to \{'alertId': 12132024420, `l1dbId': 71776805594116,
`diaSource': \{'diaSourceId':\\
73499448928374785, `ccdVisitId': 2020011570, `diaObjectId':
71776805594116, 'ssO\\
bjectId': None, `parentDiaSourceId': None, `midPointTai': 59595.37041,
'filterNa\\
me': `y', `ra': 172.24912810036074, `decl': -80.64214929176521,
`ra\_decl\_Cov': \{\\
`raSigma': 0.0003428002819418907, `declSigma': 0.00027273103478364646,
'ra\_decl\_\\
Cov': 0.000628734880592674\}, `x': 2979.08837890625, `y':
3843.328857421875, 'x\_y\\
\_Cov': \{'xSigma': 0.6135467886924744, `ySigma': 0.77132648229599,
`x\_y\_Cov': 0.0\\
007463791407644749\}, `apFlux': None, `apFluxErr': None, `snr':
0.366516500711441\\
04, `psFlux': 7.698232025177276e-07, `psRa': None, `psDecl': None,
`ps\_Cov': Non\\
e, `psLnL': None, `psChi2': None, `psNdata': None, `trailFlux': None,
`trailRa':\\
etc.

                \vspace{\dp0}
                } \end{minipage}
        \\ \midrule
    \end{longtable}

\subsection{LVV-T218 - Simple Filtering of the LSST Alert Stream}\label{lvv-t218}

\begin{longtable}[]{llllll}
\toprule
Version & Status & Priority & Verification Type & Owner
\\\midrule
1 & Approved & Normal &
Test & Eric Bellm
\\\bottomrule
\multicolumn{6}{c}{ Open \href{https://jira.lsstcorp.org/secure/Tests.jspa\#/testCase/LVV-T218}{LVV-T218} in Jira } \\
\end{longtable}

\subsubsection{Verification Elements}
\begin{itemize}
\item \href{https://jira.lsstcorp.org/browse/LVV-173}{LVV-173} - DMS-REQ-0342-V-01: Alert Filtering Service

\item \href{https://jira.lsstcorp.org/browse/LVV-179}{LVV-179} - DMS-REQ-0348-V-01: Pre-defined alert filters

\item \href{https://jira.lsstcorp.org/browse/LVV-174}{LVV-174} - DMS-REQ-0343-V-01: Number of full-size alerts

\end{itemize}

\subsubsection{Test Items}
This test will demonstrate the LSST Alert Filtering Service that returns
a subset of alerts from the full stream identified by user-provided
filters.\\[2\baselineskip]Specifically, this will demonstrate that:\\

\begin{itemize}
\tightlist
\item
  The filtering service can retrieve alerts from the full alert stream
  and filter them according to their contents; ~ ~
\item
  The filtered subset can be delivered to science users.
\end{itemize}


\subsubsection{Predecessors}
​\href{https://jira.lsstcorp.org/secure/Tests.jspa\#/testCase/LVV-T216}{LVV-T216}​​​\\
​\href{https://jira.lsstcorp.org/secure/Tests.jspa\#/testCase/LVV-T217}{LVV-T217}​​​

\subsubsection{Environment Needs}

\paragraph{Software}
The Kafka cluster and Zookeeper shall be instantiated according to the
procedure described in
\href{https://jira.lsstcorp.org/secure/Tests.jspa\#/testCase/LVV-T216}{LVV-T216}.

\paragraph{Hardware}
This test case shall be executed on the Kubernetes Commons at the LDF.\\
As discussed in https://dmtn-028.lsst.io/ and https://dmtn-081.lsst.io/,
the test machine should have at least 16 cores, 64 GB of memory and
access to at least 1.5 TB of shared storage.

\subsubsection{Input Specification}
Input data: A sample of Avro-formatted alert packets derived from LSST
simulations corresponding to one night of simulated LSST observing.

\subsubsection{Output Specification}

\subsubsection{Test Procedure}
    \begin{longtable}[]{p{1.3cm}p{2cm}p{13cm}}
    %\toprule
    Step & \multicolumn{2}{@{}l}{Description, Input Data and Expected Result} \\ \toprule
    \endhead

                \multirow{3}{*}{\parbox{1.3cm}{ 1-1
                {\scriptsize from \hyperref[lvv-t216]
                {LVV-T216} } } }

                & {\small Description} &
                \begin{minipage}[t]{13cm}{\scriptsize
                Download Kafka Docker image from
https://github.com/lsst-dm/alert\_stream.

                \vspace{\dp0}
                } \end{minipage} \\ \cdashline{2-3}
                & {\small Test Data} &
                \begin{minipage}[t]{13cm}{\scriptsize
                } \end{minipage} \\ \cdashline{2-3}
                & {\small Expected Result} &
                    \begin{minipage}[t]{13cm}{\scriptsize
                    Runs without error

                    \vspace{\dp0}
                    } \end{minipage}
                \\ \hdashline


                \multirow{3}{*}{\parbox{1.3cm}{ 1-2
                {\scriptsize from \hyperref[lvv-t216]
                {LVV-T216} } } }

                & {\small Description} &
                \begin{minipage}[t]{13cm}{\scriptsize
                Change to the alert\_stream directory and build the docker image.\\

\begin{verbatim}
docker build -t "lsst-kub001:5000/alert_stream"
\end{verbatim}

                \vspace{\dp0}
                } \end{minipage} \\ \cdashline{2-3}
                & {\small Test Data} &
                \begin{minipage}[t]{13cm}{\scriptsize
                } \end{minipage} \\ \cdashline{2-3}
                & {\small Expected Result} &
                    \begin{minipage}[t]{13cm}{\scriptsize
                    Runs without error

                    \vspace{\dp0}
                    } \end{minipage}
                \\ \hdashline


                \multirow{3}{*}{\parbox{1.3cm}{ 1-3
                {\scriptsize from \hyperref[lvv-t216]
                {LVV-T216} } } }

                & {\small Description} &
                \begin{minipage}[t]{13cm}{\scriptsize
                Register it with Kubernetes\\[2\baselineskip]docker push
lsst-kub001:5000/alert\_stream

                \vspace{\dp0}
                } \end{minipage} \\ \cdashline{2-3}
                & {\small Test Data} &
                \begin{minipage}[t]{13cm}{\scriptsize
                } \end{minipage} \\ \cdashline{2-3}
                & {\small Expected Result} &
                    \begin{minipage}[t]{13cm}{\scriptsize
                    Runs without error

                    \vspace{\dp0}
                    } \end{minipage}
                \\ \hdashline


                \multirow{3}{*}{\parbox{1.3cm}{ 1-4
                {\scriptsize from \hyperref[lvv-t216]
                {LVV-T216} } } }

                & {\small Description} &
                \begin{minipage}[t]{13cm}{\scriptsize
                From the alert\_stream/kubernetes directory, start Kafka and
Zookeeper:\\[2\baselineskip]

\begin{verbatim}
kubectl create -f zookeeper-service.yaml
kubectl create -f zookeeper-deployment.yaml
kubectl create -f kafka-deployment.yaml
kubectl create -f kafka-service.yaml
\end{verbatim}

(use kubectl get pods/services between each command to check status;
wait until each is ``Running'' before starting the next
command)\\[2\baselineskip]

                \vspace{\dp0}
                } \end{minipage} \\ \cdashline{2-3}
                & {\small Test Data} &
                \begin{minipage}[t]{13cm}{\scriptsize
                } \end{minipage} \\ \cdashline{2-3}
                & {\small Expected Result} &
                    \begin{minipage}[t]{13cm}{\scriptsize
                    Runs without error

                    \vspace{\dp0}
                    } \end{minipage}
                \\ \hdashline


                \multirow{3}{*}{\parbox{1.3cm}{ 1-5
                {\scriptsize from \hyperref[lvv-t216]
                {LVV-T216} } } }

                & {\small Description} &
                \begin{minipage}[t]{13cm}{\scriptsize
                Confirm Kafka and Zookeeper are listed when
running\\[2\baselineskip]kubectl get
pods\\[2\baselineskip]and\\[2\baselineskip]kubectl get services

                \vspace{\dp0}
                } \end{minipage} \\ \cdashline{2-3}
                & {\small Test Data} &
                \begin{minipage}[t]{13cm}{\scriptsize
                } \end{minipage} \\ \cdashline{2-3}
                & {\small Expected Result} &
                    \begin{minipage}[t]{13cm}{\scriptsize
                    Output should be similar to:\\[2\baselineskip]kubectl get pods\\
NAME ~ ~ ~ ~ ~ ~ ~ ~ ~ ~ ~ ~READY ~ ~ STATUS ~ ~RESTARTS ~ AGE\\
kafka-768ddf5564-xwgvh ~ ~ ~1/1 ~ ~ ~ Running ~ 0 ~ ~ ~ ~ ~31s\\
zookeeper-f798cc548-mgkpn ~ 1/1 ~ ~ ~ Running ~ 0 ~ ~ ~ ~
~1m\\[2\baselineskip]kubectl get services\\
NAME ~ ~ ~ ~TYPE ~ ~ ~ ~CLUSTER-IP ~ ~ ~EXTERNAL-IP ~ PORT(S) ~ ~ AGE\\
kafka ~ ~ ~ ClusterIP ~ 10.105.19.124 ~ \textless{}none\textgreater{} ~
~ ~ ~9092/TCP ~ ~6s\\
zookeeper ~ ClusterIP ~ 10.97.110.124 ~ \textless{}none\textgreater{} ~
~ ~ ~32181/TCP ~ 2m

                    \vspace{\dp0}
                    } \end{minipage}
                \\ \hdashline


        \\ \midrule

            \multirow{3}{*}{ 2 } & Description &
            \begin{minipage}[t]{13cm}{\footnotesize
            Start 100 consumers that consume the filtered streams and logs a
deserialized version of every Nth packet:\\[2\baselineskip]

\begin{verbatim}
kubectl create -f consumer1-deployment.yaml
kubectl create -f consumer2-deployment.yaml
kubectl create -f consumer3-deployment.yaml
kubectl create -f consumer4-deployment.yaml
kubectl create -f consumer5-deployment.yaml
kubectl create -f consumer6-deployment.yaml
kubectl create -f consumer7-deployment.yaml
kubectl create -f consumer8-deployment.yaml
kubectl create -f consumer9-deployment.yaml
kubectl create -f consumer10-deployment.yaml
\end{verbatim}

            \vspace{\dp0}
            } \end{minipage} \\ \cline{2-3}
            & Test Data &
            \begin{minipage}[t]{13cm}{\footnotesize
                No data.
                \vspace{\dp0}
            } \end{minipage} \\ \cline{2-3}
            & Expected Result &
                \begin{minipage}[t]{13cm}{\footnotesize
                Runs without error

                \vspace{\dp0}
                } \end{minipage}
        \\ \midrule

            \multirow{3}{*}{ 3 } & Description &
            \begin{minipage}[t]{13cm}{\footnotesize
            Start 5 filter groups:\\

\begin{verbatim}
kubectl create -f filterer1-deployment.yaml
kubectl create -f filterer2-deployment.yaml
kubectl create -f filterer3-deployment.yaml
kubectl create -f filterer4-deployment.yaml
kubectl create -f filterer5-deployment.yaml
\end{verbatim}

            \vspace{\dp0}
            } \end{minipage} \\ \cline{2-3}
            & Test Data &
            \begin{minipage}[t]{13cm}{\footnotesize
                No data.
                \vspace{\dp0}
            } \end{minipage} \\ \cline{2-3}
            & Expected Result &
                \begin{minipage}[t]{13cm}{\footnotesize
                Runs without error

                \vspace{\dp0}
                } \end{minipage}
        \\ \midrule

            \multirow{3}{*}{ 4 } & Description &
            \begin{minipage}[t]{13cm}{\footnotesize
            Start a producer that reads alert packets from disk and loads them into
the Kafka queue:\\[2\baselineskip]

\begin{verbatim}
kubectl create -f sender-deployment.yaml
\end{verbatim}

            \vspace{\dp0}
            } \end{minipage} \\ \cline{2-3}
            & Test Data &
            \begin{minipage}[t]{13cm}{\footnotesize
                No data.
                \vspace{\dp0}
            } \end{minipage} \\ \cline{2-3}
            & Expected Result &
                \begin{minipage}[t]{13cm}{\footnotesize
                Runs without error

                \vspace{\dp0}
                } \end{minipage}
        \\ \midrule

            \multirow{3}{*}{ 5 } & Description &
            \begin{minipage}[t]{13cm}{\footnotesize
            Determine the name of the alert sender pod with\\[2\baselineskip]kubectl
get pods\\[2\baselineskip]Examine output log
files.\\[2\baselineskip]kubectl logs \textless{}pod
name\textgreater{}\\[2\baselineskip]Verify that alerts are being sent
within 40 seconds by subtracting the timing measurements.

            \vspace{\dp0}
            } \end{minipage} \\ \cline{2-3}
            & Test Data &
            \begin{minipage}[t]{13cm}{\footnotesize
                No data.
                \vspace{\dp0}
            } \end{minipage} \\ \cline{2-3}
            & Expected Result &
                \begin{minipage}[t]{13cm}{\footnotesize
                Similar to\\[2\baselineskip]kubectl logs sender-7d6f98586f-nhwfj\\
visit: 1570. ~ ~ time: 1530588618.0313473\\
visits finished: 1 ~ ~ ~time: 1530588653.5614944\\
visit: 1571. ~ ~ time: 1530588657.0087624\\
visits finished: 2 ~ ~ ~time: 1530588692.506188\\
visit: 1572. ~ ~ time: 1530588696.0051727\\
visits finished: 3 ~ ~ ~time: 1530588731.5900314\\[2\baselineskip]

                \vspace{\dp0}
                } \end{minipage}
        \\ \midrule

            \multirow{3}{*}{ 6 } & Description &
            \begin{minipage}[t]{13cm}{\footnotesize
            Determine the name of the consumer pods with\\[2\baselineskip]kubectl
get pods\\[2\baselineskip]Examine output log
files.\\[2\baselineskip]kubectl logs \textless{}pod
name\textgreater{}\\[2\baselineskip]The packet log should show
deserialized alert packets with contents matching the input packets.

            \vspace{\dp0}
            } \end{minipage} \\ \cline{2-3}
            & Test Data &
            \begin{minipage}[t]{13cm}{\footnotesize
                No data.
                \vspace{\dp0}
            } \end{minipage} \\ \cline{2-3}
            & Expected Result &
                \begin{minipage}[t]{13cm}{\footnotesize
                Similar to\\[2\baselineskip]\{'alertId': 12132024420, `l1dbId':
71776805594116, `diaSource': \{'diaSourceId':\\
73499448928374785, `ccdVisitId': 2020011570, `diaObjectId':
71776805594116, 'ssO\\
bjectId': None, `parentDiaSourceId': None, `midPointTai': 59595.37041,
'filterNa\\
me': `y', `ra': 172.24912810036074, `decl': -80.64214929176521,
`ra\_decl\_Cov': \{\\
`raSigma': 0.0003428002819418907, `declSigma': 0.00027273103478364646,
'ra\_decl\_\\
Cov': 0.000628734880592674\}, `x': 2979.08837890625, `y':
3843.328857421875, 'x\_y\\
\_Cov': \{'xSigma': 0.6135467886924744, `ySigma': 0.77132648229599,
`x\_y\_Cov': 0.0\\
007463791407644749\}, `apFlux': None, `apFluxErr': None, `snr':
0.366516500711441\\
04, `psFlux': 7.698232025177276e-07, `psRa': None, `psDecl': None,
`ps\_Cov': Non\\
e, `psLnL': None, `psChi2': None, `psNdata': None, `trailFlux': None,
`trailRa':\\
etc.

                \vspace{\dp0}
                } \end{minipage}
        \\ \midrule
    \end{longtable}

\subsection{LVV-T283 - RAS-00-00: Writing well-formed raw image}\label{lvv-t283}

\begin{longtable}[]{llllll}
\toprule
Version & Status & Priority & Verification Type & Owner
\\\midrule
1 & Approved & Normal &
Test & Michelle Butler
\\\bottomrule
\multicolumn{6}{c}{ Open \href{https://jira.lsstcorp.org/secure/Tests.jspa\#/testCase/LVV-T283}{LVV-T283} in Jira } \\
\end{longtable}

\subsubsection{Verification Elements}
\begin{itemize}
\item \href{https://jira.lsstcorp.org/browse/LVV-8}{LVV-8} - DMS-REQ-0018-V-01: Raw Science Image Data Acquisition

\item \href{https://jira.lsstcorp.org/browse/LVV-9}{LVV-9} - DMS-REQ-0020-V-01: Wavefront Sensor Data Acquisition

\item \href{https://jira.lsstcorp.org/browse/LVV-96}{LVV-96} - DMS-REQ-0265-V-01: Guider Calibration Data Acquisition

\item \href{https://jira.lsstcorp.org/browse/LVV-28}{LVV-28} - DMS-REQ-0068-V-01: Raw Science Image Metadata

\item \href{https://jira.lsstcorp.org/browse/LVV-11}{LVV-11} - DMS-REQ-0024-V-01: Raw Image Assembly

\item \href{https://jira.lsstcorp.org/browse/LVV-146}{LVV-146} - DMS-REQ-0315-V-01: DMS Communication with OCS

\item \href{https://jira.lsstcorp.org/browse/LVV-115}{LVV-115} - DMS-REQ-0284-V-01: Level-1 Production Completeness

\end{itemize}

\subsubsection{Test Items}
This test will check:\\

\begin{itemize}
\tightlist
\item
  The successful integration of the Pathfinder components with the DM
  Header Service and the Level 1 Archiver;
\item
  That the raw images are well-formed and meet specifications in
  change-controlled documents \citeds{LSE-61};
\end{itemize}

~This Test Case shall be repeated for each of the different cameras
(ATScam, LSSTCam) and sensors (Science, Wavefront, and Guider)
combination.


\subsubsection{Predecessors}
None.

\subsubsection{Environment Needs}

\paragraph{Software}
\begin{itemize}
\tightlist
\item
  Level 1 software and services needed to create raw image
\item
  LSST Monitoring Service and plugins specific to monitoring Level 1
  Test Stand and services
\end{itemize}

\paragraph{Hardware}
\begin{itemize}
\tightlist
\item
  Level 1 test stand
\item
  Test machine for LSST Monitoring Service
\end{itemize}

\subsubsection{Input Specification}
None.

\subsubsection{Output Specification}
Raw image(s) that follow specifications defined in change-controlled
document \citeds{LSE-61}.

\subsubsection{Test Procedure}
    \begin{longtable}[]{p{1.3cm}p{2cm}p{13cm}}
    %\toprule
    Step & \multicolumn{2}{@{}l}{Description, Input Data and Expected Result} \\ \toprule
    \endhead

            \multirow{3}{*}{ 1 } & Description &
            \begin{minipage}[t]{13cm}{\footnotesize
            Configure system to pull appropriate data from the DAQ emulator

            \vspace{\dp0}
            } \end{minipage} \\ \cline{2-3}
            & Test Data &
            \begin{minipage}[t]{13cm}{\footnotesize
                No data.
                \vspace{\dp0}
            } \end{minipage} \\ \cline{2-3}
            & Expected Result &
                \begin{minipage}[t]{13cm}{\footnotesize
                A functional DAQ for images to be received from.~~

                \vspace{\dp0}
                } \end{minipage}
        \\ \midrule

            \multirow{3}{*}{ 2 } & Description &
            \begin{minipage}[t]{13cm}{\footnotesize
            Acquire raw data from DAQ readout and DMHS

            \vspace{\dp0}
            } \end{minipage} \\ \cline{2-3}
            & Test Data &
            \begin{minipage}[t]{13cm}{\footnotesize
                No data.
                \vspace{\dp0}
            } \end{minipage} \\ \cline{2-3}
            & Expected Result &
                \begin{minipage}[t]{13cm}{\footnotesize
                a raw image and a header from the DMHS~

                \vspace{\dp0}
                } \end{minipage}
        \\ \midrule

            \multirow{3}{*}{ 3 } & Description &
            \begin{minipage}[t]{13cm}{\footnotesize
            Fetch data and reassemble correctly, regardless of CCD/Sensor
manufacturer type (two different types will be used)

            \vspace{\dp0}
            } \end{minipage} \\ \cline{2-3}
            & Test Data &
            \begin{minipage}[t]{13cm}{\footnotesize
                No data.
                \vspace{\dp0}
            } \end{minipage} \\ \cline{2-3}
            & Expected Result &
                \begin{minipage}[t]{13cm}{\footnotesize
                Build the data into a fits file

                \vspace{\dp0}
                } \end{minipage}
        \\ \midrule

            \multirow{3}{*}{ 4 } & Description &
            \begin{minipage}[t]{13cm}{\footnotesize
            Check completeness and correctness of the raw images including format,
metadata, and image data;

\begin{itemize}
\tightlist
\item
  Check proper fetch and reassembly of image data from camera DAQ
  (correct format and data);
\item
  Check proper merge of header service data with image data;
\item
  Check correct insertion of exposure specific data needed in the data
  file that is not supplied by header service;
\item
  Check minimum required metadata (from requirements document LSE-61)
  exists in raw image header;
\end{itemize}

            \vspace{\dp0}
            } \end{minipage} \\ \cline{2-3}
            & Test Data &
            \begin{minipage}[t]{13cm}{\footnotesize
                No data.
                \vspace{\dp0}
            } \end{minipage} \\ \cline{2-3}
            & Expected Result &
                \begin{minipage}[t]{13cm}{\footnotesize
                a well formed FITS file with a proper header that has been verified to
be correct.~

                \vspace{\dp0}
                } \end{minipage}
        \\ \midrule

            \multirow{3}{*}{ 5 } & Description &
            \begin{minipage}[t]{13cm}{\footnotesize
            Check that the checksum of the file matches the previously calculated
value that will be passed on to downstream services

            \vspace{\dp0}
            } \end{minipage} \\ \cline{2-3}
            & Test Data &
            \begin{minipage}[t]{13cm}{\footnotesize
                No data.
                \vspace{\dp0}
            } \end{minipage} \\ \cline{2-3}
            & Expected Result &
                \begin{minipage}[t]{13cm}{\footnotesize
                a MD5sum number generated from the step 4 file.~~

                \vspace{\dp0}
                } \end{minipage}
        \\ \midrule

            \multirow{3}{*}{ 6 } & Description &
            \begin{minipage}[t]{13cm}{\footnotesize
            Check confirmation that the data files arrive at their destination
intact

            \vspace{\dp0}
            } \end{minipage} \\ \cline{2-3}
            & Test Data &
            \begin{minipage}[t]{13cm}{\footnotesize
                No data.
                \vspace{\dp0}
            } \end{minipage} \\ \cline{2-3}
            & Expected Result &
                \begin{minipage}[t]{13cm}{\footnotesize
                a transfer of the file to the correct location for further retrieval
from other services.~~

                \vspace{\dp0}
                } \end{minipage}
        \\ \midrule

            \multirow{3}{*}{ 7 } & Description &
            \begin{minipage}[t]{13cm}{\footnotesize
            Check that LSST Monitoring Service showed the appropriate information
successfully

            \vspace{\dp0}
            } \end{minipage} \\ \cline{2-3}
            & Test Data &
            \begin{minipage}[t]{13cm}{\footnotesize
                No data.
                \vspace{\dp0}
            } \end{minipage} \\ \cline{2-3}
            & Expected Result &
                \begin{minipage}[t]{13cm}{\footnotesize
                all systems remained green through out the test, and showed all systems
up and available. ~\\[2\baselineskip]

                \vspace{\dp0}
                } \end{minipage}
        \\ \midrule
    \end{longtable}

\subsection{LVV-T284 - RAS-00-05: (LDM-503-8b) Writing data from CCOB to the DBB for further
data processing}\label{lvv-t284}

\begin{longtable}[]{llllll}
\toprule
Version & Status & Priority & Verification Type & Owner
\\\midrule
1 & Draft & Normal &
Test & Michelle Butler
\\\bottomrule
\multicolumn{6}{c}{ Open \href{https://jira.lsstcorp.org/secure/Tests.jspa\#/testCase/LVV-T284}{LVV-T284} in Jira } \\
\end{longtable}

\subsubsection{Verification Elements}
\begin{itemize}
\item \href{https://jira.lsstcorp.org/browse/LVV-9}{LVV-9} - DMS-REQ-0020-V-01: Wavefront Sensor Data Acquisition

\item \href{https://jira.lsstcorp.org/browse/LVV-8}{LVV-8} - DMS-REQ-0018-V-01: Raw Science Image Data Acquisition

\item \href{https://jira.lsstcorp.org/browse/LVV-96}{LVV-96} - DMS-REQ-0265-V-01: Guider Calibration Data Acquisition

\item \href{https://jira.lsstcorp.org/browse/LVV-28}{LVV-28} - DMS-REQ-0068-V-01: Raw Science Image Metadata

\item \href{https://jira.lsstcorp.org/browse/LVV-11}{LVV-11} - DMS-REQ-0024-V-01: Raw Image Assembly

\item \href{https://jira.lsstcorp.org/browse/LVV-146}{LVV-146} - DMS-REQ-0315-V-01: DMS Communication with OCS

\item \href{https://jira.lsstcorp.org/browse/LVV-115}{LVV-115} - DMS-REQ-0284-V-01: Level-1 Production Completeness

\end{itemize}

\subsubsection{Test Items}
This test will check:

\begin{itemize}
\tightlist
\item
  The successful integration of the DAQ archiver components with the
  CCOB
\item
  That the file can then be ingested into the DBB and be retrieved for
  further analysis
\end{itemize}


\subsubsection{Predecessors}
None.

\subsubsection{Environment Needs}

\paragraph{Software}
\begin{itemize}
\tightlist
\item
  CCOB device and the software to produce a file to be transferred and
  kept
\item
  DBB software to produce a retrieval file for further processing
\end{itemize}

\paragraph{Hardware}
\begin{itemize}
\tightlist
\item
  CCOB
\item
  Test machine for LSST Monitoring Service
\item
  consolidate DB~
\item
  DBB ingest file system~
\item
  DBB output file system~
\item
  data transfer protocol to move data from CCOB file systems to DBB
  ingest file system~
\end{itemize}

\subsubsection{Input Specification}
None.

\subsubsection{Output Specification}
\begin{itemize}
\tightlist
\item
  CCOB (raw image) files that follow specifications;
\item
  DBB files that follow specifications;
\item
  CCOB device directs a human to where a file is wanted to be stored in
  the DBB;
\item
  Transfer the file to the DBB ingest area;
\end{itemize}

\subsubsection{Test Procedure}
    \begin{longtable}[]{p{1.3cm}p{2cm}p{13cm}}
    %\toprule
    Step & \multicolumn{2}{@{}l}{Description, Input Data and Expected Result} \\ \toprule
    \endhead

            \multirow{3}{*}{ 1 } & Description &
            \begin{minipage}[t]{13cm}{\footnotesize
            CCOB device directs a human to where a raw file is wanted to be stored
in the DBB

            \vspace{\dp0}
            } \end{minipage} \\ \cline{2-3}
            & Test Data &
            \begin{minipage}[t]{13cm}{\footnotesize
                No data.
                \vspace{\dp0}
            } \end{minipage} \\ \cline{2-3}
            & Expected Result &
                \begin{minipage}[t]{13cm}{\footnotesize
                A file with a unique file name is in a file system somewhere, and the
data is then transferred to NCSA.~ ~

                \vspace{\dp0}
                } \end{minipage}
        \\ \midrule

            \multirow{3}{*}{ 2 } & Description &
            \begin{minipage}[t]{13cm}{\footnotesize
            Move the data from the transferred directory into the DBB foreign file
ingest file system. ~

            \vspace{\dp0}
            } \end{minipage} \\ \cline{2-3}
            & Test Data &
            \begin{minipage}[t]{13cm}{\footnotesize
                No data.
                \vspace{\dp0}
            } \end{minipage} \\ \cline{2-3}
            & Expected Result &
                \begin{minipage}[t]{13cm}{\footnotesize
                A command is executed by a human with a file name and path to the file
wanted to be stored in the DBB.~ The file is transferred to NCSA's DBB
ingest area.~ ~~

                \vspace{\dp0}
                } \end{minipage}
        \\ \midrule

            \multirow{3}{*}{ 3 } & Description &
            \begin{minipage}[t]{13cm}{\footnotesize
            Have data inspected by scientist for managing that all data was
transferred.~ ~

            \vspace{\dp0}
            } \end{minipage} \\ \cline{2-3}
            & Test Data &
            \begin{minipage}[t]{13cm}{\footnotesize
                No data.
                \vspace{\dp0}
            } \end{minipage} \\ \cline{2-3}
            & Expected Result &
                \begin{minipage}[t]{13cm}{\footnotesize
                a specific Okay to move forward; or something is
broke.\\[2\baselineskip]

                \vspace{\dp0}
                } \end{minipage}
        \\ \midrule

            \multirow{3}{*}{ 4 } & Description &
            \begin{minipage}[t]{13cm}{\footnotesize
            The DBB is notified of a new file being in the ingest area, and the DBB
ingest is run manually to ingest the CCOB file.~ ~

            \vspace{\dp0}
            } \end{minipage} \\ \cline{2-3}
            & Test Data &
            \begin{minipage}[t]{13cm}{\footnotesize
                No data.
                \vspace{\dp0}
            } \end{minipage} \\ \cline{2-3}
            & Expected Result &
                \begin{minipage}[t]{13cm}{\footnotesize
                The DBB puts the resulting file into the DBB file systems depending on
what type of file it is. ~The DB is updated with metadata and providence
of the file to be kept. ~ The resulting file system is queryable by the
LSP to find the CCOB raw image.~~

                \vspace{\dp0}
                } \end{minipage}
        \\ \midrule

            \multirow{3}{*}{ 5 } & Description &
            \begin{minipage}[t]{13cm}{\footnotesize
            The LSP can review and use the CCOB raw data file that was stored
originally somewhere else such as slac

            \vspace{\dp0}
            } \end{minipage} \\ \cline{2-3}
            & Test Data &
            \begin{minipage}[t]{13cm}{\footnotesize
                No data.
                \vspace{\dp0}
            } \end{minipage} \\ \cline{2-3}
            & Expected Result &
                \begin{minipage}[t]{13cm}{\footnotesize
                LSP has the ability to find the file and view/use it.
~\\[2\baselineskip]

                \vspace{\dp0}
                } \end{minipage}
        \\ \midrule
    \end{longtable}

\subsection{LVV-T285 - RAS-00-10: Raw images in Observatory Operations Data Service}\label{lvv-t285}

\begin{longtable}[]{llllll}
\toprule
Version & Status & Priority & Verification Type & Owner
\\\midrule
1 & Approved & Normal &
Test & Michelle Butler
\\\bottomrule
\multicolumn{6}{c}{ Open \href{https://jira.lsstcorp.org/secure/Tests.jspa\#/testCase/LVV-T285}{LVV-T285} in Jira } \\
\end{longtable}

\subsubsection{Verification Elements}
    None.

\subsubsection{Test Items}
This test will check:

\begin{itemize}
\tightlist
\item
  The handoff of a raw image from the Level 1 Archiver to the OODS cache
  manager is successful;
\item
  A recently taken raw image is accessible to the Observatory Operations
  staff at the base and summit;
\end{itemize}

~This Test Case shall be repeated for each of the different cameras
(ATScam, LSSTCam) and sensors (Science, Wavefront, and Guider)
combination.


\subsubsection{Predecessors}
LVV-T283

\subsubsection{Environment Needs}

\paragraph{Software}
The following software must be installed:\\[2\baselineskip]

\begin{itemize}
\tightlist
\item
  Level 1 Test Stand (include software from LVV-T283 - RAS-00-00)
\item
  OODS cache manager
\item
  LSST Monitoring Service and plugins specific to monitoring raw images
  and OODS~
\item
  LSST stack for checking raw images
\end{itemize}

\paragraph{Hardware}
To complete all tests in a manner which reflects the real system, the
following hardware is needed. Note: If not testing inter-machine access,
the hardware can be minimized to a single machine outside of the Level 1
Test Stand.

\begin{itemize}
\tightlist
\item
  Level1TestStand(include hardware from LVV-T283 - RAS-00-00)+read/write
  access to OODS cache disk
\item
  Test Machine for OODS cache manager with read/write access to OODS
  cache disk
\item
  Test machine for Observatory Operations staff at ''base'' that can
  access OODS cache disk
\item
  Test machine for Observatory Operations staff at ''summit'' that can
  access OODS cache disk
\item
  Test machine for LSST Monitoring Service
\end{itemize}

Size of cache disk is determined by number of files to be included in
the test.

\subsubsection{Input Specification}

\subsubsection{Output Specification}
Raw image(s) that follow format defined in \citeds{LSE-61};\\
Database (may be SQLite file) that enables the raw image(s) to be
accessed via a ``Data Butler''.

\subsubsection{Test Procedure}
    \begin{longtable}[]{p{1.3cm}p{2cm}p{13cm}}
    %\toprule
    Step & \multicolumn{2}{@{}l}{Description, Input Data and Expected Result} \\ \toprule
    \endhead

            \multirow{3}{*}{ 1 } & Description &
            \begin{minipage}[t]{13cm}{\footnotesize
            Initialize all services configuring the Level 1 Archiver Service so that
the raw images are to be saved to the OODS

            \vspace{\dp0}
            } \end{minipage} \\ \cline{2-3}
            & Test Data &
            \begin{minipage}[t]{13cm}{\footnotesize
                No data.
                \vspace{\dp0}
            } \end{minipage} \\ \cline{2-3}
            & Expected Result &
                \begin{minipage}[t]{13cm}{\footnotesize
                all camera and services for images are running and reporting green
through the monitoring programs for the services. ~\\[2\baselineskip]

                \vspace{\dp0}
                } \end{minipage}
        \\ \midrule

            \multirow{3}{*}{ 2 } & Description &
            \begin{minipage}[t]{13cm}{\footnotesize
            Acquire a raw image

            \vspace{\dp0}
            } \end{minipage} \\ \cline{2-3}
            & Test Data &
            \begin{minipage}[t]{13cm}{\footnotesize
                No data.
                \vspace{\dp0}
            } \end{minipage} \\ \cline{2-3}
            & Expected Result &
                \begin{minipage}[t]{13cm}{\footnotesize
                Image present in the input folder.

                \vspace{\dp0}
                } \end{minipage}
        \\ \midrule

            \multirow{3}{*}{ 3 } & Description &
            \begin{minipage}[t]{13cm}{\footnotesize
            \emph{The handoff of the raw image from the Level 1 Archiver Service to
the test OODS automatically occurs\\
}

            \vspace{\dp0}
            } \end{minipage} \\ \cline{2-3}
            & Test Data &
            \begin{minipage}[t]{13cm}{\footnotesize
                No data.
                \vspace{\dp0}
            } \end{minipage} \\ \cline{2-3}
            & Expected Result &
                \begin{minipage}[t]{13cm}{\footnotesize
                the raw image with a proper header is written to a file area managed by
the OODS\\[2\baselineskip]

                \vspace{\dp0}
                } \end{minipage}
        \\ \midrule

            \multirow{3}{*}{ 4 } & Description &
            \begin{minipage}[t]{13cm}{\footnotesize
            For each of the expected raw images, verify that the checksum matches
the original Level 1 checksum

            \vspace{\dp0}
            } \end{minipage} \\ \cline{2-3}
            & Test Data &
            \begin{minipage}[t]{13cm}{\footnotesize
                No data.
                \vspace{\dp0}
            } \end{minipage} \\ \cline{2-3}
            & Expected Result &
                \begin{minipage}[t]{13cm}{\footnotesize
                checksum of the file is checked against the file for verification that
the OODS has the correct file and it matches the original md5sum of the
FITS file.\\[2\baselineskip]

                \vspace{\dp0}
                } \end{minipage}
        \\ \midrule

            \multirow{3}{*}{ 5 } & Description &
            \begin{minipage}[t]{13cm}{\footnotesize
            Check that LSST Monitoring Service showed the appropriate information
successfully

            \vspace{\dp0}
            } \end{minipage} \\ \cline{2-3}
            & Test Data &
            \begin{minipage}[t]{13cm}{\footnotesize
                No data.
                \vspace{\dp0}
            } \end{minipage} \\ \cline{2-3}
            & Expected Result &
                \begin{minipage}[t]{13cm}{\footnotesize
                Make sure all camera and OODS systems were available thorughout this
test.~ ~

                \vspace{\dp0}
                } \end{minipage}
        \\ \midrule
    \end{longtable}

\subsection{LVV-T286 - RAS-00-20: Raw image are part of the permanent record of survey via DBB}\label{lvv-t286}

\begin{longtable}[]{llllll}
\toprule
Version & Status & Priority & Verification Type & Owner
\\\midrule
1 & Approved & Normal &
Test & Michelle Butler
\\\bottomrule
\multicolumn{6}{c}{ Open \href{https://jira.lsstcorp.org/secure/Tests.jspa\#/testCase/LVV-T286}{LVV-T286} in Jira } \\
\end{longtable}

\subsubsection{Verification Elements}
\begin{itemize}
\item \href{https://jira.lsstcorp.org/browse/LVV-28}{LVV-28} - DMS-REQ-0068-V-01: Raw Science Image Metadata

\item \href{https://jira.lsstcorp.org/browse/LVV-177}{LVV-177} - DMS-REQ-0346-V-01: Data Availability

\item \href{https://jira.lsstcorp.org/browse/LVV-115}{LVV-115} - DMS-REQ-0284-V-01: Level-1 Production Completeness

\end{itemize}

\subsubsection{Test Items}
This test will check:\\[2\baselineskip]

\begin{itemize}
\tightlist
\item
  That the handoff of a raw image from the Level 1 Archiver Service to
  the DBB buffer manager is successful;
\item
  That the raw image is ingested into the Data Backbone successfully;
\item
  That the monitoring of the above items is successful;
\end{itemize}

This Test Case shall be repeated for each of the different cameras
(ATScam, LSSTCam) and sensors (Science, Wavefront, and Guider)
combination.\\[2\baselineskip]Note: For a complete check of the various
aspects of what it means for a raw image to be in the Data Backbone, see
the tests for the Data Backbone.


\subsubsection{Predecessors}
LVV-T283

\subsubsection{Environment Needs}

\paragraph{Software}
\begin{itemize}
\tightlist
\item
  Level 1 Test Stand
\item
  DBB buffer manager
\item
  DBB raw image ingestion
\item
  DBB database
\item
  LSST Monitoring Service and plugins specific to monitoring raw images,
  DBB buffer manager, and DBB
\end{itemize}

\paragraph{Hardware}
\begin{itemize}
\tightlist
\item
  Level 1 Test Stand (include hardware from LVV-T-283 - RAS-00-00) +
  read/write access to DBB buffer disk;
\item
  Test Machine for DBB buffer manager with read/write access to DBB
  buffer disk;
\item
  Test machine for each DBB endpoint with read/write access to DBB disk;
\item
  Test machine for LSST Monitoring Service
\end{itemize}

Size of buffer disk and DBB disk is determined by number of files to be
included in the test.\\[2\baselineskip]Note: If not testing
inter-machine operability, then the hardware can be minimized to a
single machine outside of the Level 1 test stand.

\subsubsection{Input Specification}
​​​​​None

\subsubsection{Output Specification}
\begin{itemize}
\tightlist
\item
  Raw image(s) are saved to storage and replicated to correct locations
  with checksums that match original Level 1 checksum;
\item
  Database containing information of the following types: physical,
  location, science metadata, provenance as specified in \citeds{LSE-61};
\item
  Both image(s) and database entries replicated correctly;
\end{itemize}

\subsubsection{Test Procedure}
    \begin{longtable}[]{p{1.3cm}p{2cm}p{13cm}}
    %\toprule
    Step & \multicolumn{2}{@{}l}{Description, Input Data and Expected Result} \\ \toprule
    \endhead

            \multirow{3}{*}{ 1 } & Description &
            \begin{minipage}[t]{13cm}{\footnotesize
            Initialize all services configuring the Level 1 Archiver Service so that
the raw images are to be archived to the DBB

            \vspace{\dp0}
            } \end{minipage} \\ \cline{2-3}
            & Test Data &
            \begin{minipage}[t]{13cm}{\footnotesize
                No data.
                \vspace{\dp0}
            } \end{minipage} \\ \cline{2-3}
            & Expected Result &
                \begin{minipage}[t]{13cm}{\footnotesize
                all services for the camera images and the DBB services are all running
and ready for data.~~

                \vspace{\dp0}
                } \end{minipage}
        \\ \midrule

            \multirow{3}{*}{ 2 } & Description &
            \begin{minipage}[t]{13cm}{\footnotesize
            Acquire a raw image (see LVV-T283 - RAS-00-00)\\

            \vspace{\dp0}
            } \end{minipage} \\ \cline{2-3}
            & Test Data &
            \begin{minipage}[t]{13cm}{\footnotesize
                No data.
                \vspace{\dp0}
            } \end{minipage} \\ \cline{2-3}
            & Expected Result &
                \begin{minipage}[t]{13cm}{\footnotesize
                have a raw Fits file with proper header.~~

                \vspace{\dp0}
                } \end{minipage}
        \\ \midrule

            \multirow{3}{*}{ 3 } & Description &
            \begin{minipage}[t]{13cm}{\footnotesize
            After the automatic handoff of the raw image between the Level 1
Archiver Service and the DBB buffer manager, the raw image will
automatically be ingested into the Data Backbone

            \vspace{\dp0}
            } \end{minipage} \\ \cline{2-3}
            & Test Data &
            \begin{minipage}[t]{13cm}{\footnotesize
                No data.
                \vspace{\dp0}
            } \end{minipage} \\ \cline{2-3}
            & Expected Result &
                \begin{minipage}[t]{13cm}{\footnotesize
                the DBB file systems will have the file, and metadata and providence
will be recorded in the consolidated DB. ~ The file will also be
replicated to mulitple locations for DR.~~

                \vspace{\dp0}
                } \end{minipage}
        \\ \midrule

            \multirow{3}{*}{ 4 } & Description &
            \begin{minipage}[t]{13cm}{\footnotesize
            Check that the raw image is accessible at each ~DBB endpoint and matches
original Level 1 checksum

            \vspace{\dp0}
            } \end{minipage} \\ \cline{2-3}
            & Test Data &
            \begin{minipage}[t]{13cm}{\footnotesize
                No data.
                \vspace{\dp0}
            } \end{minipage} \\ \cline{2-3}
            & Expected Result &
                \begin{minipage}[t]{13cm}{\footnotesize
                data resides at NCSA DBB end point, and Chile end point and match with
the same checksum.~~

                \vspace{\dp0}
                } \end{minipage}
        \\ \midrule

            \multirow{3}{*}{ 5 } & Description &
            \begin{minipage}[t]{13cm}{\footnotesize
            Check that LSST Monitoring Service showed the appropriate information
successfully

            \vspace{\dp0}
            } \end{minipage} \\ \cline{2-3}
            & Test Data &
            \begin{minipage}[t]{13cm}{\footnotesize
                No data.
                \vspace{\dp0}
            } \end{minipage} \\ \cline{2-3}
            & Expected Result &
                \begin{minipage}[t]{13cm}{\footnotesize
                all related systems remained up during this test.~~

                \vspace{\dp0}
                } \end{minipage}
        \\ \midrule

            \multirow{3}{*}{ 6 } & Description &
            \begin{minipage}[t]{13cm}{\footnotesize
            More complete tests of the DBB can be done by running the DBB service
tests on the raw image(s). These would check correctness and
completeness of the data stored in the database as well as checking that
the file has been replicated to all required places

            \vspace{\dp0}
            } \end{minipage} \\ \cline{2-3}
            & Test Data &
            \begin{minipage}[t]{13cm}{\footnotesize
                No data.
                \vspace{\dp0}
            } \end{minipage} \\ \cline{2-3}
            & Expected Result &
                \begin{minipage}[t]{13cm}{\footnotesize
                These would be more tests of when things go wrong to make sure that the
DBB is able to continue to work, and not be in the way of taking images
from the camera\\[2\baselineskip]

                \vspace{\dp0}
                } \end{minipage}
        \\ \midrule
    \end{longtable}

\subsection{LVV-T287 - RAS-00-30: Raw Image Archiving Availability, Throughput, Reliability,
and Heterogeneity}\label{lvv-t287}

\begin{longtable}[]{llllll}
\toprule
Version & Status & Priority & Verification Type & Owner
\\\midrule
1 & Approved & Normal &
Test & Michelle Butler
\\\bottomrule
\multicolumn{6}{c}{ Open \href{https://jira.lsstcorp.org/secure/Tests.jspa\#/testCase/LVV-T287}{LVV-T287} in Jira } \\
\end{longtable}

\subsubsection{Verification Elements}
\begin{itemize}
\item \href{https://jira.lsstcorp.org/browse/LVV-5}{LVV-5} - DMS-REQ-0008-V-01: Pipeline Availability

\item \href{https://jira.lsstcorp.org/browse/LVV-65}{LVV-65} - DMS-REQ-0162-V-01: Pipeline Throughput

\item \href{https://jira.lsstcorp.org/browse/LVV-68}{LVV-68} - DMS-REQ-0165-V-01: Infrastructure Sizing for ``catching up''

\item \href{https://jira.lsstcorp.org/browse/LVV-70}{LVV-70} - DMS-REQ-0167-V-01: Incorporate Autonomics

\item \href{https://jira.lsstcorp.org/browse/LVV-145}{LVV-145} - DMS-REQ-0314-V-01: Compute Platform Heterogeneity

\item \href{https://jira.lsstcorp.org/browse/LVV-149}{LVV-149} - DMS-REQ-0318-V-01: Data Management Unscheduled Downtime

\item \href{https://jira.lsstcorp.org/browse/LVV-140}{LVV-140} - DMS-REQ-0309-V-01: Raw Data Archiving Reliability

\end{itemize}

\subsubsection{Test Items}
This test will check:\\[2\baselineskip]

\begin{itemize}
\tightlist
\item
  Raw Image Archiving meets availability requirements;
\item
  Raw Image Archiving meets throughput requirements;
\item
  Raw Image Archiving meets reliability requirements;
\item
  Raw Image Archiving meets heterogeneity requirements;
\end{itemize}

This test case need to be completed when more information is available.


\subsubsection{Predecessors}

\subsubsection{Environment Needs}

\paragraph{Software}

\paragraph{Hardware}

\subsubsection{Input Specification}

\subsubsection{Output Specification}

\subsubsection{Test Procedure}
    \begin{longtable}[]{p{1.3cm}p{2cm}p{13cm}}
    %\toprule
    Step & \multicolumn{2}{@{}l}{Description, Input Data and Expected Result} \\ \toprule
    \endhead

            \multirow{3}{*}{ 1 } & Description &
            \begin{minipage}[t]{13cm}{\footnotesize
            these will be filled out as the service becomes more known as to what
the availablility, throughput, reliability and heterogeneity are.
~\\[2\baselineskip]

            \vspace{\dp0}
            } \end{minipage} \\ \cline{2-3}
            & Test Data &
            \begin{minipage}[t]{13cm}{\footnotesize
                No data.
                \vspace{\dp0}
            } \end{minipage} \\ \cline{2-3}
            & Expected Result &
                \begin{minipage}[t]{13cm}{\footnotesize
                The archive system will stay up through thick and thin and perform like
it's suppose to.\\[2\baselineskip]

                \vspace{\dp0}
                } \end{minipage}
        \\ \midrule
    \end{longtable}

\subsection{LVV-T362 - Installation of the LSST Science Pipelines Payloads}\label{lvv-t362}

\begin{longtable}[]{llllll}
\toprule
Version & Status & Priority & Verification Type & Owner
\\\midrule
1 & Approved & Normal &
Test & John Swinbank
\\\bottomrule
\multicolumn{6}{c}{ Open \href{https://jira.lsstcorp.org/secure/Tests.jspa\#/testCase/LVV-T362}{LVV-T362} in Jira } \\
\end{longtable}

\subsubsection{Verification Elements}
\begin{itemize}
\item \href{https://jira.lsstcorp.org/browse/LVV-29}{LVV-29} - DMS-REQ-0069-V-01: Processed Visit Images

\item \href{https://jira.lsstcorp.org/browse/LVV-98}{LVV-98} - DMS-REQ-0267-V-01: Source Catalog

\item \href{https://jira.lsstcorp.org/browse/LVV-139}{LVV-139} - DMS-REQ-0308-V-01: Software Architecture to Enable Community Re-Use

\item \href{https://jira.lsstcorp.org/browse/LVV-127}{LVV-127} - DMS-REQ-0296-V-01: Pre-cursor, and Real Data

\item \href{https://jira.lsstcorp.org/browse/LVV-15}{LVV-15} - DMS-REQ-0033-V-01: Provide Source Detection Software

\end{itemize}

\subsubsection{Test Items}
This test will check that:

\begin{itemize}
\tightlist
\item
  The Alert Production Pipeline payload is available for installation
  from documented channels;
\item
  The Data Release Production Pipeline payload is available for
  installation from documented channels;
\item
  The Calibration Products Production Pipeline payload is available for
  installation from documented channels;
\item
  These payloads can be installed on systems at the LSST Data Facility
  following available documentation;
\item
  The installed pipeline payloads are capable of successfully executing
  basic integration tests.
\end{itemize}

Note that this test assumes a 2018-era packaging of the Science
Pipelines software, in which all the above payloads are represented by a
single ``meta-package'', lsst\_distrib.


\subsubsection{Predecessors}

\subsubsection{Environment Needs}

\paragraph{Software}
Science Pipelines prerequisite software, as documented at
https://pipelines.lsst.io/, must be installed on the target system.

\paragraph{Hardware}
This test requires a workstation or equivalent system running an
operating system supported by the LSST Science Pipelines.

\subsubsection{Input Specification}

\subsubsection{Output Specification}

\subsubsection{Test Procedure}
    \begin{longtable}[]{p{1.3cm}p{2cm}p{13cm}}
    %\toprule
    Step & \multicolumn{2}{@{}l}{Description, Input Data and Expected Result} \\ \toprule
    \endhead

            \multirow{3}{*}{ 1 } & Description &
            \begin{minipage}[t]{13cm}{\footnotesize
            The LSST Science Pipelines, described by the lsst\_distrib meta-package,
should be installed following the documentation available at
https://pipelines.lsst.io/. The suggested Conda environment will be used
to ensure that a supported execution environment is available.

            \vspace{\dp0}
            } \end{minipage} \\ \cline{2-3}
            & Test Data &
            \begin{minipage}[t]{13cm}{\footnotesize
                No data.
                \vspace{\dp0}
            } \end{minipage} \\ \cline{2-3}
            & Expected Result &
                \begin{minipage}[t]{13cm}{\footnotesize
                Detailed output will depend on the installation method chosen, but will
confirm the successful installation of the Science Pipelines.

                \vspace{\dp0}
                } \end{minipage}
        \\ \midrule

            \multirow{3}{*}{ 2 } & Description &
            \begin{minipage}[t]{13cm}{\footnotesize
            The lsst\_distrib top-level metapackage will be enabled. Assuming that
the software has been installed at
\$\{LSST\_DIR\}:\\[2\baselineskip]\hspace*{0.333em} ~ ~ ~source
\$\{LSST\_DIR\}/loadLSST.bash\\
\hspace*{0.333em} ~ ~ ~setup lsst\_distrib

            \vspace{\dp0}
            } \end{minipage} \\ \cline{2-3}
            & Test Data &
            \begin{minipage}[t]{13cm}{\footnotesize
                No data.
                \vspace{\dp0}
            } \end{minipage} \\ \cline{2-3}
            & Expected Result &
                \begin{minipage}[t]{13cm}{\footnotesize
                Nothing is printed. The command\\[2\baselineskip]\hspace*{0.333em} ~eups
list -s lsst\_distrib\\[2\baselineskip]may be used to confirm that the
correct version of the codebase has been installed.

                \vspace{\dp0}
                } \end{minipage}
        \\ \midrule

            \multirow{3}{*}{ 3 } & Description &
            \begin{minipage}[t]{13cm}{\footnotesize
            The ``LSST Stack Demo'' package will be downloaded onto the test system
from https://github.com/lsst/lsst\_dm\_stack\_demo/releases. The version
corresponding to to the version of the Science Pipelines under test
should be chosen.

            \vspace{\dp0}
            } \end{minipage} \\ \cline{2-3}
            & Test Data &
            \begin{minipage}[t]{13cm}{\footnotesize
                No data.
                \vspace{\dp0}
            } \end{minipage} \\ \cline{2-3}
            & Expected Result &
                \begin{minipage}[t]{13cm}{\footnotesize
                Depends on the tool selected by the user for downloading.

                \vspace{\dp0}
                } \end{minipage}
        \\ \midrule

            \multirow{3}{*}{ 4 } & Description &
            \begin{minipage}[t]{13cm}{\footnotesize
            The stack demo package is uncompressed into a directory \$\{DEMO\_DIR\}.

            \vspace{\dp0}
            } \end{minipage} \\ \cline{2-3}
            & Test Data &
            \begin{minipage}[t]{13cm}{\footnotesize
                No data.
                \vspace{\dp0}
            } \end{minipage} \\ \cline{2-3}
            & Expected Result &
                \begin{minipage}[t]{13cm}{\footnotesize
                Depends on options given to the tar command. Should confirm the
availability of the stack demo source.

                \vspace{\dp0}
                } \end{minipage}
        \\ \midrule

            \multirow{3}{*}{ 5 } & Description &
            \begin{minipage}[t]{13cm}{\footnotesize
            The demo package will be executed by following the instructions in its
README file.~

            \vspace{\dp0}
            } \end{minipage} \\ \cline{2-3}
            & Test Data &
            \begin{minipage}[t]{13cm}{\footnotesize
                No data.
                \vspace{\dp0}
            } \end{minipage} \\ \cline{2-3}
            & Expected Result &
                \begin{minipage}[t]{13cm}{\footnotesize
                Successful execution will result in the string ``Ok'' being returned.

                \vspace{\dp0}
                } \end{minipage}
        \\ \midrule
    \end{longtable}

\subsection{LVV-T363 - Science Pipelines Release Documentation}\label{lvv-t363}

\begin{longtable}[]{llllll}
\toprule
Version & Status & Priority & Verification Type & Owner
\\\midrule
1 & Approved & Normal &
Inspection & John Swinbank
\\\bottomrule
\multicolumn{6}{c}{ Open \href{https://jira.lsstcorp.org/secure/Tests.jspa\#/testCase/LVV-T363}{LVV-T363} in Jira } \\
\end{longtable}

\subsubsection{Verification Elements}
\begin{itemize}
\item \href{https://jira.lsstcorp.org/browse/LVV-139}{LVV-139} - DMS-REQ-0308-V-01: Software Architecture to Enable Community Re-Use

\item \href{https://jira.lsstcorp.org/browse/LVV-3402}{LVV-3402} - DMS-REQ-0360-V-01: Median astrometric error on 20 arcmin scales

\end{itemize}

\subsubsection{Test Items}
This test will check:

\begin{itemize}
\tightlist
\item
  That a particular Science Pipelines release is adequately described by
  documentation at the https://pipelines.lsst.io/ site;
\item
  That the Science Pipelines release is accompanied by a
  characterization report which describes its scientific performance.
\end{itemize}


\subsubsection{Predecessors}

\subsubsection{Environment Needs}

\paragraph{Software}
A web browser.

\paragraph{Hardware}
A device with internet access.

\subsubsection{Input Specification}

\subsubsection{Output Specification}

\subsubsection{Test Procedure}
    \begin{longtable}[]{p{1.3cm}p{2cm}p{13cm}}
    %\toprule
    Step & \multicolumn{2}{@{}l}{Description, Input Data and Expected Result} \\ \toprule
    \endhead

            \multirow{3}{*}{ 1 } & Description &
            \begin{minipage}[t]{13cm}{\footnotesize
            Load the Science Pipelines website at https://pipelines.lsst.io/.

            \vspace{\dp0}
            } \end{minipage} \\ \cline{2-3}
            & Test Data &
            \begin{minipage}[t]{13cm}{\footnotesize
                No data.
                \vspace{\dp0}
            } \end{minipage} \\ \cline{2-3}
            & Expected Result &
                \begin{minipage}[t]{13cm}{\footnotesize
                The website is displayed.

                \vspace{\dp0}
                } \end{minipage}
        \\ \midrule

            \multirow{3}{*}{ 2 } & Description &
            \begin{minipage}[t]{13cm}{\footnotesize
            Identify documentation for the release under test. This should be
clearly labelled on the documentation site.\\[2\baselineskip]If the
latest release is being tested, the default page loaded when visiting
https://pipelines.lsst.io/ should be the documentation
required.\\[2\baselineskip]If this test is for another release, the site
should present clear instructions for changing the edition (or version)
of the documentation being examined, and documentation for the release
under test should be available.

            \vspace{\dp0}
            } \end{minipage} \\ \cline{2-3}
            & Test Data &
            \begin{minipage}[t]{13cm}{\footnotesize
                No data.
                \vspace{\dp0}
            } \end{minipage} \\ \cline{2-3}
            & Expected Result &
                \begin{minipage}[t]{13cm}{\footnotesize
                The documentation for the release under test is displayed.

                \vspace{\dp0}
                } \end{minipage}
        \\ \midrule

            \multirow{3}{*}{ 3 } & Description &
            \begin{minipage}[t]{13cm}{\footnotesize
            Inspect the documentation to ensure that it refers to the release under
test, and that it provides:

\begin{itemize}
\tightlist
\item
  Release notes, describing changes in this release relative to the
  previous;
\item
  Installation instructions, together with a list of supported platforms
  and prerequisites;
\item
  Getting started information.
\end{itemize}

            \vspace{\dp0}
            } \end{minipage} \\ \cline{2-3}
            & Test Data &
            \begin{minipage}[t]{13cm}{\footnotesize
                No data.
                \vspace{\dp0}
            } \end{minipage} \\ \cline{2-3}
            & Expected Result &
                \begin{minipage}[t]{13cm}{\footnotesize
                The user is satisfied that the required information is available.

                \vspace{\dp0}
                } \end{minipage}
        \\ \midrule

            \multirow{3}{*}{ 4 } & Description &
            \begin{minipage}[t]{13cm}{\footnotesize
            Locate the Characterization Metric Report corresponding to this release.
It should be linked from the main release documentation.

            \vspace{\dp0}
            } \end{minipage} \\ \cline{2-3}
            & Test Data &
            \begin{minipage}[t]{13cm}{\footnotesize
                No data.
                \vspace{\dp0}
            } \end{minipage} \\ \cline{2-3}
            & Expected Result &
                \begin{minipage}[t]{13cm}{\footnotesize
                The user is satisfied that the report is available.

                \vspace{\dp0}
                } \end{minipage}
        \\ \midrule

            \multirow{3}{*}{ 5 } & Description &
            \begin{minipage}[t]{13cm}{\footnotesize
            Verify that the characterization metric report describes the scientific
performance of the release in terms of a selection of performance
metrics drawn from high-level requirements documentation (the Science
Requirements Document, LPM-17; the LSST System Requirements, LSE-29;
and/or the Observatory System Specifications, LSE-30).

            \vspace{\dp0}
            } \end{minipage} \\ \cline{2-3}
            & Test Data &
            \begin{minipage}[t]{13cm}{\footnotesize
                No data.
                \vspace{\dp0}
            } \end{minipage} \\ \cline{2-3}
            & Expected Result &
                \begin{minipage}[t]{13cm}{\footnotesize
                Metric values describing the performance of the release, for example as
computed by validate\_drp, are described in the report.

                \vspace{\dp0}
                } \end{minipage}
        \\ \midrule
    \end{longtable}

\subsection{LVV-T368 - Loading and processing Camera test data}\label{lvv-t368}

\begin{longtable}[]{llllll}
\toprule
Version & Status & Priority & Verification Type & Owner
\\\midrule
2 & Approved & Normal &
Test & John Swinbank
\\\bottomrule
\multicolumn{6}{c}{ Open \href{https://jira.lsstcorp.org/secure/Tests.jspa\#/testCase/LVV-T368}{LVV-T368} in Jira } \\
\end{longtable}

\subsubsection{Verification Elements}
\begin{itemize}
\item \href{https://jira.lsstcorp.org/browse/LVV-129}{LVV-129} - DMS-REQ-0298-V-01: Data Product and Raw Data Access

\item \href{https://jira.lsstcorp.org/browse/LVV-63}{LVV-63} - DMS-REQ-0160-V-01: Provide User Interface Services

\item \href{https://jira.lsstcorp.org/browse/LVV-23}{LVV-23} - DMS-REQ-0060-V-01: Bias Residual Image

\end{itemize}

\subsubsection{Test Items}
This test will check:

\begin{itemize}
\tightlist
\item
  That Camera test data is available for processing in the LSST Data
  Facility, and accessible through the LSST Science Platform;
\item
  That the Data Management I/O abstraction (the ``Data Butler'') can
  load that data into the Science Platform environment;
\item
  That Data Management algorithmic ``tasks'' can be executed to process
  that data;
\item
  That results can be displayed in the Firefly display tool.
\end{itemize}


\subsubsection{Predecessors}
Executing LVV-T374 will satisfy the preconditions for this test,
assuming that \$REPOSITORY\_PATH is set equal to the output location
used in LVV-T374.

\subsubsection{Environment Needs}

\paragraph{Software}
The LSST Science Pipelines version w\_2018\_45 must be available within
the Notebook Aspect of the LSST Science Platform.

\paragraph{Hardware}
This test assumes the availability of the Notebook and Portal aspects of
the LSST Science Platform, deployed at
https://lsst-lspdev.ncsa.illinois.edu.

\subsubsection{Input Specification}
Appropriate data --- to include a ``raw'' and a ``bias'' exposure ---
from the Camera test systems must be available in a Butler data
repository on a filesystem accessible to the Notebook Aspect of the
Science Platform.\\[2\baselineskip]For the purposes of the following
discussion, we assume that:\\[2\baselineskip]

\begin{itemize}
\tightlist
\item
  Visit 258334666 from RTM (Raft Tower Module) 007 will be used;
\item
  The data is available in a repository at
  /project/bootcamp/repo\_RTM-007/ on the Data Facility GPFS filesystem
\end{itemize}

In the test script, we refer to ``258334666'' as ``\$VISIT\_ID'' and
``/project/bootcamp/repo\_RTM-007/'' as ``\$REPOSITORY\_PATH''; other
data may be substituted as appropriate.

\subsubsection{Output Specification}

\subsubsection{Test Procedure}
    \begin{longtable}[]{p{1.3cm}p{2cm}p{13cm}}
    %\toprule
    Step & \multicolumn{2}{@{}l}{Description, Input Data and Expected Result} \\ \toprule
    \endhead

            \multirow{3}{*}{ 1 } & Description &
            \begin{minipage}[t]{13cm}{\footnotesize
            Connect to the Notebook Aspect of the Science Platform following the
instructions at https://nb.lsst.io/. Log in, and ``spawn'' a new machine
with image ``Weekly 2018\_45`` and size ``small''.

            \vspace{\dp0}
            } \end{minipage} \\ \cline{2-3}
            & Test Data &
            \begin{minipage}[t]{13cm}{\footnotesize
                No data.
                \vspace{\dp0}
            } \end{minipage} \\ \cline{2-3}
            & Expected Result &
                \begin{minipage}[t]{13cm}{\footnotesize
                The JupyterLab environment appears.

                \vspace{\dp0}
                } \end{minipage}
        \\ \midrule

            \multirow{3}{*}{ 2 } & Description &
            \begin{minipage}[t]{13cm}{\footnotesize
            Create a terminal session. Use it to set up the LSST tools, then
download and build version 5c12b06e6 of
obs\_lsst:\\[2\baselineskip]\hspace*{0.333em} ~\$ source
/opt/lsst/software/stack/loadLSST.bash\\
\hspace*{0.333em} ~\$ setup lsst\_distrib\\
\hspace*{0.333em} ~\$ git clone https://github.com/lsst/obs\_lsst.git\\
\hspace*{0.333em} ~\$ cd obs\_lsst\\
\hspace*{0.333em} ~\$ git checkout 5c12b06e6\\
\hspace*{0.333em} ~\$ setup -k -r .\\
\hspace*{0.333em} ~\$ scons\\[2\baselineskip]Arrange for obs\_lsst to
automatically be added to the environment when starting a new
notebook:\\[2\baselineskip]\hspace*{0.333em} ~\$ echo ``setup -j -r
\textasciitilde{}/obs\_lsst'' \textgreater{}\textgreater{}
\textasciitilde{}/notebooks/.user\_setups\\[2\baselineskip]Exit the
terminal.

            \vspace{\dp0}
            } \end{minipage} \\ \cline{2-3}
            & Test Data &
            \begin{minipage}[t]{13cm}{\footnotesize
                No data.
                \vspace{\dp0}
            } \end{minipage} \\ \cline{2-3}
            & Expected Result &
                \begin{minipage}[t]{13cm}{\footnotesize
                No errors are seen during execution of the provided commands.

                \vspace{\dp0}
                } \end{minipage}
        \\ \midrule

            \multirow{3}{*}{ 3 } & Description &
            \begin{minipage}[t]{13cm}{\footnotesize
            Create a new ``LSST'' notebook.\\[2\baselineskip]Import the standard
libraries required for the rest of this
test:\\[2\baselineskip]\hspace*{0.333em} ~import os\\
\hspace*{0.333em} ~import lsst.afw.display as afwDisplay\\
\hspace*{0.333em} ~from lsst.daf.persistence import Butler\\
\hspace*{0.333em} ~from lsst.ip.isr import IsrTask\\
\hspace*{0.333em} ~from firefly\_client import FireflyClient\\
\hspace*{0.333em} ~from IPython.display import
IFrame\\[2\baselineskip]and execute the cell.

            \vspace{\dp0}
            } \end{minipage} \\ \cline{2-3}
            & Test Data &
            \begin{minipage}[t]{13cm}{\footnotesize
                No data.
                \vspace{\dp0}
            } \end{minipage} \\ \cline{2-3}
            & Expected Result &
                \begin{minipage}[t]{13cm}{\footnotesize
                Nothing is printed.

                \vspace{\dp0}
                } \end{minipage}
        \\ \midrule

            \multirow{3}{*}{ 4 } & Description &
            \begin{minipage}[t]{13cm}{\footnotesize
            Create a Data Butler client, and use it to retrieve the data which will
be used for this test.\\[2\baselineskip]\hspace*{0.333em} ~butler =
Butler(\$REPOSITORY\_PATH)\\
\hspace*{0.333em} ~raw = butler.get(``raw'', visit=\$VISIT\_ID,
detector=2)\\
\hspace*{0.333em} ~bias = butler.get(``bias'', visit=\$VISIT\_ID,
detector=2)

            \vspace{\dp0}
            } \end{minipage} \\ \cline{2-3}
            & Test Data &
            \begin{minipage}[t]{13cm}{\footnotesize
                No data.
                \vspace{\dp0}
            } \end{minipage} \\ \cline{2-3}
            & Expected Result &
                \begin{minipage}[t]{13cm}{\footnotesize
                Nothing is printed.

                \vspace{\dp0}
                } \end{minipage}
        \\ \midrule

            \multirow{3}{*}{ 5 } & Description &
            \begin{minipage}[t]{13cm}{\footnotesize
            Initialize the Firefly display
system:\\[2\baselineskip]\hspace*{0.333em} ~my\_channel =
`\{\}\_test\_channel'.format(os.environ{[}'USER'{]})\\
\hspace*{0.333em} ~server = `https://lsst-lspdev.ncsa.illinois.edu'\\
\hspace*{0.333em}
~ff='\{\}/firefly/slate.html?\_\_wsch=\{\}'.format(server,
my\_channel)\\
\hspace*{0.333em} ~IFrame(ff,800,600)\\
\hspace*{0.333em} ~afwDisplay.setDefaultBackend('firefly')\\
\hspace*{0.333em} ~afw\_display = afwDisplay.getDisplay(frame=1,\\
\hspace*{0.333em} ~ ~ ~ ~ ~ ~ ~ ~ ~ ~ ~ ~ ~ ~ ~ ~ ~ ~
~name=my\_channel)\\[2\baselineskip]Click on the link provided after
executing the above.

            \vspace{\dp0}
            } \end{minipage} \\ \cline{2-3}
            & Test Data &
            \begin{minipage}[t]{13cm}{\footnotesize
                No data.
                \vspace{\dp0}
            } \end{minipage} \\ \cline{2-3}
            & Expected Result &
                \begin{minipage}[t]{13cm}{\footnotesize
                A Firefly window is shown.

                \vspace{\dp0}
                } \end{minipage}
        \\ \midrule

            \multirow{3}{*}{ 6 } & Description &
            \begin{minipage}[t]{13cm}{\footnotesize
            Display the raw image data in the Firefly
window:\\[2\baselineskip]\hspace*{0.333em} afw\_display.mtv(raw)

            \vspace{\dp0}
            } \end{minipage} \\ \cline{2-3}
            & Test Data &
            \begin{minipage}[t]{13cm}{\footnotesize
                No data.
                \vspace{\dp0}
            } \end{minipage} \\ \cline{2-3}
            & Expected Result &
                \begin{minipage}[t]{13cm}{\footnotesize
                Raw image data is displayed.

                \vspace{\dp0}
                } \end{minipage}
        \\ \midrule

            \multirow{3}{*}{ 7 } & Description &
            \begin{minipage}[t]{13cm}{\footnotesize
            Configure and run an Instrument Signature Removal (ISR) task on the raw
data. Most corrections are disabled for simplicity. but the bias frame
is applied.\\
\hspace*{0.333em}

\begin{verbatim}
   isr_config = IsrTask.ConfigClass()
   isr_config.doDark=False
   isr_config.doFlat=False
   isr_config.doFringe=False
   isr_config.doDefect=False
   isr_config.doAddDistortionModel=False
   isr_config.doLinearize=False
   isr = IsrTask(config=isr_config)
   result = isr.run(raw, bias=bias)
\end{verbatim}

            \vspace{\dp0}
            } \end{minipage} \\ \cline{2-3}
            & Test Data &
            \begin{minipage}[t]{13cm}{\footnotesize
                No data.
                \vspace{\dp0}
            } \end{minipage} \\ \cline{2-3}
            & Expected Result &
                \begin{minipage}[t]{13cm}{\footnotesize
                Nothing is printed.

                \vspace{\dp0}
                } \end{minipage}
        \\ \midrule

            \multirow{3}{*}{ 8 } & Description &
            \begin{minipage}[t]{13cm}{\footnotesize
            Display the corrected image data in the Firefly
window:\\[2\baselineskip]\hspace*{0.333em}
~afw\_display.mtv(result.exposure)

            \vspace{\dp0}
            } \end{minipage} \\ \cline{2-3}
            & Test Data &
            \begin{minipage}[t]{13cm}{\footnotesize
                No data.
                \vspace{\dp0}
            } \end{minipage} \\ \cline{2-3}
            & Expected Result &
                \begin{minipage}[t]{13cm}{\footnotesize
                Processed (trimmed, bias-subtracted) image data is displayed.

                \vspace{\dp0}
                } \end{minipage}
        \\ \midrule
    \end{longtable}

\subsection{LVV-T374 - Ingesting Camera test data}\label{lvv-t374}

\begin{longtable}[]{llllll}
\toprule
Version & Status & Priority & Verification Type & Owner
\\\midrule
1 & Approved & Normal &
Test & John Swinbank
\\\bottomrule
\multicolumn{6}{c}{ Open \href{https://jira.lsstcorp.org/secure/Tests.jspa\#/testCase/LVV-T374}{LVV-T374} in Jira } \\
\end{longtable}

\subsubsection{Verification Elements}
\begin{itemize}
\item \href{https://jira.lsstcorp.org/browse/LVV-130}{LVV-130} - DMS-REQ-0299-V-01: Data Product Ingest

\item \href{https://jira.lsstcorp.org/browse/LVV-129}{LVV-129} - DMS-REQ-0298-V-01: Data Product and Raw Data Access

\end{itemize}

\subsubsection{Test Items}
This test will check:

\begin{itemize}
\tightlist
\item
  That raw Camera test data is available on a filesystem in the LSST
  Data Facility;
\item
  That raw Camera test data can be ingested and made available through
  the Data Management I/O abstraction (the ``Data Butler'').
\end{itemize}


\subsubsection{Predecessors}

\subsubsection{Environment Needs}

\paragraph{Software}
The LSST Science Pipelines version w\_2018\_45 must be available within
the Notebook Aspect of the LSST Science Platform.

\paragraph{Hardware}
This test assumes the availability of the Notebook aspect of the LSST
Science Platform, deployed at https://lsst-lspdev.ncsa.illinois.edu.

\subsubsection{Input Specification}
Appropriate raw data from Camera test systems must be available on a
filesystem within the LSST Data Facility. This test data is assumed to
include visit 258334666 (hereafter referred to as \$VISIT\_ID) from RTM
(Raft Tower Module) 007 for the purposes of this test, but other,
equivalent, may be substituted.\\[2\baselineskip]At time of writing,
suitable data may be found on the GPFS filesystem at
/project/bootcamp/data/LCA-11021\_RTM-007/7086/fe55\_raft\_acq/v0/44981.
In future, as data transport procedures to the Data Facility become more
streamlined and formalised, this data may be moved elsewhere or made
available through some other system. Throughout the test script, we use
the string ``\$INPUT\_DATA\_DIR'' as an alias for
``/project/bootcamp/data/LCA-11021\_RTM-007/7086/fe55\_raft\_acq/v0/44981'',
or wherever this data has been moved to.\\[2\baselineskip]

\subsubsection{Output Specification}

\subsubsection{Test Procedure}
    \begin{longtable}[]{p{1.3cm}p{2cm}p{13cm}}
    %\toprule
    Step & \multicolumn{2}{@{}l}{Description, Input Data and Expected Result} \\ \toprule
    \endhead

            \multirow{3}{*}{ 1 } & Description &
            \begin{minipage}[t]{13cm}{\footnotesize
            Connect to the Notebook Aspect of the Science Platform following the
instructions at https://nb.lsst.io/. Log in, and ``spawn'' a new machine
with image ``Weekly 2018\_45`` and size ``large''.

            \vspace{\dp0}
            } \end{minipage} \\ \cline{2-3}
            & Test Data &
            \begin{minipage}[t]{13cm}{\footnotesize
                No data.
                \vspace{\dp0}
            } \end{minipage} \\ \cline{2-3}
            & Expected Result &
                \begin{minipage}[t]{13cm}{\footnotesize
                The JupyterLab environment appears.

                \vspace{\dp0}
                } \end{minipage}
        \\ \midrule

            \multirow{3}{*}{ 2 } & Description &
            \begin{minipage}[t]{13cm}{\footnotesize
            Create a terminal session. Use it to set up the LSST tools, then
download and build version 5c12b06e6 of
obs\_lsst:\\[2\baselineskip]\hspace*{0.333em} ~\$ source
/opt/lsst/software/stack/loadLSST.bash\\
\hspace*{0.333em} ~\$ setup lsst\_distrib\\
\hspace*{0.333em} ~\$ git clone https://github.com/lsst/obs\_lsst.git\\
\hspace*{0.333em} ~\$ cd obs\_lsst\\
\hspace*{0.333em} ~\$ git checkout 5c12b06e6\\
\hspace*{0.333em} ~\$ setup -k -r .\\
\hspace*{0.333em} ~\$ scons

            \vspace{\dp0}
            } \end{minipage} \\ \cline{2-3}
            & Test Data &
            \begin{minipage}[t]{13cm}{\footnotesize
                No data.
                \vspace{\dp0}
            } \end{minipage} \\ \cline{2-3}
            & Expected Result &
                \begin{minipage}[t]{13cm}{\footnotesize
                No errors are seen during execution of the provided commands.

                \vspace{\dp0}
                } \end{minipage}
        \\ \midrule

            \multirow{3}{*}{ 3 } & Description &
            \begin{minipage}[t]{13cm}{\footnotesize
            Ingest RTM-007 test data by executing the following
commands:\\[2\baselineskip]\hspace*{0.333em}
~OUTPUT\_REPO\_DIR=\$OUTPUT\_DATA\_DIR\\
\hspace*{0.333em} ~INPUT\_DATA\_DIR=\$INPUT\_DATA\_DIR\\
\hspace*{0.333em} ~mkdir -p \$OUTPUT\_REPO\_DIR\\
\hspace*{0.333em} ~echo ``lsst.obs.lsst.ts8.Ts8Mapper'' \textgreater{}
\$OUTPUT\_REPO\_DIR/\_mapper\\
\hspace*{0.333em} ~ingestImages.py \$OUTPUT\_REPO\_DIR
\$INPUT\_DATA\_DIR/*/*.fits\\
\hspace*{0.333em} ~constructBias.py \$OUTPUT\_REPO\_DIR --rerun calibs
--id imageType=BIAS --batch-type smp --cores 4\\
\hspace*{0.333em} ~ingestCalibs.py \$OUTPUT\_REPO\_DIR --calibType bias
\$OUTPUT\_REPO\_DIR/rerun/calibs/bias/*/*.fits --validity 9999 --output
\$OUTPUT\_REPO\_DIR/CALIB
--mode=link\\[2\baselineskip]Where:\\[2\baselineskip]\hspace*{0.333em}
~\$OUTPUT\_DATA\_DIR is some location on shared storage to which the
user has write permission;\\
\hspace*{0.333em} ~\$INPUT\_DATA\_DIR is defined in the test case
description.

            \vspace{\dp0}
            } \end{minipage} \\ \cline{2-3}
            & Test Data &
            \begin{minipage}[t]{13cm}{\footnotesize
                No data.
                \vspace{\dp0}
            } \end{minipage} \\ \cline{2-3}
            & Expected Result &
                \begin{minipage}[t]{13cm}{\footnotesize
                Many status messages are logged to screen, and the command exits with
status 0.

                \vspace{\dp0}
                } \end{minipage}
        \\ \midrule

            \multirow{3}{*}{ 4 } & Description &
            \begin{minipage}[t]{13cm}{\footnotesize
            Demonstrate that raw and bias data for visit \$VISIT\_ID have been made
available in the repository. Load a Python interpreter (run
â\euro{}œpython") and execute the
following:\\[2\baselineskip]\hspace*{0.333em} ~from lsst.daf.persistence
import Butler\\
\hspace*{0.333em} ~visit\_id = \$VISIT\_ID\\
\hspace*{0.333em} ~b = Butler(\$OUTPUT\_DATA\_DIR)\\
\hspace*{0.333em} ~b.get(``raw'', visit=visit\_id, detector=2)\\
\hspace*{0.333em} ~b.get(``bias'', visit=visit\_id, detector=2)

            \vspace{\dp0}
            } \end{minipage} \\ \cline{2-3}
            & Test Data &
            \begin{minipage}[t]{13cm}{\footnotesize
                No data.
                \vspace{\dp0}
            } \end{minipage} \\ \cline{2-3}
            & Expected Result &
                \begin{minipage}[t]{13cm}{\footnotesize
                Each call to b.get() returns an instance of an ExposureF object.
Warnings about lack of dark-time or WCS information may be ignored.

                \vspace{\dp0}
                } \end{minipage}
        \\ \midrule
    \end{longtable}

\subsection{LVV-T376 - Verify the Calculation of Ellipticity Residuals and Correlations}\label{lvv-t376}

\begin{longtable}[]{llllll}
\toprule
Version & Status & Priority & Verification Type & Owner
\\\midrule
1 & Approved & Normal &
Test & Leanne Guy
\\\bottomrule
\multicolumn{6}{c}{ Open \href{https://jira.lsstcorp.org/secure/Tests.jspa\#/testCase/LVV-T376}{LVV-T376} in Jira } \\
\end{longtable}

\subsubsection{Verification Elements}
\begin{itemize}
\item \href{https://jira.lsstcorp.org/browse/LVV-3404}{LVV-3404} - DMS-REQ-0362-V-01: Median residual PSF ellipticity correlations on 5
arcmin scales

\item \href{https://jira.lsstcorp.org/browse/LVV-9780}{LVV-9780} - DMS-REQ-0362-V-02: Max fraction of excess ellipticity residuals on 1 and
5 arcmin scales

\end{itemize}

\subsubsection{Test Items}
Verify that the DMS includes software to enable the calculation of the
ellipticity residuals and correlation metrics defined in the OSS.~


\subsubsection{Predecessors}

\subsubsection{Environment Needs}

\paragraph{Software}

\paragraph{Hardware}

\subsubsection{Input Specification}

\subsubsection{Output Specification}

\subsubsection{Test Procedure}
    \begin{longtable}[]{p{1.3cm}p{2cm}p{13cm}}
    %\toprule
    Step & \multicolumn{2}{@{}l}{Description, Input Data and Expected Result} \\ \toprule
    \endhead

                \multirow{3}{*}{\parbox{1.3cm}{ 1-1
                {\scriptsize from \hyperref[lvv-t987]
                {LVV-T987} } } }

                & {\small Description} &
                \begin{minipage}[t]{13cm}{\scriptsize
                Identify the path to the data repository, which we will refer to as
`DATA/path', then execute the following:

                \vspace{\dp0}
                } \end{minipage} \\ \cdashline{2-3}
                & {\small Test Data} &
                \begin{minipage}[t]{13cm}{\scriptsize
                } \end{minipage} \\ \cdashline{2-3}
                & {\small Expected Result} &
                    \begin{minipage}[t]{13cm}{\scriptsize
                    Butler repo available for reading.

                    \vspace{\dp0}
                    } \end{minipage}
                \\ \hdashline


        \\ \midrule

            \multirow{3}{*}{ 2 } & Description &
            \begin{minipage}[t]{13cm}{\footnotesize
            Point the butler to an appropriate (precursor or simulated) dataset
containing data in all filters, that is sufficient for the purposes of
measuring astrometric performance metrics.

            \vspace{\dp0}
            } \end{minipage} \\ \cline{2-3}
            & Test Data &
            \begin{minipage}[t]{13cm}{\footnotesize
                No data.
                \vspace{\dp0}
            } \end{minipage} \\ \cline{2-3}
            & Expected Result &
        \\ \midrule

            \multirow{3}{*}{ 3 } & Description &
            \begin{minipage}[t]{13cm}{\footnotesize
            Execute the LSST Stack package `validate\_drp` (or an alternate package
that is relevant) on this dataset to perform the measurements of the
metrics.

            \vspace{\dp0}
            } \end{minipage} \\ \cline{2-3}
            & Test Data &
            \begin{minipage}[t]{13cm}{\footnotesize
                No data.
                \vspace{\dp0}
            } \end{minipage} \\ \cline{2-3}
            & Expected Result &
                \begin{minipage}[t]{13cm}{\footnotesize
                Measurements of validation metrics and the presence of QA plots
resulting from the validation pipeline.

                \vspace{\dp0}
                } \end{minipage}
        \\ \midrule

            \multirow{3}{*}{ 4 } & Description &
            \begin{minipage}[t]{13cm}{\footnotesize
            Compare measured ellipticity correlations to known (for simulated data)
or measured (if using precursor data) values from input (precursor or
simulated) data, and confirm that the output values for all of the
ellipticity performance metrics are as expected.

            \vspace{\dp0}
            } \end{minipage} \\ \cline{2-3}
            & Test Data &
            \begin{minipage}[t]{13cm}{\footnotesize
                No data.
                \vspace{\dp0}
            } \end{minipage} \\ \cline{2-3}
            & Expected Result &
                \begin{minipage}[t]{13cm}{\footnotesize
                Measured ellipticity metrics that are within reasonable values given the
(known) input dataset.

                \vspace{\dp0}
                } \end{minipage}
        \\ \midrule
    \end{longtable}

\subsection{LVV-T377 - Verify Calculation of Photometric Performance Metrics}\label{lvv-t377}

\begin{longtable}[]{llllll}
\toprule
Version & Status & Priority & Verification Type & Owner
\\\midrule
1 & Approved & Normal &
Test & Leanne Guy
\\\bottomrule
\multicolumn{6}{c}{ Open \href{https://jira.lsstcorp.org/secure/Tests.jspa\#/testCase/LVV-T377}{LVV-T377} in Jira } \\
\end{longtable}

\subsubsection{Verification Elements}
\begin{itemize}
\item \href{https://jira.lsstcorp.org/browse/LVV-9751}{LVV-9751} - DMS-REQ-0359-V-02: Max fraction of sensors with excess unusable pixels

\item \href{https://jira.lsstcorp.org/browse/LVV-9757}{LVV-9757} - DMS-REQ-0359-V-08: Max cross-talk imperfections

\item \href{https://jira.lsstcorp.org/browse/LVV-9755}{LVV-9755} - DMS-REQ-0359-V-06: Accuracy of photometric transformation

\item \href{https://jira.lsstcorp.org/browse/LVV-9756}{LVV-9756} - DMS-REQ-0359-V-07: RMS width of zero point in u-band

\item \href{https://jira.lsstcorp.org/browse/LVV-9753}{LVV-9753} - DMS-REQ-0359-V-04: Accuracy of zero point for colors with u-band

\item \href{https://jira.lsstcorp.org/browse/LVV-9762}{LVV-9762} - DMS-REQ-0359-V-13: Max sky brightness error

\item \href{https://jira.lsstcorp.org/browse/LVV-9760}{LVV-9760} - DMS-REQ-0359-V-11: Fraction of zero point outliers

\item \href{https://jira.lsstcorp.org/browse/LVV-9761}{LVV-9761} - DMS-REQ-0359-V-12: Max fraction of unusable pixels per sensor

\item \href{https://jira.lsstcorp.org/browse/LVV-9764}{LVV-9764} - DMS-REQ-0359-V-15: Percentage of image area with ghosts

\item \href{https://jira.lsstcorp.org/browse/LVV-9766}{LVV-9766} - DMS-REQ-0359-V-17: Max RMS of resolved/unresolved flux ratio

\item \href{https://jira.lsstcorp.org/browse/LVV-9763}{LVV-9763} - DMS-REQ-0359-V-14: RMS width of zero point in all bands except u

\item \href{https://jira.lsstcorp.org/browse/LVV-9765}{LVV-9765} - DMS-REQ-0359-V-16: Accuracy of zero point for colors without u-band

\end{itemize}

\subsubsection{Test Items}
Verify that the DMS system provides software to calculate photometric
performance metrics, and that the algorithms are properly calculating
the desired quantities. Note that because the DMS requirement is that
the software shall be provided (and not on the actual measured values of
the metrics), we verify all of the requirements via a single test case.


\subsubsection{Predecessors}

\subsubsection{Environment Needs}

\paragraph{Software}

\paragraph{Hardware}

\subsubsection{Input Specification}

\subsubsection{Output Specification}

\subsubsection{Test Procedure}
    \begin{longtable}[]{p{1.3cm}p{2cm}p{13cm}}
    %\toprule
    Step & \multicolumn{2}{@{}l}{Description, Input Data and Expected Result} \\ \toprule
    \endhead

                \multirow{3}{*}{\parbox{1.3cm}{ 1-1
                {\scriptsize from \hyperref[lvv-t987]
                {LVV-T987} } } }

                & {\small Description} &
                \begin{minipage}[t]{13cm}{\scriptsize
                Identify the path to the data repository, which we will refer to as
`DATA/path', then execute the following:

                \vspace{\dp0}
                } \end{minipage} \\ \cdashline{2-3}
                & {\small Test Data} &
                \begin{minipage}[t]{13cm}{\scriptsize
                } \end{minipage} \\ \cdashline{2-3}
                & {\small Expected Result} &
                    \begin{minipage}[t]{13cm}{\scriptsize
                    Butler repo available for reading.

                    \vspace{\dp0}
                    } \end{minipage}
                \\ \hdashline


        \\ \midrule

            \multirow{3}{*}{ 2 } & Description &
            \begin{minipage}[t]{13cm}{\footnotesize
            Point the butler to a simulated dataset containing data in all filters,
that is sufficient for the purposes of measuring photometric performance
metrics.

            \vspace{\dp0}
            } \end{minipage} \\ \cline{2-3}
            & Test Data &
            \begin{minipage}[t]{13cm}{\footnotesize
                No data.
                \vspace{\dp0}
            } \end{minipage} \\ \cline{2-3}
            & Expected Result &
        \\ \midrule

            \multirow{3}{*}{ 3 } & Description &
            \begin{minipage}[t]{13cm}{\footnotesize
            Execute the LSST Stack package `validate\_drp` (or an alternate package
that is relevant) on this dataset to perform the measurements of the
metrics.

            \vspace{\dp0}
            } \end{minipage} \\ \cline{2-3}
            & Test Data &
            \begin{minipage}[t]{13cm}{\footnotesize
                No data.
                \vspace{\dp0}
            } \end{minipage} \\ \cline{2-3}
            & Expected Result &
                \begin{minipage}[t]{13cm}{\footnotesize
                Measurements of validation metrics and the presence of QA plots
resulting from the validation pipeline.

                \vspace{\dp0}
                } \end{minipage}
        \\ \midrule

            \multirow{3}{*}{ 4 } & Description &
            \begin{minipage}[t]{13cm}{\footnotesize
            Compare measured photometry to known values from input simulated data,
and confirm that the output values for all of the photometric
performance metrics are as expected.

            \vspace{\dp0}
            } \end{minipage} \\ \cline{2-3}
            & Test Data &
            \begin{minipage}[t]{13cm}{\footnotesize
                No data.
                \vspace{\dp0}
            } \end{minipage} \\ \cline{2-3}
            & Expected Result &
                \begin{minipage}[t]{13cm}{\footnotesize
                Measured astrometry metrics that are within reasonable values given the
(known) input dataset.

                \vspace{\dp0}
                } \end{minipage}
        \\ \midrule
    \end{longtable}

\subsection{LVV-T378 - Verify Calculation of Astrometric Performance Metrics}\label{lvv-t378}

\begin{longtable}[]{llllll}
\toprule
Version & Status & Priority & Verification Type & Owner
\\\midrule
1 & Approved & Normal &
Test & Leanne Guy
\\\bottomrule
\multicolumn{6}{c}{ Open \href{https://jira.lsstcorp.org/secure/Tests.jspa\#/testCase/LVV-T378}{LVV-T378} in Jira } \\
\end{longtable}

\subsubsection{Verification Elements}
\begin{itemize}
\item \href{https://jira.lsstcorp.org/browse/LVV-9778}{LVV-9778} - DMS-REQ-0360-V-12: RMS difference between r-band and other filter
separation

\item \href{https://jira.lsstcorp.org/browse/LVV-9777}{LVV-9777} - DMS-REQ-0360-V-11: Max fraction of r-band color difference outliers

\item \href{https://jira.lsstcorp.org/browse/LVV-9779}{LVV-9779} - DMS-REQ-0360-V-13: Max fraction exceeding limit on 200 arcmin scales

\item \href{https://jira.lsstcorp.org/browse/LVV-9773}{LVV-9773} - DMS-REQ-0360-V-07: Outlier limit on 5 arcmin scales

\item \href{https://jira.lsstcorp.org/browse/LVV-9770}{LVV-9770} - DMS-REQ-0360-V-05: Outlier limit on 20 arcmin scales

\item \href{https://jira.lsstcorp.org/browse/LVV-9775}{LVV-9775} - DMS-REQ-0360-V-09: Outlier limit on 200 arcmin scales

\item \href{https://jira.lsstcorp.org/browse/LVV-9769}{LVV-9769} - DMS-REQ-0360-V-04: Median absolute error in RA, Dec

\item \href{https://jira.lsstcorp.org/browse/LVV-9774}{LVV-9774} - DMS-REQ-0360-V-08: Median astrometric error on 200 arcmin scales

\item \href{https://jira.lsstcorp.org/browse/LVV-9768}{LVV-9768} - DMS-REQ-0360-V-03: Median astrometric error on 5 arcmin scales

\item \href{https://jira.lsstcorp.org/browse/LVV-9771}{LVV-9771} - DMS-REQ-0360-V-06: Color difference outlier limit relative to r-band

\item \href{https://jira.lsstcorp.org/browse/LVV-9776}{LVV-9776} - DMS-REQ-0360-V-10: Max fraction exceeding limit on 20 arcmin scales

\item \href{https://jira.lsstcorp.org/browse/LVV-9767}{LVV-9767} - DMS-REQ-0360-V-02: Max fraction exceeding limit on 5 arcmin scales

\end{itemize}

\subsubsection{Test Items}
Verify that the DMS system provides software to calculate astrometric
performance metrics, and that the algorithms are properly calculating
the desired quantities. Note that because the DMS requirement is that
the software shall be provided (and not on the actual measured values of
the metrics), we verify all of the requirements via a single test case.


\subsubsection{Predecessors}

\subsubsection{Environment Needs}

\paragraph{Software}

\paragraph{Hardware}

\subsubsection{Input Specification}

\subsubsection{Output Specification}

\subsubsection{Test Procedure}
    \begin{longtable}[]{p{1.3cm}p{2cm}p{13cm}}
    %\toprule
    Step & \multicolumn{2}{@{}l}{Description, Input Data and Expected Result} \\ \toprule
    \endhead

                \multirow{3}{*}{\parbox{1.3cm}{ 1-1
                {\scriptsize from \hyperref[lvv-t987]
                {LVV-T987} } } }

                & {\small Description} &
                \begin{minipage}[t]{13cm}{\scriptsize
                Identify the path to the data repository, which we will refer to as
`DATA/path', then execute the following:

                \vspace{\dp0}
                } \end{minipage} \\ \cdashline{2-3}
                & {\small Test Data} &
                \begin{minipage}[t]{13cm}{\scriptsize
                } \end{minipage} \\ \cdashline{2-3}
                & {\small Expected Result} &
                    \begin{minipage}[t]{13cm}{\scriptsize
                    Butler repo available for reading.

                    \vspace{\dp0}
                    } \end{minipage}
                \\ \hdashline


        \\ \midrule

            \multirow{3}{*}{ 2 } & Description &
            \begin{minipage}[t]{13cm}{\footnotesize
            Point the butler to an appropriate (precursor or simulated) dataset
containing data in all filters, that is sufficient for the purposes of
measuring astrometric performance metrics.

            \vspace{\dp0}
            } \end{minipage} \\ \cline{2-3}
            & Test Data &
            \begin{minipage}[t]{13cm}{\footnotesize
                No data.
                \vspace{\dp0}
            } \end{minipage} \\ \cline{2-3}
            & Expected Result &
        \\ \midrule

            \multirow{3}{*}{ 3 } & Description &
            \begin{minipage}[t]{13cm}{\footnotesize
            Execute the LSST Stack package `validate\_drp` (or an alternate package
that is relevant) on this dataset to perform the measurements of the
metrics.

            \vspace{\dp0}
            } \end{minipage} \\ \cline{2-3}
            & Test Data &
            \begin{minipage}[t]{13cm}{\footnotesize
                No data.
                \vspace{\dp0}
            } \end{minipage} \\ \cline{2-3}
            & Expected Result &
                \begin{minipage}[t]{13cm}{\footnotesize
                Measurements of validation metrics and the presence of QA plots
resulting from the validation pipeline.

                \vspace{\dp0}
                } \end{minipage}
        \\ \midrule

            \multirow{3}{*}{ 4 } & Description &
            \begin{minipage}[t]{13cm}{\footnotesize
            Compare measured astrometry to known (for simulated data) or measured
(if using precursor data) values from input (precursor or simulated)
data, and confirm that the output values for all of the astrometric
performance metrics are as expected.

            \vspace{\dp0}
            } \end{minipage} \\ \cline{2-3}
            & Test Data &
            \begin{minipage}[t]{13cm}{\footnotesize
                No data.
                \vspace{\dp0}
            } \end{minipage} \\ \cline{2-3}
            & Expected Result &
                \begin{minipage}[t]{13cm}{\footnotesize
                Measured astrometry metrics that are within reasonable values given the
(known) input dataset.

                \vspace{\dp0}
                } \end{minipage}
        \\ \midrule
    \end{longtable}

\subsection{LVV-T385 - Verify implementation of minimum number of simultaneous retrievals of
CCD-sized coadd cutouts}\label{lvv-t385}

\begin{longtable}[]{llllll}
\toprule
Version & Status & Priority & Verification Type & Owner
\\\midrule
1 & Defined & Normal &
Test & Leanne Guy
\\\bottomrule
\multicolumn{6}{c}{ Open \href{https://jira.lsstcorp.org/secure/Tests.jspa\#/testCase/LVV-T385}{LVV-T385} in Jira } \\
\end{longtable}

\subsubsection{Verification Elements}
\begin{itemize}
\item \href{https://jira.lsstcorp.org/browse/LVV-3394}{LVV-3394} - DMS-REQ-0377-V-01: Min number of simultaneous single-CCD coadd cutout
image users

\end{itemize}

\subsubsection{Test Items}
Verify that at least \textbf{ccdRetrievalUsers = 20~}users can
simultaneously retrieve a single CCD-sized coadd cutout using the IVOA
SODA protocol.~


\subsubsection{Predecessors}

\subsubsection{Environment Needs}

\paragraph{Software}

\paragraph{Hardware}

\subsubsection{Input Specification}

\subsubsection{Output Specification}

\subsubsection{Test Procedure}
    \begin{longtable}[]{p{1.3cm}p{2cm}p{13cm}}
    %\toprule
    Step & \multicolumn{2}{@{}l}{Description, Input Data and Expected Result} \\ \toprule
    \endhead

            \multirow{3}{*}{ 1 } & Description &
            \begin{minipage}[t]{13cm}{\footnotesize
            Confirm that CCD-sized cutouts from coadds, also containing mask and
variance planes, are available on the SODA server. If none are
available, copy an image (or some images) to the server.

            \vspace{\dp0}
            } \end{minipage} \\ \cline{2-3}
            & Test Data &
            \begin{minipage}[t]{13cm}{\footnotesize
                No data.
                \vspace{\dp0}
            } \end{minipage} \\ \cline{2-3}
            & Expected Result &
                \begin{minipage}[t]{13cm}{\footnotesize
                At least one CCD-sized coadd cutout is available, and is a well-formed
image.

                \vspace{\dp0}
                } \end{minipage}
        \\ \midrule

            \multirow{3}{*}{ 2 } & Description &
            \begin{minipage}[t]{13cm}{\footnotesize
            Simulate SODA queries by at least \textbf{ccdRetrievalUsers = 20~}users
at the same time.

            \vspace{\dp0}
            } \end{minipage} \\ \cline{2-3}
            & Test Data &
            \begin{minipage}[t]{13cm}{\footnotesize
                No data.
                \vspace{\dp0}
            } \end{minipage} \\ \cline{2-3}
            & Expected Result &
        \\ \midrule

            \multirow{3}{*}{ 3 } & Description &
            \begin{minipage}[t]{13cm}{\footnotesize
            Confirm that all simulated users retrieved the desired image(s), and
that the returned images are well-formed, with (at least) image, mask,
and variance planes.

            \vspace{\dp0}
            } \end{minipage} \\ \cline{2-3}
            & Test Data &
            \begin{minipage}[t]{13cm}{\footnotesize
                No data.
                \vspace{\dp0}
            } \end{minipage} \\ \cline{2-3}
            & Expected Result &
                \begin{minipage}[t]{13cm}{\footnotesize
                All of the simulated~\textbf{ccdRetrievalUsers = 20~}users retrieved
images within the specified time (see related Verification Element and
Test Case).

                \vspace{\dp0}
                } \end{minipage}
        \\ \midrule
    \end{longtable}

\subsection{LVV-T454 - LDM-503-8 Enable LSP viewing of spectrograph data.}\label{lvv-t454}

\begin{longtable}[]{llllll}
\toprule
Version & Status & Priority & Verification Type & Owner
\\\midrule
1 & Approved & Normal &
Test & Michelle Gower
\\\bottomrule
\multicolumn{6}{c}{ Open \href{https://jira.lsstcorp.org/secure/Tests.jspa\#/testCase/LVV-T454}{LVV-T454} in Jira } \\
\end{longtable}

\subsubsection{Verification Elements}
\begin{itemize}
\item \href{https://jira.lsstcorp.org/browse/LVV-140}{LVV-140} - DMS-REQ-0309-V-01: Raw Data Archiving Reliability

\end{itemize}

\subsubsection{Test Items}
\begin{itemize}
\tightlist
\item
  Acquire spectrograph image data, transfer that data to NCSA, ingest
  data into a Butler (G2 or G3 when available), and enable viewing of
  data on LSP. ~
\end{itemize}


\subsubsection{Predecessors}
LDM-503-4b

\subsubsection{Environment Needs}

\paragraph{Software}

\paragraph{Hardware}
ATS storage server system housed with spectrograph.~ ~Receiver system at
NCSA for data.~~

\subsubsection{Input Specification}
Data must be well formed on Spectrograph data archiving system (ATS).
~Well-formed means ``good image'' and correct headers. (\citeds{LSE-400}) ~

\subsubsection{Output Specification}

\subsubsection{Test Procedure}
    \begin{longtable}[]{p{1.3cm}p{2cm}p{13cm}}
    %\toprule
    Step & \multicolumn{2}{@{}l}{Description, Input Data and Expected Result} \\ \toprule
    \endhead

            \multirow{3}{*}{ 1 } & Description &
            \begin{minipage}[t]{13cm}{\footnotesize
            Have data on the ATS archiver system from the spectrograph.~

            \vspace{\dp0}
            } \end{minipage} \\ \cline{2-3}
            & Test Data &
            \begin{minipage}[t]{13cm}{\footnotesize
                No data.
                \vspace{\dp0}
            } \end{minipage} \\ \cline{2-3}
            & Expected Result &
                \begin{minipage}[t]{13cm}{\footnotesize
                Well formed files on the ATS system that need to be transferred to NCSA
for further analysis

                \vspace{\dp0}
                } \end{minipage}
        \\ \midrule

            \multirow{3}{*}{ 2 } & Description &
            \begin{minipage}[t]{13cm}{\footnotesize
            A first few iterations is the human runs script to transfer data to NCSA
through secure pipeline. ~after the process is unchanging/solid, a
cronjob starts up data ``sync'' process. ~

            \vspace{\dp0}
            } \end{minipage} \\ \cline{2-3}
            & Test Data &
            \begin{minipage}[t]{13cm}{\footnotesize
                No data.
                \vspace{\dp0}
            } \end{minipage} \\ \cline{2-3}
            & Expected Result &
                \begin{minipage}[t]{13cm}{\footnotesize
                Data is transferred to NCSA, and is located in NCSA file
systems.\\[2\baselineskip]

                \vspace{\dp0}
                } \end{minipage}
        \\ \midrule

            \multirow{3}{*}{ 3 } & Description &
            \begin{minipage}[t]{13cm}{\footnotesize
            All files transferred have a ButlerG2 (or G3 when ready) ingest
process.~

            \vspace{\dp0}
            } \end{minipage} \\ \cline{2-3}
            & Test Data &
            \begin{minipage}[t]{13cm}{\footnotesize
                No data.
                \vspace{\dp0}
            } \end{minipage} \\ \cline{2-3}
            & Expected Result &
                \begin{minipage}[t]{13cm}{\footnotesize
                files now can be accessed by Butler access methods\\[2\baselineskip]

                \vspace{\dp0}
                } \end{minipage}
        \\ \midrule

            \multirow{3}{*}{ 4 } & Description &
            \begin{minipage}[t]{13cm}{\footnotesize
            LSP processes can now view spectrograph generate files~

            \vspace{\dp0}
            } \end{minipage} \\ \cline{2-3}
            & Test Data &
            \begin{minipage}[t]{13cm}{\footnotesize
                No data.
                \vspace{\dp0}
            } \end{minipage} \\ \cline{2-3}
            & Expected Result &
                \begin{minipage}[t]{13cm}{\footnotesize
                LSP jupyter notebooks can view spectrograph files.\\[2\baselineskip]

                \vspace{\dp0}
                } \end{minipage}
        \\ \midrule
    \end{longtable}

\subsection{LVV-T1085 - Short Queries Functional Test}\label{lvv-t1085}

\begin{longtable}[]{llllll}
\toprule
Version & Status & Priority & Verification Type & Owner
\\\midrule
1 & Approved & Normal &
Test & Fritz Mueller
\\\bottomrule
\multicolumn{6}{c}{ Open \href{https://jira.lsstcorp.org/secure/Tests.jspa\#/testCase/LVV-T1085}{LVV-T1085} in Jira } \\
\end{longtable}

\subsubsection{Verification Elements}
\begin{itemize}
\item \href{https://jira.lsstcorp.org/browse/LVV-33}{LVV-33} - DMS-REQ-0075-V-01: Catalog Queries

\item \href{https://jira.lsstcorp.org/browse/LVV-9787}{LVV-9787} - DMS-REQ-0356-V-04: Max time to retrieve low-volume query results

\end{itemize}

\subsubsection{Test Items}
The objective of this test is to ensure that the short queries are
performing as expected and establish a timing baseline benchmark for
these types of queries.


\subsubsection{Predecessors}

\subsubsection{Environment Needs}

\paragraph{Software}

\paragraph{Hardware}

\subsubsection{Input Specification}
QSERV has been set-up following procedure at ~LVV-T1017.

\subsubsection{Output Specification}

\subsubsection{Test Procedure}
    \begin{longtable}[]{p{1.3cm}p{2cm}p{13cm}}
    %\toprule
    Step & \multicolumn{2}{@{}l}{Description, Input Data and Expected Result} \\ \toprule
    \endhead

            \multirow{3}{*}{ 1 } & Description &
            \begin{minipage}[t]{13cm}{\footnotesize
            Execute single object selection:\\[2\baselineskip]\textbf{SELECT} *
\textbf{FROM} Object~\textbf{WHERE} deepSourceId =
9292041530376264\\[2\baselineskip]and record execution time.

            \vspace{\dp0}
            } \end{minipage} \\ \cline{2-3}
            & Test Data &
            \begin{minipage}[t]{13cm}{\footnotesize
                No data.
                \vspace{\dp0}
            } \end{minipage} \\ \cline{2-3}
            & Expected Result &
                \begin{minipage}[t]{13cm}{\footnotesize
                Query runs in less than 10 seconds.

                \vspace{\dp0}
                } \end{minipage}
        \\ \midrule

            \multirow{3}{*}{ 2 } & Description &
            \begin{minipage}[t]{13cm}{\footnotesize
            Execute spatial area selection from
Object:\\[2\baselineskip]\textbf{SELECT COUNT(*)} \textbf{FROM} Object
\textbf{WHERE}~\\

~qserv\_areaspec\_box(316.582327, −6.839078, 316.653938, −6.781822)

and record execution time.

            \vspace{\dp0}
            } \end{minipage} \\ \cline{2-3}
            & Test Data &
            \begin{minipage}[t]{13cm}{\footnotesize
                No data.
                \vspace{\dp0}
            } \end{minipage} \\ \cline{2-3}
            & Expected Result &
                \begin{minipage}[t]{13cm}{\footnotesize
                Query runs in less than 10 seconds.

                \vspace{\dp0}
                } \end{minipage}
        \\ \midrule
    \end{longtable}

\subsection{LVV-T1086 - Full Table Scans Functional Test}\label{lvv-t1086}

\begin{longtable}[]{llllll}
\toprule
Version & Status & Priority & Verification Type & Owner
\\\midrule
1 & Approved & Normal &
Test & Fritz Mueller
\\\bottomrule
\multicolumn{6}{c}{ Open \href{https://jira.lsstcorp.org/secure/Tests.jspa\#/testCase/LVV-T1086}{LVV-T1086} in Jira } \\
\end{longtable}

\subsubsection{Verification Elements}
\begin{itemize}
\item \href{https://jira.lsstcorp.org/browse/LVV-33}{LVV-33} - DMS-REQ-0075-V-01: Catalog Queries

\item \href{https://jira.lsstcorp.org/browse/LVV-188}{LVV-188} - DMS-REQ-0357-V-01: Result latency for high-volume full-sky queries on
the Object table

\item \href{https://jira.lsstcorp.org/browse/LVV-185}{LVV-185} - DMS-REQ-0354-V-01: Result latency for high-volume complex queries

\end{itemize}

\subsubsection{Test Items}
The objective of this test is to ensure that the full table scan queries
are performing as expected and establish a timing baseline benchmark for
these types of queries.


\subsubsection{Predecessors}

\subsubsection{Environment Needs}

\paragraph{Software}

\paragraph{Hardware}

\subsubsection{Input Specification}
QSERV has been set-up following procedure at ~LVV-T1017.

\subsubsection{Output Specification}

\subsubsection{Test Procedure}
    \begin{longtable}[]{p{1.3cm}p{2cm}p{13cm}}
    %\toprule
    Step & \multicolumn{2}{@{}l}{Description, Input Data and Expected Result} \\ \toprule
    \endhead

            \multirow{3}{*}{ 1 } & Description &
            \begin{minipage}[t]{13cm}{\footnotesize
            Execute query:\\[2\baselineskip]\textbf{SELECT} ra , decl , u\_psfFlux ,
g\_psfFlux , r\_psfFlux \textbf{FROM} Object\\
\textbf{WHERE} y\_shapeIxx \textbf{BETWEEN} 20 \textbf{AND}
20.1\\[3\baselineskip]and record execution time and output size.

            \vspace{\dp0}
            } \end{minipage} \\ \cline{2-3}
            & Test Data &
            \begin{minipage}[t]{13cm}{\footnotesize
                No data.
                \vspace{\dp0}
            } \end{minipage} \\ \cline{2-3}
            & Expected Result &
                \begin{minipage}[t]{13cm}{\footnotesize
                Query expected to run in less than 1 hour.\\[2\baselineskip]

                \vspace{\dp0}
                } \end{minipage}
        \\ \midrule

            \multirow{3}{*}{ 2 } & Description &
            \begin{minipage}[t]{13cm}{\footnotesize
            Execute query:\\[2\baselineskip]\textbf{SELECT} COUNT(*) \textbf{FROM}
Source \textbf{WHERE} flux\_sinc \textbf{BETWEEN} 1 \textbf{AND}
1.1\\[2\baselineskip]and record the execution time

            \vspace{\dp0}
            } \end{minipage} \\ \cline{2-3}
            & Test Data &
            \begin{minipage}[t]{13cm}{\footnotesize
                No data.
                \vspace{\dp0}
            } \end{minipage} \\ \cline{2-3}
            & Expected Result &
                \begin{minipage}[t]{13cm}{\footnotesize
                Query expected to run in less than 12 hours.

                \vspace{\dp0}
                } \end{minipage}
        \\ \midrule

            \multirow{3}{*}{ 3 } & Description &
            \begin{minipage}[t]{13cm}{\footnotesize
            Execute query:\\[2\baselineskip]\textbf{SELECT} COUNT(*) \textbf{FROM}
ForcedSource \textbf{WHERE} psfFlux \textbf{BETWEEN} 0.1 \textbf{AND}
0.2\\[2\baselineskip]and record the execution time

            \vspace{\dp0}
            } \end{minipage} \\ \cline{2-3}
            & Test Data &
            \begin{minipage}[t]{13cm}{\footnotesize
                No data.
                \vspace{\dp0}
            } \end{minipage} \\ \cline{2-3}
            & Expected Result &
                \begin{minipage}[t]{13cm}{\footnotesize
                Query expected to run in less than 12 hours.

                \vspace{\dp0}
                } \end{minipage}
        \\ \midrule
    \end{longtable}

\subsection{LVV-T1087 - Full Table Joins Functional Test}\label{lvv-t1087}

\begin{longtable}[]{llllll}
\toprule
Version & Status & Priority & Verification Type & Owner
\\\midrule
1 & Approved & Normal &
Test & Fritz Mueller
\\\bottomrule
\multicolumn{6}{c}{ Open \href{https://jira.lsstcorp.org/secure/Tests.jspa\#/testCase/LVV-T1087}{LVV-T1087} in Jira } \\
\end{longtable}

\subsubsection{Verification Elements}
\begin{itemize}
\item \href{https://jira.lsstcorp.org/browse/LVV-33}{LVV-33} - DMS-REQ-0075-V-01: Catalog Queries

\item \href{https://jira.lsstcorp.org/browse/LVV-185}{LVV-185} - DMS-REQ-0354-V-01: Result latency for high-volume complex queries

\end{itemize}

\subsubsection{Test Items}
The objective of this test is to ensure that the full table join queries
are performing as expected and establish a timing baseline benchmark for
these types of queries.


\subsubsection{Predecessors}

\subsubsection{Environment Needs}

\paragraph{Software}

\paragraph{Hardware}

\subsubsection{Input Specification}
QSERV has been set-up following procedure at ~LVV-T1017.

\subsubsection{Output Specification}

\subsubsection{Test Procedure}
    \begin{longtable}[]{p{1.3cm}p{2cm}p{13cm}}
    %\toprule
    Step & \multicolumn{2}{@{}l}{Description, Input Data and Expected Result} \\ \toprule
    \endhead

            \multirow{3}{*}{ 1 } & Description &
            \begin{minipage}[t]{13cm}{\footnotesize
            Execute query:\\[2\baselineskip]\textbf{SELECT} o.deepSourceId,
s.objectId, s.id, o.ra, o.decl\\
\textbf{~ ~ FROM} Object o, Source s WHERE o.deepSourceId=s.objectId\\
\hspace*{0.333em} ~ \textbf{AND} s . flux\_sinc \textbf{BETWEEN} 0.3
\textbf{AND} 0.31\\[2\baselineskip]and record execution time.

            \vspace{\dp0}
            } \end{minipage} \\ \cline{2-3}
            & Test Data &
            \begin{minipage}[t]{13cm}{\footnotesize
                No data.
                \vspace{\dp0}
            } \end{minipage} \\ \cline{2-3}
            & Expected Result &
                \begin{minipage}[t]{13cm}{\footnotesize
                Query expected to run in less than 12 hours.

                \vspace{\dp0}
                } \end{minipage}
        \\ \midrule

            \multirow{3}{*}{ 2 } & Description &
            \begin{minipage}[t]{13cm}{\footnotesize
            Execute query:\\[2\baselineskip]\textbf{SELECT} o.deepSourceId,
f.psfFlux \textbf{FROM} Object o, ForcedSource f\\
\textbf{~ ~ WHERE} o.deepSourceId=f.deepSourceId\\
\textbf{~ ~ AND} f . psfFlux \textbf{BETWEEN} 0.13 \textbf{AND}
0.14\\[2\baselineskip]and record execution time.

            \vspace{\dp0}
            } \end{minipage} \\ \cline{2-3}
            & Test Data &
            \begin{minipage}[t]{13cm}{\footnotesize
                No data.
                \vspace{\dp0}
            } \end{minipage} \\ \cline{2-3}
            & Expected Result &
                \begin{minipage}[t]{13cm}{\footnotesize
                Query expected to run in less than 12 hours.

                \vspace{\dp0}
                } \end{minipage}
        \\ \midrule
    \end{longtable}

\subsection{LVV-T1088 - Concurrent Scans Scaling Test}\label{lvv-t1088}

\begin{longtable}[]{llllll}
\toprule
Version & Status & Priority & Verification Type & Owner
\\\midrule
1 & Approved & Normal &
Test & Fritz Mueller
\\\bottomrule
\multicolumn{6}{c}{ Open \href{https://jira.lsstcorp.org/secure/Tests.jspa\#/testCase/LVV-T1088}{LVV-T1088} in Jira } \\
\end{longtable}

\subsubsection{Verification Elements}
\begin{itemize}
\item \href{https://jira.lsstcorp.org/browse/LVV-185}{LVV-185} - DMS-REQ-0354-V-01: Result latency for high-volume complex queries

\item \href{https://jira.lsstcorp.org/browse/LVV-188}{LVV-188} - DMS-REQ-0357-V-01: Result latency for high-volume full-sky queries on
the Object table

\item \href{https://jira.lsstcorp.org/browse/LVV-3403}{LVV-3403} - DMS-REQ-0361-V-01: Simultaneous users for high-volume queries

\end{itemize}

\subsubsection{Test Items}
This test will show that average completion-time of full-scan queries of
the Object catalog table grows sub-linearly with respect to the number
of simultaneously active full-scan queries, within the limits of machine
resource exhaustion.


\subsubsection{Predecessors}

\subsubsection{Environment Needs}

\paragraph{Software}

\paragraph{Hardware}

\subsubsection{Input Specification}
\begin{enumerate}
\tightlist
\item
  A test catalog of appropriate size (see schedule detail in \citeds{LDM-552},
  section 2.2.1), prepared and ingested into the Qserv instance under
  test as detailed in LVV-T1017.
\item
  The concurrency load execution script, runQueries.py, maintained in
  the LSST Qserv github repository here:
  https://github.com/lsst/qserv/blob/master/admin/tools/docker/deployment/in2p3/runQueries.py
\end{enumerate}

\subsubsection{Output Specification}

\subsubsection{Test Procedure}
    \begin{longtable}[]{p{1.3cm}p{2cm}p{13cm}}
    %\toprule
    Step & \multicolumn{2}{@{}l}{Description, Input Data and Expected Result} \\ \toprule
    \endhead

            \multirow{3}{*}{ 1 } & Description &
            \begin{minipage}[t]{13cm}{\footnotesize
            Repeat steps 2 through 5 below, where ``pool of interest'' is taken
first to be ``FTSObj'' and subsequently ``FTSSrc'':

            \vspace{\dp0}
            } \end{minipage} \\ \cline{2-3}
            & Test Data &
            \begin{minipage}[t]{13cm}{\footnotesize
                No data.
                \vspace{\dp0}
            } \end{minipage} \\ \cline{2-3}
            & Expected Result &
                \begin{minipage}[t]{13cm}{\footnotesize
                At end of each pass, a graph indicating scan scaling rate and machine
resource exhaustion cutoff.

                \vspace{\dp0}
                } \end{minipage}
        \\ \midrule

            \multirow{3}{*}{ 2 } & Description &
            \begin{minipage}[t]{13cm}{\footnotesize
            Inspect and modify the CONCURRENCY and TARGET\_RATES dictionaries in the
runQueries.py script. Set CONCURRENCY initially to 1 for the query pool
of interest, and to 0 for all other query pools. Set TARGET\_RATES for
the query pool of interest to the yearly value per table in LDM-552,
section 2.2.1.

            \vspace{\dp0}
            } \end{minipage} \\ \cline{2-3}
            & Test Data &
            \begin{minipage}[t]{13cm}{\footnotesize
                No data.
                \vspace{\dp0}
            } \end{minipage} \\ \cline{2-3}
            & Expected Result &
                \begin{minipage}[t]{13cm}{\footnotesize
                rueQueries.py script updated with appropriate values for test iteration

                \vspace{\dp0}
                } \end{minipage}
        \\ \midrule

            \multirow{3}{*}{ 3 } & Description &
            \begin{minipage}[t]{13cm}{\footnotesize
            Execute the runQueries.py script and let it run for at least one, but
preferably several, query cycles.

            \vspace{\dp0}
            } \end{minipage} \\ \cline{2-3}
            & Test Data &
            \begin{minipage}[t]{13cm}{\footnotesize
                No data.
                \vspace{\dp0}
            } \end{minipage} \\ \cline{2-3}
            & Expected Result &
                \begin{minipage}[t]{13cm}{\footnotesize
                Test script executes producing log file.

                \vspace{\dp0}
                } \end{minipage}
        \\ \midrule

            \multirow{3}{*}{ 4 } & Description &
            \begin{minipage}[t]{13cm}{\footnotesize
            Examine log file output and compile performance statistics to obtain a
growth curve point for the pool of interest for the test report.

            \vspace{\dp0}
            } \end{minipage} \\ \cline{2-3}
            & Test Data &
            \begin{minipage}[t]{13cm}{\footnotesize
                No data.
                \vspace{\dp0}
            } \end{minipage} \\ \cline{2-3}
            & Expected Result &
                \begin{minipage}[t]{13cm}{\footnotesize
                Logs indicate either successful test run, providing another growth point
for curve, or errors indicating machine resource exhaustion cutoff has
been reached.

                \vspace{\dp0}
                } \end{minipage}
        \\ \midrule

            \multirow{3}{*}{ 5 } & Description &
            \begin{minipage}[t]{13cm}{\footnotesize
            Adjust the CONCURRENCY value for the pool of interest and repeat from
step 3 to establish the growth trend and machine resource exhaustion
cutoff for the query pool of interest to an acceptable degree of
accuracy.

            \vspace{\dp0}
            } \end{minipage} \\ \cline{2-3}
            & Test Data &
            \begin{minipage}[t]{13cm}{\footnotesize
                No data.
                \vspace{\dp0}
            } \end{minipage} \\ \cline{2-3}
            & Expected Result &
                \begin{minipage}[t]{13cm}{\footnotesize
                Average query execution time for full scan queries of each class should
be demonstrated to grow sub-linearly in the number of concurrent queries
to the limits of machine resource exhaustion.

                \vspace{\dp0}
                } \end{minipage}
        \\ \midrule
    \end{longtable}

\subsection{LVV-T1089 - Load Test}\label{lvv-t1089}

\begin{longtable}[]{llllll}
\toprule
Version & Status & Priority & Verification Type & Owner
\\\midrule
1 & Approved & Normal &
Test & Fritz Mueller
\\\bottomrule
\multicolumn{6}{c}{ Open \href{https://jira.lsstcorp.org/secure/Tests.jspa\#/testCase/LVV-T1089}{LVV-T1089} in Jira } \\
\end{longtable}

\subsubsection{Verification Elements}
\begin{itemize}
\item \href{https://jira.lsstcorp.org/browse/LVV-9786}{LVV-9786} - DMS-REQ-0356-V-03: Min number of simultaneous low-volume query users

\item \href{https://jira.lsstcorp.org/browse/LVV-9787}{LVV-9787} - DMS-REQ-0356-V-04: Max time to retrieve low-volume query results

\item \href{https://jira.lsstcorp.org/browse/LVV-188}{LVV-188} - DMS-REQ-0357-V-01: Result latency for high-volume full-sky queries on
the Object table

\item \href{https://jira.lsstcorp.org/browse/LVV-185}{LVV-185} - DMS-REQ-0354-V-01: Result latency for high-volume complex queries

\item \href{https://jira.lsstcorp.org/browse/LVV-3403}{LVV-3403} - DMS-REQ-0361-V-01: Simultaneous users for high-volume queries

\end{itemize}

\subsubsection{Test Items}
This test will check that Qserv is able to meet average query completion
time targets per query class under a representative load of simultaneous
high and low volume queries while running against an appropriately
scaled test catalog.


\subsubsection{Predecessors}

\subsubsection{Environment Needs}

\paragraph{Software}

\paragraph{Hardware}

\subsubsection{Input Specification}
QSERV has been set-up following procedure at ~LVV-T1017

\subsubsection{Output Specification}

\subsubsection{Test Procedure}
    \begin{longtable}[]{p{1.3cm}p{2cm}p{13cm}}
    %\toprule
    Step & \multicolumn{2}{@{}l}{Description, Input Data and Expected Result} \\ \toprule
    \endhead

            \multirow{3}{*}{ 1 } & Description &
            \begin{minipage}[t]{13cm}{\footnotesize
            Inspect and modify the CONCURRENCY and TARGET\_RATES dictionaries in the
runQueries.py script. ~Set CONCURRENCY and TARGET\_RATES for all pools
to the yearly value per table in LDM-552, section 2.2.1.

            \vspace{\dp0}
            } \end{minipage} \\ \cline{2-3}
            & Test Data &
            \begin{minipage}[t]{13cm}{\footnotesize
                No data.
                \vspace{\dp0}
            } \end{minipage} \\ \cline{2-3}
            & Expected Result &
                \begin{minipage}[t]{13cm}{\footnotesize
                Script updated with appropriate values.

                \vspace{\dp0}
                } \end{minipage}
        \\ \midrule

            \multirow{3}{*}{ 2 } & Description &
            \begin{minipage}[t]{13cm}{\footnotesize
            Execute the runQueries.py script and let it run for 24 hours.

            \vspace{\dp0}
            } \end{minipage} \\ \cline{2-3}
            & Test Data &
            \begin{minipage}[t]{13cm}{\footnotesize
                No data.
                \vspace{\dp0}
            } \end{minipage} \\ \cline{2-3}
            & Expected Result &
                \begin{minipage}[t]{13cm}{\footnotesize
                Script runs without error and produces output log.

                \vspace{\dp0}
                } \end{minipage}
        \\ \midrule

            \multirow{3}{*}{ 3 } & Description &
            \begin{minipage}[t]{13cm}{\footnotesize
            Examine log file output and compile average query execution times per
query type; and compare to yearly target values per table in LDM-552,
section 2.2.1.

            \vspace{\dp0}
            } \end{minipage} \\ \cline{2-3}
            & Test Data &
            \begin{minipage}[t]{13cm}{\footnotesize
                No data.
                \vspace{\dp0}
            } \end{minipage} \\ \cline{2-3}
            & Expected Result &
                \begin{minipage}[t]{13cm}{\footnotesize
                Average query times per query type equal or less than corresponding
yearly target values in LDM-552, section 2.2.1.

                \vspace{\dp0}
                } \end{minipage}
        \\ \midrule
    \end{longtable}

\subsection{LVV-T1090 - Heavy Load Test}\label{lvv-t1090}

\begin{longtable}[]{llllll}
\toprule
Version & Status & Priority & Verification Type & Owner
\\\midrule
1 & Approved & Normal &
Test & Fritz Mueller
\\\bottomrule
\multicolumn{6}{c}{ Open \href{https://jira.lsstcorp.org/secure/Tests.jspa\#/testCase/LVV-T1090}{LVV-T1090} in Jira } \\
\end{longtable}

\subsubsection{Verification Elements}
\begin{itemize}
\item \href{https://jira.lsstcorp.org/browse/LVV-9786}{LVV-9786} - DMS-REQ-0356-V-03: Min number of simultaneous low-volume query users

\item \href{https://jira.lsstcorp.org/browse/LVV-9787}{LVV-9787} - DMS-REQ-0356-V-04: Max time to retrieve low-volume query results

\item \href{https://jira.lsstcorp.org/browse/LVV-188}{LVV-188} - DMS-REQ-0357-V-01: Result latency for high-volume full-sky queries on
the Object table

\item \href{https://jira.lsstcorp.org/browse/LVV-185}{LVV-185} - DMS-REQ-0354-V-01: Result latency for high-volume complex queries

\item \href{https://jira.lsstcorp.org/browse/LVV-3403}{LVV-3403} - DMS-REQ-0361-V-01: Simultaneous users for high-volume queries

\end{itemize}

\subsubsection{Test Items}
This test will check that Qserv is able to meet average query completion
time targets per query class under a higher than average load of
simultaneous high and low volume queries while running against an
appropriately scaled test catalog.


\subsubsection{Predecessors}

\subsubsection{Environment Needs}

\paragraph{Software}

\paragraph{Hardware}

\subsubsection{Input Specification}
QSERV has been set-up following procedure at ~LVV-T1017

\subsubsection{Output Specification}

\subsubsection{Test Procedure}
    \begin{longtable}[]{p{1.3cm}p{2cm}p{13cm}}
    %\toprule
    Step & \multicolumn{2}{@{}l}{Description, Input Data and Expected Result} \\ \toprule
    \endhead

            \multirow{3}{*}{ 1 } & Description &
            \begin{minipage}[t]{13cm}{\footnotesize
            Inspect and modify the CONCURRENCY and TARGET\_RATES dictionaries in the
runQueries.py script. ~Set CONCURRENCY and TARGET\_RATES for LV query
pool to 2020 value per table in LDM-552, section 2.2.1.~ Set CONCURRENCY
and TARGET\_RATES for all other query pools to values in next column
over from current year column (or to 2020 values +10\% if year is 2020)
per table in LDM-552, section 2.2.1.

            \vspace{\dp0}
            } \end{minipage} \\ \cline{2-3}
            & Test Data &
            \begin{minipage}[t]{13cm}{\footnotesize
                No data.
                \vspace{\dp0}
            } \end{minipage} \\ \cline{2-3}
            & Expected Result &
                \begin{minipage}[t]{13cm}{\footnotesize
                Script updated with appropriate values.

                \vspace{\dp0}
                } \end{minipage}
        \\ \midrule

            \multirow{3}{*}{ 2 } & Description &
            \begin{minipage}[t]{13cm}{\footnotesize
            Execute the runQueries.py script and let it run for 24 hrs.

            \vspace{\dp0}
            } \end{minipage} \\ \cline{2-3}
            & Test Data &
            \begin{minipage}[t]{13cm}{\footnotesize
                No data.
                \vspace{\dp0}
            } \end{minipage} \\ \cline{2-3}
            & Expected Result &
                \begin{minipage}[t]{13cm}{\footnotesize
                Script runs without error and produces output log.

                \vspace{\dp0}
                } \end{minipage}
        \\ \midrule

            \multirow{3}{*}{ 3 } & Description &
            \begin{minipage}[t]{13cm}{\footnotesize
            Examine log file output and compile average query execution times per
query type.

            \vspace{\dp0}
            } \end{minipage} \\ \cline{2-3}
            & Test Data &
            \begin{minipage}[t]{13cm}{\footnotesize
                No data.
                \vspace{\dp0}
            } \end{minipage} \\ \cline{2-3}
            & Expected Result &
                \begin{minipage}[t]{13cm}{\footnotesize
                Average query times per query type equal or less than corresponding
yearly target values in LDM-552, section 2.2.1.

                \vspace{\dp0}
                } \end{minipage}
        \\ \midrule
    \end{longtable}

\subsection{LVV-T1097 - Verify Summit Facility Network Implementation}\label{lvv-t1097}

\begin{longtable}[]{llllll}
\toprule
Version & Status & Priority & Verification Type & Owner
\\\midrule
1 & Draft & Normal &
Test & Jeff Kantor
\\\bottomrule
\multicolumn{6}{c}{ Open \href{https://jira.lsstcorp.org/secure/Tests.jspa\#/testCase/LVV-T1097}{LVV-T1097} in Jira } \\
\end{longtable}

\subsubsection{Verification Elements}
\begin{itemize}
\item \href{https://jira.lsstcorp.org/browse/LVV-71}{LVV-71} - DMS-REQ-0168-V-01: Summit Facility Data Communications

\end{itemize}

\subsubsection{Test Items}
Verify that data acquired by a AuxTel DAQ can be transferred to Summit
DWDM and loaded in the EFD without problems.


\subsubsection{Predecessors}
PMCS DMTC-7400-2400 Complete\\
PMCS T\&SC-2600-1545 Complete

\subsubsection{Environment Needs}

\paragraph{Software}
See pre-conditions

\paragraph{Hardware}
See pre-conditions.\\[2\baselineskip]

\subsubsection{Input Specification}
\begin{enumerate}
\tightlist
\item
  Summit Control Network and Camera Data Backbone installed and
  operating properly.
\item
  Summit - Base Network installed and operating properly.
\item
  EITHER: AuxTel hardware and control systems are functional with
  LATISS. AuxTel TCS, AuxTel EFD, AuxTel CCS, AuxTel DAQ are connected
  via Control Network on Summit to Rubin Observatory DWDM (with at least
  2 x 10 Gbps ethernet port client cards) OR: high-quality DAQ
  application-level simulators that match the form, volume, file paths,
  compressibility, and cadence of the expected instrument data, running
  on end node computers that are the production hardware or equivalent
  to it. Scientific validity of the data content is not essential.
\item
  AuxTel Archiver/forwarders installed in Summit and operating properly
  running on end node computers that are the production hardware or
  equivalent to it.
\item
  As-built documentation for all of the above is available.
\end{enumerate}

NOTE: This test will be repeated at increasing data volumes as
additional observatory capabilities (e.g. ComCAM, FullCam) become
available. ~Final verification will be tested at full operational
volume. After the initial test, the corresponding verification elements
will be flagged as ``Requires Monitoring'' such that those requirements
will be closed out as having been verified but will continue to be
monitored throughout commissioning to ensure they do not drop out of
compliance. ~This will also be monitored for end to end Summit - Data
Facility transfers during Commissioning.

\subsubsection{Output Specification}

\subsubsection{Test Procedure}
    \begin{longtable}[]{p{1.3cm}p{2cm}p{13cm}}
    %\toprule
    Step & \multicolumn{2}{@{}l}{Description, Input Data and Expected Result} \\ \toprule
    \endhead

            \multirow{3}{*}{ 1 } & Description &
            \begin{minipage}[t]{13cm}{\footnotesize
            Verify the pre-conditions have been satisfied

            \vspace{\dp0}
            } \end{minipage} \\ \cline{2-3}
            & Test Data &
            \begin{minipage}[t]{13cm}{\footnotesize
                NA

                \vspace{\dp0}
            } \end{minipage} \\ \cline{2-3}
            & Expected Result &
                \begin{minipage}[t]{13cm}{\footnotesize
                Pre-conditions are satisfied.

                \vspace{\dp0}
                } \end{minipage}
        \\ \midrule

            \multirow{3}{*}{ 2 } & Description &
            \begin{minipage}[t]{13cm}{\footnotesize
            Control the AuxTel through a night of Observing. ~While observing, read
out LATISS data and transfer to Rubin Observatory Summit DWDM while
monitoring latency.

            \vspace{\dp0}
            } \end{minipage} \\ \cline{2-3}
            & Test Data &
            \begin{minipage}[t]{13cm}{\footnotesize
                LATISS images and metadata

                \vspace{\dp0}
            } \end{minipage} \\ \cline{2-3}
            & Expected Result &
                \begin{minipage}[t]{13cm}{\footnotesize
                Data is fed to DWDM without delays or errors.

                \vspace{\dp0}
                } \end{minipage}
        \\ \midrule

            \multirow{3}{*}{ 3 } & Description &
            \begin{minipage}[t]{13cm}{\footnotesize
            Verify that data acquired by a AuxTel DAQ can be transferred ~and loaded
in EFD without problems.

            \vspace{\dp0}
            } \end{minipage} \\ \cline{2-3}
            & Test Data &
            \begin{minipage}[t]{13cm}{\footnotesize
                LATISS images and metadata

                \vspace{\dp0}
            } \end{minipage} \\ \cline{2-3}
            & Expected Result &
                \begin{minipage}[t]{13cm}{\footnotesize
                Examine the EFD to ensure that the data has been loaded properly.

                \vspace{\dp0}
                } \end{minipage}
        \\ \midrule
    \end{longtable}

\subsection{LVV-T1168 - Verify Summit - Base Network Integration}\label{lvv-t1168}

\begin{longtable}[]{llllll}
\toprule
Version & Status & Priority & Verification Type & Owner
\\\midrule
1 & Approved & Normal &
Inspection & Jeff Kantor
\\\bottomrule
\multicolumn{6}{c}{ Open \href{https://jira.lsstcorp.org/secure/Tests.jspa\#/testCase/LVV-T1168}{LVV-T1168} in Jira } \\
\end{longtable}

\subsubsection{Verification Elements}
\begin{itemize}
\item \href{https://jira.lsstcorp.org/browse/LVV-73}{LVV-73} - DMS-REQ-0171-V-01: Summit to Base Network

\end{itemize}

\subsubsection{Test Items}
Verify the integration of the summit to base network by demonstrating a
sustained and uninterrupted transfer of data between summit and base
over 1 day period at or exceeding rates specified in \citeds{LDM-142}. Done in 3
phases in collaboration with equipment/installation vendors (see test
procedure).


\subsubsection{Predecessors}
See pre-conditions by phase above.

\subsubsection{Environment Needs}

\paragraph{Software}
perfsonar on DTN.

\paragraph{Hardware}
OTDR, DTN.

\subsubsection{Input Specification}
PMCS DMTC-7400-2330 COMPLETE\\
By phase:

\begin{enumerate}
\tightlist
\item
  Posts from Cerro Pachon to AURA Gatehouse repaired/improved. ~Fiber
  installed on posts from Cerro Pachon to AURA Gatehouse. ~Fiber
  installed from AURA Gatehouse to AURA compound in La Serena. OTDR
  purchased.
\item
  AURA DWDM installed in caseta on Cerro Pachon and in existing computer
  room in La Serena. ~DTN installed in La Serena. ~DTN loaded with
  software and test data staged.
\item
  Base Data Center (BDC) ready for installation of LSST DWDM. ~Fiber
  connecting existing computer room to BDC. ~LSST DWDM equipment
  installed in Summit Computer Room and BDC.
\end{enumerate}

\subsubsection{Output Specification}
Fiber tested to within acceptable Db. ~Bandwidth, latency within
specifications.

\subsubsection{Test Procedure}
    \begin{longtable}[]{p{1.3cm}p{2cm}p{13cm}}
    %\toprule
    Step & \multicolumn{2}{@{}l}{Description, Input Data and Expected Result} \\ \toprule
    \endhead

            \multirow{3}{*}{ 1 } & Description &
            \begin{minipage}[t]{13cm}{\footnotesize
            Test optical fiber with OTDR:\\
Installation of fiber optic cables and Optical Time Domain Reflector
(OTDR) fiber testing (completed 20170602
\href{https://docushare.lsstcorp.org/docushare/dsweb/Get/Document-26270/RD10\%20Report\%20of\%20delivery\%20of\%20LS\%20-\%20AG\%20fiber\%20from\%20Telefonica\%20to\%20REUNA.pdf}{REUNA
deliverable RD10})

            \vspace{\dp0}
            } \end{minipage} \\ \cline{2-3}
            & Test Data &
            \begin{minipage}[t]{13cm}{\footnotesize
                OTDR generated optical data

                \vspace{\dp0}
            } \end{minipage} \\ \cline{2-3}
            & Expected Result &
                \begin{minipage}[t]{13cm}{\footnotesize
                Fiber tested to within acceptable Db.

                \vspace{\dp0}
                } \end{minipage}
        \\ \midrule

            \multirow{3}{*}{ 2 } & Description &
            \begin{minipage}[t]{13cm}{\footnotesize
            Test AURA DWDM:\\
Installation of AURA DWDM and Data Transfer Node (DTN) (completed
20171218
\href{https://docushare.lsst.org/docushare/dsweb/Get/DMTR-82/DMTR-82.pdf}{DMTR-82})

            \vspace{\dp0}
            } \end{minipage} \\ \cline{2-3}
            & Test Data &
            \begin{minipage}[t]{13cm}{\footnotesize
                DTN perfSonar generated data

                \vspace{\dp0}
            } \end{minipage} \\ \cline{2-3}
            & Expected Result &
                \begin{minipage}[t]{13cm}{\footnotesize
                Summit - Base bandwidth and latency within specifications

                \vspace{\dp0}
                } \end{minipage}
        \\ \midrule

            \multirow{3}{*}{ 3 } & Description &
            \begin{minipage}[t]{13cm}{\footnotesize
            Test LSST DWDM:\\
Installation of LSST DWDM and Bit Error Rate Tester (BERT) data
(completed 20190505
\href{https://docushare.lsstcorp.org/docushare/dsweb/View/Collection-7743}{collection-7743},
20191108
\href{https://docushare.lsstcorp.org/docushare/dsweb/Get/Document-35302/DAQ\%20DWDM\%20connection\%20tests\%2020191109.pptx}{DAQ
DWDM Connection Tests})

            \vspace{\dp0}
            } \end{minipage} \\ \cline{2-3}
            & Test Data &
            \begin{minipage}[t]{13cm}{\footnotesize
                BERT generated data

                \vspace{\dp0}
            } \end{minipage} \\ \cline{2-3}
            & Expected Result &
                \begin{minipage}[t]{13cm}{\footnotesize
                Summit - Base bandwidth, latency, bit error rate within specifications

                \vspace{\dp0}
                } \end{minipage}
        \\ \midrule
    \end{longtable}

\subsection{LVV-T1232 - Verify Implementation of Catalog Export Formats From the Portal Aspect}\label{lvv-t1232}

\begin{longtable}[]{llllll}
\toprule
Version & Status & Priority & Verification Type & Owner
\\\midrule
1 & Approved & Normal &
Test & Colin Slater
\\\bottomrule
\multicolumn{6}{c}{ Open \href{https://jira.lsstcorp.org/secure/Tests.jspa\#/testCase/LVV-T1232}{LVV-T1232} in Jira } \\
\end{longtable}

\subsubsection{Verification Elements}
\begin{itemize}
\item \href{https://jira.lsstcorp.org/browse/LVV-35}{LVV-35} - DMS-REQ-0078-V-01: Catalog Export Formats

\end{itemize}

\subsubsection{Test Items}
Verify that catalog data is exportable from the portal aspect in a
variety of community-standard formats.


\subsubsection{Predecessors}

\subsubsection{Environment Needs}

\paragraph{Software}

\paragraph{Hardware}

\subsubsection{Input Specification}

\subsubsection{Output Specification}

\subsubsection{Test Procedure}
    \begin{longtable}[]{p{1.3cm}p{2cm}p{13cm}}
    %\toprule
    Step & \multicolumn{2}{@{}l}{Description, Input Data and Expected Result} \\ \toprule
    \endhead

                \multirow{3}{*}{\parbox{1.3cm}{ 1-1
                {\scriptsize from \hyperref[lvv-t849]
                {LVV-T849} } } }

                & {\small Description} &
                \begin{minipage}[t]{13cm}{\scriptsize
                Navigate to the Portal Aspect endpoint. ~The stable version should be
used for this test and is currently located at:
https://lsst-lsp-stable.ncsa.illinois.edu/portal/app/ .

                \vspace{\dp0}
                } \end{minipage} \\ \cdashline{2-3}
                & {\small Test Data} &
                \begin{minipage}[t]{13cm}{\scriptsize
                } \end{minipage} \\ \cdashline{2-3}
                & {\small Expected Result} &
                    \begin{minipage}[t]{13cm}{\scriptsize
                    A credential-entry screen should be displayed.

                    \vspace{\dp0}
                    } \end{minipage}
                \\ \hdashline


                \multirow{3}{*}{\parbox{1.3cm}{ 1-2
                {\scriptsize from \hyperref[lvv-t849]
                {LVV-T849} } } }

                & {\small Description} &
                \begin{minipage}[t]{13cm}{\scriptsize
                Enter a valid set of credentials for an LSST user with LSP access on the
instance under test.

                \vspace{\dp0}
                } \end{minipage} \\ \cdashline{2-3}
                & {\small Test Data} &
                \begin{minipage}[t]{13cm}{\scriptsize
                } \end{minipage} \\ \cdashline{2-3}
                & {\small Expected Result} &
                    \begin{minipage}[t]{13cm}{\scriptsize
                    The Portal Aspect UI should be displayed following authentication.

                    \vspace{\dp0}
                    } \end{minipage}
                \\ \hdashline


        \\ \midrule

            \multirow{3}{*}{ 2 } & Description &
            \begin{minipage}[t]{13cm}{\footnotesize
            Select query type ``ADQL''.

            \vspace{\dp0}
            } \end{minipage} \\ \cline{2-3}
            & Test Data &
            \begin{minipage}[t]{13cm}{\footnotesize
                No data.
                \vspace{\dp0}
            } \end{minipage} \\ \cline{2-3}
            & Expected Result &
        \\ \midrule

            \multirow{3}{*}{ 3 } & Description &
            \begin{minipage}[t]{13cm}{\footnotesize
            Execute the example query given in the example code below by entering
the text in the ADQL Query box, then clicking ``Search'' at the lower
left corner of the page.

            \vspace{\dp0}
            } \end{minipage} \\ \cline{2-3}
            & Test Data &
            \begin{minipage}[t]{13cm}{\footnotesize
                No data.
                \vspace{\dp0}
            } \end{minipage} \\ \cline{2-3}
                & Example Code &
                \begin{minipage}[t]{13cm}{\footnotesize
                SELECT cntr, ra, decl, w1mpro\_ep, w2mpro\_ep, w3mpro\_ep FROM
wise\_00.allwise\_p3as\_mep WHERE CONTAINS(POINT('ICRS', ra, decl),
CIRCLE('ICRS', 192.85, 27.13, .2)) = 1

                \vspace{\dp0}
                } \end{minipage} \\ \cline{2-3}
            & Expected Result &
                \begin{minipage}[t]{13cm}{\footnotesize
                A new page will load with the search results as a table, with some plots
as well.

                \vspace{\dp0}
                } \end{minipage}
        \\ \midrule

            \multirow{3}{*}{ 4 } & Description &
            \begin{minipage}[t]{13cm}{\footnotesize
            Click the icon that looks like a floppy disk (it says ``Save the content
as an IPAC, CSV, or TSV table'' when you mouse over it).

            \vspace{\dp0}
            } \end{minipage} \\ \cline{2-3}
            & Test Data &
            \begin{minipage}[t]{13cm}{\footnotesize
                No data.
                \vspace{\dp0}
            } \end{minipage} \\ \cline{2-3}
            & Expected Result &
        \\ \midrule

            \multirow{3}{*}{ 5 } & Description &
            \begin{minipage}[t]{13cm}{\footnotesize
            \begin{itemize}
\tightlist
\item
  Select ``CSV'', then specify a destination to save the file on your
  local computer.
\item
  Select ``VOTable'', then specify a destination to save the file on
  your local computer.
\item
  Select ``FITS'', then specify a destination to save the file on your
  local computer.
\end{itemize}

            \vspace{\dp0}
            } \end{minipage} \\ \cline{2-3}
            & Test Data &
            \begin{minipage}[t]{13cm}{\footnotesize
                No data.
                \vspace{\dp0}
            } \end{minipage} \\ \cline{2-3}
            & Expected Result &
        \\ \midrule

            \multirow{3}{*}{ 6 } & Description &
            \begin{minipage}[t]{13cm}{\footnotesize
            Open each of the files (either in TOPCAT, or using Astropy io tools).
Confirm that the data tables are well-formed, and that each table
contains the same columns and the same number of rows.

            \vspace{\dp0}
            } \end{minipage} \\ \cline{2-3}
            & Test Data &
            \begin{minipage}[t]{13cm}{\footnotesize
                No data.
                \vspace{\dp0}
            } \end{minipage} \\ \cline{2-3}
            & Expected Result &
        \\ \midrule

                \multirow{3}{*}{\parbox{1.3cm}{ 7-1
                {\scriptsize from \hyperref[lvv-t850]
                {LVV-T850} } } }

                & {\small Description} &
                \begin{minipage}[t]{13cm}{\scriptsize
                Currently, there is no logout mechanism on the portal.\\
This should be updated as the system matures.\\[2\baselineskip]Simply
close the browser window.

                \vspace{\dp0}
                } \end{minipage} \\ \cdashline{2-3}
                & {\small Test Data} &
                \begin{minipage}[t]{13cm}{\scriptsize
                } \end{minipage} \\ \cdashline{2-3}
                & {\small Expected Result} &
                    \begin{minipage}[t]{13cm}{\scriptsize
                    Closed browser window. ~When navigating to the portal endpoint, expect
to execute the steps in LVV-T849.

                    \vspace{\dp0}
                    } \end{minipage}
                \\ \hdashline


        \\ \midrule
    \end{longtable}

\subsection{LVV-T1240 - Verify implementation of minimum astrometric standards per CCD}\label{lvv-t1240}

\begin{longtable}[]{llllll}
\toprule
Version & Status & Priority & Verification Type & Owner
\\\midrule
1 & Approved & Normal &
Test & Jim Bosch
\\\bottomrule
\multicolumn{6}{c}{ Open \href{https://jira.lsstcorp.org/secure/Tests.jspa\#/testCase/LVV-T1240}{LVV-T1240} in Jira } \\
\end{longtable}

\subsubsection{Verification Elements}
\begin{itemize}
\item \href{https://jira.lsstcorp.org/browse/LVV-9741}{LVV-9741} - DMS-REQ-0030-V-02: Minimum astrometric standards per CCD

\end{itemize}

\subsubsection{Test Items}
Verify that each CCD in a processed dataset had its astrometric solution
determined by at least~\textbf{astrometricMinStandards = 5~}astrometric
standards.


\subsubsection{Predecessors}

\subsubsection{Environment Needs}

\paragraph{Software}

\paragraph{Hardware}

\subsubsection{Input Specification}

\subsubsection{Output Specification}

\subsubsection{Test Procedure}
    \begin{longtable}[]{p{1.3cm}p{2cm}p{13cm}}
    %\toprule
    Step & \multicolumn{2}{@{}l}{Description, Input Data and Expected Result} \\ \toprule
    \endhead

            \multirow{3}{*}{ 1 } & Description &
            \begin{minipage}[t]{13cm}{\footnotesize
            Identify an appropriate processed dataset for this test.

            \vspace{\dp0}
            } \end{minipage} \\ \cline{2-3}
            & Test Data &
            \begin{minipage}[t]{13cm}{\footnotesize
                No data.
                \vspace{\dp0}
            } \end{minipage} \\ \cline{2-3}
            & Expected Result &
                \begin{minipage}[t]{13cm}{\footnotesize
                A dataset with Processed Visit Images.

                \vspace{\dp0}
                } \end{minipage}
        \\ \midrule

                \multirow{3}{*}{\parbox{1.3cm}{ 2-1
                {\scriptsize from \hyperref[lvv-t987]
                {LVV-T987} } } }

                & {\small Description} &
                \begin{minipage}[t]{13cm}{\scriptsize
                Identify the path to the data repository, which we will refer to as
`DATA/path', then execute the following:

                \vspace{\dp0}
                } \end{minipage} \\ \cdashline{2-3}
                & {\small Test Data} &
                \begin{minipage}[t]{13cm}{\scriptsize
                } \end{minipage} \\ \cdashline{2-3}
                & {\small Expected Result} &
                    \begin{minipage}[t]{13cm}{\scriptsize
                    Butler repo available for reading.

                    \vspace{\dp0}
                    } \end{minipage}
                \\ \hdashline


        \\ \midrule

            \multirow{3}{*}{ 3 } & Description &
            \begin{minipage}[t]{13cm}{\footnotesize
            Select a single visit from the dataset, and extract its calibration
data. For a subset of CCDs, check how many astrometric standards
contributed to the solution. Confirm that this number is at
least~\textbf{astrometricMinStandards = 5.}

            \vspace{\dp0}
            } \end{minipage} \\ \cline{2-3}
            & Test Data &
            \begin{minipage}[t]{13cm}{\footnotesize
                No data.
                \vspace{\dp0}
            } \end{minipage} \\ \cline{2-3}
            & Expected Result &
                \begin{minipage}[t]{13cm}{\footnotesize
                At least \textbf{astrometricMinStandards} from each CCD\textbf{~}were
used in determining the WCS solution.

                \vspace{\dp0}
                } \end{minipage}
        \\ \midrule
    \end{longtable}

\subsection{LVV-T1250 - Verify implementation of minimum number of simultaneous DM EFD query
users}\label{lvv-t1250}

\begin{longtable}[]{llllll}
\toprule
Version & Status & Priority & Verification Type & Owner
\\\midrule
1 & Draft & Normal &
Test & Jeffrey Carlin
\\\bottomrule
\multicolumn{6}{c}{ Open \href{https://jira.lsstcorp.org/secure/Tests.jspa\#/testCase/LVV-T1250}{LVV-T1250} in Jira } \\
\end{longtable}

\subsubsection{Verification Elements}
\begin{itemize}
\item \href{https://jira.lsstcorp.org/browse/LVV-3400}{LVV-3400} - DMS-REQ-0358-V-01: Min number of simultaneous DM EFD query users

\end{itemize}

\subsubsection{Test Items}
Verify that the DM EFD can support \textbf{dmEfdQueryUsers~= 5}
simultaneous queries. The additional requirement that each query must
last no more than \textbf{dmEfdQueryTime = 10 seconds~}will be verified
separately in
\href{https://jira.lsstcorp.org/secure/Tests.jspa\#/testCase/LVV-T1251}{LVV-T1251},
but these must be satisfied together.


\subsubsection{Predecessors}

\subsubsection{Environment Needs}

\paragraph{Software}

\paragraph{Hardware}

\subsubsection{Input Specification}

\subsubsection{Output Specification}

\subsubsection{Test Procedure}
    \begin{longtable}[]{p{1.3cm}p{2cm}p{13cm}}
    %\toprule
    Step & \multicolumn{2}{@{}l}{Description, Input Data and Expected Result} \\ \toprule
    \endhead

            \multirow{3}{*}{ 1 } & Description &
            \begin{minipage}[t]{13cm}{\footnotesize
            Send multiple (at least 5) simultaneous queries to the DM EFD.

            \vspace{\dp0}
            } \end{minipage} \\ \cline{2-3}
            & Test Data &
            \begin{minipage}[t]{13cm}{\footnotesize
                No data.
                \vspace{\dp0}
            } \end{minipage} \\ \cline{2-3}
            & Expected Result &
        \\ \midrule

            \multirow{3}{*}{ 2 } & Description &
            \begin{minipage}[t]{13cm}{\footnotesize
            Confirm that (a) the queries executed successfully, and that (b) they
return reasonable results.~

            \vspace{\dp0}
            } \end{minipage} \\ \cline{2-3}
            & Test Data &
            \begin{minipage}[t]{13cm}{\footnotesize
                No data.
                \vspace{\dp0}
            } \end{minipage} \\ \cline{2-3}
            & Expected Result &
        \\ \midrule

            \multirow{3}{*}{ 3 } & Description &
            \begin{minipage}[t]{13cm}{\footnotesize
            Repeat the above steps for different queries, and different numbers of
simultaneous queries, to confirm that the expected performance is met
regardless of the query being executed.

            \vspace{\dp0}
            } \end{minipage} \\ \cline{2-3}
            & Test Data &
            \begin{minipage}[t]{13cm}{\footnotesize
                No data.
                \vspace{\dp0}
            } \end{minipage} \\ \cline{2-3}
            & Expected Result &
        \\ \midrule
    \end{longtable}

\subsection{LVV-T1251 - Verify implementation of maximum time to retrieve DM EFD query results}\label{lvv-t1251}

\begin{longtable}[]{llllll}
\toprule
Version & Status & Priority & Verification Type & Owner
\\\midrule
1 & Draft & Normal &
Test & Jeffrey Carlin
\\\bottomrule
\multicolumn{6}{c}{ Open \href{https://jira.lsstcorp.org/secure/Tests.jspa\#/testCase/LVV-T1251}{LVV-T1251} in Jira } \\
\end{longtable}

\subsubsection{Verification Elements}
\begin{itemize}
\item \href{https://jira.lsstcorp.org/browse/LVV-9788}{LVV-9788} - DMS-REQ-0358-V-02: Max time to retrieve DM EFD query results

\end{itemize}

\subsubsection{Test Items}
Verify that the DM EFD can support \textbf{dmEfdQueryUsers~= 5}
simultaneous queries, with each query must executing in no more than
\textbf{dmEfdQueryTime = 10 seconds.~}The requirement on at least 5
simultaneous queries will be verified separately in
\href{https://jira.lsstcorp.org/secure/Tests.jspa\#/testCase/LVV-T1250}{LVV-T1250},\href{https://jira.lsstcorp.org/secure/Tests.jspa\#/testCase/LVV-T1251}{}
but these must be satisfied together.


\subsubsection{Predecessors}

\subsubsection{Environment Needs}

\paragraph{Software}

\paragraph{Hardware}

\subsubsection{Input Specification}

\subsubsection{Output Specification}

\subsubsection{Test Procedure}
    \begin{longtable}[]{p{1.3cm}p{2cm}p{13cm}}
    %\toprule
    Step & \multicolumn{2}{@{}l}{Description, Input Data and Expected Result} \\ \toprule
    \endhead

            \multirow{3}{*}{ 1 } & Description &
            \begin{minipage}[t]{13cm}{\footnotesize
            Send multiple (at least 5) simultaneous queries to the DM EFD.

            \vspace{\dp0}
            } \end{minipage} \\ \cline{2-3}
            & Test Data &
            \begin{minipage}[t]{13cm}{\footnotesize
                No data.
                \vspace{\dp0}
            } \end{minipage} \\ \cline{2-3}
            & Expected Result &
        \\ \midrule

            \multirow{3}{*}{ 2 } & Description &
            \begin{minipage}[t]{13cm}{\footnotesize
            Confirm that (a) the queries executed successfully, and that (b) they
return reasonable results. Check that the time of execution for all
queries was less than 10 seconds.

            \vspace{\dp0}
            } \end{minipage} \\ \cline{2-3}
            & Test Data &
            \begin{minipage}[t]{13cm}{\footnotesize
                No data.
                \vspace{\dp0}
            } \end{minipage} \\ \cline{2-3}
            & Expected Result &
        \\ \midrule

            \multirow{3}{*}{ 3 } & Description &
            \begin{minipage}[t]{13cm}{\footnotesize
            Repeat the above steps for different queries, and different numbers of
simultaneous queries, to confirm that the expected performance is met
regardless of the query being executed.

            \vspace{\dp0}
            } \end{minipage} \\ \cline{2-3}
            & Test Data &
            \begin{minipage}[t]{13cm}{\footnotesize
                No data.
                \vspace{\dp0}
            } \end{minipage} \\ \cline{2-3}
            & Expected Result &
        \\ \midrule
    \end{longtable}

\subsection{LVV-T1252 - Verify number of simultaneous alert filter users}\label{lvv-t1252}

\begin{longtable}[]{llllll}
\toprule
Version & Status & Priority & Verification Type & Owner
\\\midrule
1 & Defined & Normal &
Test & Eric Bellm
\\\bottomrule
\multicolumn{6}{c}{ Open \href{https://jira.lsstcorp.org/secure/Tests.jspa\#/testCase/LVV-T1252}{LVV-T1252} in Jira } \\
\end{longtable}

\subsubsection{Verification Elements}
\begin{itemize}
\item \href{https://jira.lsstcorp.org/browse/LVV-9748}{LVV-9748} - DMS-REQ-0343-V-02: Number of simultaneous users

\end{itemize}

\subsubsection{Test Items}
Verify that the DMS alert filter service supports \textbf{numBrokerUsers
= 100~}simultaneous brokers.


\subsubsection{Predecessors}

\subsubsection{Environment Needs}

\paragraph{Software}

\paragraph{Hardware}

\subsubsection{Input Specification}

\subsubsection{Output Specification}

\subsubsection{Test Procedure}
    \begin{longtable}[]{p{1.3cm}p{2cm}p{13cm}}
    %\toprule
    Step & \multicolumn{2}{@{}l}{Description, Input Data and Expected Result} \\ \toprule
    \endhead

            \multirow{3}{*}{ 1 } & Description &
            \begin{minipage}[t]{13cm}{\footnotesize
            Create a simulated alert stream.

            \vspace{\dp0}
            } \end{minipage} \\ \cline{2-3}
            & Test Data &
            \begin{minipage}[t]{13cm}{\footnotesize
                No data.
                \vspace{\dp0}
            } \end{minipage} \\ \cline{2-3}
            & Expected Result &
        \\ \midrule

            \multirow{3}{*}{ 2 } & Description &
            \begin{minipage}[t]{13cm}{\footnotesize
            Simultaneously execute user-defined alert filters for at least
\textbf{numBrokerUsers = 100} users, and confirm that the system
successfully filters the stream as requested. Confirm that the bandwidth
requirement of \textbf{numBrokerAlerts = 20~}per user was
met.Simultaneously execute user-defined alert filters for at least 100
users, and confirm that the system successfully filters the stream as
requested.

            \vspace{\dp0}
            } \end{minipage} \\ \cline{2-3}
            & Test Data &
            \begin{minipage}[t]{13cm}{\footnotesize
                No data.
                \vspace{\dp0}
            } \end{minipage} \\ \cline{2-3}
            & Expected Result &
                \begin{minipage}[t]{13cm}{\footnotesize
                All of the (simulated) \textbf{numBrokerUsers = 100~}users successfully
receive their requested filtered alerts.

                \vspace{\dp0}
                } \end{minipage}
        \\ \midrule
    \end{longtable}

\subsection{LVV-T1264 - Verify implementation of archiving camera test data}\label{lvv-t1264}

\begin{longtable}[]{llllll}
\toprule
Version & Status & Priority & Verification Type & Owner
\\\midrule
1 & Approved & Normal &
Test & Robert Gruendl
\\\bottomrule
\multicolumn{6}{c}{ Open \href{https://jira.lsstcorp.org/secure/Tests.jspa\#/testCase/LVV-T1264}{LVV-T1264} in Jira } \\
\end{longtable}

\subsubsection{Verification Elements}
\begin{itemize}
\item \href{https://jira.lsstcorp.org/browse/LVV-9637}{LVV-9637} - DMS-REQ-0372-V-01: Archiving Camera Test Data

\end{itemize}

\subsubsection{Test Items}
Verify that a subset of camera test data has been ingested into Butler
repos and is available through standard data access tools.


\subsubsection{Predecessors}

\subsubsection{Environment Needs}

\paragraph{Software}

\paragraph{Hardware}

\subsubsection{Input Specification}

\subsubsection{Output Specification}

\subsubsection{Test Procedure}
    \begin{longtable}[]{p{1.3cm}p{2cm}p{13cm}}
    %\toprule
    Step & \multicolumn{2}{@{}l}{Description, Input Data and Expected Result} \\ \toprule
    \endhead

            \multirow{3}{*}{ 1 } & Description &
            \begin{minipage}[t]{13cm}{\footnotesize
            Obtain some data on a camera test stand.

            \vspace{\dp0}
            } \end{minipage} \\ \cline{2-3}
            & Test Data &
            \begin{minipage}[t]{13cm}{\footnotesize
                No data.
                \vspace{\dp0}
            } \end{minipage} \\ \cline{2-3}
            & Expected Result &
        \\ \midrule

            \multirow{3}{*}{ 2 } & Description &
            \begin{minipage}[t]{13cm}{\footnotesize
            Wait a sufficient amount of time, then confirm that automatic
transfer/ingest of the data has occurred, and a repo is available at
NCSA.

            \vspace{\dp0}
            } \end{minipage} \\ \cline{2-3}
            & Test Data &
            \begin{minipage}[t]{13cm}{\footnotesize
                No data.
                \vspace{\dp0}
            } \end{minipage} \\ \cline{2-3}
            & Expected Result &
                \begin{minipage}[t]{13cm}{\footnotesize
                The data is present at NCSA in non-empty repos.

                \vspace{\dp0}
                } \end{minipage}
        \\ \midrule

            \multirow{3}{*}{ 3 } & Description &
            \begin{minipage}[t]{13cm}{\footnotesize
            Identify the relevant Butler repo of ingested camera test stand data.

            \vspace{\dp0}
            } \end{minipage} \\ \cline{2-3}
            & Test Data &
            \begin{minipage}[t]{13cm}{\footnotesize
                No data.
                \vspace{\dp0}
            } \end{minipage} \\ \cline{2-3}
            & Expected Result &
        \\ \midrule

                \multirow{3}{*}{\parbox{1.3cm}{ 4-1
                {\scriptsize from \hyperref[lvv-t987]
                {LVV-T987} } } }

                & {\small Description} &
                \begin{minipage}[t]{13cm}{\scriptsize
                Identify the path to the data repository, which we will refer to as
`DATA/path', then execute the following:

                \vspace{\dp0}
                } \end{minipage} \\ \cdashline{2-3}
                & {\small Test Data} &
                \begin{minipage}[t]{13cm}{\scriptsize
                } \end{minipage} \\ \cdashline{2-3}
                & {\small Expected Result} &
                    \begin{minipage}[t]{13cm}{\scriptsize
                    Butler repo available for reading.

                    \vspace{\dp0}
                    } \end{minipage}
                \\ \hdashline


        \\ \midrule

            \multirow{3}{*}{ 5 } & Description &
            \begin{minipage}[t]{13cm}{\footnotesize
            Read various repo data products with the Butler, and confirm that they
contain the expected data.

            \vspace{\dp0}
            } \end{minipage} \\ \cline{2-3}
            & Test Data &
            \begin{minipage}[t]{13cm}{\footnotesize
                No data.
                \vspace{\dp0}
            } \end{minipage} \\ \cline{2-3}
            & Expected Result &
                \begin{minipage}[t]{13cm}{\footnotesize
                Camera test stand data that is well-formed.

                \vspace{\dp0}
                } \end{minipage}
        \\ \midrule
    \end{longtable}

\subsection{LVV-T1276 - Verify implementation of latency of reporting optical transients}\label{lvv-t1276}

\begin{longtable}[]{llllll}
\toprule
Version & Status & Priority & Verification Type & Owner
\\\midrule
1 & Draft & Normal &
Test & Eric Bellm
\\\bottomrule
\multicolumn{6}{c}{ Open \href{https://jira.lsstcorp.org/secure/Tests.jspa\#/testCase/LVV-T1276}{LVV-T1276} in Jira } \\
\end{longtable}

\subsubsection{Verification Elements}
\begin{itemize}
\item \href{https://jira.lsstcorp.org/browse/LVV-9740}{LVV-9740} - DMS-REQ-0004-V-02: Latency of reporting optical transients

\end{itemize}

\subsubsection{Test Items}
Verify that alerts are generated for optical transients
within~\textbf{OTT1 = 1 minute~}of the completion of the readout of the
last image.


\subsubsection{Predecessors}

\subsubsection{Environment Needs}

\paragraph{Software}

\paragraph{Hardware}

\subsubsection{Input Specification}

\subsubsection{Output Specification}

\subsubsection{Test Procedure}
    \begin{longtable}[]{p{1.3cm}p{2cm}p{13cm}}
    %\toprule
    Step & \multicolumn{2}{@{}l}{Description, Input Data and Expected Result} \\ \toprule
    \endhead

            \multirow{3}{*}{ 1 } & Description &
            \begin{minipage}[t]{13cm}{\footnotesize
            Identify a precursor dataset containing raw images (and templates), that
is suitable for testing the Alert Production.

            \vspace{\dp0}
            } \end{minipage} \\ \cline{2-3}
            & Test Data &
            \begin{minipage}[t]{13cm}{\footnotesize
                No data.
                \vspace{\dp0}
            } \end{minipage} \\ \cline{2-3}
            & Expected Result &
        \\ \midrule

                \multirow{3}{*}{\parbox{1.3cm}{ 2-1
                {\scriptsize from \hyperref[lvv-t866]
                {LVV-T866} } } }

                & {\small Description} &
                \begin{minipage}[t]{13cm}{\scriptsize
                Perform the steps of Alert Production (including, but not necessarily
limited to, single frame processing, ISR, source detection/measurement,
PSF estimation, photometric and astrometric calibration, difference
imaging, DIASource detection/measurement, source association). During
Operations, it is presumed that these are automated for a given
dataset.~

                \vspace{\dp0}
                } \end{minipage} \\ \cdashline{2-3}
                & {\small Test Data} &
                \begin{minipage}[t]{13cm}{\scriptsize
                } \end{minipage} \\ \cdashline{2-3}
                & {\small Expected Result} &
                    \begin{minipage}[t]{13cm}{\scriptsize
                    An output dataset including difference images and DIASource and
DIAObject measurements.

                    \vspace{\dp0}
                    } \end{minipage}
                \\ \hdashline


                \multirow{3}{*}{\parbox{1.3cm}{ 2-2
                {\scriptsize from \hyperref[lvv-t866]
                {LVV-T866} } } }

                & {\small Description} &
                \begin{minipage}[t]{13cm}{\scriptsize
                Verify that the expected data products have been produced, and that
catalogs contain reasonable values for measured quantities of interest.

                \vspace{\dp0}
                } \end{minipage} \\ \cdashline{2-3}
                & {\small Test Data} &
                \begin{minipage}[t]{13cm}{\scriptsize
                } \end{minipage} \\ \cdashline{2-3}
                & {\small Expected Result} &
                \\ \hdashline


        \\ \midrule

            \multirow{3}{*}{ 3 } & Description &
            \begin{minipage}[t]{13cm}{\footnotesize
            Time processing of data starting from (pre-ingested) raw files until an
alert is available for distribution; verify that this time is less than
OTT1.

            \vspace{\dp0}
            } \end{minipage} \\ \cline{2-3}
            & Test Data &
            \begin{minipage}[t]{13cm}{\footnotesize
                No data.
                \vspace{\dp0}
            } \end{minipage} \\ \cline{2-3}
            & Expected Result &
                \begin{minipage}[t]{13cm}{\footnotesize
                Alerts are received via the alert stream within OTT1=1 minute from the
time the Alert Production payload was executed.

                \vspace{\dp0}
                } \end{minipage}
        \\ \midrule
    \end{longtable}

\subsection{LVV-T1277 - Verify processing of maximum number of calibration exposures}\label{lvv-t1277}

\begin{longtable}[]{llllll}
\toprule
Version & Status & Priority & Verification Type & Owner
\\\midrule
1 & Draft & Normal &
Test & Kian-Tat Lim
\\\bottomrule
\multicolumn{6}{c}{ Open \href{https://jira.lsstcorp.org/secure/Tests.jspa\#/testCase/LVV-T1277}{LVV-T1277} in Jira } \\
\end{longtable}

\subsubsection{Verification Elements}
\begin{itemize}
\item \href{https://jira.lsstcorp.org/browse/LVV-9745}{LVV-9745} - DMS-REQ-0131-V-02: Max number of calibs to be processed

\end{itemize}

\subsubsection{Test Items}
Verify that as many as \textbf{nCalExpProc = 25} calibration exposures
can be processed together within time calProcTime.


\subsubsection{Predecessors}

\subsubsection{Environment Needs}

\paragraph{Software}

\paragraph{Hardware}

\subsubsection{Input Specification}

\subsubsection{Output Specification}

\subsubsection{Test Procedure}
    \begin{longtable}[]{p{1.3cm}p{2cm}p{13cm}}
    %\toprule
    Step & \multicolumn{2}{@{}l}{Description, Input Data and Expected Result} \\ \toprule
    \endhead

            \multirow{3}{*}{ 1 } & Description &
            \begin{minipage}[t]{13cm}{\footnotesize
            Identify a dataset of raw calibration exposures containing at least
\textbf{nCalExpProc = 25~}exposures. (If it contains more than 25
exposures, use only 25 for the test.)

            \vspace{\dp0}
            } \end{minipage} \\ \cline{2-3}
            & Test Data &
            \begin{minipage}[t]{13cm}{\footnotesize
                No data.
                \vspace{\dp0}
            } \end{minipage} \\ \cline{2-3}
            & Expected Result &
        \\ \midrule

                \multirow{3}{*}{\parbox{1.3cm}{ 2-1
                {\scriptsize from \hyperref[lvv-t1059]
                {LVV-T1059} } } }

                & {\small Description} &
                \begin{minipage}[t]{13cm}{\scriptsize
                Execute the Daily Calibration Products Update payload. The payload uses
raw calibration images and information from the Transformed EFD to
generate a subset of Master Calibration Images and Calibration Database
entries in the Data Backbone.

                \vspace{\dp0}
                } \end{minipage} \\ \cdashline{2-3}
                & {\small Test Data} &
                \begin{minipage}[t]{13cm}{\scriptsize
                } \end{minipage} \\ \cdashline{2-3}
                & {\small Expected Result} &
                \\ \hdashline


                \multirow{3}{*}{\parbox{1.3cm}{ 2-2
                {\scriptsize from \hyperref[lvv-t1059]
                {LVV-T1059} } } }

                & {\small Description} &
                \begin{minipage}[t]{13cm}{\scriptsize
                Confirm that the expected Master Calibration images and Calibration
Database entries are present and well-formed.

                \vspace{\dp0}
                } \end{minipage} \\ \cdashline{2-3}
                & {\small Test Data} &
                \begin{minipage}[t]{13cm}{\scriptsize
                } \end{minipage} \\ \cdashline{2-3}
                & {\small Expected Result} &
                \\ \hdashline


        \\ \midrule

            \multirow{3}{*}{ 3 } & Description &
            \begin{minipage}[t]{13cm}{\footnotesize
            Confirm that the processing completed successfully within
\textbf{calProcTime = 1200 seconds.}

            \vspace{\dp0}
            } \end{minipage} \\ \cline{2-3}
            & Test Data &
            \begin{minipage}[t]{13cm}{\footnotesize
                No data.
                \vspace{\dp0}
            } \end{minipage} \\ \cline{2-3}
            & Expected Result &
                \begin{minipage}[t]{13cm}{\footnotesize
                Calibration products resulting from processed raw calibration exposures
are present within calProcTime, and are well-formed images.

                \vspace{\dp0}
                } \end{minipage}
        \\ \midrule

            \multirow{3}{*}{ 4 } & Description &
            \begin{minipage}[t]{13cm}{\footnotesize
            Perform the test again with \emph{more than} nCalExpProc = 25 images,
and confirm that the processing completes within~\textbf{calProcTime =
1200 seconds.}

            \vspace{\dp0}
            } \end{minipage} \\ \cline{2-3}
            & Test Data &
            \begin{minipage}[t]{13cm}{\footnotesize
                No data.
                \vspace{\dp0}
            } \end{minipage} \\ \cline{2-3}
            & Expected Result &
                \begin{minipage}[t]{13cm}{\footnotesize
                Calibration products resulting from processed raw calibration exposures
are present within calProcTime, and are well-formed images. (To verify
that the test with 25 images was not at the limits of what the software
can handle -- should be able to exceed that bare minimum.)

                \vspace{\dp0}
                } \end{minipage}
        \\ \midrule
    \end{longtable}

\subsection{LVV-T1332 - Verify implementation of maximum time for retrieval of CCD-sized coadd
cutouts}\label{lvv-t1332}

\begin{longtable}[]{llllll}
\toprule
Version & Status & Priority & Verification Type & Owner
\\\midrule
1 & Defined & Normal &
Test & Leanne Guy
\\\bottomrule
\multicolumn{6}{c}{ Open \href{https://jira.lsstcorp.org/secure/Tests.jspa\#/testCase/LVV-T1332}{LVV-T1332} in Jira } \\
\end{longtable}

\subsubsection{Verification Elements}
\begin{itemize}
\item \href{https://jira.lsstcorp.org/browse/LVV-9797}{LVV-9797} - DMS-REQ-0377-V-02: Max time to retrieve single-CCD coadd cutout image

\end{itemize}

\subsubsection{Test Items}
Verify that at least \textbf{ccdRetrievalUsers = 20~}users can retrieve
CCD-sized coadd cutouts using the IVOA SODA protocol within a maximum
retrieval time of~\textbf{ccdRetrievalTime = 15 seconds}.


\subsubsection{Predecessors}

\subsubsection{Environment Needs}

\paragraph{Software}

\paragraph{Hardware}

\subsubsection{Input Specification}

\subsubsection{Output Specification}

\subsubsection{Test Procedure}
    \begin{longtable}[]{p{1.3cm}p{2cm}p{13cm}}
    %\toprule
    Step & \multicolumn{2}{@{}l}{Description, Input Data and Expected Result} \\ \toprule
    \endhead

            \multirow{3}{*}{ 1 } & Description &
            \begin{minipage}[t]{13cm}{\footnotesize
            Confirm that CCD-sized cutouts from coadds, also containing mask and
variance planes, are available on the SODA server. If none are
available, copy an image (or some images) to the server.

            \vspace{\dp0}
            } \end{minipage} \\ \cline{2-3}
            & Test Data &
            \begin{minipage}[t]{13cm}{\footnotesize
                No data.
                \vspace{\dp0}
            } \end{minipage} \\ \cline{2-3}
            & Expected Result &
                \begin{minipage}[t]{13cm}{\footnotesize
                At least one CCD-sized coadd cutout is available, and is a well-formed
image.

                \vspace{\dp0}
                } \end{minipage}
        \\ \midrule

            \multirow{3}{*}{ 2 } & Description &
            \begin{minipage}[t]{13cm}{\footnotesize
            Simulate SODA queries by at least \textbf{ccdRetrievalUsers = 20~}users
at the same time.

            \vspace{\dp0}
            } \end{minipage} \\ \cline{2-3}
            & Test Data &
            \begin{minipage}[t]{13cm}{\footnotesize
                No data.
                \vspace{\dp0}
            } \end{minipage} \\ \cline{2-3}
            & Expected Result &
        \\ \midrule

            \multirow{3}{*}{ 3 } & Description &
            \begin{minipage}[t]{13cm}{\footnotesize
            Monitor the time that each query takes to complete, and confirm that all
simulated users retrieved the desired image(s)
within~\textbf{ccdRetrievalTime = 15 seconds.}

            \vspace{\dp0}
            } \end{minipage} \\ \cline{2-3}
            & Test Data &
            \begin{minipage}[t]{13cm}{\footnotesize
                No data.
                \vspace{\dp0}
            } \end{minipage} \\ \cline{2-3}
            & Expected Result &
                \begin{minipage}[t]{13cm}{\footnotesize
                All of the simulated \textbf{ccdRetrievalUsers = 20~}users retrieved
images within~\textbf{ccdRetrievalTime = 15 seconds.}

                \vspace{\dp0}
                } \end{minipage}
        \\ \midrule
    \end{longtable}

\subsection{LVV-T1524 - Verify Implementation of Exporting MOCs as FITS}\label{lvv-t1524}

\begin{longtable}[]{llllll}
\toprule
Version & Status & Priority & Verification Type & Owner
\\\midrule
1 & Draft & Normal &
Demonstration & Jeffrey Carlin
\\\bottomrule
\multicolumn{6}{c}{ Open \href{https://jira.lsstcorp.org/secure/Tests.jspa\#/testCase/LVV-T1524}{LVV-T1524} in Jira } \\
\end{longtable}

\subsubsection{Verification Elements}
\begin{itemize}
\item \href{https://jira.lsstcorp.org/browse/LVV-18222}{LVV-18222} - DMS-REQ-0384-V-01: Export MOCs As FITS\_1

\end{itemize}

\subsubsection{Test Items}
Verify that the Data Management system provides a means for exporting
the LSST-generated MOCs in the FITS serialization form defined in the
IVOA MOC Recommendation.


\subsubsection{Predecessors}

\subsubsection{Environment Needs}

\paragraph{Software}

\paragraph{Hardware}

\subsubsection{Input Specification}

\subsubsection{Output Specification}

\subsubsection{Test Procedure}
    \begin{longtable}[]{p{1.3cm}p{2cm}p{13cm}}
    %\toprule
    Step & \multicolumn{2}{@{}l}{Description, Input Data and Expected Result} \\ \toprule
    \endhead

            \multirow{3}{*}{ 1 } & Description &
            \begin{minipage}[t]{13cm}{\footnotesize
            
            \vspace{\dp0}
            } \end{minipage} \\ \cline{2-3}
            & Test Data &
            \begin{minipage}[t]{13cm}{\footnotesize
                No data.
                \vspace{\dp0}
            } \end{minipage} \\ \cline{2-3}
            & Expected Result &
        \\ \midrule
    \end{longtable}

\subsection{LVV-T1525 - Verify Implementation of Linkage Between HiPS Maps and Coadded Images}\label{lvv-t1525}

\begin{longtable}[]{llllll}
\toprule
Version & Status & Priority & Verification Type & Owner
\\\midrule
1 & Draft & Normal &
Demonstration & Jeffrey Carlin
\\\bottomrule
\multicolumn{6}{c}{ Open \href{https://jira.lsstcorp.org/secure/Tests.jspa\#/testCase/LVV-T1525}{LVV-T1525} in Jira } \\
\end{longtable}

\subsubsection{Verification Elements}
\begin{itemize}
\item \href{https://jira.lsstcorp.org/browse/LVV-18223}{LVV-18223} - DMS-REQ-0381-V-01: HiPS Linkage to Coadds\_1

\end{itemize}

\subsubsection{Test Items}
Verify that the HiPS maps produced by the Data Management system provide
for straightforward linkage from the HiPS data to the underlying LSST
coadded images, and that this has been implemented using a mechanism
supported by both the LSST Science Platform and by community tools.


\subsubsection{Predecessors}

\subsubsection{Environment Needs}

\paragraph{Software}

\paragraph{Hardware}

\subsubsection{Input Specification}

\subsubsection{Output Specification}

\subsubsection{Test Procedure}
    \begin{longtable}[]{p{1.3cm}p{2cm}p{13cm}}
    %\toprule
    Step & \multicolumn{2}{@{}l}{Description, Input Data and Expected Result} \\ \toprule
    \endhead

            \multirow{3}{*}{ 1 } & Description &
            \begin{minipage}[t]{13cm}{\footnotesize
            
            \vspace{\dp0}
            } \end{minipage} \\ \cline{2-3}
            & Test Data &
            \begin{minipage}[t]{13cm}{\footnotesize
                No data.
                \vspace{\dp0}
            } \end{minipage} \\ \cline{2-3}
            & Expected Result &
        \\ \midrule
    \end{longtable}

\subsection{LVV-T1526 - Verify Availability of Secure and Authenticated HiPS Service}\label{lvv-t1526}

\begin{longtable}[]{llllll}
\toprule
Version & Status & Priority & Verification Type & Owner
\\\midrule
1 & Draft & Normal &
Demonstration & Jeffrey Carlin
\\\bottomrule
\multicolumn{6}{c}{ Open \href{https://jira.lsstcorp.org/secure/Tests.jspa\#/testCase/LVV-T1526}{LVV-T1526} in Jira } \\
\end{longtable}

\subsubsection{Verification Elements}
\begin{itemize}
\item \href{https://jira.lsstcorp.org/browse/LVV-18224}{LVV-18224} - DMS-REQ-0380-V-01: HiPS Service\_1

\end{itemize}

\subsubsection{Test Items}
Verify that the Data Management system includes a secure and
authenticated Internet endpoint for an IVOA-compliant HiPS service.
~Confirm that this service is advertised via Registry as well as in the
HiPS community mechanism operated by CDS, or whatever equivalent
mechanism may exist in the LSST operations era.


\subsubsection{Predecessors}

\subsubsection{Environment Needs}

\paragraph{Software}

\paragraph{Hardware}

\subsubsection{Input Specification}

\subsubsection{Output Specification}

\subsubsection{Test Procedure}
    \begin{longtable}[]{p{1.3cm}p{2cm}p{13cm}}
    %\toprule
    Step & \multicolumn{2}{@{}l}{Description, Input Data and Expected Result} \\ \toprule
    \endhead

            \multirow{3}{*}{ 1 } & Description &
            \begin{minipage}[t]{13cm}{\footnotesize
            
            \vspace{\dp0}
            } \end{minipage} \\ \cline{2-3}
            & Test Data &
            \begin{minipage}[t]{13cm}{\footnotesize
                No data.
                \vspace{\dp0}
            } \end{minipage} \\ \cline{2-3}
            & Expected Result &
        \\ \midrule
    \end{longtable}

\subsection{LVV-T1527 - Verify Support for HiPS Visualization}\label{lvv-t1527}

\begin{longtable}[]{llllll}
\toprule
Version & Status & Priority & Verification Type & Owner
\\\midrule
1 & Draft & Normal &
Demonstration & Jeffrey Carlin
\\\bottomrule
\multicolumn{6}{c}{ Open \href{https://jira.lsstcorp.org/secure/Tests.jspa\#/testCase/LVV-T1527}{LVV-T1527} in Jira } \\
\end{longtable}

\subsubsection{Verification Elements}
\begin{itemize}
\item \href{https://jira.lsstcorp.org/browse/LVV-18225}{LVV-18225} - DMS-REQ-0382-V-01: HiPS Visualization\_1

\end{itemize}

\subsubsection{Test Items}
Verify that the LSST Science Platform supports the visualization of
LSST-generated HiPS image maps as well as other HiPS maps which satisfy
the IVOA HiPS Recommendation. Also verify that integrated behavior is
available, such as the overplotting of catalog entries, comparable to
that provided for individual source images (e.g., PVIs and coadd tiles).


\subsubsection{Predecessors}

\subsubsection{Environment Needs}

\paragraph{Software}

\paragraph{Hardware}

\subsubsection{Input Specification}

\subsubsection{Output Specification}

\subsubsection{Test Procedure}
    \begin{longtable}[]{p{1.3cm}p{2cm}p{13cm}}
    %\toprule
    Step & \multicolumn{2}{@{}l}{Description, Input Data and Expected Result} \\ \toprule
    \endhead

            \multirow{3}{*}{ 1 } & Description &
            \begin{minipage}[t]{13cm}{\footnotesize
            
            \vspace{\dp0}
            } \end{minipage} \\ \cline{2-3}
            & Test Data &
            \begin{minipage}[t]{13cm}{\footnotesize
                No data.
                \vspace{\dp0}
            } \end{minipage} \\ \cline{2-3}
            & Expected Result &
        \\ \midrule
    \end{longtable}

\subsection{LVV-T1528 - Verify Visualization of MOCs via Science Platform}\label{lvv-t1528}

\begin{longtable}[]{llllll}
\toprule
Version & Status & Priority & Verification Type & Owner
\\\midrule
1 & Draft & Normal &
Demonstration & Jeffrey Carlin
\\\bottomrule
\multicolumn{6}{c}{ Open \href{https://jira.lsstcorp.org/secure/Tests.jspa\#/testCase/LVV-T1528}{LVV-T1528} in Jira } \\
\end{longtable}

\subsubsection{Verification Elements}
\begin{itemize}
\item \href{https://jira.lsstcorp.org/browse/LVV-18226}{LVV-18226} - DMS-REQ-0385-V-01: MOC Visualization\_1

\end{itemize}

\subsubsection{Test Items}
Verify that the LSST Science Platform supports the visualization of the
LSST-generated MOCs as well as other MOCs which satisfy the IVOA MOC
Recommendation.


\subsubsection{Predecessors}

\subsubsection{Environment Needs}

\paragraph{Software}

\paragraph{Hardware}

\subsubsection{Input Specification}

\subsubsection{Output Specification}

\subsubsection{Test Procedure}
    \begin{longtable}[]{p{1.3cm}p{2cm}p{13cm}}
    %\toprule
    Step & \multicolumn{2}{@{}l}{Description, Input Data and Expected Result} \\ \toprule
    \endhead

            \multirow{3}{*}{ 1 } & Description &
            \begin{minipage}[t]{13cm}{\footnotesize
            
            \vspace{\dp0}
            } \end{minipage} \\ \cline{2-3}
            & Test Data &
            \begin{minipage}[t]{13cm}{\footnotesize
                No data.
                \vspace{\dp0}
            } \end{minipage} \\ \cline{2-3}
            & Expected Result &
        \\ \midrule
    \end{longtable}

\subsection{LVV-T1529 - Verify Production of All-Sky HiPS Map}\label{lvv-t1529}

\begin{longtable}[]{llllll}
\toprule
Version & Status & Priority & Verification Type & Owner
\\\midrule
1 & Draft & Normal &
Demonstration & Jeffrey Carlin
\\\bottomrule
\multicolumn{6}{c}{ Open \href{https://jira.lsstcorp.org/secure/Tests.jspa\#/testCase/LVV-T1529}{LVV-T1529} in Jira } \\
\end{longtable}

\subsubsection{Verification Elements}
\begin{itemize}
\item \href{https://jira.lsstcorp.org/browse/LVV-18227}{LVV-18227} - DMS-REQ-0379-V-01: Produce All-Sky HiPS Map\_1

\end{itemize}

\subsubsection{Test Items}
Verify that Data Release Production includes the production of an
all-sky image map for the existing coadded image area in each filter
band, and at least one pre-defined all-sky color image map, following
the IVOA HiPS Recommendation.


\subsubsection{Predecessors}

\subsubsection{Environment Needs}

\paragraph{Software}

\paragraph{Hardware}

\subsubsection{Input Specification}

\subsubsection{Output Specification}

\subsubsection{Test Procedure}
    \begin{longtable}[]{p{1.3cm}p{2cm}p{13cm}}
    %\toprule
    Step & \multicolumn{2}{@{}l}{Description, Input Data and Expected Result} \\ \toprule
    \endhead

            \multirow{3}{*}{ 1 } & Description &
            \begin{minipage}[t]{13cm}{\footnotesize
            
            \vspace{\dp0}
            } \end{minipage} \\ \cline{2-3}
            & Test Data &
            \begin{minipage}[t]{13cm}{\footnotesize
                No data.
                \vspace{\dp0}
            } \end{minipage} \\ \cline{2-3}
            & Expected Result &
        \\ \midrule
    \end{longtable}

\subsection{LVV-T1530 - Verify Production of Multi-Order Coverage Maps for Survey Data}\label{lvv-t1530}

\begin{longtable}[]{llllll}
\toprule
Version & Status & Priority & Verification Type & Owner
\\\midrule
1 & Draft & Normal &
Demonstration & Jeffrey Carlin
\\\bottomrule
\multicolumn{6}{c}{ Open \href{https://jira.lsstcorp.org/secure/Tests.jspa\#/testCase/LVV-T1530}{LVV-T1530} in Jira } \\
\end{longtable}

\subsubsection{Verification Elements}
\begin{itemize}
\item \href{https://jira.lsstcorp.org/browse/LVV-18228}{LVV-18228} - DMS-REQ-0383-V-01: Produce MOC Maps\_1

\end{itemize}

\subsubsection{Test Items}
Verify that Data Release Production includes the production of
Multi-Order Coverage maps for the survey data, conformant with the IVOA
MOC recommendation. ~Confirm that separate MOC are produced for each
filter band for the main survey, and additional MOCs are produced to
represent special-programs datasets and other collections of on-sky
data.


\subsubsection{Predecessors}

\subsubsection{Environment Needs}

\paragraph{Software}

\paragraph{Hardware}

\subsubsection{Input Specification}

\subsubsection{Output Specification}

\subsubsection{Test Procedure}
    \begin{longtable}[]{p{1.3cm}p{2cm}p{13cm}}
    %\toprule
    Step & \multicolumn{2}{@{}l}{Description, Input Data and Expected Result} \\ \toprule
    \endhead

            \multirow{3}{*}{ 1 } & Description &
            \begin{minipage}[t]{13cm}{\footnotesize
            
            \vspace{\dp0}
            } \end{minipage} \\ \cline{2-3}
            & Test Data &
            \begin{minipage}[t]{13cm}{\footnotesize
                No data.
                \vspace{\dp0}
            } \end{minipage} \\ \cline{2-3}
            & Expected Result &
        \\ \midrule
    \end{longtable}

\subsection{LVV-T1549 - LDM-503-6 Comcam verification readiness}\label{lvv-t1549}

\begin{longtable}[]{llllll}
\toprule
Version & Status & Priority & Verification Type & Owner
\\\midrule
1 & Approved & Normal &
Demonstration & Michelle Butler
\\\bottomrule
\multicolumn{6}{c}{ Open \href{https://jira.lsstcorp.org/secure/Tests.jspa\#/testCase/LVV-T1549}{LVV-T1549} in Jira } \\
\end{longtable}

\subsubsection{Verification Elements}
\begin{itemize}
\item \href{https://jira.lsstcorp.org/browse/LVV-9}{LVV-9} - DMS-REQ-0020-V-01: Wavefront Sensor Data Acquisition

\item \href{https://jira.lsstcorp.org/browse/LVV-8}{LVV-8} - DMS-REQ-0018-V-01: Raw Science Image Data Acquisition

\item \href{https://jira.lsstcorp.org/browse/LVV-28}{LVV-28} - DMS-REQ-0068-V-01: Raw Science Image Metadata

\item \href{https://jira.lsstcorp.org/browse/LVV-11}{LVV-11} - DMS-REQ-0024-V-01: Raw Image Assembly

\item \href{https://jira.lsstcorp.org/browse/LVV-146}{LVV-146} - DMS-REQ-0315-V-01: DMS Communication with OCS

\end{itemize}

\subsubsection{Test Items}
Verify that ComCam has all the services running and verified working for
retrieving an image from the ComCam DAQ and store it on file systems at
the LDF for viewing by RSP. ~


\subsubsection{Predecessors}

\subsubsection{Environment Needs}

\paragraph{Software}

\paragraph{Hardware}

\subsubsection{Input Specification}
ComCam must be In LaSerena and producing images with proper headers. ~

\subsubsection{Output Specification}

\subsubsection{Test Procedure}
    \begin{longtable}[]{p{1.3cm}p{2cm}p{13cm}}
    %\toprule
    Step & \multicolumn{2}{@{}l}{Description, Input Data and Expected Result} \\ \toprule
    \endhead

            \multirow{3}{*}{ 1 } & Description &
            \begin{minipage}[t]{13cm}{\footnotesize
            ComCam-DAQ produces an image~

            \vspace{\dp0}
            } \end{minipage} \\ \cline{2-3}
            & Test Data &
            \begin{minipage}[t]{13cm}{\footnotesize
                DAQ produces a SAL message that a image has been created~

                \vspace{\dp0}
            } \end{minipage} \\ \cline{2-3}
            & Expected Result &
                \begin{minipage}[t]{13cm}{\footnotesize
                in memory file created in DAQ

                \vspace{\dp0}
                } \end{minipage}
        \\ \midrule

            \multirow{3}{*}{ 2 } & Description &
            \begin{minipage}[t]{13cm}{\footnotesize
            ComCam-archiver and ComCam-forwarder build image with proper header from
ComCam-header service~

            \vspace{\dp0}
            } \end{minipage} \\ \cline{2-3}
            & Test Data &
            \begin{minipage}[t]{13cm}{\footnotesize
                Good image file with proper header with all 9 CCDs~

                \vspace{\dp0}
            } \end{minipage} \\ \cline{2-3}
            & Expected Result &
                \begin{minipage}[t]{13cm}{\footnotesize
                9 image files all with individual headers and then 1 header for all 9
images too. ~\\[2\baselineskip]

                \vspace{\dp0}
                } \end{minipage}
        \\ \midrule

            \multirow{3}{*}{ 3 } & Description &
            \begin{minipage}[t]{13cm}{\footnotesize
            ComCam-archiver/forwarder transfers the file to the l1-handoff machine.~

            \vspace{\dp0}
            } \end{minipage} \\ \cline{2-3}
            & Test Data &
            \begin{minipage}[t]{13cm}{\footnotesize
                l1-handoff machine has image file now on local disk.~~

                \vspace{\dp0}
            } \end{minipage} \\ \cline{2-3}
            & Expected Result &
                \begin{minipage}[t]{13cm}{\footnotesize
                image file now found on disk on L1-handoff with hardlinks to 2 different
file systems (OODS and DBB) services. ~\\[3\baselineskip]

                \vspace{\dp0}
                } \end{minipage}
        \\ \midrule

            \multirow{3}{*}{ 4 } & Description &
            \begin{minipage}[t]{13cm}{\footnotesize
            OODS service running and ingests the image file into Butler/G3 (or Gen2)
and readies the file systems for the commissioning cluster at the Base
to be able to mount and see the new files. ~ ~

            \vspace{\dp0}
            } \end{minipage} \\ \cline{2-3}
            & Test Data &
            \begin{minipage}[t]{13cm}{\footnotesize
                Image file ingested to local butler for Base~

                \vspace{\dp0}
            } \end{minipage} \\ \cline{2-3}
            & Expected Result &
                \begin{minipage}[t]{13cm}{\footnotesize
                Image file ingested\\[2\baselineskip]

                \vspace{\dp0}
                } \end{minipage}
        \\ \midrule

            \multirow{3}{*}{ 5 } & Description &
            \begin{minipage}[t]{13cm}{\footnotesize
            DBB transfers the file to NCSA thorough the DBB-gateway machines and DTN
nodes at the base.~~

            \vspace{\dp0}
            } \end{minipage} \\ \cline{2-3}
            & Test Data &
            \begin{minipage}[t]{13cm}{\footnotesize
                No data.
                \vspace{\dp0}
            } \end{minipage} \\ \cline{2-3}
            & Expected Result &
                \begin{minipage}[t]{13cm}{\footnotesize
                data file arrives at file systems at NCSA\\[2\baselineskip]

                \vspace{\dp0}
                } \end{minipage}
        \\ \midrule

            \multirow{3}{*}{ 6 } & Description &
            \begin{minipage}[t]{13cm}{\footnotesize
            Files are ingested into the butler/G3 at NCSA and moved to file systems
that are viewable by the RSP. ~

            \vspace{\dp0}
            } \end{minipage} \\ \cline{2-3}
            & Test Data &
            \begin{minipage}[t]{13cm}{\footnotesize
                No data.
                \vspace{\dp0}
            } \end{minipage} \\ \cline{2-3}
            & Expected Result &
                \begin{minipage}[t]{13cm}{\footnotesize
                data can be seen and retrieved by RSP. ~

                \vspace{\dp0}
                } \end{minipage}
        \\ \midrule
    \end{longtable}

\subsection{LVV-T1550 - LDM-503-10 DAQ Validation}\label{lvv-t1550}

\begin{longtable}[]{llllll}
\toprule
Version & Status & Priority & Verification Type & Owner
\\\midrule
1 & Approved & Normal &
Demonstration & Michelle Butler
\\\bottomrule
\multicolumn{6}{c}{ Open \href{https://jira.lsstcorp.org/secure/Tests.jspa\#/testCase/LVV-T1550}{LVV-T1550} in Jira } \\
\end{longtable}

\subsubsection{Verification Elements}
\begin{itemize}
\item \href{https://jira.lsstcorp.org/browse/LVV-8}{LVV-8} - DMS-REQ-0018-V-01: Raw Science Image Data Acquisition

\item \href{https://jira.lsstcorp.org/browse/LVV-28}{LVV-28} - DMS-REQ-0068-V-01: Raw Science Image Metadata

\item \href{https://jira.lsstcorp.org/browse/LVV-11}{LVV-11} - DMS-REQ-0024-V-01: Raw Image Assembly

\end{itemize}

\subsubsection{Test Items}
Verify that the DAQ can talk to test machines at the BDC through the
DWDM network.~


\subsubsection{Predecessors}
DAQ network at the base; forwarders and L1 handoff machine must be
available to the DAQ COB at the summit, and forwarders and other test
machines must be configured and set up on the BDC networks.~

\subsubsection{Environment Needs}

\paragraph{Software}

\paragraph{Hardware}

\subsubsection{Input Specification}
DAQ at the Summit and machines on networks at the base.~~

\subsubsection{Output Specification}

\subsubsection{Test Procedure}
    \begin{longtable}[]{p{1.3cm}p{2cm}p{13cm}}
    %\toprule
    Step & \multicolumn{2}{@{}l}{Description, Input Data and Expected Result} \\ \toprule
    \endhead

            \multirow{3}{*}{ 1 } & Description &
            \begin{minipage}[t]{13cm}{\footnotesize
            have DAQ produce image at the summit~

            \vspace{\dp0}
            } \end{minipage} \\ \cline{2-3}
            & Test Data &
            \begin{minipage}[t]{13cm}{\footnotesize
                No data.
                \vspace{\dp0}
            } \end{minipage} \\ \cline{2-3}
            & Expected Result &
                \begin{minipage}[t]{13cm}{\footnotesize
                Image on At-archiver~

                \vspace{\dp0}
                } \end{minipage}
        \\ \midrule

            \multirow{3}{*}{ 2 } & Description &
            \begin{minipage}[t]{13cm}{\footnotesize
            The forwarder at the BDC should be able to have communication with the
DAQ that the image was taken, and be able to see the file.
~\\[2\baselineskip]

            \vspace{\dp0}
            } \end{minipage} \\ \cline{2-3}
            & Test Data &
            \begin{minipage}[t]{13cm}{\footnotesize
                No data.
                \vspace{\dp0}
            } \end{minipage} \\ \cline{2-3}
            & Expected Result &
                \begin{minipage}[t]{13cm}{\footnotesize
                Image available for the forwarder at the base.~~

                \vspace{\dp0}
                } \end{minipage}
        \\ \midrule

            \multirow{3}{*}{ 3 } & Description &
            \begin{minipage}[t]{13cm}{\footnotesize
            Communication between the forwarder and the DAQ are in place with
messages being exchanged.~ ~

            \vspace{\dp0}
            } \end{minipage} \\ \cline{2-3}
            & Test Data &
            \begin{minipage}[t]{13cm}{\footnotesize
                No data.
                \vspace{\dp0}
            } \end{minipage} \\ \cline{2-3}
            & Expected Result &
                \begin{minipage}[t]{13cm}{\footnotesize
                if messages can be exchanged, the communication has been established.
~\\[3\baselineskip]

                \vspace{\dp0}
                } \end{minipage}
        \\ \midrule
    \end{longtable}

\subsection{LVV-T1556 - LDM-503-10B Large Scale CCOB Data Access}\label{lvv-t1556}

\begin{longtable}[]{llllll}
\toprule
Version & Status & Priority & Verification Type & Owner
\\\midrule
1 & Draft & Normal &
Demonstration & Michelle Butler
\\\bottomrule
\multicolumn{6}{c}{ Open \href{https://jira.lsstcorp.org/secure/Tests.jspa\#/testCase/LVV-T1556}{LVV-T1556} in Jira } \\
\end{longtable}

\subsubsection{Verification Elements}
\begin{itemize}
\item \href{https://jira.lsstcorp.org/browse/LVV-8}{LVV-8} - DMS-REQ-0018-V-01: Raw Science Image Data Acquisition

\item \href{https://jira.lsstcorp.org/browse/LVV-9}{LVV-9} - DMS-REQ-0020-V-01: Wavefront Sensor Data Acquisition

\item \href{https://jira.lsstcorp.org/browse/LVV-11}{LVV-11} - DMS-REQ-0024-V-01: Raw Image Assembly

\item \href{https://jira.lsstcorp.org/browse/LVV-146}{LVV-146} - DMS-REQ-0315-V-01: DMS Communication with OCS

\item \href{https://jira.lsstcorp.org/browse/LVV-28}{LVV-28} - DMS-REQ-0068-V-01: Raw Science Image Metadata

\end{itemize}

\subsubsection{Test Items}
Demonstrate the ability to transfer data from the SLAC test stand or
CCOB with 21 rafts from SLAC and ingested at NCSA and make available
through an instance of the RSP


\subsubsection{Predecessors}

\subsubsection{Environment Needs}

\paragraph{Software}

\paragraph{Hardware}

\subsubsection{Input Specification}
SLAC or some other test stand needs to have produced 21 rafts of data
that has some environment for transferring the data to NCSA. ~ ~The
images won't be able to be ingested at NCSA into butler repositories if
the headers are not correct. ~ ~The \citeds{LSE-400} document calls out what
fields are designed to be completed by the image properly.~ ~The headers
at this time are still in test form, and are being updated as images are
created.~ ~ The confluence page
of:~\url{https://confluence.lsstcorp.org/display/SYSENG/ComCam+Header+Information+Topic+Mapping}
has the current header configuration and what the information maps to
for the ingest information.~ ~

\subsubsection{Output Specification}

\subsubsection{Test Procedure}
    \begin{longtable}[]{p{1.3cm}p{2cm}p{13cm}}
    %\toprule
    Step & \multicolumn{2}{@{}l}{Description, Input Data and Expected Result} \\ \toprule
    \endhead

            \multirow{3}{*}{ 1 } & Description &
            \begin{minipage}[t]{13cm}{\footnotesize
            Have a system at SLAC that has the 21 raft data that needs to be
transferred to NCSA, and all accounts and scripts installed on
environment that can read that data.~ ~

            \vspace{\dp0}
            } \end{minipage} \\ \cline{2-3}
            & Test Data &
            \begin{minipage}[t]{13cm}{\footnotesize
                21 rafts of data with proper headers~

                \vspace{\dp0}
            } \end{minipage} \\ \cline{2-3}
            & Expected Result &
                \begin{minipage}[t]{13cm}{\footnotesize
                scripts are able to transfer the data to NCSA though rsync or bbcp.
~\\[2\baselineskip]

                \vspace{\dp0}
                } \end{minipage}
        \\ \midrule

            \multirow{3}{*}{ 2 } & Description &
            \begin{minipage}[t]{13cm}{\footnotesize
            Data is transferred to NCSA and ingested into Butler~

            \vspace{\dp0}
            } \end{minipage} \\ \cline{2-3}
            & Test Data &
            \begin{minipage}[t]{13cm}{\footnotesize
                21 rafts of data~

                \vspace{\dp0}
            } \end{minipage} \\ \cline{2-3}
            & Expected Result &
                \begin{minipage}[t]{13cm}{\footnotesize
                Data is transferred to NCSA, and can now be see in file systems by the
RSP. ~\\[2\baselineskip]

                \vspace{\dp0}
                } \end{minipage}
        \\ \midrule

            \multirow{3}{*}{ 3 } & Description &
            \begin{minipage}[t]{13cm}{\footnotesize
            using the RSP view the data in the ingested directory~

            \vspace{\dp0}
            } \end{minipage} \\ \cline{2-3}
            & Test Data &
            \begin{minipage}[t]{13cm}{\footnotesize
                21 rafts of data with proper headers and available with Butler.get~

                \vspace{\dp0}
            } \end{minipage} \\ \cline{2-3}
            & Expected Result &
                \begin{minipage}[t]{13cm}{\footnotesize
                data can be viewed.

                \vspace{\dp0}
                } \end{minipage}
        \\ \midrule
    \end{longtable}

\subsection{LVV-T1560 - Verify archiving of processing provenance}\label{lvv-t1560}

\begin{longtable}[]{llllll}
\toprule
Version & Status & Priority & Verification Type & Owner
\\\midrule
1 & Draft & Normal &
Inspection & Jeffrey Carlin
\\\bottomrule
\multicolumn{6}{c}{ Open \href{https://jira.lsstcorp.org/secure/Tests.jspa\#/testCase/LVV-T1560}{LVV-T1560} in Jira } \\
\end{longtable}

\subsubsection{Verification Elements}
\begin{itemize}
\item \href{https://jira.lsstcorp.org/browse/LVV-18230}{LVV-18230} - DMS-REQ-0386-V-01: Archive Processing Provenance\_1

\end{itemize}

\subsubsection{Test Items}
Verify that provenance information related to data processing, including
relevant data from other subsystems, has been archived.


\subsubsection{Predecessors}

\subsubsection{Environment Needs}

\paragraph{Software}

\paragraph{Hardware}

\subsubsection{Input Specification}

\subsubsection{Output Specification}

\subsubsection{Test Procedure}
    \begin{longtable}[]{p{1.3cm}p{2cm}p{13cm}}
    %\toprule
    Step & \multicolumn{2}{@{}l}{Description, Input Data and Expected Result} \\ \toprule
    \endhead

            \multirow{3}{*}{ 1 } & Description &
            \begin{minipage}[t]{13cm}{\footnotesize
            
            \vspace{\dp0}
            } \end{minipage} \\ \cline{2-3}
            & Test Data &
            \begin{minipage}[t]{13cm}{\footnotesize
                No data.
                \vspace{\dp0}
            } \end{minipage} \\ \cline{2-3}
            & Expected Result &
        \\ \midrule
    \end{longtable}

\subsection{LVV-T1561 - Verify provenance availability to science users}\label{lvv-t1561}

\begin{longtable}[]{llllll}
\toprule
Version & Status & Priority & Verification Type & Owner
\\\midrule
1 & Draft & Normal &
Inspection & Jeffrey Carlin
\\\bottomrule
\multicolumn{6}{c}{ Open \href{https://jira.lsstcorp.org/secure/Tests.jspa\#/testCase/LVV-T1561}{LVV-T1561} in Jira } \\
\end{longtable}

\subsubsection{Verification Elements}
\begin{itemize}
\item \href{https://jira.lsstcorp.org/browse/LVV-18231}{LVV-18231} - DMS-REQ-0387-V-01: Serve Archived Provenance\_1

\end{itemize}

\subsubsection{Test Items}
Verify that archived provenance data is available to science users
together with the associated science data products.


\subsubsection{Predecessors}

\subsubsection{Environment Needs}

\paragraph{Software}

\paragraph{Hardware}

\subsubsection{Input Specification}

\subsubsection{Output Specification}

\subsubsection{Test Procedure}
    \begin{longtable}[]{p{1.3cm}p{2cm}p{13cm}}
    %\toprule
    Step & \multicolumn{2}{@{}l}{Description, Input Data and Expected Result} \\ \toprule
    \endhead

            \multirow{3}{*}{ 1 } & Description &
            \begin{minipage}[t]{13cm}{\footnotesize
            
            \vspace{\dp0}
            } \end{minipage} \\ \cline{2-3}
            & Test Data &
            \begin{minipage}[t]{13cm}{\footnotesize
                No data.
                \vspace{\dp0}
            } \end{minipage} \\ \cline{2-3}
            & Expected Result &
        \\ \midrule
    \end{longtable}

\subsection{LVV-T1562 - Verify availability of re-run tools}\label{lvv-t1562}

\begin{longtable}[]{llllll}
\toprule
Version & Status & Priority & Verification Type & Owner
\\\midrule
1 & Draft & Normal &
Demonstration & Jeffrey Carlin
\\\bottomrule
\multicolumn{6}{c}{ Open \href{https://jira.lsstcorp.org/secure/Tests.jspa\#/testCase/LVV-T1562}{LVV-T1562} in Jira } \\
\end{longtable}

\subsubsection{Verification Elements}
\begin{itemize}
\item \href{https://jira.lsstcorp.org/browse/LVV-18232}{LVV-18232} - DMS-REQ-0388-V-01: Provide Re-Run Tools\_1

\end{itemize}

\subsubsection{Test Items}
Verify that tools are provided to use the archived provenance data to
re-run a data processing operation under the same conditions (including
LSST software version, its configuration parameters, and supporting data
such as calibration frames) as a previous run of that operation.


\subsubsection{Predecessors}

\subsubsection{Environment Needs}

\paragraph{Software}

\paragraph{Hardware}

\subsubsection{Input Specification}

\subsubsection{Output Specification}

\subsubsection{Test Procedure}
    \begin{longtable}[]{p{1.3cm}p{2cm}p{13cm}}
    %\toprule
    Step & \multicolumn{2}{@{}l}{Description, Input Data and Expected Result} \\ \toprule
    \endhead

            \multirow{3}{*}{ 1 } & Description &
            \begin{minipage}[t]{13cm}{\footnotesize
            
            \vspace{\dp0}
            } \end{minipage} \\ \cline{2-3}
            & Test Data &
            \begin{minipage}[t]{13cm}{\footnotesize
                No data.
                \vspace{\dp0}
            } \end{minipage} \\ \cline{2-3}
            & Expected Result &
        \\ \midrule
    \end{longtable}

\subsection{LVV-T1563 - Verify re-run on different system produces the same results}\label{lvv-t1563}

\begin{longtable}[]{llllll}
\toprule
Version & Status & Priority & Verification Type & Owner
\\\midrule
1 & Draft & Normal &
Demonstration & Jeffrey Carlin
\\\bottomrule
\multicolumn{6}{c}{ Open \href{https://jira.lsstcorp.org/secure/Tests.jspa\#/testCase/LVV-T1563}{LVV-T1563} in Jira } \\
\end{longtable}

\subsubsection{Verification Elements}
\begin{itemize}
\item \href{https://jira.lsstcorp.org/browse/LVV-18233}{LVV-18233} - DMS-REQ-0390-V-01: Re-Runs on Other Systems\_1

\end{itemize}

\subsubsection{Test Items}
Verify that tools are provided to use the archived provenance data to
re-run a data processing operation on different systems, and that the
results produced are the same to the extent computationally feasible.~


\subsubsection{Predecessors}

\subsubsection{Environment Needs}

\paragraph{Software}

\paragraph{Hardware}

\subsubsection{Input Specification}

\subsubsection{Output Specification}

\subsubsection{Test Procedure}
    \begin{longtable}[]{p{1.3cm}p{2cm}p{13cm}}
    %\toprule
    Step & \multicolumn{2}{@{}l}{Description, Input Data and Expected Result} \\ \toprule
    \endhead

            \multirow{3}{*}{ 1 } & Description &
            \begin{minipage}[t]{13cm}{\footnotesize
            
            \vspace{\dp0}
            } \end{minipage} \\ \cline{2-3}
            & Test Data &
            \begin{minipage}[t]{13cm}{\footnotesize
                No data.
                \vspace{\dp0}
            } \end{minipage} \\ \cline{2-3}
            & Expected Result &
        \\ \midrule
    \end{longtable}

\subsection{LVV-T1564 - Verify re-run on similar system produces the same results}\label{lvv-t1564}

\begin{longtable}[]{llllll}
\toprule
Version & Status & Priority & Verification Type & Owner
\\\midrule
1 & Draft & Normal &
Demonstration & Jeffrey Carlin
\\\bottomrule
\multicolumn{6}{c}{ Open \href{https://jira.lsstcorp.org/secure/Tests.jspa\#/testCase/LVV-T1564}{LVV-T1564} in Jira } \\
\end{longtable}

\subsubsection{Verification Elements}
\begin{itemize}
\item \href{https://jira.lsstcorp.org/browse/LVV-18234}{LVV-18234} - DMS-REQ-0389-V-01: Re-Runs on Similar Systems\_1

\end{itemize}

\subsubsection{Test Items}
Verify that a provenance-based re-run that is run on the same system, or
a system with identically configured hardware and system software,
produces the same results.~


\subsubsection{Predecessors}

\subsubsection{Environment Needs}

\paragraph{Software}

\paragraph{Hardware}

\subsubsection{Input Specification}

\subsubsection{Output Specification}

\subsubsection{Test Procedure}
    \begin{longtable}[]{p{1.3cm}p{2cm}p{13cm}}
    %\toprule
    Step & \multicolumn{2}{@{}l}{Description, Input Data and Expected Result} \\ \toprule
    \endhead

            \multirow{3}{*}{ 1 } & Description &
            \begin{minipage}[t]{13cm}{\footnotesize
            
            \vspace{\dp0}
            } \end{minipage} \\ \cline{2-3}
            & Test Data &
            \begin{minipage}[t]{13cm}{\footnotesize
                No data.
                \vspace{\dp0}
            } \end{minipage} \\ \cline{2-3}
            & Expected Result &
        \\ \midrule
    \end{longtable}

\subsection{LVV-T1612 - Verify Summit - Base Network Integration (System Level)}\label{lvv-t1612}

\begin{longtable}[]{llllll}
\toprule
Version & Status & Priority & Verification Type & Owner
\\\midrule
1 & Draft & Normal &
Inspection & Jeff Kantor
\\\bottomrule
\multicolumn{6}{c}{ Open \href{https://jira.lsstcorp.org/secure/Tests.jspa\#/testCase/LVV-T1612}{LVV-T1612} in Jira } \\
\end{longtable}

\subsubsection{Verification Elements}
\begin{itemize}
\item \href{https://jira.lsstcorp.org/browse/LVV-73}{LVV-73} - DMS-REQ-0171-V-01: Summit to Base Network

\end{itemize}

\subsubsection{Test Items}
Verify ISO Layer 3 full (22 x 10 Gbps ethernet ports on DAQ side with
test data from DAQ test stand, AURA, Camera DAQ team do test).
Demonstrate transfer of data at or exceeding rates specified in \citeds{LDM-142}.


\subsubsection{Predecessors}
See pre-conditions.

\subsubsection{Environment Needs}

\paragraph{Software}
See pre-conditions.

\paragraph{Hardware}
See pre-conditions.

\subsubsection{Input Specification}
\begin{enumerate}
\tightlist
\item
  PMCS DMTC-7400-2400 COMPLETE
\item
  \href{https://jira.lsstcorp.org/secure/Tests.jspa\#/testCase/1401}{LVV-T1168}
  Passed
\item
  EITHER: Full Camera DAQ installed on summit and loaded with data OR:
  high-quality DAQ application-level simulators that match the form,
  volume, file paths, compressibility, and cadence of the expected
  instrument data, running on end node computers that are the production
  hardware or equivalent to it. Scientific validity of the data content
  is not essential.
\item
  Archiver/forwarders installed at Base running on end node computers
  that are the production hardware or equivalent to it.
\item
  As-built documentation for all of the above is available.
\end{enumerate}

NOTE: This test will be repeated at increasing data volumes as
additional observatory capabilities (e.g. ComCAM, FullCam) become
available. Final verification will be tested at full operational
volume.After the initial test, the corresponding verification elements
will be flagged as ``Requires Monitoring'' such that those requirements
will be closed out as having been verified but will continue to be
monitored throughout commissioning to ensure they do not drop out of
compliance. This will also be monitored for end to end Summit - Data
Facility transfers during Commissioning.

\subsubsection{Output Specification}

\subsubsection{Test Procedure}
    \begin{longtable}[]{p{1.3cm}p{2cm}p{13cm}}
    %\toprule
    Step & \multicolumn{2}{@{}l}{Description, Input Data and Expected Result} \\ \toprule
    \endhead

            \multirow{3}{*}{ 1 } & Description &
            \begin{minipage}[t]{13cm}{\footnotesize
            Verify Pre-conditions are satisfied.

            \vspace{\dp0}
            } \end{minipage} \\ \cline{2-3}
            & Test Data &
            \begin{minipage}[t]{13cm}{\footnotesize
                NA

                \vspace{\dp0}
            } \end{minipage} \\ \cline{2-3}
            & Expected Result &
                \begin{minipage}[t]{13cm}{\footnotesize
                Pre-conditions are satisfied.

                \vspace{\dp0}
                } \end{minipage}
        \\ \midrule

            \multirow{3}{*}{ 2 } & Description &
            \begin{minipage}[t]{13cm}{\footnotesize
            Transfer data between summit and base over uninterrupted 1 day period.
~Monitor transfer of data at or exceeding rates specified in LDM-142.

            \vspace{\dp0}
            } \end{minipage} \\ \cline{2-3}
            & Test Data &
            \begin{minipage}[t]{13cm}{\footnotesize
                DAQ pre-loaded data

                \vspace{\dp0}
            } \end{minipage} \\ \cline{2-3}
            & Expected Result &
                \begin{minipage}[t]{13cm}{\footnotesize
                Data transfers at or exceeding rates specified in LDM-142.

                \vspace{\dp0}
                } \end{minipage}
        \\ \midrule
    \end{longtable}

\subsection{LVV-T1745 - Verify calculation of median relative astrometric measurement error on
20 arcminute scales}\label{lvv-t1745}

\begin{longtable}[]{llllll}
\toprule
Version & Status & Priority & Verification Type & Owner
\\\midrule
1 & Approved & Normal &
Test & Jeffrey Carlin
\\\bottomrule
\multicolumn{6}{c}{ Open \href{https://jira.lsstcorp.org/secure/Tests.jspa\#/testCase/LVV-T1745}{LVV-T1745} in Jira } \\
\end{longtable}

\subsubsection{Verification Elements}
\begin{itemize}
\item \href{https://jira.lsstcorp.org/browse/LVV-3402}{LVV-3402} - DMS-REQ-0360-V-01: Median astrometric error on 20 arcmin scales

\end{itemize}

\subsubsection{Test Items}
Verify that the DM system has provided the code to calculate the median
relative astrometric measurement error on 20 arcminute scales and assess
whether it meets the requirement that it shall be no more than AM2 = 10
milliarcseconds.


\subsubsection{Predecessors}

\subsubsection{Environment Needs}

\paragraph{Software}

\paragraph{Hardware}

\subsubsection{Input Specification}

\subsubsection{Output Specification}

\subsubsection{Test Procedure}
    \begin{longtable}[]{p{1.3cm}p{2cm}p{13cm}}
    %\toprule
    Step & \multicolumn{2}{@{}l}{Description, Input Data and Expected Result} \\ \toprule
    \endhead

            \multirow{3}{*}{ 1 } & Description &
            \begin{minipage}[t]{13cm}{\footnotesize
            Identify a dataset containing at least one field with multiple
overlapping visits.

            \vspace{\dp0}
            } \end{minipage} \\ \cline{2-3}
            & Test Data &
            \begin{minipage}[t]{13cm}{\footnotesize
                No data.
                \vspace{\dp0}
            } \end{minipage} \\ \cline{2-3}
            & Expected Result &
                \begin{minipage}[t]{13cm}{\footnotesize
                A dataset that has been ingested into a Butler repository.

                \vspace{\dp0}
                } \end{minipage}
        \\ \midrule

                \multirow{3}{*}{\parbox{1.3cm}{ 2-1
                {\scriptsize from \hyperref[lvv-t860]
                {LVV-T860} } } }

                & {\small Description} &
                \begin{minipage}[t]{13cm}{\scriptsize
                The `path` that you will use depends on where you are running the
science pipelines. Options:\\[2\baselineskip]

\begin{itemize}
\tightlist
\item
  local (newinstall.sh - based
  install):{[}path\_to\_installation{]}/loadLSST.bash
\item
  development cluster (``lsst-dev''):
  /software/lsstsw/stack/loadLSST.bash
\item
  LSP Notebook aspect (from a terminal):
  /opt/lsst/software/stack/loadLSST.bash
\end{itemize}

From the command line, execute the commands below in the example
code:\\[2\baselineskip]

                \vspace{\dp0}
                } \end{minipage} \\ \cdashline{2-3}
                & {\small Test Data} &
                \begin{minipage}[t]{13cm}{\scriptsize
                } \end{minipage} \\ \cdashline{2-3}
                & {\small Expected Result} &
                    \begin{minipage}[t]{13cm}{\scriptsize
                    Science pipeline software is available for use. If additional packages
are needed (for example, `obs' packages such as `obs\_subaru`), then
additional `setup` commands will be necessary.\\[2\baselineskip]To check
versions in use, type:\\
eups list -s

                    \vspace{\dp0}
                    } \end{minipage}
                \\ \hdashline


        \\ \midrule

                \multirow{3}{*}{\parbox{1.3cm}{ 3-1
                {\scriptsize from \hyperref[lvv-t1744]
                {LVV-T1744} } } }

                & {\small Description} &
                \begin{minipage}[t]{13cm}{\scriptsize
                Execute `validate\_drp` on a repository containing precursor data.
Identify the path to the data, which we will call `DATA/path', then
execute the following (with additional flags specified as needed):

                \vspace{\dp0}
                } \end{minipage} \\ \cdashline{2-3}
                & {\small Test Data} &
                \begin{minipage}[t]{13cm}{\scriptsize
                } \end{minipage} \\ \cdashline{2-3}
                & {\small Expected Result} &
                    \begin{minipage}[t]{13cm}{\scriptsize
                    JSON files (and associated figures) containing the Measurements and any
associated ``extras.''

                    \vspace{\dp0}
                    } \end{minipage}
                \\ \hdashline


        \\ \midrule

            \multirow{3}{*}{ 4 } & Description &
            \begin{minipage}[t]{13cm}{\footnotesize
            Confirm that the metric AM2 has been calculated, and that its values are
reasonable.

            \vspace{\dp0}
            } \end{minipage} \\ \cline{2-3}
            & Test Data &
            \begin{minipage}[t]{13cm}{\footnotesize
                No data.
                \vspace{\dp0}
            } \end{minipage} \\ \cline{2-3}
            & Expected Result &
                \begin{minipage}[t]{13cm}{\footnotesize
                A JSON file (and/or a report generated from that JSON file)
demonstrating that AM2 has been calculated.

                \vspace{\dp0}
                } \end{minipage}
        \\ \midrule
    \end{longtable}

\subsection{LVV-T1746 - Verify calculation of fraction of relative astrometric measurement error
on 5 arcminute scales exceeding outlier limit}\label{lvv-t1746}

\begin{longtable}[]{llllll}
\toprule
Version & Status & Priority & Verification Type & Owner
\\\midrule
1 & Approved & Normal &
Test & Jeffrey Carlin
\\\bottomrule
\multicolumn{6}{c}{ Open \href{https://jira.lsstcorp.org/secure/Tests.jspa\#/testCase/LVV-T1746}{LVV-T1746} in Jira } \\
\end{longtable}

\subsubsection{Verification Elements}
\begin{itemize}
\item \href{https://jira.lsstcorp.org/browse/LVV-9767}{LVV-9767} - DMS-REQ-0360-V-02: Max fraction exceeding limit on 5 arcmin scales

\item \href{https://jira.lsstcorp.org/browse/LVV-9773}{LVV-9773} - DMS-REQ-0360-V-07: Outlier limit on 5 arcmin scales

\end{itemize}

\subsubsection{Test Items}
Verify that the DM system has provided the code to calculate the maximum
fraction of relative astrometric measurements on 5 arcminute scales that
exceed the 5 arcminute outlier limit \textbf{AD1 = 20 milliarcseconds},
and assess whether it meets the requirement that it shall be less than
\textbf{AF1 = 10 percent.}


\subsubsection{Predecessors}

\subsubsection{Environment Needs}

\paragraph{Software}

\paragraph{Hardware}

\subsubsection{Input Specification}

\subsubsection{Output Specification}

\subsubsection{Test Procedure}
    \begin{longtable}[]{p{1.3cm}p{2cm}p{13cm}}
    %\toprule
    Step & \multicolumn{2}{@{}l}{Description, Input Data and Expected Result} \\ \toprule
    \endhead

            \multirow{3}{*}{ 1 } & Description &
            \begin{minipage}[t]{13cm}{\footnotesize
            Identify a dataset containing at least one field with multiple
overlapping visits.

            \vspace{\dp0}
            } \end{minipage} \\ \cline{2-3}
            & Test Data &
            \begin{minipage}[t]{13cm}{\footnotesize
                No data.
                \vspace{\dp0}
            } \end{minipage} \\ \cline{2-3}
            & Expected Result &
                \begin{minipage}[t]{13cm}{\footnotesize
                A dataset that has been ingested into a Butler repository.

                \vspace{\dp0}
                } \end{minipage}
        \\ \midrule

                \multirow{3}{*}{\parbox{1.3cm}{ 2-1
                {\scriptsize from \hyperref[lvv-t860]
                {LVV-T860} } } }

                & {\small Description} &
                \begin{minipage}[t]{13cm}{\scriptsize
                The `path` that you will use depends on where you are running the
science pipelines. Options:\\[2\baselineskip]

\begin{itemize}
\tightlist
\item
  local (newinstall.sh - based
  install):{[}path\_to\_installation{]}/loadLSST.bash
\item
  development cluster (``lsst-dev''):
  /software/lsstsw/stack/loadLSST.bash
\item
  LSP Notebook aspect (from a terminal):
  /opt/lsst/software/stack/loadLSST.bash
\end{itemize}

From the command line, execute the commands below in the example
code:\\[2\baselineskip]

                \vspace{\dp0}
                } \end{minipage} \\ \cdashline{2-3}
                & {\small Test Data} &
                \begin{minipage}[t]{13cm}{\scriptsize
                } \end{minipage} \\ \cdashline{2-3}
                & {\small Expected Result} &
                    \begin{minipage}[t]{13cm}{\scriptsize
                    Science pipeline software is available for use. If additional packages
are needed (for example, `obs' packages such as `obs\_subaru`), then
additional `setup` commands will be necessary.\\[2\baselineskip]To check
versions in use, type:\\
eups list -s

                    \vspace{\dp0}
                    } \end{minipage}
                \\ \hdashline


        \\ \midrule

                \multirow{3}{*}{\parbox{1.3cm}{ 3-1
                {\scriptsize from \hyperref[lvv-t1744]
                {LVV-T1744} } } }

                & {\small Description} &
                \begin{minipage}[t]{13cm}{\scriptsize
                Execute `validate\_drp` on a repository containing precursor data.
Identify the path to the data, which we will call `DATA/path', then
execute the following (with additional flags specified as needed):

                \vspace{\dp0}
                } \end{minipage} \\ \cdashline{2-3}
                & {\small Test Data} &
                \begin{minipage}[t]{13cm}{\scriptsize
                } \end{minipage} \\ \cdashline{2-3}
                & {\small Expected Result} &
                    \begin{minipage}[t]{13cm}{\scriptsize
                    JSON files (and associated figures) containing the Measurements and any
associated ``extras.''

                    \vspace{\dp0}
                    } \end{minipage}
                \\ \hdashline


        \\ \midrule

            \multirow{3}{*}{ 4 } & Description &
            \begin{minipage}[t]{13cm}{\footnotesize
            Confirm that the metric AF1 has been calculated using the outlier limit
AD1, and that its values are reasonable.

            \vspace{\dp0}
            } \end{minipage} \\ \cline{2-3}
            & Test Data &
            \begin{minipage}[t]{13cm}{\footnotesize
                No data.
                \vspace{\dp0}
            } \end{minipage} \\ \cline{2-3}
            & Expected Result &
                \begin{minipage}[t]{13cm}{\footnotesize
                A JSON file (and/or a report generated from that JSON file)
demonstrating that AF1 has been calculated (and used the limit AD1).

                \vspace{\dp0}
                } \end{minipage}
        \\ \midrule
    \end{longtable}

\subsection{LVV-T1747 - Verify calculation of relative astrometric measurement error on 5
arcminute scales}\label{lvv-t1747}

\begin{longtable}[]{llllll}
\toprule
Version & Status & Priority & Verification Type & Owner
\\\midrule
1 & Approved & Normal &
Test & Jeffrey Carlin
\\\bottomrule
\multicolumn{6}{c}{ Open \href{https://jira.lsstcorp.org/secure/Tests.jspa\#/testCase/LVV-T1747}{LVV-T1747} in Jira } \\
\end{longtable}

\subsubsection{Verification Elements}
\begin{itemize}
\item \href{https://jira.lsstcorp.org/browse/LVV-9768}{LVV-9768} - DMS-REQ-0360-V-03: Median astrometric error on 5 arcmin scales

\end{itemize}

\subsubsection{Test Items}
Verify that the DM system has provided the code to calculate the
relative astrometric measurement error on 5 arcminute scales, and assess
whether it meets the requirement that it shall be less
than\textbf{~\textbf{AM1 = 10 milliarcseconds.}}


\subsubsection{Predecessors}

\subsubsection{Environment Needs}

\paragraph{Software}

\paragraph{Hardware}

\subsubsection{Input Specification}

\subsubsection{Output Specification}

\subsubsection{Test Procedure}
    \begin{longtable}[]{p{1.3cm}p{2cm}p{13cm}}
    %\toprule
    Step & \multicolumn{2}{@{}l}{Description, Input Data and Expected Result} \\ \toprule
    \endhead

            \multirow{3}{*}{ 1 } & Description &
            \begin{minipage}[t]{13cm}{\footnotesize
            Identify a dataset containing at least one field with multiple
overlapping visits.

            \vspace{\dp0}
            } \end{minipage} \\ \cline{2-3}
            & Test Data &
            \begin{minipage}[t]{13cm}{\footnotesize
                No data.
                \vspace{\dp0}
            } \end{minipage} \\ \cline{2-3}
            & Expected Result &
                \begin{minipage}[t]{13cm}{\footnotesize
                A dataset that has been ingested into a Butler repository.

                \vspace{\dp0}
                } \end{minipage}
        \\ \midrule

                \multirow{3}{*}{\parbox{1.3cm}{ 2-1
                {\scriptsize from \hyperref[lvv-t860]
                {LVV-T860} } } }

                & {\small Description} &
                \begin{minipage}[t]{13cm}{\scriptsize
                The `path` that you will use depends on where you are running the
science pipelines. Options:\\[2\baselineskip]

\begin{itemize}
\tightlist
\item
  local (newinstall.sh - based
  install):{[}path\_to\_installation{]}/loadLSST.bash
\item
  development cluster (``lsst-dev''):
  /software/lsstsw/stack/loadLSST.bash
\item
  LSP Notebook aspect (from a terminal):
  /opt/lsst/software/stack/loadLSST.bash
\end{itemize}

From the command line, execute the commands below in the example
code:\\[2\baselineskip]

                \vspace{\dp0}
                } \end{minipage} \\ \cdashline{2-3}
                & {\small Test Data} &
                \begin{minipage}[t]{13cm}{\scriptsize
                } \end{minipage} \\ \cdashline{2-3}
                & {\small Expected Result} &
                    \begin{minipage}[t]{13cm}{\scriptsize
                    Science pipeline software is available for use. If additional packages
are needed (for example, `obs' packages such as `obs\_subaru`), then
additional `setup` commands will be necessary.\\[2\baselineskip]To check
versions in use, type:\\
eups list -s

                    \vspace{\dp0}
                    } \end{minipage}
                \\ \hdashline


        \\ \midrule

                \multirow{3}{*}{\parbox{1.3cm}{ 3-1
                {\scriptsize from \hyperref[lvv-t1744]
                {LVV-T1744} } } }

                & {\small Description} &
                \begin{minipage}[t]{13cm}{\scriptsize
                Execute `validate\_drp` on a repository containing precursor data.
Identify the path to the data, which we will call `DATA/path', then
execute the following (with additional flags specified as needed):

                \vspace{\dp0}
                } \end{minipage} \\ \cdashline{2-3}
                & {\small Test Data} &
                \begin{minipage}[t]{13cm}{\scriptsize
                } \end{minipage} \\ \cdashline{2-3}
                & {\small Expected Result} &
                    \begin{minipage}[t]{13cm}{\scriptsize
                    JSON files (and associated figures) containing the Measurements and any
associated ``extras.''

                    \vspace{\dp0}
                    } \end{minipage}
                \\ \hdashline


        \\ \midrule

            \multirow{3}{*}{ 4 } & Description &
            \begin{minipage}[t]{13cm}{\footnotesize
            Confirm that the metric AM1 has been calculated, and that its values are
reasonable.

            \vspace{\dp0}
            } \end{minipage} \\ \cline{2-3}
            & Test Data &
            \begin{minipage}[t]{13cm}{\footnotesize
                No data.
                \vspace{\dp0}
            } \end{minipage} \\ \cline{2-3}
            & Expected Result &
                \begin{minipage}[t]{13cm}{\footnotesize
                A JSON file (and/or a report generated from that JSON file)
demonstrating that AM1 has been calculated.

                \vspace{\dp0}
                } \end{minipage}
        \\ \midrule
    \end{longtable}

\subsection{LVV-T1748 - Verify calculation of median error in absolute position for RA, Dec axes}\label{lvv-t1748}

\begin{longtable}[]{llllll}
\toprule
Version & Status & Priority & Verification Type & Owner
\\\midrule
1 & Approved & Normal &
Test & Jeffrey Carlin
\\\bottomrule
\multicolumn{6}{c}{ Open \href{https://jira.lsstcorp.org/secure/Tests.jspa\#/testCase/LVV-T1748}{LVV-T1748} in Jira } \\
\end{longtable}

\subsubsection{Verification Elements}
\begin{itemize}
\item \href{https://jira.lsstcorp.org/browse/LVV-9769}{LVV-9769} - DMS-REQ-0360-V-04: Median absolute error in RA, Dec

\end{itemize}

\subsubsection{Test Items}
Verify that the DM system has provided the code to calculate the median
error in absolute position for each axis, RA and DEC, and assess whether
it meets the requirement that it shall be less than \textbf{AA1 = 50
milliarcseconds}.


\subsubsection{Predecessors}

\subsubsection{Environment Needs}

\paragraph{Software}

\paragraph{Hardware}

\subsubsection{Input Specification}

\subsubsection{Output Specification}

\subsubsection{Test Procedure}
    \begin{longtable}[]{p{1.3cm}p{2cm}p{13cm}}
    %\toprule
    Step & \multicolumn{2}{@{}l}{Description, Input Data and Expected Result} \\ \toprule
    \endhead

            \multirow{3}{*}{ 1 } & Description &
            \begin{minipage}[t]{13cm}{\footnotesize
            Identify a dataset containing at least one field with multiple
overlapping visits.

            \vspace{\dp0}
            } \end{minipage} \\ \cline{2-3}
            & Test Data &
            \begin{minipage}[t]{13cm}{\footnotesize
                No data.
                \vspace{\dp0}
            } \end{minipage} \\ \cline{2-3}
            & Expected Result &
                \begin{minipage}[t]{13cm}{\footnotesize
                A dataset that has been ingested into a Butler repository.

                \vspace{\dp0}
                } \end{minipage}
        \\ \midrule

                \multirow{3}{*}{\parbox{1.3cm}{ 2-1
                {\scriptsize from \hyperref[lvv-t860]
                {LVV-T860} } } }

                & {\small Description} &
                \begin{minipage}[t]{13cm}{\scriptsize
                The `path` that you will use depends on where you are running the
science pipelines. Options:\\[2\baselineskip]

\begin{itemize}
\tightlist
\item
  local (newinstall.sh - based
  install):{[}path\_to\_installation{]}/loadLSST.bash
\item
  development cluster (``lsst-dev''):
  /software/lsstsw/stack/loadLSST.bash
\item
  LSP Notebook aspect (from a terminal):
  /opt/lsst/software/stack/loadLSST.bash
\end{itemize}

From the command line, execute the commands below in the example
code:\\[2\baselineskip]

                \vspace{\dp0}
                } \end{minipage} \\ \cdashline{2-3}
                & {\small Test Data} &
                \begin{minipage}[t]{13cm}{\scriptsize
                } \end{minipage} \\ \cdashline{2-3}
                & {\small Expected Result} &
                    \begin{minipage}[t]{13cm}{\scriptsize
                    Science pipeline software is available for use. If additional packages
are needed (for example, `obs' packages such as `obs\_subaru`), then
additional `setup` commands will be necessary.\\[2\baselineskip]To check
versions in use, type:\\
eups list -s

                    \vspace{\dp0}
                    } \end{minipage}
                \\ \hdashline


        \\ \midrule

                \multirow{3}{*}{\parbox{1.3cm}{ 3-1
                {\scriptsize from \hyperref[lvv-t1744]
                {LVV-T1744} } } }

                & {\small Description} &
                \begin{minipage}[t]{13cm}{\scriptsize
                Execute `validate\_drp` on a repository containing precursor data.
Identify the path to the data, which we will call `DATA/path', then
execute the following (with additional flags specified as needed):

                \vspace{\dp0}
                } \end{minipage} \\ \cdashline{2-3}
                & {\small Test Data} &
                \begin{minipage}[t]{13cm}{\scriptsize
                } \end{minipage} \\ \cdashline{2-3}
                & {\small Expected Result} &
                    \begin{minipage}[t]{13cm}{\scriptsize
                    JSON files (and associated figures) containing the Measurements and any
associated ``extras.''

                    \vspace{\dp0}
                    } \end{minipage}
                \\ \hdashline


        \\ \midrule

            \multirow{3}{*}{ 4 } & Description &
            \begin{minipage}[t]{13cm}{\footnotesize
            Confirm that the metric AA1 has been calculated, and that its values are
reasonable.

            \vspace{\dp0}
            } \end{minipage} \\ \cline{2-3}
            & Test Data &
            \begin{minipage}[t]{13cm}{\footnotesize
                No data.
                \vspace{\dp0}
            } \end{minipage} \\ \cline{2-3}
            & Expected Result &
                \begin{minipage}[t]{13cm}{\footnotesize
                A JSON file (and/or a report generated from that JSON file)
demonstrating that AA1 has been calculated.

                \vspace{\dp0}
                } \end{minipage}
        \\ \midrule
    \end{longtable}

\subsection{LVV-T1749 - Verify calculation of fraction of relative astrometric measurement error
on 20 arcminute scales exceeding outlier limit}\label{lvv-t1749}

\begin{longtable}[]{llllll}
\toprule
Version & Status & Priority & Verification Type & Owner
\\\midrule
1 & Approved & Normal &
Test & Jeffrey Carlin
\\\bottomrule
\multicolumn{6}{c}{ Open \href{https://jira.lsstcorp.org/secure/Tests.jspa\#/testCase/LVV-T1749}{LVV-T1749} in Jira } \\
\end{longtable}

\subsubsection{Verification Elements}
\begin{itemize}
\item \href{https://jira.lsstcorp.org/browse/LVV-9776}{LVV-9776} - DMS-REQ-0360-V-10: Max fraction exceeding limit on 20 arcmin scales

\item \href{https://jira.lsstcorp.org/browse/LVV-9770}{LVV-9770} - DMS-REQ-0360-V-05: Outlier limit on 20 arcmin scales

\end{itemize}

\subsubsection{Test Items}
Verify that the DM system has provided the code to calculate the maximum
fraction of relative astrometric measurements on 20 arcminute scales
that exceed the 20 arcminute outlier limit \textbf{AD2 = 20
milliarcseconds}, and assess whether it meets the requirement that it
shall be less than \textbf{AF2 = 10 percent.}


\subsubsection{Predecessors}

\subsubsection{Environment Needs}

\paragraph{Software}

\paragraph{Hardware}

\subsubsection{Input Specification}

\subsubsection{Output Specification}

\subsubsection{Test Procedure}
    \begin{longtable}[]{p{1.3cm}p{2cm}p{13cm}}
    %\toprule
    Step & \multicolumn{2}{@{}l}{Description, Input Data and Expected Result} \\ \toprule
    \endhead

            \multirow{3}{*}{ 1 } & Description &
            \begin{minipage}[t]{13cm}{\footnotesize
            Identify a dataset containing at least one field with multiple
overlapping visits.

            \vspace{\dp0}
            } \end{minipage} \\ \cline{2-3}
            & Test Data &
            \begin{minipage}[t]{13cm}{\footnotesize
                No data.
                \vspace{\dp0}
            } \end{minipage} \\ \cline{2-3}
            & Expected Result &
                \begin{minipage}[t]{13cm}{\footnotesize
                A dataset that has been ingested into a Butler repository.

                \vspace{\dp0}
                } \end{minipage}
        \\ \midrule

                \multirow{3}{*}{\parbox{1.3cm}{ 2-1
                {\scriptsize from \hyperref[lvv-t860]
                {LVV-T860} } } }

                & {\small Description} &
                \begin{minipage}[t]{13cm}{\scriptsize
                The `path` that you will use depends on where you are running the
science pipelines. Options:\\[2\baselineskip]

\begin{itemize}
\tightlist
\item
  local (newinstall.sh - based
  install):{[}path\_to\_installation{]}/loadLSST.bash
\item
  development cluster (``lsst-dev''):
  /software/lsstsw/stack/loadLSST.bash
\item
  LSP Notebook aspect (from a terminal):
  /opt/lsst/software/stack/loadLSST.bash
\end{itemize}

From the command line, execute the commands below in the example
code:\\[2\baselineskip]

                \vspace{\dp0}
                } \end{minipage} \\ \cdashline{2-3}
                & {\small Test Data} &
                \begin{minipage}[t]{13cm}{\scriptsize
                } \end{minipage} \\ \cdashline{2-3}
                & {\small Expected Result} &
                    \begin{minipage}[t]{13cm}{\scriptsize
                    Science pipeline software is available for use. If additional packages
are needed (for example, `obs' packages such as `obs\_subaru`), then
additional `setup` commands will be necessary.\\[2\baselineskip]To check
versions in use, type:\\
eups list -s

                    \vspace{\dp0}
                    } \end{minipage}
                \\ \hdashline


        \\ \midrule

                \multirow{3}{*}{\parbox{1.3cm}{ 3-1
                {\scriptsize from \hyperref[lvv-t1744]
                {LVV-T1744} } } }

                & {\small Description} &
                \begin{minipage}[t]{13cm}{\scriptsize
                Execute `validate\_drp` on a repository containing precursor data.
Identify the path to the data, which we will call `DATA/path', then
execute the following (with additional flags specified as needed):

                \vspace{\dp0}
                } \end{minipage} \\ \cdashline{2-3}
                & {\small Test Data} &
                \begin{minipage}[t]{13cm}{\scriptsize
                } \end{minipage} \\ \cdashline{2-3}
                & {\small Expected Result} &
                    \begin{minipage}[t]{13cm}{\scriptsize
                    JSON files (and associated figures) containing the Measurements and any
associated ``extras.''

                    \vspace{\dp0}
                    } \end{minipage}
                \\ \hdashline


        \\ \midrule

            \multirow{3}{*}{ 4 } & Description &
            \begin{minipage}[t]{13cm}{\footnotesize
            Confirm that the metric AF2 has been calculated using the outlier limit
AD2, and that its values are reasonable.

            \vspace{\dp0}
            } \end{minipage} \\ \cline{2-3}
            & Test Data &
            \begin{minipage}[t]{13cm}{\footnotesize
                No data.
                \vspace{\dp0}
            } \end{minipage} \\ \cline{2-3}
            & Expected Result &
                \begin{minipage}[t]{13cm}{\footnotesize
                A JSON file (and/or a report generated from that JSON file)
demonstrating that AF2 has been calculated (and used the limit AD2).

                \vspace{\dp0}
                } \end{minipage}
        \\ \midrule
    \end{longtable}

\subsection{LVV-T1750 - Verify calculation of separations relative to r-band exceeding color
difference outlier limit}\label{lvv-t1750}

\begin{longtable}[]{llllll}
\toprule
Version & Status & Priority & Verification Type & Owner
\\\midrule
1 & Approved & Normal &
Test & Jeffrey Carlin
\\\bottomrule
\multicolumn{6}{c}{ Open \href{https://jira.lsstcorp.org/secure/Tests.jspa\#/testCase/LVV-T1750}{LVV-T1750} in Jira } \\
\end{longtable}

\subsubsection{Verification Elements}
\begin{itemize}
\item \href{https://jira.lsstcorp.org/browse/LVV-9771}{LVV-9771} - DMS-REQ-0360-V-06: Color difference outlier limit relative to r-band

\item \href{https://jira.lsstcorp.org/browse/LVV-9777}{LVV-9777} - DMS-REQ-0360-V-11: Max fraction of r-band color difference outliers

\end{itemize}

\subsubsection{Test Items}
Verify that the DM system has provided the code to calculate the
separations measured relative to the r-band that exceed the color
difference outlier limit \textbf{AB2 = 20 milliarcseconds}, and assess
whether it meets the requirement that it shall be less than \textbf{ABF1
= 10 percent.~}


\subsubsection{Predecessors}

\subsubsection{Environment Needs}

\paragraph{Software}

\paragraph{Hardware}

\subsubsection{Input Specification}

\subsubsection{Output Specification}

\subsubsection{Test Procedure}
    \begin{longtable}[]{p{1.3cm}p{2cm}p{13cm}}
    %\toprule
    Step & \multicolumn{2}{@{}l}{Description, Input Data and Expected Result} \\ \toprule
    \endhead

            \multirow{3}{*}{ 1 } & Description &
            \begin{minipage}[t]{13cm}{\footnotesize
            Identify a dataset containing at least one field with multiple
overlapping visits, and including at least one visit in r-band.

            \vspace{\dp0}
            } \end{minipage} \\ \cline{2-3}
            & Test Data &
            \begin{minipage}[t]{13cm}{\footnotesize
                No data.
                \vspace{\dp0}
            } \end{minipage} \\ \cline{2-3}
            & Expected Result &
                \begin{minipage}[t]{13cm}{\footnotesize
                A dataset that has been ingested into a Butler repository.

                \vspace{\dp0}
                } \end{minipage}
        \\ \midrule

                \multirow{3}{*}{\parbox{1.3cm}{ 2-1
                {\scriptsize from \hyperref[lvv-t860]
                {LVV-T860} } } }

                & {\small Description} &
                \begin{minipage}[t]{13cm}{\scriptsize
                The `path` that you will use depends on where you are running the
science pipelines. Options:\\[2\baselineskip]

\begin{itemize}
\tightlist
\item
  local (newinstall.sh - based
  install):{[}path\_to\_installation{]}/loadLSST.bash
\item
  development cluster (``lsst-dev''):
  /software/lsstsw/stack/loadLSST.bash
\item
  LSP Notebook aspect (from a terminal):
  /opt/lsst/software/stack/loadLSST.bash
\end{itemize}

From the command line, execute the commands below in the example
code:\\[2\baselineskip]

                \vspace{\dp0}
                } \end{minipage} \\ \cdashline{2-3}
                & {\small Test Data} &
                \begin{minipage}[t]{13cm}{\scriptsize
                } \end{minipage} \\ \cdashline{2-3}
                & {\small Expected Result} &
                    \begin{minipage}[t]{13cm}{\scriptsize
                    Science pipeline software is available for use. If additional packages
are needed (for example, `obs' packages such as `obs\_subaru`), then
additional `setup` commands will be necessary.\\[2\baselineskip]To check
versions in use, type:\\
eups list -s

                    \vspace{\dp0}
                    } \end{minipage}
                \\ \hdashline


        \\ \midrule

                \multirow{3}{*}{\parbox{1.3cm}{ 3-1
                {\scriptsize from \hyperref[lvv-t1744]
                {LVV-T1744} } } }

                & {\small Description} &
                \begin{minipage}[t]{13cm}{\scriptsize
                Execute `validate\_drp` on a repository containing precursor data.
Identify the path to the data, which we will call `DATA/path', then
execute the following (with additional flags specified as needed):

                \vspace{\dp0}
                } \end{minipage} \\ \cdashline{2-3}
                & {\small Test Data} &
                \begin{minipage}[t]{13cm}{\scriptsize
                } \end{minipage} \\ \cdashline{2-3}
                & {\small Expected Result} &
                    \begin{minipage}[t]{13cm}{\scriptsize
                    JSON files (and associated figures) containing the Measurements and any
associated ``extras.''

                    \vspace{\dp0}
                    } \end{minipage}
                \\ \hdashline


        \\ \midrule

            \multirow{3}{*}{ 4 } & Description &
            \begin{minipage}[t]{13cm}{\footnotesize
            Confirm that the metric ABF1 has been calculated using the outlier limit
AB2, and that its values are reasonable.

            \vspace{\dp0}
            } \end{minipage} \\ \cline{2-3}
            & Test Data &
            \begin{minipage}[t]{13cm}{\footnotesize
                No data.
                \vspace{\dp0}
            } \end{minipage} \\ \cline{2-3}
            & Expected Result &
                \begin{minipage}[t]{13cm}{\footnotesize
                A JSON file (and/or a report generated from that JSON file)
demonstrating that ABF1 has been calculated (and used the limit AB2).

                \vspace{\dp0}
                } \end{minipage}
        \\ \midrule
    \end{longtable}

\subsection{LVV-T1751 - Verify calculation of median relative astrometric measurement error on
200 arcminute scales}\label{lvv-t1751}

\begin{longtable}[]{llllll}
\toprule
Version & Status & Priority & Verification Type & Owner
\\\midrule
1 & Approved & Normal &
Test & Jeffrey Carlin
\\\bottomrule
\multicolumn{6}{c}{ Open \href{https://jira.lsstcorp.org/secure/Tests.jspa\#/testCase/LVV-T1751}{LVV-T1751} in Jira } \\
\end{longtable}

\subsubsection{Verification Elements}
\begin{itemize}
\item \href{https://jira.lsstcorp.org/browse/LVV-9774}{LVV-9774} - DMS-REQ-0360-V-08: Median astrometric error on 200 arcmin scales

\end{itemize}

\subsubsection{Test Items}
Verify that the DM system has provided the code to calculate the median
relative astrometric measurement error on 200 arcminute scales and
assess whether it meets the requirement that it shall be no more than
AM3 = 15 milliarcseconds.


\subsubsection{Predecessors}

\subsubsection{Environment Needs}

\paragraph{Software}

\paragraph{Hardware}

\subsubsection{Input Specification}

\subsubsection{Output Specification}

\subsubsection{Test Procedure}
    \begin{longtable}[]{p{1.3cm}p{2cm}p{13cm}}
    %\toprule
    Step & \multicolumn{2}{@{}l}{Description, Input Data and Expected Result} \\ \toprule
    \endhead

            \multirow{3}{*}{ 1 } & Description &
            \begin{minipage}[t]{13cm}{\footnotesize
            Identify a dataset containing at least one field with multiple
overlapping visits, and that covers an area larger than 200 arcminutes.

            \vspace{\dp0}
            } \end{minipage} \\ \cline{2-3}
            & Test Data &
            \begin{minipage}[t]{13cm}{\footnotesize
                No data.
                \vspace{\dp0}
            } \end{minipage} \\ \cline{2-3}
            & Expected Result &
                \begin{minipage}[t]{13cm}{\footnotesize
                A dataset that has been ingested into a Butler repository.

                \vspace{\dp0}
                } \end{minipage}
        \\ \midrule

                \multirow{3}{*}{\parbox{1.3cm}{ 2-1
                {\scriptsize from \hyperref[lvv-t860]
                {LVV-T860} } } }

                & {\small Description} &
                \begin{minipage}[t]{13cm}{\scriptsize
                The `path` that you will use depends on where you are running the
science pipelines. Options:\\[2\baselineskip]

\begin{itemize}
\tightlist
\item
  local (newinstall.sh - based
  install):{[}path\_to\_installation{]}/loadLSST.bash
\item
  development cluster (``lsst-dev''):
  /software/lsstsw/stack/loadLSST.bash
\item
  LSP Notebook aspect (from a terminal):
  /opt/lsst/software/stack/loadLSST.bash
\end{itemize}

From the command line, execute the commands below in the example
code:\\[2\baselineskip]

                \vspace{\dp0}
                } \end{minipage} \\ \cdashline{2-3}
                & {\small Test Data} &
                \begin{minipage}[t]{13cm}{\scriptsize
                } \end{minipage} \\ \cdashline{2-3}
                & {\small Expected Result} &
                    \begin{minipage}[t]{13cm}{\scriptsize
                    Science pipeline software is available for use. If additional packages
are needed (for example, `obs' packages such as `obs\_subaru`), then
additional `setup` commands will be necessary.\\[2\baselineskip]To check
versions in use, type:\\
eups list -s

                    \vspace{\dp0}
                    } \end{minipage}
                \\ \hdashline


        \\ \midrule

                \multirow{3}{*}{\parbox{1.3cm}{ 3-1
                {\scriptsize from \hyperref[lvv-t1744]
                {LVV-T1744} } } }

                & {\small Description} &
                \begin{minipage}[t]{13cm}{\scriptsize
                Execute `validate\_drp` on a repository containing precursor data.
Identify the path to the data, which we will call `DATA/path', then
execute the following (with additional flags specified as needed):

                \vspace{\dp0}
                } \end{minipage} \\ \cdashline{2-3}
                & {\small Test Data} &
                \begin{minipage}[t]{13cm}{\scriptsize
                } \end{minipage} \\ \cdashline{2-3}
                & {\small Expected Result} &
                    \begin{minipage}[t]{13cm}{\scriptsize
                    JSON files (and associated figures) containing the Measurements and any
associated ``extras.''

                    \vspace{\dp0}
                    } \end{minipage}
                \\ \hdashline


        \\ \midrule

            \multirow{3}{*}{ 4 } & Description &
            \begin{minipage}[t]{13cm}{\footnotesize
            Confirm that the metric AM3 has been calculated, and that its values are
reasonable.

            \vspace{\dp0}
            } \end{minipage} \\ \cline{2-3}
            & Test Data &
            \begin{minipage}[t]{13cm}{\footnotesize
                No data.
                \vspace{\dp0}
            } \end{minipage} \\ \cline{2-3}
            & Expected Result &
                \begin{minipage}[t]{13cm}{\footnotesize
                A JSON file (and/or a report generated from that JSON file)
demonstrating that AM3 has been calculated.

                \vspace{\dp0}
                } \end{minipage}
        \\ \midrule
    \end{longtable}

\subsection{LVV-T1752 - Verify calculation of fraction of relative astrometric measurement error
on 200 arcminute scales exceeding outlier limit}\label{lvv-t1752}

\begin{longtable}[]{llllll}
\toprule
Version & Status & Priority & Verification Type & Owner
\\\midrule
1 & Approved & Normal &
Test & Jeffrey Carlin
\\\bottomrule
\multicolumn{6}{c}{ Open \href{https://jira.lsstcorp.org/secure/Tests.jspa\#/testCase/LVV-T1752}{LVV-T1752} in Jira } \\
\end{longtable}

\subsubsection{Verification Elements}
\begin{itemize}
\item \href{https://jira.lsstcorp.org/browse/LVV-9779}{LVV-9779} - DMS-REQ-0360-V-13: Max fraction exceeding limit on 200 arcmin scales

\end{itemize}

\subsubsection{Test Items}
Verify that the DM system has provided the code to calculate the maximum
fraction of relative astrometric measurements on 200 arcminute scales
that exceed the 200 arcminute outlier limit \textbf{AD3 = 30
milliarcseconds}, and assess whether it meets the requirement that it
shall be less than \textbf{AF3 = 10 percent.}


\subsubsection{Predecessors}

\subsubsection{Environment Needs}

\paragraph{Software}

\paragraph{Hardware}

\subsubsection{Input Specification}

\subsubsection{Output Specification}

\subsubsection{Test Procedure}
    \begin{longtable}[]{p{1.3cm}p{2cm}p{13cm}}
    %\toprule
    Step & \multicolumn{2}{@{}l}{Description, Input Data and Expected Result} \\ \toprule
    \endhead

            \multirow{3}{*}{ 1 } & Description &
            \begin{minipage}[t]{13cm}{\footnotesize
            Identify a dataset containing at least one field with multiple
overlapping visits, and that covers an area larger than 200 arcminutes.

            \vspace{\dp0}
            } \end{minipage} \\ \cline{2-3}
            & Test Data &
            \begin{minipage}[t]{13cm}{\footnotesize
                No data.
                \vspace{\dp0}
            } \end{minipage} \\ \cline{2-3}
            & Expected Result &
                \begin{minipage}[t]{13cm}{\footnotesize
                A dataset that has been ingested into a Butler repository.

                \vspace{\dp0}
                } \end{minipage}
        \\ \midrule

                \multirow{3}{*}{\parbox{1.3cm}{ 2-1
                {\scriptsize from \hyperref[lvv-t860]
                {LVV-T860} } } }

                & {\small Description} &
                \begin{minipage}[t]{13cm}{\scriptsize
                The `path` that you will use depends on where you are running the
science pipelines. Options:\\[2\baselineskip]

\begin{itemize}
\tightlist
\item
  local (newinstall.sh - based
  install):{[}path\_to\_installation{]}/loadLSST.bash
\item
  development cluster (``lsst-dev''):
  /software/lsstsw/stack/loadLSST.bash
\item
  LSP Notebook aspect (from a terminal):
  /opt/lsst/software/stack/loadLSST.bash
\end{itemize}

From the command line, execute the commands below in the example
code:\\[2\baselineskip]

                \vspace{\dp0}
                } \end{minipage} \\ \cdashline{2-3}
                & {\small Test Data} &
                \begin{minipage}[t]{13cm}{\scriptsize
                } \end{minipage} \\ \cdashline{2-3}
                & {\small Expected Result} &
                    \begin{minipage}[t]{13cm}{\scriptsize
                    Science pipeline software is available for use. If additional packages
are needed (for example, `obs' packages such as `obs\_subaru`), then
additional `setup` commands will be necessary.\\[2\baselineskip]To check
versions in use, type:\\
eups list -s

                    \vspace{\dp0}
                    } \end{minipage}
                \\ \hdashline


        \\ \midrule

                \multirow{3}{*}{\parbox{1.3cm}{ 3-1
                {\scriptsize from \hyperref[lvv-t1744]
                {LVV-T1744} } } }

                & {\small Description} &
                \begin{minipage}[t]{13cm}{\scriptsize
                Execute `validate\_drp` on a repository containing precursor data.
Identify the path to the data, which we will call `DATA/path', then
execute the following (with additional flags specified as needed):

                \vspace{\dp0}
                } \end{minipage} \\ \cdashline{2-3}
                & {\small Test Data} &
                \begin{minipage}[t]{13cm}{\scriptsize
                } \end{minipage} \\ \cdashline{2-3}
                & {\small Expected Result} &
                    \begin{minipage}[t]{13cm}{\scriptsize
                    JSON files (and associated figures) containing the Measurements and any
associated ``extras.''

                    \vspace{\dp0}
                    } \end{minipage}
                \\ \hdashline


        \\ \midrule

            \multirow{3}{*}{ 4 } & Description &
            \begin{minipage}[t]{13cm}{\footnotesize
            Confirm that the metric AF3 has been calculated using the outlier limit
AD3, and that its values are reasonable.

            \vspace{\dp0}
            } \end{minipage} \\ \cline{2-3}
            & Test Data &
            \begin{minipage}[t]{13cm}{\footnotesize
                No data.
                \vspace{\dp0}
            } \end{minipage} \\ \cline{2-3}
            & Expected Result &
                \begin{minipage}[t]{13cm}{\footnotesize
                A JSON file (and/or a report generated from that JSON file)
demonstrating that AF3 has been calculated (and used the limit AD3).

                \vspace{\dp0}
                } \end{minipage}
        \\ \midrule
    \end{longtable}

\subsection{LVV-T1753 - Verify calculation of RMS difference of separations relative to r-band}\label{lvv-t1753}

\begin{longtable}[]{llllll}
\toprule
Version & Status & Priority & Verification Type & Owner
\\\midrule
1 & Approved & Normal &
Test & Jeffrey Carlin
\\\bottomrule
\multicolumn{6}{c}{ Open \href{https://jira.lsstcorp.org/secure/Tests.jspa\#/testCase/LVV-T1753}{LVV-T1753} in Jira } \\
\end{longtable}

\subsubsection{Verification Elements}
\begin{itemize}
\item \href{https://jira.lsstcorp.org/browse/LVV-9778}{LVV-9778} - DMS-REQ-0360-V-12: RMS difference between r-band and other filter
separation

\end{itemize}

\subsubsection{Test Items}
Verify that the DM system has provided the code to calculate the
separations measured relative to the r-band, and assess whether it meets
the requirement that it shall be less than \textbf{AB1 =
10~milliarcseconds.}


\subsubsection{Predecessors}

\subsubsection{Environment Needs}

\paragraph{Software}

\paragraph{Hardware}

\subsubsection{Input Specification}

\subsubsection{Output Specification}

\subsubsection{Test Procedure}
    \begin{longtable}[]{p{1.3cm}p{2cm}p{13cm}}
    %\toprule
    Step & \multicolumn{2}{@{}l}{Description, Input Data and Expected Result} \\ \toprule
    \endhead

            \multirow{3}{*}{ 1 } & Description &
            \begin{minipage}[t]{13cm}{\footnotesize
            Identify a dataset containing at least one field with multiple
overlapping visits, and including at least one visit in r-band.

            \vspace{\dp0}
            } \end{minipage} \\ \cline{2-3}
            & Test Data &
            \begin{minipage}[t]{13cm}{\footnotesize
                No data.
                \vspace{\dp0}
            } \end{minipage} \\ \cline{2-3}
            & Expected Result &
                \begin{minipage}[t]{13cm}{\footnotesize
                A dataset that has been ingested into a Butler repository.

                \vspace{\dp0}
                } \end{minipage}
        \\ \midrule

                \multirow{3}{*}{\parbox{1.3cm}{ 2-1
                {\scriptsize from \hyperref[lvv-t860]
                {LVV-T860} } } }

                & {\small Description} &
                \begin{minipage}[t]{13cm}{\scriptsize
                The `path` that you will use depends on where you are running the
science pipelines. Options:\\[2\baselineskip]

\begin{itemize}
\tightlist
\item
  local (newinstall.sh - based
  install):{[}path\_to\_installation{]}/loadLSST.bash
\item
  development cluster (``lsst-dev''):
  /software/lsstsw/stack/loadLSST.bash
\item
  LSP Notebook aspect (from a terminal):
  /opt/lsst/software/stack/loadLSST.bash
\end{itemize}

From the command line, execute the commands below in the example
code:\\[2\baselineskip]

                \vspace{\dp0}
                } \end{minipage} \\ \cdashline{2-3}
                & {\small Test Data} &
                \begin{minipage}[t]{13cm}{\scriptsize
                } \end{minipage} \\ \cdashline{2-3}
                & {\small Expected Result} &
                    \begin{minipage}[t]{13cm}{\scriptsize
                    Science pipeline software is available for use. If additional packages
are needed (for example, `obs' packages such as `obs\_subaru`), then
additional `setup` commands will be necessary.\\[2\baselineskip]To check
versions in use, type:\\
eups list -s

                    \vspace{\dp0}
                    } \end{minipage}
                \\ \hdashline


        \\ \midrule

                \multirow{3}{*}{\parbox{1.3cm}{ 3-1
                {\scriptsize from \hyperref[lvv-t1744]
                {LVV-T1744} } } }

                & {\small Description} &
                \begin{minipage}[t]{13cm}{\scriptsize
                Execute `validate\_drp` on a repository containing precursor data.
Identify the path to the data, which we will call `DATA/path', then
execute the following (with additional flags specified as needed):

                \vspace{\dp0}
                } \end{minipage} \\ \cdashline{2-3}
                & {\small Test Data} &
                \begin{minipage}[t]{13cm}{\scriptsize
                } \end{minipage} \\ \cdashline{2-3}
                & {\small Expected Result} &
                    \begin{minipage}[t]{13cm}{\scriptsize
                    JSON files (and associated figures) containing the Measurements and any
associated ``extras.''

                    \vspace{\dp0}
                    } \end{minipage}
                \\ \hdashline


        \\ \midrule

            \multirow{3}{*}{ 4 } & Description &
            \begin{minipage}[t]{13cm}{\footnotesize
            Confirm that the metric AB1 has been calculated, and that its values are
reasonable.

            \vspace{\dp0}
            } \end{minipage} \\ \cline{2-3}
            & Test Data &
            \begin{minipage}[t]{13cm}{\footnotesize
                No data.
                \vspace{\dp0}
            } \end{minipage} \\ \cline{2-3}
            & Expected Result &
                \begin{minipage}[t]{13cm}{\footnotesize
                A JSON file (and/or a report generated from that JSON file)
demonstrating that AB1 has been calculated.

                \vspace{\dp0}
                } \end{minipage}
        \\ \midrule
    \end{longtable}

\subsection{LVV-T1754 - Verify calculation of residual PSF ellipticity correlations for
separations less than 5 arcmin}\label{lvv-t1754}

\begin{longtable}[]{llllll}
\toprule
Version & Status & Priority & Verification Type & Owner
\\\midrule
1 & Approved & Normal &
Test & Jeffrey Carlin
\\\bottomrule
\multicolumn{6}{c}{ Open \href{https://jira.lsstcorp.org/secure/Tests.jspa\#/testCase/LVV-T1754}{LVV-T1754} in Jira } \\
\end{longtable}

\subsubsection{Verification Elements}
\begin{itemize}
\item \href{https://jira.lsstcorp.org/browse/LVV-3404}{LVV-3404} - DMS-REQ-0362-V-01: Median residual PSF ellipticity correlations on 5
arcmin scales

\end{itemize}

\subsubsection{Test Items}
Verify that the DM system has provided the code to calculate the median
residual PSF ellipticity correlations averaged over an arbitrary field
of view for separations less than 5 arcmin, and assess whether it meets
the requirement that it shall be no greater than \textbf{TE2 =
1.0e-7{[}arcminuteSeparationCorrelation{]}.}


\subsubsection{Predecessors}

\subsubsection{Environment Needs}

\paragraph{Software}

\paragraph{Hardware}

\subsubsection{Input Specification}

\subsubsection{Output Specification}

\subsubsection{Test Procedure}
    \begin{longtable}[]{p{1.3cm}p{2cm}p{13cm}}
    %\toprule
    Step & \multicolumn{2}{@{}l}{Description, Input Data and Expected Result} \\ \toprule
    \endhead

            \multirow{3}{*}{ 1 } & Description &
            \begin{minipage}[t]{13cm}{\footnotesize
            Identify a dataset containing at least one field with multiple
overlapping visits.

            \vspace{\dp0}
            } \end{minipage} \\ \cline{2-3}
            & Test Data &
            \begin{minipage}[t]{13cm}{\footnotesize
                No data.
                \vspace{\dp0}
            } \end{minipage} \\ \cline{2-3}
            & Expected Result &
                \begin{minipage}[t]{13cm}{\footnotesize
                A dataset that has been ingested into a Butler repository.

                \vspace{\dp0}
                } \end{minipage}
        \\ \midrule

                \multirow{3}{*}{\parbox{1.3cm}{ 2-1
                {\scriptsize from \hyperref[lvv-t860]
                {LVV-T860} } } }

                & {\small Description} &
                \begin{minipage}[t]{13cm}{\scriptsize
                The `path` that you will use depends on where you are running the
science pipelines. Options:\\[2\baselineskip]

\begin{itemize}
\tightlist
\item
  local (newinstall.sh - based
  install):{[}path\_to\_installation{]}/loadLSST.bash
\item
  development cluster (``lsst-dev''):
  /software/lsstsw/stack/loadLSST.bash
\item
  LSP Notebook aspect (from a terminal):
  /opt/lsst/software/stack/loadLSST.bash
\end{itemize}

From the command line, execute the commands below in the example
code:\\[2\baselineskip]

                \vspace{\dp0}
                } \end{minipage} \\ \cdashline{2-3}
                & {\small Test Data} &
                \begin{minipage}[t]{13cm}{\scriptsize
                } \end{minipage} \\ \cdashline{2-3}
                & {\small Expected Result} &
                    \begin{minipage}[t]{13cm}{\scriptsize
                    Science pipeline software is available for use. If additional packages
are needed (for example, `obs' packages such as `obs\_subaru`), then
additional `setup` commands will be necessary.\\[2\baselineskip]To check
versions in use, type:\\
eups list -s

                    \vspace{\dp0}
                    } \end{minipage}
                \\ \hdashline


        \\ \midrule

                \multirow{3}{*}{\parbox{1.3cm}{ 3-1
                {\scriptsize from \hyperref[lvv-t1744]
                {LVV-T1744} } } }

                & {\small Description} &
                \begin{minipage}[t]{13cm}{\scriptsize
                Execute `validate\_drp` on a repository containing precursor data.
Identify the path to the data, which we will call `DATA/path', then
execute the following (with additional flags specified as needed):

                \vspace{\dp0}
                } \end{minipage} \\ \cdashline{2-3}
                & {\small Test Data} &
                \begin{minipage}[t]{13cm}{\scriptsize
                } \end{minipage} \\ \cdashline{2-3}
                & {\small Expected Result} &
                    \begin{minipage}[t]{13cm}{\scriptsize
                    JSON files (and associated figures) containing the Measurements and any
associated ``extras.''

                    \vspace{\dp0}
                    } \end{minipage}
                \\ \hdashline


        \\ \midrule

            \multirow{3}{*}{ 4 } & Description &
            \begin{minipage}[t]{13cm}{\footnotesize
            Confirm that the metric TE2 has been calculated, and that its values are
reasonable.

            \vspace{\dp0}
            } \end{minipage} \\ \cline{2-3}
            & Test Data &
            \begin{minipage}[t]{13cm}{\footnotesize
                No data.
                \vspace{\dp0}
            } \end{minipage} \\ \cline{2-3}
            & Expected Result &
                \begin{minipage}[t]{13cm}{\footnotesize
                A JSON file (and/or a report generated from that JSON file)
demonstrating that TE2 has been calculated.

                \vspace{\dp0}
                } \end{minipage}
        \\ \midrule
    \end{longtable}

\subsection{LVV-T1755 - Verify calculation of residual PSF ellipticity correlations for
separations less than 1 arcmin}\label{lvv-t1755}

\begin{longtable}[]{llllll}
\toprule
Version & Status & Priority & Verification Type & Owner
\\\midrule
1 & Approved & Normal &
Test & Jeffrey Carlin
\\\bottomrule
\multicolumn{6}{c}{ Open \href{https://jira.lsstcorp.org/secure/Tests.jspa\#/testCase/LVV-T1755}{LVV-T1755} in Jira } \\
\end{longtable}

\subsubsection{Verification Elements}
\begin{itemize}
\item \href{https://jira.lsstcorp.org/browse/LVV-9782}{LVV-9782} - DMS-REQ-0362-V-04: Median residual PSF ellipticity correlations on 1
arcmin scales

\end{itemize}

\subsubsection{Test Items}
Verify that the DM system has provided the code to calculate the median
residual PSF ellipticity correlations averaged over an arbitrary field
of view for separations less than 1 arcmin, and assess whether it meets
the requirement that it shall be no greater than \textbf{TE1 =
2.0e-5{[}arcminuteSeparationCorrelation{]}.}


\subsubsection{Predecessors}

\subsubsection{Environment Needs}

\paragraph{Software}

\paragraph{Hardware}

\subsubsection{Input Specification}

\subsubsection{Output Specification}

\subsubsection{Test Procedure}
    \begin{longtable}[]{p{1.3cm}p{2cm}p{13cm}}
    %\toprule
    Step & \multicolumn{2}{@{}l}{Description, Input Data and Expected Result} \\ \toprule
    \endhead

            \multirow{3}{*}{ 1 } & Description &
            \begin{minipage}[t]{13cm}{\footnotesize
            Identify a dataset containing at least one field with multiple
overlapping visits.

            \vspace{\dp0}
            } \end{minipage} \\ \cline{2-3}
            & Test Data &
            \begin{minipage}[t]{13cm}{\footnotesize
                No data.
                \vspace{\dp0}
            } \end{minipage} \\ \cline{2-3}
            & Expected Result &
                \begin{minipage}[t]{13cm}{\footnotesize
                A dataset that has been ingested into a Butler repository.

                \vspace{\dp0}
                } \end{minipage}
        \\ \midrule

                \multirow{3}{*}{\parbox{1.3cm}{ 2-1
                {\scriptsize from \hyperref[lvv-t860]
                {LVV-T860} } } }

                & {\small Description} &
                \begin{minipage}[t]{13cm}{\scriptsize
                The `path` that you will use depends on where you are running the
science pipelines. Options:\\[2\baselineskip]

\begin{itemize}
\tightlist
\item
  local (newinstall.sh - based
  install):{[}path\_to\_installation{]}/loadLSST.bash
\item
  development cluster (``lsst-dev''):
  /software/lsstsw/stack/loadLSST.bash
\item
  LSP Notebook aspect (from a terminal):
  /opt/lsst/software/stack/loadLSST.bash
\end{itemize}

From the command line, execute the commands below in the example
code:\\[2\baselineskip]

                \vspace{\dp0}
                } \end{minipage} \\ \cdashline{2-3}
                & {\small Test Data} &
                \begin{minipage}[t]{13cm}{\scriptsize
                } \end{minipage} \\ \cdashline{2-3}
                & {\small Expected Result} &
                    \begin{minipage}[t]{13cm}{\scriptsize
                    Science pipeline software is available for use. If additional packages
are needed (for example, `obs' packages such as `obs\_subaru`), then
additional `setup` commands will be necessary.\\[2\baselineskip]To check
versions in use, type:\\
eups list -s

                    \vspace{\dp0}
                    } \end{minipage}
                \\ \hdashline


        \\ \midrule

                \multirow{3}{*}{\parbox{1.3cm}{ 3-1
                {\scriptsize from \hyperref[lvv-t1744]
                {LVV-T1744} } } }

                & {\small Description} &
                \begin{minipage}[t]{13cm}{\scriptsize
                Execute `validate\_drp` on a repository containing precursor data.
Identify the path to the data, which we will call `DATA/path', then
execute the following (with additional flags specified as needed):

                \vspace{\dp0}
                } \end{minipage} \\ \cdashline{2-3}
                & {\small Test Data} &
                \begin{minipage}[t]{13cm}{\scriptsize
                } \end{minipage} \\ \cdashline{2-3}
                & {\small Expected Result} &
                    \begin{minipage}[t]{13cm}{\scriptsize
                    JSON files (and associated figures) containing the Measurements and any
associated ``extras.''

                    \vspace{\dp0}
                    } \end{minipage}
                \\ \hdashline


        \\ \midrule

            \multirow{3}{*}{ 4 } & Description &
            \begin{minipage}[t]{13cm}{\footnotesize
            Confirm that the metric TE1 has been calculated, and that its values are
reasonable.

            \vspace{\dp0}
            } \end{minipage} \\ \cline{2-3}
            & Test Data &
            \begin{minipage}[t]{13cm}{\footnotesize
                No data.
                \vspace{\dp0}
            } \end{minipage} \\ \cline{2-3}
            & Expected Result &
                \begin{minipage}[t]{13cm}{\footnotesize
                A JSON file (and/or a report generated from that JSON file)
demonstrating that TE1 has been calculated.

                \vspace{\dp0}
                } \end{minipage}
        \\ \midrule
    \end{longtable}

\subsection{LVV-T1756 - Verify calculation of photometric repeatability in uzy filters}\label{lvv-t1756}

\begin{longtable}[]{llllll}
\toprule
Version & Status & Priority & Verification Type & Owner
\\\midrule
1 & Approved & Normal &
Test & Jeffrey Carlin
\\\bottomrule
\multicolumn{6}{c}{ Open \href{https://jira.lsstcorp.org/secure/Tests.jspa\#/testCase/LVV-T1756}{LVV-T1756} in Jira } \\
\end{longtable}

\subsubsection{Verification Elements}
\begin{itemize}
\item \href{https://jira.lsstcorp.org/browse/LVV-3401}{LVV-3401} - DMS-REQ-0359-V-01: RMS photometric repeatability in uzy

\end{itemize}

\subsubsection{Test Items}
Verify that the DM system has provided the code to calculate the RMS
photometric repeatability of bright non-saturated unresolved point
sources in the u, z, and y filters, and assess whether it meets the
requirement that it shall be less than \textbf{PA1uzy = 7.5
millimagnitudes}.


\subsubsection{Predecessors}

\subsubsection{Environment Needs}

\paragraph{Software}

\paragraph{Hardware}

\subsubsection{Input Specification}

\subsubsection{Output Specification}

\subsubsection{Test Procedure}
    \begin{longtable}[]{p{1.3cm}p{2cm}p{13cm}}
    %\toprule
    Step & \multicolumn{2}{@{}l}{Description, Input Data and Expected Result} \\ \toprule
    \endhead

            \multirow{3}{*}{ 1 } & Description &
            \begin{minipage}[t]{13cm}{\footnotesize
            Identify a dataset containing at least one field in each of the u, z,
and y filters with multiple overlapping visits.

            \vspace{\dp0}
            } \end{minipage} \\ \cline{2-3}
            & Test Data &
            \begin{minipage}[t]{13cm}{\footnotesize
                No data.
                \vspace{\dp0}
            } \end{minipage} \\ \cline{2-3}
            & Expected Result &
                \begin{minipage}[t]{13cm}{\footnotesize
                A dataset that has been ingested into a Butler repository.

                \vspace{\dp0}
                } \end{minipage}
        \\ \midrule

                \multirow{3}{*}{\parbox{1.3cm}{ 2-1
                {\scriptsize from \hyperref[lvv-t1744]
                {LVV-T1744} } } }

                & {\small Description} &
                \begin{minipage}[t]{13cm}{\scriptsize
                Execute `validate\_drp` on a repository containing precursor data.
Identify the path to the data, which we will call `DATA/path', then
execute the following (with additional flags specified as needed):

                \vspace{\dp0}
                } \end{minipage} \\ \cdashline{2-3}
                & {\small Test Data} &
                \begin{minipage}[t]{13cm}{\scriptsize
                } \end{minipage} \\ \cdashline{2-3}
                & {\small Expected Result} &
                    \begin{minipage}[t]{13cm}{\scriptsize
                    JSON files (and associated figures) containing the Measurements and any
associated ``extras.''

                    \vspace{\dp0}
                    } \end{minipage}
                \\ \hdashline


        \\ \midrule

            \multirow{3}{*}{ 3 } & Description &
            \begin{minipage}[t]{13cm}{\footnotesize
            Confirm that the metric PA1uzy has been calculated, and that its values
are reasonable.

            \vspace{\dp0}
            } \end{minipage} \\ \cline{2-3}
            & Test Data &
            \begin{minipage}[t]{13cm}{\footnotesize
                No data.
                \vspace{\dp0}
            } \end{minipage} \\ \cline{2-3}
            & Expected Result &
                \begin{minipage}[t]{13cm}{\footnotesize
                A JSON file (and/or a report generated from that JSON file)
demonstrating that PA1uzy has been calculated.

                \vspace{\dp0}
                } \end{minipage}
        \\ \midrule
    \end{longtable}

\subsection{LVV-T1757 - Verify calculation of photometric repeatability in gri filters}\label{lvv-t1757}

\begin{longtable}[]{llllll}
\toprule
Version & Status & Priority & Verification Type & Owner
\\\midrule
1 & Approved & Normal &
Test & Jeffrey Carlin
\\\bottomrule
\multicolumn{6}{c}{ Open \href{https://jira.lsstcorp.org/secure/Tests.jspa\#/testCase/LVV-T1757}{LVV-T1757} in Jira } \\
\end{longtable}

\subsubsection{Verification Elements}
\begin{itemize}
\item \href{https://jira.lsstcorp.org/browse/LVV-9759}{LVV-9759} - DMS-REQ-0359-V-10: RMS photometric repeatability in gri

\end{itemize}

\subsubsection{Test Items}
Verify that the DM system has provided the code to calculate the RMS
photometric repeatability of bright non-saturated unresolved point
sources in the g, r, and i filters, and assess whether it meets the
requirement that it shall be less than \textbf{PA1gri = 5.0
millimagnitudes}.


\subsubsection{Predecessors}

\subsubsection{Environment Needs}

\paragraph{Software}

\paragraph{Hardware}

\subsubsection{Input Specification}

\subsubsection{Output Specification}

\subsubsection{Test Procedure}
    \begin{longtable}[]{p{1.3cm}p{2cm}p{13cm}}
    %\toprule
    Step & \multicolumn{2}{@{}l}{Description, Input Data and Expected Result} \\ \toprule
    \endhead

            \multirow{3}{*}{ 1 } & Description &
            \begin{minipage}[t]{13cm}{\footnotesize
            Identify a dataset containing at least one field in each of the g, r,
and i filters with multiple overlapping visits.

            \vspace{\dp0}
            } \end{minipage} \\ \cline{2-3}
            & Test Data &
            \begin{minipage}[t]{13cm}{\footnotesize
                No data.
                \vspace{\dp0}
            } \end{minipage} \\ \cline{2-3}
            & Expected Result &
                \begin{minipage}[t]{13cm}{\footnotesize
                A dataset that has been ingested into a Butler repository.

                \vspace{\dp0}
                } \end{minipage}
        \\ \midrule

                \multirow{3}{*}{\parbox{1.3cm}{ 2-1
                {\scriptsize from \hyperref[lvv-t1744]
                {LVV-T1744} } } }

                & {\small Description} &
                \begin{minipage}[t]{13cm}{\scriptsize
                Execute `validate\_drp` on a repository containing precursor data.
Identify the path to the data, which we will call `DATA/path', then
execute the following (with additional flags specified as needed):

                \vspace{\dp0}
                } \end{minipage} \\ \cdashline{2-3}
                & {\small Test Data} &
                \begin{minipage}[t]{13cm}{\scriptsize
                } \end{minipage} \\ \cdashline{2-3}
                & {\small Expected Result} &
                    \begin{minipage}[t]{13cm}{\scriptsize
                    JSON files (and associated figures) containing the Measurements and any
associated ``extras.''

                    \vspace{\dp0}
                    } \end{minipage}
                \\ \hdashline


        \\ \midrule

            \multirow{3}{*}{ 3 } & Description &
            \begin{minipage}[t]{13cm}{\footnotesize
            Confirm that the metric PA1gri has been calculated, and that its values
are reasonable.

            \vspace{\dp0}
            } \end{minipage} \\ \cline{2-3}
            & Test Data &
            \begin{minipage}[t]{13cm}{\footnotesize
                No data.
                \vspace{\dp0}
            } \end{minipage} \\ \cline{2-3}
            & Expected Result &
                \begin{minipage}[t]{13cm}{\footnotesize
                A JSON file (and/or a report generated from that JSON file)
demonstrating that PA1gri has been calculated.

                \vspace{\dp0}
                } \end{minipage}
        \\ \midrule
    \end{longtable}

\subsection{LVV-T1758 - Verify calculation of photometric outliers in uzy bands}\label{lvv-t1758}

\begin{longtable}[]{llllll}
\toprule
Version & Status & Priority & Verification Type & Owner
\\\midrule
1 & Approved & Normal &
Test & Jeffrey Carlin
\\\bottomrule
\multicolumn{6}{c}{ Open \href{https://jira.lsstcorp.org/secure/Tests.jspa\#/testCase/LVV-T1758}{LVV-T1758} in Jira } \\
\end{longtable}

\subsubsection{Verification Elements}
\begin{itemize}
\item \href{https://jira.lsstcorp.org/browse/LVV-9758}{LVV-9758} - DMS-REQ-0359-V-09: Repeatability outlier limit in uzy

\item \href{https://jira.lsstcorp.org/browse/LVV-9752}{LVV-9752} - DMS-REQ-0359-V-03: Max fraction of outliers among non-saturated sources

\end{itemize}

\subsubsection{Test Items}
Verify that the DM system has provided the code to calculate the
photometric repeatability in the u, z, and y filters, and assess whether
it meets the requirement that no more than \textbf{PF1 =
10{[}percent{]}} of the repeatability outliers exceed the outlier limit
of \textbf{PA2uzy = 22.5 millimagnitudes}.~


\subsubsection{Predecessors}

\subsubsection{Environment Needs}

\paragraph{Software}

\paragraph{Hardware}

\subsubsection{Input Specification}

\subsubsection{Output Specification}

\subsubsection{Test Procedure}
    \begin{longtable}[]{p{1.3cm}p{2cm}p{13cm}}
    %\toprule
    Step & \multicolumn{2}{@{}l}{Description, Input Data and Expected Result} \\ \toprule
    \endhead

            \multirow{3}{*}{ 1 } & Description &
            \begin{minipage}[t]{13cm}{\footnotesize
            Identify a dataset containing at least one field in each of the u, z,
and y filters with multiple overlapping visits.

            \vspace{\dp0}
            } \end{minipage} \\ \cline{2-3}
            & Test Data &
            \begin{minipage}[t]{13cm}{\footnotesize
                No data.
                \vspace{\dp0}
            } \end{minipage} \\ \cline{2-3}
            & Expected Result &
                \begin{minipage}[t]{13cm}{\footnotesize
                A dataset that has been ingested into a Butler repository.

                \vspace{\dp0}
                } \end{minipage}
        \\ \midrule

                \multirow{3}{*}{\parbox{1.3cm}{ 2-1
                {\scriptsize from \hyperref[lvv-t860]
                {LVV-T860} } } }

                & {\small Description} &
                \begin{minipage}[t]{13cm}{\scriptsize
                The `path` that you will use depends on where you are running the
science pipelines. Options:\\[2\baselineskip]

\begin{itemize}
\tightlist
\item
  local (newinstall.sh - based
  install):{[}path\_to\_installation{]}/loadLSST.bash
\item
  development cluster (``lsst-dev''):
  /software/lsstsw/stack/loadLSST.bash
\item
  LSP Notebook aspect (from a terminal):
  /opt/lsst/software/stack/loadLSST.bash
\end{itemize}

From the command line, execute the commands below in the example
code:\\[2\baselineskip]

                \vspace{\dp0}
                } \end{minipage} \\ \cdashline{2-3}
                & {\small Test Data} &
                \begin{minipage}[t]{13cm}{\scriptsize
                } \end{minipage} \\ \cdashline{2-3}
                & {\small Expected Result} &
                    \begin{minipage}[t]{13cm}{\scriptsize
                    Science pipeline software is available for use. If additional packages
are needed (for example, `obs' packages such as `obs\_subaru`), then
additional `setup` commands will be necessary.\\[2\baselineskip]To check
versions in use, type:\\
eups list -s

                    \vspace{\dp0}
                    } \end{minipage}
                \\ \hdashline


        \\ \midrule

                \multirow{3}{*}{\parbox{1.3cm}{ 3-1
                {\scriptsize from \hyperref[lvv-t1744]
                {LVV-T1744} } } }

                & {\small Description} &
                \begin{minipage}[t]{13cm}{\scriptsize
                Execute `validate\_drp` on a repository containing precursor data.
Identify the path to the data, which we will call `DATA/path', then
execute the following (with additional flags specified as needed):

                \vspace{\dp0}
                } \end{minipage} \\ \cdashline{2-3}
                & {\small Test Data} &
                \begin{minipage}[t]{13cm}{\scriptsize
                } \end{minipage} \\ \cdashline{2-3}
                & {\small Expected Result} &
                    \begin{minipage}[t]{13cm}{\scriptsize
                    JSON files (and associated figures) containing the Measurements and any
associated ``extras.''

                    \vspace{\dp0}
                    } \end{minipage}
                \\ \hdashline


        \\ \midrule

            \multirow{3}{*}{ 4 } & Description &
            \begin{minipage}[t]{13cm}{\footnotesize
            Confirm that the metric PA2uzy has been calculated using the threshold
PF1, and that its values are reasonable.

            \vspace{\dp0}
            } \end{minipage} \\ \cline{2-3}
            & Test Data &
            \begin{minipage}[t]{13cm}{\footnotesize
                No data.
                \vspace{\dp0}
            } \end{minipage} \\ \cline{2-3}
            & Expected Result &
                \begin{minipage}[t]{13cm}{\footnotesize
                A JSON file (and/or a report generated from that JSON file)
demonstrating that PA2uzy has been calculated (and that it used PF1).

                \vspace{\dp0}
                } \end{minipage}
        \\ \midrule
    \end{longtable}

\subsection{LVV-T1759 - Verify calculation of photometric outliers in gri bands}\label{lvv-t1759}

\begin{longtable}[]{llllll}
\toprule
Version & Status & Priority & Verification Type & Owner
\\\midrule
1 & Approved & Normal &
Test & Jeffrey Carlin
\\\bottomrule
\multicolumn{6}{c}{ Open \href{https://jira.lsstcorp.org/secure/Tests.jspa\#/testCase/LVV-T1759}{LVV-T1759} in Jira } \\
\end{longtable}

\subsubsection{Verification Elements}
\begin{itemize}
\item \href{https://jira.lsstcorp.org/browse/LVV-9752}{LVV-9752} - DMS-REQ-0359-V-03: Max fraction of outliers among non-saturated sources

\item \href{https://jira.lsstcorp.org/browse/LVV-9754}{LVV-9754} - DMS-REQ-0359-V-05: Repeatability outlier limit in gri

\end{itemize}

\subsubsection{Test Items}
Verify that the DM system has provided the code to calculate the
photometric repeatability in the g, r, and i filters, and assess whether
it meets the requirement that no more than \textbf{PF1 =
10{[}percent{]}} of the repeatability outliers exceed the outlier limit
of \textbf{PA2gri = 15 millimagnitudes}.


\subsubsection{Predecessors}

\subsubsection{Environment Needs}

\paragraph{Software}

\paragraph{Hardware}

\subsubsection{Input Specification}

\subsubsection{Output Specification}

\subsubsection{Test Procedure}
    \begin{longtable}[]{p{1.3cm}p{2cm}p{13cm}}
    %\toprule
    Step & \multicolumn{2}{@{}l}{Description, Input Data and Expected Result} \\ \toprule
    \endhead

            \multirow{3}{*}{ 1 } & Description &
            \begin{minipage}[t]{13cm}{\footnotesize
            Identify a dataset containing at least one field in each of the g, r,
and i filters with multiple overlapping visits.

            \vspace{\dp0}
            } \end{minipage} \\ \cline{2-3}
            & Test Data &
            \begin{minipage}[t]{13cm}{\footnotesize
                No data.
                \vspace{\dp0}
            } \end{minipage} \\ \cline{2-3}
            & Expected Result &
                \begin{minipage}[t]{13cm}{\footnotesize
                A dataset that has been ingested into a Butler repository.

                \vspace{\dp0}
                } \end{minipage}
        \\ \midrule

                \multirow{3}{*}{\parbox{1.3cm}{ 2-1
                {\scriptsize from \hyperref[lvv-t860]
                {LVV-T860} } } }

                & {\small Description} &
                \begin{minipage}[t]{13cm}{\scriptsize
                The `path` that you will use depends on where you are running the
science pipelines. Options:\\[2\baselineskip]

\begin{itemize}
\tightlist
\item
  local (newinstall.sh - based
  install):{[}path\_to\_installation{]}/loadLSST.bash
\item
  development cluster (``lsst-dev''):
  /software/lsstsw/stack/loadLSST.bash
\item
  LSP Notebook aspect (from a terminal):
  /opt/lsst/software/stack/loadLSST.bash
\end{itemize}

From the command line, execute the commands below in the example
code:\\[2\baselineskip]

                \vspace{\dp0}
                } \end{minipage} \\ \cdashline{2-3}
                & {\small Test Data} &
                \begin{minipage}[t]{13cm}{\scriptsize
                } \end{minipage} \\ \cdashline{2-3}
                & {\small Expected Result} &
                    \begin{minipage}[t]{13cm}{\scriptsize
                    Science pipeline software is available for use. If additional packages
are needed (for example, `obs' packages such as `obs\_subaru`), then
additional `setup` commands will be necessary.\\[2\baselineskip]To check
versions in use, type:\\
eups list -s

                    \vspace{\dp0}
                    } \end{minipage}
                \\ \hdashline


        \\ \midrule

                \multirow{3}{*}{\parbox{1.3cm}{ 3-1
                {\scriptsize from \hyperref[lvv-t1744]
                {LVV-T1744} } } }

                & {\small Description} &
                \begin{minipage}[t]{13cm}{\scriptsize
                Execute `validate\_drp` on a repository containing precursor data.
Identify the path to the data, which we will call `DATA/path', then
execute the following (with additional flags specified as needed):

                \vspace{\dp0}
                } \end{minipage} \\ \cdashline{2-3}
                & {\small Test Data} &
                \begin{minipage}[t]{13cm}{\scriptsize
                } \end{minipage} \\ \cdashline{2-3}
                & {\small Expected Result} &
                    \begin{minipage}[t]{13cm}{\scriptsize
                    JSON files (and associated figures) containing the Measurements and any
associated ``extras.''

                    \vspace{\dp0}
                    } \end{minipage}
                \\ \hdashline


        \\ \midrule

            \multirow{3}{*}{ 4 } & Description &
            \begin{minipage}[t]{13cm}{\footnotesize
            Confirm that the metric PA2gri has been calculated using the threshold
PF1, and that its values are reasonable.

            \vspace{\dp0}
            } \end{minipage} \\ \cline{2-3}
            & Test Data &
            \begin{minipage}[t]{13cm}{\footnotesize
                No data.
                \vspace{\dp0}
            } \end{minipage} \\ \cline{2-3}
            & Expected Result &
                \begin{minipage}[t]{13cm}{\footnotesize
                A JSON file (and/or a report generated from that JSON file)
demonstrating that PA2gri has been calculated (and that it used PF1).

                \vspace{\dp0}
                } \end{minipage}
        \\ \midrule
    \end{longtable}

\subsection{LVV-T1830 - Verify Implementation of Scientific Visualization of Camera Image Data}\label{lvv-t1830}

\begin{longtable}[]{llllll}
\toprule
Version & Status & Priority & Verification Type & Owner
\\\midrule
1 & Draft & Normal &
Inspection & Jeffrey Carlin
\\\bottomrule
\multicolumn{6}{c}{ Open \href{https://jira.lsstcorp.org/secure/Tests.jspa\#/testCase/LVV-T1830}{LVV-T1830} in Jira } \\
\end{longtable}

\subsubsection{Verification Elements}
\begin{itemize}
\item \href{https://jira.lsstcorp.org/browse/LVV-18465}{LVV-18465} - DMS-REQ-0395-V-01: Scientific Visualization of Camera Image Data\_1

\end{itemize}

\subsubsection{Test Items}
Verify that all scientific visualization of camera image data uses the
coordinate systems defined in \href{https://lse-349.lsst.io/}{LSE-349}.


\subsubsection{Predecessors}

\subsubsection{Environment Needs}

\paragraph{Software}

\paragraph{Hardware}

\subsubsection{Input Specification}

\subsubsection{Output Specification}

\subsubsection{Test Procedure}
    \begin{longtable}[]{p{1.3cm}p{2cm}p{13cm}}
    %\toprule
    Step & \multicolumn{2}{@{}l}{Description, Input Data and Expected Result} \\ \toprule
    \endhead

            \multirow{3}{*}{ 1 } & Description &
            \begin{minipage}[t]{13cm}{\footnotesize
            
            \vspace{\dp0}
            } \end{minipage} \\ \cline{2-3}
            & Test Data &
            \begin{minipage}[t]{13cm}{\footnotesize
                No data.
                \vspace{\dp0}
            } \end{minipage} \\ \cline{2-3}
            & Expected Result &
        \\ \midrule
    \end{longtable}

\subsection{LVV-T1831 - Verify Implementation of Data Management Nightly Reporting}\label{lvv-t1831}

\begin{longtable}[]{llllll}
\toprule
Version & Status & Priority & Verification Type & Owner
\\\midrule
1 & Draft & Normal &
Demonstration & Jeffrey Carlin
\\\bottomrule
\multicolumn{6}{c}{ Open \href{https://jira.lsstcorp.org/secure/Tests.jspa\#/testCase/LVV-T1831}{LVV-T1831} in Jira } \\
\end{longtable}

\subsubsection{Verification Elements}
\begin{itemize}
\item \href{https://jira.lsstcorp.org/browse/LVV-18295}{LVV-18295} - DMS-REQ-0394-V-01: Data Management Nightly Reporting\_1

\end{itemize}

\subsubsection{Test Items}
Verify that the LSST Data Management subsystem produces a searchable -
interactive nightly report(s), from information published in the EFD by
each subsystem, summarizing performance and behavior over a user defined
period of time (e.g. the previous 24 hours).


\subsubsection{Predecessors}

\subsubsection{Environment Needs}

\paragraph{Software}

\paragraph{Hardware}

\subsubsection{Input Specification}

\subsubsection{Output Specification}

\subsubsection{Test Procedure}
    \begin{longtable}[]{p{1.3cm}p{2cm}p{13cm}}
    %\toprule
    Step & \multicolumn{2}{@{}l}{Description, Input Data and Expected Result} \\ \toprule
    \endhead

            \multirow{3}{*}{ 1 } & Description &
            \begin{minipage}[t]{13cm}{\footnotesize
            
            \vspace{\dp0}
            } \end{minipage} \\ \cline{2-3}
            & Test Data &
            \begin{minipage}[t]{13cm}{\footnotesize
                No data.
                \vspace{\dp0}
            } \end{minipage} \\ \cline{2-3}
            & Expected Result &
        \\ \midrule
    \end{longtable}

\subsection{LVV-T1836 - Verify calculation of resolved-to-unresolved flux ratio errors}\label{lvv-t1836}

\begin{longtable}[]{llllll}
\toprule
Version & Status & Priority & Verification Type & Owner
\\\midrule
1 & Draft & Normal &
Test & Jeffrey Carlin
\\\bottomrule
\multicolumn{6}{c}{ Open \href{https://jira.lsstcorp.org/secure/Tests.jspa\#/testCase/LVV-T1836}{LVV-T1836} in Jira } \\
\end{longtable}

\subsubsection{Verification Elements}
\begin{itemize}
\item \href{https://jira.lsstcorp.org/browse/LVV-9766}{LVV-9766} - DMS-REQ-0359-V-17: Max RMS of resolved/unresolved flux ratio

\end{itemize}

\subsubsection{Test Items}
Verify that the DM system has provided code to assess whether the
maximum RMS of the ratio of the error in integrated flux measurement
between bright, isolated, resolved sources less than 10 arcsec in
diameter and bright, isolated unresolved point sources is less than
\textbf{ResSource = 2}.


\subsubsection{Predecessors}

\subsubsection{Environment Needs}

\paragraph{Software}

\paragraph{Hardware}

\subsubsection{Input Specification}

\subsubsection{Output Specification}

\subsubsection{Test Procedure}
    \begin{longtable}[]{p{1.3cm}p{2cm}p{13cm}}
    %\toprule
    Step & \multicolumn{2}{@{}l}{Description, Input Data and Expected Result} \\ \toprule
    \endhead

            \multirow{3}{*}{ 1 } & Description &
            \begin{minipage}[t]{13cm}{\footnotesize
            
            \vspace{\dp0}
            } \end{minipage} \\ \cline{2-3}
            & Test Data &
            \begin{minipage}[t]{13cm}{\footnotesize
                No data.
                \vspace{\dp0}
            } \end{minipage} \\ \cline{2-3}
            & Expected Result &
        \\ \midrule
    \end{longtable}

\subsection{LVV-T1837 - Verify calculation of band-to-band color zero-point accuracy}\label{lvv-t1837}

\begin{longtable}[]{llllll}
\toprule
Version & Status & Priority & Verification Type & Owner
\\\midrule
1 & Draft & Normal &
Test & Jeffrey Carlin
\\\bottomrule
\multicolumn{6}{c}{ Open \href{https://jira.lsstcorp.org/secure/Tests.jspa\#/testCase/LVV-T1837}{LVV-T1837} in Jira } \\
\end{longtable}

\subsubsection{Verification Elements}
\begin{itemize}
\item \href{https://jira.lsstcorp.org/browse/LVV-9765}{LVV-9765} - DMS-REQ-0359-V-16: Accuracy of zero point for colors without u-band

\end{itemize}

\subsubsection{Test Items}
Verify that the DM system provides code to assess whether the accuracy
of absolute band-to-band color zero-points for all colors constructed
from any filter pair, excluding the u-band, is less than \textbf{PA5 = 5
millimagnitudes}.


\subsubsection{Predecessors}

\subsubsection{Environment Needs}

\paragraph{Software}

\paragraph{Hardware}

\subsubsection{Input Specification}

\subsubsection{Output Specification}

\subsubsection{Test Procedure}
    \begin{longtable}[]{p{1.3cm}p{2cm}p{13cm}}
    %\toprule
    Step & \multicolumn{2}{@{}l}{Description, Input Data and Expected Result} \\ \toprule
    \endhead

            \multirow{3}{*}{ 1 } & Description &
            \begin{minipage}[t]{13cm}{\footnotesize
            
            \vspace{\dp0}
            } \end{minipage} \\ \cline{2-3}
            & Test Data &
            \begin{minipage}[t]{13cm}{\footnotesize
                No data.
                \vspace{\dp0}
            } \end{minipage} \\ \cline{2-3}
            & Expected Result &
        \\ \midrule
    \end{longtable}

\subsection{LVV-T1838 - Verify calculation of image fraction affected by ghosts}\label{lvv-t1838}

\begin{longtable}[]{llllll}
\toprule
Version & Status & Priority & Verification Type & Owner
\\\midrule
1 & Draft & Normal &
Test & Jeffrey Carlin
\\\bottomrule
\multicolumn{6}{c}{ Open \href{https://jira.lsstcorp.org/secure/Tests.jspa\#/testCase/LVV-T1838}{LVV-T1838} in Jira } \\
\end{longtable}

\subsubsection{Verification Elements}
\begin{itemize}
\item \href{https://jira.lsstcorp.org/browse/LVV-9764}{LVV-9764} - DMS-REQ-0359-V-15: Percentage of image area with ghosts

\end{itemize}

\subsubsection{Test Items}
Verify that the DM system provides code to assess whether the percentage
of image area that has ghosts with surface brightness gradient amplitude
of more than 1/3 of the sky noise over 1 arcsec is less than
\textbf{GhostAF = 1 percent}.


\subsubsection{Predecessors}

\subsubsection{Environment Needs}

\paragraph{Software}

\paragraph{Hardware}

\subsubsection{Input Specification}

\subsubsection{Output Specification}

\subsubsection{Test Procedure}
    \begin{longtable}[]{p{1.3cm}p{2cm}p{13cm}}
    %\toprule
    Step & \multicolumn{2}{@{}l}{Description, Input Data and Expected Result} \\ \toprule
    \endhead

            \multirow{3}{*}{ 1 } & Description &
            \begin{minipage}[t]{13cm}{\footnotesize
            
            \vspace{\dp0}
            } \end{minipage} \\ \cline{2-3}
            & Test Data &
            \begin{minipage}[t]{13cm}{\footnotesize
                No data.
                \vspace{\dp0}
            } \end{minipage} \\ \cline{2-3}
            & Expected Result &
        \\ \midrule
    \end{longtable}

\subsection{LVV-T1839 - Verify calculation of RMS width of photometric zeropoint}\label{lvv-t1839}

\begin{longtable}[]{llllll}
\toprule
Version & Status & Priority & Verification Type & Owner
\\\midrule
1 & Draft & Normal &
Test & Jeffrey Carlin
\\\bottomrule
\multicolumn{6}{c}{ Open \href{https://jira.lsstcorp.org/secure/Tests.jspa\#/testCase/LVV-T1839}{LVV-T1839} in Jira } \\
\end{longtable}

\subsubsection{Verification Elements}
\begin{itemize}
\item \href{https://jira.lsstcorp.org/browse/LVV-9763}{LVV-9763} - DMS-REQ-0359-V-14: RMS width of zero point in all bands except u

\end{itemize}

\subsubsection{Test Items}
Verify that the DM system provides code to assess whether the RMS width
of the internal photometric zero-point (precision of system uniformity
across the sky) for all bands except u-band is less than \textbf{PA3 =
10 millimagnitudes}.


\subsubsection{Predecessors}

\subsubsection{Environment Needs}

\paragraph{Software}

\paragraph{Hardware}

\subsubsection{Input Specification}

\subsubsection{Output Specification}

\subsubsection{Test Procedure}
    \begin{longtable}[]{p{1.3cm}p{2cm}p{13cm}}
    %\toprule
    Step & \multicolumn{2}{@{}l}{Description, Input Data and Expected Result} \\ \toprule
    \endhead

            \multirow{3}{*}{ 1 } & Description &
            \begin{minipage}[t]{13cm}{\footnotesize
            
            \vspace{\dp0}
            } \end{minipage} \\ \cline{2-3}
            & Test Data &
            \begin{minipage}[t]{13cm}{\footnotesize
                No data.
                \vspace{\dp0}
            } \end{minipage} \\ \cline{2-3}
            & Expected Result &
        \\ \midrule
    \end{longtable}

\subsection{LVV-T1840 - Verify calculation of sky brightness precision}\label{lvv-t1840}

\begin{longtable}[]{llllll}
\toprule
Version & Status & Priority & Verification Type & Owner
\\\midrule
1 & Draft & Normal &
Test & Jeffrey Carlin
\\\bottomrule
\multicolumn{6}{c}{ Open \href{https://jira.lsstcorp.org/secure/Tests.jspa\#/testCase/LVV-T1840}{LVV-T1840} in Jira } \\
\end{longtable}

\subsubsection{Verification Elements}
\begin{itemize}
\item \href{https://jira.lsstcorp.org/browse/LVV-9762}{LVV-9762} - DMS-REQ-0359-V-13: Max sky brightness error

\end{itemize}

\subsubsection{Test Items}
Verify that the DM system provides software to assess whether the
maximum error in the precision of the sky brightness determination is
less than \textbf{SBPrec = 1 percent.}


\subsubsection{Predecessors}

\subsubsection{Environment Needs}

\paragraph{Software}

\paragraph{Hardware}

\subsubsection{Input Specification}

\subsubsection{Output Specification}

\subsubsection{Test Procedure}
    \begin{longtable}[]{p{1.3cm}p{2cm}p{13cm}}
    %\toprule
    Step & \multicolumn{2}{@{}l}{Description, Input Data and Expected Result} \\ \toprule
    \endhead

            \multirow{3}{*}{ 1 } & Description &
            \begin{minipage}[t]{13cm}{\footnotesize
            
            \vspace{\dp0}
            } \end{minipage} \\ \cline{2-3}
            & Test Data &
            \begin{minipage}[t]{13cm}{\footnotesize
                No data.
                \vspace{\dp0}
            } \end{minipage} \\ \cline{2-3}
            & Expected Result &
        \\ \midrule
    \end{longtable}

\subsection{LVV-T1841 - Verify calculation of scientifically unusable pixel fraction}\label{lvv-t1841}

\begin{longtable}[]{llllll}
\toprule
Version & Status & Priority & Verification Type & Owner
\\\midrule
1 & Draft & Normal &
Test & Jeffrey Carlin
\\\bottomrule
\multicolumn{6}{c}{ Open \href{https://jira.lsstcorp.org/secure/Tests.jspa\#/testCase/LVV-T1841}{LVV-T1841} in Jira } \\
\end{longtable}

\subsubsection{Verification Elements}
\begin{itemize}
\item \href{https://jira.lsstcorp.org/browse/LVV-9761}{LVV-9761} - DMS-REQ-0359-V-12: Max fraction of unusable pixels per sensor

\end{itemize}

\subsubsection{Test Items}
Verify that the DM system provides software to assess whether the
maximum fraction of pixels scientifically unusable per sensor out of the
total allowable fraction of sensors meeting this performance is less
than~\textbf{PixFrac = 1 percent}.


\subsubsection{Predecessors}

\subsubsection{Environment Needs}

\paragraph{Software}

\paragraph{Hardware}

\subsubsection{Input Specification}

\subsubsection{Output Specification}

\subsubsection{Test Procedure}
    \begin{longtable}[]{p{1.3cm}p{2cm}p{13cm}}
    %\toprule
    Step & \multicolumn{2}{@{}l}{Description, Input Data and Expected Result} \\ \toprule
    \endhead

            \multirow{3}{*}{ 1 } & Description &
            \begin{minipage}[t]{13cm}{\footnotesize
            
            \vspace{\dp0}
            } \end{minipage} \\ \cline{2-3}
            & Test Data &
            \begin{minipage}[t]{13cm}{\footnotesize
                No data.
                \vspace{\dp0}
            } \end{minipage} \\ \cline{2-3}
            & Expected Result &
        \\ \midrule
    \end{longtable}

\subsection{LVV-T1842 - Verify calculation of zeropoint error fraction exceeding the outlier
limit}\label{lvv-t1842}

\begin{longtable}[]{llllll}
\toprule
Version & Status & Priority & Verification Type & Owner
\\\midrule
1 & Draft & Normal &
Test & Jeffrey Carlin
\\\bottomrule
\multicolumn{6}{c}{ Open \href{https://jira.lsstcorp.org/secure/Tests.jspa\#/testCase/LVV-T1842}{LVV-T1842} in Jira } \\
\end{longtable}

\subsubsection{Verification Elements}
\begin{itemize}
\item \href{https://jira.lsstcorp.org/browse/LVV-9760}{LVV-9760} - DMS-REQ-0359-V-11: Fraction of zero point outliers

\end{itemize}

\subsubsection{Test Items}
Verify that the DM system provides software to calculate the fraction of
zeropoint errors that exceed the zero point error outlier limit, and
confirm that it is less than \textbf{PF2 = 10 percent.}


\subsubsection{Predecessors}

\subsubsection{Environment Needs}

\paragraph{Software}

\paragraph{Hardware}

\subsubsection{Input Specification}

\subsubsection{Output Specification}

\subsubsection{Test Procedure}
    \begin{longtable}[]{p{1.3cm}p{2cm}p{13cm}}
    %\toprule
    Step & \multicolumn{2}{@{}l}{Description, Input Data and Expected Result} \\ \toprule
    \endhead

            \multirow{3}{*}{ 1 } & Description &
            \begin{minipage}[t]{13cm}{\footnotesize
            
            \vspace{\dp0}
            } \end{minipage} \\ \cline{2-3}
            & Test Data &
            \begin{minipage}[t]{13cm}{\footnotesize
                No data.
                \vspace{\dp0}
            } \end{minipage} \\ \cline{2-3}
            & Expected Result &
        \\ \midrule
    \end{longtable}

\subsection{LVV-T1843 - Verify calculation of significance of imperfect crosstalk corrections}\label{lvv-t1843}

\begin{longtable}[]{llllll}
\toprule
Version & Status & Priority & Verification Type & Owner
\\\midrule
1 & Draft & Normal &
Test & Jeffrey Carlin
\\\bottomrule
\multicolumn{6}{c}{ Open \href{https://jira.lsstcorp.org/secure/Tests.jspa\#/testCase/LVV-T1843}{LVV-T1843} in Jira } \\
\end{longtable}

\subsubsection{Verification Elements}
\begin{itemize}
\item \href{https://jira.lsstcorp.org/browse/LVV-9757}{LVV-9757} - DMS-REQ-0359-V-08: Max cross-talk imperfections

\end{itemize}

\subsubsection{Test Items}
Verify that the DM system provides software to assess whether the
maximum local significance integrated over the PSF of imperfect
crosstalk corrections is less than \textbf{Xtalk = 3 sigma}.


\subsubsection{Predecessors}

\subsubsection{Environment Needs}

\paragraph{Software}

\paragraph{Hardware}

\subsubsection{Input Specification}

\subsubsection{Output Specification}

\subsubsection{Test Procedure}
    \begin{longtable}[]{p{1.3cm}p{2cm}p{13cm}}
    %\toprule
    Step & \multicolumn{2}{@{}l}{Description, Input Data and Expected Result} \\ \toprule
    \endhead

            \multirow{3}{*}{ 1 } & Description &
            \begin{minipage}[t]{13cm}{\footnotesize
            
            \vspace{\dp0}
            } \end{minipage} \\ \cline{2-3}
            & Test Data &
            \begin{minipage}[t]{13cm}{\footnotesize
                No data.
                \vspace{\dp0}
            } \end{minipage} \\ \cline{2-3}
            & Expected Result &
        \\ \midrule
    \end{longtable}

\subsection{LVV-T1844 - Verify calculation of u-band photometric zero-point RMS}\label{lvv-t1844}

\begin{longtable}[]{llllll}
\toprule
Version & Status & Priority & Verification Type & Owner
\\\midrule
1 & Draft & Normal &
Test & Jeffrey Carlin
\\\bottomrule
\multicolumn{6}{c}{ Open \href{https://jira.lsstcorp.org/secure/Tests.jspa\#/testCase/LVV-T1844}{LVV-T1844} in Jira } \\
\end{longtable}

\subsubsection{Verification Elements}
\begin{itemize}
\item \href{https://jira.lsstcorp.org/browse/LVV-9756}{LVV-9756} - DMS-REQ-0359-V-07: RMS width of zero point in u-band

\end{itemize}

\subsubsection{Test Items}
Verify that the DM system provides software to assess whether the RMS
width of internal photometric zero-point (precision of system uniformity
across the sky) in the u-band is less than \textbf{PA3u = 20
millimagnitudes}.


\subsubsection{Predecessors}

\subsubsection{Environment Needs}

\paragraph{Software}

\paragraph{Hardware}

\subsubsection{Input Specification}

\subsubsection{Output Specification}

\subsubsection{Test Procedure}
    \begin{longtable}[]{p{1.3cm}p{2cm}p{13cm}}
    %\toprule
    Step & \multicolumn{2}{@{}l}{Description, Input Data and Expected Result} \\ \toprule
    \endhead

            \multirow{3}{*}{ 1 } & Description &
            \begin{minipage}[t]{13cm}{\footnotesize
            
            \vspace{\dp0}
            } \end{minipage} \\ \cline{2-3}
            & Test Data &
            \begin{minipage}[t]{13cm}{\footnotesize
                No data.
                \vspace{\dp0}
            } \end{minipage} \\ \cline{2-3}
            & Expected Result &
        \\ \midrule
    \end{longtable}

\subsection{LVV-T1845 - Verify accuracy of photometric transformation to physical scale}\label{lvv-t1845}

\begin{longtable}[]{llllll}
\toprule
Version & Status & Priority & Verification Type & Owner
\\\midrule
1 & Draft & Normal &
Test & Jeffrey Carlin
\\\bottomrule
\multicolumn{6}{c}{ Open \href{https://jira.lsstcorp.org/secure/Tests.jspa\#/testCase/LVV-T1845}{LVV-T1845} in Jira } \\
\end{longtable}

\subsubsection{Verification Elements}
\begin{itemize}
\item \href{https://jira.lsstcorp.org/browse/LVV-9755}{LVV-9755} - DMS-REQ-0359-V-06: Accuracy of photometric transformation

\end{itemize}

\subsubsection{Test Items}
Verify that the DM system provides software to assess whether the
accuracy of the transformation of internal LSST photometry to a physical
scale (e.g. AB magnitudes) is less than \textbf{PA6 = 10
millimagnitudes}.


\subsubsection{Predecessors}

\subsubsection{Environment Needs}

\paragraph{Software}

\paragraph{Hardware}

\subsubsection{Input Specification}

\subsubsection{Output Specification}

\subsubsection{Test Procedure}
    \begin{longtable}[]{p{1.3cm}p{2cm}p{13cm}}
    %\toprule
    Step & \multicolumn{2}{@{}l}{Description, Input Data and Expected Result} \\ \toprule
    \endhead

            \multirow{3}{*}{ 1 } & Description &
            \begin{minipage}[t]{13cm}{\footnotesize
            
            \vspace{\dp0}
            } \end{minipage} \\ \cline{2-3}
            & Test Data &
            \begin{minipage}[t]{13cm}{\footnotesize
                No data.
                \vspace{\dp0}
            } \end{minipage} \\ \cline{2-3}
            & Expected Result &
        \\ \midrule
    \end{longtable}

\subsection{LVV-T1846 - Verify calculation of band-to-band color zero-point accuracy including
u-band}\label{lvv-t1846}

\begin{longtable}[]{llllll}
\toprule
Version & Status & Priority & Verification Type & Owner
\\\midrule
1 & Draft & Normal &
Test & Jeffrey Carlin
\\\bottomrule
\multicolumn{6}{c}{ Open \href{https://jira.lsstcorp.org/secure/Tests.jspa\#/testCase/LVV-T1846}{LVV-T1846} in Jira } \\
\end{longtable}

\subsubsection{Verification Elements}
\begin{itemize}
\item \href{https://jira.lsstcorp.org/browse/LVV-9753}{LVV-9753} - DMS-REQ-0359-V-04: Accuracy of zero point for colors with u-band

\end{itemize}

\subsubsection{Test Items}
Verify that the DM system provides software to assess whether the
accuracy of absolute band-to-band color zero-points for all colors
constructed from any filter pair, including the u-band, is less than
\textbf{PA5u = 10 millimagnitudes}.


\subsubsection{Predecessors}

\subsubsection{Environment Needs}

\paragraph{Software}

\paragraph{Hardware}

\subsubsection{Input Specification}

\subsubsection{Output Specification}

\subsubsection{Test Procedure}
    \begin{longtable}[]{p{1.3cm}p{2cm}p{13cm}}
    %\toprule
    Step & \multicolumn{2}{@{}l}{Description, Input Data and Expected Result} \\ \toprule
    \endhead

            \multirow{3}{*}{ 1 } & Description &
            \begin{minipage}[t]{13cm}{\footnotesize
            
            \vspace{\dp0}
            } \end{minipage} \\ \cline{2-3}
            & Test Data &
            \begin{minipage}[t]{13cm}{\footnotesize
                No data.
                \vspace{\dp0}
            } \end{minipage} \\ \cline{2-3}
            & Expected Result &
        \\ \midrule
    \end{longtable}

\subsection{LVV-T1847 - Verify calculation of sensor fraction with unusable pixels}\label{lvv-t1847}

\begin{longtable}[]{llllll}
\toprule
Version & Status & Priority & Verification Type & Owner
\\\midrule
1 & Draft & Normal &
Test & Jeffrey Carlin
\\\bottomrule
\multicolumn{6}{c}{ Open \href{https://jira.lsstcorp.org/secure/Tests.jspa\#/testCase/LVV-T1847}{LVV-T1847} in Jira } \\
\end{longtable}

\subsubsection{Verification Elements}
\begin{itemize}
\item \href{https://jira.lsstcorp.org/browse/LVV-9751}{LVV-9751} - DMS-REQ-0359-V-02: Max fraction of sensors with excess unusable pixels

\end{itemize}

\subsubsection{Test Items}
Verify that the DM system provides software to assess whether the
maximum allowable fraction of sensors with \textbf{PixFrac
\textgreater{} 1} percent scientifically unusable pixels is less
than~\textbf{SensorFraction = 15 percent.}


\subsubsection{Predecessors}

\subsubsection{Environment Needs}

\paragraph{Software}

\paragraph{Hardware}

\subsubsection{Input Specification}

\subsubsection{Output Specification}

\subsubsection{Test Procedure}
    \begin{longtable}[]{p{1.3cm}p{2cm}p{13cm}}
    %\toprule
    Step & \multicolumn{2}{@{}l}{Description, Input Data and Expected Result} \\ \toprule
    \endhead

            \multirow{3}{*}{ 1 } & Description &
            \begin{minipage}[t]{13cm}{\footnotesize
            
            \vspace{\dp0}
            } \end{minipage} \\ \cline{2-3}
            & Test Data &
            \begin{minipage}[t]{13cm}{\footnotesize
                No data.
                \vspace{\dp0}
            } \end{minipage} \\ \cline{2-3}
            & Expected Result &
        \\ \midrule
    \end{longtable}

\subsection{LVV-T1862 - Verify determining effectiveness of dark current frame}\label{lvv-t1862}

\begin{longtable}[]{llllll}
\toprule
Version & Status & Priority & Verification Type & Owner
\\\midrule
1 & Draft & Normal &
Test & Jeffrey Carlin
\\\bottomrule
\multicolumn{6}{c}{ Open \href{https://jira.lsstcorp.org/secure/Tests.jspa\#/testCase/LVV-T1862}{LVV-T1862} in Jira } \\
\end{longtable}

\subsubsection{Verification Elements}
\begin{itemize}
\item \href{https://jira.lsstcorp.org/browse/LVV-18881}{LVV-18881} - DMS-REQ-0282-V-02: Dark Current Correction Frame Effectiveness

\end{itemize}

\subsubsection{Test Items}
Verify that the DMS can determine the effectiveness of a dark correction
and determine how often it should be updated.


\subsubsection{Predecessors}
Execution of
\href{https://jira.lsstcorp.org/secure/Tests.jspa\#/testCase/LVV-T90}{LVV-T90}.

\subsubsection{Environment Needs}

\paragraph{Software}

\paragraph{Hardware}

\subsubsection{Input Specification}

\subsubsection{Output Specification}

\subsubsection{Test Procedure}
    \begin{longtable}[]{p{1.3cm}p{2cm}p{13cm}}
    %\toprule
    Step & \multicolumn{2}{@{}l}{Description, Input Data and Expected Result} \\ \toprule
    \endhead

            \multirow{3}{*}{ 1 } & Description &
            \begin{minipage}[t]{13cm}{\footnotesize
            Identify the path to a dataset containing dark frames (i.e., exposures
taken with the shutter closed).

            \vspace{\dp0}
            } \end{minipage} \\ \cline{2-3}
            & Test Data &
            \begin{minipage}[t]{13cm}{\footnotesize
                No data.
                \vspace{\dp0}
            } \end{minipage} \\ \cline{2-3}
            & Expected Result &
        \\ \midrule

                \multirow{3}{*}{\parbox{1.3cm}{ 2-1
                {\scriptsize from \hyperref[lvv-t1060]
                {LVV-T1060} } } }

                & {\small Description} &
                \begin{minipage}[t]{13cm}{\scriptsize
                Execute the Calibration Products Production payload. The payload uses
raw calibration images and information from the Transformed EFD to
generate a subset of Master Calibration Images and Calibration Database
entries in the Data Backbone.

                \vspace{\dp0}
                } \end{minipage} \\ \cdashline{2-3}
                & {\small Test Data} &
                \begin{minipage}[t]{13cm}{\scriptsize
                } \end{minipage} \\ \cdashline{2-3}
                & {\small Expected Result} &
                \\ \hdashline


                \multirow{3}{*}{\parbox{1.3cm}{ 2-2
                {\scriptsize from \hyperref[lvv-t1060]
                {LVV-T1060} } } }

                & {\small Description} &
                \begin{minipage}[t]{13cm}{\scriptsize
                Confirm that the expected Master Calibration images and Calibration
Database entries are present and well-formed.

                \vspace{\dp0}
                } \end{minipage} \\ \cdashline{2-3}
                & {\small Test Data} &
                \begin{minipage}[t]{13cm}{\scriptsize
                } \end{minipage} \\ \cdashline{2-3}
                & {\small Expected Result} &
                \\ \hdashline


        \\ \midrule

            \multirow{3}{*}{ 3 } & Description &
            \begin{minipage}[t]{13cm}{\footnotesize
            Determining whether the dark correction is being done properly will
require on-sky science data. The dark correction can be applied to these
frames and the results inspected to ensure that the correction was
correctly measured and applied.

            \vspace{\dp0}
            } \end{minipage} \\ \cline{2-3}
            & Test Data &
            \begin{minipage}[t]{13cm}{\footnotesize
                No data.
                \vspace{\dp0}
            } \end{minipage} \\ \cline{2-3}
            & Expected Result &
                \begin{minipage}[t]{13cm}{\footnotesize
                Applying the dark correction to a dataset produces noticeable
differences between the original frame(s) and the corrected outputs.

                \vspace{\dp0}
                } \end{minipage}
        \\ \midrule
    \end{longtable}

\subsection{LVV-T1863 - Verify ability to process Special Programs data alongside normal
processing}\label{lvv-t1863}

\begin{longtable}[]{llllll}
\toprule
Version & Status & Priority & Verification Type & Owner
\\\midrule
1 & Draft & Normal &
Test & Jeffrey Carlin
\\\bottomrule
\multicolumn{6}{c}{ Open \href{https://jira.lsstcorp.org/secure/Tests.jspa\#/testCase/LVV-T1863}{LVV-T1863} in Jira } \\
\end{longtable}

\subsubsection{Verification Elements}
\begin{itemize}
\item \href{https://jira.lsstcorp.org/browse/LVV-18847}{LVV-18847} - DMS-REQ-0397-V-01: Prompt/DR Processing of Data from Special Programs\_1

\end{itemize}

\subsubsection{Test Items}
Verify that Special Programs data can be processed alongside either
prompt-products or data-release processing with little or no extra
effort by DM staff.


\subsubsection{Predecessors}

\subsubsection{Environment Needs}

\paragraph{Software}

\paragraph{Hardware}

\subsubsection{Input Specification}

\subsubsection{Output Specification}

\subsubsection{Test Procedure}
    \begin{longtable}[]{p{1.3cm}p{2cm}p{13cm}}
    %\toprule
    Step & \multicolumn{2}{@{}l}{Description, Input Data and Expected Result} \\ \toprule
    \endhead

            \multirow{3}{*}{ 1 } & Description &
            \begin{minipage}[t]{13cm}{\footnotesize
            
            \vspace{\dp0}
            } \end{minipage} \\ \cline{2-3}
            & Test Data &
            \begin{minipage}[t]{13cm}{\footnotesize
                No data.
                \vspace{\dp0}
            } \end{minipage} \\ \cline{2-3}
            & Expected Result &
        \\ \midrule
    \end{longtable}

\subsection{LVV-T1865 - Verify implementation of time to L1 public release for Special Programs}\label{lvv-t1865}

\begin{longtable}[]{llllll}
\toprule
Version & Status & Priority & Verification Type & Owner
\\\midrule
1 & Draft & Normal &
Test & Jeffrey Carlin
\\\bottomrule
\multicolumn{6}{c}{ Open \href{https://jira.lsstcorp.org/secure/Tests.jspa\#/testCase/LVV-T1865}{LVV-T1865} in Jira } \\
\end{longtable}

\subsubsection{Verification Elements}
\begin{itemize}
\item \href{https://jira.lsstcorp.org/browse/LVV-18229}{LVV-18229} - DMS-REQ-0344-V-01: Time to L1 public release

\end{itemize}

\subsubsection{Test Items}
~Verify that data from Special Programs are made available via public
release within \textbf{L1PublicT = 24{[}hour{]}} from the acquisition of
science data.


\subsubsection{Predecessors}

\subsubsection{Environment Needs}

\paragraph{Software}

\paragraph{Hardware}

\subsubsection{Input Specification}

\subsubsection{Output Specification}

\subsubsection{Test Procedure}
    \begin{longtable}[]{p{1.3cm}p{2cm}p{13cm}}
    %\toprule
    Step & \multicolumn{2}{@{}l}{Description, Input Data and Expected Result} \\ \toprule
    \endhead

            \multirow{3}{*}{ 1 } & Description &
            \begin{minipage}[t]{13cm}{\footnotesize
            
            \vspace{\dp0}
            } \end{minipage} \\ \cline{2-3}
            & Test Data &
            \begin{minipage}[t]{13cm}{\footnotesize
                No data.
                \vspace{\dp0}
            } \end{minipage} \\ \cline{2-3}
            & Expected Result &
        \\ \midrule
    \end{longtable}

\subsection{LVV-T1866 - Verify latency of reporting optical transients from Special Programs}\label{lvv-t1866}

\begin{longtable}[]{llllll}
\toprule
Version & Status & Priority & Verification Type & Owner
\\\midrule
1 & Draft & Normal &
Test & Jeffrey Carlin
\\\bottomrule
\multicolumn{6}{c}{ Open \href{https://jira.lsstcorp.org/secure/Tests.jspa\#/testCase/LVV-T1866}{LVV-T1866} in Jira } \\
\end{longtable}

\subsubsection{Verification Elements}
\begin{itemize}
\item \href{https://jira.lsstcorp.org/browse/LVV-9744}{LVV-9744} - DMS-REQ-0344-V-02: Latency of reporting optical transients

\end{itemize}

\subsubsection{Test Items}
Verify that optical transients (Level 1 data products) are reported
within OTT1 = 1 minute of last image readout for Special Programs.


\subsubsection{Predecessors}

\subsubsection{Environment Needs}

\paragraph{Software}

\paragraph{Hardware}

\subsubsection{Input Specification}

\subsubsection{Output Specification}

\subsubsection{Test Procedure}
    \begin{longtable}[]{p{1.3cm}p{2cm}p{13cm}}
    %\toprule
    Step & \multicolumn{2}{@{}l}{Description, Input Data and Expected Result} \\ \toprule
    \endhead

            \multirow{3}{*}{ 1 } & Description &
            \begin{minipage}[t]{13cm}{\footnotesize
            
            \vspace{\dp0}
            } \end{minipage} \\ \cline{2-3}
            & Test Data &
            \begin{minipage}[t]{13cm}{\footnotesize
                No data.
                \vspace{\dp0}
            } \end{minipage} \\ \cline{2-3}
            & Expected Result &
        \\ \midrule
    \end{longtable}

\subsection{LVV-T1867 - Verify implementation of at least numStreams alert streams supported}\label{lvv-t1867}

\begin{longtable}[]{llllll}
\toprule
Version & Status & Priority & Verification Type & Owner
\\\midrule
1 & Draft & Normal &
Test & Jeffrey Carlin
\\\bottomrule
\multicolumn{6}{c}{ Open \href{https://jira.lsstcorp.org/secure/Tests.jspa\#/testCase/LVV-T1867}{LVV-T1867} in Jira } \\
\end{longtable}

\subsubsection{Verification Elements}
\begin{itemize}
\item \href{https://jira.lsstcorp.org/browse/LVV-18297}{LVV-18297} - DMS-REQ-0391-V-01: Alert Stream Distribution nStreams

\end{itemize}

\subsubsection{Test Items}
Verify that the LSST system supports the transmission of at least
\textbf{numStreams=5} full alert streams out of the alert distribution
system within \textbf{OTT1=1 minute}.~


\subsubsection{Predecessors}

\subsubsection{Environment Needs}

\paragraph{Software}

\paragraph{Hardware}

\subsubsection{Input Specification}

\subsubsection{Output Specification}

\subsubsection{Test Procedure}
    \begin{longtable}[]{p{1.3cm}p{2cm}p{13cm}}
    %\toprule
    Step & \multicolumn{2}{@{}l}{Description, Input Data and Expected Result} \\ \toprule
    \endhead

            \multirow{3}{*}{ 1 } & Description &
            \begin{minipage}[t]{13cm}{\footnotesize
            
            \vspace{\dp0}
            } \end{minipage} \\ \cline{2-3}
            & Test Data &
            \begin{minipage}[t]{13cm}{\footnotesize
                No data.
                \vspace{\dp0}
            } \end{minipage} \\ \cline{2-3}
            & Expected Result &
        \\ \midrule
    \end{longtable}

\subsection{LVV-T1868 - Verify implementation of alert streams distributed within latency limit}\label{lvv-t1868}

\begin{longtable}[]{llllll}
\toprule
Version & Status & Priority & Verification Type & Owner
\\\midrule
1 & Draft & Normal &
Test & Jeffrey Carlin
\\\bottomrule
\multicolumn{6}{c}{ Open \href{https://jira.lsstcorp.org/secure/Tests.jspa\#/testCase/LVV-T1868}{LVV-T1868} in Jira } \\
\end{longtable}

\subsubsection{Verification Elements}
\begin{itemize}
\item \href{https://jira.lsstcorp.org/browse/LVV-18911}{LVV-18911} - DMS-REQ-0391-V-02: Alert Stream Distribution Latency

\end{itemize}

\subsubsection{Test Items}
Verify that the LSST system supports the transmission of full alert
streams out of the alert distribution system within \textbf{OTT1=1
minute}.


\subsubsection{Predecessors}

\subsubsection{Environment Needs}

\paragraph{Software}

\paragraph{Hardware}

\subsubsection{Input Specification}

\subsubsection{Output Specification}

\subsubsection{Test Procedure}
    \begin{longtable}[]{p{1.3cm}p{2cm}p{13cm}}
    %\toprule
    Step & \multicolumn{2}{@{}l}{Description, Input Data and Expected Result} \\ \toprule
    \endhead

            \multirow{3}{*}{ 1 } & Description &
            \begin{minipage}[t]{13cm}{\footnotesize
            
            \vspace{\dp0}
            } \end{minipage} \\ \cline{2-3}
            & Test Data &
            \begin{minipage}[t]{13cm}{\footnotesize
                No data.
                \vspace{\dp0}
            } \end{minipage} \\ \cline{2-3}
            & Expected Result &
        \\ \midrule
    \end{longtable}

\subsection{LVV-T1946 - Verify implementation of measurements in catalogs from coadds}\label{lvv-t1946}

\begin{longtable}[]{llllll}
\toprule
Version & Status & Priority & Verification Type & Owner
\\\midrule
1 & Approved & Normal &
Test & Jeffrey Carlin
\\\bottomrule
\multicolumn{6}{c}{ Open \href{https://jira.lsstcorp.org/secure/Tests.jspa\#/testCase/LVV-T1946}{LVV-T1946} in Jira } \\
\end{longtable}

\subsubsection{Verification Elements}
\begin{itemize}
\item \href{https://jira.lsstcorp.org/browse/LVV-178}{LVV-178} - DMS-REQ-0347-V-01: Measurements in catalogs

\end{itemize}

\subsubsection{Test Items}
Verify that source measurements in catalogs containing measurements from
coadd images are in flux units.


\subsubsection{Predecessors}

\subsubsection{Environment Needs}

\paragraph{Software}

\paragraph{Hardware}

\subsubsection{Input Specification}

\subsubsection{Output Specification}

\subsubsection{Test Procedure}
    \begin{longtable}[]{p{1.3cm}p{2cm}p{13cm}}
    %\toprule
    Step & \multicolumn{2}{@{}l}{Description, Input Data and Expected Result} \\ \toprule
    \endhead

                \multirow{3}{*}{\parbox{1.3cm}{ 1-1
                {\scriptsize from \hyperref[lvv-t987]
                {LVV-T987} } } }

                & {\small Description} &
                \begin{minipage}[t]{13cm}{\scriptsize
                Identify the path to the data repository, which we will refer to as
`DATA/path', then execute the following:

                \vspace{\dp0}
                } \end{minipage} \\ \cdashline{2-3}
                & {\small Test Data} &
                \begin{minipage}[t]{13cm}{\scriptsize
                } \end{minipage} \\ \cdashline{2-3}
                & {\small Expected Result} &
                    \begin{minipage}[t]{13cm}{\scriptsize
                    Butler repo available for reading.

                    \vspace{\dp0}
                    } \end{minipage}
                \\ \hdashline


        \\ \midrule

            \multirow{3}{*}{ 2 } & Description &
            \begin{minipage}[t]{13cm}{\footnotesize
            Identify and read an appropriate processed precursor dataset containing
coadds with the Butler.

            \vspace{\dp0}
            } \end{minipage} \\ \cline{2-3}
            & Test Data &
            \begin{minipage}[t]{13cm}{\footnotesize
                No data.
                \vspace{\dp0}
            } \end{minipage} \\ \cline{2-3}
            & Expected Result &
        \\ \midrule

            \multirow{3}{*}{ 3 } & Description &
            \begin{minipage}[t]{13cm}{\footnotesize
            Verify that the coadd catalog provides measurements in flux units.

            \vspace{\dp0}
            } \end{minipage} \\ \cline{2-3}
            & Test Data &
            \begin{minipage}[t]{13cm}{\footnotesize
                No data.
                \vspace{\dp0}
            } \end{minipage} \\ \cline{2-3}
            & Expected Result &
                \begin{minipage}[t]{13cm}{\footnotesize
                Confirmation of measurements in catalogs encoded in flux units.

                \vspace{\dp0}
                } \end{minipage}
        \\ \midrule
    \end{longtable}

\subsection{LVV-T1947 - Verify implementation of measurements in catalogs from difference images}\label{lvv-t1947}

\begin{longtable}[]{llllll}
\toprule
Version & Status & Priority & Verification Type & Owner
\\\midrule
1 & Approved & Normal &
Test & Jeffrey Carlin
\\\bottomrule
\multicolumn{6}{c}{ Open \href{https://jira.lsstcorp.org/secure/Tests.jspa\#/testCase/LVV-T1947}{LVV-T1947} in Jira } \\
\end{longtable}

\subsubsection{Verification Elements}
\begin{itemize}
\item \href{https://jira.lsstcorp.org/browse/LVV-178}{LVV-178} - DMS-REQ-0347-V-01: Measurements in catalogs

\end{itemize}

\subsubsection{Test Items}
Verify that source measurements in catalogs containing measurements from
difference images are in flux units.


\subsubsection{Predecessors}

\subsubsection{Environment Needs}

\paragraph{Software}

\paragraph{Hardware}

\subsubsection{Input Specification}

\subsubsection{Output Specification}

\subsubsection{Test Procedure}
    \begin{longtable}[]{p{1.3cm}p{2cm}p{13cm}}
    %\toprule
    Step & \multicolumn{2}{@{}l}{Description, Input Data and Expected Result} \\ \toprule
    \endhead

                \multirow{3}{*}{\parbox{1.3cm}{ 1-1
                {\scriptsize from \hyperref[lvv-t987]
                {LVV-T987} } } }

                & {\small Description} &
                \begin{minipage}[t]{13cm}{\scriptsize
                Identify the path to the data repository, which we will refer to as
`DATA/path', then execute the following:

                \vspace{\dp0}
                } \end{minipage} \\ \cdashline{2-3}
                & {\small Test Data} &
                \begin{minipage}[t]{13cm}{\scriptsize
                } \end{minipage} \\ \cdashline{2-3}
                & {\small Expected Result} &
                    \begin{minipage}[t]{13cm}{\scriptsize
                    Butler repo available for reading.

                    \vspace{\dp0}
                    } \end{minipage}
                \\ \hdashline


        \\ \midrule

            \multirow{3}{*}{ 2 } & Description &
            \begin{minipage}[t]{13cm}{\footnotesize
            Identify and read an appropriate processed precursor dataset containing
difference images with the Butler.

            \vspace{\dp0}
            } \end{minipage} \\ \cline{2-3}
            & Test Data &
            \begin{minipage}[t]{13cm}{\footnotesize
                No data.
                \vspace{\dp0}
            } \end{minipage} \\ \cline{2-3}
            & Expected Result &
        \\ \midrule

            \multirow{3}{*}{ 3 } & Description &
            \begin{minipage}[t]{13cm}{\footnotesize
            Verify that the difference image source catalog provides measurements in
flux units.

            \vspace{\dp0}
            } \end{minipage} \\ \cline{2-3}
            & Test Data &
            \begin{minipage}[t]{13cm}{\footnotesize
                No data.
                \vspace{\dp0}
            } \end{minipage} \\ \cline{2-3}
            & Expected Result &
                \begin{minipage}[t]{13cm}{\footnotesize
                Confirmation of measurements in catalogs encoded in flux units.

                \vspace{\dp0}
                } \end{minipage}
        \\ \midrule
    \end{longtable}

\newpage
\section{Reusable Test Cases}

Test cases in this section are made up of commonly encountered steps that have been factored out into modular, reusable scripts.
These test cases are meant solely for the building of actual tests used for verification, to be inserted in test scripts via the “Call to Test” functionality in Jira/ATM.
They streamline the process of writing test scripts by providing pre-designed steps, while also ensuring homogeneity throughout the test suite.
These reusable modules are not themselves verifying requirements.
Also, these test cases shall not call other reusable test cases in their script.




\subsection{LVV-T216 - Installation of the Alert Distribution payloads.}\label{lvv-t216}

\begin{longtable}[]{llllll}
\toprule
Version & Status & Priority & Verification Type & Owner
\\\midrule
1 & Approved & Normal &
Test & Eric Bellm
\\\bottomrule
\multicolumn{6}{c}{ Open \href{https://jira.lsstcorp.org/secure/Tests.jspa\#/testCase/LVV-T216}{LVV-T216} in Jira } \\
\end{longtable}

\subsubsection{Test Items}
This test will check:\\

\begin{itemize}
\tightlist
\item
  That the Alert Distribution payloads are available from documented
  channels.
\item
  That the Alert Distribution payloads can be installed on LSST Data
  Facility-managed systems.
\item
  That the Alert Distribution payloads can be executed by LSST Data
  Facility-managed systems.
\end{itemize}



\subsubsection{Environment Needs}


\paragraph{Hardware}
This test case shall be executed on the Kubernetes Commons at the LDF.\\
As discussed in https://dmtn-028.lsst.io/ and https://dmtn-081.lsst.io/,
the test machine should have at least 16 cores, 64 GB of memory and
access to at least 1.5 TB of shared storage.



\subsubsection{Test Procedure}
    \begin{longtable}[]{p{1.3cm}p{2cm}p{13cm}}
    %\toprule
    Step & \multicolumn{2}{@{}l}{Description, Input Data and Expected Result} \\ \toprule
    \endhead
            \multirow{3}{*}{\parbox{1.3cm}{ 1
} }
& {\small Description} &
\begin{minipage}[t]{13cm}{\scriptsize
Download Kafka Docker image from
https://github.com/lsst-dm/alert\_stream.

\vspace{\dp0}
} \end{minipage} \\ \cdashline{2-3}
& {\small Test Data} &
\\ \cdashline{2-3}
& {\small Expected Result} &
\begin{minipage}[t]{13cm}{\scriptsize
Runs without error

\vspace{\dp0}
} \end{minipage}
\\ \hdashline
        \\ \midrule
            \multirow{3}{*}{\parbox{1.3cm}{ 2
} }
& {\small Description} &
\begin{minipage}[t]{13cm}{\scriptsize
Change to the alert\_stream directory and build the docker image.\\

\begin{verbatim}
docker build -t "lsst-kub001:5000/alert_stream"
\end{verbatim}

\vspace{\dp0}
} \end{minipage} \\ \cdashline{2-3}
& {\small Test Data} &
\\ \cdashline{2-3}
& {\small Expected Result} &
\begin{minipage}[t]{13cm}{\scriptsize
Runs without error

\vspace{\dp0}
} \end{minipage}
\\ \hdashline
        \\ \midrule
            \multirow{3}{*}{\parbox{1.3cm}{ 3
} }
& {\small Description} &
\begin{minipage}[t]{13cm}{\scriptsize
Register it with Kubernetes\\[2\baselineskip]docker push
lsst-kub001:5000/alert\_stream

\vspace{\dp0}
} \end{minipage} \\ \cdashline{2-3}
& {\small Test Data} &
\\ \cdashline{2-3}
& {\small Expected Result} &
\begin{minipage}[t]{13cm}{\scriptsize
Runs without error

\vspace{\dp0}
} \end{minipage}
\\ \hdashline
        \\ \midrule
            \multirow{3}{*}{\parbox{1.3cm}{ 4
} }
& {\small Description} &
\begin{minipage}[t]{13cm}{\scriptsize
From the alert\_stream/kubernetes directory, start Kafka and
Zookeeper:\\[2\baselineskip]

\begin{verbatim}
kubectl create -f zookeeper-service.yaml
kubectl create -f zookeeper-deployment.yaml
kubectl create -f kafka-deployment.yaml
kubectl create -f kafka-service.yaml
\end{verbatim}

(use kubectl get pods/services between each command to check status;
wait until each is ``Running'' before starting the next
command)\\[2\baselineskip]

\vspace{\dp0}
} \end{minipage} \\ \cdashline{2-3}
& {\small Test Data} &
\\ \cdashline{2-3}
& {\small Expected Result} &
\begin{minipage}[t]{13cm}{\scriptsize
Runs without error

\vspace{\dp0}
} \end{minipage}
\\ \hdashline
        \\ \midrule
            \multirow{3}{*}{\parbox{1.3cm}{ 5
} }
& {\small Description} &
\begin{minipage}[t]{13cm}{\scriptsize
Confirm Kafka and Zookeeper are listed when
running\\[2\baselineskip]kubectl get
pods\\[2\baselineskip]and\\[2\baselineskip]kubectl get services

\vspace{\dp0}
} \end{minipage} \\ \cdashline{2-3}
& {\small Test Data} &
\\ \cdashline{2-3}
& {\small Expected Result} &
\begin{minipage}[t]{13cm}{\scriptsize
Output should be similar to:\\[2\baselineskip]kubectl get pods\\
NAME ~ ~ ~ ~ ~ ~ ~ ~ ~ ~ ~ ~READY ~ ~ STATUS ~ ~RESTARTS ~ AGE\\
kafka-768ddf5564-xwgvh ~ ~ ~1/1 ~ ~ ~ Running ~ 0 ~ ~ ~ ~ ~31s\\
zookeeper-f798cc548-mgkpn ~ 1/1 ~ ~ ~ Running ~ 0 ~ ~ ~ ~
~1m\\[2\baselineskip]kubectl get services\\
NAME ~ ~ ~ ~TYPE ~ ~ ~ ~CLUSTER-IP ~ ~ ~EXTERNAL-IP ~ PORT(S) ~ ~ AGE\\
kafka ~ ~ ~ ClusterIP ~ 10.105.19.124 ~ \textless{}none\textgreater{} ~
~ ~ ~9092/TCP ~ ~6s\\
zookeeper ~ ClusterIP ~ 10.97.110.124 ~ \textless{}none\textgreater{} ~
~ ~ ~32181/TCP ~ 2m

\vspace{\dp0}
} \end{minipage}
\\ \hdashline
        \\ \midrule
    \end{longtable}


\subsection{LVV-T837 - Authenticate to Notebook Aspect}\label{lvv-t837}

\begin{longtable}[]{llllll}
\toprule
Version & Status & Priority & Verification Type & Owner
\\\midrule
1 & Draft & Normal &
Test & Jeffrey Carlin
\\\bottomrule
\multicolumn{6}{c}{ Open \href{https://jira.lsstcorp.org/secure/Tests.jspa\#/testCase/LVV-T837}{LVV-T837} in Jira } \\
\end{longtable}

\subsubsection{Test Items}
Not specifically a test -- modular script to be used in multiple other
Test Scripts.






\subsubsection{Input Specification}
Must have a user account on the LSP.


\subsubsection{Test Procedure}
    \begin{longtable}[]{p{1.3cm}p{2cm}p{13cm}}
    %\toprule
    Step & \multicolumn{2}{@{}l}{Description, Input Data and Expected Result} \\ \toprule
    \endhead
            \multirow{3}{*}{\parbox{1.3cm}{ 1
} }
& {\small Description} &
\begin{minipage}[t]{13cm}{\scriptsize
Authenticate to the notebook aspect of the LSST Science Platform
(NB-LSP). ~This is currently at
https://lsst-lsp-stable.ncsa.illinois.edu/nb.

\vspace{\dp0}
} \end{minipage} \\ \cdashline{2-3}
& {\small Test Data} &
\\ \cdashline{2-3}
& {\small Expected Result} &
\begin{minipage}[t]{13cm}{\scriptsize
Redirection to the spawner page of the NB-LSP allowing selection of the
containerized stack version and machine flavor.

\vspace{\dp0}
} \end{minipage}
\\ \hdashline
        \\ \midrule
            \multirow{3}{*}{\parbox{1.3cm}{ 2
} }
& {\small Description} &
\begin{minipage}[t]{13cm}{\scriptsize
Spawn a container by:\\
1) choosing an appropriate stack version: e.g. the latest weekly.\\
2) choosing an appropriate machine flavor: e.g. medium\\
3) click ``Spawn''

\vspace{\dp0}
} \end{minipage} \\ \cdashline{2-3}
& {\small Test Data} &
\\ \cdashline{2-3}
& {\small Expected Result} &
\begin{minipage}[t]{13cm}{\scriptsize
Redirection to the JupyterLab environment served from the chosen
container containing the correct stack version.

\vspace{\dp0}
} \end{minipage}
\\ \hdashline
        \\ \midrule
    \end{longtable}


\subsection{LVV-T838 - Access an empty notebook in the Notebook Aspect}\label{lvv-t838}

\begin{longtable}[]{llllll}
\toprule
Version & Status & Priority & Verification Type & Owner
\\\midrule
1 & Draft & Normal &
Test & Simon Krughoff
\\\bottomrule
\multicolumn{6}{c}{ Open \href{https://jira.lsstcorp.org/secure/Tests.jspa\#/testCase/LVV-T838}{LVV-T838} in Jira } \\
\end{longtable}

\subsubsection{Test Items}
The steps here cover just those necessary to gain access to an empty
notebook after authentication is complete.






\subsubsection{Input Specification}
Authentication to the Notebook aspect.


\subsubsection{Test Procedure}
    \begin{longtable}[]{p{1.3cm}p{2cm}p{13cm}}
    %\toprule
    Step & \multicolumn{2}{@{}l}{Description, Input Data and Expected Result} \\ \toprule
    \endhead
            \multirow{3}{*}{\parbox{1.3cm}{ 1
} }
& {\small Description} &
\begin{minipage}[t]{13cm}{\scriptsize
Open a new launcher by navigating in the top menu bar ``File''
-\textgreater{} ``New Launcher''

\vspace{\dp0}
} \end{minipage} \\ \cdashline{2-3}
& {\small Test Data} &
\\ \cdashline{2-3}
& {\small Expected Result} &
\begin{minipage}[t]{13cm}{\scriptsize
A launcher window with several sections, potentially with several kernel
versions for each.

\vspace{\dp0}
} \end{minipage}
\\ \hdashline
        \\ \midrule
            \multirow{3}{*}{\parbox{1.3cm}{ 2
} }
& {\small Description} &
\begin{minipage}[t]{13cm}{\scriptsize
Select the option under ``Notebook'' labeled ``LSST'' by clicking on the
icon.

\vspace{\dp0}
} \end{minipage} \\ \cdashline{2-3}
& {\small Test Data} &
\\ \cdashline{2-3}
& {\small Expected Result} &
\begin{minipage}[t]{13cm}{\scriptsize
An empty notebook with a single empty cell. ~The kernel show up as
``LSST'' in the top right of the notebook.

\vspace{\dp0}
} \end{minipage}
\\ \hdashline
        \\ \midrule
    \end{longtable}


\subsection{LVV-T849 - Authenticate to the portal aspect of the LSP}\label{lvv-t849}

\begin{longtable}[]{llllll}
\toprule
Version & Status & Priority & Verification Type & Owner
\\\midrule
2 & Draft & Normal &
Test & Simon Krughoff
\\\bottomrule
\multicolumn{6}{c}{ Open \href{https://jira.lsstcorp.org/secure/Tests.jspa\#/testCase/LVV-T849}{LVV-T849} in Jira } \\
\end{longtable}

\subsubsection{Test Items}
Obtain an authenticated session in the portal aspect of the LSST Science
Platform








\subsubsection{Test Procedure}
    \begin{longtable}[]{p{1.3cm}p{2cm}p{13cm}}
    %\toprule
    Step & \multicolumn{2}{@{}l}{Description, Input Data and Expected Result} \\ \toprule
    \endhead
            \multirow{3}{*}{\parbox{1.3cm}{ 1
} }
& {\small Description} &
\begin{minipage}[t]{13cm}{\scriptsize
Navigate to the Portal Aspect endpoint. ~The stable version should be
used for this test and is currently located at:
https://lsst-lsp-stable.ncsa.illinois.edu/portal/app/ .

\vspace{\dp0}
} \end{minipage} \\ \cdashline{2-3}
& {\small Test Data} &
\\ \cdashline{2-3}
& {\small Expected Result} &
\begin{minipage}[t]{13cm}{\scriptsize
A credential-entry screen should be displayed.

\vspace{\dp0}
} \end{minipage}
\\ \hdashline
        \\ \midrule
            \multirow{3}{*}{\parbox{1.3cm}{ 2
} }
& {\small Description} &
\begin{minipage}[t]{13cm}{\scriptsize
Enter a valid set of credentials for an LSST user with LSP access on the
instance under test.

\vspace{\dp0}
} \end{minipage} \\ \cdashline{2-3}
& {\small Test Data} &
\\ \cdashline{2-3}
& {\small Expected Result} &
\begin{minipage}[t]{13cm}{\scriptsize
The Portal Aspect UI should be displayed following authentication.

\vspace{\dp0}
} \end{minipage}
\\ \hdashline
        \\ \midrule
    \end{longtable}


\subsection{LVV-T850 - Log out of the portal aspect of the LSP}\label{lvv-t850}

\begin{longtable}[]{llllll}
\toprule
Version & Status & Priority & Verification Type & Owner
\\\midrule
1 & Draft & Normal &
Test & Simon Krughoff
\\\bottomrule
\multicolumn{6}{c}{ Open \href{https://jira.lsstcorp.org/secure/Tests.jspa\#/testCase/LVV-T850}{LVV-T850} in Jira } \\
\end{longtable}

\subsubsection{Test Items}
Leave the portal aspect of the LSST Science Platform in a clean state








\subsubsection{Test Procedure}
    \begin{longtable}[]{p{1.3cm}p{2cm}p{13cm}}
    %\toprule
    Step & \multicolumn{2}{@{}l}{Description, Input Data and Expected Result} \\ \toprule
    \endhead
            \multirow{3}{*}{\parbox{1.3cm}{ 1
} }
& {\small Description} &
\begin{minipage}[t]{13cm}{\scriptsize
Currently, there is no logout mechanism on the portal.\\
This should be updated as the system matures.\\[2\baselineskip]Simply
close the browser window.

\vspace{\dp0}
} \end{minipage} \\ \cdashline{2-3}
& {\small Test Data} &
\\ \cdashline{2-3}
& {\small Expected Result} &
\begin{minipage}[t]{13cm}{\scriptsize
Closed browser window. ~When navigating to the portal endpoint, expect
to execute the steps in LVV-T849.

\vspace{\dp0}
} \end{minipage}
\\ \hdashline
        \\ \midrule
    \end{longtable}


\subsection{LVV-T860 - Initialize science pipelines}\label{lvv-t860}

\begin{longtable}[]{llllll}
\toprule
Version & Status & Priority & Verification Type & Owner
\\\midrule
1 & Draft & Normal &
Test & Jeffrey Carlin
\\\bottomrule
\multicolumn{6}{c}{ Open \href{https://jira.lsstcorp.org/secure/Tests.jspa\#/testCase/LVV-T860}{LVV-T860} in Jira } \\
\end{longtable}

\subsubsection{Test Items}
Initialize the science pipelines software for use.~






\subsubsection{Input Specification}
An installed software stack, either locally, on `lsst-dev`, or through
the Notebook aspect.


\subsubsection{Test Procedure}
    \begin{longtable}[]{p{1.3cm}p{2cm}p{13cm}}
    %\toprule
    Step & \multicolumn{2}{@{}l}{Description, Input Data and Expected Result} \\ \toprule
    \endhead
            \multirow{3}{*}{\parbox{1.3cm}{ 1
} }
& {\small Description} &
\begin{minipage}[t]{13cm}{\scriptsize
The `path` that you will use depends on where you are running the
science pipelines. Options:\\[2\baselineskip]

\begin{itemize}
\tightlist
\item
  local (newinstall.sh - based
  install):{[}path\_to\_installation{]}/loadLSST.bash
\item
  development cluster (``lsst-dev''):
  /software/lsstsw/stack/loadLSST.bash
\item
  LSP Notebook aspect (from a terminal):
  /opt/lsst/software/stack/loadLSST.bash
\end{itemize}

From the command line, execute the commands below in the example
code:\\[2\baselineskip]

\vspace{\dp0}
} \end{minipage} \\ \cdashline{2-3}
& {\small Test Data} &
\\ \cdashline{2-3}
& Example Code &
\begin{minipage}[t]{13cm}{\footnotesize
source `path`\\
setup lsst\_distrib

\vspace{\dp0}
} \end{minipage} \\ \cline{2-3}
& {\small Expected Result} &
\begin{minipage}[t]{13cm}{\scriptsize
Science pipeline software is available for use. If additional packages
are needed (for example, `obs' packages such as `obs\_subaru`), then
additional `setup` commands will be necessary.\\[2\baselineskip]To check
versions in use, type:\\
eups list -s

\vspace{\dp0}
} \end{minipage}
\\ \hdashline
        \\ \midrule
    \end{longtable}


\subsection{LVV-T866 - Run Alert Production Payload}\label{lvv-t866}

\begin{longtable}[]{llllll}
\toprule
Version & Status & Priority & Verification Type & Owner
\\\midrule
1 & Draft & Normal &
Test & Jeffrey Carlin
\\\bottomrule
\multicolumn{6}{c}{ Open \href{https://jira.lsstcorp.org/secure/Tests.jspa\#/testCase/LVV-T866}{LVV-T866} in Jira } \\
\end{longtable}

\subsubsection{Test Items}
Execute Alert Production payload on a dataset. Generate all (or a subset
of) Prompt science data products including Alerts (with the exception of
Solar System object orbits) and load them into the Data Backbone and
Prompt Products Database. ~~








\subsubsection{Test Procedure}
    \begin{longtable}[]{p{1.3cm}p{2cm}p{13cm}}
    %\toprule
    Step & \multicolumn{2}{@{}l}{Description, Input Data and Expected Result} \\ \toprule
    \endhead
            \multirow{3}{*}{\parbox{1.3cm}{ 1
} }
& {\small Description} &
\begin{minipage}[t]{13cm}{\scriptsize
Perform the steps of Alert Production (including, but not necessarily
limited to, single frame processing, ISR, source detection/measurement,
PSF estimation, photometric and astrometric calibration, difference
imaging, DIASource detection/measurement, source association). During
Operations, it is presumed that these are automated for a given
dataset.~

\vspace{\dp0}
} \end{minipage} \\ \cdashline{2-3}
& {\small Test Data} &
\\ \cdashline{2-3}
& {\small Expected Result} &
\begin{minipage}[t]{13cm}{\scriptsize
An output dataset including difference images and DIASource and
DIAObject measurements.

\vspace{\dp0}
} \end{minipage}
\\ \hdashline
        \\ \midrule
            \multirow{3}{*}{\parbox{1.3cm}{ 2
} }
& {\small Description} &
\begin{minipage}[t]{13cm}{\scriptsize
Verify that the expected data products have been produced, and that
catalogs contain reasonable values for measured quantities of interest.

\vspace{\dp0}
} \end{minipage} \\ \cdashline{2-3}
& {\small Test Data} &
\\ \cdashline{2-3}
& {\small Expected Result} &
\\ \hdashline
        \\ \midrule
    \end{longtable}


\subsection{LVV-T901 - Run MOPS payload}\label{lvv-t901}

\begin{longtable}[]{llllll}
\toprule
Version & Status & Priority & Verification Type & Owner
\\\midrule
1 & Draft & Normal &
Test & Jeffrey Carlin
\\\bottomrule
\multicolumn{6}{c}{ Open \href{https://jira.lsstcorp.org/secure/Tests.jspa\#/testCase/LVV-T901}{LVV-T901} in Jira } \\
\end{longtable}

\subsubsection{Test Items}
Run MOPS payload on a dataset (for example, one night's data). Generate
entries in the MOPS Database and the Prompt Products Database, including
Solar System Object records, measurements, and orbits. Perform precovery
forced photometry of transients.


\subsubsection{Predecessors}
Uses results loaded into Prompt Products database and Data Backbone
services in
\href{https://jira.lsstcorp.org/secure/Tests.jspa\#/testCase/LVV-T866}{LVV-T866}.






\subsubsection{Test Procedure}
    \begin{longtable}[]{p{1.3cm}p{2cm}p{13cm}}
    %\toprule
    Step & \multicolumn{2}{@{}l}{Description, Input Data and Expected Result} \\ \toprule
    \endhead
            \multirow{3}{*}{\parbox{1.3cm}{ 1
} }
& {\small Description} &
\begin{minipage}[t]{13cm}{\scriptsize
Perform the steps of Moving Object Pipeline (MOPS) processing on newly
detected DIASources, and generate Solar System data products including
Solar System objects with associated Keplerian orbits, errors, and
detected DIASources. This includes running processes to link DIASource
detections within a night (called tracklets), to link these tracklets
across multiple nights (into tracks), to fit the tracks with an orbital
model to identify those tracks that are consistent with an asteroid
orbit, to match these new orbits with existing SSObjects, and to update
the SSObject table. ~ ~ ~ ~ ~ ~ ~ ~ ~ ~ ~ ~ ~ ~ ~ ~ ~ ~ ~~

\vspace{\dp0}
} \end{minipage} \\ \cdashline{2-3}
& {\small Test Data} &
\\ \cdashline{2-3}
& {\small Expected Result} &
\begin{minipage}[t]{13cm}{\scriptsize
An output dataset consisting of an updated SSObject database with
SSObjects both added and pruned as the orbital fits have been refined,
and an updated DIASource database with DIASources assigned and
unassigned to SSObjects.

\vspace{\dp0}
} \end{minipage}
\\ \hdashline
        \\ \midrule
            \multirow{3}{*}{\parbox{1.3cm}{ 2
} }
& {\small Description} &
\begin{minipage}[t]{13cm}{\scriptsize
Verify that the expected data products have been produced, and that
catalogs contain reasonable values for measured quantities of interest.

\vspace{\dp0}
} \end{minipage} \\ \cdashline{2-3}
& {\small Test Data} &
\\ \cdashline{2-3}
& {\small Expected Result} &
\\ \hdashline
        \\ \midrule
    \end{longtable}


\subsection{LVV-T987 - Instantiate the Butler for reading data}\label{lvv-t987}

\begin{longtable}[]{llllll}
\toprule
Version & Status & Priority & Verification Type & Owner
\\\midrule
1 & Draft & Normal &
Test & Jeffrey Carlin
\\\bottomrule
\multicolumn{6}{c}{ Open \href{https://jira.lsstcorp.org/secure/Tests.jspa\#/testCase/LVV-T987}{LVV-T987} in Jira } \\
\end{longtable}

\subsubsection{Test Items}
Create a Butler client to read data from an input repository.






\subsubsection{Input Specification}
\href{https://jira.lsstcorp.org/secure/Tests.jspa\#/testCase/LVV-T860}{LVV-T860}
must be executed to initialize the science pipelines.


\subsubsection{Test Procedure}
    \begin{longtable}[]{p{1.3cm}p{2cm}p{13cm}}
    %\toprule
    Step & \multicolumn{2}{@{}l}{Description, Input Data and Expected Result} \\ \toprule
    \endhead
            \multirow{3}{*}{\parbox{1.3cm}{ 1
} }
& {\small Description} &
\begin{minipage}[t]{13cm}{\scriptsize
Identify the path to the data repository, which we will refer to as
`DATA/path', then execute the following:

\vspace{\dp0}
} \end{minipage} \\ \cdashline{2-3}
& {\small Test Data} &
\\ \cdashline{2-3}
& Example Code &
\begin{minipage}[t]{13cm}{\footnotesize
\begin{verbatim}
import lsst.daf.persistence as dafPersist
butler = dafPersist.Butler(inputs='DATA/path')
\end{verbatim}

\vspace{\dp0}
} \end{minipage} \\ \cline{2-3}
& {\small Expected Result} &
\begin{minipage}[t]{13cm}{\scriptsize
Butler repo available for reading.

\vspace{\dp0}
} \end{minipage}
\\ \hdashline
        \\ \midrule
    \end{longtable}


\subsection{LVV-T1059 - Run Daily Calibration Products Update Payload}\label{lvv-t1059}

\begin{longtable}[]{llllll}
\toprule
Version & Status & Priority & Verification Type & Owner
\\\midrule
1 & Draft & Normal &
Test & Jeffrey Carlin
\\\bottomrule
\multicolumn{6}{c}{ Open \href{https://jira.lsstcorp.org/secure/Tests.jspa\#/testCase/LVV-T1059}{LVV-T1059} in Jira } \\
\end{longtable}

\subsubsection{Test Items}
Execute the Daily Calibration Products Update payload to create a subset
of Master Calibration images and Calibration Database entries.








\subsubsection{Test Procedure}
    \begin{longtable}[]{p{1.3cm}p{2cm}p{13cm}}
    %\toprule
    Step & \multicolumn{2}{@{}l}{Description, Input Data and Expected Result} \\ \toprule
    \endhead
            \multirow{3}{*}{\parbox{1.3cm}{ 1
} }
& {\small Description} &
\begin{minipage}[t]{13cm}{\scriptsize
Execute the Daily Calibration Products Update payload. The payload uses
raw calibration images and information from the Transformed EFD to
generate a subset of Master Calibration Images and Calibration Database
entries in the Data Backbone.

\vspace{\dp0}
} \end{minipage} \\ \cdashline{2-3}
& {\small Test Data} &
\\ \cdashline{2-3}
& {\small Expected Result} &
\\ \hdashline
        \\ \midrule
            \multirow{3}{*}{\parbox{1.3cm}{ 2
} }
& {\small Description} &
\begin{minipage}[t]{13cm}{\scriptsize
Confirm that the expected Master Calibration images and Calibration
Database entries are present and well-formed.

\vspace{\dp0}
} \end{minipage} \\ \cdashline{2-3}
& {\small Test Data} &
\\ \cdashline{2-3}
& {\small Expected Result} &
\\ \hdashline
        \\ \midrule
    \end{longtable}


\subsection{LVV-T1060 - Run Periodic Calibration Products Production Payload}\label{lvv-t1060}

\begin{longtable}[]{llllll}
\toprule
Version & Status & Priority & Verification Type & Owner
\\\midrule
1 & Draft & Normal &
Test & Jeffrey Carlin
\\\bottomrule
\multicolumn{6}{c}{ Open \href{https://jira.lsstcorp.org/secure/Tests.jspa\#/testCase/LVV-T1060}{LVV-T1060} in Jira } \\
\end{longtable}

\subsubsection{Test Items}
Execute the Calibration Products Production payload to create a subset
of Master Calibration images and Calibration Database entries.








\subsubsection{Test Procedure}
    \begin{longtable}[]{p{1.3cm}p{2cm}p{13cm}}
    %\toprule
    Step & \multicolumn{2}{@{}l}{Description, Input Data and Expected Result} \\ \toprule
    \endhead
            \multirow{3}{*}{\parbox{1.3cm}{ 1
} }
& {\small Description} &
\begin{minipage}[t]{13cm}{\scriptsize
Execute the Calibration Products Production payload. The payload uses
raw calibration images and information from the Transformed EFD to
generate a subset of Master Calibration Images and Calibration Database
entries in the Data Backbone.

\vspace{\dp0}
} \end{minipage} \\ \cdashline{2-3}
& {\small Test Data} &
\\ \cdashline{2-3}
& {\small Expected Result} &
\\ \hdashline
        \\ \midrule
            \multirow{3}{*}{\parbox{1.3cm}{ 2
} }
& {\small Description} &
\begin{minipage}[t]{13cm}{\scriptsize
Confirm that the expected Master Calibration images and Calibration
Database entries are present and well-formed.

\vspace{\dp0}
} \end{minipage} \\ \cdashline{2-3}
& {\small Test Data} &
\\ \cdashline{2-3}
& {\small Expected Result} &
\\ \hdashline
        \\ \midrule
    \end{longtable}


\subsection{LVV-T1064 - Run Data Release Production Payload}\label{lvv-t1064}

\begin{longtable}[]{llllll}
\toprule
Version & Status & Priority & Verification Type & Owner
\\\midrule
1 & Draft & Normal &
Test & Jeffrey Carlin
\\\bottomrule
\multicolumn{6}{c}{ Open \href{https://jira.lsstcorp.org/secure/Tests.jspa\#/testCase/LVV-T1064}{LVV-T1064} in Jira } \\
\end{longtable}

\subsubsection{Test Items}
Execute the Data Release Production payload, starting from raw images
and producing science data products.








\subsubsection{Test Procedure}
    \begin{longtable}[]{p{1.3cm}p{2cm}p{13cm}}
    %\toprule
    Step & \multicolumn{2}{@{}l}{Description, Input Data and Expected Result} \\ \toprule
    \endhead
            \multirow{3}{*}{\parbox{1.3cm}{ 1
} }
& {\small Description} &
\begin{minipage}[t]{13cm}{\scriptsize
Process data with the Data Release Production payload, starting from raw
science images and generating science data products, placing them in the
Data Backbone.

\vspace{\dp0}
} \end{minipage} \\ \cdashline{2-3}
& {\small Test Data} &
\\ \cdashline{2-3}
& {\small Expected Result} &
\\ \hdashline
        \\ \midrule
    \end{longtable}


\subsection{LVV-T1207 - Execute a simple ADQL query using the TAP service in the notebook aspect}\label{lvv-t1207}

\begin{longtable}[]{llllll}
\toprule
Version & Status & Priority & Verification Type & Owner
\\\midrule
1 & Draft & Normal &
Test & Jeffrey Carlin
\\\bottomrule
\multicolumn{6}{c}{ Open \href{https://jira.lsstcorp.org/secure/Tests.jspa\#/testCase/LVV-T1207}{LVV-T1207} in Jira } \\
\end{longtable}

\subsubsection{Test Items}
Extract a small amount of data from a catalog via the LSST TAP service.






\subsubsection{Input Specification}
One must have access to the LSST Notebook Aspect, and have logged in and
opened an empty notebook.


\subsubsection{Test Procedure}
    \begin{longtable}[]{p{1.3cm}p{2cm}p{13cm}}
    %\toprule
    Step & \multicolumn{2}{@{}l}{Description, Input Data and Expected Result} \\ \toprule
    \endhead
            \multirow{3}{*}{\parbox{1.3cm}{ 1
} }
& {\small Description} &
\begin{minipage}[t]{13cm}{\scriptsize
Execute a query in a notebook to select a small number of stars. In the
example code below, we query the WISE catalog, then extract the results
to an Astropy table.

\vspace{\dp0}
} \end{minipage} \\ \cdashline{2-3}
& {\small Test Data} &
\\ \cdashline{2-3}
& Example Code &
\begin{minipage}[t]{13cm}{\footnotesize
\begin{verbatim}
import pandas
import pyvo
service = pyvo.dal.TAPService('http://lsst-lsp-stable.ncsa.illinois.edu/api/tap')
\end{verbatim}

results = service.search(``SELECT ra, decl, w1mpro\_ep, w2mpro\_ep,
w3mpro\_ep FROM wise\_00.allwise\_p3as\_mep WHERE CONTAINS(POINT('ICRS',
ra, decl), CIRCLE('ICRS', 192.85, 27.13, .2)) = 1'')\\
tab = results.to\_table()

\vspace{\dp0}
} \end{minipage} \\ \cline{2-3}
& {\small Expected Result} &
\\ \hdashline
        \\ \midrule
    \end{longtable}


\subsection{LVV-T1208 - Log out of the notebook aspect of the LSP}\label{lvv-t1208}

\begin{longtable}[]{llllll}
\toprule
Version & Status & Priority & Verification Type & Owner
\\\midrule
1 & Draft & Normal &
Test & Simon Krughoff
\\\bottomrule
\multicolumn{6}{c}{ Open \href{https://jira.lsstcorp.org/secure/Tests.jspa\#/testCase/LVV-T1208}{LVV-T1208} in Jira } \\
\end{longtable}

\subsubsection{Test Items}
Leave the notebook aspect of the LSST Science Platform in a clean state








\subsubsection{Test Procedure}
    \begin{longtable}[]{p{1.3cm}p{2cm}p{13cm}}
    %\toprule
    Step & \multicolumn{2}{@{}l}{Description, Input Data and Expected Result} \\ \toprule
    \endhead
            \multirow{3}{*}{\parbox{1.3cm}{ 1
} }
& {\small Description} &
\begin{minipage}[t]{13cm}{\scriptsize
Under the `File' menu at the top of your Jupyter notebook session,
select one of the following:\\[2\baselineskip]

\begin{itemize}
\tightlist
\item
  Save All, Exit, and Log Out
\item
  Exit and Log Out Without Saving
\end{itemize}

\vspace{\dp0}
} \end{minipage} \\ \cdashline{2-3}
& {\small Test Data} &
\\ \cdashline{2-3}
& {\small Expected Result} &
\begin{minipage}[t]{13cm}{\scriptsize
You will be returned to the LSP landing page:
\url{https://lsst-lsp-stable.ncsa.illinois.edu/} It is now safe to close
the browser window.~

\vspace{\dp0}
} \end{minipage}
\\ \hdashline
        \\ \midrule
    \end{longtable}


\subsection{LVV-T1744 - Run validate\_drp on precursor data}\label{lvv-t1744}

\begin{longtable}[]{llllll}
\toprule
Version & Status & Priority & Verification Type & Owner
\\\midrule
1 & Defined & Normal &
Analysis & Jeffrey Carlin
\\\bottomrule
\multicolumn{6}{c}{ Open \href{https://jira.lsstcorp.org/secure/Tests.jspa\#/testCase/LVV-T1744}{LVV-T1744} in Jira } \\
\end{longtable}

\subsubsection{Test Items}
Run the validate\_drp code on a precursor dataset to evaluate the
metrics that have been implemented in validate\_drp.








\subsubsection{Test Procedure}
    \begin{longtable}[]{p{1.3cm}p{2cm}p{13cm}}
    %\toprule
    Step & \multicolumn{2}{@{}l}{Description, Input Data and Expected Result} \\ \toprule
    \endhead
            \multirow{3}{*}{\parbox{1.3cm}{ 1
} }
& {\small Description} &
\begin{minipage}[t]{13cm}{\scriptsize
Execute `validate\_drp` on a repository containing precursor data.
Identify the path to the data, which we will call `DATA/path', then
execute the following (with additional flags specified as needed):

\vspace{\dp0}
} \end{minipage} \\ \cdashline{2-3}
& {\small Test Data} &
\\ \cdashline{2-3}
& Example Code &
\begin{minipage}[t]{13cm}{\footnotesize
validateDrp.py `DATA/path`

\vspace{\dp0}
} \end{minipage} \\ \cline{2-3}
& {\small Expected Result} &
\begin{minipage}[t]{13cm}{\scriptsize
JSON files (and associated figures) containing the Measurements and any
associated ``extras.''

\vspace{\dp0}
} \end{minipage}
\\ \hdashline
        \\ \midrule
    \end{longtable}





\newpage
\section{Deprecated Test Cases}

This section includes all test cases that have been marked as deprecated.
These test cases will never be executed again, but have been in the past.
For this reason it is important to keep them in the baseline as a reference.

  \textit{No deprecated test cases found.}

\newpage
\appendix
