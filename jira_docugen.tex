\newpage
\hypertarget{test-case-summary}{%
\section{Test Case Summary}\label{test-case-summary}}

\begin{longtable}[]{@{}ll@{}}
\toprule
Jira Id & Test Name\tabularnewline
\midrule
\endhead
\protect\hyperlink{lvv-t54---verify-alert-content-dms-req-0274-implementation}{LVV-T54}
& Verify Alert Content (DMS-REQ-0274) implementation\tabularnewline
\protect\hyperlink{lvv-t76---verify-all-sky-visualization-of-data-releases-dms-req-0329-implementation}{LVV-T76}
& Verify All-Sky Visualization of Data Releases (DMS-REQ-0329)
implementation\tabularnewline
\protect\hyperlink{lvv-t61---verify-associate-sources-to-objects--dms-req-0034-implementation}{LVV-T61}
& Verify Associate Sources to Objects (DMS-REQ-0034)
implementation\tabularnewline
\protect\hyperlink{lvv-t77---verify-best-seeing-coadds-dms-req-0330-implementation}{LVV-T77}
& Verify Best Seeing Coadds (DMS-REQ-0330) implementation\tabularnewline
\protect\hyperlink{lvv-t57---verify-calculating-ssobject-parameters-dms-req-0323-implementation}{LVV-T57}
& Verify Calculating SSObject Parameters (DMS-REQ-0323)
implementation\tabularnewline
\protect\hyperlink{lvv-t56---verify-characterizing-variability-dms-req-0319-implementation}{LVV-T56}
& Verify Characterizing Variability (DMS-REQ-0319)
implementation\tabularnewline
\protect\hyperlink{lvv-t72---verify-coadd-image-method-constraints-dms-req-0278-implementation}{LVV-T72}
& Verify Coadd Image Method Constraints (DMS-REQ-0278)
implementation\tabularnewline
\protect\hyperlink{lvv-t70---verify-coadd-source-catalog-dms-req-0277-implementation}{LVV-T70}
& Verify Coadd Source Catalog (DMS-REQ-0277)
implementation\tabularnewline
\protect\hyperlink{lvv-t64---verify-coadded-image-provenance-dms-req-0106-implementation}{LVV-T64}
& Verify Coadded Image Provenance (DMS-REQ-0106)
implementation\tabularnewline
\protect\hyperlink{lvv-t24---verify-computing-derived-quantities-dms-req-0331-implementation}{LVV-T24}
& Verify Computing Derived Quantities (DMS-REQ-0331)
Implementation\tabularnewline
\protect\hyperlink{lvv-t55---verify-diaforcedsource-catalog-dms-req-0317-implementation}{LVV-T55}
& Verify DIAForcedSource Catalog (DMS-REQ-0317)
implementation\tabularnewline
\protect\hyperlink{lvv-t51---verify-diaobject-catalog-dms-req-0271-implementation}{LVV-T51}
& Verify DIAObject Catalog (DMS-REQ-0271) Implementation\tabularnewline
\protect\hyperlink{lvv-t49---verify-diasource-catalog-dms-req-0269-implementation}{LVV-T49}
& Verify DIASource Catalog (DMS-REQ-0269) Implementation\tabularnewline
\protect\hyperlink{lvv-t73---verify-deep-detection-coadds-dms-req-0279-implementation}{LVV-T73}
& Verify Deep Detection Coadds (DMS-REQ-0279)
implementation\tabularnewline
\protect\hyperlink{lvv-t25---verify-denormalizing-database-tables-dms-req-0332-implementation}{LVV-T25}
& Verify Denormalizing Database Tables (DMS-REQ-0332)
Implementation\tabularnewline
\protect\hyperlink{lvv-t71---verify-detecting-extended--low-surface-brightness-objects-dms-req-0349-implementation}{LVV-T71}
& Verify Detecting extended low surface brightness objects
(DMS-REQ-0349) implementation\tabularnewline
\protect\hyperlink{lvv-t37---verify-difference-exposure-attributes-dms-req-0074-implementation}{LVV-T37}
& Verify Difference Exposure Attributes (DMS-REQ-0074)
Implementation\tabularnewline
\protect\hyperlink{lvv-t36---verify-difference-exposures--dms-req-0010-implementation}{LVV-T36}
& Verify Difference Exposures (DMS-REQ-0010)
Implementation\tabularnewline
\protect\hyperlink{lvv-t48---verify-exposure-catalog-dms-req-0266-implementation}{LVV-T48}
& Verify Exposure Catalog (DMS-REQ-0266) Implementation\tabularnewline
\protect\hyperlink{lvv-t50---verify-faint-diasource-measurements-dms-req-0270-implementation}{LVV-T50}
& Verify Faint DIASource Measurements (DMS-REQ-0270)
Implementation\tabularnewline
\protect\hyperlink{lvv-t66---verify-forced-source-catalog-dms-req-0268-implementation}{LVV-T66}
& Verify Forced-Source Catalog (DMS-REQ-0268)
implementation\tabularnewline
\protect\hyperlink{lvv-t58---verify-matching-diasources-to-objects-dms-req-0324-implementation}{LVV-T58}
& Verify Matching DIASources to Objects (DMS-REQ-0324)
implementation\tabularnewline
\protect\hyperlink{lvv-t28---verify-measurements-in-catalogs-dms-req-0347-implementation}{LVV-T28}
& Verify Measurements in catalogs (DMS-REQ-0347)
Implementation\tabularnewline
\protect\hyperlink{lvv-t75---verify-multi-band-coadds-dms-req-0281-implementation}{LVV-T75}
& Verify Multi-band Coadds (DMS-REQ-0281) implementation\tabularnewline
\protect\hyperlink{lvv-t35---verify-nightly-data-accessible-within-24-hrs-dms-req-0004-implementation}{LVV-T35}
& Verify Nightly Data Accessible Within 24 hrs (DMS-REQ-0004)
Implementation\tabularnewline
\protect\hyperlink{lvv-t78---verify-persisting-data-products-dms-req-0334-implementation}{LVV-T78}
& Verify Persisting Data Products (DMS-REQ-0334)
implementation\tabularnewline
\protect\hyperlink{lvv-t38---verify-processed-visit-images-dms-req-0069-implementation}{LVV-T38}
& Verify Processed Visit Images (DMS-REQ-0069)
Implementation\tabularnewline
\protect\hyperlink{lvv-t63---verify-produce-images-for-epo-dms-req-0103-implementation}{LVV-T63}
& Verify Produce Images for EPO (DMS-REQ-0103)
implementation\tabularnewline
\protect\hyperlink{lvv-t62---verify-provide-psf-for-coadded-images-dms-req-0047-implementation}{LVV-T62}
& Verify Provide PSF for Coadded Images (DMS-REQ-0047)
implementation\tabularnewline
\protect\hyperlink{lvv-t68---verify-provide-photometric-redshifts-of-galaxies-dms-req-0046-implementation}{LVV-T68}
& Verify Provide Photometric Redshifts of Galaxies (DMS-REQ-0046)
implementation\tabularnewline
\protect\hyperlink{lvv-t53---verify-ssobject-catalog-dms-req-0273-implementation}{LVV-T53}
& Verify SSObject Catalog (DMS-REQ-0273) implementation\tabularnewline
\protect\hyperlink{lvv-t74---verify-template-coadds-dms-req-0280-implementation}{LVV-T74}
& Verify Template Coadds (DMS-REQ-0280) implementation\tabularnewline
\protect\hyperlink{lvv-t23---verify-test-storing-approximations-of-per-pixel-metadata-dms-req-0326-implementation}{LVV-T23}
& Verify Test Storing Approximations of Per-pixel Metadata
(DMS-REQ-0326) Implementation\tabularnewline
\protect\hyperlink{lvv-t80---verify-implementation-of-detecting-faint-variable-objects-dms-req-0337}{LVV-T80}
& Verify implementation of Detecting faint variable objects
(DMS-REQ-0337)\tabularnewline
\protect\hyperlink{lvv-t79---verify-implementation-of-psf-matched-coadds-dms-req-0335}{LVV-T79}
& Verify implementation of PSF-Matched Coadds
(DMS-REQ-0335)\tabularnewline
\protect\hyperlink{lvv-t82---verify-implementation-of-tracking-characterization-changes-between-data-releases-dms-req-0339}{LVV-T82}
& Verify implementation of Tracking Characterization Changes Between
Data Releases (DMS-REQ-0339)\tabularnewline
\bottomrule
\end{longtable}

\newpage
\hypertarget{test-cases}{%
\section{Test Cases}\label{test-cases}}

\hypertarget{lvv-t54---verify-alert-content-dms-req-0274-implementation}{%
\subsection{\texorpdfstring{\href{https://jira.lsstcorp.org/secure/Tests.jspa\#/testCase/LVV-T54}{LVV-T54}
- Verify Alert Content (DMS-REQ-0274)
implementation}{LVV-T54 - Verify Alert Content (DMS-REQ-0274) implementation}}\label{lvv-t54---verify-alert-content-dms-req-0274-implementation}}

\begin{longtable}[]{@{}llllll@{}}
\toprule
Version & Status & Priority & Verification Type & Critical Event &
Owner\tabularnewline
\midrule
\endhead
1 & Draft & Normal & Test & False & ebellm\tabularnewline
\bottomrule
\end{longtable}

\hypertarget{test-items}{%
\subsubsection{Test Items}\label{test-items}}

\hypertarget{requirements}{%
\subsubsection{Requirements}\label{requirements}}

\hypertarget{requirements-1}{%
\subsubsection{Requirements}\label{requirements-1}}

\begin{itemize}
\tightlist
\item
  \href{https://jira.lsstcorp.org/browse/LVV-105}{LVV-105} -
  DMS-REQ-0274-V-01: Alert Content
\end{itemize}

\hypertarget{test-script}{%
\subsubsection{Test Script}\label{test-script}}

\textbf{Step 1}\\
The DM Stack and Alert Processing packaged shall be initialized as
described in LVT-T17 (AG-00-00).\\
~\\
\textbf{Step 2}\\
The alert generation processing will be executed using the verification
cluster:\\
\hspace*{0.333em} ~ ~ ~python
ap\_verify/bin/prepare\_demo\_slurm\_files.py\\
\hspace*{0.333em} ~ ~ ~\# At present we must run a single ccd+visit to
handle ingestion before\\
\hspace*{0.333em} ~ ~ ~\# parallel processing can begin\\
\hspace*{0.333em} ~ ~ ~./ap\_verify/bin/exec\_demo\_run\_1ccd.sh 410915
25\\
\hspace*{0.333em} ~ ~ ~ln -s ap\_verify/bin/demo\_run.sl\\
\hspace*{0.333em} ~ ~ ~ln -s ap\_verify/bin/demo\_cmds.conf\\
\hspace*{0.333em} ~ ~ ~sbatch demo\_run.sl\\
and any errors or failures reported.\\
~\\
\textbf{Step 3}\\
A ``Data Butler'' will be initialized to access the repository.\\
~\\
\textbf{Step 4}\\
For each of the expected data products types (listed in §4.2.2) and each
of the expected units (PVIs, catalogs, etc.), the data product will be
retrieved from the Butler and verified to be non-empty.\\
~\\
\textbf{Step 5}\\
DIAObjects are currently only stored in a database, without shims to the
Butler, so the existence of the database table and its non-empty
contents will be verified by directly accessing it using sqlite3 and
executing appropriate SQL queries.\\
~\\

\hypertarget{lvv-t76---verify-all-sky-visualization-of-data-releases-dms-req-0329-implementation}{%
\subsection{\texorpdfstring{\href{https://jira.lsstcorp.org/secure/Tests.jspa\#/testCase/LVV-T76}{LVV-T76}
- Verify All-Sky Visualization of Data Releases (DMS-REQ-0329)
implementation}{LVV-T76 - Verify All-Sky Visualization of Data Releases (DMS-REQ-0329) implementation}}\label{lvv-t76---verify-all-sky-visualization-of-data-releases-dms-req-0329-implementation}}

\begin{longtable}[]{@{}llllll@{}}
\toprule
Version & Status & Priority & Verification Type & Critical Event &
Owner\tabularnewline
\midrule
\endhead
1 & Draft & Normal & Test & False & gpdf\tabularnewline
\bottomrule
\end{longtable}

\hypertarget{test-items-1}{%
\subsubsection{Test Items}\label{test-items-1}}

\hypertarget{requirements-2}{%
\subsubsection{Requirements}\label{requirements-2}}

\hypertarget{requirements-3}{%
\subsubsection{Requirements}\label{requirements-3}}

\begin{itemize}
\tightlist
\item
  \href{https://jira.lsstcorp.org/browse/LVV-160}{LVV-160} -
  DMS-REQ-0329-V-01: All-Sky Visualization of Data Releases
\end{itemize}

\hypertarget{precondition}{%
\subsubsection{Precondition}\label{precondition}}

Input Data\\
Dataset of perhaps \textasciitilde{}100 square degrees. ~The first HSC
Public Data Release might be big enough, but may want to revisit with
larger datasets.

\hypertarget{test-script-1}{%
\subsubsection{Test Script}\label{test-script-1}}

\textbf{Step 1}\\
The DM Stack shall be initialized using the loadLSST script (as
described in DRP-00-00).\\
~\\
\textbf{Step 2}\\
A ``Data Butler'' will be initialized to access the repository.\\
~\\
\textbf{Step 3}\\
For each of the expected data products types (listed in Test Items
section §4.3.2) and each of the expected units (PVIs, coadds, etc), the
data product will be retrieved from the Butler and verified to be
non-empty.\\
~\\
~\\
~\\
\textbf{Step 4}\\
Verify that available images can be zoomed and navigated easily across
the entire available dataset.\\
~\\
\textbf{Step 5}\\
Clone the Github lsst/LDM-540 package. Record the SHA1 for the version
of the package to be used. {[}Additional procedures, e.g., tagging, are
still to be confirmed.{]}\\
~\\
\textbf{Step 6}\\
Log host and networking details of the client host to be used. Log the
Web browser version to be used. Log the version of Jupyter to be used.\\
~\\
\textbf{Step 7}\\
Establish VPN connectivity to the PDAC at NCSA.\\
~\\
\textbf{Step 8}\\
\emph{}Execute the LDM-540/test\_scripts/lsp-00-25.ipynb notebook to
perform the tests listed above against the API Aspect.\\
~\\
\textbf{Step 9}\\
Preserve any outputs of the script that are in the form of files outside
the notebook.\\
~\\
\textbf{Step 10}\\
Manually perform, and log, the following steps against the Portal
Aspect:

\begin{enumerate}
\tightlist
\item
  Navigate to the PDAC Portal. Log the URL used to do so.
\item
  Perform image queries against the SDSS Stripe 82 coadded and
  single-epoch image data. These queries should match ones in the API
  Aspect test notebook. Document the query results with screen shots.
\item
  Download an example of each type of image.
\item
  Using the public SDSS archive, attempt to retrieve corresponding
  images and visu- ally compare them.
\item
  Perform image cutout requests for 300 × 300 fields at the same targets
  used in the full-image queries.
\item
  Download the cutouts.
\end{enumerate}

\hypertarget{lvv-t61---verify-associate-sources-to-objects-dms-req-0034-implementation}{%
\subsection{\texorpdfstring{\href{https://jira.lsstcorp.org/secure/Tests.jspa\#/testCase/LVV-T61}{LVV-T61}
- Verify Associate Sources to Objects (DMS-REQ-0034)
implementation}{LVV-T61 - Verify Associate Sources to Objects (DMS-REQ-0034) implementation}}\label{lvv-t61---verify-associate-sources-to-objects-dms-req-0034-implementation}}

\begin{longtable}[]{@{}llllll@{}}
\toprule
Version & Status & Priority & Verification Type & Critical Event &
Owner\tabularnewline
\midrule
\endhead
1 & Draft & Normal & Test & False & ktl\tabularnewline
\bottomrule
\end{longtable}

\hypertarget{test-items-2}{%
\subsubsection{Test Items}\label{test-items-2}}

\hypertarget{requirements-4}{%
\subsubsection{Requirements}\label{requirements-4}}

\hypertarget{requirements-5}{%
\subsubsection{Requirements}\label{requirements-5}}

\begin{itemize}
\tightlist
\item
  \href{https://jira.lsstcorp.org/browse/LVV-16}{LVV-16} -
  DMS-REQ-0034-V-01: Associate Sources to Objects
\end{itemize}

\hypertarget{test-script-2}{%
\subsubsection{Test Script}\label{test-script-2}}

\textbf{Step 1}\\
The DM Stack shall be initialized using the loadLSST script (as
described in DRP-00-00).\\
~\\
\textbf{Step 2}\\
A ``Data Butler'' will be initialized to access the repository.\\
~\\
\textbf{Step 3}\\
For each of the expected data products types (listed in Test Items
section §4.3.2) and each of the expected units (PVIs, coadds, etc), the
data product will be retrieved from the Butler and verified to be
non-empty.\\
~\\
~\\
~\\
\textbf{Step 4}\\
Verify that sources have objects\\
~\\
\textbf{Step 5}\\
Verify that objects list sources that seem reasonably near them.\\
~\\

\hypertarget{lvv-t77---verify-best-seeing-coadds-dms-req-0330-implementation}{%
\subsection{\texorpdfstring{\href{https://jira.lsstcorp.org/secure/Tests.jspa\#/testCase/LVV-T77}{LVV-T77}
- Verify Best Seeing Coadds (DMS-REQ-0330)
implementation}{LVV-T77 - Verify Best Seeing Coadds (DMS-REQ-0330) implementation}}\label{lvv-t77---verify-best-seeing-coadds-dms-req-0330-implementation}}

\begin{longtable}[]{@{}llllll@{}}
\toprule
Version & Status & Priority & Verification Type & Critical Event &
Owner\tabularnewline
\midrule
\endhead
1 & Draft & Normal & Test & False & ktl\tabularnewline
\bottomrule
\end{longtable}

\hypertarget{test-items-3}{%
\subsubsection{Test Items}\label{test-items-3}}

\hypertarget{requirements-6}{%
\subsubsection{Requirements}\label{requirements-6}}

\hypertarget{requirements-7}{%
\subsubsection{Requirements}\label{requirements-7}}

\begin{itemize}
\tightlist
\item
  \href{https://jira.lsstcorp.org/browse/LVV-161}{LVV-161} -
  DMS-REQ-0330-V-01: Best Seeing Coadds
\end{itemize}

\hypertarget{test-script-3}{%
\subsubsection{Test Script}\label{test-script-3}}

\textbf{Step 1}\\
~\\
~\\
\textbf{Step 2}\\
The DM Stack shall be initialized using the loadLSST script (as
described in DRP-00-00).\\
~\\
\textbf{Step 3}\\
A ``Data Butler'' will be initialized to access the repository.\\
~\\
\textbf{Step 4}\\
For each of the expected data products types (listed in Test Items
section §4.3.2) and each of the expected units (PVIs, coadds, etc), the
data product will be retrieved from the Butler and verified to be
non-empty.\\
~\\
~\\
~\\
\textbf{Step 5}\\
Explicitly create a coadd for a specified seeing range in each filter.\\
~\\
\textbf{Step 6}\\
Verify that these coadds exist.\\
~\\

\hypertarget{lvv-t57---verify-calculating-ssobject-parameters-dms-req-0323-implementation}{%
\subsection{\texorpdfstring{\href{https://jira.lsstcorp.org/secure/Tests.jspa\#/testCase/LVV-T57}{LVV-T57}
- Verify Calculating SSObject Parameters (DMS-REQ-0323)
implementation}{LVV-T57 - Verify Calculating SSObject Parameters (DMS-REQ-0323) implementation}}\label{lvv-t57---verify-calculating-ssobject-parameters-dms-req-0323-implementation}}

\begin{longtable}[]{@{}llllll@{}}
\toprule
Version & Status & Priority & Verification Type & Critical Event &
Owner\tabularnewline
\midrule
\endhead
1 & Draft & Normal & Test & False & ebellm\tabularnewline
\bottomrule
\end{longtable}

\hypertarget{test-items-4}{%
\subsubsection{Test Items}\label{test-items-4}}

\hypertarget{requirements-8}{%
\subsubsection{Requirements}\label{requirements-8}}

\hypertarget{requirements-9}{%
\subsubsection{Requirements}\label{requirements-9}}

\begin{itemize}
\tightlist
\item
  \href{https://jira.lsstcorp.org/browse/LVV-154}{LVV-154} -
  DMS-REQ-0323-V-01: Calculating SSObject Parameters
\end{itemize}

\hypertarget{test-script-4}{%
\subsubsection{Test Script}\label{test-script-4}}

\textbf{Step 1}\\
Run MOPS\\
~\\
\textbf{Step 2}\\
~\\
~\\
\textbf{Step 3}\\
Load results\\
~\\
\textbf{Step 4}\\
Inspect SSOBject catalog\\
~\\
\textbf{Step 5}\\
Computer the phase angle, reduced and absolute asteroid magnitudes for
objects identified in SSObject Catalog\\
~\\

\hypertarget{lvv-t56---verify-characterizing-variability-dms-req-0319-implementation}{%
\subsection{\texorpdfstring{\href{https://jira.lsstcorp.org/secure/Tests.jspa\#/testCase/LVV-T56}{LVV-T56}
- Verify Characterizing Variability (DMS-REQ-0319)
implementation}{LVV-T56 - Verify Characterizing Variability (DMS-REQ-0319) implementation}}\label{lvv-t56---verify-characterizing-variability-dms-req-0319-implementation}}

\begin{longtable}[]{@{}llllll@{}}
\toprule
Version & Status & Priority & Verification Type & Critical Event &
Owner\tabularnewline
\midrule
\endhead
1 & Draft & Normal & Test & False & ebellm\tabularnewline
\bottomrule
\end{longtable}

\hypertarget{test-items-5}{%
\subsubsection{Test Items}\label{test-items-5}}

\hypertarget{requirements-10}{%
\subsubsection{Requirements}\label{requirements-10}}

\hypertarget{requirements-11}{%
\subsubsection{Requirements}\label{requirements-11}}

\begin{itemize}
\tightlist
\item
  \href{https://jira.lsstcorp.org/browse/LVV-150}{LVV-150} -
  DMS-REQ-0319-V-01: Characterizing Variability
\end{itemize}

\hypertarget{test-script-5}{%
\subsubsection{Test Script}\label{test-script-5}}

\textbf{Step 1}\\
The DM Stack and Alert Processing packaged shall be initialized as
described in LVT-T17 (AG-00-00).\\
~\\
\textbf{Step 2}\\
The alert generation processing will be executed using the verification
cluster:\\
\hspace*{0.333em} ~ ~ ~python
ap\_verify/bin/prepare\_demo\_slurm\_files.py\\
\hspace*{0.333em} ~ ~ ~\# At present we must run a single ccd+visit to
handle ingestion before\\
\hspace*{0.333em} ~ ~ ~\# parallel processing can begin\\
\hspace*{0.333em} ~ ~ ~./ap\_verify/bin/exec\_demo\_run\_1ccd.sh 410915
25\\
\hspace*{0.333em} ~ ~ ~ln -s ap\_verify/bin/demo\_run.sl\\
\hspace*{0.333em} ~ ~ ~ln -s ap\_verify/bin/demo\_cmds.conf\\
\hspace*{0.333em} ~ ~ ~sbatch demo\_run.sl\\
and any errors or failures reported.\\
~\\
\textbf{Step 3}\\
A ``Data Butler'' will be initialized to access the repository.\\
~\\
\textbf{Step 4}\\
For each of the expected data products types (listed in §4.2.2) and each
of the expected units (PVIs, catalogs, etc.), the data product will be
retrieved from the Butler and verified to be non-empty.\\
~\\
\textbf{Step 5}\\
DIAObjects are currently only stored in a database, without shims to the
Butler, so the existence of the database table and its non-empty
contents will be verified by directly accessing it using sqlite3 and
executing appropriate SQL queries.\\
~\\
\textbf{Step 6}\\
Verify that the issued alerts contain measurements during the
diaCharacterizationCutoff.\\
~\\

\hypertarget{lvv-t72---verify-coadd-image-method-constraints-dms-req-0278-implementation}{%
\subsection{\texorpdfstring{\href{https://jira.lsstcorp.org/secure/Tests.jspa\#/testCase/LVV-T72}{LVV-T72}
- Verify Coadd Image Method Constraints (DMS-REQ-0278)
implementation}{LVV-T72 - Verify Coadd Image Method Constraints (DMS-REQ-0278) implementation}}\label{lvv-t72---verify-coadd-image-method-constraints-dms-req-0278-implementation}}

\begin{longtable}[]{@{}llllll@{}}
\toprule
Version & Status & Priority & Verification Type & Critical Event &
Owner\tabularnewline
\midrule
\endhead
1 & Draft & Normal & Test & False & ktl\tabularnewline
\bottomrule
\end{longtable}

\hypertarget{test-items-6}{%
\subsubsection{Test Items}\label{test-items-6}}

\hypertarget{requirements-12}{%
\subsubsection{Requirements}\label{requirements-12}}

\hypertarget{requirements-13}{%
\subsubsection{Requirements}\label{requirements-13}}

\begin{itemize}
\tightlist
\item
  \href{https://jira.lsstcorp.org/browse/LVV-109}{LVV-109} -
  DMS-REQ-0278-V-01: Coadd Image Method Constraints
\end{itemize}

\hypertarget{test-script-6}{%
\subsubsection{Test Script}\label{test-script-6}}

\textbf{Step 1}\\
The DM Stack shall be initialized using the loadLSST script (as
described in DRP-00-00).\\
~\\
\textbf{Step 2}\\
A ``Data Butler'' will be initialized to access the repository.\\
~\\
\textbf{Step 3}\\
For each of the expected data products types (listed in Test Items
section §4.3.2) and each of the expected units (PVIs, coadds, etc), the
data product will be retrieved from the Butler and verified to be
non-empty.\\
~\\
~\\
~\\
\textbf{Step 4}\\
Verify that coadds were created following specification\\
~\\

\hypertarget{lvv-t70---verify-coadd-source-catalog-dms-req-0277-implementation}{%
\subsection{\texorpdfstring{\href{https://jira.lsstcorp.org/secure/Tests.jspa\#/testCase/LVV-T70}{LVV-T70}
- Verify Coadd Source Catalog (DMS-REQ-0277)
implementation}{LVV-T70 - Verify Coadd Source Catalog (DMS-REQ-0277) implementation}}\label{lvv-t70---verify-coadd-source-catalog-dms-req-0277-implementation}}

\begin{longtable}[]{@{}llllll@{}}
\toprule
Version & Status & Priority & Verification Type & Critical Event &
Owner\tabularnewline
\midrule
\endhead
1 & Draft & Normal & Test & False & jbosch\tabularnewline
\bottomrule
\end{longtable}

\hypertarget{test-items-7}{%
\subsubsection{Test Items}\label{test-items-7}}

\hypertarget{requirements-14}{%
\subsubsection{Requirements}\label{requirements-14}}

\hypertarget{requirements-15}{%
\subsubsection{Requirements}\label{requirements-15}}

\begin{itemize}
\tightlist
\item
  \href{https://jira.lsstcorp.org/browse/LVV-108}{LVV-108} -
  DMS-REQ-0277-V-01: Coadd Source Catalog
\end{itemize}

\hypertarget{test-script-7}{%
\subsubsection{Test Script}\label{test-script-7}}

\textbf{Step 1}\\
The DM Stack shall be initialized using the loadLSST script (as
described in DRP-00-00).\\
~\\
\textbf{Step 2}\\
A ``Data Butler'' will be initialized to access the repository.\\
~\\
\textbf{Step 3}\\
For each of the expected data products types (listed in Test Items
section §4.3.2) and each of the expected units (PVIs, coadds, etc), the
data product will be retrieved from the Butler and verified to be
non-empty.\\
~\\
~\\
~\\
\textbf{Step 4}\\
Verify that there exists a catalog of merged sources.\\
~\\

\hypertarget{lvv-t64---verify-coadded-image-provenance-dms-req-0106-implementation}{%
\subsection{\texorpdfstring{\href{https://jira.lsstcorp.org/secure/Tests.jspa\#/testCase/LVV-T64}{LVV-T64}
- Verify Coadded Image Provenance (DMS-REQ-0106)
implementation}{LVV-T64 - Verify Coadded Image Provenance (DMS-REQ-0106) implementation}}\label{lvv-t64---verify-coadded-image-provenance-dms-req-0106-implementation}}

\begin{longtable}[]{@{}llllll@{}}
\toprule
Version & Status & Priority & Verification Type & Critical Event &
Owner\tabularnewline
\midrule
\endhead
1 & Draft & Normal & Test & False & ktl\tabularnewline
\bottomrule
\end{longtable}

\hypertarget{test-items-8}{%
\subsubsection{Test Items}\label{test-items-8}}

\hypertarget{requirements-16}{%
\subsubsection{Requirements}\label{requirements-16}}

\hypertarget{requirements-17}{%
\subsubsection{Requirements}\label{requirements-17}}

\begin{itemize}
\tightlist
\item
  \href{https://jira.lsstcorp.org/browse/LVV-46}{LVV-46} -
  DMS-REQ-0106-V-01: Coadded Image Provenance
\end{itemize}

\hypertarget{test-script-8}{%
\subsubsection{Test Script}\label{test-script-8}}

\textbf{Step 1}\\
~\\
~\\
\textbf{Step 2}\\
The DM Stack shall be initialized using the loadLSST script (as
described in DRP-00-00).\\
~\\
\textbf{Step 3}\\
A ``Data Butler'' will be initialized to access the repository.\\
~\\
\textbf{Step 4}\\
For each of the expected data products types (listed in Test Items
section §4.3.2) and each of the expected units (PVIs, coadds, etc), the
data product will be retrieved from the Butler and verified to be
non-empty.\\
~\\
~\\
~\\
\textbf{Step 5}\\
Query and verify provenance of input images, and software versions that
went into producing stack.\\
~\\
\textbf{Step 6}\\
Test re-generating 10 different coadds tract+patches based on the
provenance image given\\
~\\

\hypertarget{lvv-t24---verify-computing-derived-quantities-dms-req-0331-implementation}{%
\subsection{\texorpdfstring{\href{https://jira.lsstcorp.org/secure/Tests.jspa\#/testCase/LVV-T24}{LVV-T24}
- Verify Computing Derived Quantities (DMS-REQ-0331)
Implementation}{LVV-T24 - Verify Computing Derived Quantities (DMS-REQ-0331) Implementation}}\label{lvv-t24---verify-computing-derived-quantities-dms-req-0331-implementation}}

\begin{longtable}[]{@{}llllll@{}}
\toprule
Version & Status & Priority & Verification Type & Critical Event &
Owner\tabularnewline
\midrule
\endhead
1 & Draft & Normal & Test & False & lguy\tabularnewline
\bottomrule
\end{longtable}

\hypertarget{test-items-9}{%
\subsubsection{Test Items}\label{test-items-9}}

Common derived quantities shall be made available to end-users by either
providing pre-computed columns or providing functions that can be used
dynamically in queries. These should at least include the ability to
calculate the reduced chi-squared of fitted models and make it as easy
as possible to calculate color-color diagrams.

\hypertarget{requirements-18}{%
\subsubsection{Requirements}\label{requirements-18}}

\hypertarget{requirements-19}{%
\subsubsection{Requirements}\label{requirements-19}}

\begin{itemize}
\tightlist
\item
  \href{https://jira.lsstcorp.org/browse/LVV-162}{LVV-162} -
  DMS-REQ-0331-V-01: Computing Derived Quantities
\end{itemize}

\hypertarget{test-script-9}{%
\subsubsection{Test Script}\label{test-script-9}}

\textbf{Step 1}\\
The DM Stack shall be initialized using the loadLSST script (as
described in DRP-00-00).\\
~\\
\textbf{Step 2}\\
A ``Data Butler'' will be initialized to access the repository.\\
~\\
\textbf{Step 3}\\
For each of the expected data products types (listed in Test Items
section §4.3.2) and each of the expected units (PVIs, coadds, etc), the
data product will be retrieved from the Butler and verified to be
non-empty.\\
~\\
~\\
~\\
\textbf{Step 4}\\
Load into DPDD+Science Platform\\
~\\
\textbf{Step 5}\\
Constructing color-color diagram and and fitting stellar locus in
Science Platform.\\
~\\
\textbf{Step 6}\\
Invite three members of commissioning team to create color-color diagram
from coadd catalogs based on merged coadd reference catalog.\\
~\\

\hypertarget{lvv-t55---verify-diaforcedsource-catalog-dms-req-0317-implementation}{%
\subsection{\texorpdfstring{\href{https://jira.lsstcorp.org/secure/Tests.jspa\#/testCase/LVV-T55}{LVV-T55}
- Verify DIAForcedSource Catalog (DMS-REQ-0317)
implementation}{LVV-T55 - Verify DIAForcedSource Catalog (DMS-REQ-0317) implementation}}\label{lvv-t55---verify-diaforcedsource-catalog-dms-req-0317-implementation}}

\begin{longtable}[]{@{}llllll@{}}
\toprule
Version & Status & Priority & Verification Type & Critical Event &
Owner\tabularnewline
\midrule
\endhead
1 & Draft & Normal & Test & False & ebellm\tabularnewline
\bottomrule
\end{longtable}

\hypertarget{test-items-10}{%
\subsubsection{Test Items}\label{test-items-10}}

\hypertarget{requirements-20}{%
\subsubsection{Requirements}\label{requirements-20}}

\hypertarget{requirements-21}{%
\subsubsection{Requirements}\label{requirements-21}}

\begin{itemize}
\tightlist
\item
  \href{https://jira.lsstcorp.org/browse/LVV-148}{LVV-148} -
  DMS-REQ-0317-V-01: DIAForcedSource Catalog
\end{itemize}

\hypertarget{test-script-10}{%
\subsubsection{Test Script}\label{test-script-10}}

\textbf{Step 1}\\
The DM Stack and Alert Processing packaged shall be initialized as
described in LVT-T17 (AG-00-00).\\
~\\
\textbf{Step 2}\\
The alert generation processing will be executed using the verification
cluster:\\
\hspace*{0.333em} ~ ~ ~python
ap\_verify/bin/prepare\_demo\_slurm\_files.py\\
\hspace*{0.333em} ~ ~ ~\# At present we must run a single ccd+visit to
handle ingestion before\\
\hspace*{0.333em} ~ ~ ~\# parallel processing can begin\\
\hspace*{0.333em} ~ ~ ~./ap\_verify/bin/exec\_demo\_run\_1ccd.sh 410915
25\\
\hspace*{0.333em} ~ ~ ~ln -s ap\_verify/bin/demo\_run.sl\\
\hspace*{0.333em} ~ ~ ~ln -s ap\_verify/bin/demo\_cmds.conf\\
\hspace*{0.333em} ~ ~ ~sbatch demo\_run.sl\\
and any errors or failures reported.\\
~\\
\textbf{Step 3}\\
A ``Data Butler'' will be initialized to access the repository.\\
~\\
\textbf{Step 4}\\
For each of the expected data products types (listed in §4.2.2) and each
of the expected units (PVIs, catalogs, etc.), the data product will be
retrieved from the Butler and verified to be non-empty.\\
~\\
\textbf{Step 5}\\
DIAObjects are currently only stored in a database, without shims to the
Butler, so the existence of the database table and its non-empty
contents will be verified by directly accessing it using sqlite3 and
executing appropriate SQL queries.\\
~\\

\hypertarget{lvv-t51---verify-diaobject-catalog-dms-req-0271-implementation}{%
\subsection{\texorpdfstring{\href{https://jira.lsstcorp.org/secure/Tests.jspa\#/testCase/LVV-T51}{LVV-T51}
- Verify DIAObject Catalog (DMS-REQ-0271)
Implementation}{LVV-T51 - Verify DIAObject Catalog (DMS-REQ-0271) Implementation}}\label{lvv-t51---verify-diaobject-catalog-dms-req-0271-implementation}}

\begin{longtable}[]{@{}llllll@{}}
\toprule
Version & Status & Priority & Verification Type & Critical Event &
Owner\tabularnewline
\midrule
\endhead
1 & Draft & Normal & Test & False & ebellm\tabularnewline
\bottomrule
\end{longtable}

\hypertarget{test-items-11}{%
\subsubsection{Test Items}\label{test-items-11}}

\hypertarget{requirements-22}{%
\subsubsection{Requirements}\label{requirements-22}}

\hypertarget{requirements-23}{%
\subsubsection{Requirements}\label{requirements-23}}

\begin{itemize}
\tightlist
\item
  \href{https://jira.lsstcorp.org/browse/LVV-102}{LVV-102} -
  DMS-REQ-0271-V-01: DIAObject Catalog
\end{itemize}

\hypertarget{test-script-11}{%
\subsubsection{Test Script}\label{test-script-11}}

\textbf{Step 1}\\
The DM Stack shall be initialized using the loadLSST script (as
described in LVV-T17 - AG-00-00).\\
~\\
\textbf{Step 2}\\
A ``Data Butler'' will be initialized to access the repository.\\
~\\
\textbf{Step 3}\\
DIASource records will be accessed by querying the Butler, then examined
interactively at a Python prompt.\\
~\\
\textbf{Step 4}\\
The DM Stack and Alert Processing packaged shall be initialized as
described in LVT-T17 (AG-00-00).\\
~\\
\textbf{Step 5}\\
The alert generation processing will be executed using the verification
cluster:\\
\hspace*{0.333em} ~ ~ ~python
ap\_verify/bin/prepare\_demo\_slurm\_files.py\\
\hspace*{0.333em} ~ ~ ~\# At present we must run a single ccd+visit to
handle ingestion before\\
\hspace*{0.333em} ~ ~ ~\# parallel processing can begin\\
\hspace*{0.333em} ~ ~ ~./ap\_verify/bin/exec\_demo\_run\_1ccd.sh 410915
25\\
\hspace*{0.333em} ~ ~ ~ln -s ap\_verify/bin/demo\_run.sl\\
\hspace*{0.333em} ~ ~ ~ln -s ap\_verify/bin/demo\_cmds.conf\\
\hspace*{0.333em} ~ ~ ~sbatch demo\_run.sl\\
and any errors or failures reported.\\
~\\
\textbf{Step 6}\\
A ``Data Butler'' will be initialized to access the repository.\\
~\\
\textbf{Step 7}\\
For each of the expected data products types (listed in §4.2.2) and each
of the expected units (PVIs, catalogs, etc.), the data product will be
retrieved from the Butler and verified to be non-empty.\\
~\\
\textbf{Step 8}\\
DIAObjects are currently only stored in a database, without shims to the
Butler, so the existence of the database table and its non-empty
contents will be verified by directly accessing it using sqlite3 and
executing appropriate SQL queries.\\
~\\
\textbf{Step 9}\\
The DM Stack shall be initialized using the loadLSST script (as
described in LVV-T17 - AG-00-00).\\
~\\
\textbf{Step 10}\\
sqlite3 or Python's sqlalchemy module will be used to access the Level 1
database.\\
~\\
\textbf{Step 11}\\
Verify that DIAObjects have diaNearbyObjMaxStar and
diaNearbyObjMaxGalaxies that point to the Object catalog and are within
dianNearbyObjRadius; the probability of association; and the required
DIAObject properties.\\
~\\

\hypertarget{lvv-t49---verify-diasource-catalog-dms-req-0269-implementation}{%
\subsection{\texorpdfstring{\href{https://jira.lsstcorp.org/secure/Tests.jspa\#/testCase/LVV-T49}{LVV-T49}
- Verify DIASource Catalog (DMS-REQ-0269)
Implementation}{LVV-T49 - Verify DIASource Catalog (DMS-REQ-0269) Implementation}}\label{lvv-t49---verify-diasource-catalog-dms-req-0269-implementation}}

\begin{longtable}[]{@{}llllll@{}}
\toprule
Version & Status & Priority & Verification Type & Critical Event &
Owner\tabularnewline
\midrule
\endhead
1 & Draft & Normal & Test & False & ebellm\tabularnewline
\bottomrule
\end{longtable}

\hypertarget{test-items-12}{%
\subsubsection{Test Items}\label{test-items-12}}

\hypertarget{requirements-24}{%
\subsubsection{Requirements}\label{requirements-24}}

\hypertarget{requirements-25}{%
\subsubsection{Requirements}\label{requirements-25}}

\begin{itemize}
\tightlist
\item
  \href{https://jira.lsstcorp.org/browse/LVV-100}{LVV-100} -
  DMS-REQ-0269-V-01: DIASource Catalog
\end{itemize}

\hypertarget{test-script-12}{%
\subsubsection{Test Script}\label{test-script-12}}

\textbf{Step 1}\\
Verify that products are produced for DIASource catalog\\
~\\
\textbf{Step 2}\\
The DM Stack and Alert Processing packaged shall be initialized as
described in LVT-T17 (AG-00-00).\\
~\\
\textbf{Step 3}\\
The alert generation processing will be executed using the verification
cluster:\\
\hspace*{0.333em} ~ ~ ~python
ap\_verify/bin/prepare\_demo\_slurm\_files.py\\
\hspace*{0.333em} ~ ~ ~\# At present we must run a single ccd+visit to
handle ingestion before\\
\hspace*{0.333em} ~ ~ ~\# parallel processing can begin\\
\hspace*{0.333em} ~ ~ ~./ap\_verify/bin/exec\_demo\_run\_1ccd.sh 410915
25\\
\hspace*{0.333em} ~ ~ ~ln -s ap\_verify/bin/demo\_run.sl\\
\hspace*{0.333em} ~ ~ ~ln -s ap\_verify/bin/demo\_cmds.conf\\
\hspace*{0.333em} ~ ~ ~sbatch demo\_run.sl\\
and any errors or failures reported.\\
~\\
\textbf{Step 4}\\
A ``Data Butler'' will be initialized to access the repository.\\
~\\
\textbf{Step 5}\\
For each of the expected data products types (listed in §4.2.2) and each
of the expected units (PVIs, catalogs, etc.), the data product will be
retrieved from the Butler and verified to be non-empty.\\
~\\
\textbf{Step 6}\\
DIAObjects are currently only stored in a database, without shims to the
Butler, so the existence of the database table and its non-empty
contents will be verified by directly accessing it using sqlite3 and
executing appropriate SQL queries.\\
~\\

\hypertarget{lvv-t73---verify-deep-detection-coadds-dms-req-0279-implementation}{%
\subsection{\texorpdfstring{\href{https://jira.lsstcorp.org/secure/Tests.jspa\#/testCase/LVV-T73}{LVV-T73}
- Verify Deep Detection Coadds (DMS-REQ-0279)
implementation}{LVV-T73 - Verify Deep Detection Coadds (DMS-REQ-0279) implementation}}\label{lvv-t73---verify-deep-detection-coadds-dms-req-0279-implementation}}

\begin{longtable}[]{@{}llllll@{}}
\toprule
Version & Status & Priority & Verification Type & Critical Event &
Owner\tabularnewline
\midrule
\endhead
1 & Draft & Normal & Test & False & ktl\tabularnewline
\bottomrule
\end{longtable}

\hypertarget{test-items-13}{%
\subsubsection{Test Items}\label{test-items-13}}

\hypertarget{requirements-26}{%
\subsubsection{Requirements}\label{requirements-26}}

\hypertarget{requirements-27}{%
\subsubsection{Requirements}\label{requirements-27}}

\begin{itemize}
\tightlist
\item
  \href{https://jira.lsstcorp.org/browse/LVV-110}{LVV-110} -
  DMS-REQ-0279-V-01: Deep Detection Coadds
\end{itemize}

\hypertarget{test-script-13}{%
\subsubsection{Test Script}\label{test-script-13}}

\textbf{Step 1}\\
The DM Stack shall be initialized using the loadLSST script (as
described in DRP-00-00).\\
~\\
\textbf{Step 2}\\
A ``Data Butler'' will be initialized to access the repository.\\
~\\
\textbf{Step 3}\\
For each of the expected data products types (listed in Test Items
section §4.3.2) and each of the expected units (PVIs, coadds, etc), the
data product will be retrieved from the Butler and verified to be
non-empty.\\
~\\
~\\
~\\
\textbf{Step 4}\\
Verify through inspection that per-filter coadds exist for each
tract+patch possible\\
~\\
\textbf{Step 5}\\
Verify through inspection that the images used to generate those coadds
met specified conditions\\
~\\
\textbf{Step 6}\\
Visually inspect a subset of the coadds to verify that they visually
appear reasonable and to be from good quality data.\\
~\\

\hypertarget{lvv-t25---verify-denormalizing-database-tables-dms-req-0332-implementation}{%
\subsection{\texorpdfstring{\href{https://jira.lsstcorp.org/secure/Tests.jspa\#/testCase/LVV-T25}{LVV-T25}
- Verify Denormalizing Database Tables (DMS-REQ-0332)
Implementation}{LVV-T25 - Verify Denormalizing Database Tables (DMS-REQ-0332) Implementation}}\label{lvv-t25---verify-denormalizing-database-tables-dms-req-0332-implementation}}

\begin{longtable}[]{@{}llllll@{}}
\toprule
Version & Status & Priority & Verification Type & Critical Event &
Owner\tabularnewline
\midrule
\endhead
1 & Draft & Normal & Test & False & lguy\tabularnewline
\bottomrule
\end{longtable}

\hypertarget{test-items-14}{%
\subsubsection{Test Items}\label{test-items-14}}

The database tables shall contain views presented to the users that will
be appropriately denormalized for ease of use.

\hypertarget{requirements-28}{%
\subsubsection{Requirements}\label{requirements-28}}

\hypertarget{requirements-29}{%
\subsubsection{Requirements}\label{requirements-29}}

\begin{itemize}
\tightlist
\item
  \href{https://jira.lsstcorp.org/browse/LVV-163}{LVV-163} -
  DMS-REQ-0332-V-01: Denormalizing Database Tables
\end{itemize}

\hypertarget{test-script-14}{%
\subsubsection{Test Script}\label{test-script-14}}

\textbf{Step 1}\\
The DM Stack shall be initialized using the loadLSST script (as
described in DRP-00-00).\\
~\\
\textbf{Step 2}\\
A ``Data Butler'' will be initialized to access the repository.\\
~\\
\textbf{Step 3}\\
For each of the expected data products types (listed in Test Items
section §4.3.2) and each of the expected units (PVIs, coadds, etc), the
data product will be retrieved from the Butler and verified to be
non-empty.\\
~\\
~\\
~\\
\textbf{Step 4}\\
Log host and networking details of the client host to be used. Log the
Web browser version to be used.\\
~\\
\textbf{Step 5}\\
Establish VPN connectivity to the PDAC at NCSA.\\
~\\
\textbf{Step 6}\\
Manually perform, and log, the following steps against the Portal
Aspect:

\begin{enumerate}
\tightlist
\item
  Navigate to the PDAC Portal. Log the URL used to do so.
\item
  Perform a cone search around (ra=0,dec=0), radius 300 arcseconds, in
  each of the Object-like, ForcedSource-like, and a Source-like catalog.
  Choose a row from each search and record the primary key value for
  each for later use. Take screen shots of the search form and of the
  results of the three searches. Record the wall clock time required for
  the searches, if long enough to measure.
\item
  Perform a multi-object cone search based on the coordinates in the
  file LDM-540/test\_scripts/lsp-00-15.coords in the Object-like table.
  Take screen shots of the search form and of the results of the search.
  Record the wall clock time required for the searches, if long enough
  to measure.
\item
  Perform a search on each of the Object-like, ForcedSource-like, and
  Source-like catalogs for the IDs previously saved. Confirm that each
  search is successful and re- turns the same information as in the
  original search from which the ID was taken. Perform a search on the
  ForcedSource-like catalog using the ID from the Object- like catalog.
  Confirm that a time series of measurements of that object in multiple
  epochs is returned. Take screen shots of the search forms and of the
  results of the searches. Record the wall clock time required for the
  searches, if long enough to measure.
\item
  On each of the Object-like catalog and a Source-like catalog, by
  performing searches over small regions of sky and exploring the
  results, choose a set of attributes and search parameters which should
  select a relatively small number of rows (\textless{} 100, 000) when
  applied to the entire sky. This may require some iterative
  experimentation at increasingly larger scales. Take screen shots of
  the final search forms and of the results of the searches. Record the
  wall clock time required for the searches, if long enough to measure.
\end{enumerate}

\textbf{Step 7}\\
{Take 20 sampled queries and determine which are easily done on views
and which are complicated joins. ~Discuss the complicated ones and
determine if the design is easily changed or if there are clear choices
about which queries will be more complicated.}\\
~\\

\hypertarget{lvv-t71---verify-detecting-extended-low-surface-brightness-objects-dms-req-0349-implementation}{%
\subsection{\texorpdfstring{\href{https://jira.lsstcorp.org/secure/Tests.jspa\#/testCase/LVV-T71}{LVV-T71}
- Verify Detecting extended low surface brightness objects
(DMS-REQ-0349)
implementation}{LVV-T71 - Verify Detecting extended low surface brightness objects (DMS-REQ-0349) implementation}}\label{lvv-t71---verify-detecting-extended-low-surface-brightness-objects-dms-req-0349-implementation}}

\begin{longtable}[]{@{}llllll@{}}
\toprule
Version & Status & Priority & Verification Type & Critical Event &
Owner\tabularnewline
\midrule
\endhead
1 & Draft & Normal & Test & False & jbosch\tabularnewline
\bottomrule
\end{longtable}

\hypertarget{test-items-15}{%
\subsubsection{Test Items}\label{test-items-15}}

\hypertarget{requirements-30}{%
\subsubsection{Requirements}\label{requirements-30}}

\hypertarget{requirements-31}{%
\subsubsection{Requirements}\label{requirements-31}}

\begin{itemize}
\tightlist
\item
  \href{https://jira.lsstcorp.org/browse/LVV-180}{LVV-180} -
  DMS-REQ-0349-V-01: Detecting extended low surface brightness objects
\end{itemize}

\hypertarget{precondition-1}{%
\subsubsection{Precondition}\label{precondition-1}}

Input Data\\
\hspace*{0.333em}-\/- HSC Public Data Release

\hypertarget{test-script-15}{%
\subsubsection{Test Script}\label{test-script-15}}

\textbf{Step 1}\\
Verify that low surface brightness objects exist. ~I.e., objects at SNR
\textasciitilde{} 10 with clear extendedness.\\
~\\
\textbf{Step 2}\\
The DM Stack shall be initialized using the loadLSST script (as
described in DRP-00-00).\\
~\\
\textbf{Step 3}\\
A ``Data Butler'' will be initialized to access the repository.\\
~\\
\textbf{Step 4}\\
For each of the expected data products types (listed in Test Items
section §4.3.2) and each of the expected units (PVIs, coadds, etc), the
data product will be retrieved from the Butler and verified to be
non-empty.\\
~\\
~\\
~\\

\hypertarget{lvv-t37---verify-difference-exposure-attributes-dms-req-0074-implementation}{%
\subsection{\texorpdfstring{\href{https://jira.lsstcorp.org/secure/Tests.jspa\#/testCase/LVV-T37}{LVV-T37}
- Verify Difference Exposure Attributes (DMS-REQ-0074)
Implementation}{LVV-T37 - Verify Difference Exposure Attributes (DMS-REQ-0074) Implementation}}\label{lvv-t37---verify-difference-exposure-attributes-dms-req-0074-implementation}}

\begin{longtable}[]{@{}llllll@{}}
\toprule
Version & Status & Priority & Verification Type & Critical Event &
Owner\tabularnewline
\midrule
\endhead
1 & Draft & Normal & Test & False & ebellm\tabularnewline
\bottomrule
\end{longtable}

\hypertarget{test-items-16}{%
\subsubsection{Test Items}\label{test-items-16}}

For each Difference Exposure, the DMS shall store: the identify of the
input exposures and related provenance information, and a set of
metadata attributes including at least a representation of the PSF
matching kernel used in the differencing

\hypertarget{requirements-32}{%
\subsubsection{Requirements}\label{requirements-32}}

\hypertarget{requirements-33}{%
\subsubsection{Requirements}\label{requirements-33}}

\begin{itemize}
\tightlist
\item
  \href{https://jira.lsstcorp.org/browse/LVV-32}{LVV-32} -
  DMS-REQ-0074-V-01: Difference Exposure Attributes
\end{itemize}

\hypertarget{test-script-16}{%
\subsubsection{Test Script}\label{test-script-16}}

\textbf{Step 1}\\
The DM Stack and Alert Processing packaged shall be initialized as
described in LVT-T17 (AG-00-00).\\
~\\
\textbf{Step 2}\\
The alert generation processing will be executed using the verification
cluster:\\
\hspace*{0.333em} ~ ~ ~python
ap\_verify/bin/prepare\_demo\_slurm\_files.py\\
\hspace*{0.333em} ~ ~ ~\# At present we must run a single ccd+visit to
handle ingestion before\\
\hspace*{0.333em} ~ ~ ~\# parallel processing can begin\\
\hspace*{0.333em} ~ ~ ~./ap\_verify/bin/exec\_demo\_run\_1ccd.sh 410915
25\\
\hspace*{0.333em} ~ ~ ~ln -s ap\_verify/bin/demo\_run.sl\\
\hspace*{0.333em} ~ ~ ~ln -s ap\_verify/bin/demo\_cmds.conf\\
\hspace*{0.333em} ~ ~ ~sbatch demo\_run.sl\\
and any errors or failures reported.\\
~\\
\textbf{Step 3}\\
A ``Data Butler'' will be initialized to access the repository.\\
~\\
\textbf{Step 4}\\
For each of the expected data products types (listed in §4.2.2) and each
of the expected units (PVIs, catalogs, etc.), the data product will be
retrieved from the Butler and verified to be non-empty.\\
~\\
\textbf{Step 5}\\
DIAObjects are currently only stored in a database, without shims to the
Butler, so the existence of the database table and its non-empty
contents will be verified by directly accessing it using sqlite3 and
executing appropriate SQL queries.\\
~\\
\textbf{Step 6}\\
For each of HSC PDR and DECAM HiTS data: set up three different
templates and run subtractions on 10 different images from at least two
different filters. ~Verify that we can recover the provenance
information about which template was used for each subtraction, which
input images were used for that template, and that we can successfull
extract the PSF matching kernel.\\
~\\

\hypertarget{lvv-t36---verify-difference-exposures-dms-req-0010-implementation}{%
\subsection{\texorpdfstring{\href{https://jira.lsstcorp.org/secure/Tests.jspa\#/testCase/LVV-T36}{LVV-T36}
- Verify Difference Exposures (DMS-REQ-0010)
Implementation}{LVV-T36 - Verify Difference Exposures (DMS-REQ-0010) Implementation}}\label{lvv-t36---verify-difference-exposures-dms-req-0010-implementation}}

\begin{longtable}[]{@{}llllll@{}}
\toprule
Version & Status & Priority & Verification Type & Critical Event &
Owner\tabularnewline
\midrule
\endhead
1 & Draft & Normal & Test & False & ebellm\tabularnewline
\bottomrule
\end{longtable}

\hypertarget{test-items-17}{%
\subsubsection{Test Items}\label{test-items-17}}

The DMS shall create a Difference Exposure from each Processed Visit
Image by subtracting a re-projected, scaled, PSF-matched Template Image
in the same passband.

\hypertarget{requirements-34}{%
\subsubsection{Requirements}\label{requirements-34}}

\hypertarget{requirements-35}{%
\subsubsection{Requirements}\label{requirements-35}}

\begin{itemize}
\tightlist
\item
  \href{https://jira.lsstcorp.org/browse/LVV-7}{LVV-7} -
  DMS-REQ-0010-V-01: Difference Exposures
\end{itemize}

\hypertarget{test-script-17}{%
\subsubsection{Test Script}\label{test-script-17}}

\textbf{Step 1}\\
The DM Stack and Alert Processing packaged shall be initialized as
described in LVT-T17 (AG-00-00).\\
~\\
\textbf{Step 2}\\
The alert generation processing will be executed using the verification
cluster:\\
\hspace*{0.333em} ~ ~ ~python
ap\_verify/bin/prepare\_demo\_slurm\_files.py\\
\hspace*{0.333em} ~ ~ ~\# At present we must run a single ccd+visit to
handle ingestion before\\
\hspace*{0.333em} ~ ~ ~\# parallel processing can begin\\
\hspace*{0.333em} ~ ~ ~./ap\_verify/bin/exec\_demo\_run\_1ccd.sh 410915
25\\
\hspace*{0.333em} ~ ~ ~ln -s ap\_verify/bin/demo\_run.sl\\
\hspace*{0.333em} ~ ~ ~ln -s ap\_verify/bin/demo\_cmds.conf\\
\hspace*{0.333em} ~ ~ ~sbatch demo\_run.sl\\
and any errors or failures reported.\\
~\\
\textbf{Step 3}\\
A ``Data Butler'' will be initialized to access the repository.\\
~\\
\textbf{Step 4}\\
For each of the expected data products types (listed in §4.2.2) and each
of the expected units (PVIs, catalogs, etc.), the data product will be
retrieved from the Butler and verified to be non-empty.\\
~\\
\textbf{Step 5}\\
DIAObjects are currently only stored in a database, without shims to the
Butler, so the existence of the database table and its non-empty
contents will be verified by directly accessing it using sqlite3 and
executing appropriate SQL queries.\\
~\\
\textbf{Step 6}\\
Demonstrate successful creation of a template image from HSC PDF and
DECAM HiTS data. ~Demonstrate successful creation of a Difference
Exposure for at least 10 other images from survey, ideally at a range of
arimass. ~In particular, HiTS has 2013A u-band data. ~While the Blanco
4-m does have an ADC, there are still some chromatic effects and we
should demonstrate that we can successfully produce Difference Exposures
and templates for diferent airmass bins.\\
~\\

\hypertarget{lvv-t48---verify-exposure-catalog-dms-req-0266-implementation}{%
\subsection{\texorpdfstring{\href{https://jira.lsstcorp.org/secure/Tests.jspa\#/testCase/LVV-T48}{LVV-T48}
- Verify Exposure Catalog (DMS-REQ-0266)
Implementation}{LVV-T48 - Verify Exposure Catalog (DMS-REQ-0266) Implementation}}\label{lvv-t48---verify-exposure-catalog-dms-req-0266-implementation}}

\begin{longtable}[]{@{}llllll@{}}
\toprule
Version & Status & Priority & Verification Type & Critical Event &
Owner\tabularnewline
\midrule
\endhead
1 & Draft & Normal & Test & False & jbosch\tabularnewline
\bottomrule
\end{longtable}

\hypertarget{test-items-18}{%
\subsubsection{Test Items}\label{test-items-18}}

\hypertarget{requirements-36}{%
\subsubsection{Requirements}\label{requirements-36}}

\hypertarget{requirements-37}{%
\subsubsection{Requirements}\label{requirements-37}}

\begin{itemize}
\tightlist
\item
  \href{https://jira.lsstcorp.org/browse/LVV-97}{LVV-97} -
  DMS-REQ-0266-V-01: Exposure Catalog
\end{itemize}

\hypertarget{test-script-18}{%
\subsubsection{Test Script}\label{test-script-18}}

\textbf{Step 1}\\
Verify that Exposure Catalogs contained required elements\\
~\\
\textbf{Step 2}\\
The DM Stack shall be initialized using the loadLSST script (as
described in DRP-00-00).\\
~\\
\textbf{Step 3}\\
A ``Data Butler'' will be initialized to access the repository.\\
~\\
\textbf{Step 4}\\
For each of the expected data products types (listed in Test Items
section §4.3.2) and each of the expected units (PVIs, coadds, etc), the
data product will be retrieved from the Butler and verified to be
non-empty.\\
~\\
~\\
~\\

\hypertarget{lvv-t50---verify-faint-diasource-measurements-dms-req-0270-implementation}{%
\subsection{\texorpdfstring{\href{https://jira.lsstcorp.org/secure/Tests.jspa\#/testCase/LVV-T50}{LVV-T50}
- Verify Faint DIASource Measurements (DMS-REQ-0270)
Implementation}{LVV-T50 - Verify Faint DIASource Measurements (DMS-REQ-0270) Implementation}}\label{lvv-t50---verify-faint-diasource-measurements-dms-req-0270-implementation}}

\begin{longtable}[]{@{}llllll@{}}
\toprule
Version & Status & Priority & Verification Type & Critical Event &
Owner\tabularnewline
\midrule
\endhead
1 & Draft & Normal & Test & False & ebellm\tabularnewline
\bottomrule
\end{longtable}

\hypertarget{test-items-19}{%
\subsubsection{Test Items}\label{test-items-19}}

\hypertarget{requirements-38}{%
\subsubsection{Requirements}\label{requirements-38}}

\hypertarget{requirements-39}{%
\subsubsection{Requirements}\label{requirements-39}}

\begin{itemize}
\tightlist
\item
  \href{https://jira.lsstcorp.org/browse/LVV-101}{LVV-101} -
  DMS-REQ-0270-V-01: Faint DIASource Measurements
\end{itemize}

\hypertarget{precondition-2}{%
\subsubsection{Precondition}\label{precondition-2}}

Input Data\\
\hspace*{0.333em}DECam HiTS data.

\hypertarget{test-script-19}{%
\subsubsection{Test Script}\label{test-script-19}}

\textbf{Step 1}\\
The DM Stack and Alert Processing packaged shall be initialized as
described in LVT-T17 (AG-00-00).\\
~\\
\textbf{Step 2}\\
The alert generation processing will be executed using the verification
cluster:\\
\hspace*{0.333em} ~ ~ ~python
ap\_verify/bin/prepare\_demo\_slurm\_files.py\\
\hspace*{0.333em} ~ ~ ~\# At present we must run a single ccd+visit to
handle ingestion before\\
\hspace*{0.333em} ~ ~ ~\# parallel processing can begin\\
\hspace*{0.333em} ~ ~ ~./ap\_verify/bin/exec\_demo\_run\_1ccd.sh 410915
25\\
\hspace*{0.333em} ~ ~ ~ln -s ap\_verify/bin/demo\_run.sl\\
\hspace*{0.333em} ~ ~ ~ln -s ap\_verify/bin/demo\_cmds.conf\\
\hspace*{0.333em} ~ ~ ~sbatch demo\_run.sl\\
and any errors or failures reported.\\
~\\
\textbf{Step 3}\\
A ``Data Butler'' will be initialized to access the repository.\\
~\\
\textbf{Step 4}\\
For each of the expected data products types (listed in §4.2.2) and each
of the expected units (PVIs, catalogs, etc.), the data product will be
retrieved from the Butler and verified to be non-empty.\\
~\\
\textbf{Step 5}\\
DIAObjects are currently only stored in a database, without shims to the
Butler, so the existence of the database table and its non-empty
contents will be verified by directly accessing it using sqlite3 and
executing appropriate SQL queries.\\
~\\
\textbf{Step 6}\\
As an example of selecting with constrains, Re-run source detection as
an afterburner to select isolated sources (defined as more than 2
arcseconds away from any other objects in the single-image-depth
catalog) that are fainter than the fiducial transSNR cut.\\
~\\

\hypertarget{lvv-t66---verify-forced-source-catalog-dms-req-0268-implementation}{%
\subsection{\texorpdfstring{\href{https://jira.lsstcorp.org/secure/Tests.jspa\#/testCase/LVV-T66}{LVV-T66}
- Verify Forced-Source Catalog (DMS-REQ-0268)
implementation}{LVV-T66 - Verify Forced-Source Catalog (DMS-REQ-0268) implementation}}\label{lvv-t66---verify-forced-source-catalog-dms-req-0268-implementation}}

\begin{longtable}[]{@{}llllll@{}}
\toprule
Version & Status & Priority & Verification Type & Critical Event &
Owner\tabularnewline
\midrule
\endhead
1 & Draft & Normal & Test & False & jbosch\tabularnewline
\bottomrule
\end{longtable}

\hypertarget{test-items-20}{%
\subsubsection{Test Items}\label{test-items-20}}

\hypertarget{requirements-40}{%
\subsubsection{Requirements}\label{requirements-40}}

\hypertarget{requirements-41}{%
\subsubsection{Requirements}\label{requirements-41}}

\begin{itemize}
\tightlist
\item
  \href{https://jira.lsstcorp.org/browse/LVV-99}{LVV-99} -
  DMS-REQ-0268-V-01: Forced-Source Catalog
\end{itemize}

\hypertarget{test-script-20}{%
\subsubsection{Test Script}\label{test-script-20}}

\textbf{Step 1}\\
The DM Stack and Alert Processing packaged shall be initialized as
described in LVT-T17 (AG-00-00).\\
~\\
\textbf{Step 2}\\
The alert generation processing will be executed using the verification
cluster:\\
\hspace*{0.333em} ~ ~ ~python
ap\_verify/bin/prepare\_demo\_slurm\_files.py\\
\hspace*{0.333em} ~ ~ ~\# At present we must run a single ccd+visit to
handle ingestion before\\
\hspace*{0.333em} ~ ~ ~\# parallel processing can begin\\
\hspace*{0.333em} ~ ~ ~./ap\_verify/bin/exec\_demo\_run\_1ccd.sh 410915
25\\
\hspace*{0.333em} ~ ~ ~ln -s ap\_verify/bin/demo\_run.sl\\
\hspace*{0.333em} ~ ~ ~ln -s ap\_verify/bin/demo\_cmds.conf\\
\hspace*{0.333em} ~ ~ ~sbatch demo\_run.sl\\
and any errors or failures reported.\\
~\\
\textbf{Step 3}\\
A ``Data Butler'' will be initialized to access the repository.\\
~\\
\textbf{Step 4}\\
For each of the expected data products types (listed in §4.2.2) and each
of the expected units (PVIs, catalogs, etc.), the data product will be
retrieved from the Butler and verified to be non-empty.\\
~\\
\textbf{Step 5}\\
DIAObjects are currently only stored in a database, without shims to the
Butler, so the existence of the database table and its non-empty
contents will be verified by directly accessing it using sqlite3 and
executing appropriate SQL queries.\\
~\\
\textbf{Step 6}\\
The DM Stack shall be initialized using the loadLSST script (as
described in DRP-00-00).\\
~\\
\textbf{Step 7}\\
A ``Data Butler'' will be initialized to access the repository.\\
~\\
\textbf{Step 8}\\
For each of the expected data products types (listed in Test Items
section §4.3.2) and each of the expected units (PVIs, coadds, etc), the
data product will be retrieved from the Butler and verified to be
non-empty.\\
~\\
~\\
~\\
\textbf{Step 9}\\
Verify that there exist entries in the forced-photometry table for all
coadd objects for the PVIs on which the object should appear.\\
~\\
\textbf{Step 10}\\
Verify that there exist entries in a forced-photometry table for each
image for all DIAObjects.\\
~\\
~\\
~\\

\hypertarget{lvv-t58---verify-matching-diasources-to-objects-dms-req-0324-implementation}{%
\subsection{\texorpdfstring{\href{https://jira.lsstcorp.org/secure/Tests.jspa\#/testCase/LVV-T58}{LVV-T58}
- Verify Matching DIASources to Objects (DMS-REQ-0324)
implementation}{LVV-T58 - Verify Matching DIASources to Objects (DMS-REQ-0324) implementation}}\label{lvv-t58---verify-matching-diasources-to-objects-dms-req-0324-implementation}}

\begin{longtable}[]{@{}llllll@{}}
\toprule
Version & Status & Priority & Verification Type & Critical Event &
Owner\tabularnewline
\midrule
\endhead
1 & Draft & Normal & Test & False & ebellm\tabularnewline
\bottomrule
\end{longtable}

\hypertarget{test-items-21}{%
\subsubsection{Test Items}\label{test-items-21}}

\hypertarget{requirements-42}{%
\subsubsection{Requirements}\label{requirements-42}}

\hypertarget{requirements-43}{%
\subsubsection{Requirements}\label{requirements-43}}

\begin{itemize}
\tightlist
\item
  \href{https://jira.lsstcorp.org/browse/LVV-155}{LVV-155} -
  DMS-REQ-0324-V-01: Matching DIASources to Objects
\end{itemize}

\hypertarget{test-script-21}{%
\subsubsection{Test Script}\label{test-script-21}}

\textbf{Step 1}\\
The DM Stack and Alert Processing packaged shall be initialized as
described in LVT-T17 (AG-00-00).\\
~\\
\textbf{Step 2}\\
The alert generation processing will be executed using the verification
cluster:\\
\hspace*{0.333em} ~ ~ ~python
ap\_verify/bin/prepare\_demo\_slurm\_files.py\\
\hspace*{0.333em} ~ ~ ~\# At present we must run a single ccd+visit to
handle ingestion before\\
\hspace*{0.333em} ~ ~ ~\# parallel processing can begin\\
\hspace*{0.333em} ~ ~ ~./ap\_verify/bin/exec\_demo\_run\_1ccd.sh 410915
25\\
\hspace*{0.333em} ~ ~ ~ln -s ap\_verify/bin/demo\_run.sl\\
\hspace*{0.333em} ~ ~ ~ln -s ap\_verify/bin/demo\_cmds.conf\\
\hspace*{0.333em} ~ ~ ~sbatch demo\_run.sl\\
and any errors or failures reported.\\
~\\
\textbf{Step 3}\\
A ``Data Butler'' will be initialized to access the repository.\\
~\\
\textbf{Step 4}\\
For each of the expected data products types (listed in §4.2.2) and each
of the expected units (PVIs, catalogs, etc.), the data product will be
retrieved from the Butler and verified to be non-empty.\\
~\\
\textbf{Step 5}\\
DIAObjects are currently only stored in a database, without shims to the
Butler, so the existence of the database table and its non-empty
contents will be verified by directly accessing it using sqlite3 and
executing appropriate SQL queries.\\
~\\
\textbf{Step 6}\\
Do a DRP and Data Release with same dataset\\
~\\
\textbf{Step 7}\\
Verify that a cross match table between the Prompt products and DRP
Products is available.\\
~\\

\hypertarget{lvv-t28---verify-measurements-in-catalogs-dms-req-0347-implementation}{%
\subsection{\texorpdfstring{\href{https://jira.lsstcorp.org/secure/Tests.jspa\#/testCase/LVV-T28}{LVV-T28}
- Verify Measurements in catalogs (DMS-REQ-0347)
Implementation}{LVV-T28 - Verify Measurements in catalogs (DMS-REQ-0347) Implementation}}\label{lvv-t28---verify-measurements-in-catalogs-dms-req-0347-implementation}}

\begin{longtable}[]{@{}llllll@{}}
\toprule
Version & Status & Priority & Verification Type & Critical Event &
Owner\tabularnewline
\midrule
\endhead
1 & Draft & Normal & Test & False & lguy\tabularnewline
\bottomrule
\end{longtable}

\hypertarget{test-items-22}{%
\subsubsection{Test Items}\label{test-items-22}}

All catalogs shall record source measurements in flux units.

\hypertarget{requirements-44}{%
\subsubsection{Requirements}\label{requirements-44}}

\hypertarget{requirements-45}{%
\subsubsection{Requirements}\label{requirements-45}}

\begin{itemize}
\tightlist
\item
  \href{https://jira.lsstcorp.org/browse/LVV-178}{LVV-178} -
  DMS-REQ-0347-V-01: Measurements in catalogs
\end{itemize}

\hypertarget{test-script-22}{%
\subsubsection{Test Script}\label{test-script-22}}

\textbf{Step 1}\\
The DM Stack and Alert Processing packaged shall be initialized as
described in LVT-T17 (AG-00-00).\\
~\\
\textbf{Step 2}\\
The alert generation processing will be executed using the verification
cluster:\\
\hspace*{0.333em} ~ ~ ~python
ap\_verify/bin/prepare\_demo\_slurm\_files.py\\
\hspace*{0.333em} ~ ~ ~\# At present we must run a single ccd+visit to
handle ingestion before\\
\hspace*{0.333em} ~ ~ ~\# parallel processing can begin\\
\hspace*{0.333em} ~ ~ ~./ap\_verify/bin/exec\_demo\_run\_1ccd.sh 410915
25\\
\hspace*{0.333em} ~ ~ ~ln -s ap\_verify/bin/demo\_run.sl\\
\hspace*{0.333em} ~ ~ ~ln -s ap\_verify/bin/demo\_cmds.conf\\
\hspace*{0.333em} ~ ~ ~sbatch demo\_run.sl\\
and any errors or failures reported.\\
~\\
\textbf{Step 3}\\
A ``Data Butler'' will be initialized to access the repository.\\
~\\
\textbf{Step 4}\\
For each of the expected data products types (listed in §4.2.2) and each
of the expected units (PVIs, catalogs, etc.), the data product will be
retrieved from the Butler and verified to be non-empty.\\
~\\
\textbf{Step 5}\\
DIAObjects are currently only stored in a database, without shims to the
Butler, so the existence of the database table and its non-empty
contents will be verified by directly accessing it using sqlite3 and
executing appropriate SQL queries.\\
~\\
\textbf{Step 6}\\
The DM Stack shall be initialized using the loadLSST script (as
described in DRP-00-00).\\
~\\
\textbf{Step 7}\\
A ``Data Butler'' will be initialized to access the repository.\\
~\\
\textbf{Step 8}\\
For each of the expected data products types (listed in Test Items
section §4.3.2) and each of the expected units (PVIs, coadds, etc), the
data product will be retrieved from the Butler and verified to be
non-empty.\\
~\\
~\\
~\\
\textbf{Step 9}\\
Verify that each of the single-visit, coadd, and difference image
catalogs from HSC reporcessing and HiTS reprocessing (which may be the
first source of regular difference images) provide measurements in flux
units.\\
~\\

\hypertarget{lvv-t75---verify-multi-band-coadds-dms-req-0281-implementation}{%
\subsection{\texorpdfstring{\href{https://jira.lsstcorp.org/secure/Tests.jspa\#/testCase/LVV-T75}{LVV-T75}
- Verify Multi-band Coadds (DMS-REQ-0281)
implementation}{LVV-T75 - Verify Multi-band Coadds (DMS-REQ-0281) implementation}}\label{lvv-t75---verify-multi-band-coadds-dms-req-0281-implementation}}

\begin{longtable}[]{@{}llllll@{}}
\toprule
Version & Status & Priority & Verification Type & Critical Event &
Owner\tabularnewline
\midrule
\endhead
1 & Draft & Normal & Test & False & ktl\tabularnewline
\bottomrule
\end{longtable}

\hypertarget{test-items-23}{%
\subsubsection{Test Items}\label{test-items-23}}

\hypertarget{requirements-46}{%
\subsubsection{Requirements}\label{requirements-46}}

\hypertarget{requirements-47}{%
\subsubsection{Requirements}\label{requirements-47}}

\begin{itemize}
\tightlist
\item
  \href{https://jira.lsstcorp.org/browse/LVV-112}{LVV-112} -
  DMS-REQ-0281-V-01: Multi-band Coadds
\end{itemize}

\hypertarget{precondition-3}{%
\subsubsection{Precondition}\label{precondition-3}}

\hypertarget{test-script-23}{%
\subsubsection{Test Script}\label{test-script-23}}

\textbf{Step 1}\\
~\\
~\\
\textbf{Step 2}\\
The DM Stack shall be initialized using the loadLSST script (as
described in DRP-00-00).\\
~\\
\textbf{Step 3}\\
A ``Data Butler'' will be initialized to access the repository.\\
~\\
\textbf{Step 4}\\
For each of the expected data products types (listed in Test Items
section §4.3.2) and each of the expected units (PVIs, coadds, etc), the
data product will be retrieved from the Butler and verified to be
non-empty.\\
~\\
~\\
~\\
\textbf{Step 5}\\
The DM Stack shall be initialized using the loadLSST script (as
described in LVV-T10 - DRP-00-00)\\
~\\
\textbf{Step 6}\\
A ``Data Butler'' will be initialized to access the repository.\\
~\\
\textbf{Step 7}\\
For each combination of tract/patch/filter, the PVI will be retrieved
from the Butler, and the existence of all components described in Test
items section §4.6.2 will be verified.\\
~\\
\textbf{Step 8}\\
Scripts from the pipe\_analysis package will be run on every visit to
check for the presence of data products and make plots\\
~\\
\textbf{Step 9}\\
Ten patches will be chosen at random and inspected by eye for unmasked
artifacts.\\
~\\
\textbf{Step 10}\\
Verify that deep detection coadds exist based on all filters.\\
~\\
~\\
~\\

\hypertarget{lvv-t35---verify-nightly-data-accessible-within-24-hrs-dms-req-0004-implementation}{%
\subsection{\texorpdfstring{\href{https://jira.lsstcorp.org/secure/Tests.jspa\#/testCase/LVV-T35}{LVV-T35}
- Verify Nightly Data Accessible Within 24 hrs (DMS-REQ-0004)
Implementation}{LVV-T35 - Verify Nightly Data Accessible Within 24 hrs (DMS-REQ-0004) Implementation}}\label{lvv-t35---verify-nightly-data-accessible-within-24-hrs-dms-req-0004-implementation}}

\begin{longtable}[]{@{}llllll@{}}
\toprule
Version & Status & Priority & Verification Type & Critical Event &
Owner\tabularnewline
\midrule
\endhead
1 & Draft & Normal & Test & False & ebellm\tabularnewline
\bottomrule
\end{longtable}

\hypertarget{test-items-24}{%
\subsubsection{Test Items}\label{test-items-24}}

With the exception of alerts and Solar System Objects, all Level 1 Data
Products shall be made public within time L1PublicT (LSR-REQ-0104) of
the acquisition of the raw image data. Alerts shall be made available
within time OTT1 (LSR-REQ-0101) from the conclusion of readout of the
raw exposures used to generate each alert to the distribution of the
alert to community distribution mechanisms. Solar System Object orbits
shall, on average, be calculated before the following night's observing
finishes and the results shall be made available within time L1PublicT
of those calculations being completed.

\hypertarget{requirements-48}{%
\subsubsection{Requirements}\label{requirements-48}}

\hypertarget{requirements-49}{%
\subsubsection{Requirements}\label{requirements-49}}

\begin{itemize}
\tightlist
\item
  \href{https://jira.lsstcorp.org/browse/LVV-4}{LVV-4} -
  DMS-REQ-0004-V-01: Nightly Data Accessible Within 24 hrs
\end{itemize}

\hypertarget{test-script-24}{%
\subsubsection{Test Script}\label{test-script-24}}

\textbf{Step 1}\\
The DM Stack and Alert Processing packaged shall be initialized as
described in LVT-T17 (AG-00-00).\\
~\\
\textbf{Step 2}\\
The alert generation processing will be executed using the verification
cluster:\\
\hspace*{0.333em} ~ ~ ~python
ap\_verify/bin/prepare\_demo\_slurm\_files.py\\
\hspace*{0.333em} ~ ~ ~\# At present we must run a single ccd+visit to
handle ingestion before\\
\hspace*{0.333em} ~ ~ ~\# parallel processing can begin\\
\hspace*{0.333em} ~ ~ ~./ap\_verify/bin/exec\_demo\_run\_1ccd.sh 410915
25\\
\hspace*{0.333em} ~ ~ ~ln -s ap\_verify/bin/demo\_run.sl\\
\hspace*{0.333em} ~ ~ ~ln -s ap\_verify/bin/demo\_cmds.conf\\
\hspace*{0.333em} ~ ~ ~sbatch demo\_run.sl\\
and any errors or failures reported.\\
~\\
\textbf{Step 3}\\
A ``Data Butler'' will be initialized to access the repository.\\
~\\
\textbf{Step 4}\\
For each of the expected data products types (listed in §4.2.2) and each
of the expected units (PVIs, catalogs, etc.), the data product will be
retrieved from the Butler and verified to be non-empty.\\
~\\
\textbf{Step 5}\\
DIAObjects are currently only stored in a database, without shims to the
Butler, so the existence of the database table and its non-empty
contents will be verified by directly accessing it using sqlite3 and
executing appropriate SQL queries.\\
~\\
\textbf{Step 6}\\
Test generation of alerts from a set of DECam HiTS data. ~Ingest,
process, subtract, detect, measure, and verify that from
ingest-\textgreater{}alert satisfies OTT1.\\
~\\
\textbf{Step 7}\\
Verify that the fuller set of Level 1 Data Productscan ~from the
processing above ~is made available in L1PublicT.\\
~\\
\textbf{Step 8}\\
Feed outputs of 1 night equivalent of LSST observing worth of DECam HiTS
data and verify that Solar System Object orbits can be updated within 24
hours.\\
~\\

\hypertarget{lvv-t78---verify-persisting-data-products-dms-req-0334-implementation}{%
\subsection{\texorpdfstring{\href{https://jira.lsstcorp.org/secure/Tests.jspa\#/testCase/LVV-T78}{LVV-T78}
- Verify Persisting Data Products (DMS-REQ-0334)
implementation}{LVV-T78 - Verify Persisting Data Products (DMS-REQ-0334) implementation}}\label{lvv-t78---verify-persisting-data-products-dms-req-0334-implementation}}

\begin{longtable}[]{@{}llllll@{}}
\toprule
Version & Status & Priority & Verification Type & Critical Event &
Owner\tabularnewline
\midrule
\endhead
1 & Draft & Normal & Test & False & ktl\tabularnewline
\bottomrule
\end{longtable}

\hypertarget{test-items-25}{%
\subsubsection{Test Items}\label{test-items-25}}

\hypertarget{requirements-50}{%
\subsubsection{Requirements}\label{requirements-50}}

\hypertarget{requirements-51}{%
\subsubsection{Requirements}\label{requirements-51}}

\begin{itemize}
\tightlist
\item
  \href{https://jira.lsstcorp.org/browse/LVV-165}{LVV-165} -
  DMS-REQ-0334-V-01: Persisting Data Products
\end{itemize}

\hypertarget{precondition-4}{%
\subsubsection{Precondition}\label{precondition-4}}

Precursor data from HSC PDR.

\hypertarget{test-script-25}{%
\subsubsection{Test Script}\label{test-script-25}}

\textbf{Step 1}\\
~\\
~\\

\hypertarget{lvv-t38---verify-processed-visit-images-dms-req-0069-implementation}{%
\subsection{\texorpdfstring{\href{https://jira.lsstcorp.org/secure/Tests.jspa\#/testCase/LVV-T38}{LVV-T38}
- Verify Processed Visit Images (DMS-REQ-0069)
Implementation}{LVV-T38 - Verify Processed Visit Images (DMS-REQ-0069) Implementation}}\label{lvv-t38---verify-processed-visit-images-dms-req-0069-implementation}}

\begin{longtable}[]{@{}llllll@{}}
\toprule
Version & Status & Priority & Verification Type & Critical Event &
Owner\tabularnewline
\midrule
\endhead
1 & Draft & Normal & Test & False & ebellm\tabularnewline
\bottomrule
\end{longtable}

\hypertarget{test-items-26}{%
\subsubsection{Test Items}\label{test-items-26}}

The DMS shall produce Processed Visit Images, in which the corresponding
raw sensor array data has been trimmed of overscan and corrected for
instrumental signature. Images obtained in pairs during a standard visit
are combined.

\hypertarget{requirements-52}{%
\subsubsection{Requirements}\label{requirements-52}}

\hypertarget{requirements-53}{%
\subsubsection{Requirements}\label{requirements-53}}

\begin{itemize}
\tightlist
\item
  \href{https://jira.lsstcorp.org/browse/LVV-29}{LVV-29} -
  DMS-REQ-0069-V-01: Processed Visit Images
\end{itemize}

\hypertarget{test-script-26}{%
\subsubsection{Test Script}\label{test-script-26}}

\textbf{Step 1}\\
Process HSC data, DECAM data. ~Verify that Processed Visit Images are
generated at correct size and with significant instrumental artifacts
removed.\\
~\\
\textbf{Step 2}\\
Run camera test stand data through full acquisition+backbone+ISR.\\
~\\
\textbf{Step 3}\\
Run simulated LSST data with calibrations through prompt processing
system and inspect Processed Visit images to verify that they have been
cleaned of significant artifacts and are of the correct, shape, and
described orientation.\\
~\\

\hypertarget{lvv-t63---verify-produce-images-for-epo-dms-req-0103-implementation}{%
\subsection{\texorpdfstring{\href{https://jira.lsstcorp.org/secure/Tests.jspa\#/testCase/LVV-T63}{LVV-T63}
- Verify Produce Images for EPO (DMS-REQ-0103)
implementation}{LVV-T63 - Verify Produce Images for EPO (DMS-REQ-0103) implementation}}\label{lvv-t63---verify-produce-images-for-epo-dms-req-0103-implementation}}

\begin{longtable}[]{@{}llllll@{}}
\toprule
Version & Status & Priority & Verification Type & Critical Event &
Owner\tabularnewline
\midrule
\endhead
1 & Draft & Normal & Test & False & gpdf\tabularnewline
\bottomrule
\end{longtable}

\hypertarget{test-items-27}{%
\subsubsection{Test Items}\label{test-items-27}}

\hypertarget{requirements-54}{%
\subsubsection{Requirements}\label{requirements-54}}

\hypertarget{requirements-55}{%
\subsubsection{Requirements}\label{requirements-55}}

\begin{itemize}
\tightlist
\item
  \href{https://jira.lsstcorp.org/browse/LVV-45}{LVV-45} -
  DMS-REQ-0103-V-01: Produce Images for EPO
\end{itemize}

\hypertarget{test-script-27}{%
\subsubsection{Test Script}\label{test-script-27}}

\textbf{Step 1}\\
The DM Stack shall be initialized using the loadLSST script (as
described in DRP-00-00).\\
~\\
\textbf{Step 2}\\
A ``Data Butler'' will be initialized to access the repository.\\
~\\
\textbf{Step 3}\\
For each of the expected data products types (listed in Test Items
section §4.3.2) and each of the expected units (PVIs, coadds, etc), the
data product will be retrieved from the Butler and verified to be
non-empty.\\
~\\
~\\
~\\
\textbf{Step 4}\\
Query the produced data to produce required EPO data products.\\
~\\

\hypertarget{lvv-t62---verify-provide-psf-for-coadded-images-dms-req-0047-implementation}{%
\subsection{\texorpdfstring{\href{https://jira.lsstcorp.org/secure/Tests.jspa\#/testCase/LVV-T62}{LVV-T62}
- Verify Provide PSF for Coadded Images (DMS-REQ-0047)
implementation}{LVV-T62 - Verify Provide PSF for Coadded Images (DMS-REQ-0047) implementation}}\label{lvv-t62---verify-provide-psf-for-coadded-images-dms-req-0047-implementation}}

\begin{longtable}[]{@{}llllll@{}}
\toprule
Version & Status & Priority & Verification Type & Critical Event &
Owner\tabularnewline
\midrule
\endhead
1 & Draft & Normal & Test & False & ktl\tabularnewline
\bottomrule
\end{longtable}

\hypertarget{test-items-28}{%
\subsubsection{Test Items}\label{test-items-28}}

\hypertarget{requirements-56}{%
\subsubsection{Requirements}\label{requirements-56}}

\hypertarget{requirements-57}{%
\subsubsection{Requirements}\label{requirements-57}}

\begin{itemize}
\tightlist
\item
  \href{https://jira.lsstcorp.org/browse/LVV-20}{LVV-20} -
  DMS-REQ-0047-V-01: Provide PSF for Coadded Images
\end{itemize}

\hypertarget{test-script-28}{%
\subsubsection{Test Script}\label{test-script-28}}

\textbf{Step 1}\\
~\\
~\\
\textbf{Step 2}\\
The DM Stack shall be initialized using the loadLSST script (as
described in LVV-T10 - DRP-00-00)\\
~\\
\textbf{Step 3}\\
A ``Data Butler'' will be initialized to access the repository.\\
~\\
\textbf{Step 4}\\
For each combination of tract/patch/filter, the PVI will be retrieved
from the Butler, and the existence of all components described in Test
items section §4.6.2 will be verified.\\
~\\
\textbf{Step 5}\\
Scripts from the pipe\_analysis package will be run on every visit to
check for the presence of data products and make plots\\
~\\
\textbf{Step 6}\\
Ten patches will be chosen at random and inspected by eye for unmasked
artifacts.\\
~\\
\textbf{Step 7}\\
Query range of positions on 10 different coadd images. ~Verify that
reasonable PSFs are returned.\\
~\\

\hypertarget{lvv-t68---verify-provide-photometric-redshifts-of-galaxies-dms-req-0046-implementation}{%
\subsection{\texorpdfstring{\href{https://jira.lsstcorp.org/secure/Tests.jspa\#/testCase/LVV-T68}{LVV-T68}
- Verify Provide Photometric Redshifts of Galaxies (DMS-REQ-0046)
implementation}{LVV-T68 - Verify Provide Photometric Redshifts of Galaxies (DMS-REQ-0046) implementation}}\label{lvv-t68---verify-provide-photometric-redshifts-of-galaxies-dms-req-0046-implementation}}

\begin{longtable}[]{@{}llllll@{}}
\toprule
Version & Status & Priority & Verification Type & Critical Event &
Owner\tabularnewline
\midrule
\endhead
1 & Draft & Normal & Test & False & ktl\tabularnewline
\bottomrule
\end{longtable}

\hypertarget{test-items-29}{%
\subsubsection{Test Items}\label{test-items-29}}

\hypertarget{requirements-58}{%
\subsubsection{Requirements}\label{requirements-58}}

\hypertarget{requirements-59}{%
\subsubsection{Requirements}\label{requirements-59}}

\begin{itemize}
\tightlist
\item
  \href{https://jira.lsstcorp.org/browse/LVV-19}{LVV-19} -
  DMS-REQ-0046-V-01: Provide Photometric Redshifts of Galaxies
\end{itemize}

\hypertarget{precondition-5}{%
\subsubsection{Precondition}\label{precondition-5}}

Input Data\\
HSC Public Data Release

\hypertarget{test-script-29}{%
\subsubsection{Test Script}\label{test-script-29}}

\textbf{Step 1}\\
load LSST stack\\
~\\
\textbf{Step 2}\\
Ingest HSC PDR data\\
~\\
\textbf{Step 3}\\
processCcd.py\\
multiBandDriver.py\\
~\\
\textbf{Step 4}\\
Estimate Photometric Redshifts\\
~\\
\textbf{Step 5}\\
Load into DRP Database\\
~\\
\textbf{Step 6}\\
Inspect database to verify that photometric redshifts are present for
all objects\\
~\\

\hypertarget{lvv-t53---verify-ssobject-catalog-dms-req-0273-implementation}{%
\subsection{\texorpdfstring{\href{https://jira.lsstcorp.org/secure/Tests.jspa\#/testCase/LVV-T53}{LVV-T53}
- Verify SSObject Catalog (DMS-REQ-0273)
implementation}{LVV-T53 - Verify SSObject Catalog (DMS-REQ-0273) implementation}}\label{lvv-t53---verify-ssobject-catalog-dms-req-0273-implementation}}

\begin{longtable}[]{@{}llllll@{}}
\toprule
Version & Status & Priority & Verification Type & Critical Event &
Owner\tabularnewline
\midrule
\endhead
1 & Draft & Normal & Test & False & ebellm\tabularnewline
\bottomrule
\end{longtable}

\hypertarget{test-items-30}{%
\subsubsection{Test Items}\label{test-items-30}}

\hypertarget{requirements-60}{%
\subsubsection{Requirements}\label{requirements-60}}

\hypertarget{requirements-61}{%
\subsubsection{Requirements}\label{requirements-61}}

\begin{itemize}
\tightlist
\item
  \href{https://jira.lsstcorp.org/browse/LVV-104}{LVV-104} -
  DMS-REQ-0273-V-01: SSObject Catalog
\end{itemize}

\hypertarget{test-script-30}{%
\subsubsection{Test Script}\label{test-script-30}}

\textbf{Step 1}\\
The DM Stack and Alert Processing packaged shall be initialized as
described in LVT-T17 (AG-00-00).\\
~\\
\textbf{Step 2}\\
The alert generation processing will be executed using the verification
cluster:\\
\hspace*{0.333em} ~ ~ ~python
ap\_verify/bin/prepare\_demo\_slurm\_files.py\\
\hspace*{0.333em} ~ ~ ~\# At present we must run a single ccd+visit to
handle ingestion before\\
\hspace*{0.333em} ~ ~ ~\# parallel processing can begin\\
\hspace*{0.333em} ~ ~ ~./ap\_verify/bin/exec\_demo\_run\_1ccd.sh 410915
25\\
\hspace*{0.333em} ~ ~ ~ln -s ap\_verify/bin/demo\_run.sl\\
\hspace*{0.333em} ~ ~ ~ln -s ap\_verify/bin/demo\_cmds.conf\\
\hspace*{0.333em} ~ ~ ~sbatch demo\_run.sl\\
and any errors or failures reported.\\
~\\
\textbf{Step 3}\\
A ``Data Butler'' will be initialized to access the repository.\\
~\\
\textbf{Step 4}\\
For each of the expected data products types (listed in §4.2.2) and each
of the expected units (PVIs, catalogs, etc.), the data product will be
retrieved from the Butler and verified to be non-empty.\\
~\\
\textbf{Step 5}\\
DIAObjects are currently only stored in a database, without shims to the
Butler, so the existence of the database table and its non-empty
contents will be verified by directly accessing it using sqlite3 and
executing appropriate SQL queries.\\
~\\
\textbf{Step 6}\\
Run MOPS\\
~\\
\textbf{Step 7}\\
Load results\\
~\\
\textbf{Step 8}\\
Inspect SSOBject catalog\\
~\\

\hypertarget{lvv-t74---verify-template-coadds-dms-req-0280-implementation}{%
\subsection{\texorpdfstring{\href{https://jira.lsstcorp.org/secure/Tests.jspa\#/testCase/LVV-T74}{LVV-T74}
- Verify Template Coadds (DMS-REQ-0280)
implementation}{LVV-T74 - Verify Template Coadds (DMS-REQ-0280) implementation}}\label{lvv-t74---verify-template-coadds-dms-req-0280-implementation}}

\begin{longtable}[]{@{}llllll@{}}
\toprule
Version & Status & Priority & Verification Type & Critical Event &
Owner\tabularnewline
\midrule
\endhead
1 & Draft & Normal & Test & False & ktl\tabularnewline
\bottomrule
\end{longtable}

\hypertarget{test-items-31}{%
\subsubsection{Test Items}\label{test-items-31}}

\hypertarget{requirements-62}{%
\subsubsection{Requirements}\label{requirements-62}}

\hypertarget{requirements-63}{%
\subsubsection{Requirements}\label{requirements-63}}

\begin{itemize}
\tightlist
\item
  \href{https://jira.lsstcorp.org/browse/LVV-111}{LVV-111} -
  DMS-REQ-0280-V-01: Template Coadds
\end{itemize}

\hypertarget{test-script-31}{%
\subsubsection{Test Script}\label{test-script-31}}

\textbf{Step 1}\\
~\\
~\\
\textbf{Step 2}\\
The DM Stack and Alert Processing packaged shall be initialized as
described in LVT-T17 (AG-00-00).\\
~\\
\textbf{Step 3}\\
The alert generation processing will be executed using the verification
cluster:\\
\hspace*{0.333em} ~ ~ ~python
ap\_verify/bin/prepare\_demo\_slurm\_files.py\\
\hspace*{0.333em} ~ ~ ~\# At present we must run a single ccd+visit to
handle ingestion before\\
\hspace*{0.333em} ~ ~ ~\# parallel processing can begin\\
\hspace*{0.333em} ~ ~ ~./ap\_verify/bin/exec\_demo\_run\_1ccd.sh 410915
25\\
\hspace*{0.333em} ~ ~ ~ln -s ap\_verify/bin/demo\_run.sl\\
\hspace*{0.333em} ~ ~ ~ln -s ap\_verify/bin/demo\_cmds.conf\\
\hspace*{0.333em} ~ ~ ~sbatch demo\_run.sl\\
and any errors or failures reported.\\
~\\
\textbf{Step 4}\\
A ``Data Butler'' will be initialized to access the repository.\\
~\\
\textbf{Step 5}\\
For each of the expected data products types (listed in §4.2.2) and each
of the expected units (PVIs, catalogs, etc.), the data product will be
retrieved from the Butler and verified to be non-empty.\\
~\\
\textbf{Step 6}\\
DIAObjects are currently only stored in a database, without shims to the
Butler, so the existence of the database table and its non-empty
contents will be verified by directly accessing it using sqlite3 and
executing appropriate SQL queries.\\
~\\

\hypertarget{lvv-t23---verify-test-storing-approximations-of-per-pixel-metadata-dms-req-0326-implementation}{%
\subsection{\texorpdfstring{\href{https://jira.lsstcorp.org/secure/Tests.jspa\#/testCase/LVV-T23}{LVV-T23}
- Verify Test Storing Approximations of Per-pixel Metadata
(DMS-REQ-0326)
Implementation}{LVV-T23 - Verify Test Storing Approximations of Per-pixel Metadata (DMS-REQ-0326) Implementation}}\label{lvv-t23---verify-test-storing-approximations-of-per-pixel-metadata-dms-req-0326-implementation}}

\begin{longtable}[]{@{}llllll@{}}
\toprule
Version & Status & Priority & Verification Type & Critical Event &
Owner\tabularnewline
\midrule
\endhead
1 & Draft & Normal & Test & False & lguy\tabularnewline
\bottomrule
\end{longtable}

\hypertarget{test-items-32}{%
\subsubsection{Test Items}\label{test-items-32}}

Image depth and mask information shall be available in a parameterized
approximate form in addition to a full per-pixel form.

\hypertarget{requirements-64}{%
\subsubsection{Requirements}\label{requirements-64}}

\hypertarget{requirements-65}{%
\subsubsection{Requirements}\label{requirements-65}}

\begin{itemize}
\tightlist
\item
  \href{https://jira.lsstcorp.org/browse/LVV-157}{LVV-157} -
  DMS-REQ-0326-V-01: Storing Approximations of Per-pixel Metadata
\end{itemize}

\hypertarget{test-script-32}{%
\subsubsection{Test Script}\label{test-script-32}}

\textbf{Step 1}\\
The DM Stack shall be initialized using the loadLSST script (as
described in DRP-00-00).\\
~\\
\textbf{Step 2}\\
A ``Data Butler'' will be initialized to access the repository.\\
~\\
\textbf{Step 3}\\
For each of the expected data products types (listed in Test Items
section §4.3.2) and each of the expected units (PVIs, coadds, etc), the
data product will be retrieved from the Butler and verified to be
non-empty.\\
~\\
~\\
~\\
\textbf{Step 4}\\
Produce coverage map from HSC PDR data. ~Compare depth and mask
parameterization (MOC) to pixel-level depth and mask products.
~Parameterized mask should predict at least XXX (99\% ?) of pixel-level
mask. ~Total depth in parameterized depth map and pixel-level depth map
should agree to YYY (0.1\% ?).\\
~\\

\hypertarget{lvv-t80---verify-implementation-of-detecting-faint-variable-objects-dms-req-0337}{%
\subsection{\texorpdfstring{\href{https://jira.lsstcorp.org/secure/Tests.jspa\#/testCase/LVV-T80}{LVV-T80}
- Verify implementation of Detecting faint variable objects
(DMS-REQ-0337)}{LVV-T80 - Verify implementation of Detecting faint variable objects (DMS-REQ-0337)}}\label{lvv-t80---verify-implementation-of-detecting-faint-variable-objects-dms-req-0337}}

\begin{longtable}[]{@{}llllll@{}}
\toprule
Version & Status & Priority & Verification Type & Critical Event &
Owner\tabularnewline
\midrule
\endhead
1 & Draft & Normal & Test & False & ebellm\tabularnewline
\bottomrule
\end{longtable}

\hypertarget{test-items-33}{%
\subsubsection{Test Items}\label{test-items-33}}

\hypertarget{requirements-66}{%
\subsubsection{Requirements}\label{requirements-66}}

\hypertarget{requirements-67}{%
\subsubsection{Requirements}\label{requirements-67}}

\begin{itemize}
\tightlist
\item
  \href{https://jira.lsstcorp.org/browse/LVV-168}{LVV-168} -
  DMS-REQ-0337-V-01: Detecting faint variable objects
\end{itemize}

\hypertarget{precondition-6}{%
\subsubsection{Precondition}\label{precondition-6}}

Input Data\\
DECam HiTS data.

\hypertarget{test-script-33}{%
\subsubsection{Test Script}\label{test-script-33}}

\textbf{Step 1}\\
The DM Stack and Alert Processing packaged shall be initialized as
described in LVT-T17 (AG-00-00).\\
~\\
\textbf{Step 2}\\
The alert generation processing will be executed using the verification
cluster:\\
\hspace*{0.333em} ~ ~ ~python
ap\_verify/bin/prepare\_demo\_slurm\_files.py\\
\hspace*{0.333em} ~ ~ ~\# At present we must run a single ccd+visit to
handle ingestion before\\
\hspace*{0.333em} ~ ~ ~\# parallel processing can begin\\
\hspace*{0.333em} ~ ~ ~./ap\_verify/bin/exec\_demo\_run\_1ccd.sh 410915
25\\
\hspace*{0.333em} ~ ~ ~ln -s ap\_verify/bin/demo\_run.sl\\
\hspace*{0.333em} ~ ~ ~ln -s ap\_verify/bin/demo\_cmds.conf\\
\hspace*{0.333em} ~ ~ ~sbatch demo\_run.sl\\
and any errors or failures reported.\\
~\\
\textbf{Step 3}\\
A ``Data Butler'' will be initialized to access the repository.\\
~\\
\textbf{Step 4}\\
For each of the expected data products types (listed in §4.2.2) and each
of the expected units (PVIs, catalogs, etc.), the data product will be
retrieved from the Butler and verified to be non-empty.\\
~\\
\textbf{Step 5}\\
DIAObjects are currently only stored in a database, without shims to the
Butler, so the existence of the database table and its non-empty
contents will be verified by directly accessing it using sqlite3 and
executing appropriate SQL queries.\\
~\\
\textbf{Step 6}\\
Identify 100 objects from Gaia with proper motions high enough to have
detectably moved during HSC observations.\\
~\\
\textbf{Step 7}\\
Measure reported proper motion of these objects in DM Stack processing.
~Verify that it is consistent with Gaia objects.\\
~\\
\textbf{Step 8}\\
Identify 100 quasars from color-space or existing extragalactic
spectroscopic catalog.\\
~\\
\textbf{Step 9}\\
Measure lightcurves of these quasars. ~Determine if structure function
is reasonable.\\
~\\

\hypertarget{lvv-t79---verify-implementation-of-psf-matched-coadds-dms-req-0335}{%
\subsection{\texorpdfstring{\href{https://jira.lsstcorp.org/secure/Tests.jspa\#/testCase/LVV-T79}{LVV-T79}
- Verify implementation of PSF-Matched Coadds
(DMS-REQ-0335)}{LVV-T79 - Verify implementation of PSF-Matched Coadds (DMS-REQ-0335)}}\label{lvv-t79---verify-implementation-of-psf-matched-coadds-dms-req-0335}}

\begin{longtable}[]{@{}llllll@{}}
\toprule
Version & Status & Priority & Verification Type & Critical Event &
Owner\tabularnewline
\midrule
\endhead
1 & Draft & Normal & Test & False & ktl\tabularnewline
\bottomrule
\end{longtable}

\hypertarget{test-items-34}{%
\subsubsection{Test Items}\label{test-items-34}}

\hypertarget{requirements-68}{%
\subsubsection{Requirements}\label{requirements-68}}

\hypertarget{requirements-69}{%
\subsubsection{Requirements}\label{requirements-69}}

\begin{itemize}
\tightlist
\item
  \href{https://jira.lsstcorp.org/browse/LVV-166}{LVV-166} -
  DMS-REQ-0335-V-01: PSF-Matched Coadds
\end{itemize}

\hypertarget{test-script-34}{%
\subsubsection{Test Script}\label{test-script-34}}

\textbf{Step 1}\\
~\\
~\\
\textbf{Step 2}\\
The DM Stack shall be initialized using the loadLSST script (as
described in DRP-00-00).\\
~\\
\textbf{Step 3}\\
A ``Data Butler'' will be initialized to access the repository.\\
~\\
\textbf{Step 4}\\
For each of the expected data products types (listed in Test Items
section §4.3.2) and each of the expected units (PVIs, coadds, etc), the
data product will be retrieved from the Butler and verified to be
non-empty.\\
~\\
~\\
~\\
\textbf{Step 5}\\
Verify that PSF-matched coadds were created.\\
~\\

\hypertarget{lvv-t82---verify-implementation-of-tracking-characterization-changes-between-data-releases-dms-req-0339}{%
\subsection{\texorpdfstring{\href{https://jira.lsstcorp.org/secure/Tests.jspa\#/testCase/LVV-T82}{LVV-T82}
- Verify implementation of Tracking Characterization Changes Between
Data Releases
(DMS-REQ-0339)}{LVV-T82 - Verify implementation of Tracking Characterization Changes Between Data Releases (DMS-REQ-0339)}}\label{lvv-t82---verify-implementation-of-tracking-characterization-changes-between-data-releases-dms-req-0339}}

\begin{longtable}[]{@{}llllll@{}}
\toprule
Version & Status & Priority & Verification Type & Critical Event &
Owner\tabularnewline
\midrule
\endhead
1 & Draft & Normal & Test & False & jbosch\tabularnewline
\bottomrule
\end{longtable}

\hypertarget{test-items-35}{%
\subsubsection{Test Items}\label{test-items-35}}

\hypertarget{requirements-70}{%
\subsubsection{Requirements}\label{requirements-70}}

\hypertarget{requirements-71}{%
\subsubsection{Requirements}\label{requirements-71}}

\begin{itemize}
\tightlist
\item
  \href{https://jira.lsstcorp.org/browse/LVV-170}{LVV-170} -
  DMS-REQ-0339-V-01: Tracking Characterization Changes Between Data
  Releases
\end{itemize}

\hypertarget{test-script-35}{%
\subsubsection{Test Script}\label{test-script-35}}

\textbf{Step 1}\\
The DM Stack shall be initialized using the loadLSST script (as
described in DRP-00-00).\\
~\\
\textbf{Step 2}\\
A ``Data Butler'' will be initialized to access the repository.\\
~\\
\textbf{Step 3}\\
For each of the expected data products types (listed in Test Items
section §4.3.2) and each of the expected units (PVIs, coadds, etc), the
data product will be retrieved from the Butler and verified to be
non-empty.\\
~\\
~\\
~\\
\textbf{Step 4}\\
The DM Stack shall be initialized using the loadLSST script (as
described in LVV-T10 - DRP-00-00).\\
~\\
~\\
~\\
\textbf{Step 5}\\
A ``Data Butler'' will be initialized to access the repository.\\
~\\
\textbf{Step 6}\\
Scripts from the pipe\_analysis package will be run on every visit to
check for the presence of data products and make plots.\\
~\\
\textbf{Step 7}\\
The DM Stack shall be initialized using the loadLSST script (as
described in LVV-T10 - DRP-00-00).\\
~\\
\textbf{Step 8}\\
A ``Data Butler'' will be initialized to access the repository.\\
~\\
\textbf{Step 9}\\
Scripts from the pipe\_analysis package will be run on every tract to
check for the presence of data products and make plots\\
~\\
\textbf{Step 10}\\
The DM Stack shall be initialized using the loadLSST script (as
described in LVV-T10 - DRP-00-00).\\
~\\
\textbf{Step 11}\\
A ``Data Butler'' will be initialized to access the repository.\\
~\\
\textbf{Step 12}\\
For each processed CCD, the PVI will be retrieved from the Butler, and
the existence of all components described in section Test Items (§4.6.2)
will be verified.\\
~\\
\textbf{Step 13}\\
Scripts from the pipe\_analysis package will be run on every visit to
check for the presence of data products and make plots\\
~\\
\textbf{Step 14}\\
Five sensors will be chosen at random from each of two visits and
inspected by eye for unmasked artifacts.\\
~\\
\textbf{Step 15}\\
The DM Stack shall be initialized using the loadLSST script (as
described in LVV-T10 - DRP-00-00)\\
~\\
\textbf{Step 16}\\
A ``Data Butler'' will be initialized to access the repository.\\
~\\
\textbf{Step 17}\\
For each combination of tract/patch/filter, the PVI will be retrieved
from the Butler, and the existence of all components described in Test
items section §4.6.2 will be verified.\\
~\\
\textbf{Step 18}\\
Scripts from the pipe\_analysis package will be run on every visit to
check for the presence of data products and make plots\\
~\\
\textbf{Step 19}\\
Ten patches will be chosen at random and inspected by eye for unmasked
artifacts.\\
~\\
\textbf{Step 20}\\
Prepare a second DRP run -\textgreater{} DPDD with different
configuration parameters for this second test Data Release.\\
~\\
\textbf{Step 21}\\
Stage subset of products from first test Data Release to separate
storage.\\
~\\
\textbf{Step 22}\\
Scientifically compare the results of the subset of that region of sky
to those in the second test Data Release comparing the results of the
DRP Scientific Verification tests.\\
~\\

